\NormalFont\ShortTitle{ΕΠΙΣΤΟΛΗ ΙΕΡΕΜΙΟΥ}
{\MT ΕΠΙΣΤΟΛΗ ΙΕΡΕΜΙΟΥ

\par }\OneChap {\PP \VerseOne{1}ΑΝΤΙΓΡΑΦΟΝ ἐπιστολῆς ἧς ἀπέστειλεν Ἱερεμίας πρὸς τοὺς ἀχθησομένους αἰχμαλώτους εἰς Βαβυλῶνα ὑπὸ τοῦ βασιλέως τῶν Βαβυλωνίων, ἀναγγεῖλαι αὐτοῖς καθότι ἐπετάγη αὐτῷ ὑπὸ τοῦ Θεοῦ.
\par }{\PP \VS{2}Διὰ τὰς ἁμαρτίας ἃς ἡμαρτήκατε ἐναντίον τοῦ Θεοῦ, ἀχθήσεσθε εἰς Βαβυλῶνα αἰχμάλωτοι ὑπὸ Ναβουχοδονόσορ βασιλέως τῶν Βαβυλωνίων.
\VS{3}Εἰσελθόντες οὖν εἰς Βαβυλῶνα, ἔσεσθε ἐκεῖ ἔτη πλείονα καὶ χρόνον μακρὸν, ἕως γενεῶν ἑπτά· μετὰ τοῦτο δὲ ἐξάξω ὑμᾶς ἐκεῖθεν μετʼ εἰρήνης.
\par }{\PP \VS{4}Νυνὶ δὲ ὄψεσθε ἐν Βαβυλῶνι θεοὺς ἀργυροῦς καὶ χρυσοῦς καὶ ξυλίνους ἐπʼ ὤμοις αἰρομένους, δεικνύντας φόβον τοῖς ἔθνεσιν.
\VS{5}Εὐλαβήθητε οὖν μὴ καὶ ὑμεῖς ἀφομοιωθέντες τοῖς ἀλλοφύλοις ἀφομοιωθῆτε, καὶ φόβος ὑμᾶς λὰβῃ ἐπʼ αὐτοῖς, ἰδόντας ὄχλον ἔμπροσθεν καὶ ὄπισθεν αὐτῶν προσκυνοῦντας αὐτά.
\VS{6}Εἴπατε δὲ τῇ διανοίᾳ, σοὶ δεῖ προσκυνεῖν, δέσποτα.
\VS{7}Ὁ γὰρ ἄγγελός μου μεθʼ ὑμῶν ἐστιν, αὐτός τε ἐκζητῶν τὰς ψυχὰς ὑμῶν.
\par }{\PP \VS{8}Γλῶσσα γὰρ αὐτῶν ἐστι κατεξυσμένη ὑπὸ τέκτονος, αὐτά τε περίχρυσα καὶ περιάργυρα, ψευδῆ δʼ ἐστὶ, καὶ οὐ δύνανται λαλεῖν.
\VS{9}Καὶ ὥσπερ παρθένῳ φιλοκόσμῳ λαμβάνοντες χρυσίον, κατασκευάζουσι στεφάνους ἐπὶ τὰς κεφαλὰς τῶν θεῶν αὐτῶν.
\VS{10}Ἔστι δὲ καὶ ὅτε ὑφαιρούμενοι οἱ ἱερεῖς ἀπὸ τῶν θεῶν αὐτῶν χρυσίον καὶ ἀργύριον εἰς ἑαυτοὺς καταναλοῦσι.
\VS{11}Δώσουσι δὲ ἀπʼ αὐτῶν καὶ ταῖς ἐπὶ τοῦ στέγους πόρναις· κοσμοῦσί τε αὐτοὺς, ὡς ἀνθρώπους, τοῖς ἐνδύμασι, θεοὺς ἀργυροῦς, καὶ θεοὺς χρυσοῦς, καὶ ξυλίνους.
\par }{\PP \VS{12}Οὗτοι δὲ οὐ διασώζονται ἀπὸ ἰοῦ καὶ βρωμάτων, περιβεβλημένων αὐτῶν ἱματισμὸν πορφυροῦν.
\VS{13}Ἐκμάσσονται τὸ πρόσωπον αὐτῶν διὰ τὸν ἐκ τῆς οἰκίας κονιορτὸν, ὅς ἐστι πλείων ἐπʼ αὐτοῖς.
\VS{14}Καὶ σκῆπτρον ἔχει ὡς ἄνθρωπος κριτὴς χώρας, ὃς τὸν εἰς αὐτὸν ἁμαρτάνοντα οὐκ ἀνελεῖ.
\VS{15}Ἔχει δὲ ἐγχειρίδιον δεξιᾷ, καὶ πέλεκυν· ἑαυτὸν δὲ ἐκ πολέμον καὶ λῃστῶν οὐκ ἐξελεῖται.
\VS{16}Ὅθεν γνώριμοί εἰσιν οὐκ ὄντες θεοί· μὴ οὖν φοβηθῆτε αὐτούς.
\par }{\PP \VS{17}Ὥσπερ γὰρ σκεῦος ἀνθρώπου συντριβὲν ἀχρεῖον γῖνεται, τοιοῦτοι ὑπάρχουσιν οἱ θεοὶ αὐτῶν, καθιδρυμένων αὐτῶν ἐν τοῖς οἴκοις· οἱ ὀφθαλμοὶ αὐτῶν πλήρεις εἰσὶ κονιορτοῦ ἀπὸ τῶν ποδῶν τῶν εἰσπορευομένων.
\VS{18}Καὶ ὥσπερ τινὶ ἠδικηκότι βασιλέα, περιπεφραγμέναι εἰσὶν αἱ αὐλαὶ, ὡς ἐπὶ θανάτῳ ἀπηγμένῳ· τοὺς οἴκους αὐτῶν ὀχυροῦσιν οἱ ἱερεῖς θυρώμασί τε καὶ κλείθροις καὶ μοχλοῖς, ὅπως ὑπὸ τῶν λῃστῶν μὴ συληθῶσι.
\par }{\PP \VS{19}Λύχνους καίουσι, καὶ πλείους ἢ ἑαυτοῖς, ὧν οὐδένα δύνανται ἰδεῖν.
\VS{20}Ἔστι μὲν ὥσπερ δοκὸς τῶν ἐκ τῆς οἰκίας, τὰς δὲ καρδίας αὐτῶν φασιν ἐκλείχεσθαι τῶν ἀπὸ τῆς γῆς ἑρπετῶν, κατεσθόντων αὐτούς τε καὶ τὸν ἱματισμὸν αὐτῶν οὐκ αἰσθάνονται·
\VS{21}Μεμελανωμένοι τὸ πρόσωπον αὐτῶν ἀπὸ τοῦ καπνοῦ τοῦ ἐκ τῆς οἰκίας.
\VS{22}Ἐπὶ τὸ σῶμα αὐτῶν καὶ ἐπὶ τὴν κεφαλὴν αὐτῶν ἐφίπτανται νυκτερίδες, χελιδόνες, καὶ τὰ ὄρνεα, ὡσαύτως δὲ καὶ οἱ αἴλουροι.
\VS{23}Ὅθεν γνώσεσθε ὅτι οὐκ εἰσὶ θεοί· μὴ οὖν φοβεῖσθε αὐτά.
\par }{\PP \VS{24}Τὸ γὰρ χρυσίον ὃ περίκεινται εἰς κάλλος, ἐὰν μή τις ἐκμάξῃ τὸν ἰὸν, οὐ μὴ στίλψωσιν, οὐδὲ γὰρ ὅτε ἐχωνεύοντο, ᾐσθάνοντο.
\VS{25}Ἐκ πάσης τιμῆς ἠγορασμένα ἐστὶν, ἐν οἷς οὐκ ἔστι πνεῦμα.
\VS{26}Ἄνευ ποδῶν ἐπʼ ὤμοις φέρονταὶ, ἐνδεικνύμενοι τὴν ἑαυτῶν ἀτιμίαν τοῖς ἀνθρώποις.
\par }{\PP \VS{27}Αἰσχύνονταί τε καὶ οἱ θεραπεύοντες αὐτὰ, διὰ τὸ, εἴποτε ἐπὶ τὴν γῆν πέσῃ, μὴ διʼ αὐτῶν ἀνίστασθαι, μήτε ἐάν τις αὐτὸ ὀρθὸν στήσῃ, διʼ ἑαυτοῦ κινηθήσεται, μητε ἐὰν κλιθῇ, οὐ μὴ ὀρθωθῇ, ἀλλʼ ὥσπερ νεκροῖς τὰ δῶρα αὐτοῖς παρατίθεται.
\par }{\PP \VS{28}Τὰς δὲ θυσίας αὐτῶν ἀποδόμενοι οἱ ἱερεῖς αὐτῶν καταχρῶνται· ὡσαύτως δὲ καὶ αἱ γυναῖκες ἀπʼ αὐτῶν ταριχεύουσαι, οὐτε πτωχῷ οὔτε ἀδυνάτῳ μὴ μεταδῶσι.
\VS{29}Τῶν θυσιῶν αὐτῶν ἀποκαθημένη καὶ λεχὼ ἅπτονται· γνόντες οὖν ἀπὸ τούτων ὅτι οὐκ εἰσὶ θεοὶ, μὴ φοβηθῆτε αὐτούς.
\VS{30}Πόθεν γὰρ κληθείησαν θεοί; ὅτι γυναῖκες παρατιθέασι θεοῖς ἀργυροῖς και χρυσοῖς καὶ ξυλίνοις.
\VS{31}Καὶ ἐν τοῖς οἴκοις αὐτῶν οἱ ἱερεῖς διφρεύουσιν, ἔχοντες τοὺς χιτῶνας διεῤῥωγότας, καὶ τὰς κεφαλὰς καὶ τοὺς πώγωνας ἐξυρημένους, ὧν αἱ κεφαλαὶ ἀκάλυπτοί εἰσιν.
\VS{32}Ὠρύονται δὲ βοῶντες ἐναντίον τῶν θεῶν αὐτῶν, ὥσπερ τινὲς ἐν περιδείπνῳ νεκροῦ.
\par }{\PP \VS{33}Ἀπὸ τοῦ ἱματισμοῦ αὐτῶν ἀφελόμενοι οἱ ἱερεῖς, ἐνδύσουσι τὰς γυναῖκας αὐτῶν καὶ τὰ παιδία.
\VS{34}Οὔτε ἐὰν κακὸν πάθωσιν ὑπό τινος, οὔτε ἐὰν ἀγαθὸν, δυνήσονται ἀνταποδοῦναι· οὔτε καταστῆσαι βασιλέα δύνανται, οὔτε ἀφελέσθαι.
\VS{35}Ὡσαύτως οὔτε πλοῦτον οὔτε χαλκὸν οὐ μὴ δύνωνται διδόναι· ἐάν τις εὐχὴν αὐτοῖς εὐξάμενος μὴ ἀποδῷ, οὐ μὴ ἐπιζητήσωσιν.
\VS{36}Ἐκ θανάτου ἄνθρωπον οὐ μὴ ῥύσωνται, οὔτε ἥττονα ἀπὸ ἰσχυροῦ μὴ ἐξέλωνται.
\VS{37}Ἄνθρωπον τυφλὸν εἰς ὅρασιν οὐ μὴ περιστήσωσιν, ἐν ἀνάγκῃ ἄνθρωπον ὄντα οὐ μὴ ἐξέλωνται.
\VS{38}Χήραν οὐ μὴ ἐλεήσωσιν, οὔτε ὀρφανὸν εὖ ποιήσωσι.
\par }{\PP \VS{39}Τοῖς ἀπὸ τοῦ ὄρους λίθοις ὡμοιωμὲνοι εἰσὶ τὰ ξύλινα, καὶ τὰ περίχρυσα, καὶ τὰ περιάργυρα, οἱ δὲ θεραπεύοντες αὐτὰ καταισχυνθήσονται.
\par }{\PP \VS{40}Πῶς οὖν νομιστέον ἢ κλητέον ὑπάρχειν αὐτοὺς θεοὺς, ἔτι δὲ καὶ αὐτῶν τῶν Χαλδαίων ἀτιμαζόντων αὐτά;
\VS{41}Οἳ ὅταν ἴδωσιν ἐνεὸν μὴ δυνάμενον λαλῆσαι, προσενεγκάμενοι τὸν Βῆλον, ἀξιοῦσι φωνῆσαι, ὠς δυνατοῦ ὄντος αὐτοῦ αἰσθὲσθαι.
\VS{42}Καὶ οὐ δύνανται αὐτοὶ νοήσαντες καταλιπεῖν αὐτὰ, αἴσθησιν γὰρ οὐκ ἐχουσιν.
\par }{\PP \VS{43}Αἱ δὲ γυναῖκες περιθέμεναι σχοινὶα, ἐν ταῖς ὁδοῖς ἐγκάθηνται, θυμιῶσαι τὰ πίτυρα· ὅτας δέ τις αὐτῶν ἐφελκυσθεῖσα ὑπό τινος τῶν παραπορευομένων κοιμηθῇ, τὴν πλησίον ὀνειδίζει, ὅτι οὐκ ἠξίωται ὥσπερ καὶ αὐτὴ, οὔτε τὸ σχοινίον αὐτῆς διεῤῥάγη.
\VS{44}Πάντα τὰ γενόμενα ἐν αὐτοῖς ἐστι ψευδῆ· πῶς οὖν νομιστέον ἢ κλητέον ὡς θεοὺς αὐτοὺς ὑπάρχειν;
\par }{\PP \VS{45}Ὑπὸ τεκτόνων καὶ χρυσοχόων κατεσκευασμένα εἰσίν· οὐθὲν ἄλλο μὴ γὲνηται, ἢ ὃ βούλονται οἱ τεχνίται αὐτὰ γενέσθαι.
\VS{46}Αὐτοί τε οἱ κατασκευάζοντες αὐτὰ οὐ μὴ γένωνται πολυχρόνιοι· πῶς τε δὴ μέλλει τὰ ὑπʼ αὐτῶν κατασκευασθέντα;
\par }{\PP \VS{47}Κατέλιπον γὰρ ψεύδη καὶ ὄνειδος τοῖς ἐπιγινομένοις.
\VS{48}Ὅταν γὰρ ἐπέλθῃ ἐπʼ αὐτὰ πόλεμος καὶ κακὰ, βουλεύονται πρὸς ἑαυτοὺς οἱ ἱερεῖς, ποῦ συναποκρυβῶσι μετʼ αὐτῶν.
\VS{49}Πῶς οὖν οὐκ ἔστιν αἰσθέσθαι ὅτι οὐκ εἰσί θεοί, οἳ οὔτε σώζουσιν ἑαυτοὺς ἐκ πολέμου, οὔτε ἐκ κακῶν;
\VS{50}Ὑπάρχοντα γὰρ ξύλινα καὶ περίχρυσα καὶ περιάργυρα, γνωσθήσεται μετὰ ταῦτα ὅτι ἐστὶ ψευδῆ.
\VS{51}Τοῖς ἔθνεσι πᾶσι τοῖς τε βασιλεῦσι φανερὸν ἔσται ὃτι οὐκ εἰσὶ θεοὶ, ἀλλὰ ἔργα χειρῶν ἀνθρώπων, καὶ οὐδὲν Θεοῦ ἔργον ἐν αὐτοῖς ἐστι.
\par }{\PP \VS{52}Τίνι οὖν γνωστέον ἐστὶν ὅτι οὐκ εἰσὶ θεοί;
\VS{53}Βασιλὲα γὰρ χώρας οὐ μὴ ἀναστήσωσιν, οὔτε ὑετὸν ἀνθρώποις οὐ μὴ δῶσι.
\VS{54}Κρίσιν τε οὐ μὴ διακρίνωσιν ἑαυτῶν, οὐδὲ μὴ ῥύσωνται ἀδίκημα, ἀδὺνατοι ὄντες· ὥσπερ γὰρ κορῶναι ἀναμέσον τοῦ οὐρανοῦ καὶ τῆς γῆς.
\par }{\PP \VS{55}Καὶ γὰρ ὅταν ἐμπέσῃ εἰς οἰκίαν θεῶν ξυλίνων ἡ περιχρύσων ἢ περιαργύρων πῦρ, οἱ μὲν ἱερεῖν φεύξονται καὶ διασωθήσονται, αὐτοὶ δὲ ὥσπερ δοκοὶ μέσοι κατακαυθήσονται.
\VS{56}Βασιλεῖ δὲ καὶ πολεμίοις οὐ μὴ ἀντιστῶσι· πῶς οὖν ἐκδεκτέον ἢ νομιστέον ὅτι εἰσὶ θεοί;
\VS{57}Οὔτε ἀπὸ κλεπτῶν, οὔτε ἀπὸ λῃστῶν οὐ μὴ διασωθῶσι θεοὶ ξύλινον, καὶ περιάργυροι, καὶ περίχρυσοι·
\VS{58}ὧν οἱ ἰσχύοντες περιελοῦνται τὸ χρυσίον καὶ τὸ ἀργύριον, καὶ τὸν ἱματισμὸν τὸν περικείμενον αὐτοῖς ἀπελεύσονται ἔχοντες, οὔτε ἑαυτοῖς οὐ μὴ βοηθήσωσιν.
\par }{\PP \VS{59}Ὥστε κρεῖσσον εἶναι βασιλέα ἐπιδεικνύμενον τὴν ἑαυτοῦ ἀνδρείαν, ἢ σκεῦος ἐν οἰκίᾳ χρήσιμον ἐφʼ ᾧ κεχρήσεται ὁ κεκτημένος, ἢ οἱ ψευδεῖς θεοί· ἢ καὶ θύρα ἐν οἰκίᾳ διασώζουσα τὰ ἐν αὐτῇ ὄντα, ἢ οἱ ψευδεῖς θεοί· καὶ ξύλινος στύλος ἐν βασιλείοις, ἤ οἱ ψευδεῖς θεοί.
\par }{\PP \VS{60}Ἥλιος μὲν γὰρ καὶ σελήνη καὶ ἄστρα ὄντα λαμπρὰ, καὶ ἀποστελλόμενα ἐπὶ χρείας, εὐήκοά εἰσιν.
\VS{61}Ὡσαύτως καὶ ἀστραπὴ ὅταν ἐπιφανῇ, εὔοπτός ἐστι· τὸ δʼ αὐτὸ καὶ πνεῦμα ἐν πάσῃ χώρᾳ πνεῖ.
\VS{62}Καὶ νεφέλαις ὅταν ἐπιταγῆ ὑπὸ τοῦ Θεοῦ ἐπιπορεύεσθαι ἐπιπορεύεσθαι ἐφʼ ὅλην τὴν οἰκουμένην, συντελοῦσι τὸ ταχθέν.
\VS{63}Τό, τε πῦρ ἐξαποσταλὲν ἄνωθεν ἐξαναλῶσαι ὄρη καὶ δρυμοὺς, ποιεῖ τὸ συνταχθὲν· ταῦτα δὲ οὔτε ταῖς εἰδέαις οὔτε ταῖς δυνάμεσιν αὐτῶν ἀφωμοιωμένα ἐστίν.
\par }{\PP \VS{64}Ὅθεν οὔτε νομιστέον οὔτε κλητέον ὑπάρχειν αὐτοὺς θεοὺς, οὐ δυνατῶν ὄντων αὐτῶν οὔτε κλητέον κρίναι, οὔτε εὖ ποιῆσαι ἀνθρώποις.
\VS{65}Γνόντες οὖν ὅτι οὐκ εἰσὶ θεοί, μὴ φοβηθῆτε αὐτούς·
\par }{\PP \VS{66}Οὔτε γὰρ βασιλεῦσιν οὐ μὴ καταράσωνται, οὔτε μὴ εὐλογήσωσι.
\VS{67}Σημεῖά τε ἐν ἔθνεσιν ἐν οὐρανῷ οὐ μὴ δείξωσιν, οὐδὲ ὡς ὁ ἥλιος λάμψουσιν, οὔτε φωτιοῦσιν ὡς ἡ σελήνη.
\VS{68}Τὰ θηρία αὐτῶν ἐστι κρείττω, ἃ δύνανται ἐκφυγόντα εἰς σκέπην ἑαυτὰ ὠφελῆσαι.
\VS{69}Κατʼ οὐδένα οὖν τρόπον ἡμῖν ἑστι φανερὸν ὅτι εἰσὶ θεοί· διὸ μὴ φοβηθῆτε αὐτούς.
\par }{\PP \VS{70}Ὥσοερ γὰρ ἐν σικυηράτῳ προβασκάνιον οὐδὲν φυλάσσον, οὕτως οἱ θεοὶ αὐτῶν εἰσι ξύλινοι καὶ περίχρυσον καὶ περιάργυροι.
\VS{71}τὸν αὐτὸν τρόπον καὶ τῇ. ἐν κήπῳ ῥάμνῳ, ἐφʼ ἧς πᾶν ὄρνεον ἐπικάθηται, ὡσαύτως δὲ καὶ νεκρῷ ἐῤῥιμμένῳ ἐν σκότει ἀφωμοίωνται οἱ θεοὶ αὐτῶν ξύλινοι καὶ περίχρυσοι καὶ περιάργυροι.
\VS{72}Ἀπό τε τῆς πορφύρας καὶ τῆς μαρμάρου τῆς ἐπʼ αὐτοὺς σηπομένης γνωσθήσονται ὅτι οὐκ εἰσὶ θεοί· αὐτά τε ἐξ ὑστέρου βρωθήσονται, καὶ ἔσται ὄνειδος ἐν τῇ χώρᾳ.
\par }{\PP \VS{73}Κρεῖσσον οὖ ἄνθρωπος δίκαιος οὐκ ἔχων εἴδωλα, ἔσται γὰρ μακρὰν ἀπὸ ὀνειδισμοῦ.
\par }