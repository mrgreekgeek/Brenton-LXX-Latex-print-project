\NormalFont\ShortTitle{ΠΑΡΑΛΕΙΠΟΜΕΝΩΝ Α}
{\MT ΠΑΡΑΛΕΙΠΟΜΕΝΩΝ Α

\par }\ChapOne{1}{\PP \VerseOne{1}ἈΔΑΜ, Σὴθ, Ἐνὼς,
\VS{2}καὶ Καϊνᾶν, Μαλελεὴλ, Ἰάρεδ,
\VS{3}Ἐνὼχ, Μαθουσάλα, Λάμεχ,
\VS{4}Νῶε· υἱοὶ Νῶε, Σὴμ, Χὰμ, Ἰάφεθ.
\par }{\PP \VS{5}Υἱοὶ Ἰάφεθ, Γαμὲρ, Μαγὼγ, Μαδαῒμ, Ἰωϋὰν, Ἑλισὰ, Θοβὲλ, Μοσὸχ, καὶ Θίρας.
\VS{6}Καὶ οἱ υἱοὶ Γαμὲρ, Ἀσχανὰζ, καὶ Ῥιφὰθ, καὶ Θοργαμά.
\VS{7}Καὶ οἱ υἱοὶ Ἰωϋὰν, Ἑλισὰ, καὶ Θαρσὶς, Κίτιοι, καὶ Ῥόδιοι.
\par }{\PP \VS{8}Καὶ υἱοὶ Χὰμ, Χοὺς, καὶ Μεσραῒμ, Φοὺδ, καὶ Χαναάν.
\VS{9}Καὶ υἱοὶ Χοὺς, Σαβὰ, καὶ Εὐιλὰ, καὶ Σαβαθὰ, καὶ Ῥεγμὰ, καὶ Σεβεθαχά· καὶ υἱοὶ Ῥεγμὰ, Σαβὰ, καὶ Δαδάν.
\VS{10}Καὶ Χοὺς ἐγέννησε τὸν Νεβρώδ· οὗτος ἤρξατο εἶναι γίγας κυνηγὸς ἐπὶ τῆς γῆς.
\par }{\PP \VS{17}Υἱοὶ Σὴμ, Αἰλὰμ, καὶ Ἀσσοὺρ,
\VS{24}καὶ Ἀρφαξὰδ, Σάλα,
\VS{25}Ἔβερ, Φαλὲγ, Ῥαγὰν,
\VS{26}Σεροὺχ, Ναχὼρ, Θάῤῥα,
\VS{27}Ἁβραάμ·
\par }{\PP \VS{28}Υἱοὶ δὲ Ἁβραὰμ, Ἰσαὰκ, καὶ Ἰσμαήλ.
\VS{29}Αὗται δὲ αἱ γενέσεις αὐτῶν· πρωτότοκος Ἰσμαὴλ, Ναβαιὼθ, καὶ Κηδὰρ, Ναβδεὴλ, Μασσὰμ,
\VS{30}Μασμὰ, Ἰδουμὰ, Μασσὴ, Χονδὰν, Θαιμὰν,
\VS{31}Ἰεττοὺρ, Ναφὲς, Κεδμά· οὗτοι υἱοὶ Ἰσμαήλ.
\par }{\PP \VS{32}Καὶ υἱοὶ Χεττούρας παλλακῆς Ἁβραάμ· καὶ ἔτεκεν αὐτῷ τὸν Ζεμβρὰμ, Ἰεξὰν, Μαδιὰμ, Μαδὰμ, Σοβὰκ, Σωέ· καὶ υἱοὶ Ἰεξὰν, Δαιδὰν, καὶ Σαβαΐ.
\VS{33}Καὶ υἱοὶ Μαδιὰμ, Γεφὰρ, καὶ Ὀφὲρ, καὶ Ἐνὼχ, καὶ Ἀβιδὰ, καὶ Ἐλδαδά· πάντες οὗτοι υἱοὶ Χεττούρας.
\par }{\PP \VS{34}Καὶ ἐγέννησεν Ἁβραὰμ τὸν Ἰσαάκ· καὶ υἱοὶ Ἰσαὰκ, Ἰακὼβ, καὶ Ἡσαῦ.
\VS{35}Υἱοὶ Ἡσαῦ, Ἐλιφὰζ, καὶ Ῥαγουὴλ, καὶ Ἰεοὺλ, καὶ Ἰεγλὸμ, καὶ Κορέ.
\VS{36}Υἱοὶ Ἐλιφὰζ, Θαιμὰν, καὶ Ὠμὰρ, Σωφὰρ, καὶ Γοωθὰμ, καὶ Κενὲζ, καὶ Θαμνὰ, καὶ Ἀμαλήκ.
\VS{37}Καὶ υἱοὶ Ῥαγουὴλ, Ναχὲς, Ζαρὲ, Σομὲ, καὶ Μοζέ.
\VS{38}Υἱοὶ Σηῒρ, Λωτὰν, Σωβὰλ, Σεβεγὼν, Ἀνὰ, Δησὼν, Ὠσὰρ, καὶ Δισάν.
\VS{39}Καὶ υἱοὶ Λωτὰν, Χοῤῥὶ, καὶ Αἰμὰν· ἀδελφὴ δὲ Λωτὰν Θαμνά. Υἱοὶ Σωβὰλ, Ἀλὼν, Μαχανὰθ, Ταιβὴλ, Σωφὶ, καὶ Ὠνὰν·
\VS{40}υἱοὶ δὲ Σεβεγὼν, Ἀῒθ, καὶ Σωνάν.
\VS{41}Υἱοὶ Σωνὰν. Δαισών· υἱοὶ δὲ Δαισὼν, Ἐμερὼν, καὶ Ἀσεβὼν, καὶ Ἰεθρὰμ, καὶ Χαῤῥάν.
\VS{42}Καὶ υἱοὶ Ὡσὰρ, Βαλαὰμ, καὶ Ζουκὰμ, καὶ Ἀκάν· υἱοὶ Δισὰν, Ὢς καὶ Ἀράν.
\par }{\PP \VS{43}Καὶ οὗτοι οἱ βασιλεῖς αὐτῶν· Βαλὰκ υἱὸς Βεὼρ, καὶ ὄνομα τῇ πόλει αὐτοῦ Δενναβά.
\VS{44}Καὶ ἀπέθανε Βαλὰκ, καὶ ἐβασίλευσεν ἀντʼ αὐτοῦ Ἰωβὰβ υἱὸς Ζαρὰ ἐκ Βοσόῤῥας.
\VS{45}Καὶ ἀπέθανεν Ἰωβὰβ καὶ ἐβασίλευσεν ἀντʼ αὐτοῦ Ἁσὸμ ἐκ γῆς Θαιμανών.
\VS{46}Καὶ ἀπέθανεν Ἁσόμ, καὶ ἐβασίλευσεν ἀντʼ αὐτοῦ Ἀδὰδ υἱὸς Βαρὰδ, ὁ πατάξας Μαδιὰμ ἐν τῷ πεδίῳ Μωὰβ· καὶ ὄνομα τῇ πόλει αὐτοῦ Γεθαίμ.
\VS{47}Καὶ ἀπέθανεν Ἁδὰδ, καὶ ἐβασίλευσεν ἀντʼ αὐτοῦ Σεβλὰ ἐκ Μασεκκάς.
\VS{48}Καὶ ἀπέθανε Σεβλὰ, καὶ ἐβασίλευσεν ἀντʼ αὐτοῦ Σαοὺλ ἐκ Ῥωβὼθ τῆς παρὰ ποταμόν.
\VS{49}Καὶ ἀπέθανε Σαοὺλ, καὶ ἐβασίλευσεν ἀντʼ αὐτοῦ Βαλαεννὼρ υἱὸς Ἀχωβώρ.
\VS{50}Καὶ ἀπέθανε Βαλαεννὼρ, καὶ ἐβασίλευσεν ἀντʼ αὐτοῦ Ἀδὰδ υἱὸς Βαρὰδ, καὶ ὄνομα τῇ πόλει αὐτοῦ, Φογώρ.
\par }{\PP \VS{51}Ἡγεμόνες Ἐδώμ· ἡγεμὼν Θαμνὰ, ἡγεμὼν Γωλαδὰ, ἡγεμὼν Ἰεθὲρ,
\VS{52}ἡγεμὼν Ἐλιβαμὰς, ἡγεμὼν Ἠλὰς, ἡγεμὼν Φινὼν,
\VS{53}ἡγεμὼν Κενέζ, ἡγεμὼν Θαιμὰν, ἡγεμὼν Βαβσὰρ,
\VS{54}ἡγεμὼν Μαγεδιὴλ, ἡγεμὼν Ζαφωΐν· οὗτοι ἡγεμόνες Ἐδώμ.

\par }\Chap{2}{\PP \VerseOne{1}Ταῦτα τὰ ὀνόματα τῶν υἱῶν Ἰσραήλ· Ῥουβὴν, Συμεὼν, Λευὶ, Ἰούδα, Ἰσσάχαρ, Ζαβουλὼν,
\VS{2}Δὰν, Ἰωσὴφ, Βενιαμὶν, Νεφθαλὶ, Γὰδ, Ἀσήρ.
\par }{\PP \VS{3}Υἱοὶ Ἰούδα, Ἢρ, Αὐνὰν, Σηλώμ· τρεῖς ἐγεννήθησαν αὐτῷ ἐκ τῆς θυγατρὸς Σαύας τῆς Χανανίτιδος· καὶ ἦν Ἢρ ὁ πρωτότοκος Ἰούδα πονηρὸς ἐναντίον Κυρίου, καὶ ἀπέκτεινεν αὐτόν·
\VS{4}Καὶ Θάμαρ ἡ νύμφη αὐτοῦ ἔτεκεν αὐτῷ τὸν Φαρὲς, καὶ τὸν Ζαρα· πάντες υἱοὶ Ἰούδα πέντε.
\par }{\PP \VS{5}Υἱοὶ Φαρὲς, Ἐσρὼμ, καὶ Ἰεμουήλ.
\VS{6}Καὶ υἱοὶ Ζαρὰ, Ζαμβρὶ, καὶ Αἰθὰμ, καὶ Αἰμουὰν, καὶ Καλχὰλ, καὶ Δαρὰδ, πάντες πέντε.
\par }{\PP \VS{7}Καὶ υἱοὶ Χαρμὶ, Ἀχὰρ ὁ ἐμποδοστάτης Ἰσραὴλ, ὃς ἠθέτησεν εἰς τὸ ἀνάθεμα.
\VS{8}Καὶ υἱοὶ Αἰθὰμ, Ἀζαρίας.
\VS{9}Καὶ υἱοὶ Ἐσρὼμ οἳ ἐτέχθησαν αὐτῷ, ὁ Ἱεραμεὴλ, καὶ ὁ Ἀρὰμ, καὶ ὁ Χαλέβ.
\par }{\PP \VS{10}Καὶ Ἀρὰμ ἐγέννησε τὸν Ἀμιναδὰβ, καὶ Ἀμιναδὰβ ἐγέννησε τὸν Ναασσὼν ἄρχοντα οἴκου Ἰούδα,
\VS{11}καὶ Ναασσὼν ἐγέννησε τὸν Σαλμὼν, καὶ Σαλμὼν ἐγέννησε τὸν Βοὸζ,
\VS{12}καὶ Βοὸζ ἐγέννησε τὸν Ὠβὴδ, καὶ Ὠβὴδ ἐγέννησε τὸν Ἰεσσαὶ,
\VS{13}καὶ Ἰεσσαὶ ἐγέννησε τὸν πρωτότοκον αὐτοῦ τὸν Ἐλιὰβ, Ἀμιναδὰβ ὁ δεύτερος, Σαμαὰ ὁ τρίτος,
\VS{14}Ναθαναὴλ ὁ τέταρτος, Ζαβδαῒ ὁ πέμπτος,
\VS{15}Ἀσὰμ ὁ ἕκτος, Δαυὶδ ὁ ἕβδομος.
\VS{16}Καὶ ἡ ἀδελφὴ αὐτῶν Σαρουία, καὶ Ἀβιγαία· καὶ υἱοὶ Σαρουία, Ἀβισὰ, καὶ Ἰωὰβ, καὶ Ἀσαὴλ, τρεῖς.
\VS{17}Καὶ Ἀβειγαία ἐγέννησε τὸν Ἀμεσσάβ· καὶ πατὴρ Ἀμεσσὰβ Ἰοθὸρ ὁ Ἰσμαηλίτης.
\par }{\PP \VS{18}Καὶ Χαλὲβ υἱὸς Ἐσρὼμ ἔλαβε τὴν Γαζουβὰ γυναῖκα, καὶ τὴν Ἰεριώθ· καὶ οὗτοι υἱοὶ αὐτῆς, Ἰασὰρ, καὶ Σουβὰβ, καὶ Ἀρδών.
\VS{19}Καὶ ἀπέθανε Γαζουβὰ, καὶ ἔλαβεν ἑαυτῷ Χαλὲβ τὴν Ἐφρὰθ, καὶ ἔτεκεν αὐτῷ τὸν Ὥρ.
\VS{20}Καὶ Ὣρ ἐγέννησε τὸν Οὐρί· καὶ Οὐρὶ ἐγέννησε τὸν Βεσελεήλ.
\VS{21}Καὶ μετὰ ταῦτα εἰσῆλθεν Ἐσρὼν πρὸς τὴν θυγατέρα Μαχὶρ πατρὸς Γαλαὰδ, καὶ αὐτὸς ἔλαβεν αὐτὴν, καὶ αὐτὸς ἑξηκονταπέντε ἐτῶν ἦν· καὶ ἔτεκεν αὐτῷ τὸν Σερούχ.
\VS{22}Καὶ Σεροὺχ ἐγέννησε τὸν Ἰαΐρ. καὶ ἦσαν αὐτῷ εἴκοσι καὶ τρεῖς πόλεις ἐν τῇ Γαλαάδ.
\VS{23}Καὶ ἔλαβε Γεδσοὺρ καὶ Ἀρὰμ τὰς κώμας Ἰαῒρ ἐξ αὐτῶν, τὴν Κανὰθ καὶ τὰς κώμας αὐτῆς, ἑξήκοντα πόλεις· πᾶσαι αὗται υἱῶν Μαχὶρ πατρὸς Γαλαάδ.
\VS{24}Καὶ μετὰ τὸ ἀποθανεῖν Ἐσρὼν, ἦλθε Χαλὲβ εἰς Ἐφραθά· καὶ ἡ γυνὴ Ἐσρὼν Ἀβιά· καὶ ἔτεκεν αὐτῷ τὸν Ἀσχὼ πατέρα Θεκωέ.
\par }{\PP \VS{25}καὶ ἦσαν οἱ υἱοὶ Ἱεραμεὴλ πρωτοτόκου Ἐσρὼν, ὁ πρωτότοκος Ῥὰμ, καὶ Βαναὰ, καὶ Ἀρὰμ, καὶ Ἀσὰν ἀδελφὸς αὐτοῦ.
\VS{26}Καὶ ἦν γυνὴ ἑτέρα τῷ Ἱεραμεὴλ, καὶ ὄνομα αὐτῇ Ἀτάρα· αὕτη ἐστὶ μήτηρ Ὀζόμ.
\VS{27}Καὶ ἦσαν υἱοὶ Ῥὰμ πρωτοτόκου Ἱεραμεὴλ, Μαὰς, καὶ Ἰαμὶν, καὶ Ἀκόρ.
\VS{28}Καὶ ἦσαν υἱοὶ Ὀζὸμ, Σαμαῒ, καὶ Ἰαδαέ· καὶ υἱοὶ Σαμαῒ, Ναδὰβ καὶ Ἀβισούρ.
\VS{29}Καὶ ὄνομα τῆς γυναικὸς Ἀβισοὺρ, Ἀβιχαία· καὶ ἔτεκεν αὐτῷ τὸν Ἀχαβὰρ, καὶ τὸν Μωήλ.
\VS{30}Καὶ υἱοὶ Ναδὰβ, Σαλὰδ, καὶ Ἀπφαίν· καὶ ἀπέθανε Σαλὰδ οὐκ ἔχων τέκνα.
\VS{31}Καὶ υἱοὶ Ἀπφαὶν, Ἰσεμιήλ· καὶ υἱοὶ Ἰσεμιὴλ, Σωσάν· καὶ υἱοὶ Σωσὰν, Δαδαί.
\VS{32}Καὶ υἱοὶ Δαδαὶ, Ἀχισαμὰς, Ἰεθὲρ, Ἰωνάθαν· καὶ ἀπέθανεν Ἰεθὲρ οὐκ ἔχων τέκνα.
\VS{33}Καὶ υἱοὶ Ἰωνάθαν, Φαλὲθ, καὶ Ὁζάμ· οὗτοι ἦσαν υἱοὶ Ἱεραμεήλ.
\par }{\PP \VS{34}Καὶ οὐκ ἦσαν τῷ Σωσὰν υἱοὶ, ἀλλʼ ἢ θυγατέρες· καὶ τῷ Σωσὰν παῖς Αἰγύπτιος, καὶ ὄνομα αὐτῷ Ἰωχήλ.
\VS{35}Καὶ ἔδωκε Σωσὰν τὴν θυγατέρα αὐτοῦ τῷ Ἰωχὴλ παιδὶ αὐτοῦ εἰς γυναῖκα, καὶ ἔτεκεν αὐτῷ τὸν Ἐθὶ,
\VS{36}καὶ Ἐθὶ ἐγέννησε τὸν Ναθὰν, καὶ Ναθὰν ἐγέννησε τὸν Ζαβὲδ,
\VS{37}καὶ Ζαβὲδ ἐγέννησε τὸν Ἀφαμὴλ, καὶ Ἀφαμὴλ ἐγέννησε τὸν Ὠβὴδ,
\VS{38}καὶ Ὠβὴδ ἐγέννησε τὸν Ἰηοὺ, καὶ Ἰηοὺ ἐγέννησε τὸν Ἀζαρίαν,
\VS{39}καὶ Ἀζαρίας ἐγέννησε τὸν Χελλὴς, καὶ Χελλὴς ἐγέννησε τὸν Ἐλεασὰ,
\VS{40}καὶ Ἐλεασὰ ἐγέννησε τὸν Σοσομαῒ, καὶ Σοσομαῒ ἐγέννησε τὸν Σαλοὺμ,
\VS{41}καὶ Σαλοὺμ ἐγέννησε τὸν Ἰεχεμίαν, καὶ Ἰεχεμίας ἐγέννησε τὸν Ἐλισαμὰ, καὶ Ἐλισαμὰ ἐγέννησε τὸν Ἰσμαήλ.
\par }{\PP \VS{42}Καὶ υἱοὶ Χαλὲβ ἀδελφοῦ Ἱεραμεὴλ, Μαρισὰ ὁ πρωτότοκος αὐτοῦ· οὗτος πατὴρ Ζίφ· καὶ υἱοὶ Μαρισὰ πατρὸς Χεβρών.
\VS{43}Καὶ υἱοὶ Χεβρὼν, Κορὲ, καὶ Θαπφοὺς, καὶ Ῥεκὸμ, καὶ Σαμαά.
\VS{44}Καὶ Σαμαὰ ἐγέννησε τὸν Ῥαὲμ πατέρα Ἰεκλὰν, καὶ Ἰεκλὰν ἐγέννησε τὸν Σαμαΐ.
\VS{45}Καὶ υἱὸς αὐτοῦ Μαών· καὶ Μαὼν πατὴρ Βαιθσούρ.
\VS{46}Καὶ Γαιφὰ ἡ παλλακὴ Χαλὲβ ἐγέννησε τὸν Ἀρὰμ, καὶ τὸν Μοσὰ, καὶ τὸν Γεζουέ.
\VS{47}Καὶ υἱοὶ Ἀδδαῒ, Ῥαγὲμ, καὶ Ἰωάθαμ, καὶ Σωγὰρ, καὶ Φαλὲκ, καὶ Γαιφὰ, καὶ Σαγαέ.
\VS{48}Καὶ ἡ παλλακὴ Χαλὲβ Μωχὰ ἐγέννησε τὸν Σαβὲρ, καὶ τὸν Θαράμ.
\VS{49}Καὶ ἐγέννησε Σαγαὲ πατέρα Μαδμηνὰ, καὶ τὸν Σαοὺ πατέρα Μαχαβηνὰ, καὶ πατέρα Γαιβάλ· καὶ θυγάτηρ Χαλὲβ, Ἀσχά.
\par }{\PP \VS{50}Οὗτοι ἦσαν υἱοὶ Χαλέβ· υἱοὶ Ὢρ πρωτοτόκου Ἐφραθά· Σωβὰλ πατὴρ Καριαθιαρὶμ,
\VS{51}Σαλωμὼν πατὴρ Βαιθὰ, Λαμμὼν πατὴρ Βαιθαλαὲμ, καὶ Ἀρὶμ πατὴρ Βεθγεδώρ.
\VS{52}Καὶ ἦσαν υἱοὶ τῷ Σωβὰλ πατρὶ Καριαθιαρὶμ Ἀραὰ, καὶ Αἰσὶ, καὶ Ἀμμανὶθ,
\VS{53}καὶ Οὐμασφαὲ, πόλεις Ἰαῒρ, Αἰθαλὶμ, καὶ Μιφιθὶμ, καὶ Ἡσαμαθὶμ, καὶ Ἡμασαραΐμ· ἐκ τούτων ἐξήλθοσαν οἱ Σαραθαῖοι, καὶ υἱοὶ Ἐσθαάμ.
\VS{54}Υἱοὶ Σαλωμὼν Βαιθαλαὲμ, ὁ Νετωφατὶ, Ἀταρὼθ οἴκου Ἰωὰβ, καὶ ἥμισυ τῆς Μαλαθὶ, Ἠσαρὶ·
\VS{55}Πατριαὶ γραμματέων κατοικοῦντες ἐν Ἰάβις Θαργαθιῒμ, καὶ Σαμαθιῒμ, καὶ Σωχαθίμ· οὗτοι οἱ Κιναῖοι οἱ ἐλθόντες ἐξ Αἱμὰθ πατρὸς οἴκου Ῥηχάβ.

\par }\Chap{3}{\PP \VerseOne{1}Καὶ οὗτοι ἦσαν υἱοὶ Δαυὶδ οἱ τεχθέντες αὐτῷ ἐν Χεβρών· ὁ πρωτότοκος Ἀμνὼν τῇ Ἀχιναὰμ τῇ Ἰεζραηλίτιδι· ὁ δεύτερος Δαμνιὴλ τῇ Ἀβιγαίᾳ τῇ Καρμηλίᾳ·
\VS{2}Ὁ τρίτος Ἀβεσσαλὼμ, υἱὸς Μωχὰ θυγατρὸς Θολμαῒ βασιλέως Γεδσούρ· ὁ τέταρτος Ἀδωνία υἱὸς Ἀγγίθ·
\VS{3}Ὁ πέμπτος Σαφατία τῆς Ἀβιτὰλ· ὁ ἕκτος Ἰεθραὰμ τῇ Ἀγλᾷ γυναικὶ αὐτοῦ.
\VS{4}Ἓξ ἐγεννήθησαν αὐτῷ ἐν Χεβρών· καὶ ἐβασίλευσεν ἐκεῖ ἑπτὰ ἔτη, καὶ ἑξάμηνον· καὶ τριάκοντα καὶ τρία ἔτη ἐβασίλευσεν ἐν Ἱερουσαλήμ.
\VS{5}Καὶ οὗτοι ἐτέχθησαν αὐτῷ ἐν Ἱερουσαλήμ· Σαμαὰ, Σωβὰβ, Νάθαν, καὶ Σαλωμών· τέσσαρες τῇ Βηρσαβεὲ θυγατρὶ Ἀμιήλ·
\VS{6}Καὶ Ἐβαὰρ, καὶ Ἐλισὰ, καὶ Ἐλιφαλὴθ,
\VS{7}καὶ Ναγαὶ, καὶ Ναφὲκ, καὶ Ἰαφιὲ,
\VS{8}καὶ Ἑλισαμὰ, καὶ Ἐλιαδὰ, καὶ Ἐλιφαλὰ, ἐννέα.
\VS{9}Πάντες υἱοὶ Δαυὶδ, πλὴν τῶν υἱῶν τῶν παλλακῶν, καὶ Θήμαρ ἀδελφὴ αὐτῶν.
\par }{\PP \VS{10}Υἱοὶ Σαλωμὼν, Ῥοβοάμ, Ἀβιὰ υἱὸς αὐτοῦ, Ἀσὰ υἱὸς αὐτοῦ, Ἰωσαφὰτ υἱὸς αὐτοῦ,
\VS{11}Ἰωρὰμ υἱὸς αὐτοῦ, Ὀχοζίας υἱὸς αὐτοῦ, Ἰωὰς υἱὸς αὐτοῦ,
\VS{12}Ἀμασίας υἱὸς αὐτοῦ, Ἀζαρίας υἱὸς αὐτοῦ, Ἰωάθαν υἱὸς αὐτοῦ,
\VS{13}Ἄχαζ υἱὸς αὐτοῦ, Ἐζεκίας υἱὸς αὐτοῦ, Μανασσῆς υἱὸς αὐτοῦ,
\VS{14}Ἀμὼν υἱὸς αὐτοῦ, Ἰωσία υἱὸς αὐτοῦ.
\VS{15}Καὶ υἱοὶ Ἰωσία, πρωτότοκος Ἰωανὰν, ὁ δεύτερος Ἰωακεὶμ, ὁ τρίτος Σεδεκίας, ὁ τέταρτος Σαλούμ.
\VS{16}Καὶ υἱοὶ Ἰωακείμ, Ἰεχονίας υἱὸς αὐτοῦ, Σεδεκίας υἱὸς αὐτοῦ.
\VS{17}Καὶ υἱοὶ Ἰεχονία, Ἀσὶρ, Σαλαθιὴλ υἱὸς αὐτοῦ,
\VS{18}Μελχειρὰμ, καὶ Φαδαΐας, καὶ Σανεσὰρ, καὶ Ἰεκιμία, καὶ Ὡσαμὰθ, καὶ Ναβαδίας.
\par }{\PP \VS{19}Καὶ υἱοὶ Φαδαΐας, Ζοροβάβελ, καὶ Σεμεΐ· καὶ υἱοὶ Ζοροβάβελ, Μοσολλὰμ, καὶ Ἀνανία, καὶ Σαλωμεθὶ ἀδελφὴ αὐτῶν,
\VS{20}καὶ Ἀσουβὲ, καὶ Ὀὸλ, καὶ Βαραχὶα, καὶ Ἀσαδία, καὶ Ἀσοβὲδ, πέντε.
\par }{\PP \VS{21}Καὶ υἱοὶ Ἀνανία, Φαλεττία, καὶ Ἰεσίας υἱὸς αὐτοῦ, Ῥαφὰλ υἱὸς αὐτοῦ, Ὀρνὰ υἱὸς αὐτοῦ, Ἀβδία υἱὸς αὐτοῦ, Σεχενίας υἱὸς αὐτοῦ.
\VS{22}Καὶ υἱὸς Σεχενία, Σαμαΐα· καὶ υἱοὶ Σαμαΐα, Χαττοὺς, καὶ Ἰωὴλ, καὶ Βεῤῥὶ, καὶ Νωαδία, καὶ Σαφὰθ, ἕξ.
\par }{\PP \VS{23}Καὶ υἱοὶ Νωαδία, Ἐλιθενὰν, καὶ Ἐζεκία, καὶ Ἐζρικὰμ, τρεῖς.
\par }{\PP \VS{24}Καὶ υἱοὶ Ἐλιθενὰν, Ὀδολία, καὶ Ἑλιασεβὼν, καὶ Φαδαΐα, καὶ Ἀκοὺβ, καὶ Ἰωανὰν, καὶ Δαλααΐα, καὶ Ἀνὰν, ἑπτά.

\par }\Chap{4}{\PP \VerseOne{1}Καὶ υἱοὶ Ἰούδα, Φαρὲς, Ἐσρὼμ, καὶ Χαρμὶ, καὶ Ὢρ, Σουβὰλ,
\VS{2}καὶ Ῥάδα υἱὸς αὐτοῦ· καὶ Σουβὰλ ἐγέννησε τὸν Ἰέθ· καὶ Ἰὲθ ἐγέννησε τὸν Ἀχιμαῒ, καὶ τὸν Λαάδ· αὗται αἱ γενέσεις τοῦ Ἀραθί.
\VS{3}Καὶ οὗτοι υἱοὶ Αἰτὰμ, Ἰεζραὴλ, καὶ Ἰεσμὰν, καὶ Ἰεβδάς· καὶ ὄνομα ἀδελφῆς αὐτῶν Ἐσηλεββών.
\VS{4}Καὶ Φανουὴλ πατὴρ Γεδὼρ, καὶ Ἰαζὴρ πατὴρ Ὠσάν· οὗτοι υἱοὶ Ὢρ τοῦ πρωτοτόκου Ἐφραθὰ πατρὸς Βαιθαλαέν.
\par }{\PP \VS{5}Καὶ τῷ Ἀσοὺρ πατρὶ Θεκωὲ ἦσαν δύο γυναῖκες, Ἀωδὰ, καὶ Θοαδά.
\VS{6}Καὶ ἔτεκεν αὐτῷ Ἀωδὰ τὸν Ὠχαία, καὶ τὸν Ἠφὰλ, καὶ τὸν Θαιμὰν, καὶ τὸν Ἀασθήρ· πάντες οὗτοι υἱοὶ Ἀωδᾶς.
\VS{7}Καὶ υἱοὶ Θοαδᾶς, Σερὲθ, καὶ Σαὰρ, καὶ Ἐσθανάμ.
\VS{8}Καὶ Κωὲ ἐγέννησε τὸν Ἐνὼβ, καὶ τὸν Σαβαθά· καὶ γεννήσεις ἀδελφοῦ Ῥηχὰβ, υἱοῦ Ἰαρίν.
\VS{9}Καὶ ἦν Ἰγαβὴς ἔνδοξος ὑπὲρ τοὺς ἀδελφοὺς αὐτοῦ· καὶ ἡ μήτηρ ἐκάλεσε τὸ ὄνομα αὐτοῦ Ἰγαβὴς, λέγουσα, ἔτεκον ὡς γαβής.
\VS{10}Καὶ ἐπεκαλέσατο Ἰγαβὴς τὸν Θεὸν Ἰσραὴλ, λέγων, ἐὰν εὐλογῶν εὐλογήσῃς με, καὶ πληθύνῃς τὰ ὅριά μου, καὶ ᾖ ἡ χείρ σου μετʼ ἐμοῦ, καὶ ποιήσῃς γνῶσιν τοῦ μὴ ταπεινῶσαί με· καὶ ἐπήγαγεν ὁ Θεὸς πάντα ὅσα ᾐτήσατο.
\par }{\PP \VS{11}Καὶ Χαλὲβ πατὴρ Ἀσχὰ ἐγέννησε τὸν Μαχίρ· οὗτος πατὴρ Ἀσσαθών.
\VS{12}Ἐγέννησε τὸν Βαθραίαν, καὶ τὸν Βεσσηὲ, καὶ τὸν Θαιμὰν πατέρα πόλεως Ναᾶς ἀδελφοῦ Ἐσελὼμ τοῦ Κενεζί· οὗτοι ἄνδρες Ῥηχάβ.
\VS{13}Καὶ υἱοὶ Κενὲζ, Γοθονιὴλ, καὶ Σαραΐα· καὶ υἱοὶ Γοθονιὴλ, Ἀθάθ.
\VS{14}Καὶ Μαναθὶ ἐγέννησε τὸν Γοφερά. καὶ Σαραΐα ἐγέννησε τὸν Ἰωβὰβ, πατέρα Ἀγεαδδαῒρ, ὅτι τέκτονες ἦσαν.
\VS{15}Καὶ υἱοὶ Χαλὲβ υἱοῦ Ἰεφοννὴ, Ἢρ, Ἀδὰ, καὶ Νοόμ· καὶ υἱοὶ Ἀδὰ, Κενέζ.
\VS{16}Καὶ υἱοὶ Ἀλεὴλ, Ζὶβ, καὶ Ζεφὰ, καὶ Θιριὰ, καὶ Ἐσερήλ.
\VS{17}Καὶ υἱοὶ Ἐσρὶ, Ἰεθὲρ, Μωρὰδ, καὶ Ἄφερ, καὶ Ἰαμών· καὶ ἐγέννησεν Ἰεθὲρ τὸν Μαρὼν, καὶ τὸν Σεμεῒ, καὶ τὸν Ἰεσβὰ πατέρα Ἐσθαίμων·
\VS{18}Καὶ ἡ γυνὴ αὐτοῦ αὕτη Ἀδία, ἔτεκε τὸν Ἰάρεδ πατέρα Γεδὼρ, καὶ τὸν Ἀβὲρ πατέρα Σωχὼν, καὶ τὸν Χετιὴλ πατέρα Ζαμών· καὶ οὗτοι υἱοὶ Βετθία θυγατρὸς Φαραὼ, ἣν ἔλαβε Μωρήδ.
\VS{19}Καὶ υἱοὶ γυναικὸς τῆς Ἰδουίας ἀδελφῆς Ναχαῒμ πατρὸς Κεϊλὰ, Γαρμὶ, καὶ Ἐσθαιμὼν Νωχαθί.
\VS{20}Καὶ υἱοὶ Σεμὼν, Αμνὼν, καὶ Ἀνὰ υἱὸς Φανὰ, καὶ Ἰνών· καὶ υἱοὶ Σεῒ, Ζωὰν, καὶ υἱοὶ Ζωάβ.
\par }{\PP \VS{21}Υἱοὶ Σηλὼμ υἱοῦ Ἰούδα, Ἢρ πατὴρ Ληχὰβ, καὶ Λααδὰ πατὴρ Μαρισά· καὶ γενέσεις οἰκείων Ἐφραθαβὰκ τῷ οἴκῳ Ἐσοβὰ,
\VS{22}καὶ Ἰωακὶμ, καὶ ἄνδρες Χωζηβὰ, καὶ Ἰωὰς, καὶ Σαρὰφ, οἳ κατῴκησαν ἐν Μωάβ· καὶ ἀπέστρεψεν αὐτος ἀβεδηρὶν, ἀθουκιΐμ·
\VS{23}Οὗτοι κεραμεῖς οἱ κατοικοῦντες ἐν Ἀταῒμ καὶ Γαδιρὰ μετὰ τοῦ βασιλέως, ἐν τῇ βασιλείᾳ αὐτοῦ ἐνίσχυσαν, καὶ κατῴκησαν ἐκεῖ.
\par }{\PP \VS{24}Υἱοὶ Σεμεὼν, Ναμουὴλ, καὶ Ἰαμὶν, Ἰαρὶβ, Ζαρὲς, Σαοὺλ,
\VS{25}Σαλὲμ υἱὸς αὐτοῦ, Μαβασὰμ υἱὸς αὐτοῦ, Μασμὰ υἱὸς αὐτοῦ,
\VS{26}Σεμεῒ υἱὸς αὐτοῦ·
\VS{27}Τῷ Σεμεῒ υἱοὶ ἑκκαίδεκα, καὶ θυγατέρες ἕξ· καὶ τοῖς ἀδελφοῖς αὐτῶν οὐκ ἦσαν υἱοὶ πολλοί· καὶ πᾶσαι αἱ πατριαὶ αὐτῶν οὐκ ἐπλεόνασαν ὡς υἱοὶ Ἰούδα.
\VS{28}Καὶ κατῴκησαν ἐν Βηρσαβεὲ, καὶ Μωλαδὰ, καὶ ἐν Ἑσερσουὰλ,
\VS{29}καὶ ἐν Βαλαὰ, καὶ ἐν Αἰσὲμ, καὶ ἐν Θωλὰδ,
\VS{30}καὶ ἐν Ἑρμὰ, καὶ ἐν Σικελὰγ,
\VS{31}καὶ ἐν Βαιθμαριμὼθ, καὶ Ἡμισουσεωσὶν, καὶ οἴκου Βαρουσεωρίμ· αὗται αἱ πόλεις αὐτῶν ἕως βασιλέως Δαυίδ.
\VS{32}Καὶ ἐπαύλεις αὐτῶν Αἰτὰν, καὶ Ἢν, Ῥεμνὼν, καὶ Θοκκὰ, καὶ Αἰσὰρ, πόλεις πέντε.
\VS{33}Καὶ πᾶσαι ἐπαύλεις αὐτῶν κύκλῳ τῶν πόλεων τούτων ἕως Βάαλ· αὕτη κατάσχεσις αὐτῶν, καὶ ὁ καταλοχισμὸς αὐτῶν.
\VS{34}Καὶ Μοσωβὰβ, καὶ Ἰεμολὸχ, καὶ Ἰωσία υἱὸς Ἀμασία,
\VS{35}καὶ Ἰωὴλ, καὶ Ἰηοὺ υἱὸς Ἀσαβία, υἱὸς Σαραῦ, υἱὸς Ἀσιὴλ,
\VS{36}καὶ Ἐλιωναῒ, καὶ Ἰωκαβὰ, καὶ Ἰασουία, καὶ Ἀσαΐα, καὶ Ἰεδιὴλ, καὶ Ἰσμαὴλ, καὶ Βαναίας, καὶ Ζουζὰ υἱὸς Σαφαῒ,
\VS{37}υἱοῦ Ἀλὼν, υἱοῦ Ἰεδιὰ, υἱοῦ Σεμρὶ, υἱοῦ Σαμαίου.
\VS{38}Οὗτοι οἱ διελθόντες ἐν ὀνόμασιν ἀρχόντων ἐν ταῖς γενέσεσιν αὐτῶν, καὶ ἐν οἴκοις πατριῶν αὐτῶν ἐπληθύνθησαν εἰς πλῆθος.
\par }{\PP \VS{39}Καὶ ἐπορεύθησαν ἕως τοῦ ἐλθεῖν Γέραρα ἕως τῶν ἀνατολῶν τῆς Γαὶ, τοῦ ζητῆσαι νομὰς τοῖς κτήνεσιν αὐτῶν.
\VS{40}Καὶ εὗρον νομὰς πλεὶονας καὶ ἀγαθάς· καὶ ἡ γῆ πλατεῖα ἐναντίον αὐτῶν, καὶ εἰρήνη καὶ ἡσυχία, ὅτι ἐκ τῶν υἱῶν Χὰμ τὼν κατοικούντων ἐκεῖ ἔμπροσθεν.
\VS{41}Καὶ ἤλθοσαν οὗτοι οἱ γεγραμμένοι ἐπʼ ὀνόματος ἐν ἡμέραις Ἐζεκίου βασιλέως Ἰούδα, καὶ ἐπάταξαν τοὺς οἴκους αὐτῶν καὶ τοὺς Μιναίους οὕς εὕροσαν ἐκεῖ, καὶ ἀνεθεμάτισαν αὐτοὺς ἕως τῆς ἡμέρας ταύτης· καὶ ᾤκησαν ἀντʼ αὐτῶν, ὅτι νομαὶ τοῖς κτήνεσιν αὐτῶν ἐκεῖ.
\VS{42}Καὶ ἐξ αὐτῶν ἀπὸ τῶν υἱῶν Συμεὼν ἐπορεύθησαν εἰς ὄρος Σηὶρ ἄνδρες πεντακόσιοι, καὶ Φαλαεττία, καὶ Νωαδία, καὶ Ῥαφαΐα, καὶ Ὀζιὴλ υἱοὶ Ἰεσὶ ἄρχοντες αὐτῶν.
\VS{43}Καὶ ἐπάταξαν τοὺς καταλοίπους τοὺς καταλειφθέντας τοῦ Ἀμαλὴκ ἕως ἡμέρας ταύτης.

\par }\Chap{5}{\PP \VerseOne{1}Καὶ υἱοὶ Ῥουβὴν πρωτοτόκου Ἰσραήλ· ὅτι οὗτος ὁ πρωτότοκος, καὶ ἐν τῷ ἀναβῆναι ἐπὶ τὴν κοίτην τοῦ πατρὸς αὐτοῦ ἔδωκεν εὐλογίαν αὐτοῦ τῷ υἱῷ αὐτοῦ Ἰωσὴφ υἱῷ Ἰσραήλ, καὶ οὐκ ἐγενεαλογήθη εἰς πρωτοτοκεῖα,
\VS{2}ὅτι Ἰούδας δυνατὸς ἰσχύι καὶ ἐν τοῖς ἀδελφοῖς αὐτοῦ, καὶ εἰς ἡγούμενον ἐξ αὐτοῦ, καὶ ἡ εὐλογία τοῦ Ἰωσήφ·
\VS{3}Υἱοὶ Ῥουβὴν πρωτοτόκου Ἰσραὴλ, Ἐνὼχ, καὶ Φαλλοὺς, Ἀσρὼμ, καὶ Χαρμί.
\VS{4}Υἱοὶ Ἰωὴλ, Σεμεῒ, καὶ Βαναία υἱὸς αὐτοῦ· καὶ υἱοὶ Γοὺγ υἱοῦ Σεμεῒ,
\VS{5}υἱὸς αὐτοῦ Μιχὰ, υἱὸς αὐτοῦ Ῥηχά, υἱὸς αὐτοῦ Ἰωὴλ,
\VS{6}υἱὸς αὐτοῦ Βεὴλ, ὃν μετῴκισε Θαγλαφαλλασὰρ βασιλεὺς Ἀσσούρ· οὗτος ἄρχων τῶν Ῥουβήν.
\par }{\PP \VS{7}Καὶ ἀδελφοὶ αὐτοῦ τῇ πατρίδι αὐτοῦ ἐν τοῖς καταλοχισμοῖς αὐτῶν κατὰ γενέσεις αὐτῶν, ὁ ἄρχων Ἰωὴλ, καὶ Ζαχαρία,
\VS{8}καὶ Βαλὲκ υἱὸς Ἀζοὺζ, υἱὸς Σαμὰ, υἱὸς Ἰωήλ· οὗτος κατῴκησεν ἐν Ἀροὴρ, καὶ ἐπὶ Ναβαῦ, καὶ Βεελμασσών.
\VS{9}Καὶ πρὸς ἀνατολὰς κατῴκησεν ἕως ἐρχομένων τῆς ἐρήμου, ἀπὸ τοῦ ποταμοῦ Εὐφράτου, ὅτι κτήνη αὐτῶν πολλὰ ἐν γῇ Γαλαάδ.
\VS{10}Καὶ ἐν ἡμέραις Σαοὺλ ἐποίησαν πόλεμον πρὸς τοὺς παροίκους, καὶ ἔπεσον ἐν χερσὶν αὐτῶν κατοικοῦντες ἐν σκηναῖς αὐτῶν πάντες κατʼ ἀνατολὰς τῆς Γαλαάδ.
\par }{\PP \VS{11}Υἱοὶ Γὰδ κατέναντι αὐτῶν κατῴκησαν ἐν γῇ Βασὰν ἕως Σελά·
\VS{12}Ἰωὴλ πρωτότοκος, καὶ Σαφὰμ ὁ δεύτερος, καὶ Ἰανὶν ὁ γραμματεὺς ἐν Βασάν.
\VS{13}Καὶ οἱ ἀδελφοὶ αὐτῶν κατʼ οἴκους πατριῶν αὐτῶν, Μιχαὴλ, Μοσολλὰμ, καὶ Σεβεὲ, καὶ Ἰωρεὲ, καὶ Ἰωαχὰν, καὶ Ζουὲ, καὶ Ὠβὴδ, ἑπτά.
\VS{14}Οὗτοι υἱοὶ Ἀβιχαία υἱοῦ Οὐρὶ, υἱοῦ Ἰδαῒ, υἱοῦ Γαλαὰδ, υἱοῦ Μιχαὴλ, υἱοῦ Ἰεσαῒ, υἱοῦ Ἰεδδαῒ, υἱοῦ Βοὺζ ἀδελφοῦ
\VS{15}υἱοῦ Ἀβδιὴλ, υἱοῦ Γουνὶ, ἄρχων οἴκου πατριῶν.
\VS{16}Κατῴκουν ἐν Γαλαὰδ, ἐν Βασὰν, καὶ ἐν ταῖς κώμαις αὐτῶν, καὶ πάντα τὰ περίχωρα Σαρὼν ἕως ἐξόδου.
\VS{17}Πάντων ὁ καταλοχισμὸς ἐν ἡμέραις Ἰωάθαμ βασιλέως Ἰούδα, καὶ ἐν ἡμέραις Ἱεροβοὰμ βασιλέως Ἰσραήλ.
\par }{\PP \VS{18}Υἱοὶ Ῥουβὴν καὶ Γὰδ καὶ ἥμισυ φυλῆς Μανασσῆ ἐξ υἱῶν δυνάμεως, ἄνδρες αἴροντες ἀσπίδας καὶ μάχαιραν, καὶ τείνοντες τόξον, καὶ δεδιδαγμένοι πόλεμον, τεσσαράκοντα καὶ τέσσαρες χιλιάδες καὶ ἑπτακόσιοι καὶ ἑξήκοντα ἐκπορευόμενοι εἰς παράταξιν.
\VS{19}Καὶ ἐποίουν πόλεμον μετὰ τῶν Ἀγαρηνῶν, καὶ Ἰτουραίων, καὶ Ναφισαίων, καὶ Ναδαβαίων,
\VS{20}καὶ κατίσχυσαν ἐπʼ αὐτῶν· καὶ ἐδόθησαν εἰς χεῖρας αὐτῶν Ἀγαραῖοι, καὶ πάντα τὰ σκηνώματα αὐτῶν, ὅτι πρὸς τὸν Θεὸν ἐβόησαν ἐν τῷ πολέμῳ, καὶ ἐπήκουσεν αὐτοῖς, ὅτι ἤλπισαν ἐπʼ αὐτόν.
\VS{21}Καὶ ᾐχμαλώτευσαν τὴν ἀποσκευὴν αὐτῶν, καμήλους πεντακισχιλίας, καὶ προβάτων διακοσίας πεντήκοντα χιλιάδας, ὄνους δισχιλίους, καὶ ψυχὰς ἀνδρῶν ἑκατὸν χιλιάδας.
\VS{22}Ὅτι τραυματίαι πολλοὶ ἔπεσον, ὅτι παρὰ τοῦ Θεοῦ ὁ πόλεμος· καὶ κατῴκησαν ἀντʼ αὐτῶν ἕως μετοικεσίας.
\par }{\PP \VS{23}Καὶ οἱ ἡμίσεις φυλῆς Μανασσῆ κατῴκησαν ἀπὸ Βασὰν ἕως Βαὰλ, Ἐρμὼν, καὶ Σανὶρ, καὶ ὄρος Ἀερμών· καὶ ἐν τῷ Λιβάνῳ αὐτοὶ ἐπλεονάσθησαν.
\VS{24}Καὶ οὗτοι ἀρχηγοὶ οἴκου πατριῶν αὐτῶν· Ὀφὲρ, καὶ Σεῒ, καὶ Ἐλιὴλ, καὶ Ἱερεμία, καὶ Ὠδουΐα, καὶ Ἰεδιήλ· ἄνδρες ἰσχυροὶ δυνάμει, ἄνδρες ὀνομαστοὶ, ἄρχοντες τῶν οἴκων πατριῶν αὐτῶν.
\par }{\PP \VS{25}Καὶ ἠθέτησαν ἐν Θεῷ πατέρων αὐτῶν, καὶ ἐπόρνευσαν ὀπίσω θεῶν τῶν λαῶν τῆς γῆς, οὓς ἐξῇρεν ὁ Θεὸς ἀπὸ προσώπου αὐτῶν.
\VS{26}Καὶ ἐπήγειρεν ὁ Θεὸς Ἰσραὴλ τὸ πνεῦμα Φαλὼχ βασιλέως Ἀσσοὺρ, καὶ τὸ πνεῦμα Θαγλαφαλλασὰρ βασιλέως Ἀσσοὺρ, καὶ μετῴκισε τὸν Ῥουβὴν, καὶ τὸν Γαδδὶ, καὶ τὸ ἥμισυ φυλῆς Μανασσῆ, καὶ ἤγαγεν αὐτοὺς εἰς Χαὰχ, καὶ Χαβὼρ, καὶ ἐπὶ ποταμὸν Γωζὰν ἕως τῆς ἡμέρας ταύτης.
\par }{\PP \VS{27}Υἱοὶ Λευὶ, Γεδσὼν, Καὰθ, καὶ Μεραρί.
\VS{28}Καὶ υἱοὶ Καὰθ, Ἄμβραμ, καὶ Ἰσσαὰρ, Χεβρὼν, καὶ Ὀζιήλ.
\VS{29}Καὶ υἱοὶ Ἄμβραμ, Ἀαρὼν, καὶ Μωυσῆς, καὶ Μαριάμ· καὶ υἱοὶ Ἀαρών, Ναδὰβ, καὶ Ἀβιοὺδ, Ἐλεάζαρ, καὶ Ἰθάμαρ.
\VS{30}Ἐλεάζαρ ἐγέννησε τὸν Φινεὲς, Φινεὲς ἐγέννησε τὸν Αβισοὺ,
\VS{31}Αβισοὺ ἐγέννησεν τὸν Βοκκὶ, καὶ Βοκκὶ ἐγέννησε τὸν Ὀζὶ,
\VS{32}Ὀζὶ ἐγέννησε τὸν Ζαραία, Ζαραία ἐγέννησε τὸν Μαριὴλ,
\VS{33}καὶ Μαριὴλ ἐγέννησε τὸν Ἀμαρία, καὶ Ἀμαρία ἐγέννησε τὸν Ἀχιτὼβ,
\VS{34}καὶ Ἀχιτὼβ ἐγέννησε τὸν Σαδὼκ, καὶ Σαδὼκ ἐγέννησε τὸν Ἀχιμάας,
\VS{35}καὶ Ἀχιμάας ἐγέννησε τὸν Ἀζαρίαν, καὶ Ἀζαρίας ἐγέννησε τὸν Ἰωανὰν,
\VS{36}καὶ Ἰωανὰν ἐγέννησε τὸν Ἀζαρίαν, οὗτος ἱεράτευσεν ἐν τῷ οἴκῳ ᾧ ᾠκοδόμησε Σαλωμὼν ἐν Ἱερουσαλήμ.
\VS{37}Καὶ ἐγέννησεν Ἀζαρίας τὸν Ἀμαρία, καὶ Ἀμαρία ἐγέννησε τὸν Ἀχιτὼβ,
\VS{38}καὶ Ἀχιτὼβ ἐγέννησε τὸν Σαδὼκ, καὶ Σαδὼκ ἐγέννησε τὸν Σαλὼμ,
\VS{39}καὶ Σαλὼμ ἐγέννησε τὸν Χελκίαν, καὶ Χελκίας ἐγέννησε τὸν Ἀζαρίαν,
\VS{40}καὶ Ἀζαρίας ἐγέννησε τὸν Σαραία, καὶ Σαραίας ἐγέννησε τὸν Ἰωσαδάκ.
\VS{41}Καὶ Ἰωσαδὰκ ἐπορεύθη ἐν τῇ μετοικίᾳ μετὰ Ἰούδα καὶ Ἱερουσαλὴμ ἐν χειρὶ Ναβουχοδονόσορ.

\par }\Chap{6}{\PP \VerseOne{1}Υἱοὶ Λευὶ, Γεδσὼν, Καὰθ, καὶ Μεραρί.
\VS{2}Καὶ ταῦτα τὰ ὀνόματα τῶν υἱῶν Γεδσὼν, Λοβενὶ, καὶ Σεμεΐ.
\VS{3}Υἱοὶ Καὰθ, Ἄμβραμ, καὶ Ἰσσαὰρ, Χεβρὼν, καὶ Ὀζιήλ.
\VS{4}Υἱοὶ Μεραρὶ, Μοολὶ, καὶ ὁ Μουσί· καὶ αὗται αἱ πατριαὶ τοῦ Λευὶ κατὰ πατριὰς αὐτῶν.
\VS{5}Τῷ Γεδσὼν, τῷ Λοβενὶ υἱῷ αὐτοῦ, Ἰὲθ υἱὸς αὐτοῦ, Ζαμμὰθ υἱὸς αὐτοῦ,
\VS{6}Ἰωὰβ υἱὸς αὐτοῦ, Ἀδδὶ υἱὸς αὐτοῦ, Ζαρὰ υἱὸς αὐτοῦ, Ἰεθρὶ υἱὸς αὐτοῦ.
\VS{7}Υἱοὶ Καὰθ, Ἀμιναδὰβ υἱὸς αὐτοῦ, Κορὲ υἱὸς αὐτοῦ, Ἀσὴρ υἱὸς αὐτοῦ,
\VS{8}Ἑλκανὰ υἱὸς αὐτοῦ, Ἀβισὰφ υἱὸς αὐτοῦ, Ἀσὴρ υἱὸς αὐτοῦ,
\VS{9}Θαὰθ υἱὸς αὐτοῦ, Οὐριὴλ υἱὸς αὐτοῦ, Ὀζία υἱὸς αὐτοῦ, Σαοὺλ υἱὸς αὐτοῦ.
\VS{10}Καὶ υἱοὶ Ἑλκανὰ, Ἀμεσσὶ, καὶ Ἀχιμὼθ,
\VS{11}Ἑλκανὰ υἱὸς αὐτοῦ, Σουφὶ υἱὸς αὐτοῦ, Καιναὰθ υἱὸς αὐτοῦ,
\VS{12}Ἐλιὰβ υἱὸς αὐτοῦ, Ιεροβοὰμ υἱὸς αὐτοῦ, Ἑλκανὰ υἱὸς αὐτοῦ.
\VS{13}Υἱοὶ Σαμουὴλ, ὁ πρωτότοκος Σανὶ, καὶ Ἀβιά.
\VS{14}Υἱοὶ Μεραρὶ, Μοολὶ, Λοβενὶ υἱὸς αὐτοῦ, Σεμεῒ υἱὸς αὐτοῦ, Ὀζὰ υἱὸς αὐτοῦ,
\VS{15}Σαμαὰ υἱὸς αὐτοῦ, Ἀγγία υἱὸς αὐτοῦ, Ἀσαῒας υἱὸς αὐτοῦ.
\par }{\PP \VS{16}Καὶ οὗτοι οὓς κατέστησε Δαυὶδ ἐπὶ χεῖρας ᾀδόντων ἐν οἴκῳ Κυρίου ἐν τῇ καταπαύσει τῆς κιβωτοῦ.
\VS{17}Καὶ ἦσαν λειτουργοῦντες ἐναντίον τῆς σκηνῆς τοῦ μαρτυρίου ἐν ὀργάνοις, ἕως οὗ ᾠκοδόμησε Σαλωμὼν τὸν οἶκον Κυρίου ἐν Ἱερουσαλήμ· καὶ ἔστησαν κατὰ τὴν κρίσιν αὐτῶν ἐπὶ τὰς λειτουργίας αὐτῶν.
\par }{\PP \VS{18}Καὶ οὗτοι οἱ ἑστηκότες, καὶ υἱοὶ αὐτῶν ἐκ τῶν υἱῶν τοῦ Καὰθ, Αἰμὰν ὁ ψαλτῳδὸς υἱὸς Ἰωὴλ, υἱοῦ Σαμουὴλ,
\VS{19}υἱοῦ Ἑλκανὰ, υἱοῦ Ἱεροβοὰμ, υἱοῦ Ἐλιὴλ, υἱοῦ Θοοὺ,
\VS{20}υἱοῦ Σοὺφ, υἱοῦ Ἑλκανὰ, υἱοῦ Μαὰθ, υἱοῦ Ἀμαθὶ,
\VS{21}υἱοῦ Ἑλκανὰ, υἱοῦ Ἰωὴλ, υἱοῦ Ἀζαρία, υἱοῦ Σαφανία,
\VS{22}υἱοῦ Θαὰθ, υἱοῦ Ἀσὴρ, υἱοῦ Ἀβιασὰφ, υἱοῦ Κορὲ,
\VS{23}υἱοῦ Ἰσαὰρ, υἱοῦ Καὰθ, υἱοῦ Λευὶ, υἱοῦ Ἰσραήλ.
\VS{24}Καὶ ὁ ἀδελφὸς αὐτοῦ Ἀσὰρ ὁ ἑστηκὼς ἐν δεξιᾷ αὐτοῦ· Ἀσὰρ υἱὸς Βαραχία, υἱοῦ Σαμαὰ,
\VS{25}υἱοῦ Μιχαὴλ, υἱοῦ Βαασία, υἱοῦ Μελχία,
\VS{26}υἱοῦ Ἀθανὶ, υἱοῦ Ζααραῒ, υἱοῦ Ἀδαῒ,
\VS{27}υἱοῦ Αἰθὰμ, υἱοῦ Ζαμμὰμ, υἱοῦ Σεμεῒ,
\VS{28}υἱοῦ Ἰεὲθ, υἱοῦ Γεδσὼν, υἱοῦ Λευί.
\VS{29}Καὶ υἱοὶ Μεραρὶ οἱ ἀδελφοὶ αὐτῶν ἐξ ἀριστερῶν· Αἰθὰμ υἱὸς Κισὰ, υἱοῦ Ἀβαῒ, υἱοῦ Μαλῶχ,
\VS{30}υἱοῦ Ἀσεβὶ, υἱοῦ Ἀμεσσία,
\VS{31}υἱοῦ Βανὶ, υἱοῦ Σεμὴρ,
\VS{32}υἱοῦ Μοολὶ, υἱοῦ Μουσὶ, υἱοῦ Μεραρὶ, υἱοῦ Λευί.
\VS{33}Καὶ οἱ ἀδελφοὶ αὐτῶν κατʼ οἴκους πατριῶν αὐτῶν, οἱ Λευῖται οἱ δεδομένοι εἰς πᾶσαν ἐργασίαν λειτουργίας σκηνῆς οἴκου τοῦ Θεοῦ.
\par }{\PP \VS{34}Καὶ Ἀαρὼν καὶ υἱοὶ αὐτοῦ θυμιῶντες ἐπὶ τὸ θυσιαστήριον τῶν ὁλοκαυτωμάτων, καὶ ἐπὶ τὸ θυσιαστήριον τῶν θυμιαμάτων εἰς πᾶσαν ἐργασίαν ἅγια τῶν ἁγίων, καὶ ἐξιλάσκεσθαι περὶ Ἰσραὴλ, κατὰ πάντα ὅσα ἐνετείλατο Μωυσῆς παῖς τοῦ Θεοῦ.
\VS{35}Καὶ οὗτοι υἱοὶ Ἀαρών· Ἐλεάζαρ υἱὸς αὐτοῦ, Φινεὲς υἱὸς αὐτοῦ, Ἀβισοὺ υἱὸς αὐτοῦ,
\VS{36}Βοκκὶ υἱὸς αὐτοῦ, Ὀζὶ υἱὸς αὐτοῦ, Σαραῒα υἱὸς αὐτοῦ,
\VS{37}Μαριὴλ υἱὸς αὐτοῦ, Ἀμαρία υἱὸς αὐτοῦ, Ἀχιτὼβ υἱὸς αὐτοῦ,
\VS{38}Σαδὼκ υἱὸς αὐτοῦ, Ἀχιμάας υἱὸς αὐτοῦ.
\par }{\PP \VS{39}Καὶ αὗται αἱ κατοικίαι αὐτῶν ἐν ταῖς κώμαις αὐτῶν, ἐν τοῖς ὁρίοις αὐτῶν, τοῖς υἱοῖς Ἀαρὼν τῇ πατριᾷ αὐτῶν τοῖς Κααθὶ, ὅτι αὐτοῖς ἐγένετο ὁ κλῆρος.
\VS{40}Καὶ ἔδωκαν αὐτοῖς τὴν Χεβρὼν ἐν γῇ Ἰούδα, καὶ τὰ περισπόρια αὐτῆς κύκλῳ αὐτῆς.
\VS{41}Καὶ τὰ πεδία τῆς πόλεως, καὶ τὰς κώμας αὐτῆς ἔδωκαν τῷ Χαλὲβ υἱῷ Ἰεφοννή.
\VS{42}Καὶ τοῖς υἱοῖς Ἀαρὼν ἔδωκαν τὰς πόλεις τῶν φυγαδευτηρίων, τὴν Χεβρὼν, καὶ τὴν Λοβνὰ καὶ τὰ περισπόρια αὐτῆς, καὶ τὴν Σελνὰ καὶ τὰ περισπόρια αὐτῆς, καὶ τὴν Ἐσθαμὼ καὶ τὰ περισπόρια αὐτῆς,
\VS{43}καὶ τὴν Ἰεθὰρ καὶ τὰ περισπόρια αὐτῆς, καὶ τὴν Δαβὶρ καὶ τὰ περισπόρια αὐτῆς,
\VS{44}καὶ τὴν Ἀσὰν καὶ τὰ περισπόρια αὐτῆς, καὶ τὴν Βαιθσαμὺς καὶ τὰ περισπόρια αὐτῆς·
\VS{45}Καὶ ἐκ φυλῆς Βενιαμὶν τὴν Γαβαῒ καὶ τὰ περισπόρια αὐτῆς, καὶ τὴν Γαλεμὰθ καὶ τὰ περισπόρια αὐτῆς, καὶ τὴν Ἀναθὼθ καὶ τὰ περισπόρια αὐτῆς· πᾶσαι αἱ πόλεις αὐτῶν τρισκαίδεκα πόλεις κατὰ πατριὰς αὐτῶν.
\par }{\PP \VS{46}Καὶ τοῖς υἱοῖς Καὰθ τοῖς καταλοίποις ἐκ τῶν πατριῶν ἐκ τῆς φυλῆς ἐκ τοῦ ἡμίσους φυλῆς Μανασσῆ, κλήρῳ πόλεις δέκα.
\VS{47}Καὶ τοῖς υἱοῖς Γεδσὼν κατὰ πατριὰς αὐτῶν ἐκ φυλῆς Ἰσσάχαρ, ἐκ φυλῆς Ἀσὴρ, ἀπὸ φυλῆς Νεφθαλὶ, ἐκ φυλῆς Μανασσῆ ἐν τῇ Βασὰν, πόλεις τρισκαίδεκα.
\VS{48}Καὶ τοῖς υἱοῖς Μεραρὶ κατὰ πατριὰς αὐτῶν ἐκ φυλῆς Ῥουβὴν, ἐκ φυλῆς Γὰδ, ἐκ φυλῆς Ζαβουλὼν, κλήρῳ πόλεις δεκαδύο.
\VS{49}Καὶ ἔδωκαν οἱ υἱοὶ Ἰσραὴλ τοῖς Λευίταις τὰς πόλεις καὶ τὰ περισπόρια αὐτῶν.
\VS{50}Καὶ ἔδωκαν ἐν κλήρῳ ἐκ φυλῆς υἱῶν Ἰούδα, καὶ ἐκ φυλῆς υἱῶν Συμεὼν, καὶ ἐκ φυλῆς υἱῶν Βενιαμὶν τὰς πόλεις ταύτας ἃς ἐκάλεσαν αὐτὰς ἐπʼ ὀνόματος.
\par }{\PP \VS{51}Καὶ ἀπὸ τῶν πατριῶν υἱῶν Καὰθ, καὶ ἐγένοντο πόλεις τῶν ὁρίων αὐτῶν ἐκ φυλῆς Ἐφραίμ.
\VS{52}Καὶ ἔδωκαν αὐτοῖς τὰς πόλεις τῶν φυγαδευτηρίων, τὴν Συχὲμ καὶ τὰ περισπόρια αὐτῆς ἐν ὄρει Ἐφραὶμ, καὶ τὴν Γαζὲρ καὶ τὰ περισπόρια αὐτῆς,
\VS{53}καὶ τὴν Ἰεκμαὰν καὶ τὰ περισπόρια αὐτῆς, καὶ τὴν Βαιθωρὼν καὶ τὰ περισπόρια αὐτῆς,
\VS{54}καὶ τὴν Αἰλὼν καὶ τὰ περισπόρια αὐτῆς, καὶ τὴν Γεθρεμμὼν καὶ τὰ περισπόρια αὐτῆς·
\VS{55}Καὶ ἀπὸ τοῦ ἡμίσους φυλῆς Μανασσῆ τὴν Ἀνὰρ καὶ τὰ περισπόρια αὐτῆς, καὶ τὴν Ἰεμβλάαν καὶ τὰ περισπόρια αὐτῆς, κατὰ πατριὰν τοῖς υἱοῖς Καὰθ τοῖς καταλοίποις.
\par }{\PP \VS{56}Τοῖς υἱοῖς Γεδσὼν ἀπὸ πατριῶν ἡμίσους φυλῆς Μανασσῆ τὴν Γωλὰν ἐκ τῆς Βασὰν καὶ τὰ περιπόλια αὐτῆς, καὶ τὴν Ἀσηρὼθ καὶ τὰ περιπόλια αὐτῆς·
\VS{57}Καὶ ἐκ φυλῆς Ἰσσάχαρ τὴν Κέδες καὶ τὰ περισπόρια αὐτῆς, καὶ τὴν Δεβερὶ καὶ τὰ περισπόρια αὐτῆς, καὶ τὴν Δαβὼρ καὶ τὰ περισπόρια αὐτῆς,
\VS{58}καὶ τὴν Ῥαμὼθ, καὶ τὴν Αἰνὰν καὶ τὰ περισπόρια αὐτῆς.
\VS{59}Καὶ ἐκ φυλῆς Ἀσὴρ τὴν Μαασὰλ καὶ τὰ περισπόρια αὐτῆς, καὶ τὴν Ἀβδὼν καὶ τὰ περισπόρια αὐτῆς,
\VS{60}καὶ τὴν Ἀκὰκ καὶ τὰ περισπόρια αὐτῆς, καὶ τὴν Ῥοὼβ καὶ τὰ περισπόρια αὐτῆς·
\VS{61}Καὶ ἀπὸ φυλῆς Νεφθαλὶ τὴν Κέδες ἐν τῇ Γαλιλαίᾳ καὶ τὰ περισπόρια αὐτῆς, καὶ τὴν Χαμὼθ καὶ τὰ περισπόρια αὐτῆς, καὶ τὴν Καριαθαῒμ καὶ τὰ περισπόρια αὐτῆς.
\par }{\PP \VS{62}Τοῖς υἱοῖς Μεραρὶ τοῖς καταλοίποις ἐκ φυλῆς Ζαβουλὼν τὴν Ῥεμμὼν καὶ τὰ περισπόρια αὐτῆς, καὶ τὴν Θαβὼρ καὶ τὰ περισπόρια αὐτῆς,
\VS{63}ἐκ τοῦ πέραν τοῦ Ἰορδάνου τὴν Ἱεριχὼ κατὰ δυσμὰς τοῦ Ἰορδάνου· ἐκ φυλῆς Ῥουβὴν τὴν Βοσὸρ ἐν τῇ ἐρήμῳ καὶ τὰ περισπόρια αὐτῆς, καὶ τὴν Ἰασὰ καὶ τὰ περισπόρια αὐτῆς,
\VS{64}καὶ τὴν Καδμὼθ καὶ τὰ περισπόρια αὐτῆς, καὶ τὴν Μαεφλὰ καὶ τὰ περισπόρια αὐτῆς·
\VS{65}Ἐκ φυλῆς Γὰδ τὴν Ῥαμμὼθ Γαλαὰδ καὶ τὰ περισπόρια αὐτῆς, καὶ τὴν Μααναῒμ καὶ τὰ περισπόρια αὐτῆς,
\VS{66}καὶ τὴν Ἐσεβὼν καὶ τὰ περισπόρια αὐτῆς, καὶ τὴν Ἰαζὴρ καὶ τὰ περισπόρια αὐτῆς.

\par }\Chap{7}{\PP \VerseOne{1}Καὶ τοῖς υἱοῖς Ἰσσάχαρ, Θωλὰ, καὶ Φουὰ, καὶ Ἰασοὺβ, καὶ Σεμερὼν, τέσσαρες.
\VS{2}Καὶ υἱοὶ Θωλὰ, Ὀζὶ, Ῥαφαΐα, καὶ ʼΙεριὴλ, καὶ ʼΙαμαῒ, καὶ ʼΙεμασὰν, καὶ Σαμουὴλ, ἄρχοντες οἴκων πατριῶν αὐτῶν τῷ Θωλὰ, ἰσχυροὶ δυνάμει κατὰ γενέσεις αὐτῶν, ὁ ἀριθμὸς αὐτῶν ἐν ἡμέραις Δαυὶδ, εἴκοσι καὶ δύο χιλιάδες καὶ ἑξακόσιοι.
\VS{3}Καὶ υἱοὶ ʼΟζὶ, ʼΙεζραῒα· καὶ υἱοὶ ʼΙεζραΐ, Μιχαὴλ, ʼΑβδιοὺ, καὶ ʼΙωὴλ, καὶ ʼΙεσία, πέντε, ἄρχοντες πάντες.
\par }{\PP \VS{4}Καὶ ἐπʼ αὐτῶν, κατὰ γενέσεις αὐτῶν, κατʼ οἴκους πατριῶν αὐτῶν, ἰσχυροὶ παρατάξασθαι εἰς πόλεμον, τριάκοντα καὶ ἓξ χιλιάδες, ὅτι ἐπλήθυναν γυναῖκας καὶ υἱούς.
\VS{5}Καὶ ἀδελφοὶ αὐτῶν εἰς πάσας πατριὰς Ἰσσάχαρ, καὶ ἰσχυροὶ δυνάμει, ὀγδοήκοντα καὶ ἑπτὰ χιλιάδες, ὁ ἀριθμὸς αὐτῶν τῶν πάντων.
\par }{\PP \VS{6}Υἱοὶ Βενιαμὶν, καὶ Βαλὲ, καὶ Βαχὶρ, καὶ ʼΙεδιὴλ, τρεῖς.
\VS{7}Καὶ υἱοὶ Βαλὲ, ʼΕσεβὼν, καὶ Ὀζὶ, καὶ Ὀζιὴλ, καὶ Ἰεριμοὺθ, καὶ Οὐρὶ, πέντε, ἄρχοντες οἴκων πατρικῶν ἰσχυροὶ δυνάμει· καὶ ὁ ἀριθμὸς αὐτῶν, εἴκοσι καὶ δύο χιλιάδες καὶ τριακοντατέσσαρες.
\VS{8}Καὶ υἱοὶ Βαχὶρ, Ζεμιρὰ, καὶ Ἰωὰς, καὶ Ἐλιέζερ, καὶ Ἐλιθενὰν, καὶ Ἀμαρία, καὶ ʼΙεριμοὺθ, καὶ Ἀβιοὺδ, καὶ Ἀναθὼθ, καὶ ʼΕληεμέθ· πάντες οὗτοι υἱοὶ Βαχίρ.
\VS{9}Καὶ ὁ ἀριθμὸς αὐτῶν κατὰ γενέσεις αὐτῶν, ἄρχοντες οἴκων πατριῶν αὐτῶν ἰσχυροὶ δυνάμει, εἴκοσι χιλιάδες καὶ διακόσιοι.
\VS{10}Καὶ υἱοὶ ʼΙεδιὴλ, Βαλαάν· καὶ υἱοὶ Βαλαὰν, Ἰαοὺς, καὶ Βενιαμὶν, καὶ Ἀὼθ, καὶ Χανανὰ, καὶ Ζαιθὰν, καὶ Θαρσὶ, καὶ Ἀχισαάρ.
\VS{11}Πάντες οὗτοι υἱοὶ ʼΙεδιὴλ, ἄρχοντες τῶν πατριῶν ἰσχυροὶ δυνάμει, ἑπτα καίδεκα χιλιάδες καὶ διακόσιοι, ἐκπορευόμενοι δυνάμει πολεμεῖν.
\VS{12}Καὶ Σαπφὶν, καὶ Ἀπφὶν, καὶ υἱοὶ Ὢρ, ʼΑσὼμ, υἱὸς αὐτοῦ Ἀόρ.
\par }{\PP \VS{13}Υἱοὶ Νεφθαλὶ, Ἰασιὴλ, Γωνὶ, καὶ Ἀσὴρ, καὶ Σελλοὺμ, υἱοὶ αὐτοῦ, Βαλὰμ υἱὸς αὐτοῦ.
\par }{\PP \VS{14}Υἱοὶ Μανασσῆ, Ἐσριὴλ, ὃν ἔτεκεν ἡ παλλακὴ αὐτοῦ ἡ Σύρα, ἔτεκε δὲ αὐτῷ καὶ Μαχὶρ πατέρα Γαλαάδ.
\VS{15}Καὶ Μαχὶρ ἔλαβε γυναῖκα τῷ Ἀπφὶν καὶ Σαπφίν· καὶ ὄνομα ἀδελφῆς αὐτοῦ Μοωχὰ, καὶ ὄνομα τῷ δευτέρῳ Σαπφαάδ· ἐγεννήθησαν δὲ τῷ Σαπφαὰδ θυγατέρες.
\VS{16}Καὶ ἔτεκε Μοωχὰ γυνὴ Μαχὶρ υἱὸν, καὶ ἐκάλεσε τὸ ὄνομα αὐτοῦ Φαρές· καὶ ὄνομα ἀδελφοῦ αὐτοῦ Σοῦρος· υἱοὶ αὐτοῦ Οὐλὰμ, καὶ ʼΡοκόμ.
\VS{17}Καὶ υἱοὶ Οὐλὰμ, Βαδάμ· οὗτοι υἱοὶ Γαλαὰδ, υἱοῦ Μαχὶρ, υἱοῦ Μανασσῆ.
\VS{18}Καὶ ἡ ἀδελφὴ αὐτοῦ ἡ Μαλεχὲθ ἔτεκε τὸν Ἰσοὺδ, καὶ τὸν Ἀβιέζερ, καὶ τὸν Μαελά.
\VS{19}Καὶ ἦσαν υἱοὶ Σεμιρὰ, ʼΑῒμ, καὶ Συχὲμ, καὶ Λακὶμ, καὶ Ἀνιάν.
\par }{\PP \VS{20}Καὶ υἱοὶ Ἐφραὶμ, Σωθαλὰθ, καὶ Βαρὰδ υἱὸς αὐτοῦ, καὶ Θαὰθ υἱὸς αὐτοῦ, ʼΕλαδὰ υἱὸς αὐτοῦ, Σαὰθ υἱὸς αὐτοῦ,
\VS{21}καὶ Ζαβὰδ υἱὸς αὐτοῦ, Σωθελὲ υἱὸς αὐτοῦ, καὶ ʼΑζὲρ, καὶ ʼΕλεάδ· καὶ ἀπέκτειναν αὐτοὺς οἱ ἄνδρες Γὲθ οἱ τεχθέντες ἐν τῇ γῇ, ὅτι κατέβησαν τοῦ λαβεῖν τὰ κτήνη αὐτῶν.
\VS{22}Καὶ ἐπένθησεν Ἐφραὶμ ὁ πατὴρ αὐτῶν ἡμέρας πολλάς· καὶ ἦλθον ἀδελφοὶ αὐτοῦ τοῦ παρακαλέσαι αὐτόν.
\VS{23}Καὶ εἰσῆλθε πρὸς τὴν γυναῖκα αὐτοῦ, καὶ ἔλαβεν ἐν γαστρὶ, καὶ ἔτεκεν υἱόν· καὶ ἐκάλεσε τὸ ὄνομα αὐτοῦ Βεριὰ, ὅτι ἐν κακοῖς ἐγένετο ἐν οἴκῳ μου.
\VS{24}Καὶ ἡ θυγάτηρ αὐτοῦ Σαραά· καὶ ἐν ἐκείνοις τοῖς καταλοίποις· καὶ ᾠκοδόμησε τὴν Βαιθωρὼν τὴν κάτω καὶ τὴν ἄνω· καὶ υἱοὶ Ὀζὰν Σεηρὰ,
\VS{25}καὶ Ῥαφὴ υἱὸς αὐτοῦ, Σαρὰφ καὶ Θαλεὲς υἱοὶ αὐτοῦ, Θαὲν υἱὸς αὐτοῦ.
\VS{26}Τῷ Λααδὰν υἱῷ αὐτοῦ υἱὸς Ἀμιοὺδ, υἱὸς Ἐλισαμαῒ,
\VS{27}υἱὸς Νοὺν, υἱὸς Ἰησουὲ, υἱοὶ αὐτοῦ.
\par }{\PP \VS{28}Καὶ κατάσχεσις αὐτῶν καὶ κατοικία αὐτῶν Βαιθὴλ καὶ αἱ κῶμαι αὐτῆς, κατʼ ἀνατολὰς Νοαρὰν, πρὸς δυσμαῖς Γάζερ καὶ αἱ κῶμαι αὐτῆς, καὶ Συχὲμ καὶ αἱ κῶμαι αὐτῆς ἕως Γάζης, καὶ αἱ κῶμαι αὐτῆς,
\VS{29}καὶ ἕως ὁρίων υἱῶν Μανασσῆ, Βαιθσαὰν καὶ αἱ κῶμαι αὐτῆς, Θανὰχ καὶ αἱ κῶμαι αὐτῆς, Μαγεδδὼ καὶ αἱ κῶμαι αὐτῆς, Δὼρ καὶ αἱ κῶμαι αὐτῆς· ἐν ταύτῃ κατῴκησαν υἱοὶ Ἰωσὴφ υἱοῦ Ἰσραήλ.
\par }{\PP \VS{30}Υἱοὶ Ἀσὴρ, Ἰεμνὰ, καὶ Σουΐα, καὶ Ἰσουῒ, καὶ Βεριὰ, καὶ Σορὲ ἀδελφὴ αὐτῶν.
\VS{31}Καὶ υἱοὶ Βεριὰ, Χάβερ, καὶ Μελχιήλ· οὗτος πατὴρ Βερθαΐθ.
\VS{32}Καὶ Χάβερ ἐγέννησε τὸν Ἰαφλὴτ, καὶ τὸν Σαμὴρ, καὶ τὸν Χωθὰν, καὶ τὴν Σωλὰ ἀδελφὴν αὐτῶν.
\VS{33}Καὶ υἱοὶ Ἰαφλὴτ, Φασὲκ, καὶ Βαμαὴλ, καὶ Ἀσίθ· οὗτοι υἱοὶ Ἰαφλήτ.
\VS{34}Καὶ υἱοὶ Σεμμὴρ, Ἀχὶρ, καὶ Ῥοογὰ, καὶ ʼΙαβὰ, καὶ Αρὰμ,
\VS{35}καὶ Βανὴ ʼΕλὰμ ἀδελφοῦ αὐτοῦ Σωφὰ, καὶ ʼΙμανὰ, καὶ Σελλὴς, καὶ ʼΑμάλ.
\VS{36}Υἱοὶ Σωφὰς, Σουὲ, καὶ Ἁρναφὰρ, καὶ Σουδὰ, καὶ Βαρὶν, καὶ ʼΙμρὰν,
\VS{37}καὶ Βασὰν, καὶ ʼΩὰ, καὶ Σαμὰ, καὶ Σαλισὰ, καὶ ʼΙεθρὰ, καὶ Βεηρά.
\VS{38}Καὶ υἱοὶ Ἰεθὴρ, Ἰεφινὰ, καὶ Φασφὰ, καὶ Ἀρά.
\VS{39}καὶ υἱοὶ ʼΟλὰ, Ὀρὲχ, Ἀνιὴλ, καὶ Ῥασιά.
\par }{\PP \VS{40}Πάντες οὗτοι υἱοὶ Ἀσὴρ, πάντες ἄρχοντες πατριῶν, ἐκλεκτοὶ ἰσχυροὶ δυνάμει, ἄρχοντες ἡγούμενοι· ὁ ἀριθμὸς αὐτῶν εἰς παράταξιν τοῦ πολεμεῖν, ἀριθμὸς αὐτῶν ἄνδρες εἰκοσιὲξ χιλιάδες.

\par }\Chap{8}{\PP \VerseOne{1}Καὶ Βενιαμὶν ἐγέννησε Βαλὲ πρωτότοκον αὐτοῦ, καὶ ʼΑσβὴλ τὸν δεύτερον, Ἀαρὰ τὸν τρίτον,
\VS{2}Νωὰ τὸν τέταρτον, καὶ Ῥαφὰ τὸν πέμπτον.
\VS{3}Καὶ ἦσαν υἱοὶ τῷ Βαλὲ, Ἀδὶρ, καὶ Γηρὰ, καὶ Ἀβιοὺδ,
\VS{4}καὶ Ἀβεσσουὲ, καὶ Νοαμὰ, καὶ Ἀχιὰ,
\VS{5}καὶ Γερὰ, καὶ Σεφουφὰμ, καὶ Οὐράμ.
\VS{6}Οὗτοι υἱοὶ Ἀὼδ, οὗτοί εἰσιν ἄρχοντες πατριῶν τοῖς κατοικοῦσι Γαβεέ· καὶ μετῴκισαν αὐτοὺς εἰς Μαχαναθὶ,
\VS{7}καὶ Νοομὰ, καὶ Ἀχιὰ, καὶ Γηρά· οὗτος ἰεγλαὰμ, καὶ ἐγέννησε τὸν ʼΑζὰ, καὶ τὸν Ἰαχιχώ.
\par }{\PP \VS{8}Καὶ Σααρὶν ἐγέννησεν ἐν τῷ πεδίῳ Μωὰβ μετὰ τὸ ἀποστεῖλαι αὐτὸν ʼΩσὶν καὶ τὴν Βααδὰ γυναῖκα αὐτοῦ.
\par }{\PP \VS{9}Καὶ ἐγέννησεν ἐκ τῆς ʼΑδὰ γυναικὸς αὐτοῦ τὸν Ἰωλὰβ, καὶ τὸν Σεβιὰ, καὶ τὸν Μισὰ, καὶ τὸν Μελχὰς,
\VS{10}καὶ τὸν Ἰεβοὺς, καὶ τὸν Ζαβιὰ καὶ τὸν Μαρμά· οὗτοι ἄρχοντες πατριῶν.
\VS{11}Καὶ ἐκ τῆς Ὠσὶν ἐγέννησε τὸν Ἀβιτὼλ, καὶ τὸν Ἀλφαάλ.
\VS{12}Καὶ υἱοὶ Ἀλφαὰλ, Ὠβὴδ, Μισαὰλ, Σεμμήρ· οὗτος ᾠκοδόμησε τὴν Ὠνὰν, καὶ τὴν Λὼδ καὶ τὰς κώμας αὐτῆς·
\VS{13}Καὶ Βεριὰ, καὶ Σαμά· οὗτοι ἄρχοντες τῶν πατριῶν τοῖς κατοικοῦσιν Αἰλὰμ, καὶ οὗτοι ἐξεδίωξαν τοὺς κατοικοῦντας Γέθ.
\VS{14}καὶ ἀδελφὸς αὐτοῦ Σωσὴκ, καὶ ʼΑριμὼθ,
\VS{15}καὶ Ζαβαδία, καὶ Ὠρὴδ, καὶ Ἔδερ,
\VS{16}καὶ Μιχαὴλ, καὶ ʼΙεσφὰ, καὶ Ἰωδὰ, υἱοὶ Βεριά.
\VS{17}Καὶ Ζαβαδία, καὶ Μοσολλὰμ, καὶ Ἀζακὶ, καὶ Ἀβὰρ,
\VS{18}καὶ Ἰσαμαρὶ, καὶ Ἰεξλίας, καὶ Ἰωβὰβ, υἱοὶ Ἐλφαάλ.
\VS{19}Καὶ Ἰακὶμ, καὶ Ζαχρὶ, καὶ Ζαβδὶ,
\VS{20}καὶ Ἐλιωναῒ, καὶ Σαλαθὶ, καὶ Ἐλιηλὶ,
\VS{21}καὶ Ἀδαΐα, καὶ Βαραΐα, καὶ Σαμαρὰθ, υἱοὶ Σαμαΐθ.
\VS{22}καὶ Ἰεσφὰν, καὶ Ὠβὴδ, καὶ Ἐλεὴλ,
\VS{23}καὶ Ἀβδὼν, καὶ Ζεχρὶ, καὶ Ἀνὰν,
\VS{24}καὶ Ἀνανία, καὶ Ἀμβρὶ, καὶ Αἰλὰμ, καὶ Ἀναθὼθ,
\VS{25}καὶ Ἰαθὶν, καὶ Ἰεφαδίας, καὶ Φανουὴλ, υἱοὶ Σωσήκ.
\VS{26}Καὶ Σαμσαρὶ, καὶ Σααρίας, καὶ Γοθολία,
\VS{27}καὶ Ἰαρασία, καὶ Ἐριὰ, καὶ Ζεχρὶ υἱὸς Ἰροάμ.
\VS{28}Οὗτοι ἄρχοντες πατριῶν κατὰ γενέσεις αὐτῶν ἄρχοντες· οὗτοι κατῴκησαν ἐν Ἱερουσαλήμ.
\par }{\PP \VS{29}Καὶ ἐν Γαβαὼν κατῴκησε πατὴρ Γαβαών· καὶ ὄνομα γυναικὶ αὐτοῦ Μοαχά.
\VS{30}Καὶ ὁ υἱὸς αὐτῆς ὁ πρωτότοκος Ἀβδὼν, καὶ Σοὺρ, καὶ Κὶς, καὶ Βαὰλ, καὶ Ναδὰβ,
\VS{31}καὶ Νὴρ, καὶ Γεδοὺρ καὶ ἀδελφὸς αὐτοῦ, καὶ Ζακχοὺρ, καὶ Μακελώθ.
\VS{32}Καὶ Μακελὼθ ἐγέννησε τὸν Σαμαά· καὶ γὰρ οὗτοι κατέναντι τῶν ἀδελφῶν αὐτῶν κατῴκησαν ἐν Ἱερουσαλὴμ μετὰ τῶν ἀδελφῶν αὐτῶν.
\VS{33}Καὶ Νὴρ ἐγέννησε τὸν Κὶς, καὶ Κὶς ἐγέννησε τὸν Σαοὺλ, καὶ Σαοὺλ ἐγέννησε τὸν Ἰωνάθαν, καὶ τὸν Μελχισουὲ, καὶ τὸν Ἀμιναδὰβ, καὶ τὸν Ἀσαβάλ.
\VS{34}καὶ υἱὸς Ἰωνάθαν Μεριβαάλ· καὶ Μεριβαὰλ ἐγέννησε τὸν Μιχά.
\VS{35}καὶ υἱοὶ Μιχὰ, Φιθὼν, καὶ Μελὰχ, καὶ Θαρὰχ, καὶ Ἀχάζ.
\VS{36}καὶ Ἀχὰζ ἐγέννησε τὸν Ἰαδά· καὶ Ἰαδὰ ἐγέννησε τὸν Σαλαιμὰθ, καὶ τὸν Ἀσμὼθ, καὶ τὸν Ζαμβρί· καὶ Ζαμβρὶ ἐγέννησε τὸν Μαισά.
\VS{37}Καὶ Μαισὰ ἐγέννησε τὸν Βαανά· Ῥαφαία υἱὸς αὐτοῦ, Ἐλασὰ υἱὸς αὐτοῦ, Ἐσὴλ υἱὸς αὐτοῦ.
\par }{\PP \VS{38}Καὶ τῷ Ἐσὴλ ἓξ υἱοί· καὶ ταῦτα τὰ ὀνόματα αὐτῶν· Ἐζρικὰμ πρωτότοκος αὐτοῦ, καὶ Ἰσμαὴλ, καὶ Σαραΐα, καὶ Ἀβδία, καὶ Ἀνὰν, καὶ Ἀσά· πάντες οὗτοι υἱοὶ Ἐσήλ.
\VS{39}Καὶ υἱοὶ Ἀσὴλ ἀδελφοῦ αὐτοῦ, Αἰλὰμ πρωτότοκος αὐτοῦ, καὶ Ἰὰς ὁ δεύτερος, καὶ Ἐλιφαλὲτ ὁ τρίτος.
\VS{40}Καὶ ἦσαν υἱοὶ Αἰλὰμ ἰσχυροὶ ἄνδρες δυνάμει, τείνοντες τόξον, καὶ πληθύνοντες υἱοὺς καὶ υἱοὺς τῶν υἱῶν, ἑκατὸν πεντήκοντα· πάντες οὗτοι ἐξ υἱῶν Βενιαμίν.

\par }\Chap{9}{\PP \VerseOne{1}Καὶ πᾶς Ἰσραὴλ ὁ συλλοχισμὸς αὐτῶν· καὶ οὗτοι καταγεγραμμένοι ἐν βιβλίῳ τῶν βασιλέων Ἰσραὴλ καὶ Ἰούδα, μετὰ τῶν ἀποικισθέντων εἰς Βαβυλῶνα ἐν ταῖς ἀνομίαις αὐτῶν,
\VS{2}καὶ οἱ κατοικοῦντες πρότερον ἐν ταῖς κατασχέσεσιν αὐτῶν ἐν ταῖς πόλεσιν Ἰσραὴλ, οἱ ἱερεῖς, οἱ Λευῖται, καὶ οἱ δεδομένοι.
\par }{\PP \VS{3}Καὶ ἐν Ἱερουσαλὴμ κατῴκησαν ἀπὸ τῶν υἱῶν Ἰούδα, καὶ ἀπὸ τῶν υἱῶν Βενιαμὶν, καὶ ἀπὸ τῶν υἱῶν Ἐφραῒμ, καὶ Μανασσῆ.
\VS{4}Καὶ Γνωθὶ, καὶ υἱὸς Σαμιοὺδ, υἱοῦ Ἀμρὶ, υἱοῦ Ἀμβραῒμ, υἱοῦ Βουνὶ, υἱοῦ υἱῶν Φαρὲς, υἱοῦ Ἰούδα.
\VS{5}Καὶ ἐκ τῶν Σηλωνὶ, Ἀσαΐα πρωτότοκος αὐτοῦ, καὶ ὁ ὑοὶ αὐτοῦ.
\VS{6}Ἐκ τῶν υἱῶν Ζαρὰ, Ἰεὴλ, καὶ ἀδελφοὶ αὐτῶν ἑξακόσιοι καὶ ἐννενήκοντα.
\par }{\PP \VS{7}Καὶ ἐκ τῶν υἱῶν Βενιαμὶν Σαλὼμ υἱὸς Μοσλλὰμ, υἱοῦ Ὠδουΐα, υἱοῦ Ἀσινοῦ,
\VS{8}καὶ Ἰεμναὰ υἱὸς Ἱεροβοὰμ, καὶ Ἠλώ· οὗτοι υἱοὶ Ὀζὶ υἱοῦ Μαχίρ· καὶ Μοσολλὰμ υἱὸς Σαφατία, υἱοῦ Ῥαγουὴλ, υἱοῦ Ἰεμναῒ,
\VS{9}καὶ ἀδελφοὶ αὐτῶν κατὰ γενέσεις αὐτῶν ἐννακόσιοι πεντηκονταὲξ, πάντες οἱ ἄνδρες ἄρχοντες πατριῶν κατʼ οἴκους πατριῶν αὐτῶν.
\par }{\PP \VS{10}Καὶ ἀπὸ τῶν ἱερέων, Ἰωδαὲ, καὶ Ἰωαρὶμ, καὶ Ἰαχὶν,
\VS{11}καὶ Ἀζαρία υἱὸς Χελκία υἱοῦ Μοσολλὰμ, υἱοῦ Σαδὼκ, υἱοῦ Μαραϊὼθ, υἱοῦ Ἀχιτὼβ ἡγουμένου οἴκου τοῦ Θεοῦ,
\VS{12}καὶ Ἀδαΐα υἱὸς Ἰραὰμ, υἱοῦ Φασχὼρ, υἱοῦ Μελχία, καὶ Μαασαία υἱὸς Ἀδιὴλ, υἱοῦ Ἐζιρὰ, υἱοῦ Μοσολλὰμ, υἱοῦ Μασελμὼθ, υἱοῦ Ἐμμὴρ,
\VS{13}καὶ ἀδελφοὶ αὐτῶν ἄρχοντες οἴκων πατριῶν αὐτῶν, χίλιοι καὶ ἑπτακόσιοι καὶ ἑξήκοντα, ἰσχυροὶ δυνάμει εἰς ἐργασίαν λειτουργίας οἴκου τοῦ Θεοῦ.
\par }{\PP \VS{14}Καὶ ἐκ τῶν Λευιτῶν, Σαμαΐα υἱὸς Ἀσὼβ, υἱοῦ Ἐζρικὰμ, υἱοῦ Ἀσαβία, ἐκ τῶν υἱῶν Μεραρί.
\VS{15}Καὶ Βακβακὰρ, καὶ Ἀρὴς, καὶ Γαλαὰλ, καὶ Ματθανίας υἱὸς Μιχὰ, υἱοῦ Ζεχρὶ, υἱοῦ Ἀσάφ.
\VS{16}Καὶ Ἀβδία υἱὸς Σαμία, υἱοῦ Γαλαὰλ, υἱοῦ Ἰδιθοὺν, καὶ Βαραχία υἱὸς Ὀσσὰ, υἱοῦ Ἑλκανὰ, ὁ κατοικῶν ἐν ταῖς κώμαις Νωτεφατί.
\VS{17}Οἱ πυλωροὶ, Σαλὼμ, Ἀκοὺμ, Τελμὼν, καὶ Διμὰν, καὶ ἀδελφοὶ αὐτῶν· Σαλὼμ ὁ ἄρχων,
\VS{18}καὶ ἕως ταύτης ἐν τῇ πύλῃ τοῦ βασιλέως κατʼ ἀνατολάς· αὗται αἱ πύλαι τῶν παρεμβολῶν υἱῶν Λευί.
\VS{19}Καὶ Σελλοὺμ υἱὸς Κορὲ, υἱοῦ Ἀβιασὰφ, υἱοῦ Κορέ· καὶ οἱ ἀδελφοὶ αὐτοῦ εἰς οἶκον πατρὸς αὐτοῦ, οἱ Κορῖται ἐπὶ τῶν ἔργων τῆς λειτουργίας φυλάσσοντες τὰς φυλακὰς τῆς σκηνῆς· καὶ πατέρες αὐτῶν ἐπὶ τῆς παρεμβολῆς Κυρίου φυλάσσοντες τὴν εἴσοδον.
\par }{\PP \VS{20}Καὶ Φινεὲς υἱὸς Ἐλεάζαρ ἡγούμενος ἦν ἐπʼ αὐτῶν ἔμπροσθεν Κυρίου, καὶ οὗτοι μετʼ αὐτοῦ.
\VS{21}Ζαχαρίας υἱὸς Μοσολλαμὶ πυλωρὸς τῆς θύρας τῆς σκηνῆς τοῦ μαρτυρίου.
\VS{22}Πάντες οἱ ἐκλεκτοὶ ἐπὶ τῆς πύλης ἐν ταῖς πύλαις διακόσιοι καὶ δεκαδύο· οὗτοι ἐν ταῖς αὐλαῖς αὐτῶν, ὁ καταλοχισμὸς αὐτῶν· τούτους ἔστησε Δαυὶδ καὶ Σαμουὴλ ὁ βλέπων τῇ πίστει αὐτῶν.
\VS{23}Καὶ οὗτοι καὶ οἱ υἱοὶ αὐτῶν ἐπὶ τῶν πύλων ἐν οἴκῳ Κυρίου, καὶ ἐν οἴκῳ τῆς σκηνῆς τοῦ φυλάσσειν.
\VS{24}Κατὰ τοὺς τέσσαρας ἀνέμους ἦσαν αἱ πύλαι, κατὰ ἀνατολὰς, θάλασσαν, Βοῤῥᾶν, Νότον.
\VS{25}Καὶ ἀδελφοὶ αὐτῶν ἐν ταῖς αὐλαῖς αὐτῶν τοῦ εἰσπορεύεσθαι κατὰ ἑπτὰ ἡμέρας ἀπὸ καιροῦ εἰς καιρὸν μετὰ τούτων·
\VS{26}Ὅτι ἐν πίστει εἰσὶ τέσσαρες δυνατοὶ τῶν πυλῶν· καὶ οἱ Λευῖται ἦσαν ἐπὶ τῶν παστοφορίων, καὶ ἐπὶ τῶν θησαυρῶν οἴκου τοῦ Θεοῦ παρεμβάλλουσιν,
\VS{27}ὅτι ἐπʼ αὐτοὺς ἡ φυλακή· καὶ οὗτοι ἐπὶ τῶν κλειδῶν τοπρωῒ πρωῒ ἀνοίγειν τὰς θύρας τοῦ ἱεροῦ.
\par }{\PP \VS{28}Καὶ ἐξ αὐτῶν ἐπὶ τὰ σκεύη τῆς λειτουργίας, ὅτι ἐν ἀριθμῷ εἰσοίσουσι, καὶ ἐν ἀριθμῷ ἐξόσουσι.
\VS{29}Καὶ ἐξ αὐτῶν καθεσταμένοι ἐπὶ τὰ σκεύη, καὶ ἐπὶ πάντα σκεύη τὰ ἅγια, καὶ ἐπὶ τῆς σεμιδάλεως, τοῦ οἴνου, τοῦ ἐλαίου, τοῦ λιβανωτοῦ, καὶ τῶν ἀρωμάτων.
\VS{30}Καὶ ἀπὸ τῶν υἱῶν τῶν ἱερέων ἦσαν μυρεψοὶ τοῦ μύρου, καὶ εἰς τὰ ἀρώματα.
\VS{31}Καὶ Ματθαθίας ἐκ τῶν Λευιτῶν, οὗτος ὁ πρωτότοκος τῷ Σαλὼμ τῷ Κορείτῃ, ἐν τῇ πίστει ἐπὶ τὰ ἔργα τῆς θυσίας τοῦ τηγάνου τοῦ μεγάλου ἱερέως.
\VS{32}Καὶ Βαναΐας ὁ Κααθίτης ἐκ τῶν ἀδελφῶν αὐτῶν ἐπὶ τῶν ἄρτων τῆς προθέσεως, τοῦ ἑτοιμάσαι σάββατον κατὰ σάββατον.
\VS{33}Καὶ οὗτοι ψαλτῳδοὶ ἄρχοντες τῶν πατριῶν τῶν Λευιτῶν διατεταγμέναι ἐφημερίαι, ὅτι ἡμέρα καὶ νὺξ ἐπʼ αὐτοῖς ἐν τοῖς ἔργοις.
\VS{34}Οὗτοι ἄρχοντες τῶν πατριῶν τῶν Λευιτῶν κατὰ γενέσεις αὐτῶν, ἄρχοντες οὗτοι κατῴκησαν ἐν Ἱερουσαλήμ.
\par }{\PP \VS{35}Καὶ ἐν Γαβαὼν κατῴκησε πατὴρ Γαβαὼν Ἰεήλ· καὶ ὄνομα γυναικὸς αὐτοῦ Μοωχά.
\VS{36}Καὶ υἱὸς αὐτοῦ ὁ πρωτότοκος Ἀβδὼν, καὶ Σοὺρ, καὶ Κὶς, καὶ Βαὰλ, καὶ Νὴρ, καὶ Ναδὰβ,
\VS{37}καὶ Γεδοὺρ καὶ ἀδελφὸς, καὶ Ζακχοὺρ, καὶ Μακελώθ.
\VS{38}Καὶ Μακελὼθ ἐγέννησε τὸν Σαμαά· καὶ οὗτοι ἐν μέσῳ τῶν ἀδελφῶν αὐτῶν κατῴκησαν ἐν Ἱερουσαλὴμ, ἐν μέσῳ τῶν ἀδελφῶν αὐτῶν.
\par }{\PP \VS{39}Καὶ Νὴρ ἐγέννησε τὸν Κὶς, καὶ Κὶς ἐγέννησε τὸν Σαοὺλ, καὶ Σαοὺλ ἐγέννησε τὸν Ἰωνάθαν, καὶ τὸν Μελχισουὲ, καὶ τὸν Ἀμιναδὰβ, καὶ τὸν Ἀσαβάλ.
\VS{40}Καὶ υἱὸς Ἰωνάθαν Μεριβάλ· καὶ Μεριβαὰλ ἐγέννησε τὸν Μιχά.
\VS{41}Καὶ υἱοὶ Μιχὰ Φιθὼν, καὶ Μαλὰχ, καὶ Θαράχ.
\VS{42}Καὶ Ἀχὰζ ἐγέννησε τὸν Ἰαδά· καὶ Ἰαδὰ ἐγέννησε τὸν Γαλεμὲθ, καὶ τὸν Γαζμὼθ, καὶ τὸν Ζαμβρί· και Ζαμβρὶ ἐγέννησε τὸν Μασσά.
\VS{43}Καὶ Μασσὰ ἐγέννησε τὸν Βαανὰ, καὶ Ῥαφαΐα υἱὸς αὐτοῦ, Ἐλασὰ υἱὸς αὐτοῦ, Ἐσὴλ υἱὸς αὐτοῦ.
\VS{44}Καὶ τῷ Ἐσὴλ ἓξ υἱοί· καὶ ταῦτα τὰ ὀνόματα αὐτῶν· Ἐζρικὰμ πρωτότοκος αὐτοῦ, καὶ Ἰσμαὴλ, καὶ Σαραΐα, καὶ Ἀβδία, καὶ Ἀνὰν, καὶ Ἀσά· οὗτοι υἱοὶ Ἐσήλ.

\par }\Chap{10}{\PP \VerseOne{1}Καὶ ἀλλόφυλοι ἐπολέμησαν πρὸς τὸν Ἰσραὴλ, καὶ ἔφυγον ἀπὸ προσώπου ἀλλοφύλων, καὶ ἔπεσον τραυματίαι ἐν ὄρει Γελβουέ.
\VS{2}Καὶ κατεδίωξαν οἱ ἀλλόφυλοι ὀπίσω Σαοὺλ καὶ ὀπίσω τῶν υἱῶν αὐτοῦ· καὶ ἐπάταξαν ἀλλόφυλοι τὸν Ἰωνάθαν, καὶ τὸν Ἀμιναδὰβ, καὶ τὸν Μελχισουὲ, υἱοὺς Σαούλ.
\VS{3}Καὶ ἐβαρύνθη ὁ πόλεμος ἐπὶ Σαούλ· καὶ εὗρον αὐτὸν οἱ τοξόται ἐν τόξοις καὶ πόνοις, καὶ ἐπονέσαν ἀπὸ τῶν τόξων.
\VS{4}Καὶ εἶπε Σαοὺλ τῷ αἴροντι τὰ σκεύη αὐτοῦ, σπάσαι τὴν ῥομφαίαν σου, καὶ ἐκκέντησόν με ἐν αὐτῇ, μὴ ἔλθωσιν οἱ ἀπερίτμητοι οὗτοι καὶ ἐμπαίξωσί μοι· καὶ οὐκ ἐβούλετο ὁ αἴρων τὰ σκεύη αὐτοῦ, ὅτι ἐφοβεῖτο σφόδρα· καὶ ἔλαβε Σαοὺλ τὴν ῥομφαίαν, καὶ ἐπέπεσεν ἐπʼ αὐτήν.
\VS{5}Καὶ εἶδεν ὁ αἴρων τὰ σκεύη αὐτοῦ, ὅτι ἀπέθανε Σαοὺλ, καὶ ἔπεσε καὶ αὐτὸς ἐπὶ τὴν ῥομφαίαν αὐτοῦ.
\VS{6}Καὶ ἀπέθανε Σαοὺλ, καὶ τρεῖς υἱοὶ αὐτοῦ ἐν τῇ ἡμέρᾳ ἐκείνῃ· καὶ πᾶς ὁ οἶκος αὐτοῦ ἐπὶ τὸ αὐτὸ ἀπέθανε.
\VS{7}Καὶ εἶδε πᾶς ἀνὴρ Ἰσραὴλ ὁ ἐν τῷ αὐλῶνι ὅτι ἔφυγεν Ἰσραὴλ, καὶ ὅτι ἀπέθανε Σαοὺλ καὶ οἱ υἱοὶ αὐτοῦ, καὶ κατέλιπον τὰς πόλεις αὐτῶν καὶ ἔφυγον· καὶ ἦλθον ἀλλόφυλοι, καὶ κατῴκησαν ἐν αὐταῖς.
\par }{\PP \VS{8}Καὶ ἐγένετο τῇ ἐχομένῃ, καὶ ἦλθον ἀλλόφυλοι τοῦ σκυλεύειν τοὺς τραυματίας, καὶ εὗρον τὸν Σαοὺλ καὶ τοὺς υἱοὺς αὐτοῦ πεπτωκότας ἐν τῷ ὄρει Γελβουέ.
\VS{9}Καὶ ἐξέδυσαν αὐτὸν, καὶ ἔλαβον τὴν κεφαλὴν αὐτοῦ καὶ τὰ σκεύη αὐτοῦ, καὶ ἀπέστειλαν εἰς γῆν ἀλλοφύλων κύκλῳ τοῦ εὐαγγελίσασθαι τοῖς εἰδώλοις αὐτῶν, καὶ τῷ λαῷ·
\VS{10}Καὶ ἔθηκαν τὰ σκεύη αὐτῶν ἐν οἴκῳ θεοῦ αὐτῶν· καὶ τὴν κεφαλὴν αὐτοῦ ἔθηκαν ἐν οἴκῳ Δαγών.
\par }{\PP \VS{11}Καὶ ἤκουσαν πάντες οἱ κατοικοῦντες Γαλαὰδ ἅπαντα ἃ ἐποίησαν οἱ ἀλλόφυλοι τῷ Σαοὺλ καὶ τῷ Ἰσραήλ.
\VS{12}Καὶ ἠγέρθησαν ἐκ Γαλαὰδ πᾶς ἀνὴρ δυνατὸς, καὶ ἔλαβον τὸ σῶμα Σαοὺλ καὶ τὸ σῶμα τῶν υἱῶν αὐτοῦ, καὶ ἤνεγκαν αὐτὰ εἰς Ἰαβὶς, καὶ ἔθαψαν τὰ ὀστᾶ αὐτῶν ὑπὸ τὴν δρὺν ἐν Ἰαβίς· καὶ ἐνήστευσαν ἑπτὰ ἡμέρας.
\VS{13}Καὶ ἀπέθανε Σαοὺλ ἐν ταῖς ἀνομίαις αὐτοῦ, αἷς ἠνόμησε τῷ Θεῷ κατὰ τὸν λόγον Κυρίου, διότι οὐκ ἐφύλαξεν, ὅτι ἐπηρώτησε Σαοὺλ ἐν τῷ ἐγγαστριμύθῳ τοῦ ζητῆσαι, καὶ ἀπεκρίνατο αὐτῷ Σαμουὴλ ὁ προφήτης,
\VS{14}καὶ οὐκ ἐζήτησε Κύριον· καὶ ἀπέκτεινεν αὐτὸν, καὶ ἐπέστρεψε τὴν βασιλείαν τῷ Δαυὶδ υἱῷ Ἰεσσαί.

\par }\Chap{11}{\PP \VerseOne{1}Καὶ ἦλθε πᾶς Ἰσραὴλ πρὸς Δαυὶδ ἐν Χεβρὼν, λέγοντες, ἰδοὺ ὀστᾶ σου καὶ σάρκες σου ἡμεῖς,
\VS{2}καὶ ἐχθὲς καὶ τρίτην ὄντος Σαοὺλ βασιλέως, σὺ ἦσθα ὁ ἐξάγων καὶ εἰσάγων τὸν Ἰσραήλ· καὶ εἶπεν Ἰσραὴλ Κύριός σοι, σὺ ποιμανεῖς τὸν λαόν μου τὸν Ἰσραὴλ, καὶ σὺ ἔσῃ εἰς ἡγούμενον ἐπὶ Ἰσραήλ.
\VS{3}Καὶ ἦλθον πάντες πρεσβύτεροι Ἰσραὴλ πρὸς τὸν βασιλέα εἰς Χεβρών· καὶ διέθετο αὐτοῖς ὁ βασιλεὺς Δαυὶδ διαθήκην ἐν Χεβρὼν ἔναντι Κυρίου· καὶ ἔχρισαν τὸν Δαυὶδ εἰς βασιλέα ἐπὶ Ἰσραὴλ κατὰ τὸν λόγον Κυρίου διὰ χειρὸς Σαμουήλ.
\par }{\PP \VS{4}Καὶ ἐπορεύθη ὁ βασιλεὺς καὶ ἄνδρες αὐτοῦ εἰς Ἱερουσαλὴμ, αὕτη Ἰεβοὺς, καὶ ἐκεῖ οἱ κατοικοῦντες τὴν γῆν εἶπον τῷ Δανιδ, οὐκ εἰσελεύση ὧδε·
\VS{5}καὶ προκατελάβετο τὴν περιοχὴν Σιών· αὕτη ἡ πόλις Δαυίδ.
\VS{6}Καὶ εἶπε Δαυὶδ, πᾶς τύπτων Ἰεβουσαῖον ἐν πρώτοις, καὶ ἔσται εἰς ἄρχοντα καὶ εἰς στρατηγόν· καὶ ἀνέβη ἐπʼ αὐτὴν ἐν πρώτοις Ἰωὰβ υἱὸς Σαρουία, καὶ ἐγένετο εἰς ἄρχοντα.
\VS{7}Καὶ ἐκάθισε Δαυὶδ ἐν τῇ περιοχῇ· διὰ τοῦτο ἐκάλεσεν αὐτὴν πόλιν Δαυίδ.
\VS{8}Καὶ ᾠκοδόμησε τὴν πόλιν κύκλῳ.
\VS{9}Καὶ ἐπορεύετο Δαυὶδ πορευόμενος καὶ μεγαλυνόμενος, καὶ Κύριος παντοκράτωρ μετʼ αὐτοῦ.
\VS{10}καὶ οὗτοι οἱ ἄρχοντες τῶν δυνατῶν, οἳ ἦσαν τῷ Δαυὶδ, οἱ κατισχύοντες μετʼ αὐτοῦ ἐν τῇ βασιλείᾳ αὐτοῦ μετὰ παντὸς Ἰσραὴλ, τοῦ βασιλεῦσαι αὐτὸν κατὰ τὸν λόγον Κυρίου ἐπὶ Ἰσραήλ.
\par }{\PP \VS{11}Καὶ οὗτος ὁ ἀριθμὸς τῶν δυνατῶν τοῦ Δαυίδ· Ἰεσεβαδὰ υἱὸς ʼΑχαμὰν πρῶτος τῶν τριάκοντα· οὗτος ἐσπάσατο τὴν ῥομφαίαν αὐτοῦ ἅπαξ ἐπὶ τριακοσίους τραυματίας ἐν καιρῷ ἑνί.
\VS{12}Καὶ μετʼ αὐτὸν Ἐλεάζαρ υἱὸς Δωδαῒ ὁ Ἀχωχί· οὗτος ἦν ἐν τοῖς τρισὶ δυνατοῖς.
\VS{13}Οὗτος ἦν μετὰ Δαυὶδ ἐν Φασσδαμὶν, καὶ οἱ ἀλλόφυλοι συνήχθησαν ἐκεῖ εἰς πόλεμον, καὶ ἦν μερὶς τοῦ ἀγροῦ πλήρης κριθῶν, καὶ ὁ λαὸς ἔφυγεν ἀπὸ προσώπου ἀλλοφύλων.
\VS{14}Καὶ ἔστη ἐν μέσῳ τῆς μερίδος, καὶ ἔσωσεν αὐτὴν, καὶ ἐπάταξε τοὺς ἀλλοφύλους, καὶ ἐποίησε Κύριος σωτηρίαν μεγάλην.
\par }{\PP \VS{15}Καὶ κατέβησαν τρεῖς ἐκ τῶν τριάκοντα ἀρχόντων εἰς τὴν πέτραν πρὸς Δαυὶδ εἰς τὸ σπήλαιον Ὀδολλὰμ, καὶ παρεμβολὴ τῶν ἀλλοφύλων ἐν τῇ κοιλάδι τῶν γιγάντων.
\VS{16}Καὶ Δαυὶδ τότε ἐν τῇ περιοχῇ, καὶ τὸ σύστημα τῶν ἀλλοφύλων τότε ἐν Βηθλεέμ.
\VS{17}Καὶ ἐπεθύμησε Δαυὶδ, καὶ εἶπε, τίς ποτιεῖ με ὕδωρ ἐκ τοῦ λάκκου Βηθλεὲμ τοῦ ἐν τῇ πύλῃ;
\VS{18}Καὶ διέῤῥηξαν οἱ τρεῖς τὴν παρεμβολὴν τῶν ἀλλοφύλων· καὶ ὑδρεύσαντο ὕδωρ ἐκ τοῦ λάκκου τοῦ ἐν Βηθλεὲμ, ὃς ἦν ἐν τῇ πύλῃ, καὶ ἔλαβον καὶ ἤλθον πρὸς Δαυίδ· καὶ οὐκ ἠθέλησε Δαυὶδ τοῦ πιεῖν αὐτὸ, καὶ ἔσπεισεν αὐτὸ τῷ Κυρίῳ,
\VS{19}καὶ εἶπεν, ἵλεώς μοι ὁ θεὸς τοῦ ποιῆσαι τὸ ῥῆμα τοῦτο· εἰ αἷμα ἀνδρῶν τούτων πίομαι ἐν ψυχαῖς αὐτῶν; ὅτι ἐν ψυχαῖς αὐτῶν ἤνεγκαν· καὶ οὐκ ἐβούλετο πιεῖν αὐτό· ταῦτα ἐποίησαν οἱ τρεῖς δυνατοί.
\par }{\PP \VS{20}Καὶ Ἀβισὰ ἀδελφὸς Ἰωὰβ, οὗτος ἦν ἄρχων τῶν τριῶν· οὗτος ἐσπάσατο τὴν ῥομφαίαν αὐτοῦ ἐπὶ τριακοσίους τραυματίας ἐν καιρῷ ἑνὶ, καὶ οὗτος ἦν ὀνομαστὸς ἐν τοῖς τρισίν.
\VS{21}Ἀπὸ τῶν τριῶν ὑπὲρ τοὺς δύο ἔνδοξος, καὶ ἦν αὐτοῖς εἰς ἄρχοντα, καὶ ἕως τῶν τριῶν οὐκ ἤρχετο.
\par }{\PP \VS{22}Καὶ Βαναία υἱὸς Ἰωδαὲ υἱὸς ἄνδρος δυνατοῦ, πολλὰ ἔργα αὐτοῦ ὑπὲρ Καβασαήλ· οὗτος ἐπάταξε τοὺς δύο ἀριὴλ Μωὰβ, καὶ οὗτος κατέβη καὶ ἐπάταξε τὸν λέοντα ἐν τῷ λάκκῳ ἐν ἡμέρᾳ χιόνος.
\VS{23}Καὶ οὗτος ἐπάταξε τὸν ἄνδρα τὸν Αἰγύπτιον, ἄνδρα ὁρατὸν πεντάπηχυν, καὶ ἐν χειρὶ τοῦ Αἰγυπτίου δόρυ ὡς ἀντίον ὑφαινόντων· καὶ κατέβη ἐπʼ αὐτὸν Βαναία ἐν ῥάβδῳ, καὶ ἀφείλατο ἐκ τῆς χειρὸς τοῦ Αἰγυπτίου τὸ δόρυ, καὶ ἀπέκτεινεν αὐτὸν ἐν τῷ δόρατι αὐτοῦ.
\VS{24}Ταῦτα ἐποίησε Βαναία υἱὸς Ἰωδαὲ, καὶ τούτῳ ὄνομα ἐν τοῖς τρισὶ τοῖς δυνατοῖς. Ὑπὲρ τοὺς τριάκοντα ἦν ἔνδοξος οὗτος, καὶ πρὸς τοὺς τρεῖς οὐκ ἤρχετο· καὶ κατέστησεν αὐτὸν Δαυὶδ ἐπὶ τὴν πατριὰν αὐτοῦ.
\VS{25}Ὑπὲρ τοὺς τριάκοντα ἦν ἔνδοξος οὗτος, καὶ πρὸς τοὺς τρεῖς οὐκ ἤρχετο· καὶ κατέστησεν αὐτὸν Δαυεὶδ ἐπὶ τὴν πατριὰν αὐτοῦ.
\par }{\PP \VS{26}Καὶ δυνατοὶ τῶν δυνάμεων, Ἀσαὴλ ἀδελφὸς Ἰωὰβ, Ἐλεανὰν υἱὸς Δωδωὲ ἐκ Βηθλεὲμ,
\VS{27}Σαμαὼθ ὁ ʼΑρωρὶ, Χελλὴς ὁ Φελωνὶ,
\VS{28}Ὠρὰ υἱὸς Ἐκκὶς ὁ Θεκωί, Ἀβιέζερ ὁ Ἀναθωθὶ,
\VS{29}Σοβοχαὶ ὁ Οὐσαθὶ, Ἠλὶ ὁ Ἀχωνὶ,
\VS{30}Μαραῒ ὁ Νετωφαθὶ, Χθαὸδ υἱὸς Νοοζὰ ὁ Νετωφαθὶ,
\VS{31}Αἰρεὶ υἱὸς Ῥεβιὲ ἐκ βουνοῦ Βενιαμὶν, Βαναίας ὁ Φαραθωνὶ,
\VS{32}Οὐρὶ ἐκ Ναχαλὶ Γάας, Ἀβιὴλ ὁ Γαραβαιθὶ,
\VS{33}Ἀζβὼν ὁ Βαρμὶ, Ἐλιαβὰ ὁ Σαλαβωνὶ,
\VS{34}υἱὸς Ἀσὰμ τοῦ Γιζωνίτου, Ἰωνάθαν υἱὸς Σωλὰ ὁ Ἀραρὶ,
\VS{35}Ἀχὶμ υἱὸς Ἀχὰρ ὁ ʼΑραρὶ, Ἐλφὰτ υἱὸς Θυροφὰρ
\VS{36}ὁ Μεχωραθρὶ, Ἀχία ὁ Φελλωνὶ,
\VS{37}ʼΗσερὲ ὁ Χαρμαδαῒ, Νααραὶ υἱὸς Ἀζοβαὶ,
\VS{38}Ἰωὴλ υἱὸς Νάθαν, Μεβαὰλ υἱὸς ʼΑγαρὶ,
\VS{39}Σελὴ ὁ Ἀμμωνὶ, Ναχὼρ ὁ Βηρωθὶ, αἴρων σκεύη υἱῷ Σαρουία,
\VS{40}Ἰρὰ ὁ Ἰεθρὶ, Γαβὴρ ὁ Ἰεθρὶ,
\VS{41}Οὐρία ὁ Χεττὶ, Ζαβὲτ υἱὸς Ἀχαϊὰ,
\VS{42}Ἀδινὰ υἱὸς Σαιζὰ τοῦ Ῥουβὴν ἄρχων, καὶ ἐπʼ αὐτῷ τριάκοντα,
\VS{43}ʼΑνὰν υἱὸς Μοωχὰ, καὶ Ἰωσαφὰτ ὁ Ματθανί,
\VS{44}Ὀζία ὁ Ἀσταρωθὶ, Σαμαθὰ καὶ Ἰεϊὴλ υἱοὶ Χωθὰν τοῦ Ἀραρὶ,
\VS{45}Ἰεδιὴλ υἱὸς Σαμερὶ, καὶ Ἰωζαὲ ὁ ἀδελφὸς αὐτοῦ ὁ Θωσαῒ,
\VS{46}Ἐλιὴλ ὁ Μαωὶ, καὶ Ἰαριβὶ, καὶ Ἰωσία υἱὸς αὐτοῦ, Ἐλλαὰμ, καὶ Ἰεθαμὰ ὁ Μωαβίτης,
\VS{47}Δαλιὴλ, καὶ Ὠβὴθ, καὶ Ἰεσσειὴλ ὁ Μεσωβία.

\par }\Chap{12}{\PP \VerseOne{1}Καὶ οὗτοι οἱ ἐλθόντες πρὸς Δαυὶδ εἰς Σικελὰγ, ἔτι συνεχομένου ἀπὸ προσώπου Σαοὺλ υἱοῦ Κίς· καὶ οὗτοι ἐν τοῖς δυνατοῖς βοηθοῦντες ἐν πολέμῳ,
\VS{2}καὶ τόξῳ ἐκ δεξιῶν καὶ ἐκ ἀριστερῶν, καὶ σφενδονῆται ἐν λίθοις καὶ τόξοις· ἐκ τῶν ἀδελφῶν Σαοὺλ ἐκ Βενιαμὶν
\VS{3}ὁ ἄρχων Ἀχιέζερ, καὶ Ἰωὰς υἱὸς Ἀσμὰ τοῦ Γαβαθίτου, καὶ Ἰωὴλ, καὶ Ἰωφαλὴτ υἱοὶ Ἀσμὼθ, καὶ Βερχία, καὶ Ἰηοὺλ ὁ Ἀναθωθὶ,
\VS{4}καὶ Σαμαΐας ὁ Γαβαωνείτης δυνατὸς ἐν τοῖς τριάκοντα, καὶ ἐπὶ τῶν τριάκοντα
\VS{5}Ἱερεμία, καὶ Ἰεζιὴλ, καὶ Ἰωανὰν, καὶ Ἰωαζαβὰθ ὁ Γαδαραθιῒμ,
\VS{6}Ἀζαῒ, καὶ Ἀριμοὺθ, καὶ Βααλιὰ, καὶ Σαμαραΐα, καὶ Σαφατίας ὁ Χαραιφιὴλ,
\VS{7}Ἑλκανὰ, καὶ Ἰησουνὶ, καὶ Ὀζριὴλ, καὶ Ἰωζαρὰ, καὶ Σοβοκὰμ, καὶ οἱ Κορῖται,
\VS{8}καὶ Ἰελία καὶ Ζαβαδία υἱοὶ ʼΙροὰμ, καὶ οἱ τοῦ Γεδώρ.
\par }{\PP \VS{9}Καὶ ἀπὸ τοῦ Γαδδὶ ἐχωρίσθησαν πρὸς Δαυὶδ ἀπὸ τῆς ἐρήμου ἰσχυροὶ δυνατοὶ ἄνδρες παρατάξεως πολέμου, αἴροντες θυρεοὺς καὶ δόρατα, καὶ πρόσωπον λέοντος τὰ πρόσωπα αὐτῶν, καὶ κοῦφοι ὡς δορκάδες ἐπὶ τῶν ὀρέων τῷ τάχει·
\VS{10}ʼΑζὰ ὁ ἄρχων, Ἀβδία ὁ δεύτερος, Ἐλιὰβ ὁ τρίτος,
\VS{11}Μασμανὰ ὁ τέταρτος, Ἱερεμίας ὁ πέμπτος,
\VS{12}Ἰεθὶ ὁ ἕκτος, Ἐλιὰβ ὁ ἕβδομος,
\VS{13}Ἰωανὰν ὁ ὄγδοος, Ἐλιαζὲρ ὁ ἔννατος,
\VS{14}Ἱερεμία ὁ δέκατος, Μελχαβαναὶ ὁ ἑνδέκατος.
\VS{15}Οὗτοι ἐκ τῶν υἱῶν Γὰδ ἄρχοντες τῆς στρατιᾶς, εἷς τοῖς ἑκατὸν μικρὸς, καὶ μέγας τοῖς χιλίοις.
\VS{16}Οὗτοι οἱ διαβάντες τὸν Ἰορδάνην ἐν τῷ μηνὶ τῷ πρώτῳ· καὶ οὗτος πεπληρωκὼς ἐπὶ πᾶσαν κρηπίδα αὐτοῦ· καὶ ἐξεδίωξαν πάντας τοὺς κατοικοῦντας αὐλῶνας ἀπὸ ἀνατολῶν ἕως δυσμῶν.
\par }{\PP \VS{17}Καὶ ἦλθον ἀπὸ τῶν υἱῶν Βενιαμὶν καὶ Ἰούδα εἰς βοήθειαν τοῦ Δαυίδ.
\VS{18}Καὶ Δαυὶδ ἐξῆλθεν εἰς ἀπάντησιν αὐτῶν, καὶ εἶπεν αὐτοῖς, εἰ εἰς εἰρήνην ἥκατε πρὸς μέ, εἴη μοι καρδία καθʼ ἑαυτὴν ἐφʼ ὑμᾶς· καὶ εἰ τοῦ παραδοῦναί με τοῖς ἐχθροῖς μου οὐκ ἐν ἀληθείᾳ χειρὸς, ἴδοι ὁ Θεὸς τῶν πατέρων ὑμῶν καὶ ἐλέγξαιτο.
\VS{19}Καὶ πνεῦμα ἐνέδυσε τὸν Ἀμασαὶ ἄρχοντα τῶν τριάκοντα, καὶ εἶπε, πορεύου καὶ ὁ λαός σου Δαυὶδ υἱὸς Ἰεσσαὶ, εἰρήνη εἰρήνη σοι, καὶ εἰρήνη τοῖς βοηθοῖς σου, ὅτι ἐβοήθησέ σοι ὁ Θεός σου· καὶ προσεδέξατο αὐτοὺς Δαυὶδ, καὶ κατέστησεν αὐτοὺς ἄρχοντας τῶν δυνάμεων.
\par }{\PP \VS{20}Καὶ ἀπὸ Μανασσῆ προσεχώρησαν πρὸς Δαυὶδ ἐν τῷ ἐλθεῖν τοὺς ἀλλοφύλους ἐπὶ Σαοὺλ εἰς πόλεμον· καὶ οὐκ ἐβοήθησεν αὐτοῖς, ὅτι ἐν βουλῇ ἐγένετο παρὰ τῶν στρατηγῶν τῶν ἀλλοφύλων λεγόντων, ἐν ταῖς κεφαλαῖς τῶν ἀνδρῶν ἐκείνων ἐπιστρέψει πρὸς τὸν κυρίον αὐτοῦ Σαούλ.
\VS{21}Ἐν τῷ πορευθῆναι τὸν Δαυίδ εἰς Σικελὰγ προσεχώρησαν αὐτῷ ἀπὸ Μανασσῆ, Ἐδνὰ, καὶ Ἰωζαβὰθ, καὶ Ῥωδιὴλ, καὶ Μιχαὴλ, καὶ Ἰωσαβαὶθ, καὶ Ἐλιμοὺθ, καὶ Σεμαθὶ· ἀρχηγοὶ χιλιάδων εἰσὶ τοῦ Μανασσῆ.
\VS{22}Καὶ αὐτοὶ συνεμάχησαν τῷ Δαυὶδ ἐπὶ τὸν γεδδοὺρ, ὅτι δυνατοὶ ἰσχύος πάντες· καὶ ἦσαν ἡγούμενοι ἐν τῇ στρατιᾷ ἐν τῇ δυνάμει,
\VS{23}ὅτι ἡμέραν ἐξ ἡμέρας ἤρχοντο πρὸς Δαυὶδ εἰς δύναμιν μεγάλην ὡς δύναμις τοῦ Θεοῦ.
\par }{\PP \VS{24}Καὶ ταῦτα τὰ ὀνόματα τῶν ἀρχόντων τῆς στρατιᾶς, οἱ ἐλθόντες πρὸς Δαυὶδ εἰς Χεβρὼν τοῦ ἀποστρέψαι τὴν βασιλείαν Σαοὺλ πρὸς αὐτὸν κατὰ τὸν λόγον Κυρίου.
\VS{25}Υἱοὶ Ἰούδα θυρεοφόροι καὶ δορατοφόροι, ἓξ χιλιάδες καὶ ὀκτακόσιοι δυνατοὶ παρατάξεως.
\VS{26}Τῶν υἱῶν Συμεὼν, δυνατοὶ ἰσχύος εἰς παράταξιν, ἑπτὰ χιλιάδες καὶ ἑκατόν.
\VS{27}Τῶν υἱῶν Λευὶ, τετρακισχίλιοι καὶ ἑξακόσιοι.
\VS{28}Καὶ Ἰωαδὰς ὁ ἡγούμενος τῷ Ἀαρὼν, καὶ μετʼ αὐτοῦ τρεῖς χιλιάδες καὶ ἑπτακόσιοι.
\VS{29}Καὶ Σαδὼκ νέος δυνατὸς ἰσχύϊ, καὶ τῆς πατρικῆς οἰκίας αὐτοῦ ἄρχοντες εἰκοσιδύο.
\VS{30}Καὶ τῶν υἱῶν Βενιαμὶν τῶν ἀδελφῶν Σαοὺλ, τρεῖς χιλιάδες· καὶ ἔτι τὸ πλεῖστον αὐτῶν ἀπεσκόπει τὴν φυλακὴν οἴκου Σαούλ.
\VS{31}Καὶ ἀπὸ υἱῶν Ἐφραὶμ, εἴκοσι χιλιάδες καὶ ὀκτακόσιοι, δυνατοὶ ἰσχύϊ ἄνδρες ὀνομαστοὶ κατʼ οἴκους πατριῶν αὐτῶν.
\VS{32}Καὶ ἀπὸ τοῦ ἡμίσους φυλῆς Μανασσῆ, δεκαοκτὼ χιλιάδες, καὶ οἳ ὠνομάσθησαν ἐν ὀνόματι τοῦ βασιλεῦσαι τὸν Δαυίδ.
\VS{33}Καὶ ἀπὸ τῶν υἱῶν Ἰσσάχαρ γινώσκοντες σύνεσιν εἰς τοὺς καιροὺς, γινώσκοντες τί ποιῆσαι Ἰσραὴλ, διακόσιοι, καὶ πάντες ἀδελφοὶ αὐτῶν μετʼ αὐτῶν.
\par }{\PP \VS{34}Καὶ ἀπὸ Ζαβουλὼν ἐκπορευόμενοι εἰς παράταξιν πολέμου ἐν πᾶσι σκεύεσι πολεμικοῖς πεντήκοντα χιλιάδες βοηθῆσαι τῷ Δαυὶδ οὐ χεροκένως.
\VS{35}Καὶ ἀπὸ Νεφθαλὶ ἄρχοντες χίλιοι, καὶ μετʼ αὐτῶν ἐν θυρεοῖς καὶ δόρασι, τριακονταεπτὰ χιλιάδες.
\VS{36}Καὶ ἀπὸ τῶν Δανιτῶν παρατασσόμενοι εἰς πόλεμον, εἰκοσιοκτὼ χιλιάδες καὶ ὀκτακόσιοι.
\VS{37}Καὶ ἀπὸ τοῦ Ἀσὴρ ἐκπορευόμενοι βοηθῆσαι εἰς πόλεμον, τεσσαράκοντα χιλιάδες.
\VS{38}Καὶ ἐκ πέραν τοῦ Ἰορδάνου ἀπὸ Ῥουβὴν, καὶ Γαδδὶ, καὶ ἀπὸ τοῦ ἡμίσους φυλῆς Μανασσῆ ἐν πᾶσι σκεύεσι πολεμικοῖς, ἑκατὸν εἴκοσι χιλιάδες.
\par }{\PP \VS{39}Πάντες οὗτοι ἄνδρες πολεμισταὶ παρατασσόμενοι παράταξιν ἐν ψυχῇ εἰρηνικῇ· καὶ ἦλθον εἰς Χεβρὼν τοῦ βασιλεῦσαι τὸν Δαυὶδ ἐπὶ πάντα Ἰσραήλ· καὶ ὁ κατάλοιπος Ἰσραὴλ ψυχὴ μία τοῦ βασιλεῦσαι τὸν Δαυίδ.
\VS{40}Καὶ ἦσαν ἐκεῖ ἡμέρας τρεῖς ἐσθίοντες καὶ πίνοντες, ὅτι ἡτοίμασαν οἱ ἀδελφοὶ αὐτῶν.
\VS{41}Καὶ οἱ ὁμοροῦντες αὐτοῖς ἕως Ἰσσάχαρ καὶ Ζαβουλὼν καὶ Νεφθαλὶ, ἔφερον αὐτοῖς ἐπὶ τῶν καμήλων καὶ τῶν ὄνων καὶ τῶν ἡμιόνων καὶ ἐπὶ τῶν μόσχων βρώματα, ἄλευρα, παλάθας, σταφίδας, οἶνον, καὶ ἔλαιον, μόσχους καὶ πρόβατα εἰς πλῆθος, ὅτι εὐφροσύνη ἐν Ἰσραήλ.

\par }\Chap{13}{\PP \VerseOne{1}Καὶ ἐβουλεύσατο Δαυὶδ μετὰ τῶν χιλιάρχων καὶ τῶν ἑκατοντάρχων, παντὶ ἡγουμένῳ.
\VS{2}Καὶ εἶπε Δαυὶδ πάσῃ ἐκκλησίᾳ Ἰσραὴλ, εἰ ἐφʼ ὑμῖν ἀγαθὸν, καὶ παρὰ Κυρίου τοῦ Θεοῦ ἡμῶν εὐοδωθῇ, ἀποστείλωμεν πρὸς τοὺς ἀδελφοὺς ἡμῶν τοὺς ὑπολελειμμένους ἐν πάσῃ γῇ Ἰσραὴλ, καὶ μετʼ αὐτῶν οἱ ἱερεῖς οἱ Λευῖται ἐν πόλεσι κατασχέσεως αὐτῶν, καὶ συναχθήσονται πρὸς ἡμᾶς,
\VS{3}καὶ μετενέγκωμεν τὴν κιβωτὸν τοῦ Θεοῦ ἡμῶν πρὸς ἡμᾶς, ὅτι οὐκ ἐζήτησαν αὐτὴν ἀφʼ ἡμερῶν Σαούλ.
\VS{4}Καὶ εἶπε πᾶσα ἡ ἐκκλησία τοῦ ποιῆσαι οὕτως, ὅτι εὐθὺς ὁ λόγος ἐν ὀφθαλμοῖς παντὸς τοῦ λαοῦ.
\par }{\PP \VS{5}Καὶ ἐξεκκλησίασε Δαυὶδ τὸν πάντα Ἰσραὴλ ἀπὸ ὁρίων Αἰγύπτου καὶ ἕως εἰσόδου Ἡμὰθ τοῦ εἰσενέγκαι τὴν κιβωτὸν τοῦ Θεοῦ ἐκ πόλεως Ἰαρίμ.
\VS{6}Καὶ ἀνήγαγεν αὐτὴν Δαυίδ· καὶ πᾶς Ἰσραὴλ ἀνέβη εἰς πόλιν Δαυὶδ, ἣ ἦν τοῦ Ἰούδα, τοῦ ἀναγαγεῖν ἐκεῖθεν τὴν κιβωτὸν τοῦ Θεοῦ Κυρίου καθημένου ἐπὶ χερουβὶμ, οὗ ἐπεκλήθη ὄνομα αὐτοῦ.
\VS{7}Καὶ ἐπέθηκαν τὴν κιβωτὸν τοῦ Θεοῦ ἐφʼ ἅμαξαν καινὴν ἐξ οἴκου Ἀμιναδάβ· καὶ Ὀζὰ καὶ οἱ ἀδελφοὶ αὐτοῦ ἦγον τὴν ἅμαξαν.
\par }{\PP \VS{8}Καὶ Δαυὶδ καὶ πᾶς Ἰσραὴλ παίζοντες ἐναντίον τοῦ Θεοῦ ἐν πάσῃ δυνάμει, καὶ ἐν ψαλτῳδοῖς, καὶ ἐν κινύραις, καὶ ἐν νάβλαις, ἐν τυμπάνοις, καὶ ἐν κυμβάλοις, καὶ ἐν σάλπιγξι.
\VS{9}Καὶ ἤλθοσαν ἕως τῆς ἅλωνος· καὶ ἐξέτεινεν Ὀζὰ τὴν χεῖρα αὐτοῦ τοῦ κατασχεῖν τὴν κιβωτὸν, ὅτι ἐξέκλινεν αὐτὴν ὁ μόσχος.
\VS{10}Καὶ ἐθυμώθη Κύριος ὀργῇ ἐπὶ Ὀζά, καὶ ἐπάταξεν αὐτὸν ἐκεῖ διὰ τὸ ἐκτεῖναι τὴν χεῖρα αὑτοῦ ἐπὶ τὴν κιβωτὸν, καὶ ἀπέθανεν ἐκεῖ ἀπέναντι τοῦ Θεοῦ·
\VS{11}Καὶ ἠθύμησε Δαυὶδ, ὅτι διέκοψε Κύριος διακοπὴν ἐν Ὀζὰ, καὶ ἐκάλεσε τὸν τόπον ἐκεῖνον, Διακοπὴ Ὀζὰ, ἕως τῆς ἡμέρας ταύτης.
\VS{12}Καὶ ἐφοβήθη Δαυὶδ τὸν Θεὸν ἐν τῇ ἡμέρᾳ ἐκείνῃ, λέγων, πῶς εἰσοίσω τὴν κιβωτὸν τοῦ Θεοῦ πρὸς ἐμαυτόν;
\VS{13}Καὶ οὐκ ἀπέστρεψε Δαυὶδ τὴν κιβωτὸν πρὸς ἑαυτὸν εἰς πόλιν Δαυὶδ, καὶ ἐξέκλινεν αὐτὴν εἰς οἶκον Ἀβεδδαρὰ τοῦ Γεθαίου.
\par }{\PP \VS{14}Καὶ ἐκάθισεν ἡ κιβωτὸς τοῦ Θεοῦ ἐν οἴκῳ Ἀβεδδαρὰ τρεῖς μῆνας· καὶ εὐλόγησεν ὁ Θεὸς Ἀβεδδαρὰ, καὶ πάντα τὰ αὐτοῦ.

\par }\Chap{14}{\PP \VerseOne{1}Καὶ ἀπέστειλε Χειρὰμ βασιλεὺς Τύρου ἀγγέλους πρὸς Δαυὶδ, καὶ ξύλα κέδρινα, καὶ οἰκοδόμους, καὶ τέκτονας ξύλων, τοῦ οἰκοδομῆσαι αὐτῷ οἶκον.
\VS{2}Καὶ ἔγνω Δαυὶδ ὅτι ἡτοίμασεν αὐτὸν Κύριος εἰς βασιλέα ἐπὶ Ἰσραὴλ, ὅτι ηὐξήθη εἰς ὕψος ἡ βασιλεία αὐτοῦ διὰ τὸν λαὸν αὐτοῦ Ἰσραήλ.
\par }{\PP \VS{3}Καὶ ἔλαβε Δαυὶδ ἔτι γυναῖκας ἐν Ἱερουσαλήμ· καὶ ἐτέχθησαν Δαυὶδ ἔτι υἱοὶ καὶ θυγατέρες.
\VS{4}Καὶ ταῦτα τὰ ὀνόματα αὐτῶν τῶν τεχθέντων, οἳ ἦσαν αὐτῷ ἐν Ἱερουσαλήμ· Σαμαὰ, Σωβὰβ, Νάθαν, καὶ Σαλωμὼν,
\VS{5}καὶ Βαὰρ, καὶ Ἐλισὰ, καὶ Ἐλιφαλὴθ,
\VS{6}καὶ Ναγὲθ, καὶ Ναφὰθ, καὶ Ἰαφιὲ,
\VS{7}καὶ Ἐλισαμαὲ, καὶ Ἐλιαδὲ, καὶ Ἐλιφαλά.
\par }{\PP \VS{8}Καὶ ἤκουσαν ἀλλόφυλοι ὅτι ἐχρίσθη Δαυὶδ βασιλεὺς ἐπὶ πάντα Ἰσραήλ· καὶ ἀνέβησαν πάντες οἱ ἀλλόφυλοι ζητῆσαι τὸν Δαυίδ· καὶ ἤκουσε Δαυὶδ, καὶ ἐξῆλθεν εἰς ἀπάντησιν αὐτοῖς·
\VS{9}Καὶ ἀλλόφυλοι ἦλθον, καὶ συνέπεσον ἐν τῇ κοιλάδι τῶν γιγάντων.
\VS{10}Καὶ ἐπηρώτησε Δαυὶδ διὰ τοῦ Θεοῦ, λέγων, εἰ ἀναβῶ ἐπὶ τοὺς ἀλλοφύλους, καὶ δώσεις αὐτοὺς εἰς τὰς χεῖράς μου; καὶ εἶπεν αὐτῷ Κύριος, ἀνάβηθι, καὶ δώσω αὐτοὺς εἰς τὰς χεῖράς σου.
\VS{11}Καὶ ἀνέβη εἰς Βαὰλ Φαρασὶν, καὶ ἐπάταξεν αὐτοὺς ἐκεῖ Δανίδ· καὶ εἶπε Δαυὶδ, διέκοψεν ὁ Θεὸς τοὺς ἐχθρούς μου ἐν χειρί μου, ὡς διακοπὴν ὕδατος· διὰ τοῦτο ἐκάλεσε τὸ ὄνομα τοῦ τόπου ἐκείνου, Διακοπὴ Φαρσίν.
\VS{12}Καὶ ἐγκατέλιπον ἐκεῖ τοὺς θεοὺς αὐτῶν οἱ ἀλλόφυλοι· καὶ εἶπε Δαυὶδ κατακαῦσαι ἐν πυρί.
\par }{\PP \VS{13}Καὶ προσέθεντο ἔτι ἀλλόφυλοι, καὶ συνέπεσαν ἔτι ἐν τῇ κοιλάδι τῶν γιγάντων.
\VS{14}Καὶ ἠρώτησε Δαυὶδ ἔτι ἐν Θεῷ· καὶ εἶπεν αὐτῷ ὁ Θεὸς, οὐ πορεύσῃ ὀπίσω αὐτῶν· ἀποστρέφου ἀπʼ αὐτῶν, καὶ παρέσῃ αὐτοῖς πλησίον τῶν ἀπίων.
\VS{15}Καὶ ἔσται ἐν τῷ ἀκοῦσαί σε τὴν φωνὴν τοῦ συσσεισμοῦ αὐτῶν ἄκρων τῶν ἀπίων, τότε εἰσελεύσῃ εἰς τὸν πόλεμον, ὅτι ἐξῆλθεν ὁ Θεὸς ἔμπροσθέν σου τοῦ πατάξαι τὴν παρεμβολὴν τῶν ἀλλοφύλων.
\VS{16}Καὶ ἐποίησε καθὼς ἐνετείλατο αὐτῷ ὁ Θεός· καὶ ἐπάταξε τὴν παρεμβολὴν τῶν ἀλλοφύλων ἀπὸ Γαβαὼν ἕως Γαζηρά.
\VS{17}Καὶ ἐγένετο ὄνομα Δαυὶδ ἐν πάσῃ τῇ γῇ, καὶ Κύριος ἔδωκε τὸν φόβον αὐτοῦ ἐπὶ πάντα τὰ ἔθνη.

\par }\Chap{15}{\PP \VerseOne{1}Καὶ ἐποίησεν αὐτῷ οἰκίας ἐν πόλει Δαυὶδ, καὶ ἡτοίμασε τὸν τόπον τῇ κιβωτῷ τοῦ Θεοῦ, καὶ ἐποίησεν αὐτῇ σκηνήν.
\VS{2}Τότε εἶπε Δαυὶδ, οὐκ ἔστιν ἆραι τὴν κιβωτὸν τοῦ Θεοῦ, ἀλλʼ ἢ τοὺς Λευίτας, ὅτι αὐτοὺς ἐξελέξατο Κύριος αἴρειν τὴν κιβωτὸν Κυρίου, καὶ λειτουργεῖν αὐτῷ ἕως αἰῶνος.
\par }{\PP \VS{3}Καὶ ἐξεκκλησίασε Δαυὶδ τὸν πάντα Ἰσραὴλ ἐν Ἱερουσαλὴμ, τοῦ ἀνενέγκαι τὴν κιβωτὸν Κυρίου εἰς τὸν τόπον ὃν ἡτοίμασεν αὐτῇ.
\VS{4}Καὶ συνήγαγε Δαυὶδ τοὺς υἱοὺς Ἀαρὼν τοὺς Λευίτας.
\VS{5}Τῶν υἱῶν Καὰθ, Οὐριὴλ ὁ ἄρχων καὶ οἱ ἀδελφοὶ αὐτοῦ, ἑκατὸν εἴκοσι.
\VS{6}Τῶν υἱῶν Μεραρὶ, Ἀσαΐα ὁ ἄρχων καὶ οἱ ἀδελφοὶ αὐτοῦ, διακόσιοι εἴκοσι.
\VS{7}Τῶν υἱῶν Γεδσὼν, Ἰωὴλ ὁ ἄρχων καὶ οἱ ἀδελφοὶ αὐτοῦ, ἑκατὸν τριάκοντα·
\VS{8}Τῶν υἱῶν Ἐλισαφὰτ, Σεμεῒ ὁ ἄρχων καὶ οἱ ἀδελφοὶ αὐτοῦ, διακόσιοι.
\VS{9}Τῶν υἱῶν Χεβρὼμ, Ἐλιὴλ ὁ ἄρχων καὶ οἱ ἀδελφοὶ αὐτοῦ, ὀγδοήκοντα.
\VS{10}Τῶν υἱῶν Ὀζιὴλ, Ἀμιναδὰβ ὁ ἄρχων καὶ οἱ ἀδελφοὶ αὐτοῦ, ἑκατὸν δεκαδύο.
\par }{\PP \VS{11}Καὶ ἐκάλεσε Δαυὶδ τὸν Σαδὼκ καὶ Ἀβιάθαρ τοὺς ἱερεῖς, καὶ τοὺς Λευίτας, τὸν Οὐριὴλ, Ἀσαΐαν, καὶ Ἰωὴλ, καὶ Σεμαίαν, καὶ Ἐλιὴλ, καὶ Ἀμιναδὰβ,
\VS{12}καὶ εἶπεν αὐτοῖς, ὑμεῖς ἄρχοντες πατριῶν τῶν Λευιτῶν, ἁγνίσθητε ὑμεῖς καὶ οἱ ἀδελφοὶ ὑμῶν, καὶ ἀνοίσετε τὴν κιβωτὸν τοῦ Θεοῦ Ἰσραὴλ, οὗ ἡτοίμασα αὐτῇ·
\VS{13}Ὅτι οὐκ ἐν τῷ πρότερον ὑμᾶς εἶναι, διέκοψεν ὁ Θεὸς ἡμῶν ἐν ἡμῖν, ὅτι οὐκ ἐξεζητήσαμεν ἐν κρίματι.
\VS{14}Καὶ ἡγνίσθησαν οἱ ἱερεῖς καὶ οἱ Λευῖται, τοῦ ἀνενέγκαι τὴν κιβωτὸν Θεοῦ Ἰσραήλ.
\VS{15}Καὶ ἔλαβον οἱ υἱοὶ τῶν Λευιτῶν τὴν κιβωτὸν τοῦ Θεοῦ, ὡς ἐνετείλατο Μωυσῆς ἐν λόγῳ Θεοῦ κατὰ τὴν γραφὴν, ἐν ἀναφορεῦσιν ἐπʼ αὐτούς.
\par }{\PP \VS{16}Καὶ εἶπε Δανὶδ τοῖς ἄρχουσι τῶν Λευιτῶν, στήσατε τοὺς ἀδελφοὺς αὐτῶν τοὺς ψαλτῳδοὺς ἐν ὀργάνοις, νάβλαις, κινύραις, καὶ κυμβάλοις τοῦ φωνῆσαι εἰς ὕψος ἐν φωνῇ εὐφροσύνης.
\VS{17}Καὶ ἔστησαν οἱ Λευῖται τὸν Αἰμὰν υἱὸν Ἰωήλ· ἐκ τῶν ἀδελφῶν αὐτοῦ Ἀσὰφ υἱὸς Βαραχία· καὶ ἐκ τῶν νἱῶν Μεραρὶ ἀδελφῶν αὐτοῦ Αἰθὰν υἱὸς Κισαίου·
\VS{18}Καὶ μετʼ αὐτῶν οἱ ἀδελφοὶ αὐτῶν οἱ δεύτεροι, Ζαχαρίας, καὶ Ὀζιὴλ, καὶ Σεμιραμὼθ, καὶ Ἰεϊὴλ, καὶ Ἐλιωὴλ, καὶ Ἐλιὰβ, καὶ Βαναία, καὶ Μαασαΐα, καὶ Ματθαθία, καὶ Ἐλιφενὰ, καὶ Μακελλία, καὶ Ἀβδεδὸμ, καὶ Ἰεϊὴλ, καὶ Ὀζίας, οἱ πυλωροί.·
\VS{19}Καὶ οἱ ψαλτῳδοὶ, Αἰμὰν, Ἀσὰφ, καὶ Αἰθὰν ἐν κυμβάλοις χαλκοῖς τοῦ ἀκουσθῆναι ποιῆσαι.
\VS{20}Ζαχαρίας, καὶ Ὀζιὴλ, Σεμιραμὼθ, Ἰεϊὴλ, Ὠνι, Ἐλιὰβ, Μαασαίας, Βαναίας ἐν νάβλαις ἐπὶ ἀλαιμώθ.
\VS{21}Καὶ Ματταθίας, καὶ Ἐλιφαλοὺ, καὶ Μακενία, καὶ Ἀβδεδὸμ, καὶ Ἰεϊὴλ, καὶ Ὀζίας ἐν κινύραις ἁμασενὶθ τοῦ ἐνισχῦσαι.
\par }{\PP \VS{22}Καὶ Χωνενία ἄρχων τῶν Λευιτῶν ἄρχων τῶν ᾠδῶν, ὅτι συνετὸς ἦν.
\VS{23}Καὶ Βαραχία καὶ Ἐλκανὰ πυλωροὶ τῆς κιβωτοῦ.
\VS{24}Καὶ Σομνία, καὶ Ἰωσαφὰτ, καὶ Ναθαναὴλ, καὶ Ἀμασαῒ, καὶ Ζαχαρία, καὶ Βαναΐα, καὶ Ἐλιέζερ οἱ ἱερεῖς σαλπίζοντες ταῖς σάλπιγξιν ἔμπροσθεν τῆς κιβωτοῦ τοῦ Θεοῦ· καὶ Ἀβδεδὸμ καὶ Ἰεΐα πυλωροὶ τῆς κιβωτοῦ τοῦ Θεοῦ.
\par }{\PP \VS{25}Καὶ ἦν Δαυὶδ καὶ οἱ πρεσβύτεροι Ἰσραὴλ καὶ οἱ χιλίαρχοι οἱ πορευόμενοι τοῦ ἀναγαγεῖν τὴν κιβωτὸν τῆς διαθήκης ἐξ οἴκου Ἀβδεδὸμ ἐν εὐφροσύνῃ.
\VS{26}Καὶ ἐγένετο ἐν τῷ κατισχῦσαι τὸν Θεὸν τοὺς Λευίτας αἴροντας τὴν κιβωτὸν τῆς διαθήκης Κυρίου, καὶ ἔθυσαν ἀνʼ ἑπτὰ μόσχους, καὶ ἀνʼ ἑπτὰ κριούς.
\VS{27}Καὶ Δαυὶδ περιεζωσμένος ἐν στολῇ βυσσίνῃ, καὶ πάντες οἱ Λευῖται αἴροντες τὴν κιβωτὸν διαθήκης Κυρίου, καὶ οἱ ψαλτῳδοὶ, καὶ Χωνενίας ὁ ἄρχων τῶν ᾠδῶν τῶν ᾀδόντων, καὶ ἐπὶ Δαυὶδ στολὴ βυσσίνη.
\VS{28}Καὶ πᾶς Ἰσραὴλ ἀνάγοντες τὴν κιβωτὸν διαθήκης Κυρίου ἐν σημασίᾳ, καὶ ἐν φωνῇ σωφὲρ, καὶ ἐν σάλπιγξι, καὶ ἐν κυμβάλοις, ἀναφωνοῦντες ἐν νάβλαις καὶ ἐν κινύραις.
\VS{29}Καὶ ἐγένετο ἡ κιβωτὸς διαθήκης Κυρίου, καὶ ἦλθεν ἕως πόλεως Δαυίδ· καὶ Μελχὸλ ἡ θυγάτηρ Σαοὺλ παρέκυψε διὰ τῆς θυρίδος, καὶ εἶδε τὸν βασιλέα Δαυὶδ ὀρχούμενον καὶ παίζοντα, καὶ ἐξουδένωσεν αὐτὸν ἐν τῇ ψυχῇ αὐτῆς.

\par }\Chap{16}{\PP \VerseOne{1}Καὶ εἰσήνεγκαν τὴν κιβωτὸν τοῦ Θεοῦ, καὶ ἀπηρείσαντο αὐτὴν ἐν μέσῳ τῆς σκηνῆς ἧς ἔπηξεν αὐτῇ Δαυὶδ, καὶ προσήνεγκαν ὁλοκαυτώματα καὶ σωτηρίου ἐναντίον τοῦ Θεοῦ.
\VS{2}Καὶ συνετέλεσε Δαυὶδ ἀναφέρων ὁλοκαυτώματα καὶ σωτηρίου, καὶ εὐλόγησε τὸν λαὸν ἐν ὀνόματι Κυρίου.
\VS{3}Καὶ διεμέρισε παντὶ ἀνδρὶ Ἰσραὴλ ἀπὸ ἀνδρὸς καὶ ἕως γυναικὸς, τῷ ἀνδρὶ ἄρτον ἕνα ἀρτοκοπικὸν, καὶ ἀμορείτην.
\VS{4}Καὶ ἔταξε κατὰ πρόσωπον τῆς κιβωτοῦ διαθήκης Κυρίου ἐκ τῶν Λευιτῶν λειτουργοῦντας ἀναφωνοῦντας, καὶ ἐξομολογεῖσθαι καὶ αἰνεῖν Κύριον τὸν Θεὸν Ἰσραήλ·
\VS{5}Ἀσὰφ ὁ ἡγούμενος, καὶ δευτερεύων αὐτῷ Ζαχαρίας, Ἰεϊὴλ, Σεμιραμὼθ, καὶ Ἰεϊὴλ, Ματταθίας, Ἐλιὰβ, καὶ Βαναΐας, καὶ Ἀβδεδομ· καὶ Ἰεϊὴλ ἐν ὀργάνοις, νάβλαις, κινύραις, καὶ Ἀσὰφ ἐν κυμβάλοις ἀναφωνῶν·
\VS{6}Καὶ Βαναίας καὶ Ὀζιὴλ οἱ ἱερεῖς ἐν ταῖς σάλπιγξι διαπαντὸς ἐναντίον τῆς κιβωτοῦ τῆς διαθήκης τοῦ Θεοῦ.
\par }{\PP \VS{7}Ἐν τῇ ἡμέρᾳ ἐκείνῃ. Τότε ἔταξε Δαυὶδ ἐν ἀρχῇ τοῦ αἰνεῖν τὸν Κύριον ἐν χειρὶ Ἀσὰφ καὶ τῶν ἀδελφῶν αὐτοῦ.
\par }{\PP \VS{8}Ὠδη. Ἐξομολογεῖσθε τῷ Κυρίῳ, ἐπικαλεῖσθε αὐτὸν ἐν ὀνόματι αὐτοῦ, γνωρίσατε ἐν λαοῖς τὰ ἐπιτηδεύματα αὐτοῦ.
\VS{9}Ἄσατε αὐτῷ καὶ ὑμνήσατε αὐτῷ, διηγήσασθε πᾶσι τὰ θαυμάσια αὐτοῦ, ἃ ἐποίησε Κύριος.
\VS{10}Αἰνεῖτε ἐν ὀνόματι ἁγίῳ αὐτοῦ, εὐφρανθήσεται καρδία ζητοῦσα τὴν εὐδοκίαν αὐτοῦ.
\VS{11}Ζητήσατε τὸν Κύριον καὶ ἰσχύσατε, ζητήσατε τὸ πρόσωπον αὐτοῦ διαπαντός.
\VS{12}Μνημονεύετε τὰ θαυμάσια αὐτοῦ ἃ ἐποίησε, τέρατα καὶ κρίματα τοῦ στόματος αὐτοῦ.
\VS{13}σπέρμα Ἰσραὴλ παῖδες αὐτοῦ, υἱοὶ Ἰακὼβ ἐκλεκτοὶ αὐτοῦ.
\VS{14}Αὐτὸς Κύριος ὁ Θεὸς ἡμῶν, ἐν πάσῃ τῇ γῇ τὰ κρίματα αὐτοῦ.
\VS{15}Μνημονεύωμεν εἰς αἰῶνα διαθήκης αὐτοῦ, λόγον αὐτοῦ ὃν ἐνετείλατο εἰς χιλίας γενεὰς,
\VS{16}ὃν διέθετο τῷ Ἁβραὰμ, καὶ τὸν ὅρκον αὐτοῦ τῷ Ἰσαάκ.
\VS{17}Ἔστησεν αὐτὸν τῷ Ἰακὼβ εἰς πρόσταγμα, τῷ Ἰσραὴλ διαθήκην αἰώνιον,
\VS{18}λέγων, σοὶ δώσω τὴν γῆν Χαναὰν σχοίνισμα κληρονομίας ὑμῶν.
\VS{19}Ἐν τῷ γενέσθαι αὐτοὺς ὀλιγοστοὺς ἀριθμῷ, ὡς ἐσμικρύνθησαν, καὶ παρῴκησαν ἐν αὐτῇ,
\VS{20}καὶ ἐπορεύθησαν ἀπὸ ἔθνους εἰς ἔθνος, καὶ ἀπὸ βασιλείας εἰς λαὸν ἕτερον,
\VS{21}οὐκ ἀφῆκεν ἄνδρα τοῦ δυναστεῦσαι αὐτοὺς, καὶ ἤλεγξε περὶ αὐτῶν βασιλεῖς.
\VS{22}Μὴ ἅψησθε τῶν χριστῶν μου, καὶ ἐν τοῖς πρόφήταις μου μὴ πονηρεύεσθε.
\par }{\PP \VS{23}Ἄσατε τῷ Κυρίῳ πᾶσα ἡ γῆ, ἀναγγείλατε ἐξ ἡμέρας εἰς ἡμέραν σωτηρίαν αὐτοῦ.
\VS{23a}Ἐξηγεῖσθε ἐν τοῖς ἔθνεσι τὴν δόξαν αὐτοῦ, ἐν πᾶσι τοῖς λαοῖς τὰ θαυμάσια αὐτοῦ.
\VS{25}Ὅτι μέγας Κύριος καὶ αἰνετὸς σφόδρα, φοβερός ἐστιν ἐπὶ πάντας τοὺς θεούς.
\VS{26}Ὅτι πάντες οἱ θεοὶ τῶν ἐθνῶν εἴδωλα, καὶ ὁ Θεὸς ἡμῶν οὐρανοὺς ἐποίησε.
\VS{27}Δόξα καὶ ἔπαινος κατὰ πρόσωπον αὐτοῦ, ἰσχὺς καὶ καύχημα ἐν τόπῳ αὐτοῦ.
\VS{28}Δότε τῷ Κυρίῳ αἱ πατριαὶ τῶν ἐθνῶν, δότε τῷ Κυρίῳ δόξαν καὶ ἰσχὺν,
\VS{29}δότε τῷ Κυρίῳ δόξαν ὀνόματι αὐτοῦ· λάβετε δῶρα καὶ ἐνέγκατε κατὰ πρόσωπον αὐτοῦ, καὶ προσκυνήσατε Κυρίῳ ἐν αὐλαῖς ἁγίαις αὐτοῦ.
\VS{30}Φοβηθήτω ἀπὸ προσώπου αὐτοῦ πᾶσα ἡ γῆ, κατορθωθήτω ἡ γῆ, καὶ μὴ σαλευθήτω.
\VS{31}Εὐφρανθήτω ὁ οὐρανὸς, καὶ ἀγαλλιάσθω ἡ γῆ, καὶ εἰπάτωσαν ἐν τοῖς ἔθνεσι, Κύριος βασιλεύων.
\VS{32}Βομβήσει ἡ θάλασσα σὺν τῷ πληρώματι, καὶ ξὺλον ἀγροῦ καὶ πάντα τὰ ἐν αὐτῷ
\VS{33}Τότε εὐφρανθήσεται τὰ ξύλατοῦ δρυμοῦ ἀπὸ προσώπου Κυρίου, ὅτι ἦλθε κρίναι τὴν γῆν.
\VS{34}Ἐξομολογεῖσθε τῷ Κυρίῳ, ὅτι ἀγαθὸν, ὅτι εἰς τὸν αἰῶνα τὸ ἔλεος αὐτοῦ.
\VS{35}Καὶ εἴπατε, σῶσον ἡμᾶς, ὁ Θεὸς τῆς σωτηρίας ἡμῶν καὶ ἄθροισον ἡμᾶς. καὶ ἐξελοῦ ἡμᾶς ἐκ τῶν ἐθνῶν, τοῦ αἴνειν τὸ ὄνομα τὸ ἅγιόν σου, καὶ καυχᾶσθαι ἐν ταῖς αἰνέσεσί σου.
\VS{36}Εὐλογημένος Κύριος ὁ Θεὸς Ἰσραὴλ ἀπὸ τοῦ αἰῶνος καὶ ἕως τοῦ αἰῶνος·
\par }{\PP Καὶ ἐρεῖ πᾶς ὁ λαὸς, ἀμήν· καὶ ᾔνεσαν τῷ Κυρίῳ.
\par }{\PP \VS{37}Καὶ κατέλιπον ἐκεῖ ἔναντι τῆς κιβωτοῦ διαθήκης Κυρίου τὸν Ἀσὰφ καὶ τοὺς ἀδελφοὺς αὐτοῦ, τοῦ λειτουργεῖν ἐναντίον τῆς κιβωτοῦ διαπαντὸς τὸ τῆς ἡμέρας εἰς ἡμέραν.
\VS{38}Καὶ Ἀβδεδὸμ καὶ οἱ ἀδελφοὶ αὐτοῦ, ἑξήκοντα καὶ ὀκτώ· καὶ Ἀβδεδὸμ υἱὸς Ἰδιθοὺν, καὶ Ὀσὰ, εἰς τοὺς πυλωρούς.
\VS{39}Καὶ τὸν Σαδὼκ τὸν ἱερέα καὶ τοὺς ἀδελφοὺς αὐτοῦ τοὺς ἱερεῖς ἐναντίον τῆς σκηνῆς Κυρίου ἐν βαμὰ τῇ ἐν Γαβαὼν,
\VS{40}τοῦ ἀναφέρειν ὁλοκαυτώματα τῷ Κυρίῳ ἐπὶ τοῦ θυσιαστηρίου τῶν ὁλοκαυτωμάτων διαπαντὸς τοπρωῒ καὶ τοεσπέρας, καὶ κατὰ πάντα τὰ γεγραμμένα ἐν νόμῳ Κυρίου ὅσα ἐνετείλατο ἐφʼ υἱοῖς Ἰσραὴλ ἐν χειρὶ Μωυσῆ τοῦ θεράποντος τοῦ Θεοῦ.
\VS{41}Καὶ μετʼ αὐτοῦ Αἰμὰν καὶ Ἰδιθοὺν, καὶ οἱ λοιποὶ ἐκλεγέντες ἐπʼ ὀνόματος τοῦ αἰνεῖν τὸν Κύριον, ὅτι εἰς τὸν αἰῶνα τὸ ἔλεος αὐτοῦ.
\VS{42}Καὶ μετʼ αὐτῶν σάλπιγγες καὶ κὺμβαλα τοῦ ἀναφωνεῖν καὶ ὄργανα τῶν ῷδῶν τοῖ Θεοῦ, οἱ δὲ υἱοὶ Ἰδιθοὺν εἰς τὴν πύλην.
\par }{\PP \VS{43}Καὶ ἐπορεύθη πᾶς ὁ λαὸς ἕκαστος εἰς τὸν οἶκον αὐτοῦ, καὶ ἐπέστρεψε Δαυὶδ τοῦ εὐλογῆσαι τὸν οἶκον αὐτοῦ.

\par }\Chap{17}{\PP \VerseOne{1}Καὶ ἐγένετο ὡς κατῴκησε Δαυὶδ ἐν οἴκῳ αὐτοῦ, καὶ εἶπε Δαυὶδ πρὸς Νάθαν τὸν προφήτην, ἰδοὺ ἐγὼ κατοικῶ ἐν οἴκῳ κεδρίνῳ, καὶ ἡ κιβωτὸς διαθήκης Κυρίου ὑποκάτω δέῤῥεων.
\par }{\PP \VS{2}Καὶ εἶπε Νάθαν πρὸς Δαυὶδ, πᾶν τὸ ἐν τῇ ψυχῇ σου ποίει, ὅτι Θεὸς μετὰ σοῦ.
\par }{\PP \VS{3}Καὶ ἐγένετο ἐν τῇ νυκτὶ ἐκείνῃ, καὶ ἐγένετο λόγος Κυρίου πρὸς Νάθαν·
\VS{4}Πορεύου καὶ εἶπον πρὸς Δαυὶδ τὸν δοῦλόν μου, οὕτως εἶπε Κύριος, οὐ σὺ οἰκοδομήσεις μοι οἶκον τοῦ κατοικῆσαί με ἐν αὐτῷ.
\VS{5}Ὅτι οὐ κατῴκησα ἐν οἴκῳ ἀπὸ τῆς ἡμέρας ἧς ἀνήγαγον τὸν Ἰσραὴλ ἕως τῆς ἡμέρας ταύτης,
\VS{6}καὶ ἤμην ἐν σκηνῇ καὶ ἐν καλὺμματι ἐν πᾶσιν οἷς διῆλθον ἐν παντὶ Ἰσραήλ· εἰ λαλῶν ἐλάλησα πρὸς μίαν φυλὴν τοῦ Ἰσραὴλ οἷς ἐνετειλάμην τοῦ ποιμαίνειν τὸν λαόν μου, λέγων, ὅτι οὐκ ᾠκοδομήσατέ μοι οἶκον κέδρινον;
\VS{7}Καὶ νῦν οὕτως ἐρεῖς τῷ δούλῳ μου Δαυὶδ, τάδε λέγει Κύριος παντοκράτωρ, ἐγὼ ἔλαβον σε ἐκ τῆς μάνδρας ἐξόπισθεν τῶν ποιμνίων τοῦ εἶναι εἰς ἡγούμενον ἐπὶ τὸν λαόν μου Ἰσραήλ·
\VS{8}Καὶ ἤμην μετὰ σοῦ ἐν πᾶσιν οἷς ἐπορεύθης, καὶ ἐξωλόθρευσα πάντας τοὺς ἐχθρούς σου ἀπὸ προσώπου σου, καὶ ἐποίησά σοι ὄνομα κατὰ τὸ ὄνομα τῶν μεγάλων τῶν ἐπὶ τῆς γῆς.
\VS{9}Καὶ θήσομαι τόπον τῷ λαῷ μου Ἰσραὴλ, καὶ καταφυτεύσω αὐτὸν, καὶ κατασκηνώσει καθʼ ἑαυτὸν, καὶ οὐ μεριμνήσει ἔτι, καὶ οὐ προσθήσει υἱὸς ἀδικίας τοῦ ταπεινῶσαι αὐτὸν καθὼς ἀρχῆς, καὶ ἀφʼ ἡμερῶν ὧν ἔταξα κριτὰς ἐπὶ τὸν λαόν μου Ἰσραήλ·
\VS{10}καὶ ἐταπείνωσα πάντας τοὺς ἐχθρούς σου, καὶ αὐξήσω σε, καὶ οἶκον οἰκοδομήσει σοι Κύριος.
\VS{11}Καὶ ἔσται ὅταν πληρωθῶσιν ἡμέραι σου καὶ κοιμηθήσῃ μετὰ τῶν πατέρων σου, καὶ ἀναστήσω τὸ σπέρμα σου μετὰ σὲ ὅς ἔσται ἐκ τῆς κοιλίας σου, καὶ ἑτοιμάσω τὴν βασιλείαν αὐτοῦ.
\VS{12}Αὐτὸς οἰκοδομήσει μοι οἶκον, καὶ ἀνορθώσω τὸν θρόνον αὐτοῦ ἕως αἰῶνος.
\VS{13}Ἐγὼ ἔσομαι αὐτῷ εἰς πατέρα, καὶ αὐτὸς ἔσται μοι εἰς υἱόν· καὶ τὸ ἔλεός μου οὐκ ἀποστήσω ἀπʼ αὐτοῦ, ὡς ἀπέστησα ἀπὸ τῶν ὄντων ἔμπροσθέν σου.
\VS{14}Καὶ πιστώσω αὐτὸν ἐν οἴκῳ μου καὶ ἐν βασιλείᾳ αὐτοῦ ἕως αἰῶνος, καὶ ὁ θρόνος αὐτοῦ ἔσται ἀνωρθωμένος ἕως αἰῶνος.
\par }{\PP \VS{15}Κατὰ πάντας τοὺς λόγους τούτους, καὶ κατὰ πᾶσαν τὴν ὅρασιν ταύτην, οὕτως ἐλάλησε Νάθαν πρὸς Δαυίδ.
\par }{\PP \VS{16}Καὶ ἦλθεν ὁ βασιλεὺς Δαυὶδ καὶ ἐκὰθισεν ἀπέναντι Κυρίου, καὶ εἶπε, τίς εἰμι ἐγὼ Κύριε ὁ Θεός; καὶ τίς ὁ οἶκός μου, ὅτι ἠγάπησάς με ἕως αἰῶνος;
\VS{17}Καὶ ἐσμικρύνθη ταῦτα ἐνώπιόν σου ὁ Θεὸς, καὶ ἐλάλησας ἐπὶ τὸν οἶκον τοῦ παιδός σου ἐκ μακρῶν, καὶ ἐπεῖδές με ὡς ὅρασις ἀνθρώπου, καὶ ὕψωσάς με Κύριε ὁ Θεός.
\VS{18}Τί προσθήσει ἔτι Δαυὶδ πρὸς σέ τοῦ δοξάσαι; καὶ σὺ τὸν δοῦλόν σου οἶδας,
\VS{19}καὶ κατὰ τὴν καρδίαν σου ἐποίησας τὴν πᾶσαν μεγαλωσύνην.
\VS{20}Κύριε, οὐκ ἔστιν ὅμοιός σοι, καὶ οὐκ ἔστι Θεὸς πλὴν σοῦ, κατὰ πάντα ὅσα ἠκούσαμεν ἐν ὠσὶν ἡμῶν.
\VS{21}Καὶ οὐκ ἔστιν ὡς ὁ λαός σου Ἰσραὴλ ἔθνος ἔτι ἐτὶ τῆς γῆς, ὡς ὡδήγησεν αὐτὸν ὁ Θεὸς τοῦ λυτρώσασθαι λαὸν ἑαυτῷ, τοῦ θέσθαι ἑαυτῷ ὄνομα μέγα καὶ ἐπιφανὲς, τοῦ ἐκβαλεῖν ἀπὸ προσώπου λαοῦ σου οὕς ἐλυτρώσω ἐξ Αἰγύπτου ἔθνη.
\VS{22}Καὶ ἔδωκας τὸν λαόν σου Ἰσραὴλ, σεαυτῷ λαὸν ἕως αἰῶνος, καὶ σὺ Κύριος ἐγενήθης αὐτοῖς εἰς Θεόν.
\VS{23}Καὶ νῦν, Κύριε, ὁ λόγος σου ὅν ἐλάλησας πρὸς τὸν παῖδά σου καὶ ἐπὶ τὸν οἶκον αὐτοῦ, πιστωθήτω ἕως αἰῶνος· καὶ ποιήσον καθὼς ἐλάλησας,
\VS{24}καὶ πιστωθήτω καὶ μεγαλυνθήτω τὸ ὄνομά σου ἕως αἰῶνος, λεγόντων, Κύριε Κύριε παντοκράτωρ Θεὸς Ἰσραὴλ, καὶ ὁ οἶκος Δαυὶδ παιδός σου ἀνωρθωμένος ἐναντίον σου.
\VS{25}Ὅτι σὺ Κύριος ὁ Θεός μου ἤνοιξας τὸ οὖς τοῦ παιδός σου τοῦ οἰκοδομῆσαι αὐτῷ οἶκον, διὰ τοῦτο εὗρεν ὁ παῖς σου τοῦ προσεύξασθαι κατὰ πρόσωπόν σου.
\VS{26}Καὶ νῦν, Κύριε, σὺ εἶ αὐτὸς Θεὸς, καὶ ἐλάλησας ἐπὶ τὸν δοῦλόν σου τὰ ἀγαθὰ ταῦτα.
\VS{27}Καὶ νῦν, ἦρξαι τοῦ εὐλογῆσαι τὸν οἶκον τοῦ παιδός σου, τοῦ εἶναι εἰς τὸν αἰῶνα ἐναντίον σου· ὅτι σὺ Κύριε εὑλόγησας, καὶ εὐλόγησον εἰς τὸν αἰῶνα.

\par }\Chap{18}{\PP \VerseOne{1}Καὶ ἐγένετο μετὰ ταῦτα, καὶ ἐπάταξε Δαυὶδ τοὺς ἀλλοφύλους καὶ ἐτροπώσατο αὐτοὺς, καὶ ἔλαβε τὴν Γὲθ καὶ τὰς κώμας αὐτῆς ἐκ χειρὸς ἀλλοφύλων.
\par }{\PP \VS{2}Καὶ ἐπάταξε τὴν Μωὰβ, καὶ ἦσαν Μωὰβ παῖδες τῷ Δαυὶδ φέροντες δῶρα·
\par }{\PP \VS{3}Καὶ ἐπάταξε Δαυὶδ τὸν Ἀδρααζὰρ βασιλέα Σουβὰ Ἠμάθ, πορευομένου αὐτοῦ ἐπιστῆσαι χεῖρα αὐτοῦ ἐπὶ ποταμὸν Εὐφράτην·
\VS{4}Καὶ προκατελάβετο Δαυὶδ αὐτῶν χίλια ἅρματα καὶ ἑπτὰ χιλιάδας ἵππων καὶ εἴκοσι χιλιάδας ἀνδρῶν πεζῶν· καὶ παρέλυσεν Δαυεὶδ πάντα τὰ ἅρματα, καὶ ὑπελίπετο ἐξ αὐτῶν ἑκατὸν ἅρματα.
\VS{5}Καὶ ἦλθεν Σύρος ἐκ Δαμασκοῦ βοηθῆσαι Ἀδρααζὰρ βασιλεῖ Σουβά, καὶ ἐπάταξε Δαυὶδ ἐν τῷ Σύρῳ εἴκοσι καὶ δύο χιλιάδας ἀνδρῶν.
\VS{6}Καὶ ἔθετο Δαυὶδ φρουρὰν ἐν Συρίᾳ τῇ κατὰ Δαμασκὸν, καὶ ἦσαν τῷ Δαυεὶδ εἰς παῖδας φέροντας δῶρα· καὶ ἔσωσε Κύριος Δαυίδ ἐν πᾶσιν οἷς ἐπορεύετο.
\VS{7}Καὶ ἔλαβεν Δαυεὶδ τοὺς κλοιοὺς τοὺς χρυσοῦς οἳ ἦσαν ἐπὶ τοὺς παῖδας Ἁδραάζαρ, καὶ ἤνεγκεν αὐτοὺς εἰς Ἰερουσαλήμ.
\VS{8}Καὶ ἐκ τῆς Μαγαβὲθ καὶ ἐκ τῶν ἐκλεκτῶν πόλεων τῶν Ἀδρααζὰρ ἔλαβε Δαυὶδ χαλκὸν πολὺν σφόδρα· ἐξ αὐτοῦ ἐποίησε Σαλωμὼν τὴν θάλασσαν τὴν χαλκῆν, καὶ τοὺς στύλους καὶ τὰ σκεύη τὰ χαλκᾶ.
\par }{\PP \VS{9}καὶ ἤκουσε Θωὰ βασιλεὺς Ἠμὰθ, ὅτι ἐπάταξε Δαυὶδ τὴν πᾶσαν δύναμιν Ἀδρααζὰρ βασιλέως Σουβά·
\VS{10}Καὶ ἀπέστειλεν τὸν Ἀδουρὰμ υἱὸν αὐτοῦ πρὸς τὸν βασιλέα Δαυὶδ τοῦ ἐρωτῆσαι αὐτὸν τὰ εἰς εἰρήνην, καὶ του εὐλογῆσαι αὐτὸν ὑπὲρ οὗ ἐπολέμησε τὸν Ἁδρααζαρ, καὶ ἐπάταξεν αὐτόν, ὅτι ἀνὴρ πολέμιος Θῶα ἦν τῷ Ἁδραάζαρ· καὶ πάντα τὰ σκεύη τὰ χρυσᾶ, καὶ τὰ ἀργυρᾶ,
\VS{11}Καὶ τὰ χαλκᾶ, καὶ ταῦτα ἡγίασεν ὁ βασιλεὺς Δαυὶδ τῷ κυρίῳ, μετὰ τοῦ ἀργυρίου καὶ τοῦ χρυσίου οὗ ἔλαβεν ἐκ πάντων τῶν ἐθνῶν, ἐξ Ἰδουμαίας, καὶ Μωὰβ, καὶ ἐξ υἱῶν Ἀμμὼν, καὶ ἐκ τῶν ἀλλοφύλων, καὶ ἐξ Ἀμαλήκ.
\par }{\PP \VS{12}Καὶ Ἀβεσὰ υἱὸς Σαρουίας ἐπάταξε τὴν Ἰδουμαίαν ἐν κοιλάδι τῶν ἁλῶν, ὀκτὼκαὶδεκα χιλιάδας.
\VS{13}Καὶ ἔθετο ἐν τῇ κοιλάδι φρουρὰς, καὶ ἦσαν πάντες οἱ Ἰδουμαῖοι παῖδες Δαυίδ· καὶ ἔσωζε Κύριος τὸν Δανὶδ ἐν πᾶσιν οἷς ἐπορεύετο.
\par }{\PP \VS{14}Καὶ ἐβασίλευσε Δαυὶδ ἐπὶ πάντα Ἰσραὴλ, καὶ ἦν ποιῶν κρίμα καὶ δικαιοσύνην τῷ παντὶ λαῷ αὐτοῦ.
\VS{15}Καὶ Ἰωὰβ υἱὸς Σαρουίας ἐπὶ τῆς στρατιᾶς, καὶ Ἰωσαφὰτ υἱὸς Ἀχιλοὺδ ὁ ὑπομνηματογράφος,
\VS{16}καὶ Σαδὼκ υἱὸς Ἀχειτὼβ καὶ Αχειμέλεχ υἱὸς Ἀβιάθαρ οἱ ἱερεῖς, καὶ Σουσὰ γραμματεύς,
\VS{17}καὶ Βαναίας υἱὸς Ἰωδαὲ ἐπὶ τοῦ Χερεθὶ καὶ ἐπὶ τοῦ Φελεθί. καὶ υἱοὶ Δαυὶδ οἱ πρῶτοι διάδοχοι τοῦ βασιλέως.

\par }\Chap{19}{\PP \VerseOne{1}Καὶ ἐγένετο μετὰ ταῦτα ἀπέθανε Ναὰς βασιλεῦς υἱῶν Ἀμμών, καὶ ἐβασίλευσεν Ἀνὰν υἱὸς αὐτοῦ ἀντʼ αὐτοῦ.
\VS{2}Καὶ εἶπε Δαυίδ, Ποιήσω ἔλεος μετὰ Ἀνὰν υἱοῦ Ναὰς, ὡς ἐποίησεν ὁ πατὴρ αὐτοῦ μετʼ ἐμοῦ ἔλεος· καὶ ἀγγέλοους ἀγγέλους Δαυὶδ τοῦ παρακαλέσαι αὐτὸν περὶ τοῦ πατρὸς αὐτοῦ· καὶ ἦλθον παῖδες Δαυὶδ εἰς γῆν νἱῶν Αμμὼν πρὸς Ἀνὰν τοῦ παρακαλέσαι αὐτόν.
\VS{3}Καὶ εἶπον ἄρχοντες υἱῶν Ἀμμὼν πρὸς Ἀνάν, Μὴ δοξάζων Δαυὶδ τὸν πατέρα σου ἐναντίον σου ἀπέστειλέ σοι παρακαλοῦντας; οὐχ ὅπως ἐξερενήσωσι τὴν πόλιν, καὶ τοῦ κατασκοπῆσαι τὴν γῆν, ἦλθον παῖδες αὐτοῦ πρὸς σέ;
\VS{4}Καὶ ἔλαβεν Ἁνὰν τοὺς παῖδας Ααυὶδ, καὶ ἐξύρησεν αὐτοὺς, καὶ ἀφεῖλε τῶν μανδυῶν αὐτῶν τὸ ἥμισυ ἕως τῆς ἀναβολῆς, καὶ ἀπέστειλεν αὐτούς.
\VS{5}Καὶ ἦλθον ἀπαγγεῖλαι τῷ Δαυὶδ περὶ τῶν ἀνδρῶν· καὶ ἀπέστειλεν εἰς ἀπάντησιν αὐτοῖς, ὅτι ἦσαν ἠτιμωμένοι σφόδρα· καὶ εἶπεν ὁ βασιλεύς, Καθίσατε ἐν Ἱερειχὼ ἕως τοῦ ἀνατεῖλαι τοὺς πώγωνας ὑμῶν, καὶ ἀνακάμψατε.
\par }{\PP \VS{6}Καὶ εἶδον οἱ υἱοὶ Ἀμμὼν ὅτι ἠπχὺνθη λαὸς Δαυίδ, καὶ ἀπέστειλεν Ἁνὰν καὶ οἱ υἱοὶ Ἀμμὼν χίλια τάλαντα ἀργυρίου τοῦ μισθώσασθαι ἑαυτοῖς ἐκ Συρίας Μεσοποταμίας καὶ ἐκ Συρίας Μααχὰ καὶ παρὰ Σωβὰλ ἅρματα καὶ ἱππεῖς.
\VS{7}καὶ ἐμισθώσαντο ἑαυτοῖς δύο καὶ τριάκοντα χιλιάδας ἁρμάτων, καὶ τὸν βασιλέα Μααχὰ καὶ τὸν λαὸν αὐτοῦ· καὶ ἦλθον καὶ παρενέβαλον κατέναντι Μηδαβά· καὶ οἱ υἱοὶ Ἀμμὼν συνήχθησαν ἐκ τῶν πόλεων αὐτῶν, καὶ ἦλθον εἰς τὸ πολεμῆσαι.
\par }{\PP \VS{8}Καὶ ἤκουσεν Δαυίδ, καὶ απέστειλε τὸν Ἰωὰβ καὶ πᾶσαν τὴν στρατίαν τῶν δυνατῶν.
\VS{9}καὶ ἐξῆλθον οἱ υἱοὶ Ἀμμὼν, καὶ παρατάσσονται εἰς πόλεμον παρὰ τὸν πυλῶνα τῆς πόλεως· καὶ οἱ βασιλεῖς οἱ ἐλθόντες παρενέβαλον καθʼ ἑαυτοὺς ἐν τῷ πεδίῳ·
\VS{10}καὶ εἶδεν Ἰωὰβ ὅτι γεγόνασιν ἀντιπρόσωποι τοῦ πολεμεῖν πρὸς αὐτὸν κατὰ πρόσωπον καὶ ἐξόπισθε, καὶ ἐξελέξατο ἐκ παντὸς νεανίου ἐξ Ἰσραὴλ, καὶ παρετάξαντο ἐναντίον τοῦ Σύρου.
\VS{11}καὶ τὸ κατάλοιπον τοῦ λαοῦ ἔδωκεν ἐν χειρὶ Ἀβεσσὰ ἀδελφοῦ αὐτοῦ, καὶ παρετάξαντο ἐξεναντίας υἱῶν Ἀμμών.
\VS{12}καὶ εἶπεν, Ἐὰν κρατήσῃ ὑπὲρ ἐμὲ Σύρος, καὶ ἔσῃ μοι εἰς σωτηρίαν· καὶ ἐὰν υἱοὶ Ἀμμὼν κρατήσωσιν ὑπὲρ σέ, καὶ σώσω σε.
\VS{13}καὶ Κύριος τὸ ἀγαθὸν ποιήσει.
\par }{\PP \VS{14}Καὶ παρετάξατο Ἰωὰβ καὶ ὁ λαὸς ὁ μετʼ αὐτοῦ κατέναντι Σύρων εἰς πόλεμον, καὶ ἔφυγον ἀπʼ αὐτοῦ.
\VS{15}καὶ οἱ υἱοὶ Ἀμμὼν εἶδον ὅτι ἔφυγον Σύροι, καὶ ἔφυγον καὶ αὐτοὶ ἀπὸ προσώπου Ἰωὰβ καὶ ἀπὸ προσώπου ἀδελφοῦ αὐτοῦ, καὶ ἦλθον εἰς τὴν πόλιν· καὶ ἦλθεν Ἰωὰβ εἰς Ἰερουσαλήμ.
\par }{\PP \VS{16}Καὶ εἶδεν Σύρος ὅτι ἐτροπώσατο αὐτὸν Ἰσραήλ, καὶ ἀπέστειλαν ἀγγέλους· καὶ ἐξήγαγον τὸν Σύρον ἐκ τοῦ πέραν τοῦ ποταμοῦ, καὶ Σωφὰρ ἀρχιστράτηγος δυνάμεως Ἁδραάζαρ ἔμπροσθεν αὐτῶν.
\VS{17}καὶ ἀπηγγέλη τῷ Δαυείδ, καὶ συνήγαγεν τὸν πάντα Ἰσραήλ, καὶ διέβη τὸν Ἰορδάνην καὶ ἦλθεν ἐπʼ αὐτοὺς καὶ παρετάξατο ἐπʼ αὐτούς. καὶ παρατάσσεται Σύρος ἐξ ἐναντίας Δαυεὶδ καὶ ἐπολέμησαν αὐτόν.
\VS{18}καὶ ἔφυγεν Σύρος ἀπὸ προσώπου Δαυείδ, καὶ ἀπέκτεινεν Δαυεὶδ ἀπὸ τοῦ Σύρου ἐπτὰ χιλιάδας ἁρμάτων καὶ τεσσεράκοντα χιλιάδας πεζῶν, καὶ τὸν Σαφὰθ ἀρχιστράτηγον δυνάμεως ἀπέκτεινεν.
\VS{19}καὶ εἶδον παῖδες Ἁδραάζαρ ὅτι ἐπταίκασιν ἀπὸ προσώπου Ἰσραήλ, καὶ διέθεντο μετὰ Δαυεὶδ καὶ ἐδούλευσαν αὐτῷ· καὶ οὐκ ἠθέλησεν Σύρος τοῦ βοηθῆσαι Ἀμμὼν ἔτι.

\par }\Chap{20}{\PP \VerseOne{1}Καὶ ἐγένετο ἐν τῷ ἐπιόντι ἔτει ἐν τῇ ἐξόδῳ τῶν βασιλέων καὶ ἤγαγεν Ἰωὰβ πᾶσαν τὴν δὺναμιν τῆς στρατείας, καὶ ἔφθειραν τὴν χώραν υἱῶν Ἀμμών· καὶ ἦλθεν καὶ περιεκάθισεν τὴν Ῥάββαν. καὶ Δαυεὶδ ἐκάθητο ἐν Ἰερουσαλήμ· καὶ ἐπάταξεν τὴν Ῥαββὰ καὶ κατέσκαψεν αὐτήν.
\VS{2}καὶ ἔλαβεν Δαυεὶδ τὸν στέφανον Μολχὸλ βασιλέως αὐτῶν ἀπὸ τῆς κεφαλῆς αὐτοῦ, καὶ εὑρέθη ὁ σταθμὸς αὐτοῦ τάλαντον χρυσίου, καὶ ἐν αὐτῷ λίθος τίμιος, καὶ ἦν ἐπὶ τὴν κεφαλὴν Δαυείδ· καὶ σκῦλα τῆς πόλεως ἐξήνεγκεν πολλὰ σφόδρα.
\VS{3}καὶ τὸν λαὸν τὸν ἐν αὐτῇ ἐξήγαγεν καὶ διέπρισεν πρίοσιν καὶ ἐν σκεπάρνοις σιδηροῖς, καὶ οὕτως ἐποίησεν Δαυεὶδ τοῖς παισὶν υἱοῖς Ἀμμών· καὶ ἀνέστρεψεν Δαυεὶδ καὶ πᾶς ὁ λαὸς αὐτοῦ εἰς Ἰερουσαλήμ.
\par }{\PP \VS{4}Καὶ ἐγένετο μετὰ ταῦτα καὶ ἐγένετο ἔτι πόλεμος ἐν Γάζερ μετὰ τῶν ἀλλοφύλων· τότε ἐπάταξεν Σοβοχαὶ Θωσαθεὶ τὸν Σαφοὺτ ἀπὸ τῶν υἱῶν τῶν γιγάντων καὶ ἐταπείνωσεν αὐτόν.
\par }{\PP \VS{5}Καὶ ἐγένετο ἔτι πόλεμος μετὰ τῶν ἀλλοφύλων, καὶ ἐπάταξεν Ἐλλὰν υἱὸς Ἰαεὶρ τὸν Ἐλεμεὲ ἀδελφὸν Γολιὰθ τοῦ Γεθθαίου, καὶ ξύλον δόρατος αὐτοῦ ὡς ἀντίον ὑφαινόντων.
\par }{\PP \VS{6}Καὶ ἐγένετο ἔτι πόλεμος ἐν Γέθ, καὶ ἦν ἀνὴρ ὑπερμεγέθης, καὶ δάκτυλοι αὐτοῦ ἓξ καὶ ἓξ, εἴκοσι τέσσαρες· καὶ οὗτος ἦν ἀπόγονος γιγάντων.
\VS{7}Καὶ ὠνείδισεν τὸν Ἰσραήλ, καὶ ἐπάταξεν αὐτὸν Ἰωναθὰν υἱὸς Σαμαά, υἱὸς ἀδελφοῦ Δαυείδ.
\VS{8}Οὗτος ἐγένετο Ῥαφὰ ἐν Γέθ· πάντες ἦσαν τέσσαρες γίγαντες, καὶ ἔπεσον ἐν χειρὶ Δαυεὶδ καὶ ἐν χειρὶ παίδων αὐτοῦ.

\par }\Chap{21}{\PP \VerseOne{1}Καὶ ἔστη διάβολος ἐν τῷ Ἰσραήλ, καὶ ἐπέσεισεν τὸν Δαυὶδ τοῦ ἀριθμῆσαι τὸν ἰσραήλ.
\VS{2}Καὶ εἶπεν ὁ βασιλεὺς δαυὶδ πρὸς Ἰωὰβ καὶ τοὺς ἄρχοντας τῆς δυνάμεως, πορεύθητε, ἀριθμήσατε τὸν Ἰσραὴλ ἀρὸ Βηρσαβεὲ καὶ ἕως Δᾶν, καὶ ἐνέγκατε πρὸς μὲ, καὶ γνώσομαι τὸν ἀριθμὸν αὐτῶν.
\VS{3}Καὶ εἶπεν Ἰωάβ, Προσθείη Κύριος ἐπὶ τὸν λαὸν αὐτοῦ, ὡς αὐτοὶ ἑκατονταπλασίως, καὶ οἱ ὀφθαλμοὶ τοῦ κυρίου μυυ τοῦ βασιλέως βλέποντες· πάντες τῷ κυρίῳ μου παῖδεσ· ἱνατί ζητεῖ κύριός μου τοῦτο; ἳνα μὴ γένηται εἰς ἁμαρτίαν τῷ Ἰσραήλ.
\VS{4}Τὸ δὲ ρῆμα τοῦ βασιλέως ἴσχυσεν ἐπὶ Ιωὰβ, καὶ ἐξῆλθεν Ιωὰβ, καὶ διῆθεν ἐν παντὶ Ἰσραὴλ, καὶ ἦλθεν εἰς Ἰερουσλήμ.
\VS{5}Καὶ ἔδωκεν Ἰωὰβ τὸν ἀριθμὸν τῆς ἐπισκέψεως τοῦ λαοῦ τῷ Δαυίδ· καὶ ἦν πᾶς Ἰσραὴλ χίλιαι χιλιάδες καὶ ἑκατὸν χιλιάδες ἀνδρῶν ἐσπασμένων μάχαιραν·
\VS{6}Καὶ τὸν Λευὶ καὶ τὸν Βενιαμεὶν οὐκ ἠρίθμησεν ἐν μέσῳ αὐτῶν, ὅτι κατίσχυσν λόγος τοῦ βασιλέως τὸν Ἰωάβ.
\par }{\PP \VS{7}καὶ πονηρὸν ἐναντίον τοῦ Θεοῦ περὶ τοῦ πράγματος τούτου, καὶ ἐπάταξεν τὸν Ἰσραήλ.
\VS{8}Καὶ εἶπε Δαυὶδ πρὸς τὸν Θεόν, ἡμάρτηκα σφόδρα, ὅτι ἐποίησα τὸ πρᾶγμα τοῦτο, καὶ νῦν περίελε δὴ τὴν κακίαν παιδός σου, ὅτι ἐματαιώθην σφόδρα.
\par }{\PP \VS{9}Καὶ ἐλάλησε Κύριος πρὸς Γὰδ τὸν ὁρῶντα,
\VS{10}Πορεύου καὶ λάλησον πρὸς Δαυὶδ, λέγων, Οὕτως λέγει Κύριος, τρία αἴρω ἐγὼ ἐπι σέ, ἔκλεξαι σεαυτῷ ἓν ἐξ αὐτῶν, καὶ ποιήσω σοι.
\VS{11}καὶ ἦλθεν Γὰδ πρὸς Δαυὶδ, καὶ εἶπεν αὐτῷ, Οὕτως λέγει Κύριος, Ἔκλεξαι σεαυτῷ
\VS{12}ἢ τρία ἔτη λιμοῦ, ἢ τρεῖς μῆνας φεύγειν σε ἐκ προσώπου ἐχθρῶν σου, καὶ μάχαιρα ἐξ ἐχθρῶν σου τοῦ ἐξολεθρεῦσαι, ἢ τρεῖς ἡμέρας ῥομφαίαν Κυρίου καὶ θάνατον ἐν τῇ γῇ, καὶ ἄγγελος Κυρίου ἐξολεθρεύων ἐν πάσῃ κληρονομίᾳ Ἰσραήλ· καὶ νῦν ἴδε τί ἀποκριθῶ τῷ ἀποστείλαντι λόγον.
\par }{\PP \VS{13}Καὶ εἶπεν Δαυεὶδ πρὸς Γάδ Στενά μοι καὶ τὰ τρία σφόδρα· ἐμπεσοῦμαι δὴ εἰς χεῖρας Κυρίου, ὅτι πολλοὶ οἱ οἰκτειρμοὶ αὐτοῦ σφόδρα, καὶ εἰς χεῖρας ἀνθρώπων οὐ μὴ ἐμπέσω.
\par }{\PP \VS{14}Καὶ ἔδωκεν Κύριος θάνατον ἐν Ἰσραήλ, καὶ ἔπεσον ἐξ Ἰσραὴλ ἑβδομήκοντα χιλιάδες ἀνδρῶν.
\VS{15}καὶ ἀπέστειλεν Θεὸς ἄγγελο εἰς Ἰερουσαλὴμ τοῦ ἐξολοθρεῦσαι αὐτήν. καὶ ὡς ἐξολόθρευεν, εἶδε Κύριος, καὶ μετεμελήθη ἐπὶ τῇ κακίᾳ. καὶ εἶπε τῷ ἀγγέλῳ τῷ ἐξολοθρεύονντι, Ἱκανούσθω σοι, ἄνες τὴν χεῖρά σου· καὶ ὁ ἄγγελος Κυρίου ἑστὼς ἐν τῷ ἅλῳ Ὀρνὰ τοῦ Ἰεβουσαίου.
\VS{16}Καὶ ἐπῆρε Δαυὶδ τοὺς ὀφθαλμοὺς αὐτοῦ, καὶ εἶδε τὸν ἄγγελον Κυρίου ἑστῶτα ἀναμέσον τῆς γῆς καὶ τοῦ οὐρανοῦ, καὶ ἡ ῥομφαία αὐτοῦ ἐσπασμένη ἐν τῇ χειρὶ αὐτοῦ ἐκτεταμένη ἐπὶ Ἰερουσαλὴμ, καὶ ἔπεσε Δαυὶδ καὶ οἱ πρεσβύτεροι περιβεβλημένοι ἐν σάκκοις ἐπὶ πρόσωπον αὐτῶν.
\VS{17}Καὶ εἶπεν Δαὶδ πρὸς τὸν Θθεόν, Οὐκ ἐγὼ εἶπα τοῦ ἀριθμῆσαι τῷ λαῷ; καὶ ἐγώ εἰμι ὁ ἁμαρτών, κακοποιῶν ἐκακοποίησα, καὶ ταῦτα τὰ πρόβατα τί ἐποίησαν; Κύριε ὁ Θθεός, γενηθήτω ἡ χείρ σου ἐν ἐμοὶ καὶ ἐν τῷ οἴκῳ τοῦ πατρός μου, καὶ μὴ ἐν τῷ λαῷ σου εἰς ἀπώλειαν, Κύριε.
\par }{\PP \VS{18}Καὶ ἄγγελος Κυρίου εἶπε τῷ Γὰδ τοῦ εἰπεῖν πρὸς Δαυὶδ ἵνα ἀναβῇ τοῦ στῆσαι θυσιαστήριον κυρίῳ ἐν ἅλῳ Ὀρνὰ τοῦ Ἰεβουσαίου.
\VS{19}Καὶ ἀνέβη Δαυὶδ κατὰ τὸν λόγον Γὰδ, ὃν ἐλάλησεν ἐν ὀνόματι Κυρίου.
\VS{20}Καὶ ἐπέστρεψεν Ὀρνά, καὶ εἶδε τὸν βασιλέα καὶ τέσσαρας υἱοὺς αὐτοῦ μετʼ αὐτοῦ μεθʼ ἁχαβίν· καὶ Ὀρνὰ ἦν ἀλοῶν πυρούς.
\VS{21}Καὶ ἦλθεν Δαυὶδ πρὸς Ὀρνᾶν, καὶ Ὀρνὰ ἐξῆλθεν ἐκ τῆς ἅλω καὶ προσεκύνησεν τῷ Δαυὶδ τῷ προσώπῳ ἐπὶ τὴν γῆν.
\VS{22}Καὶ εἶπε Δαυὶδ πρὸς Ὀρνὰ, δός μοι τὸν τόπον σου τῆς ἅλω, καὶ οἰκοδομήσω ἐπʼ αὐτῷ θυσιαστήριον τῷ κυρίῳ· ἐν ἀργυρίῳ ἀξίῳ δός μοι αὐτόν, καὶ παύσεται ἡ πληγὴ ἐκ τοῦ λαοῦ.
\VS{23}Καὶ εἶπεν Ὀρνὰ πρὸς Δαυίδ, Λάβε σεαυτῷ, καὶ ποιησάτω ὁ κύριός μου ὁ βασιλεὺς τὸ ἀγαθὸν ἐναντίον εαὐτοῦ· ἴδε δέδωκα τοὺς μόσχους εἰς ὁλοκαύτωσιν, καὶ τὸ ἄροτρον εἰς ξύλα, καὶ τὸν σῖτον εἰς θυαίαν, τὰ πάντα δέδωκα.
\VS{24}Καὶ εἶπεν ὁ βασιλεὺς Δαυὶδ τῷ Ὀρνὰ, Οὐχὶ, ὅτι ἀγοράζων ἀγοράσω ἐν ἀργυρίῳ ἀξίῳ, ὅτι οὐ μὴ λάβω ἅ ἐστί σοι Κυρίῳ, τοῦ ἀνενέγκαι ὁλοκαύτωσιν δωρεὰν Κυρίῳ.
\VS{25}καὶ ἔδωκεν Δαυὶδ τῷ Ὀρνὰ ἐν τῷ τόπῳ αὐτοῦ σίκλους χρυσίου ὁλκῆς ἑξακοσίους.
\VS{26}Καὶ ᾠκοδόμησεν ἐκεῖ Δανδ θυσιαστήριον Κυρίῳ, καὶ ἀνήνεγκεν ὁλοκαυτώματα καὶ σωτηρίου· καὶ ἐβόησε πρὸς Κύριον, καὶ ἐπήκουσεν αὐτῷ ἐν πυρὶ ἐκ τοῦ οὐρανοῦ ἐπὶ τὸ θυσιαστήριον τῆς ὁλοκαυτώσεως καὶ κατανάλωσεν τὴν ὁλοκαύτωσιν.
\VS{27}Καὶ εἶπε Κύριος πρὸς τὸν ἄγγελον· καὶ κατέθηκε τὴν ῥομφαίαν εἰς τὸν κολεὸν αὐτῆς.
\par }{\PP \VS{28}Ἐν τῷ καιρῷ ἐκείνῳ ἐν τῷ ἰδεῖν τὸν Δαυὶδ ὅτι ἐπήκουσεν αὐτῷ Κύριος ἐν ἅλῳ Ὀρνὰ τοῦ Ἰεβουσαίου, καὶ ἐθυσίασεν ἐκεῖ.
\VS{29}Καὶ σκηνὴ Κυρίου ἣν ἐποίησε Μωυσῆς ἐν τῇ ἐρήμῳ, καὶ θυσιαστήριον τῶν ὁλοκαυτωμάτων ἐν τῷ καιρῷ ἐκείνῳ ἐν Βαμὰ ἐν Γαβαών.
\VS{30}Καὶ οὐκ ἐδύνατο Δαυὶδ τοῦ πορευθῆναι ἔμπροσθεν αὐτοῦ τοῦ ζητῆσαι τὸν Θεόν, ὅτι οὐ κατέσπευσεν ἀπὸ προσώπου τῆς ῥομφαίας ἀγγέλου Κυρίου.

\par }\Chap{22}{\PP \VerseOne{1}Καὶ εἶπε Δαυὶδ, Οὗτός ἐστιν ὁ οἶκος Κυρίου τοῦ Θεοῦ, καὶ τοῦτο τὸ θυσιαστήριον εἰς ὁλοκαύτωσιν τῷ Ἰσραήλ.
\par }{\PP \VS{2}Καὶ εἶπε Δαυὶδ συναγαγεῖν πάντας τοὺς προσηλύτους τοὺς ἐν γῇ Ἰσραὴλ, καὶ κατέστησε λατόμους λατομῆσαι λίθους ξυστοὺς τοῦ οἰκοδομῆσαι οἶκον τῷ Θεῷ.
\VS{3}Καὶ σίδηρον πολὺν εἰς τοὺς ἥλους τῶν θυρωμάτων καὶ τῶν πυλῶν, καὶ τοὺς στροφεῖς ἡτοίμασε Δαυὶδ καὶ χαλκὸν εἰς πλῆθος, οὐκ ἦν σταθμός.
\VS{4}Καὶ ξύλα κέδρινα, οὐκ ἦν ἀριθμός· ὅτι ἐφέροσαν οἱ Σιδώνιοι καὶ οἱ Τύριοι ξύλα κέδρινα εἰς πλῆθος τῷ Δαυίδ.
\VS{5}Καὶ εἶπε Δαυὶδ, Σαλωμὼν ὁ υἱός μου παιδάριον ἁπαλὸν, καὶ ὁ οἶκος τοῦ οἰκοδομῆσαι τῷ Κυρίῳ εἰς μεγαλωσύνην ἄνω, εἰς ὄνομα καὶ εἰς δόξαν εἰς πᾶσαν τὴν γῆν· ἑτοιμάσω αὐτῷ· καὶ ἡτοίμασε Δαυεὶδ εἰς πλῆθος ἔμπροσθεν τῆς τελευτῆς αὐτοῦ.
\par }{\PP \VS{6}Καὶ ἐκάλεσε Σαλωμὼν τὸν υἱὸν αὐτοῦ, καὶ ἐνετείλατο αὐτῷ τοῦ οἰκοδομῆσαι τὸν οἶκον τῷ Κυρίῳ Θεῷ Ἰσραήλ.
\VS{7}Καὶ εἶπε Δαυὶδ Σαλωμών, τέκνον, ἐμοὶ ἐγένετο ἐπὶ ψυχῇ τοῦ οἰκοδομῆσαι οἶκον τῷ ὀνόματι Κυρίου Θεοῦ.
\VS{8}Καὶ ἐγένετό μοι λόγος Κυρίου, λέγων, αἷμα εἰς πλῆθος ἐξέχεας, καὶ πολέμους μεγάλους ἐποίησας· οὐκ οἰκοδομήσεις οἶκον τῷ ὀνόματί μου, ὅτι αἵματα πολλὰ ἐξέχεας ἐπὶ τὴν γῆν ἐναντίον ἐμου.
\VS{9}Ἰδοὺ υἱὸς τίκτεταί σοι, οὗτος ἔσται ἀνὴρ ἀναπαύσεως, καὶ ἀναπαύσω αὐτὸν ἀπὸ πάντων τῶν ἐχθρῶν αὐτοῦ κυκλόθεν, ὅτι Σαλωμὼν ὄνομα αὐτῷ, καὶ εἰρήνην καὶ ἡσυχίαν δώσω ἐπὶ Ἰσραὴλ ἐν ταῖς ἡμέραις αὐτοῦ.
\VS{10}Οὗτος οἰκοδομήσει οἶκον τῷ ὀνόματί μου, καὶ οὗτος ἔσται μοι εἰς υἱὸν, κἀγὼ αὐτῷ εἰς πατέρα, καὶ ἀνορθώσω θρόνον βασιλείας αὐτοῦ ἐν Ἰσραὴλ ἕως αἰῶνος.
\VS{11}Καὶ νῦν, υἱέμου, ἔσται μετὰ σοῦ Κύριος, καὶ εὐοδώσει, καὶ οἰκοδομήσεις οἶκον τῷ Κυρίῳ Θεῷ σου. ὡς ἐλάλησε περὶ σοῦ.
\VS{12}Ἀλλʼ ἢ δῴη σοι σοφίαν καὶ σύνεσιν Κύριος καὶ κατισχύσαι σε ἐπὶ Ἰσραὴλ, καὶ τοῦ φυλάσσεσθαι καὶ τοῦ ποιεῖν τὸν νόμον Κυρίου τοῦ Θεοῦ σου.
\VS{13}Τότε εὐοδώσει ἐὰν φυλάξῃς τοῦ ποιεῖν τὰ προστάγματα καὶ τὰ κρίματα ἃ ἐνετείλατο Κύριος τῷ Μωυσῇ ἐπὶ Ἰσραήλ· ἀνδρίζου καὶ ἴσχυε, μὴ φοβοῦ μηδὲ πτοηθῇς.
\par }{\PP \VS{14}Καὶ ἰδοὺ ἐγὼ κατὰ τὴν πτωχείαν μου ἡτοίμασα εἰς οἶκον Κυρίου χρυσίου ταλάντων ἑκατὸν χιλιάδας, καὶ ἀργυρίου ταλάντων χιλίας χιλιάδας, καὶ χαλκὸν καὶ σίδηρον οὗ οὐκ ἔστι σταθμὸς, ὅτιι εἰς πλῆθός ἐστι· καὶ ξύλα καὶ λίθους ἡτοίμασα, καὶ πρὸς ταῦτα πρόσθες.
\VS{15}Καὶ μετὰ σοῦ πρόσθες εἰς πλῆθος ποιούντων ἔργα, τεχνῖται καὶ οἰκοδόμοι λίθων, καὶ τέκτονες ξύλων, καὶ πᾶς σοφὸς ἐν παντὶ ἔργῳ,
\VS{16}ἐν παντὶ ἔργῳ, ἐν χρυσίῳ καὶ ἀργυρίῳ, χαλκῷ καὶ ἐν σιδήρῳ, οὐκ ἔστιν ἀριθμός· ἀνάστηθι καὶ ποίει, καὶ Κύριος μετὰ σοῦ.
\par }{\PP \VS{17}Καὶ ἐνετείλατο Δαυὶδ τοῖς πᾶσιν ἄρχουσιν Ἰσραὴλ ἀντιλαβέσθαι τῷ Σαλωμὼν υἱῷ αὐτοῦ.
\VS{18}Οὐχὶ Κύριος μεθʼ ὑμῶν; καὶ ἀνέπαυσεν ὑμᾶς κυκλόθεν, ὅτι ἔδωκεν ἐν ὑμῶν τοὺς κατοικοῦντας τὴν γῆν, καὶ ὑπετάγη ἡ γῆ ἐναντίον Κυρίου καὶ ἐναντίον λαοῦ αὐτοῦ.
\VS{19}Νῦν δότε καρδίας ὑμῶν καὶ ψυχὰς ὑμῶν τοῦ ζητῆσαι τῷ κυρίῳ θεῷ ὑμῶν, καὶ ἐγέρθητε καὶ οἰκοδομήσατε ἁγίασμα τῷ Θεῷ ὑμῶν, τοῦ εἰσενέγκαι τὴν κιβωτὸν διαθήκης Κυρίου, καὶ σκεύη τὰ ἅγια τοῦ Θεοῦ εἰς οἶκον τὸν οἰκοδομούμενον τῷ ὀνόματι Κυρίου.

\par }\Chap{23}{\PP \VerseOne{1}Καὶ Δαυὶδ πρεσβύτης καὶ πλήρης ἡμερῶν, καὶ ἐβασίλευσε Σαλωμὼν τὸν υἱὸν αὐτοῦ ἀνθʼ αὐτοῦ ἐπὶ Ἰσραήλ.
\VS{2}καὶ συνήγαγε τοὺς πάντας ἄρχοντας Ἰσραὴλ καὶ τοὺς ἱερεῖς καὶ τοὺς Λευείτας.
\VS{3}Καὶ ἠρίθμησαν οἱ Λευῖται ἀπὸ τριακονταετοῦς καὶ ἐπάνω, καὶ ἐγένετο ὁ ἀριθμὸς αὐτῶν κατὰ κεφαλὴν αὐτῶν εἰς ἄνδρας τριάκοντα καὶ ὀκτὼ χιλιάδας.
\VS{4}Ἀπὸ τῶν ἐργοδιωκτῶν ἐπὶ τὰ ἔργα οἴκου Κυρίου εἴκοσι τέσσαρες χιλιάδες, καὶ γραμματεῖς καὶ κριταὶ ἑξακισχίλιοι,
\VS{5}καὶ τέσσαρες χιλιάδες πυλωροὶ καὶ τώσσαρες χιλιάδες αἰνοῦντες τῷ κυρίῳ ἐν ὀργάνοιν οἷς ἐποίησε τοῦ αἰνεῖν τῷ κυρίῳ.
\par }{\PP \VS{6}καὶ διεῖλεν αὐτοὺς Δαυὶδ ἐφημερίας τοῖς υἱοῖς Λευῖ, τῷ Γεδσών, Καάθ, καὶ Μαραρί·
\VS{7}καὶ τῷ Γεδσὼν, Ἐδὰν, καὶ Σεμεΐ.
\VS{8}γἱοὶ τῷ Ἐδάν, ἄρχων Ἰεϊὴλ, καὶ Ζηθὸμ, καὶ Ἰωήλ, καὶ τρεῖς.
\VS{9}γἱοὶ Σεμεῒ, Σαλωμὶθ, Ἰεϊὴλ, καὶ Δὰν, τρεῖς· οὗτοι ἄρχοντες πατριῶν τῶν Ἐδάν·
\VS{10}καὶ τοῖς υἱοῖς Σεμεΐ· Ἰὲθ, καὶ Ζιζὰ, καὶ Ἰῶας, καὶ Βεριά· οὗτοι υἱοὶ Σεμεΐ, τέσσαρες.
\VS{11}καὶ ἦν Ἰὲθ ὁ ἄρχων· καὶ Ζιζὰ ὁ δεύτερος· καὶ Ἰωὰς καὶ Βεριὰ οὐκ ἐπλήθυναν υἱοὺς, καὶ ἐγένετο εἰς οἶκον πατριᾶς εἰς ἐπίσκεψιν μίαν.
\par }{\PP \VS{12}γἱοὶ Καάθ, Ἀμβρὰμ, Ἰσαάρ, Χεβρών, Ὀζιὴλ, τέσσαρες.
\VS{13}γἱοὶ Ἀμβρὰμ, Ἀαρὼν καὶ Μωυσῆς· καὶ διεστάλη Ἀαρὼν τοῦ ἁγιασθῆναι ἅγια ἁγίων, αὐτὸς καὶ οἱ υἱοὶ αὐτοῦ ἕως αἰῶνος, τοῦ θυμιᾷν ἐναντίον τοῦ κυρίου, λειτουργεῖν καὶ ἐπεύχεσθαι ἐπὶ τῷ ὀνόματι αὐτοῦ ἕως αἰῶνος.
\VS{14}Καὶ Μωυσῆς ἄνθρωπος τοῦ Θεοῦ, υἱοὶ αὐτοῦ ἐκλήθησαν εἰς φυλὴν τοῦ Λευεί.
\VS{15}γἱοὶ Μωυσῆ, Γηρσὰμ, καὶ Ἐλιέζερ.
\VS{16}γἱοὶ Γηρσὰμ, Σουβαὴλ ὁ ἄρχων.
\VS{17}Καὶ ἦσαν υἱοὶ τῷ Ἐλιέζερ, Ῥαβιὰ ὁ ἄρχων· καὶ οὐκ ἦσαν τῷ Ἐλιέζερ υἱοὶ ἕτεροι· καὶ υἱοὶ Ῥαβιὰ ηὐξήθησαν εἰς ὕψος.
\VS{18}γἱοὶ Ἰσαάρ, Σαλωμὼθ ὁ ἄρχων.
\VS{19}γἱοὶ Χεβρὼν, Ἰεριὰ ὁ ἄρχων, Ἀμαριὰ ὁ δεύτερος, Ἱεζιὴλ ὁ τρίτος, Ἰκεμίας ὁ τέταρτος.
\VS{20}γἱοὶ Ὀζιὴλ, Μιχὰ ὁ ἄρχων, καὶ Ἰσιὰ ὁ δεύτερος.
\par }{\PP \VS{21}γἱοὶ Μεραρί, Μοολὶ καὶ Ὁ Μουσί· υἱοὶ Μοολὶ, Ἐλεαζαρ, καὶ Κείς. καὶ Κίς.
\VS{22}Καὶ ἀπέθανεν Ἐλεάζαρ· καὶ οὐκ ἦσαν αὐτῷ υἱοὶ, ἀλλʼ ἢ θυγατέρες· καὶ ἔλαβον αὐτὰς υἱοὶ Κὶς ἀδελφοὶ αὐτῶν.
\VS{23}Υἱοὶ Μουσὶ, Μοολὶ, καὶ Ἐδὲρ, καὶ Ἰαριμὼθ, τρεῖς.
\par }{\PP \VS{24}Οὗτοι υἱοὶ Λευὶ κατʼ οἴκους πατριῶν αὐτῶν, ἄρχοντες τῶν πατριῶν αὐτῶν κατὰ τὴν ἐπίσκεψιν αὐτῶν, κατὰ τὸν ἀριθμὸν ὀνομάτων αὐτῶν, κατὰ κεφαλὴν αὐτῶν, ποιοῦντες τὰ ἔργα λειτουργείας οἴκου Κυρίου ἀπὸ εἰκοσαετοῦς καὶ ἐπάνω.
\VS{25}Ὅτι εἶπε Δαυὶδ, κατέπαυσε Κύριος ὁ Θεὸς Ἰσραὴλ τῷ λαῷ αὐτοῦ, καὶ κατεσκήνωσεν ἐν Ἱερουσαλὴμ ἕως αἰῶνος.
\VS{26}Καὶ οἱ Λευῖται οὐκ ἦσαν αἴροντες τὴν σκηνὴν καὶ τὰ πάντα σκεύη αὐτῆς εἰς τὴν λειτουργείαν αὐτῆς·
\VS{27}Ὅτι ἐν τοῖς λόγοις Δαυὶδ τοῖς ἐσχάτοις ἐστὶν ὁ ἀριθμὸς υἱῶν Λευὶ ἀπὸ εἰκοσαετοῦς καὶ ἐπάνω·
\VS{28}Ὅτι ἔστησεν αὐτοὺς ἐπὶ χειρὶ Ἀαρὼν, τοῦ λειτουργεῖν ἐν οἴκῳ Κυρίου ἐπὶ τὰς αὐλὰς, καὶ ἐπὶ τὰ παστοφόρια, καὶ ἐπὶ τὸν καθαρισμὸν τῶν πάντων ἁγίων, καὶ ἐπὶ τὰ ἔργα λειτουργείας οἴκου τοῦ Θεοῦ,
\VS{29}καὶ εἰς τοὺς ἄρτους τῆς προθέσεως, καὶ εἰς τὴν σεμίδαλιν τῆς θυσίας, καὶ εἰς τὰ λάγανα τὰ ἄζυμα, καὶ εἰς τήγανον, καὶ εἰς τὴν πεφυραμένην, καὶ εἰς πᾶν μέτρον,
\VS{30}καὶ τοῦ στῆναι πρωῒ τοῦ αἰνεῖν καὶ ἐξομολογεῖσθαι τῷ Κυρίῳ, καὶ οὕτω τοεσπέρας·
\VS{31}Καὶ ἐπὶ πάντων τῶν ἀναφερομένων ὁλοκαυτωμάτων τῷ Κυρίῳ ἐν τοῖς σαββάτοις καὶ ἐν ταῖς νεομηνίαις καὶ ἐν ταῖς ἑορταῖς, κατὰ ἀριθμὸν, κατὰ τὴν κρίσιν ἐπʼ αὐτοῖς διαπαντὸς τῷ Κυρίῳ.
\VS{32}Καὶ φυλάξουσι τὰς φυλακὰς σκηνῆς τοῦ μαρτυρίου, καὶ τὴν φυλακὴν τοῦ ἁγίου, καὶ τὰς φυλακὰς υἱῶν Ἀαρὼν ἀδελφῶν αὐτῶν, τοῦ λειτουργεῖν ἐν οἴκῳ Κυρίου.

\par }\Chap{24}{\PP \VerseOne{1}Καὶ τοὺς υἱοὺς Ἀαρὼν διαιρέσει Ναδὰβ, καὶ Ἀβιοὺδ, καὶ Ἐλεάζαρ, καὶ Ἰθάμαρ.
\VS{2}Καὶ ἀπέθανε Ναδὰβ καὶ Ἀβιοὺδ ἐναντίον τοῦ πατρὸς αὐτῶν, καὶ υἱοὶ οὐκ ἦσαν αὐτοῖς· καὶ ἱεράτευσεν Ἐλεάζαρ καὶ Ἰθάμαρ υἱοὶ Ἀαρών·
\VS{3}Καὶ διεῖλεν αὐτοὺς Δαυὶδ, καὶ Σαδὼκ ἐκ τῶν υἱῶν Ἐλεάζαρ, καὶ Ἀχιμέλεχ ἐκ τῶν υἱῶν Ἰθάμαρ, κατὰ τὴν ἐπίσκεψιν αὐτῶν, κατὰ τὴν λειτουργείαν αὐτῶν, κατʼ οἴκους πατριῶν αὐτῶν.
\par }{\PP \VS{4}Καὶ εὑρέθησαν οἱ υἱοὶ Ἐλεάζαρ πλείους εἰς ἄρχοντας τῶν δυνατῶν παρὰ τοὺς υἱοὺς Ἰθάμαρ· καὶ διεῖλεν αὐτοὺς, τοῖς υἱοῖς Ἐλεάζαρ ἄρχοντας εἰς οἴκους πατριῶν ἑκκαίδεκα, τοῖς υἱοῖς Ἰθάμαρ κατʼ οἴκους πατριῶν ὀκτώ.
\VS{5}Καὶ διεῖλεν αὐτοὺς κατὰ κλήρους τούτους πρὸς τούτους, ὅτι ἦσαν ἄρχοντες τῶν ἁγίων, καὶ ἄρχοντες Κυρίου ἐν τοῖς υἱοῖς Ἐλεάζαρ καὶ ἐν τοῖς υἱοῖς Ἰθάμαρ.
\par }{\PP \VS{6}Καὶ ἔγραψεν αὐτοὺς Σαμαΐας υἱὸς Ναθαναὴλ ὁ γραμματεὺς ἐκ τοῦ Λευὶ κατέναντι τοῦ βασιλέως καὶ τῶν ἀρχόντων, καὶ Σαδὼκ ὁ ἱερεὺς, καὶ Ἀχιμέλεχ υἱὸς Ἀβιάθαρ, καὶ ἄρχοντες τῶν πατριῶν τῶν ἱερέων καὶ τῶν Λευιτῶν οἴκου πατριᾶς, εἷς εἷς τῷ Ἐλεάζαρ, καὶ εἷς εἷς τῷ Ἰθάμαρ.
\par }{\PP \VS{7}Καὶ ἐξῆλθεν ὁ κλῆρος ὁ πρῶτος τῷ Ἰωαρὶμ, τῷ Ἰεδίᾳ ὁ δεύτερος,
\VS{8}τῷ Χαρὶβ ὁ τρίτος, τῷ Σεωρὶμ ὁ τέταρτος,
\VS{9}τῷ Μελχίᾳ ὁ πέμπτος, τῷ Μεϊαμὶν ὁ ἕκτος,
\VS{10}τῷ Κὼς ὁ ἕβδομος, τῷ Ἀβίᾳ ὁ ὄγδοος,
\VS{11}τῷ Ἰησοῦ ὁ ἔννατος, τῷ Σεχενίᾳ ὁ δέκατος,
\VS{12}τῷ Ἐλιαβὶ ὁ ἑνδέκατος, τῷ Ἰακὶμ ὁ δωδέκατος,
\VS{13}τῷ Ὀπφᾷ ὁ τρισκαιδέκατος, τῷ Ἰεσβαὰλ ὁ τεσσαρεσκαιδέκατος, τῷ Βελγὰ ὁ πεντεκαιδέκατος,
\VS{14}τῷ Ἐμμὴρ ὁ ἑκκαιδέκατος, τῷ Χηζὶν ὁ ἑπτακαιδέκατος,
\VS{15}τῷ Ἀφεσὴ ὁ ὀκτωκαιδέκατος, τῷ Φεταίᾳ ὁ ἐννεακαιδέκατος,
\VS{16}τῷ Ἐζεκὴλ ὁ εἰκοστὸς, τῷ Ἀχὶμ ὁ εἷς καὶ εἰκοστὸς,
\VS{17}τῷ Γαμοὺλ ὁ δεύτερος καὶ εἰκοστὸς, τῷ Ἀδαλλαὶ ὁ τρίτος καὶ εἰκοστὸς,
\VS{18}τῷ Μαασαὶ ὁ τέταρτος καὶ εἰκοστός.
\par }{\PP \VS{19}Αὕτη ἡ ἐπίσκεψις αὐτῶν κατὰ τὴν λειτουργίαν αὐτῶν τοῦ εἰσπορεύεσθαι εἰς οἶκον Κυρίου κατὰ τὴν κρίσιν αὐτῶν διὰ χειρὸς Ἀαρὼν πατρὸς αὐτῶν, ὡς ἐνετείλατο Κύριος ὁ Θεὸς Ἰσραήλ.
\par }{\PP \VS{20}Καὶ τοῖς υἱοῖς Λευὶ τοῖς καταλοίποις, τοῖς υἱοῖς Ἀμβρὰμ, Σωβαήλ· τοῖς υἱοῖς Σωβαὴλ Ἰεδία.
\VS{21}Τῷ Ῥααβία ὁ ἄρχων.
\VS{22}Καὶ τῷ Ἰσααρὶ, Σαλωμώθ· τοῖς υἱοῖς Σαλωμὼθ, Ἰάθ.
\VS{23}υἱοὶ Ἐκδιοῦ, Ἀμαδία ὁ δεύτερος, Ἰαζιὴλ ὁ τρίτος, Ἰεκμοὰμ ὁ τέταρτος.
\VS{24}Τοῖς υἱοῖς Ὀζιὴλ, Μιχά· υἱοὶ Μιχὰ, Σαμήρ·
\VS{25}Ἀδελφὸς Μιχὰ, Ἰσία· υἱὸς Ἰσία, Ζαχαρία.
\VS{26}Υἱοὶ Μεραρὶ, Μοολὶ καὶ ὁ Μουσί·
\VS{27}υἱοὶ Ὀζία τοῦ Μεραρὶ τῷ Ὀζίᾳ· υἱοὶ αὐτοῦ Ἰσοὰμ, καὶ Σακχοὺρ, καὶ Ἀβαΐ.
\VS{28}Τῷ Μοολὶ Ἐλεάζαρ, καὶ Ἰθάμαρ· καὶ ἀπέθανεν Ἐλεάζαρ καὶ οὐκ ἦσαν αὐτῷ υἱοί.
\VS{29}Τῷ Κὶς, υἱοὶ τοῦ Κὶς Ἱεραμεήλ.
\VS{30}Καὶ υἱοὶ τοῦ Μουσὶ, Μοολὶ, καὶ Ἐδὲρ, καὶ Ἰεριμώθ· οὗτοι υἱοὶ τῶν Λευιτῶν κατʼ οἴκους πατριῶν αὐτῶν.
\VS{31}Καὶ ἔλαβον καὶ αὐτοὶ κλήρους καθὼς οἱ ἀδελφοὶ αὐτῶν υἱοὶ Ἀαρὼν ἐναντίον τοῦ βασιλέως, καὶ Σαδὼκ, καὶ Ἀχιμέλεχ, καὶ οἱ ἄρχοντες τῶν πατριῶν τῶν ἱερέων καὶ τῶν Λευειτῶν πατριάρχαι Ἀραὰβ, καθὼς οἱ ἀδελφοὶ αὐτοῦ οἱ νεώτεροι.

\par }\Chap{25}{\PP \VerseOne{1}Καὶ ἔστησε Δαυεὶδ ὁ βασιλεὺς καὶ οἱ ἄρχοντες τῆς δυνάμεως εἰς τὰ ἔργα τοὺς υἱοὺς Ἀσὰφ, καὶ Αἰμὰν, καὶ Ἰδιθοὺν, τοὺς ἀποφεγγομένους ἐν κινύραις, καὶ ἐν νάβλαις, καὶ ἐν κυμβάλοις· καὶ ἐγένετο ὁ ἀριθμὸς αὐτῶν κατὰ κεφαλὴν αὐτῶν ἐργαζομένων ἐν τοῖς ἔργοις αὐτῶν.
\par }{\PP \VS{2}Υἱοὶ Ἀσάφ, Σακχοὺρ, Ἰωσὴφ, καὶ Ναθαυίας, καὶ Ἐραήλ· υἱοὶ Ἀσὰφ ἐχόμενοι τοῦ βασιλέως.
\par }{\PP \VS{3}Τῷ Ἰδιθοὺν, υἱοὶ Ἰδιθοὺν, Τοδολίας, καὶ Σουρὶ, καὶ Ἰσέας, καὶ Ἀσαβίας, καὶ Ματθαθίας, ἕξ μετὰ τὸν πατέρα αὐτῶν Ἰδιθὺν, ἐν κινύρᾳ ἀνακρουόμενοι ἐξομολόγησιν καὶ αἴνεσιν τῷ κυρίῳ.
\par }{\PP \VS{4}Τῷ Αἰμὰν, υἱοὶ Αἰμὰν, Βουκίας, καὶ Ματθανίας, καὶ Ὀζιὴλ, καὶ Σουβαὴλ, καὶ Ἰεριμὼθ, καὶ Ἀνανίας, καὶ Ἀνὰν, καὶ Ἑλιαθὰ, καὶ Γοδολλαθὶ, καὶ Ῥωμετθιέζερ, καὶ Ἰεσβασακὰ, καὶ Μαλλιθὶ, καὶ Ὠθηρὶ, καὶ Μεαζώθ.
\VS{5}Πάντες οὗτοι υἱοὶ τῷ Αἰμὰν τῷ ἀνακρουομένῳ τῷ βασιλεῖ ἐν λόγοις Θεοῦ, ὑψῶσαι κέρας· καὶ ἔδωκεν ὁ Θεὸς τῷ Αἰμὰν υἱοὺς τεσσαρεσκαίδεκα, καὶ θυγατέρας τρεῖς.
\VS{6}Πάντες οὗτοι μετὰ τοῦ πατρὸς αὐτῶν ὑμνῳδοῦντες ἐν οἴκῳ Θεοῦ, ἐν κυμβάλοις, καὶ ἐν νάβλαις, καὶ ἐν κινύραις εἰς τὴν δουλείαν οἴκου τοῦ Θεοῦ, ἐχόμενα τοῦ βασιλέως, καὶ Ἀσὰφ, καὶ Ἰδιθοὺν, καὶ Αἱμάν.
\par }{\PP \VS{7}Καὶ ἐγένετο ὁ ἀριθμὸς αὐτῶν μετὰ τοὺς ἀδελφοὺς αὐτῶν δεδιδαγμένοι ᾄδειν Κυρίῳ πᾶς συνιὼν, διακόσιοι ὀγδοήκοντα καὶ ὀκτώ.
\par }{\PP \VS{8}Καὶ ἔβαλον καὶ αὐτοὶ κλήρους ἐφημεριῶν κατὰ τὸν μικρὸν καὶ κατὰ τὸν μέγαν τελείων καὶ μανθανόντων.
\VS{9}Καὶ ἐξῆλθεν ὁ κλῆρος ὁ πρῶτος υἱῶν αὑτοῦ καὶ ἀδελφῶν αὐτοῦ τῷ Ἀσὰφ τοῦ Ἰωσὴφ, Γοδολίας· ὁ δεύτερος Ἡνεία, υἱοὶ αὐτοῦ καὶ ἀδελφοὶ αὐτοῦ δεκαδύο·
\VS{10}Ὁ τρίτος Ζακχοὺρ, υἱοὶ αὐτοῦ καὶ ἀδελφοὶ αὐτοῦ δεκαδύο·
\VS{11}Ὁ τέταρτος Ἰεσρὶ, υἱοὶ αὐτοῦ καὶ ἀδελφοὶ αὐτοῦ δεκαδύο·
\VS{12}Ὁ πέμπτος Νάθαν, υἱοὶ αὐτοῦ καὶ ἀδελφοὶ αὐτοῦ δεκαδύο·
\VS{13}Ὁ ἕκτος Βουκίας, υἱοὶ αὐτοῦ καὶ ἀδελφοὶ αὐτοῦ δεκαδύο·
\VS{14}Ὁ ἕβδομος Ἰσεριὴλ, υἱοὶ αὐτοῦ καὶ ἀδελφοὶ αὐτοῦ δεκαδύο·
\VS{15}Ὁ ὄγδοος Ἰωσία, υἱοὶ αὐτοῦ καὶ ἀδελφοὶ αὐτοῦ δεκαδύο·
\VS{16}Ὁ ἔννατος Ματθανίας, υἱοὶ αὐτοῦ καὶ ἀδελφοὶ αὐτοῦ δεκαδύο·
\VS{17}Ὁ δέκατος Σεμεΐα, υἱοὶ αὐτοῦ καὶ ἀδελφοὶ αὐτοῦ δεκαδύο·
\VS{18}Ὁ ἑνδέκατος Ἀσριὴλ, υἱοὶ αὐτοῦ καὶ ἀδελφοὶ αὐτοῦ δεκαδύο·
\VS{19}Ὁ δωδέκατος Ἀσαβία, υἱοὶ αὐτοῦ καὶ ἀδελφοὶ αὐτοῦ δεκαδύο·
\VS{20}Ὁ τρισκαιδέκατος Σουβαὴλ, υἱοὶ αὐτοῦ καὶ ἀδελφοὶ αὐτοῦ δεκαδύο·
\VS{21}Ὁ τεσσαρεσκαιδέκατος Ματθαθίας, υἱοὶ αὐτοῦ καὶ ἀδελφοὶ αὐτοῦ δεκαδύο·
\VS{22}Ὁ πεντεκαιδέκατος Ἰεριμὼθ, υἱοὶ αὐτοῦ καὶ ἀδελφοὶ αὐτοῦ δεκαδύο·
\VS{23}Ὁ ἑκκαιδέκατος Ἁνανίας, υἱοὶ αὐτοῦ καὶ ἀδελφοὶ αὐτοῦ δεκαδύο·
\VS{24}Ὁ ἑπτακαιδέκατος Ἰεσβασακὰ, υἱοὶ αὐτοῦ καὶ ἀδελφοὶ αὐτοῦ δεκαδύο·
\VS{25}Ὁ ὀκτωκαιδέκατος Ἀνανίας, υἱοὶ αὐτοῦ καὶ ἀδελφοὶ αὐτοῦ δεκαδύο·
\VS{26}Ὁ ἐννεακαιδέκατος Μαλλιθὶ, υἱοὶ αὐτοῦ καὶ ἀδελφοὶ αὐτοῦ δεκαδύο·
\VS{27}Ὁ εἰκοστὸς Ἑλιαθὰ, υἱοὶ αὐτοῦ καὶ ἀδελφοὶ αὐτοῦ δεκαδύο·
\VS{28}Ὁ εἰκοστὸς πρῶτος Ὠθηρὶ, υἱοὶ αὐτοῦ καὶ ἀδελφοὶ αὐτοῦ δεκαδύο·
\VS{29}Ὁ εἰκοστὸς δεύτερος Γοδολλαθὶ, υἱοὶ αὐτοῦ καὶ ἀδελφοὶ αὐτοῦ δεκαδύο·
\VS{30}Ὁ εἰκοστὸς τρίτος Μεαζὼθ, υἱοὶ αὐτοῦ καὶ ἀδελφοὶ αὐτοῦ δεκαδύο·
\VS{31}Ὁ εἰκοστὸς τέταρτος Ῥωμετθιέζερ, υἱοὶ αὐτοῦ καὶ ἀδελφοὶ αὐτοῦ δεκαδύο·

\par }\Chap{26}{\PP \VerseOne{1}Καὶ εἰς διαιρέσεις τῶν πυλῶν, υἱοὶ Κορεῒμ Μοσελλεμία ἐκ τῶν υἱῶν Ἀσάφ.
\VS{2}Καὶ τῷ Μοσελλαμίᾳ υἱὸς Ζαχαρίας ὁ πρωτότοκος, Ἰαδιὴλ ὁ δεύτερος, Ζαβαδία ὁ τρίτος, Ἰενουὴλ ὁ τέταρτος,
\VS{3}Ἰωλὰμ ὁ πέμπτος, Ἰωνάθαν ὁ ἕκτος, Ἐλιωναῒ ὁ ἕβδομος. Ἀβδεδὸμ ὁ ὄγδοος.
\VS{4}Καὶ τῷ Ἀβδεδὸμ υἱοὶ, Σαμαίας ὁ πρωτότοκος, Ἰωζαβὰθ ὁ δεύτερος, Ἰωὰθ ὁ τρίος, Σαχὰρ ὁ τέταρτος, Ναθαναὴλ ὁ πέμπτος,
\VS{5}Ἀμιὴλ ὁ ἕκτος, Ἰσσάχαρ ὁ ἕβδομος, Φελαθὶ ὁ ὄγδοος, ὅτι εὐλόγησεν αὐτὸν ὁ Θεός.
\VS{6}Καὶ τῷ Σαμαίᾳ υἱῶ αὑτοῦ ἐτέχθησαν υἱοὶ τοῦ πρωτοτόκου Ῥωσαὶ εἰς τὸν οἶκον τὸν πατρικὸν αὐτοῦ, ὅτι δυνατοὶ ἦσαν.
\VS{7}Υἱοὶ Σαμαῒ, Ὀθνὶ καὶ Ῥαφαὴλ, καὶ Ὠβὴδ, καὶ Ἐλζαβὰθ, καὶ Ἀχιοὺδ, υἱοὶ δυνατοὶ, Ἑλιοῦ, καὶ Σαβαχία, καὶ Ἰσβακώμ.
\VS{8}Πάντες ἀπὸ τῶν υἱῶν Ἀβδεδὸμ, αὐτοὶ καὶ οἱ υἱοὶ αὐτῶν καὶ οἱ ἀδελφοὶ αὐτῶν ποιοῦντες δυνατῶς ἐν τῇ ἐργασίᾳ, οἱ πάντες ἑξηκονταδύο τῷ Ἀβδεδόμ.
\par }{\PP \VS{9}Καὶ τῷ Μοσελλεμίᾳ υἱοὶ καὶ ἀδελφοὶ δεκακαιοκτὼ δυνατοί.
\VS{10}Καὶ τῷ Ὀσᾷ τῶν υἱῶν Μεραρὶ υἱοὶ φυλάσσοντες τὴν ἀρχὴν, ὅτι οὐκ ἦν πρωτότοκος· καὶ ἐποίησεν αὐτὸν ὁ πατὴρ αὐτοῦ ἄρχοντα τῆς διαιρέσεως τῆς δευτέρας.
\VS{11}Χελκίας ὁ δεύτερος, Ταβλαὶ ὁ τρίτος, Ζαχαρίας ὁ τέταρτος· πάντες οὗτοι υἱοὶ καὶ ἀδελφοὶ τῷ Ὀσᾷ τρισκαίδεκα.
\par }{\PP \VS{12}Τούτοις αἱ διαιρέσεις τῶν πυλῶν τοῖς ἄρχουσι τῶν δυνατῶν ἐφημερίαι, καθὼς οἱ ἀδελφοὶ αὐτῶν λειτουργεῖν ἐν οἴκῳ Κυρίου.
\VS{13}Καὶ ἔβαλον κλήρους κατὰ τὸν μικρὸν καὶ κατὰ τὸν μέγαν κατʼ οἴκους πατριῶν αὐτῶν εἰς πυλῶνα καὶ πυλῶνα.
\VS{14}Καὶ ἔπεσεν ὁ κλῆρος τῶν πρὸς ἀνατολὰς τῷ Σελεμίᾳ, καὶ Ζαχαρίᾳ· υἱοὶ Σωὰζ τῷ Μελχίᾳ ἔβαλον κλήρους, καὶ ἐξῆλθεν ὁ κλῆρος Βοῥῥᾶ.
\VS{15}Τῷ Ἀβδεδὸμ Νότον κατέναντι οἴκου Ἐσεφίμ.
\VS{16}Εἰς δεύτερον τῷ Ὀσᾷ πρὸς δυσμαῖς μετὰ τὴν πύλην παστοφορίου τῆς ἀναβάσεως· φυλακὴ κατέναντι φυλακῆς.
\VS{17}Πρὸς ἀνατολὰς ἓξ τὴν ἡμέραν· Βοῤῥᾶ τῆς ἡμέρας τέσσαρες· Νότον τῆς ἡμέρας τέσσαρες· καὶ εἰς τὸν Ἐσεφὶμ δύο
\VS{18}εἰς διαδεχομένους· καὶ τῷ Ὀσᾷ πρὸς δυσμαῖς μετὰ τὴν πύλην τοῦ παστοφορίου τρεῖς· φυλακὴ κατέναντι φυλακῆς τῆς ἀναβάσεως πρὸς ἀνατολὰς τῆς ἡμέρας ἓξ, καὶ τῷ Βοῤῥᾷ τέσσαρες, καὶ τῷ Νότῳ τέσσαρες, καὶ Ἐσεφὶμ δύο εἰς διαδεχομένους, καὶ πρὸς δυσμαῖς τέσσαρες, καὶ εἰς τὸν τρίβον δύο διαδεχομένους.
\VS{19}Αὗται αἱ διαιρέσεις τῶν πυλωρῶν τοῖς υἱοῖς τοῦ Κορὲ, καὶ τοῖς υἱοῖς Μεραρί.
\par }{\PP \VS{20}Καὶ οἱ Λευῖται ἀδελφοὶ αὐτῶν ἐπὶ τῶν θησαυρῶν οἴκου Κυρίου, καὶ ἐπὶ τῶν θησαυρῶν τῶν καθηγιασμένων.
\VS{21}Υἱοὶ Λαδὰν οὗτοι, υἱοὶ τῷ Γηρσωνί· τῷ Λαδὰν ἄρχοντες πατριῶν, τῷ Λαδὰν, τῷ Γηρσωνὶ Ἰεϊήλ.
\VS{22}Υἱοὶ Ἰεϊὴλ Ζεθὸμ καὶ Ἰωὴλ, οἱ ἀδελφοὶ ἐπὶ τῶν θησαυρῶν οἴκου Κυρίου.
\VS{23}Τῷ Ἀμβρὰμ καὶ Ἰσσαὰρ, Χεβρὼν, καὶ Ὀζιήλ.
\VS{24}Καὶ Σουβαὴλ ὁ τοῦ Γηρσὰμ τοῦ Μωυσῆ ἐπὶ τῶν θησαυρῶν.
\VS{25}Καὶ τῷ ἀδελφῷ αὐτοῦ Ἐλιέζερ Ῥαβίας υἱὸς, καὶ Ἰωσίας, καὶ Ἰωρὰμ, καὶ Ζεχρὶ, καὶ Σαλωμώθ.
\VS{26}Αὐτὸς Σαλωμὼθ καὶ οἱ ἀδελφοὶ αὐτοῦ ἐπὶ πάντων τῶν θησαυρῶν τῶν ἁγίων, οὓς ἡγίασε Δαυὶδ ὁ βασιλεὺς καὶ οἱ ἄρχοντες τῶν πατριῶν, χιλίαρχοι καὶ ἑκατόνταρχοι καὶ ἀρχηγοὶ τῆς δυνάμεως,
\VS{27}ἃ ἔλαβεν ἐκ πόλεων καὶ ἐκ τῶν λαφύρων, καὶ ἡγίασεν ἀπʼ αὐτῶν τοῦ μὴ καθυστερῆσαι τὴν οἰκοδομὴν τοῦ οἴκου τοῦ Θεοῦ·
\VS{28}καὶ ἐπὶ πάντων τῶν ἁγίων τοῦ Θεοῦ Σαμουὴλ τοῦ προφήτου, καὶ Σαοὺλ τοῦ Κὶς, καὶ Ἀβεννὴρ τοῦ Νὴρ, καὶ Ἰωὰβ τοῦ Σαρουία, πᾶν ὃ ἡγίασαν διὰ χειρὸς Σαλωμὼθ καὶ τῶν ἀδελφῶν αὐτοῦ.
\par }{\PP \VS{29}Τῷ Ἰσσααρὶ Χωνενία, καὶ υἱοὶ τῆς ἐργασίας τῆς ἔξω ἐπὶ τὸν Ἰσραὴλ τοῦ γραμματεύειν καὶ διακρίνειν.
\VS{30}Τῷ Χεβρωνὶ Ἀσαβίας καὶ οἱ ἀδελφοὶ αὐτοῦ υἱοὶ δυνατοὶ χίλιοι καὶ ἐπτακόσιοι ἐπὶ τῆς ἐπισκέψεως τοῦ Ἰσραὴλ πέραν τοῦ Ἰορδάνου πρὸς δυσμαῖς, εἰς πᾶσαν λειτουργίαν Κυρίου καὶ ἐργασίαν τοῦ βασιλέως.
\VS{31}Τοῦ Χεβρωνὶ Οὐρίας ὁ ἄρχων τῶν Χεβρωνὶ κατὰ γενέσεις αὐτῶν, κατὰ πατριὰς, ἐν τῷ τεσσαρακοστῷ ἔτει τῆς βασιλείας αὐτοῦ ἐπεσκέπησαν, καὶ εὑρέθη ἀνὴρ δυνατὸς ἐν αὐτοῖς ἐν Ἰαζὴρ τῆς Γαλααδίτιδος·
\VS{32}Καὶ οἱ ἀδελφοὶ αὐτοῦ υἱοὶ δυνατοὶ δισχίλιοι ἑπτακόσιοι οἱ ἄρχοντες τῶν πατριῶν, καὶ κατέστησεν αὐτοὺς Δαυὶδ ὁ βασιλεὺς ἐπὶ τοῦ Ῥουβηνὶ, καὶ Γαδδὶ, καὶ ἡμίσους φυλῆς Μανασσῆ εἰς πᾶν πρόσταγμα Κυρίου καὶ λόγον βασιλέως.

\par }\Chap{27}{\PP \VerseOne{1}Καὶ υἱοὶ Ἰσραὴλ κατὰ ἀριθμὸν αὐτῶν ἄρχοντες τῶν πατριῶν, χιλίαρχοι καὶ ἑκατόνταρχοι, καὶ γραμματεῖς οἱ λειτουργοῦντες τῷ βασιλεῖ καὶ εἰς πᾶν λόγον τοῦ βασιλέως κατὰ διαιρέσεις, πᾶν λόγον τοῦ εἰσπορευομένου καὶ ἐκπορευομένου μῆνα ἐκ μηνὸς, εἰς πάντας τοὺς μῆνας τοῦ ἐνιαυτοῦ, διαίρεσις μία εἴκοσι καὶ τέσσαρες χιλιάδες.
\par }{\PP \VS{2}Καὶ ἐπὶ τῆς διαιρέσεως τῆς πρώτης τοῦ μηνὸς τοῦ πρώτου, Ἰσβοὰζ ὁ τοῦ Ζαβδιὴλ, ἐπὶ τῆς διαιρέσεως αὐτοῦ εἴκοσι καὶ τέσσαρες χιλιάδες·
\VS{3}Ἀπὸ τῶν υἱῶν Φαρὲς, ἄρχων πάντων τῶν ἀρχόντων τῆς δυνάμεως τοῦ μηνὸς τοῦ πρώτου.
\VS{4}Καὶ ἐπὶ τῆς διαιρέσεως τοῦ μηνὸς τοῦ δευτέρου Δωδία ὁ Ἐκχὼκ, καὶ ἐπὶ τῆς διαιρέσεως αὐτοῦ, καὶ Μακελλὼθ ὁ ἡγούμενος, καὶ ἐπὶ τῆς διαιρέσεως αὐτοῦ εἴκοσι καὶ τέσσαρες χιλιάδες ἄρχοντες δυνάμεως.
\VS{5}Ὁ τρίτος τὸν μῆνα τὸν τρίτον Βαναίας ὁ τοῦ Ἰωδαὲ ὁ ἱερεὺς ὁ ἄρχων, καὶ ἐπὶ τῆς διαιρέσεως αὐτοῦ εἴκοσι καὶ τέσσαρες χιλιάδες.
\VS{6}Αὐτὸς Βαναίας ὁ δυνατώτερος τῶν τριάκοντα καὶ ἐπὶ τῶν τριάκοντα· καὶ ἐπὶ τῆς διαιρέσεως αὐτοῦ Ζαβὰδ ὁ υἱὸς αὐτοῦ.
\VS{7}Ὁ τέταρτος εἰς τὸν μῆνα τὸν τέταρτον Ἀσαὴλ ὁ ἀδελφὸς Ἰωὰβ, καὶ Ζαβαδίας υἱὸς αὐτοῦ, καὶ οἱ ἀδελφοὶ, καὶ ἐπὶ τῆς διαιρέσεως αὐτοῦ εἴκοσι καὶ τέσσαρες χιλιάδες.
\VS{8}Ὁ πέμπτος τῷ μηνὶ τῷ πέμπτῳ ὁ ἡγούμενος Σαμαὼθ ὁ Ἰεσραὲ, καὶ ἐπὶ τῆς διαιρέσεως αὐτοῦ εἴκοσι καὶ τέσσαρες χιλιάδες.
\VS{9}Ὁ ἕκτος τῷ μηνὶ τῷ ἕκτῳ Ὁδουίας ὁ τοῦ Ἐκκῆς ὁ Θεκωΐτης, καὶ ἐπὶ τῆς διαιρέσεως αὐτοῦ εἴκοσι καὶ τέσσαρες χιλιάδες.
\VS{10}Ὁ ἕβδομος τῷ μηνὶ τῷ ἐβδόμῳ Χελλὴς ὁ ἐκ Φαλλοῦς ἀπὸ τῶν υἱῶν Ἐφραὶμ, καὶ ἑπὶ τῆς διαιρέσεως αὐτοῦ εἴκοσι καὶ τέσσαρες χιλιάδες.
\VS{11}Ὁ ὄγδοος τῷ μηνὶ τῷ ὀγδόῳ Σοβοχαῒ ὁ Οὐσαθὶ τῷ Ζαραῒ, καὶ ἐπὶ τῆς διαιρέσεως αὐτοῦ εἴκοσι καὶ τέσσαρες χιλιάδες.
\VS{12}Ὁ ἔννατος τῷ μηνὶ τῷ ἐννάτῳ Ἀβιέζερ ὁ ἐξ Ἀναθὼθ ὁ ἐκ γῆς Βενιαμὶν, καὶ ἐπὶ τῆς διαιρέσεως αὐτοῦ τέσσαρες καὶ εἴκοσι χιλιάδες.
\VS{13}Ὁ δέκατος τῷ μηνὶ τῷ δεκάτῳ Μεηρὰ ὁ ἐκ Νετωφαθὶ τῷ Ζαραῒ, καὶ ἐπὶ τῆς διαιρέσεως αὐτοῦ εἴκοσι καὶ τέσσαρες χιλιάδες.
\VS{14}Ὁ ἑνδέκατος τῷ μηνὶ τῷ ἑνδεκάτῳ Βαναίας ὁ ἐκ Φαραθὼν ἐκ τῶν υἱῶν Ἐφραὶμ, καὶ ἐπὶ τῆς διαιρέσεως αὐτοῦ εἴκοσι καὶ τέσσαρες χιλιάδες.
\VS{15}Ὁ δωδέκατος εἰς τὸν μῆνα τὸν δωδέκατον Χολδία ὁ ἐκ Νετωφαθὶ τῷ Γοθονιὴλ, καὶ ἐπὶ τῆς διαιρέσεως αὐτοῦ εἴκοσι καὶ τέσσαρες χιλιάδες.
\par }{\PP \VS{16}Καὶ ἐπὶ τῶν φυλῶν Ἰσραὴλ, τῷ Ῥουβὴν ἡγούμενος Ἐλιέζερ ὁ τοῦ Ζεχρί· τῷ Συμεὼν, Σαφατίας ὁ τοῦ Μααχά.
\VS{17}Τῷ Λευὶ, Ἀσαβίας ὁ τοῦ Καμουήλ· τῷ Ἀαρὼν, Σαδώκ.
\VS{18}Τῷ Ἰούδα, Ἐλιὰβ τῶν ἀδελφῶν Δαυίδ· τῷ Ἰσσάχαρ, Ἀμβρὶ ὁ τοῦ Μιχαήλ.
\VS{19}Τῷ Ζαβουλὼν, Σαμαΐας ὁ τοῦ Ἀβδίου· τῷ Νεφθαλὶ, Ἰεριμὼθ ὁ τοῦ Ὀζιήλ.
\VS{20}Τῷ Ἐφραὶμ, Ὠσὴ ὁ τοῦ Ὀζίου· τῷ ἡμίσει φυλῆς Μανασσῆ, Ἰωὴλ υἱὸς Φαδαΐα.
\VS{21}Τῷ ἡμίσει φυλῆς Μανασσῆ τῷ ἐν γῇ Γαλαὰδ, Ἰαδαῒ ὁ τοῦ Ζαδαίου· τοῖς υἱοῖς Βενιαμὶν, Ἰασιὴλ ὁ τοῦ Ἀβεννήρ.
\VS{22}Τῷ Δὰν, Ἀζαριὴλ ὁ τοῦ Ἰρωάβ· οὗτοι πατριάρχαι τῶν φυλῶν Ἰσραήλ.
\par }{\PP \VS{23}Καὶ οὐκ ἔλαβε Δαυὶδ τὸν ἀριθμὸν αὐτῶν ἀπὸ εἰκοσαετοῦς καὶ κάτω, ὅτι εἶπε Κύριος πληθῦναι τὸν Ἰσραὴλ ὡς τοὺς ἀστέρας τοῦ οὐρανοῦ.
\VS{24}Καὶ Ἰωὰβ ὁ τοῦ Σαρουία ἤρξατο ἀριθμεῖν ἐν τῷ λαῷ, καὶ οὐ συνετέλεσε· καὶ ἐγένετο ἐν τούτοις ὀργὴ ἐπὶ Ἰσραήλ· καὶ οὐ κατεχωρίσθη ὁ ἀριθμὸς ἐν βιβλίῳ λόγων τῶν ἡμερῶν τοῦ βασιλέως Δαυίδ.
\par }{\PP \VS{25}Καὶ ἐπὶ τῶν θησαυρῶν τοῦ βασιλέως, Ἀσμὼθ ὁ τοῦ Ὀδιὴλ, καὶ ἐπὶ τῶν θησαυρῶν τῶν ἐν ἀγρῷ καὶ ἐν ταῖς κώμαις καὶ ἐν τοῖς ἐποικίοις καὶ ἐν τοῖς πύργοις, Ἰωνάθαν ὁ τοῦ Ὀζίου.
\VS{26}Καὶ ἐπὶ τῶν γεωργούντων τὴν γῆν τῶν ἐργαζομένων, Ἐσδρὶ ὁ τοῦ Χελούβ.
\VS{27}Καὶ ἐπὶ τῶν χωρίων, Σεμεῒ ὁ ἐκ Ῥαὴλ, καὶ ἐπὶ τῶν θησαυρῶν τῶν ἐν τοῖς χωρίοις τοῦ οἴνου, Ζαβδὶ ὁ τοῦ Σεφνί.
\VS{28}Καὶ ἐπὶ τῶν ἐλαιώνων, καὶ ἐπὶ τῶν συκαμίνων τῶν ἐν τῇ πεδινῇ, Βαλλανὰν ὁ Γεδωρίτης· ἐπὶ δὲ τῶν θησαυρῶν τοῦ ἐλαίου, Ἰωάς.
\VS{29}Καὶ ἐπὶ τῶν βοῶν τῶν νομάδων τῶν ἐν τῷ Σαρὼν, Σατραῒ ὁ Σαρωνίτης· καὶ ἐπὶ τῶν βοῶν τῶν ἐν τοῖς αὐλῶσι, Σωφὰτ, ὁ τοῦ Ἀδλί·
\VS{30}Ἐπὶ δὲ τῶν καμήλων, Ἀβίας ὁ Ἰσμαηλίτης· ἐπὶ δὲ τῶν ὄνων, Ἰαδίας ὁ ἐκ Μεραθών.
\VS{31}Καὶ ἐπὶ τῶν προβάτων, Ἰαζὶζ ὁ Ἀγαρίτης· πάντες οὗτοι προστάται ὑπαρχόντων Δαυὶδ τοῦ βασιλέως.
\par }{\PP \VS{32}Καὶ Ἰωνάθαν ὁ πατράδελφος Δαυὶδ σύμβουλος, ἄνθρωπος συνετός· καὶ Ἰεὴλ ὁ τοῦ Ἀχαμὶ μετὰ τῶν υἱῶν τοῦ βασιλέως.
\VS{33}Ἀχιτόφελ σύμβουλος τοῦ βασιλέως, καὶ Χουσὶ ὁ πρῶτος φίλος τοῦ βασιλέως·
\VS{34}Καὶ μετὰ τοῦτον Ἀχιτόφελ ἐχόμενος Ἰωδαὲ ὁ τοῦ Βαναίου, καὶ Ἀβιάθαρ· καὶ Ἰωὰβ ἀρχιστράτηγος τοῦ βασιλέως.

\par }\Chap{28}{\PP \VerseOne{1}Καὶ ἐξεκκλησίασε Δαυὶδ πάντας τοὺς ἄρχοντας Ἰσραὴλ, ἄρχοντας τῶν κριτῶν, καὶ πάντας τοὺς ἄρχοντας τῶν ἐφημεριῶν τῶν περὶ τὸ σῶμα τοῦ βασιλέως, καὶ ἄρχοντας τῶν χιλιάδων καὶ τῶν ἑκατοντάδων, καὶ τοὺς γαζοφύλακας, καὶ τοὺς ἐπὶ τῶν ὑπαρχόντων αὐτοῦ, καὶ πάσης τῆς κτήσεως τοῦ βασιλέως, καὶ τῶν υἱῶν αὐτοῦ, σὺν τοῖς εὐνούχοις, καὶ τοὺς δυνάστας, καὶ τοὺς μαχητὰς τῆς στρατιᾶς ἐν Ἱερουσαλήμ.
\par }{\PP \VS{2}Καὶ ἔστη Δαυὶδ ἐν μέσῳ τῆς ἐκκλησίας, καὶ εἶπεν, ἀκούσατέ μου ἀδελφοί μου, καὶ λαός μου· ἐμοὶ ἐγένετο ἐπὶ καρδίαν οἰκοδομῆσαι οἶκον ἀναπαύσεως τῆς κιβωτοῦ διαθήκης Κυρίου, καὶ στάσιν ποδῶν Κυρίου ἡμῶν, καὶ ἡτοίμασα τὰ εἰς τὴν κατασκήνωσιν ἐπιτήδεια.
\VS{3}Καὶ ὁ Θεὸς εἶπεν, οὐκ οἰκοδομήσεις ἐμοὶ οἶκον τοῦ ἐπονομάσαι τὸ ὄνομά μου ἐπʼ αὐτῷ, ὅτι ἄνθρωπος πολεμιστὴς εἶ σὺ, καὶ αἷμα ἐξέχεας.
\VS{4}Καὶ ἐξελέξατο Κύριος ὁ Θεὸς Ἰσραὴλ ἐν ἐμοὶ ἀπὸ παντὸς οἴκου πατρός μου εἶναι βασιλέα ἐπὶ Ἰσραὴλ εἰς τὸν αἰῶνα, καὶ ἐν Ἰούδα ᾑρέτικε τὸ βασίλειον, καὶ ἐξ οἴκου Ἰούδα τὸν οἶκον τοῦ πατρός μου· καὶ ἐν τοῖς υἱοῖς τοῦ πατρός μου, ἐν ἐμοὶ ἠθέλησε τοῦ γενέσθαι με εἰς βασιλέα ἐπὶ παντὶ Ἰσραήλ.
\VS{5}Καὶ ἀπὸ πάντων τῶν υἱῶν μου, ὅτι πολλοὺς υἱοὺς ἔδωκέ μοι Κύριος, ἐξελέξατο ἐν Σαλωμὼν τῷ υἱῷ μου καθίσαι αὐτὸν ἐπὶ θρόνου βασιλείας Κυρίου ἐπὶ τὸν Ἰσραήλ.
\VS{6}Καὶ εἶπέ μοι ὁ Θεὸς, Σαλωμὼν ὁ υἱός σου οἰκοδομήσει τὸν οἶκόν μου καὶ τὴν αὐλήν μου, ὅτι ᾑρέτικα ἐν αὐτῷ εἶναί μου υἱὸν, κᾀγὼ ἔσομαι αὐτῷ εἰς πατέρα.
\VS{7}Καὶ κατορθώσω τὴν βασιλείαν αὐτοῦ ἕως αἰῶνος, ἐὰν ἰσχύσῃ τοῦ φυλάξασθαι τὰς ἐντολάς μου, καὶ τὰ κρίματά μου, ὡς ἡ ἡμέρα αὕτη.
\VS{8}Καὶ νῦν κατὰ πρόσωπον πάσης ἐκκλησίας Κυρίου, καὶ ἐν ὠσὶ Θεοῦ ἡμῶν, φυλάξασθε καὶ ζητήσατε πάσας τὰς ἐντολὰς Κυρίου τοῦ Θεοῦ ἡμῶν, ἵνα κληρονομήσητε τὴν γῆν τὴν ἀγαθὴν, καὶ κατακληρονομήσητε τοῖς υἱοῖς ὑμῶν μεθʼ ὑμᾶς ἕως αἰῶνος.
\par }{\PP \VS{9}Καὶ νῦν Σαλωμὼν υἱὲ, γνῶθι τὸν Θεὸν τῶν πατέρων σου, καὶ δούλευε αὐτῷ ἐν καρδίᾳ τελείᾳ· καὶ ψυχῇ θελούσῃ, ὅτι πάσας καρδίας ἐτάζει Κύριος, καὶ πᾶν ἐνθύμημα γινώσκει· ἐὰν ζητήσῃς αὐτὸν, εὑρεθήσεταί σοι, καὶ ἐὰν καταλείψῃς αὐτὸν, καταλείψει σε εἰς τέλος.
\VS{10}Ἴδε νῦν, ὅτι Κύριος ᾑρέτικέ σε οἰκοδομῆσαι αὐτῷ οἶκον εἰς ἁγίασμα, ἴσχυε καὶ ποίει.
\par }{\PP \VS{11}Καὶ ἔδωκε Δαυὶδ Σαλωμὼν τῷ υἱῷ αὐτοῦ τὸ παράδειγμα τοῦ ναοῦ καὶ τῶν οἴκων αὐτοῦ, καὶ τῶν ζακχῶν αὐτοῦ, καὶ τῶν ὑπερῴων, καὶ τῶν ἀποθηκῶν τῶν ἐσωτέρων, καὶ τοῦ οἴκου τοῦ ἐξιλασμοῦ,
\VS{12}καὶ τὸ παράδειγμα ὃ εἶχεν ἐν πνεύματι αὐτοῦ, τῶν αὐλῶν οἴκου Κυρίου, καὶ πάντων τῶν παστοφορίων τῶν κύκλῳ τῶν εἰς τὰς ἀποθήκας οἴκου Κυρίου, καὶ τῶν ἀποθηκῶν τῶν ἁγίων,
\VS{13}καὶ τῶν καταλυμάτων, καὶ τῶν ἐφημεριῶν τῶν ἱερέων καὶ τῶν Λευιτῶν εἰς πᾶσαν ἐργασίαν λειτουργίας οἴκου Κυρίου, καὶ τῶν ἀποθηκῶν τῶν λειτουργησίμων σκευῶν τῆς λατρείας οἴκου Κυρίου.
\VS{14}Καὶ τὸν σταθμὸν τῆς ὁλκῆς αὐτῶν τῶν τε χρυσῶν καὶ ἀργυρῶν λυχνιῶν τὴν ὁλκὴν ἔδωκεν αὐτῷ,
\VS{15}καὶ τῶν λύχνων.
\VS{16}Ἔδωκεν αὐτῷ ὁμοίως τὸν σταθμὸν τῶν τραπεζῶν τῆς προθέσεως, ἑκάστης τραπέζης χρυσῆς, καὶ ὡσαύτως τῶν ἀργυρῶν,
\VS{17}καὶ τῶν κρεαγρῶν καὶ σπονδείων καὶ τῶν φιαλῶν τῶν χρυσῶν· καὶ τὸν σταθμὸν τῶν χρυσῶν καὶ τῶν ἀργυρῶν, καὶ θυΐσκων κεφουρὲ, ἑκάστου σταθμοῦ.
\VS{18}Καὶ τῶν τοῦ θυσιαστηρίου τῶν θυμιαμάτων ἐκ χρυσίου δοκίμου σταθμὸν ὑπέδειξεν αὐτῷ, καὶ τὸ παράδειγμα τοῦ ἅρματος τῶν χερουβὶμ τῶν διαπεπετασμένων ταῖς πτέρυξι, καὶ σκιαζόντων ἐπὶ τῆς κιβωτοῦ διαθήκης Κυρίου·
\VS{19}πάντα ἐν γραφῇ χειρὸς Κυρίου ἔδωκε Δαυὶδ Σαλωμὼν, κατὰ τὴν περιγενηθεῖσαν αὐτῷ σύνεσιν τῆς κατεργασίας τοῦ παραδείγματος.
\par }{\PP \VS{20}Καὶ εἶπε Δαυὶδ Σαλωμὼν τῷ υἱῷ αὐτοῦ, ἴσχυε καὶ ἀνδρίζου καὶ ποίει, μὴ φοβοῦ μηδὲ πτοηθῇς, ὅτι Κύριος ὁ Θεός μου μετὰ σοῦ, οὐκ ἀνήσει σε, καὶ οὐ μὴ ἐγκαταλίπῃ ἕως τοῦ συντελέσαι σε πᾶσαν ἐργασίαν λειτουργίας οἴκου Κυρίου· καὶ ἰδοὺ τὸ παράδειγμα τοῦ ναοῦ καὶ τοῦ οἴκου αὐτοῦ, καὶ ζακχῶ αὐτοῦ, καὶ τὰ ὑπερῷα καὶ τὰς ἀποθήκας τὰς ἐσωτέρας, καὶ τὸν οἶκον τοῦ ἱλασμοῦ, καὶ τὸ παράδειγμα οἴκου Κυρίου.
\VS{21}Καὶ ἰδοὺ αἱ ἐφημερίαι τῶν ἱερέων καὶ τῶν Λευιτῶν εἰς πᾶσαν λειτουργίαν οἴκου Κυρίου, καὶ μετὰ σοῦ ἐν πάσῃ πραγματείᾳ, καὶ πᾶς πρόθυμος ἐν σοφίᾳ κατὰ πᾶσαν τέχνην, καὶ οἱ ἄρχοντες καὶ πᾶς ὁ λαὸς εἰς πάντας τοὺς λόγους σου.

\par }\Chap{29}{\PP \VerseOne{1}Καὶ εἶπε Δαυὶδ ὁ βασιλεὺς πάσῃ τῇ ἐκκλησίᾳ, Σαλωμὼν ὁ υἱός μου, εἰς ὃν ᾑρέτικεν ἐν αὐτῷ Κύριος, νέος καὶ ἁπαλὸς, καὶ τὸ ἔργον μέγα, ὅτι οὐκ ἀνθρώπῳ, ἀλλʼ ἢ Κυρίῳ Θεῷ.
\VS{2}Κατὰ πᾶσαν τὴν δύναμιν ἡτοίμακα εἰς οἶκον Θεοῦ μου χρυσὶον, ἀργύριον, χαλκόν, σίδηρον, ξύλα, λίθους σοὰμ, καὶ πληρώσεως λίθους πολυτελεῖς καὶ ποικίλους, καὶ πάντα λίθον τίμιον, καὶ Πάριον πολύν.
\VS{3}Καὶ ἔτι ἐν τῷ εὐδοκῆσαί με ἐν οἴκῳ Θεοῦ μου, ἔστι μοι ὃ περιπεποίημαι χρυσίον καὶ ἀργύριον, καὶ ἰδοὺ δέδωκα εἰς οἶκον Θεοῦ μου εἰς ὕψος, ἐκτὸς ὧν ἡτοίμακα εἰς τὸν οἶκον τῶν ἁγίων,
\VS{4}τρισχίλια τάλαντα χρυσίου τοῦ ἐκ Σουφὶρ, καὶ ἑπτακισχίλια τάλαντα ἀργυρίου δοκίμου, ἐξαλειφῆναι ἐν αὐτοῖς τοὺς τοίχους τοῦ ἱεροῦ,
\VS{5}εἰς τὸ χρυσίον τῷ χρυσίῳ, καὶ εἰς τὸ ἀργύριον τῷ ἀργυρίῳ, καὶ εἰς πᾶν ἔργον διὰ χειρὸς τῶν τεχνιτῶν· καὶ τίς ὁ προθυμούμενος πληρῶσαι τὰς χεῖρας αὐτοῦ σήμερον Κυρίῳ;
\par }{\PP \VS{6}Καὶ προεθυμήθησαν ἄρχοντες πατριῶν, καὶ οἱ ἄρχοντες τῶν υἱῶν Ἰσραὴλ, καὶ οἱ χιλίαρχοι καὶ οἱ ἑκατόνταρχοι, καὶ οἱ προστάται τῶν ἔργων, καὶ οἱ οἰκοδόμοι τοῦ βασιλέως.
\VS{7}Καὶ ἔδωκαν εἰς τὰ ἔργα τοῦ οἴκου Κυρίου χρυσίου τάλαντα πεντακισχίλια, καὶ χρυσοῦς μυρίους, καὶ ἀργυρίου ταλάντων δέκα χιλιάδας, καὶ χαλκοῦ τάλαντα μύρια ὀκτακισχίλια, καὶ σιδήρου ταλάντων χιλιάδας ἑκατόν.
\VS{8}Καὶ οἷς εὑρέθη παρʼ αὐτοῖς λίθος, ἔδωκαν εἰς τὰς ἀποθήκας οἴκου Κυρίου διὰ χειρὸς Ἰεϊὴλ τοῦ Γεδσωνί.
\VS{9}Καὶ εὐφράνθη ὁ λαὸς ὑπὲρ τοῦ προθυμηθῆναι, ὅτι ἐν καρδίᾳ πλήρει προεθυμήθησαν τῷ Κυρίῳ· καὶ Δαυὶδ ὁ βασιλεὺς εὐφράνθη μεγάλως.
\par }{\PP \VS{10}Καὶ εὐλόγησεν ὁ βασιλεὺς Δαυὶδ τὸν Κύριον ἐνώπιον τῆς ἐκκλησίας, λέγων,
\par }{\PP \VS{11}Εὐλογητὸς εἶ Κύριε ὁ Θεὸς Ἰσραὴλ, ὁ πατὴρ ἡμῶν, ἀπὸ τοῦ αἰῶνος καὶ ἕως τοῦ αἰῶνος. Σοὶ Κύριε ἡ μεγαλωσύνη, καὶ ἡ δύναμις, καὶ τὸ καύχημα, καὶ ἡ νίκη, καὶ ἡ ἰσχὺς, ὅτι σὺ πάντων τῶν ἐν τῷ οὐρανῷ καὶ ἐπὶ τῆς γῆς δεσπόζεις· ἀπὸ προσώπου σου ταράσσεται πᾶς βασιλεὺς, καὶ ἔθνος.
\VS{12}Παρὰ σοῦ ὁ πλοῦτος καὶ ἡ δόξα, σὺ πάντων ἄρχεις Κύριε, ὁ ἄρχων πάσης ἀρχῆς, καὶ ἐν χειρί σου ἰσχὺς καὶ δυναστεία, καὶ ἐν χειρί σου παντοκράτωρ μεγαλῦναι καὶ κατισχύσαι τὰ πάντα.
\VS{13}Καὶ νῦν, Κύριε ἐξομολογούμεθά σοι, καὶ αἰνοῦμεν τὸ ὄνομα τῆς καυχήσεώς σου.
\VS{14}Καὶ τίς εἰμι ἐγὼ, καὶ τίς ὁ λαός μου ὅτι ἰσχύσαμεν προθυμηθῆναί σοι κατὰ ταῦτα; ὅτι σὰ τὰ πάντα, καὶ ἐκ τῶν σῶν δεδώκαμέν σοι.
\VS{15}Ὅτι πάροικοι ἐσμὲν ἐναντίον σου, καὶ παροικοῦντες ὡς πάντες οἱ πατέρες ἡμων· ὡς σκιὰ αἱ ἡμέραι ἡμῶν ἐπὶ γῆς, καὶ οὐκ ἔστιν ὑπομονή.
\VS{16}Κύριε ὁ Θεὸς ἡμῶν, πρὸς πᾶν τὸ πλῆθος τοῦτο ὃ ἡτοίμακα οἰκοδομηθῆναι οἶκον τῷ ὀνόματι τῷ ἁγίῳ σου, ἐκ χειρός σου ἐστὶ, καὶ σοὶ τὰ πάντα.
\VS{17}Καὶ ἔγνων, Κύριε, ὅτι σὺ εἶ ὁ ἐτάζων καρδίας, καὶ δικαιοσύνην ἀγαπᾷς· ἐν ἁπλότητι καρδίας προεθυμήθην ταῦτα πάντα, καὶ νῦν τὸν λαόν σου τὸν εὑρεθέντα ὧδε εἶδον ἐν εὐφροσύνῃ προθυμηθέντα σοι.
\VS{18}Κύριε ὁ Θεὸς Ἁβραὰμ, καὶ Ἰσαὰκ, καὶ Ἰσραὴλ, τῶν πατέρων ἡμῶν, φύλαξον ταῦτα ἐν διανοίᾳ καρδίας λαοῦ σου εἰς τὸν αἰῶνα, καὶ κατεύθυνον τὰς καρδίας αὐτῶν πρὸς σέ.
\VS{19}Καὶ Σαλωμὼν τῷ υἱῷ μου δὸς καρδίαν ἀγαθὴν ποιεῖν τὰς ἐντολάς σου, καὶ τὰ μαρτύριά σου, καὶ τὰ προστάγματά σου, καὶ τοῦ ἐπὶ τέλος ἀγαγεῖν τὴν κατασκευὴν τοῦ οἴκου σου.
\par }{\PP \VS{20}Καὶ εἶπε Δαυὶδ πάσῃ τῇ ἐκκλησίᾳ, εὐλογήσατε Κύριον τὸν Θεὸν ἡμῶν· καὶ εὐλόγησε πᾶσα ἡ ἐκκλησία Κύριον τὸν Θεὸν τῶν πατέρων αὐτῶν· καὶ κάμψαντες τὰ γόνατα προσεκύνησαν Κυρίῳ, καὶ τῷ βασιλεῖ.
\VS{21}Καὶ ἔθυσε Δαυὶδ τῷ Κυρίῳ θυσίας, καὶ ἀνήνεγκεν ὁλοκαυτώματα τῷ Θεῷ τῇ ἐπαύριον τῆς πρώτης ἡμέρας μόσχους χιλίους, κριοὺς χιλίους, ἄρνας χιλίους, καὶ τὰς σπονδὰς αὐτῶν, καὶ θυσίας εἰς πλῆθος παντὶ τῷ Ἰσραήλ.
\VS{22}Καὶ ἔφαγον καὶ ἔπιον ἐναντίον τοῦ Κυρίου ἐν ἐκείνῃ τῇ ἡμέρᾳ μετὰ χαρᾶς· καὶ ἐβασίλευσαν ἐκ δευτέρου τὸν Σαλωμὼν υἱὸν Δαυὶδ, καὶ ἔχρισαν αὐτὸν τῷ Κυρίῳ εἰς βασιλέα, καὶ Σαδὼκ εἰς ἱερωσύνην.
\par }{\PP \VS{23}Καὶ ἐκάθισε Σαλωμὼν ἐπὶ θρόνου Δαυὶδ τοῦ πατρὸς αὐτοῦ, καὶ εὐδοκήθη, καὶ ὑπήκουσαν αὐτοῦ πᾶς Ἰσραήλ.
\VS{24}Οἱ ἄρχοντες, καὶ οἱ δυνάσται, καὶ πάντες υἱοὶ Δαυὶδ τοῦ βασιλέως τοῦ πατρὸς αὐτοῦ, ὑπετάγησαν αὐτῷ.
\VS{25}Καὶ ἐμεγάλυνε Κύριος τὸν Σαλωμὼν ἐπάνωθεν παντὸς Ἰσραὴλ, καὶ ἔδωκεν αὐτῷ δόξαν βασιλέως, ὃ οὐκ ἐγένετο ἐπὶ παντὸς βασιλέως ἔμπροσθεν αὐτοῦ.
\par }{\PP \VS{26}Καὶ Δαυὶδ υἱὸς Ἰεσσαὶ ἐβασίλευσεν ἐπὶ Ἰσραὴλ
\VS{27}ἔτη τεσσράκοντα, ἐν Χεβρὼν ἔτη ἑπτὰ, καὶ ἐν Ἱερουσαλὴμ ἔτη τριακοντατρία.
\VS{28}Καὶ ἐτελεύτησεν ἐν γήρᾳ καλῷ, πλήρης ἡμερῶν, πλούτῳ καὶ δόξῃ, καὶ ἐβασίλευσε Σαλωμὼν υἱὸς αὐτοῦ ἀντʼ αὐτοῦ.
\VS{29}Οἱ δὲ λοιποὶ λόγοι τοῦ βασιλέως Δαυὶδ οἱ πρότεροι καὶ οἱ ὕστεροι γεγραμμένοι εἰσὶν ἐν λόγοις Σαμουὴλ τοῦ βλέποντος, καὶ ἐπὶ λόγων Νάθαν τοῦ προφήτου, καὶ ἐπὶ λόγων Γὰδ τοῦ βλέποντος,
\VS{30}περὶ πάσης τῆς βασιλείας αὐτοῦ, καὶ τῆς δυναστείας αὐτοῦ, καὶ οἱ καιροὶ οἳ ἐγένοντο ἐπʼ αὐτῷ, καὶ ἐπὶ τὸν Ἰσραὴλ, καὶ ἐπὶ πάσας βασιλείας τῆς γῆς.
\par }