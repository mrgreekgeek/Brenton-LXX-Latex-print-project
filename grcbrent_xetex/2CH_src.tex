\NormalFont\ShortTitle{ΠΑΡΑΛΕΙΠΟΜΕΝΩΝ Β}
{\MT ΠΑΡΑΛΕΙΠΟΜΕΝΩΝ Β

\par }\ChapOne{1}{\PP \VerseOne{1}ΚΑΙ ἐνίσχυσε Σαλωμὼν υἱὸς Δαυὶδ ἐπὶ τὴν βασιλείαν αὐτοῦ, καὶ Κύριος ὁ Θεὸς αὐτοῦ μετʼ αὐτοῦ, καὶ ἐμεγάλυνεν αὐτὸν εἰς ὕψος.
\VS{2}Καὶ εἶπε Σαλωμὼν πρὸς πάντα Ἰσραὴλ, τοῖς χιλιάρχοις, καὶ τοῖς ἑκατοντάρχοις, καὶ τοῖς κριταῖς, καὶ πᾶσι τοῖς ἄρχουσιν ἐναντίον Ἰσραὴλ τοῖς ἄρχουσι τῶν πατριῶν.
\VS{3}καὶ ἐπορεύθη Σαλωμὼν καὶ πᾶσα ἡ ἐκκλησία εἰς τὴν ὑψηλὴν τὴν ἐν Γαβαὼν, οὗ ἐκεῖ ἦν ἡ σκηνὴ τοῦ μαρτυρίου τοῦ Θεοῦ, ἣν ἐποίησε Μωυσῆς παῖς Κυρίου ἐν τῇ ἐρήμῳ.
\VS{4}Ἀλλὰ κιβωτὸν τοῦ Θεοῦ ἀνήνεγκε Δαυὶδ ἐκ πόλεως Καριαθιαρὶμ, ὅτι ἡτοίμασεν αὐτῇ Δαυὶδ, ὅτι ἔπηξεν αὐτῇ σκηνὴν ἐν Ἱερουσαλήμ.
\VS{5}Καὶ τὸ θυσιαστήριον τὸ χαλκοῦν ὃ ἐποίησε Βεσελεὴλ υἱὸς Οὐρίου υἱοῦ Ὢρ, ἐκεῖ ἦν ἔναντι τῆς σκηνῆς Κυρίου· καὶ ἐξεζήτησεν αὐτὸ Σαλωμὼν καὶ ἡ ἐκκλησία,
\VS{6}καὶ ἤνεγκε Σαλωμὼν ἐκεῖ ἐπὶ τὸ θυσιαστήριον τὸ χαλκοῦν ἐνώπιον Κυρίου τὸ ἐν τῇ σκηνῇ, καὶ ἤνεγκεν ἐπʼ αὐτῷ ὁλοκαύτωσιν χιλίαν.
\par }{\PP \VS{7}Ἐν τῇ νυκτὶ ἐκείνῃ ὤφθη Θεὸς τῷ Σαλωμὼν, καὶ εἶπεν αὐτῷ, αἴτησαι τί σοι δῶ.
\VS{8}Καὶ εἶπε Σαλωμὼν πρὸς τὸν Θεὸν, σὺ ἐποίησας μετὰ Δαυὶδ τοῦ πατρός μου ἔλεος μέγα, καὶ ἐβασίλευσάς με ἀντʼ αὐτοῦ.
\VS{9}Καὶ νῦν, Κύριε ὁ Θεὸς, πιστωθήτω δὴ τὸ ὄνομά σου ἐπὶ Δαυὶδ τὸν πατέρα μου, ὅτι σὺ ἐβασίλευσάς με ἐπὶ λαὸν πολὺν, ὡς ὁ χοῖς τῆς γῆς.
\VS{10}Νῦν σοφίαν καὶ σύνεσιν δός μοι, καὶ ἐξελεύσομαι ἐνώπιον τοῦ λαοῦ τούτου καὶ εἰσελεύσομαι, ὅτι τίς κρινεῖ τὸν λαόν σου τὸν μέγαν τοῦτον;
\par }{\PP \VS{11}Καὶ εἶπεν ὁ Θεὸς πρὸς Σαλωμὼν, ἀνθʼ ὧν ἐγένετο τοῦτο ἐν τῇ καρδίᾳ σου, καὶ οὐκ ᾐτήσω πλοῦτον χρημάτων, οὐδὲ δόξαν, οὐδὲ τὴν ψυχὴν τῶν ὑπεναντίων, καὶ ἡμέρας πολλὰς οὐκ ᾐτήσω, καὶ ᾔτησας σεαυτῷ σοφίαν καὶ σύνεσιν, ὅπως κρίνῃς τὸν λαόν μου, ἐφʼ ὃν ἐβασίλευσά σε ἐπʼ αὐτὸν,
\VS{12}τὴν σοφίαν καὶ τὴν σύνεσιν δίδωμί σοι, καὶ πλοῦτον καὶ χρήματα καὶ δόξαν δώσω σοι, ὡς οὐκ ἐγενήθη ὅμοιός σοι ἐν τοῖς βασιλεῦσι τοῖς ἔμπροσθέν σου, καὶ μετὰ σὲ οὐκ ἔσται οὕτως.
\par }{\PP \VS{13}Καὶ ἦλθε Σαλωμὼν ἐκ βαμὰ τῆς ἐν Γαβαὼν εἰς Ἱερουσαλὴμ πρὸ προσώπου τῆς σκηνῆς τοῦ μαρτυρίου, καὶ ἐβασίλευσεν ἐπὶ Ἰσραήλ.
\par }{\PP \VS{14}Καὶ συνήγαγε Σαλωμὼν ἅρματα καὶ ἱππεῖς, καὶ ἐγένοντο αὐτῷ χιλία καὶ τετρακόσια ἅρματα, καὶ δώδεκα χιλιάδες ἱππέων· καὶ κατέλιπεν αὐτὰ ἐν πόλεσι τῶν ἁρμάτων, καὶ ὁ λαὸς μετὰ τοῦ βασιλέως ἐν Ἱερουσαλήμ.
\VS{15}Καὶ ἔθηκεν ὁ βασιλεὺς τὸ ἀργύριον καὶ τὸ χρυσίον ἐν Ἱερουσαλὴμ ὡς λίθους, καὶ τὰς κέδρους ἐν τῇ Ἰουδαίᾳ ὡς συκαμίνους τὰς ἐν τῇ πεδινῇ εἰς πλῆθος.
\VS{16}Καὶ ἡ ἔξοδος τῶν ἵππων Σαλωμὼν ἐξ Αἰγύπτου, καὶ ἡ τιμὴ τῶν ἐμπόρων τοῦ βασιλέως πορεύεσθαι, καὶ ἠγόραζον,
\VS{17}καὶ ἐνέβαινον καὶ ἐξῆγον ἐξ Αἰγύπτου ἅρμα ἓν ἑξακοσίων ἀργυρίου, καὶ ἵππον πεντήκοντα καὶ ἑκατὸν ἀργυρίου· καὶ οὕτω πᾶσι τοῖς βασιλεῦσι τῶν Χετταίων καὶ τοῖς βασιλεῦσι Συρίας ἐν χερσὶν αὐτῶν ἔφερον.
\par }{\PP \VS{18}Καὶ εἶπε Σαλωμὼν τοῦ οἰκοδομῆσαι οἶκον τῷ ὀνόματι Κυρίου, καὶ οἶκον τῷ βασιλείᾳ αὐτοῦ.

\Chap{2}\VerseOne{1}Καὶ συνήγαγε Σαλωμὼν ἑβδομήκοντα χιλιάδας ἀνδρῶν νωτοφόρων, καὶ ὀγδοήκοντα χιλιάδας λατόμων ἐν τῷ ὄρει, καὶ οἱ ἐπιστάται ἐπʼ αὐτῶν τρισχίλιοι ἑξακόσιοι.
\par }{\PP \VS{2}Καὶ ἀπέστειλε Σαλωμὼν πρὸς Χιρὰμ βασιλέα Τύρου, λέγων, ὡς ἐποίησας μετὰ Δαυὶδ τοῦ πατρός μου, καὶ ἀπέστειλας αὐτῷ κέδρους τοῦ οἰκοδομῆσαι ἑαυτῷ οἶκον κατοικῆσαι ἐν αὐτῷ,
\VS{3}καὶ ἰδοὺ ἐγὼ ὁ υἱὸς αὐτοῦ οἰκοδομῶ οἶκον τῷ ὀνόματι Κυρίου Θεοῦ μου, ἁγιάσαι αὐτὸν αὐτῷ τοῦ θυμιᾷν ἀπέναντι αὐτοῦ θυμίαμα καὶ πρόθεσιν διαπαντὸς, καὶ τοῦ ἀναφέρειν ὁλοκαυτώματα διαπαντὸς τοπρωῒ καὶ τοδείλης, καὶ ἐν τοῖς σαββάτοις, καὶ ἐν ταῖς νουμηνίαις, καὶ ἐν ταῖς ἑορταῖς τοῦ Κυρίου Θεοῦ ἡμῶν· εἰς τὸν αἰῶνα τοῦτο ἐπὶ τὸν Ἰσραήλ.
\VS{4}Καὶ ὁ οἶκος ὃν ἐγὼ οἰκοδομῶ, μέγας, ὅτι μέγας Κύριος ὁ Θεὸς ἡμῶν παρὰ πάντας τοὺς θεούς.
\VS{5}Καὶ τίς ἰσχύσει οἰκοδομῆσαι αὐτῷ οἶκον; ὅτι ὁ οὐρανὸς, καὶ ὁ οὐρανὸς τοῦ οὐρανοῦ οὐ φέρουσι τὴς δόξαν αὐτοῦ· καὶ τίς ἐγὼ οἰκοδομῶν αὐτῷ οἶκον; ὅτι ἀλλʼ ἢ τοῦ θυμιᾷν κατέναντι αὐτοῦ.
\VS{6}Καὶ νῦν ἀπόστειλόν μοι ἄνδρα σοφὸν καὶ εἰδότα τοῦ ποιῆσαι ἐν τῷ χρυσίῳ, καὶ ἐν τῷ ἀργυρίῳ, καὶ ἐν τῷ χαλκῷ, καὶ ἐν τῷ σιδήρῳ, καὶ ἐν τῇ πορφύρᾳ, καὶ ἐν τῷ κοκκίνῳ, καὶ ἐν τῇ ὑακίνθῳ, καὶ ἐπιστάμενον γλύψαι γλυφὴν μετὰ τῶν σοφῶν τῶν μετʼ ἐμοῦ ἐν Ἰούδα καὶ ἐν Ἱερουσαλὴμ, ἃ ἡτοίμασε Δαυὶδ ὁ πατήρ μου.
\VS{7}Καὶ ἀπόστειλόν μοι ξύλα κέδρινα καὶ ἀρκεύθινα καὶ πεύκινα ἐκ τοῦ Λιβάνου, ὅτι ἐγὼ οἶδα ὡς οἱ δοῦλοί σου οἴδασι κόπτειν ξύλα ἐκ τοῦ Λιβάνου· καὶ ἰδοὺ οἱ παῖδές σου μετὰ τῶν παίδων μου
\VS{8}πορεύσονται ἑτοιμάσαι μοι ξύλα εἰς πλῆθος, ὅτι ὁ οἶκος ὃν ἐγὼ οἰκοδομῶ μέγας καὶ ἔνδοξος.
\VS{9}Καὶ ἰδοὺ τοῖς ἐργαζομένοις τοῖς κόπτουσι ξύλα, εἰς βρώματα δέδωκα σῖτον εἰς δόματα τοῖς παισί σου κόρων πυροῦ εἴκοσι χιλιάδας, καὶ κριθῶν κόρων εἴκοσι χιλιάδας, καὶ οἴνου μέτρων εἴκοσι χιλιάδας, καὶ ἐλαίου μέτρων εἴκοσι χιλιάδας.
\par }{\PP \VS{10}Καὶ εἶπε Χιρὰμ βασιλεὺς Τύρου ἐν γραφῇ, καὶ ἀνέστειλε πρὸς Σαλωμὼν, λέγων, ἐν τῷ ἀγαπῆσαι Κύριον τὸν λαὸν αὐτοῦ, ἔδωκέ σε ἐπʼ αὐτοὺς βασιλέα.
\VS{11}Καὶ εἶπε Χιρὰμ, εὐλογητὸς Κύριος ὁ Θεὸς Ἰσραὴλ, ὃς ἐποίησε τὸν οὐρανὸν καὶ τὴν γῆν, ὃς ἔδωκε τῷ Δαυὶδ τῷ βασιλεῖ υἱὸν σοφὸν, καὶ ἐπιστάμενον ἐπιστήμην καὶ σύνεσιν, ὃς οἰκοδομήσει οἶκον τῷ Κυρίῳ, καὶ οἶκον τῇ βασιλείᾳ αὐτοῦ.
\VS{12}Καὶ νῦν ἀπέστειλά σοι ἄνδρα σοφὸν καὶ εἰδότα σύνεσιν Χιρὰμ τὸν πατέρα μου,
\VS{13}ἡ μήτηρ αὐτοῦ ἀπὸ θυγατέρων Δὰν, καὶ ὁ πατὴρ αὐτοῦ ἀνὴρ Τύριος, εἰδότα ποιῆσαι ἐν χρυσίῳ, καὶ ἐν ἀργυρίῳ, καὶ ἐν χαλκῷ, καὶ ἐν σιδήρῳ, καὶ ἐν λίθοις καὶ ξύλοις, καὶ ὑφαίνειν ἐν τῇ πορφύρᾳ, καὶ ἐν τῇ ὑακίνθῳ, καὶ ἐν τῇ βύσσῳ, καὶ ἐν τῷ κοκκίνῳ, καὶ γλύψαι γλυφὰς, καὶ διανοεῖσθαι πᾶσαν διανόησιν, ὅσα ἂν δῷς αὐτῷ μετὰ τῶν σοφῶν σου, καὶ σοφῶν Δαυὶδ κυρίου μου πατρός σου.
\VS{14}Καὶ νῦν τὸν σῖτον καὶ τὴν κριθὴν, καὶ τὸ ἔλαιον, καὶ τὸν οἶνον ἃ εἶπεν ὁ κύριός μου, ἀποστειλάτω τοῖς παισὶν αὐτοῦ.
\VS{15}Καὶ ἡμεῖς κόψομεν ξύλα ἐκ τοῦ Λιβάνου κατὰ πᾶσαν τὴν χρείαν σου, καὶ ἄξομεν αὐτὰ σχεδίαις ἐπὶ θάλασσαν Ἰόππης, καὶ σὺ ἄξεις αὐτὰ εἰς Ἱερουσαλήμ.
\par }{\PP \VS{16}Καὶ συνήγαγε Σαλωμὼν πάντας τοὺς ἄνδρας τοὺς προσηλύτους τοὺς ἐν γῇ Ἰσραὴλ μετὰ τὸν ἀριθμὸν ὃν ἠρίθμησεν αὐτοὺς Δαυὶδ ὁ πατὴρ αὐτοῦ, καὶ εὑρέθησαν ἑκατὸν πεντήκοντα χιλιάδες καὶ τρισχίλιοι ἑξακόσιοι.
\VS{17}Καὶ ἐποίησεν ἐξ αὐτῶν ἑβδομήκοντα χιλιάδας νωτοφόρων, καὶ ὀγδοήκοντα χιλιάδας λατόμων, καὶ τρισχιλίους ἑξακοσίους ἐργοδιώκτας ἐπὶ τὸν λαόν.

\par }\Chap{3}{\PP \VerseOne{1}Καὶ ἤρξατο Σαλωμὼν τοῦ οἰκοδομεῖν τὸν οἶκον Κυρίου ἐν Ἱερουσαλὴμ ἐν ὄρει τοῦ Ἀμωρία, οὗ ὤφθη Κύριος τῷ Δαυὶδ πατρὶ αὐτοῦ, ἐν τῷ τόπῳ ᾧ ἡτοίμασε Δαυὶδ ἐν ἅλῳ Ὀρνὰ τοῦ Ἰεβουσαίου.
\VS{2}Καὶ ἤρξατο οἰκοδομῆσαι ἐν τῷ μηνὶ τῷ δευτέρῳ ἐν τῷ ἔτει τῷ τετάρτῳ τῆς βασιλείας αὐτοῦ.
\par }{\PP \VS{3}Καὶ ταῦτα ἤρξατο Σαλωμὼν τοῦ οἰκοδομῆσαι τὸν οἶκον τοῦ Θεοῦ· μῆκος πήχεων ἡ διαμέτρησις ἡ πρώτη πήχεων ἑξήκοντα, καὶ εὖρος πήχεων εἴκοσι.
\VS{4}Καὶ αἰλὰμ κατὰ πρόσωπον τοῦ οἴκου, μῆκος ἐπὶ πρόσωπον πλάτους τοῦ οἴκου πήχεων εἴκοσι, καὶ ὕψος πήχεων ἑκατὸν εἴκοσι, καὶ κατεχρύσωσεν αὐτὸν ἔσωθεν χρυσίῳ καθαρῷ.
\VS{5}Καὶ τὸν οἶκον τὸν μέγαν ἐξύλωσε ξύλοις κεδρίνοις, καὶ κατεχρύσωσε χρυσίῳ καθαρῷ, καὶ ἔγλυψεν ἐπʼ αὐτοῦ φοίνικας καὶ χαλαστά.
\VS{6}Καὶ ἐκόσμησε τὸν οἶκον λίθοις τιμίοις εἰς δόξαν, καὶ ἐχρύσωσε χρυσίῳ χρυσίου τοῦ ἐκ Φαρουΐμ.
\VS{7}Καὶ ἐχρύσωσε τὸν οἶκον, καὶ τοὺς τοίχους αὐτοῦ, καὶ τοὺς πυλῶνας, καὶ τὰ ὀροφώματα, καὶ τὰ θυρώματα χρυσίῳ, καὶ ἔγλυψε χερουβὶμ ἐπὶ τῶν τοίχων.
\par }{\PP \VS{8}Καὶ ἐποίησε τὸν οἶκον τοῦ ἁγίου τῶν ἁγίων, μῆκος αὐτοῦ ἐπὶ πρόσωπον, πλάτος τοῦ οἴκου πήχεων εἴκοσι, καὶ τὸ μῆκος πήχεων εἴκοσι, καὶ ἐχρύσωσεν αὐτὸν χρυσίῳ καθαρῷ εἰς χερουβὶμ εἰς τάλαντα ἑξακόσια.
\VS{9}Καὶ ὁλκὴ τῶν ἥλων, ὁλκὴ τοῦ ἑνὸς πεντήκοντα σίκλοι χρυσίου, καὶ τὸ ὑπερῷον ἐχρύσωσε χρυσίῳ.
\par }{\PP \VS{10}Καὶ ἐποίησεν ἐν τῷ οἴκῳ τῷ ἁγίῳ τῶν ἁγίων χερουβὶμ δύο, ἔργον ἐκ ξύλων· καὶ ἐχρύσωσεν αὐτὰ χρυσίῳ.
\VS{11}Καὶ αἱ πτέρυγες τῶν χερουβὶμ τὸ μῆκος πήχεων εἴκοσι, καὶ ἡ πτέρυξ ἡ μία πήχεων πέντε ἁπτομένη τοῦ τοίχου τοῦ οἴκου, καὶ ἡ πτέρυξ ἡ ἑτέρα πήχεων πέντε ἁπτομένη τῆς πτέρυγος τοῦ χερουβὶμ τοῦ ἑτέρου.
\VS{13}Καὶ αἱ πτέρυγες τῶν χερουβὶμ τούτων διαπεπετασμέναι πήχεων εἴκοσι, καὶ αὐτὰ ἑστηκότα ἐπὶ τοὺς πόδας αὐτῶν, καὶ τὰ πρόσωπα αὐτῶν εἰς τὸν οἶκον.
\VS{14}Καὶ ἐποίησε τὸ καταπέτασμα ὑακίνθου, καὶ πορφύρας, καὶ κοκκίνου, καὶ βύσσου, καὶ ὕφανεν ἐν αὐτῷ χερουβίμ.
\par }{\PP \VS{15}Καὶ ἐποίησεν ἔμπροσθεν τοῦ οἴκου στύλους δύο, πήχεων τριακονταπέντε τὸ ὕψος, καὶ τὰς κεφαλὰς αὐτῶν πήχεων πέντε.
\VS{16}Καὶ ἐποίησε σερσερὼθ ἐν τῷ δαβὶρ, καὶ ἔδωκεν ἐπὶ τῶν κεφαλῶν τῶν στύλων· καὶ ἐποίησε ῥοΐσκους ἑκατὸν, καὶ ἔθηκεν ἐπὶ τῶν χαλαστῶν.
\VS{17}Καὶ ἔστησε τοὺς στύλους κατὰ πρόσωπον τοῦ ναοῦ, ἕνα ἐκ δεξιῶν, καὶ τὸν ἕνα ἐξ εὐωνύμων· καὶ ἐκάλεσε τὸ ὄνομα τοῦ ἐκ δεξιῶν, Κατόρθωσις, καὶ τὸ ὄνομα τοῦ ἐξ ἀριστερῶν, Ἰσχύς.

\par }\Chap{4}{\PP \VerseOne{1}Καὶ ἐποίησε θυσιαστήριον χαλκοῦν, εἴκοσι πήχεων τὸ μῆκος, καὶ εἴκοσι πήχεων τὸ εὖρος, καὶ δέκα πήχεων τὸ ὕψος.
\VS{2}Καὶ ἐποίησε τὴν θάλασσαν χυτὴν, δέκα πήχεων τὴν διαμέτρησιν, στρογγύλην κυκλόθεν, καὶ πέντε πήχεων τὸ ὕψος, καὶ τὸ κύκλωμα τριάκοντα πήχεων.
\VS{3}Καὶ ὁμοίωμα μόσχων ὑποκάτω αὐτῆς, κύκλῳ κυκλοῦσιν αὐτήν· δέκα πήχεις περιέχουσι τὸν λουτῆρα κυκλόθεν· δύο γένη ἐχώνευσαν τοὺς μόσχους ἐν τῇ χωνεύσει αὐτῶν
\VS{4}ᾗ ἐποίησαν αὐτοὺς δώδεκα μόσχους, οἱ τρεῖς βλέποντες Βοῤῥᾶν, καὶ οἱ τρεῖς δυσμὰς, καὶ οἱ τρεῖς Νότον, καὶ οἱ τρεῖς κατʼ ἀνατολὰς, καὶ ἡ θάλασσα ἐπʼ αὐτῶν ἄνω, ἦσαν τὰ ὀπίσθια αὐτῶν ἔσω.
\VS{5}Καὶ τὸ πάχος αὐτῆς παλαιστὴς, καὶ τὸ χεῖλος αὐτῆς ὡς χεῖλος ποτηρίου, διαγεγλυμμένα βλαστοὺς κρίνου, χωροῦσαν μετρητὰς τρισχιλίους· καὶ ἐξετέλεσε.
\par }{\PP \VS{6}Καὶ ἐποίησε λουτῆρας δέκα, καὶ ἔθηκε τοὺς πέντε ἐκ δεξιῶν καὶ τοὺς πέντε ἐξ ἀριστερῶν, τοῦ πλύνειν ἐν αὐτοῖς τὰ ἔργα τῶν ὁλοκαυτωμάτων, καὶ ἀποκλύζειν ἐν αὐτοῖς, καὶ ἡ θάλασσα εἰς τὸ νίπτεσθαι τοὺς ἱερεῖς ἐν αὐτῇ.
\par }{\PP \VS{7}Καὶ ἐποίησε τὰς λυχνίας τὰς χρυσᾶς δέκα κατὰ τὸ κρίμα αὐτῶν, καὶ ἔθηκεν ἐν τῷ ναῷ, πέντε ἐκ δεξιῶν καὶ πέντε ἐξ ἀριστερῶν.
\par }{\PP \VS{8}Καὶ ἐποίησε τραπέζας δέκα, καὶ ἔθηκεν ἐν τῷ ναῷ, πέντε ἐκ δεξιῶν καὶ πέντε ἐξ εὐωνύμων, καὶ ἐποίησε φιάλας χρυσᾶς ἑκατὸν,
\VS{9}καὶ ἐποίησε τὴν αὐλὴν τῶν ἱερέων, καὶ τὴν αὐλὴν τὴν μεγάλην, καὶ θύρας τῇ αὐλῇ, καὶ θυρώματα αὐτῶν κατακεχαλκωμένα χαλκῷ.
\VS{10}Καὶ τὴν θάλασσαν ἔθηκεν ἀπὸ γωνίας τοῦ οἴκου ἐκ δεξιῶν ὡς πρὸς ἀνατολὰς κατέναντι.
\par }{\PP \VS{11}Καὶ ἐποίησε Χιρὰμ τὰς κρεάγρας, καὶ τὰ πυρεῖα, καὶ τὴν ἐσχάραν τοῦ θυσιαστηρίου, καὶ πάντα τὰ σκεύη αὐτοῦ· καὶ συνετέλεσε Χιρὰμ ποιῆσαι πᾶσαν τὴν ἐργασίαν ἣν ἐποίησε Σαλωμὼν τῷ βασιλεῖ ἐν οἴκῳ τοῦ Θεοῦ,
\VS{12}στύλους δύο, καὶ ἐπʼ αὐτῶν γελὰθ τῇ χωθαρὲθ ἐπὶ τῶν κεφαλῶν τῶν στύλων δύο, καὶ δίκτυα δύο συγκαλύψαι τὰς κεφαλὰς τῶν χωθαρὲθ ἅ ἐστιν ἐπὶ τῶν κεφαλῶν τῶν στύλων,
\VS{13}καὶ κώδωνας χρυσοῦς τετρακοσίους εἰς τὰ δύο δίκτυα, καὶ δύο γένη ῥοΐσκων ἐν τῷ δικτύῳ τῷ ἑνὶ τοῦ συγκαλύψαι τὰς δύο γωλὰθ τῶν χωθαρὲθ ἅ ἐστιν ἐπάνω τῶν στύλων·
\VS{14}Καὶ τοὺς μεχωνὼθ ἐποίησε δέκα, καὶ τοὺς λουτῆρας ἐποίησεν ἐπὶ τοὺς μεχωνὼθ,
\VS{15}καὶ τὴν θάλασσαν μίαν, καὶ τοὺς μόσχους τοὺς δώδεκα ὑποκάτω αὐτῆς,
\VS{16}καὶ τοὺς ποδιστῆρας, καὶ τοὺς ἀναλημπτῆρας, καὶ τοὺς λέβητας, καὶ τὰς κρεάγρας, καὶ πάντα τὰ σκεύη αὐτῶν ἃ ἐποίησε Χιρὰμ, καὶ ἀνήνεγκε τῷ βασιλεῖ Σαλωμὼν ἐν οἴκῳ Κυρίου, χαλκοῦ καθαροῦ.
\VS{17}Ἐν τῷ περιχώρῳ τοῦ Ἰορδάνου ἐχώνευσεν αὐτὰ ὁ βασιλεὺς ἐν τῷ πάχει τῆς γῆς ἐν οἴκῳ Σοκχὼθ καὶ ἀναμέσον Σαρηδαθά.
\par }{\PP \VS{18}Καὶ ἐποίησε Σαλωμὼν πάντα τὰ σκεύη ταῦτα εἰς πλῆθος σφόδρα, ὅτι οὐκ ἐξέλιπεν ὁλκὴ τοῦ χαλκοῦ.
\VS{19}Καὶ ἐποίησε Σαλωμὼν πάντα τὰ σκεύη οἴκου Κυρίου, καὶ τὸ θυσιαστήριον τὸ χρυσοῦν, καὶ τὰς τραπέζας, καὶ ἐπʼ αὐτῶν ἄρτοι προθέσεως,
\VS{20}καὶ τὰς λυχνίας, καὶ τοὺς λύχνους τοῦ φωτὸς κατὰ τὸ κρίμα καὶ κατὰ πρόσωπον τοῦ δαβὶρ χρυσίου καθαροῦ,
\VS{21}καὶ λαβίδες αὐτῶν, καὶ οἱ λύχνοι αὐτῶν, καὶ τὰς φιάλας, καὶ τὰς θυΐσκας, καὶ τὰ πυρεῖα χρυσίου καθαροῦ,
\VS{22}καὶ ἡ θύρα τοῦ οἴκου ἡ ἐσωτέρα εἰς τὰ ἅγια τῶν ἁγίων, καὶ τὰς θύρας τοῦ οἴκου τοῦ ναοῦ χρυσᾶς· καὶ συνετελέσθη πᾶσα ἡ ἐργασία ἣν ἐποίησε Σαλωμὼν ἐν οἴκῳ Κυρίου.

\par }\Chap{5}{\PP \VerseOne{1}Καὶ εἰσήνεγκε Σαλωμὼν τὰ ἅγια Δαυὶδ τοῦ πατρὸς αὐτοῦ, τὸ ἀργύριον, καὶ τὸ χρυσίον, καὶ τὰ σκεύη, καὶ ἔδωκεν εἰς θησαυρὸν οἴκου Κυρίου.
\par }{\PP \VS{2}Τότε ἐξεκκλησίασε Σαλωμὼν πάντας τοὺς πρεσβυτέρους Ἰσραὴλ, καὶ πάντας τοὺς ἄρχοντας τῶν φυλῶν τοὺς ἡγουμένους πατριῶν υἱῶν Ἰσραὴλ εἰς Ἱερουσαλὴμ, τοῦ ἀνενέγκαι κιβωτὸν διαθήκης Κυρίου ἐκ πόλεως Δαυὶδ, αὕτη Σιών.
\VS{3}Καὶ ἐξεκκλησιάσθησαν πρὸς τὸν βασιλέα πᾶς Ἰσραὴλ ἐν τῇ ἑορτῇ, οὗτος ὁ μὴν ἕβδομος.
\VS{4}Καὶ ἦλθον πάντες οἱ πρεσβύτεροι Ἰσραὴλ, καὶ ἔλαβον πάντες οἱ Λευῖται τὴν κιβωτὸν,
\VS{5}καὶ τὴν σκηνὴν τοῦ μαρτυρίου, καὶ πάντα τὰ σκεύη τὰ ἅγια τὰ ἐν τῇ σκηνῇ, καὶ ἀνήνεγκαν αὐτὴν οἱ ἱερεῖς καὶ οἱ Λευῖται.
\VS{6}Καὶ ὁ βασιλεὺς Σαλωμὼν, καὶ πᾶσα συναγωγὴ Ἰσραὴλ, καὶ οἱ φοβούμενοι, καὶ οἱ ἐπισυνηγμένοι αὐτῶν ἔμπροσθεν τῆς κιβωτοῦ θύοντες μόσχους καὶ πρόβατα, οἳ οὐκ ἀριθμηθήσονται καὶ οἳ οὐ λογισθήσονται ἀπὸ τοῦ πλήθους.
\VS{7}Καὶ εἰσήνεγκαν οἱ ἱερεῖς τὴν κιβωτὸν διαθήκης Κυρίου εἰς τὸν τόπον αὐτῆς, εἰς τὸ δαβὶρ τοῦ οἴκου εἰς τὰ ἅγια τῶν ἁγίων, ὑποκάτω τῶν πτερύγων τῶν χερουβίμ.
\VS{8}Καὶ ἦν τὰ χερουβὶμ διαπεπετακότα τὰς πτέρυγας αὐτῶν ἐπὶ τὸν τόπον τῆς κιβωτοῦ, καὶ συνεκάλυπτε τὰ χερουβὶμ ἐπὶ τὴν κιβωτὸν, καὶ ἐπὶ τοὺς ἀναφορεῖς αὐτῆς ἐπάνωθεν,
\VS{9}καὶ ὑπερεῖχον οἱ ἀναφορεῖς, καὶ ἐβλέποντο αἱ κεφαλαὶ τῶν ἀναφορέων ἐκ τῶν ἁγίων εἰς πρόσωπον τοῦ δαβὶρ, οὐκ ἐβλέποντο ἔξω, καὶ ἦσαν ἐκεῖ ἕως τῆς ἡμέρας ταύτης.
\VS{10}Οὐκ ἦν ἐν τῇ κιβωτῷ πλὴν δύο πλάκες ἃς ἔθηκε Μωυσῆς ἐν Χωρὴβ, ἃ διέθετο Κύριος μετὰ τῶν υἱῶν Ἰσραὴλ, ἐν τῷ ἐξελθεῖν αὐτοὺς ἐκ γῆς Αἰγύπτου.
\par }{\PP \VS{11}Καὶ ἐγένετο ἐν τῷ ἐξελθεῖν τοὺς ἱερεῖς ἐκ τῶν ἁγίων, ὅτι πάντες οἱ ἱερεῖς οἱ εὑρεθέντες ἡγιάσθησαν, οὐκ ἦσαν διατεταγμένοι κατʼ ἐφημερίαν.
\VS{12}Καὶ οἱ Λευῖται οἱ ψαλτῳδοὶ πάντες τοῖς υἱοῖς Ἀσὰφ τῷ Αἰμὰν τῷ Ἰδιθοὺν καὶ τοῖς υἱοῖς αὐτοῦ, καὶ τοῖς ἀδελφοῖς αὐτοῦ τῶν ἐνδεδυμένων στολὰς βασσίνας ἐν κυμβάλοις καὶ ἐν νάβλαις καὶ ἐν κινύραις, ἑστηκότες κατέναντι τοῦ θυσιαστηρίου, καὶ μετʼ αὐτῶν ἱερεῖς ἑκατὸν εἴκοσι σαλπίζοντες ταῖς σάλπιγξι.
\VS{13}Καὶ ἐγένετο μία φωνὴ ἐν τῷ σαλπίζειν καὶ ἐν τῷ ψαλτῳδεῖν καὶ ἐν τῷ ἀναφωνεῖν φωνῇ μιᾷ τοῦ ἐξομολογεῖσθαι καὶ αἰνεῖν τῷ Κυρίῳ· καὶ ὡς ὕψωσαν φωνὴν ἐν σάλπιγξι, καὶ ἐν κυμβάλοις, καὶ ἐν ὀργάνοις τῶν ᾠδῶν, καὶ ἔλεγον, ἐξομολογεῖσθε τῷ Κυρίῳ, ὅτι ἀγαθὸν, ὅτι εἰς τὸν αἰῶνα τὸ ἔλεος αὐτοῦ· καὶ ὁ οἶκος ἐνεπλήσθη νεφέλης δόξης Κυρίου.
\VS{14}Καὶ οὐκ ἠδύναντο οἱ ἱερεῖς τοῦ στῆναι λειτουργεῖν ἀπὸ προσώπου τῆς νεφέλης, ὅτι ἐνέπλησε δόξα Κυρίου τὸν οἶκον τοῦ Θεοῦ.

\par }\Chap{6}{\PP \VerseOne{1}Τότε εἰπε Σαλωμὼν, Κύριος εἶπε τοῦ κατασκηνῶσαι ἐν γνόφῳ,
\VS{2}καὶ ἐγὼ ᾠκοδόμηκα οἶκον τῷ ὀνόματί σου ἅγιόν σοι καὶ ἕτοιμον τοῦ κατασκηνῶσαι εἰς τοὺς αἰῶνας.
\par }{\PP \VS{3}Καὶ ἐπέστρεψεν ὁ βασιλεὺς τὸ πρόσωπον αὐτοῦ, καὶ εὐλόγησε τὴν πᾶσαν ἐκκλησίαν Ἰσραὴλ, καὶ πᾶσα ἡ ἐκκλησία Ἰσραὴλ παρειστήκει,
\VS{4}καὶ εἶπεν, εὐλογητὸς Κύριος ὁ Θεὸς Ἰσραὴλ, ὡς ἐλάλησεν ἐν στόματι αὐτοῦ πρὸς Δαυὶδ τὸν πατέρα μου, καὶ ἐν χερσὶν αὐτοῦ ἐπλήρωσε, λέγων,
\VS{5}ἀπὸ τῆς ἡμέρας ἧς ἀνήγαγον τὸν λαόν μου ἐκ γῆς Αἰγύπτου, οὐκ ἐξελεξάμην ἐν πόλει ἀπὸ πασῶν φυλῶν Ἰσραὴλ, τοῦ οἰκοδομῆσαι οἶκον τοῦ εἶναι τὸ ὄνομά μου ἐκεῖ, καὶ οὐκ ἐξελεξάμην ἐν ἀνδρὶ τοῦ εἶναι εἰς ἡγούμενον ἐπὶ τὸν λαόν μου Ἰσραήλ·
\VS{6}Καὶ ἐξελεξάμην τὴν Ἱερουσαλὴμ γενέσθαι τὸ ὄνομά μου ἐκεῖ, καὶ ἐξελεξάμην ἐν Δαυὶδ τοῦ εἶναι ἐπὶ τὸν λαόν μου Ἰσραήλ.
\VS{7}Καὶ ἐγένετο ἐπὶ καρδίαν Δαυὶδ τοῦ πατρός μου, τοῦ οἰκοδομῆσαι οἶκον τῷ ὀνόματι Κυρίου Θεοῦ Ἰσραήλ.
\VS{8}Καὶ εἶπε Κύριος πρὸς Δαυὶδ πατέρα μου, διότι ἐγένετο ἐπὶ καρδίαν σου τοῦ οἰκοδομῆσαι οἶκον τῷ ὀνόματί μου, καλῶς ἐποίησας, ὅτι ἐγένετο ἐπὶ τὴν καρδίαν σου.
\VS{9}Πλὴν σὺ οὐκ οἰκοδομήσεις τὸν οἶκον, ὅτι ὁ υἱός σου ὃς ἐξελεύσεται ἐκ τῆς ὀσφύος σου, οὗτος οἰκοδομήσει τὸν οἶκον τῷ ὀνόματί μου.
\VS{10}Καὶ ἀνέστησε Κύριος τὸν λόγον τοῦτον, ὃν ἐλάλησε, καὶ ἐγενήθην ἀντὶ Δαυὶδ πατρός μου, καὶ ἐκάθισα ἐπὶ τὸν θρόνον Ἰσραὴλ, καθὼς ἐλάλησε Κύριος· καὶ ᾠκοδόμησα τὸν οἶκον τῷ ὀνόματι Κυρίου Θεοῦ Ἰσραὴλ,
\VS{11}καὶ ἔθηκα ἐκεῖ τὴν κιβωτὸν ἐν ᾗ ἐκεῖ διαθήκη Κυρίου, ἣν διέθετο τῷ Ἰσραήλ.
\par }{\PP \VS{12}Καὶ ἔστη κατέναντι τοῦ θυσιαστηρίου Κυρίου ἔναντι πάσης ἐκκλησίας Ἰσραὴλ, καὶ διεπέτασε τὰς χεῖρας αὐτοῦ·
\VS{13}Ὅτι ἐποίησε Σαλωμὼν βάσιν χαλκῆν, καὶ ἔθηκεν αὐτὴν ἐν μέσῳ τῆς αὐλῆς τοῦ ἱεροῦ, πέντε πήχεων τὸ μῆκος αὐτῆς, καὶ πέντε πήχεων τὸ εὖρος αὐτῆς, καὶ τριῶν πήχεων τὸ ὕψος αὐτῆς· καὶ ἔστη ἐπʼ αὐτῆς, καὶ ἔπεσεν ἐπὶ τὰ γόνατα ἔναντι πάσης ἐκκλησίας Ἰσραὴλ, καὶ διεπέτασε τὰς χεῖρας αὐτοῦ εἰς τὸν οὐρανὸν,
\VS{14}καὶ εἶπε,
\par }{\PP Κύριε ὁ Θεὸς Ἰσραὴλ, οὐκ ἔστιν ὅμοιός σοι Θεὸς ἐν οὐρανῷ καὶ ἐπὶ τῆς γῆς, φυλάσσων τὴν διαθήκην καὶ τὸ ἔλεος τοῖς παισὶ σου τοῖς πορευομένοις ἐναντίον σου ἐν ὅλῃ καρδίᾳ·
\VS{15}Ἃ ἐφύλαξας τῷ παιδί σου Δαυὶδ τῷ πατρί μου, ἃ ἐλάλησας αὐτῷ, λέγων· καὶ ἐλάλησας ἐν στόματί σου, καὶ ἐν χερσί σου ἐπλήρωσας, ὡς ἡ ἡμέρα αὕτη.
\VS{16}Καὶ νῦν, Κύριε ὁ Θεὸς Ἰσραὴλ, φύλαξον τῷ παιδί σου τῷ Δαυὶδ τῷ πατρί μου ἃ ἐλάλησας αὐτῷ, λέγων, οὐκ ἐκλείψει σοι ἀνὴρ ἀπὸ προσώπου μου καθήμενος ἐπὶ θρόνου Ἰσραὴλ, πλὴν ἐὰν φυλάξωσιν οἱ υἱοί σου τὴν ὁδὸν αὐτῶν τοῦ πορεύεσθαι ἐν τῷ νόμῳ μου, ὡς ἐπορεύθης ἐναντίον μου.
\VS{17}Καὶ νῦν, Κύριε ὁ Θεὸς Ἰσραὴλ, πιστωθήτω δὴ τὸ ῥῆμά σου ὃ ἐλάλησας τῷ παιδί σου τῷ Δαυίδ.
\par }{\PP \VS{18}Ὅτι εἰ ἀληθῶς κατοικήσει Θεὸς μετὰ ἀνθρώπων ἐπὶ τῆς γῆς; εἰ ὁ οὐρανὸς καὶ ὁ οὐρανὸς τοῦ οὐρανοῦ οὐκ ἀρκέσουσί σοι, καὶ τίς ὁ οἶκος οὗτος ὃν ᾠκοδόμησα;
\VS{19}Καὶ ἐπιβλέψῃ ἐπὶ τὴν προσευχὴν παιδός σου καὶ ἐπὶ τὴν δέησίν μου, Κύριε ὁ Θεὸς, τοῦ ἐπακοῦσαι τῆς δεήσεως καὶ τῆς προσευχῆς ἧς ὁ παῖς σου προσεύχεται ἐναντίον σου σήμερον,
\VS{20}τοῦ εἶναι ὀφθαλμούς σου ἀνεῳγμένους ἐπὶ τὸν οἶκον τοῦτον ἡμέρας καὶ νυκτὸς εἰς τὸν τόπον τοῦτον, ὃν εἶπας ἐπικληθῆναι τὸ ὄνομά σου ἐκεῖ, τοῦ ἀκοῦσαι τῆς προσευχῆς ἧς προσεύχεται ὁ παῖς σου εἰς τὸν τόπον τοῦτον·
\VS{21}Καὶ ἀκούσῃ τῆς δεήσεως τοῦ παιδός σου, καὶ λαοῦ σου Ἰσραὴλ, ἃ ἂν προσεύξωνται εἰς τὸν τόπον τοῦτον· καὶ σὺ εἰσακούσῃ ἐν τῷ τόπῳ τῆς κατοικήσεώς σου ἐκ τοῦ οὐρανοῦ, καὶ ἀκούσῃ καὶ ἵλεως ἔσῃ.
\par }{\PP \VS{22}Ἐὰν ἁμάρτῃ ἀνὴρ τῷ πλησίον αὐτοῦ καὶ λάβῃ ἐπʼ αὐτὸν ἀρὰν τοῦ ἀρᾶσθαι αὐτὸν, καὶ ἔλθῃ καὶ ἀράσηται κατέναντι τοῦ θυσιαστηρίου ἐν τῷ οἴκῳ τούτῳ,
\VS{23}καὶ σὺ εἰσακούσῃ ἐκ τοῦ οὐρανοῦ, καὶ ποιήσεις, καὶ κρινεῖς τοὺς δούλους σου, τοῦ ἀποδοῦναι τῷ ἀνόμῳ, καὶ ἀποδοῦναι ὁδοὺς αὐτοῦ εἰς κεφαλὴν αὐτοῦ, καὶ τοῦ δικαιῶσαι δίκαιον, τοῦ ἀποδοῦναι αὐτῷ κατὰ τὴν δικαιοσύνην αὐτοῦ.
\par }{\PP \VS{24}Καὶ ἐὰν θραυσθῇ ὁ λαός σου Ἰσραὴλ κατέναντι τοῦ ἐχθροῦ, ἐὰν ἁμάρτωσί σοι, καὶ ἐπιστρέψωσι καὶ ἐξομολογήσωνται τῷ ὀνόματί σου, καὶ προσεύξωνται καὶ δεηθῶσιν ἐναντίον σου ἐν τῷ οἴκῳ τούτῳ,
\VS{25}καὶ σὺ εἰσακούσῃ ἐκ τοῦ οὐρανοῦ, καὶ ἵλεως ἔσῃ ταῖς ἁμαρτίαις λαοῦ σου Ἰσραὴλ, καὶ ἀποστρέψεις αὐτοὺς εἰς τὴν γῆν ἣν ἔδωκας αὐτοῖς καὶ τοῖς πατράσιν αὐτῶν.
\par }{\PP \VS{26}Ἐν τῷ συσχεθῆναι τὸν οὐρανὸν καὶ μὴ γενέσθαι ὑετὸν, ὅτι ἁμαρτήσονταί σοι, καὶ προσεύξονται εἰς τὸν τόπον τοῦτον, καὶ αἰνέσουσι τὸ ὄνομά σου, καὶ ἀπὸ τῶν ἁμαρτιῶν αὐτῶν ἐπιστρέψουσιν, ὅτι ταπεινώσεις αὐτοὺς,
\VS{27}καὶ σὺ εἰσακούσῃ ἐκ τοῦ οὐρανοῦ, καὶ ἵλεως ἔσῃ ταῖς ἁμαρτίαις τῶν παίδων καὶ τοῦ λαοῦ σου Ἰσραὴλ, ὅτι δηλώσεις αὐτοῖς τὴν ὁδὸν τὴν ἀγαθὴν, ἐν ᾗ πορεύσονται ἐν αὐτῇ, καὶ δώσεις ὑετὸν ἐπὶ τὴν γῆν σου, ἣν ἔδωκας τῷ λαῷ σου εἰς κληρονομίαν.
\par }{\PP \VS{28}Λιμὸς ἐὰν γένηται ἐπὶ τῆς γῆς, θάνατος ἐὰν γένηται, ἀνεμοφθορία καὶ ἴκτερος, ἀκρὶς καὶ βροῦχος ἐὰν γένηται, καὶ ἐὰν θλίψῃ αὐτὸν ὁ ἐχθρὸς κατέναντι τῶν πόλεων αὐτῶν, κατὰ πᾶσαν πληγὴν καὶ πάντα πόνον,
\VS{29}καὶ πᾶσα προσευχὴ, καὶ πᾶσα δέησις ἣ ἐὰν γένηται παντὶ ἀνθρώπῳ καὶ παντὶ λαῷ σου Ἰσραὴλ, ἐὰν γνῷ ἄνθρωπος τὴν ἁφὴν αὐτοῦ καὶ τὴν μαλακίαν αὐτοῦ, καὶ διαπετάσῃ τὰς χεῖρας αὐτοῦ εἰς τὸν οἶκον τοῦτον,
\VS{30}καὶ σὺ εἰσακούσῃ ἐκ τοῦ οὐρανοῦ ἐξ ἑτοίμου κατοικητηρίου σου, καὶ ἱλάσῃ, καὶ δώσεις ἀνδρὶ κατὰ τὰς ὁδοὺς αὐτοῦ, ὡς ἂν γνῷς τὴν καρδίαν αὐτοῦ, ὅτι μόνος γινώσκεις τὴν καρδίαν υἱῶν ἀνθρώπων,
\VS{31}ὅπως φοβῶνται πάσας ὁδούς σου πάσας τὰς ἡμέρας ἃς αὐτοὶ ζῶσιν ἐπὶ πρόσωπον τῆς γῆς, ἧς ἔδωκας τοῖς πατράσιν ἡμῶν.
\par }{\PP \VS{32}Καὶ πᾶς ἀλλότριος ὃς οὐκ ἐκ τοῦ λαοῦ σου Ἰσραήλ ἐστιν αὐτὸς, καὶ ἔλθῃ ἐκ γῆς μακρόθεν διὰ τὸ ὄνομά σου τὸ μέγα, καὶ τὴν χεῖρά σου τὴν κραταιὰν, καὶ τὸν βραχίονά σου τὸν ὑψηλὸν, καὶ ἔλθωσι καὶ προσεύξωνται εἰς τὸν τόπον τοῦτον,
\VS{33}καὶ σὺ εἰσακούσῃ ἐκ τοῦ οὐρανοῦ ἐξ ἑτοίμου κατοικητηρίου σου, καὶ ποιήσεις κατὰ πάντα ὅσα ἂν ἐπικαλέσηταί σε ὁ ἀλλότριος, ὅπως γνῶσι πάντες οἱ λαοὶ τῆς γῆς τὸ ὄνομά σου, καὶ τοῦ φοβεῖσθαί σε, ὡς ὁ λαός σου Ἰσραὴλ, καὶ τοῦ γνῶναι ὅτι τὸ ὄνομά σου ἐπικέκληται ἐπὶ τὸν οἶκον τοῦτον, ὃν ᾠκοδόμησα.
\par }{\PP \VS{34}Ἐὰν δὲ ἐξέλθῃ ὁ λαός σου εἰς πόλεμον ἐπὶ τοὺς ἐχθροὺς αὐτοῦ ἐν ὁδῷ ᾗ ἀποστελεῖς αὐτούς, καὶ προσεύξωνται πρὸς σὲ κατὰ τὴν ὁδὸν τῆς πόλεως ταύτης ἣν ἐξελέξω ἐν αὐτῇ, καὶ οἴκου οὗ ᾠκοδόμηκα τῷ ὀνόματί σου,
\VS{35}καὶ ἀκούσῃ ἐκ τοῦ οὐρανοῦ τῆς προσευχῆς αὐτῶν καὶ τῆς δεήσεως αὐτῶν, καὶ ποιήσεις τὸ δικαίωμα αὐτῶν.
\par }{\PP \VS{36}Ὅτι ἁμαρτήσονταί σοι, ὅτι οὐκ ἔσται ἄνθρωπος ὃς οὐχ ἁμαρτήσεται, καὶ πατάξεις αὐτοὺς καὶ παραδώσεις αὐτοὺς κατὰ πρόσωπον ἐχθρῶν καὶ αἰχμαλωτεύσουσιν αὐτοὺς οἱ αἰχμαλωτεύοντες αὐτοὺς εἰς γῆν ἐχθρῶν εἰς γῆν μακρὰν ἢ ἐγγὺς,
\VS{37}καὶ ἐπιστρέψωσι καρδίαν αὐτῶν ἐν τῇ γῇ αὐτῶν οὗ μετήχθησαν ἐκεῖ, καί γε ἐπιστρέψωσι καὶ δεηθῶσί σου ἐν τῇ αἰχμαλωσίᾳ αὐτῶν, λέγοντες, ἡμάρτομεν, ἠνομήσαμεν, ἠδικήσαμεν,
\VS{38}καὶ ἐπιστρέψωσι πρὸς σὲ ἐν ὅλῃ καρδίᾳ καὶ ἐν ὅλῃ ψυχῇ αὐτῶν ἐν γῇ αἰχμαλωτευσάντων αὐτοὺς, ὅπου ᾐχμαλώτευσαν αὐτοὺς, καὶ προσεύξωνται ὁδὸν γῆς αὐτῶν ἧς ἔδωκας τοῖς πατράσιν αὐτῶν, καὶ τῆς πόλεως ἧς ἐξελέξω, καὶ τοῦ οἴκου οὗ ᾠκοδόμησα τῷ ὀνόματί σου,
\VS{39}καὶ ἀκούσῃ ἐκ τοῦ οὐρανοῦ ἐξ ἑτοίμου κατοικητηρίου σου τῆς προσευχῆς αὐτῶν καὶ τῆς δεήσεως αὐτῶν, καὶ ποιήσεις κρίματα, καὶ ἵλεως ἔσῃ τῷ λαῷ τῷ ἁμαρτῶντί σοι.
\par }{\PP \VS{40}Καὶ νῦν Κύριε ἔστωσαν δὴ οἱ ὀφθαλμοί σου ἀνεῳγμένοι, καὶ τὰ ὦτά σου ἐπήκοα εἰς τὴν δέησιν τοῦ τόπου τούτου.
\VS{41}Καὶ νῦν ἀνάστηθι Κύριε ὁ Θεὸς εἰς τὴν κατάπαυσίν σου, σὺ καὶ ἡ κιβωτὸς τῆς ἰσχύος σου· ἱερεῖς σου Κύριε ὁ Θεὸς ἐνδύσαιντο σωτηρίαν, καὶ οἱ υἱοί σου εὐφρανθήτωσαν ἐν ἀγαθοῖς.
\VS{42}Κύριε ὁ Θεὸς, μὴ ἀποστρέψῃς τὸ πρόσωπον τοῦ χριστοῦ σου, μνήσθητι τὰ ἐλέη Δαυὶδ τοῦ δούλου σου.

\par }\Chap{7}{\PP \VerseOne{1}Καὶ ὡς συνετέλεσε Σαλωμὼν προσευχόμενος, καὶ τὸ πῦρ κατέβη ἐκ τοῦ οὐρανοῦ, καὶ κατέφαγε τὰ ὁλοκαυτώματα καὶ τὰς θυσίας, καὶ δόξα Κυρίου ἔπλησε τὸν οἶκον.
\VS{2}Καὶ οὐκ ἠδύναντο οἱ ἱερεῖς εἰσελθεῖν εἰς τὸν οἶκον Κυρίου ἐν τῷ καιρῷ ἐκείνῳ ὅτι ἔπλησε δόξα Κυρίου τὸν οἶκον.
\VS{3}Καὶ πάντες οἱ υἱοὶ Ισραὴλ ἑώρων καταβαῖνον τὸ πῦρ, καὶ ἡ δόξα Κυρίου ἐπὶ τὸν οἶκον· καὶ ἔπεσον ἐπὶ πρόσωπον ἐπὶ τὴν γῆν ἐπὶ τὸ λιθόστρωτον, καὶ προσεκύνησαν καὶ ᾔνουν τῷ Κυρίῳ, ὅτι ἀγαθὸν, ὅτι εἰς τὸν αἰῶνα τὸ ἔλεος αὐτοῦ.
\par }{\PP \VS{4}Καὶ ὁ βασιλεὺς καὶ πᾶς ὁ λαὸς θύοντες θύματα ἔναντι Κυρίου.
\VS{5}Καὶ ἐθυσίασεν ὁ βασιλεὺς Σαλωμὼν τὴν θυσίαν μόσχων εἴκοσι καὶ δύο χιλιάδας, βοσκημάτων ἑκατὸν καὶ εἴκοσι χιλιάδας, καὶ ἐνεκαίνισε τὸν οἶκον τοὔ Θεοῦ ὁ βασιλεὺς καὶ πᾶς ὁ λαός.
\VS{6}Καὶ οἱ ἱερεῖς ἐπὶ τὰς φυλακὰς αὐτῶν ἑστηκότως, καὶ οἱ Λευῖται ἐν ὀργάνοις ᾠδῶν Κυρίου τοῦ Δαυὶδ τοῦ βασιλέως, τοῦ ἐξομολογεῖσθαι ἔναντι Κυρίου, ὅτι εἰς τὸν αἰῶνα τὸ ἔλεος αὐτοῦ, ἐν ὕμνοις Δαυὶδ διὰ χειρὸς αὐτῶν· καὶ οἱ ἱερεῖς σαλπίζοντες ταῖς σάλπιγξιν ἐναντίον αὐτῶν, καὶ πᾶς Ἰσραὴλ ἑστηκώς.
\VS{7}Καὶ ἡγίασε Σαλωμὼν τὸ μέσον τῆς αὐλῆς τῆς ἐν οἴκῳ Κυρίου, ὅτι ἐποίησεν ἐκεῖ τὰ ὁλοκαυτώματα καὶ τὰ στέατα τῶν σωτηρίων, ὅτι τὸ θυσιαστήριον τὸ χαλκοῦν ὃ ἐποίησε Σαλωμὼν, οὐκ ἐξεποίει δέξασθαι τὰ ὁλοκαυτώματα καὶ τὰ μαναὰ καὶ τὰ στέατα.
\par }{\PP \VS{8}Καὶ ἐποίησε Σαλωμὼν τὴν ἑορτὴν ἐν τῷ καιρῷ ἐκείνῳ ἑπτὰ ἡμέρας, καὶ πᾶς Ἰσραὴλ μετʼ αὐτοῦ, ἐκκλησία μεγάλη σφόδρα ἀπὸ εἰσόδου Αἰμὰθ καὶ ἕως χειμάῤῥου Αἰγύπτου.
\VS{9}Καὶ ἐποίησεν ἐν τῇ ἡμέρᾳ τῇ ὀγδόῃ ἐξόδιον, ὅτι ἐγκαινισμὸν τοῦ θυσιαστηρίου ἐποίησεν ἑπτὰ ἡμέρας ἑορτήν.
\VS{10}Καὶ ἐν τῇ τρίτῃ καὶ εἰκοστῇ τοῦ μηνὸς τοῦ ἑβδόμου ἀπέστειλε τὸν λαὸν εἰς τὰ σκηνώματα αὐτῶν εὐφραινομένους, καὶ ἀγαθῇ καρδίᾳ ἐπὶ τοῖς ἀγαθοῖς οἷς ἐποίησε Κύριος τῷ Δαυὶδ, καὶ τῷ Σαλωμῶντι, καὶ τῷ Ἰσραὴλ λαῷ αὐτοῦ.
\par }{\PP \VS{11}Καὶ συνετέλεσε Σαλωμὼν τὸν οἶκον Κυρίου, καὶ τὸν οἶκον τοῦ βασιλέως· καὶ πάντα ὅσα ἠθέλησεν ἐν τῇ ψυχῇ Σαλωμὼν τοῦ ποιῆσαι ἐν οἴκῳ Κυρίου καὶ ἐν οἴκῳ αὐτοῦ, εὐωδώθη.
\par }{\PP \VS{12}Καὶ ὤφθη Κύριος τῷ Σαλωμὼν τὴν νύκτα, καὶ εἶπεν αὐτῷ, ἤκουσα τῆν προσευχῆς σου, καὶ ἐξελεξάμην ἐν τῷ τόπῳ τούτῳ ἐμαυτῷ εἰς οἶκον θυσίας.
\VS{13}Ἐὰν συσχῶ τὸν οὐρανὸν καὶ μὴ γένηται ὑετὸς, καὶ ἐὰν ἐντείλωμαι τῇ ἀκρίδι καταφαγεῖν τὸ ξύλον, καὶ ἐὰν ἀποστείλω θάνατον ἐν τῷ λαῷ μου,
\VS{14}καὶ ἐὰν ἐντραπῇ ὁ λαός μου ἐφʼ οὓς ἐπικέκληται τὸ ὄνομά μου ἐπʼ αὐτοὺς, καὶ προσεύξωνται καὶ ζητήσωσι τὸ πρόσωπόν μου, καὶ ἀποστρέψωσιν ἀπὸ τῶν ὁδῶν αὐτῶν τῶν πονηρῶν, καὶ ἐγὼ εἰσακούσομαι ἐκ τοῦ οὐρανοῦ, καὶ ἵλεως ἔσομαι ταῖς ἁμαρτίαις αὐτῶν, καὶ ἰάσομαι τὴν γῆν αὐτῶν.
\VS{15}Καὶ νῦν οἱ ὀφθαλμοί μου ἔσονται ἀνεῳγμένοι, καὶ τὰ ὦτά μου ἐπήκοα τῇ προσευχῇ τοῦ τόπου τούτου.
\VS{16}Καὶ νῦν ἐξελεξάμην καὶ ἡγίακα τὸν οἶκον τοῦτον, τοῦ εἶναι ὄνομά μου ἐκεῖ ἕως αἰῶνος, καὶ ἔσονται οἱ ὀφθαλμοί μου καὶ ἡ καρδία μου ἐκεῖ πάσας τὰς ἡμέρας.
\par }{\PP \VS{17}Καὶ σὺ ἐὰν πορευθῇς ἐναντίον μου ὡς Δαυὶδ ὁ πατήρ σου, καὶ ποιήσῃς κατὰ πάντα ἃ ἐνετειλάμην σοι, καὶ τὰ προστάγματά μου καὶ τὰ κρίματά μου φυλάξῃ,
\VS{18}καὶ ἀναστήσω τὸν θρόνον τῆς βασιλείας σου ὡς διεθέμην Δαυὶδ τῷ πατρί σου, λέγων, οὐκ ἐξαρθήσεταί σοι ἡγούμενος ἀνὴρ ἐν Ἰσραήλ.
\par }{\PP \VS{19}Καὶ ἐὰν ἀποστρέψητε ὑμεῖς, καὶ ἐγκαταλείπητε τὰ προστάγματά μου καὶ τὰς ἐντολάς μου ἃς ἔδωκα ἐναντίον ὑμῶν, καὶ πορευθῆτε καὶ λατρεύσητε θεοῖς ἑτέροις καὶ προσκυνήσητε αὐτοῖς,
\VS{20}καὶ ἐξαρῶ ὑμᾶς ἀπὸ τῆς γῆς ἧς ἔδωκα αὐτοῖς· καὶ τὸν οἶκον τοῦτον ὃν ἡγίασα τῷ ὀνόματί μου ἀποστρέψω ἐκ προσώπου μου, καὶ δώσω αὐτὸν εἰς παραβολὴν καὶ εἰς διήγημα ἐν πᾶσι τοῖς ἔθνεσι.
\VS{21}Καὶ ὁ οἶκος οὗτος ὁ ὑψηλὸς πᾶς ὁ διαπορευόμενος αὐτὸν ἐκστήσεται, καὶ ἐρεῖ, Χάριν τίνος ἐποίησε Κύριος τῇ γῇ ταύτῃ καὶ τῷ οἴκῳ τούτῳ;
\VS{22}Καὶ ἐροῦσι, διότι ἐγκατέλιπον Κυρίον τὸν Θεὸν τῶν πατέρων αὐτῶν, τὸν ἐξαγαγόντα αὐτοὺς ἐκ γῆς Αἰγύπτου, καὶ ἀντελάβοντο θεῶν ἑτέρων, καὶ προσεκύνησαν αὐτοῖς, καὶ ἐδούλευσαν αὐτοῖς, καὶ διὰ τοῦτο ἐπήγαγεν ἐπʼ αὐτοὺς πᾶσαν τὴν κακίαν ταύτην.

\par }\Chap{8}{\PP \VerseOne{1}Καὶ ἐγένετο μετὰ εἴκοσι ἔτη ἐν οἷς ᾠκοδόμησε Σαλωμὼν τὸν οἶκον Κυρίου, καὶ τὸν οἶκον αὑτοῦ,
\VS{2}καὶ τὰς πόλεις ἃς ἔδωκε Χιρὰμ τῷ Σαλωμὼν, ᾠκοδόμησεν αὐτὰς Σαλωμὼν, καὶ κατῴκισεν ἐκεῖ τοὺς υἱοὺς Ἰσραήλ.
\par }{\PP \VS{3}Καὶ ἦλθε Σαλωμὼν εἰς Βαισωβὰ, καὶ κατίσχυσεν αὐτήν.
\VS{4}Καὶ ᾠκοδόμησε τὴν Θοεδμὸρ ἐν τῇ ἐρήμῳ, καὶ πάσας τὰς πόλεις τὰς ὀχυρὰς ἃς ᾠκοδόμησεν ἐν Ἠμάθ.
\VS{5}Καὶ ᾠκοδόμησε τὴν Βαιθωρὼν τὴν ἄνω καὶ τὴν Βαιθωρὼν τὴν κάτω, πόλεις ὀχυράς· τείχη, πύλαι, καὶ μοχλοί·
\VS{6}Καὶ τὴν Βαλαὰθ, καὶ πάσας τὰς πόλεις τὰς ὀχυρὰς αἳ ἦσαν τῷ Σαλωμὼν, καὶ πάσας τὰς πόλεις τῶν ἁρμάτων, καὶ τὰς πόλεις τῶν ἱππέων· καὶ ὅσα ἐπεθύμησε Σαλωμὼν κατὰ τὴν ἐπιθυμίαν τοῦ οἰκοδομῆσαι ἐν Ἱερουσαλὴμ, καὶ ἐν τῷ Λιβάνῳ, καὶ ἐν πάσῃ τῇ βασιλείᾳ αὐτοῦ.
\par }{\PP \VS{7}Πᾶς ὁ λαὸς ὁ καταλειφθεὶς ἀπὸ τοῦ Χετταίου, καὶ τοῦ Ἀμοῥῥαίου, καὶ τοῦ Φερεζαίου, καὶ τοῦ Εὐαίου, καὶ τοῦ Ἰεβουσαίου, οἳ οὐκ εἰσὶν ἐκ τοῦ Ἰσραὴλ,
\VS{8}ἀλλʼ ἦσαν ἐκ τῶν υἱῶν αὐτῶν τῶν καταλειφθέντων μετʼ αὐτοὺς ἐν τῇ γῇ, οὓς οὐκ ἐξωλόθρευσαν οἱ υἱοὶ Ἰσραὴλ, καὶ ἀνήγαγεν αὐτοὺς Σαλωμὼν εἰς φόρον ἕως τῆς ἡμέρας ταύτης.
\VS{9}Καὶ ἐκ τῶν υἱῶν Ἰσραὴλ οὐκ ἔδωκε Σαλωμὼν εἰς παῖδας τῇ βασιλείᾳ αὐτοῦ· ὅτι ἰδοὺ ἄνδρες πολεμισταὶ καὶ ἄρχοντες, καὶ οἱ δυνατοὶ καὶ ἄρχοντες ἁρμάτων καὶ ἱππέων.
\VS{10}Καὶ οὗτοι ἄρχοντες τῶν προστατῶν βασιλέως Σαλωμὼν, πεντήκοντα καὶ διακόσιοι ἐργοδιωκτοῦντες ἐν τῷ λαῷ.
\par }{\PP \VS{11}Καὶ τὴν θυγατέρα Φαραὼ ἀνήγαγε Σαλωμὼν ἐκ πόλεως Δαυὶδ εἰς τὸν οἶκον ὃν ᾠκοδόμησεν αὐτῇ, ὅτι εἶπεν, οὐ κατοικήσει ἡ γυνή μου ἐν πόλει Δαυὶδ τοῦ βασιλέως Ἰσραὴλ, ὅτι ἅγιός ἐστιν οὗ εἰσῆλθεν ἐκεῖ κιβωτὸς Κυρίου.
\par }{\PP \VS{12}Τότε ἀνήνεγκε Σαλωμὼν ὁλοκαυτώματα τῷ Κυρίῳ ἐπὶ τὸ θυσιαστήριον, ὃ ᾠκοδόμησε Κυρίῳ ἀπέναντι τοῦ ναοῦ
\VS{13}κατὰ τὸν λόγον ἡμέρας ἐν ἡμέρᾳ, τοῦ ἀναφέρειν κατὰ τὰς ἐντολὰς Μωυσῆ ἐν τοῖς σαββάτοις, καὶ ἐν τοῖς μησὶ, καὶ ἐν ταῖς ἑορταῖς, τρεῖς καιροὺς τοῦ ἐνιαυτοῦ, ἐν τῇ ἑορτῇ τῶν ἀζύμων, καὶ ἐν τῇ ἐορτῇ τῶν ἐβδομάδων, καὶ ἐν τῇ ἑορτῇ τῶν σκηνῶν.
\VS{14}Καὶ ἔστησε κατὰ τὴν κρίσιν Δαυὶδ τοῦ πατοὸς αὐτοῦ τὰς διαιρέσεις τῶν ἱερέων, καὶ κατὰ τὰς λειτουργίας αὐτῶν· καὶ οἱ Λευῖται ἐπὶ τὰς φυλακὰς αὐτῶν, τοῦ αἰνεῖν καὶ λειτουργεῖν κατέναντι τῶν ἱερέων κατὰ τὸν λόγον ἡμέρας ἐν τῇ ἡμερᾳ· καὶ οἱ πυλωροὶ κατὰ τὰς διαιρέσεις αὐτῶν εἰς πύλην καὶ πύλην, ὅτι οὕτως ἐντολαὶ Δαυὶδ ἀνθρώπου τοῦ Θεοῦ·
\VS{15}Οὐ παρῆλθον τὰς ἐντολὰς τοῦ βασιλέως περὶ τῶν ἱερέων καὶ τῶν Λευιτῶν εἰς πάντα λόγον, καὶ εἰς τοὺς θησαυρούς.
\VS{16}Καὶ ἡτοιμάσθη πᾶσα ἡ ἐργασία ἀφʼ ἧς ἡμέρας ἐθεμελιώθη ἕως οὗ ἐτελείωσε Σαλωμὼν τὸν οἶκον Κυρίου.
\par }{\PP \VS{17}Τότε ᾤχετο Σαλωμὼν εἰς Γασιὼν Γαβὲρ, καὶ εἰς τὴν Αἰλὰθ τὴν παραθαλασσίαν ἐν γῇ Ἰδουμαίᾳ.
\VS{18}Καὶ ἀπέστειλε Χιρὰμ ἐν χειρὶ παίδων αὐτοῦ πλοῖα καὶ παῖδας εἰδότας θάλασσαν, καὶ ᾤχοντο μετὰ τῶν παίδων Σαλωμὼν εἰς Σωφιρὰ, καὶ ἔλαβον ἐκεῖθεν τὰ τετρακόσια καὶ πεντήκοντα τάλαντα χρυσίου, καὶ ἦλθον πρὸς τὸν βασιλέα Σαλωμών.

\par }\Chap{9}{\PP \VerseOne{1}Καὶ βασίλισσα Σαβὰ ἤκουσε τὸ ὄνομα Σαλωμὼν, καὶ ἦλθε τοῦ πειρᾶσαι Σαλωμὼν ἐν αἰνίγμασιν εἰς Ἱερουσαλὴμ ἐν δυνάμει βαρείᾳ σφόδρα, καὶ κάμηλοι αἴρουσαι ἀρώματα εἰς πλῆθος, καὶ χρυσίον, καὶ λίθον τίμιον· καὶ ἦλθε πρὸς Σαλωμὼν, καὶ ἐλάλησε πρὸς αὐτὸν πάντα ὅσα ἦν ἐν τῇ ψυχῇ αὐτῆς.
\VS{2}Καὶ ἀνήγγειλεν αὐτῇ Σαλωμὼν πάντας τοὺς λόγους αὐτῆς, καὶ οὐ παρῆλθε λόγος ἀπὸ Σαλωμὼν ὃν οὐκ ἀπήγγειλεν αὐτῇ.
\par }{\PP \VS{3}Καὶ εἶδε βασίλισσα Σαβὰ τὴν σοφίαν Σαλωμὼν καὶ τὸν οἶκον ὃν ᾠκοδόμησε,
\VS{4}καὶ τὰ βρώματα τῶν τραπεζῶν, καὶ καθέδραν παίδων αὐτοῦ, καὶ στάσιν λειτουργῶν αὐτοῦ, καὶ ἱματισμὸν αὐτῶν, καὶ οἰνοχόους αὐτοῦ, καὶ στολισμὸν αὐτῶν, καὶ τὰ ὁλοκαυτώματα ἃ ἀνέφερεν ἐν οἴκῳ Κυρίου, καὶ ἐξ ἑαυτῆς ἐγένετο.
\VS{5}Καὶ εἶπε πρὸς τὸν βασιλέα, ἀληθινὸς ὁ λόγος ὃν ἤκουσα ἐν τῇ γῇ μου περὶ τῶν λόγων σου, καὶ περὶ τῆς σοφίας σου.
\VS{6}Καὶ οὐκ ἐπίστευσα τοῖς λόγοις ἕως οὗ ἦλθον καὶ εἶδον οἱ ὀφθαλμοί μου, καὶ ἰδοὺ οὐκ ἀπηγγέλη μοι ἥμισυ τοῦ πλήθους τῆς σοφίας σου· προσέθηκας ἐπὶ τὴν ἀκοὴν ἣν ἤκουσα.
\VS{7}Μακάριοι οἱ ἄνδρες σου, μακάριοι οἱ παῖδες οὗτοι οἱ παρεστηκότες σοι διαπαντὸς καὶ ἀκούοντες τὴς σοφίαν σου.
\VS{8}Ἔστω Κύριος ὁ Θεός σου εὐλογημένος, ὃς ἠθέλησεν ἐν σοὶ τοῦ δοῦναί σε ἐπὶ θρόνον αὐτοῦ εἰς βασιλέα Κυρίῳ Θεῷ σου· ἐν τῷ ἀγαπῆσαι Κύριον τὸν Θεόν σου τὸν Ἰσραὴλ τοῦ στῆσαι αὐτὸν εἰς αἰῶνα, καὶ ἔδωκέ σε ἐπʼ αὐτοὺς εἰς βασιλέα τοῦ ποιῆσαι κρίμα καὶ δικαιοσύνην.
\VS{9}Καὶ ἔδωκε τῷ βασιλεῖ ἑκατὸν εἴκοσι τάλαντα χρυσίου, καὶ ἀρώματα εἰς πλῆθος πολὺ, καὶ λίθον τίμιον· καὶ οὐκ ἦν κατὰ τὰ ἀρώματα ἐκεῖνα ἃ ἔδωκε βασίλισσα Σαβὰ τῷ βασιλεῖ Σαλωμών.
\par }{\PP \VS{10}Καὶ οἱ παῖδες Σαλωμὼν καὶ οἱ παῖδες Χιρὰμ ἔφερον χρυσίον τῷ Σαλωμὼν ἐκ Σουφὶρ, καὶ ξύλα πεύκινα, καὶ λίθον τίμιον.
\VS{11}Καὶ ἐποίησεν ὁ βασιλεὺς τὰ ξύλα τὰ πεύκινα ἀναβάσεις τῷ οἴκῳ Κυρίου, καὶ τῷ οἴκῳ τοῦ βασιλέως, καὶ κιθάρας καὶ νάβλας τοῖς ᾠδοῖς· καὶ οὐκ ὤφθησαν τοιαῦτα ἔμπροσθεν ἐν γῇ Ἰούδα.
\VS{12}Καὶ ὁ βασιλεὺς Σαλωμὼν ἔδωκε τῇ βασιλίσσῃ Σαβὰ πάντα τὰ θελήματα αὐτῆς ἃ ᾔτησεν, ἐκτὸς πάντων ὧν ἤνεγκε τῷ βασιλεῖ Σαλωμὼν· καὶ ἀπέστρεψεν εἰς τὴν γῆν αὐτῆς.
\par }{\PP \VS{13}Καὶ ἦν ὁ σταθμὸς τοῦ χρυσίου τοῦ ἐνεχθέντος τῷ Σαλωμὼν ἐν ἐνιαυτῷ ἑνὶ, ἑξακόσια ἑξηκονταὲξ τάλαντα χρυσίου,
\VS{14}πλὴν τῶν ἀνδρῶν τῶν ὑποτεταγμένων καὶ τῶν ἐμπορευομένων ὧν ἔφερον· καὶ πάντων τῶν βασιλέων τῆς Ἀραβίας καὶ σατραπῶν τῆς γῆς, πάντες ἔφερον χρυσίον καὶ ἀργύριον τῷ βασιλεῖ Σαλωμών.
\VS{15}Καὶ ἐποίησεν ὁ βασιλεὺς Σαλωμὼν διακοσίους θυρεοὺς χρυσοῦς ἐλατοὺς, ἑξακόσιοι χρυσοῖ καθαροὶ ἐπῆσαν ἐπὶ τὸν ἕνα θυρεόν.
\VS{16}Καὶ τριακοσίας ἀσπίδας ἐλατὰς χρυσᾶς, τριακοσίων χρυσῶν ἀνεφέρετο ἐπὶ τὴν ἀσπίδα ἑκάστην, καὶ ἔδωκεν αὐτὰς ὁ βασιλεὺς ἐν οἴκῳ δρυμοῦ τοῦ Λιβάνου.
\par }{\PP \VS{17}Καὶ ἐποίησεν ὁ βασιλεὺς θρόνον ἐλεφαντίνων ὀδόντων μέγαν, καὶ κατεχρύσωσεν αὐτὸν χρυσίῳ δοκίμῳ.
\VS{18}Καὶ ἓξ ἀναβαθμοὶ τῷ θρόνῳ ἐνδεδεμένοι χρυσίῳ, καὶ ἀγκῶνες ἔνθεν καὶ ἔνθεν ἐπὶ τοῦ θρόνου τῆς καθέδρας, καὶ δύο λέοντες ἑστηκότες παρὰ τοὺς ἀγκῶνας,
\VS{19}καὶ δώδεκα λέοντες ἑστηκότες ἐκεῖ ἐπὶ τῶν ἓξ ἀναβαθμῶν ἔνθεν καὶ ἔνθεν· οὐκ ἐγενήθη οὕτως ἐν πάσῃ τῇ βασιλείᾳ.
\par }{\PP \VS{20}Καὶ πάντα τὰ σκεύη τοῦ βασιλέως Σαλωμὼν χρυσίου, καὶ πάντα τὰ σκεύη οἴκου δρυμοῦ τοῦ Λιβάνου χρυσίῳ κατειλημμένα· οὐκ ἦν ἀργύριον λογιζόμεμον ἐν ἡμέραις Σαλωμὼν εἰς οὐθέν.
\VS{21}Ὅτι ναῦς τῷ βασιλεῖ ἐπορεύετο εἰς Θαρσεῖς μετὰ τῶν παίδων Χιρὰμ, ἅπαξ διὰ τριῶν ἐτῶν ἤρχετο πλοῖα ἐκ Θαρσεὶς τῷ βασιλεῖ γέμοντα χρυσίου καὶ ἀργυρίου, καὶ ὀδόντων ἐλεφαντίνων, καὶ πιθήκων.
\par }{\PP \VS{22}Καὶ ἐμεγαλύνθη Σαλωμὼν ὑπὲρ πάντας τοὺς βασιλεῖς καὶ πλούτῳ καὶ σοφίᾳ.
\VS{23}Καὶ πάντες οἱ βασιλεῖς τῆς γῆς ἐζήτουν τὸ πρόσωπον Σαλωμὼν ἀκοῦσαι τῆς σοφίας αὐτοῦ, ἧς ἔδωκεν ὁ Θεὸς ἐν καρδίᾳ αὐτοῦ.
\VS{24}Καὶ αὐτοὶ ἔφερον ἕκαστος τὰ δῶρα αὐτοῦ, σκεύη ἀργυρᾶ καὶ σκεύη χρυσᾶ, καὶ ἱματισμὸν, στακτὴν καὶ ἡδύσματα, ἵππους καὶ ἡμιόνους τὸ κατʼ ἐνιαυτὸν ἐνιαυτόν.
\par }{\PP \VS{25}Καὶ ἦσαν τῷ Σαλωμὼν τέσσαρες χιλιάδες θήλειαι ἵπποι εἰς ἅρματα, καὶ δώδεκα χιλιάδες ἱππέων, καὶ ἔθετο αὐτοὺς ἐν πόλεσι τῶν ἁρμάτων, καὶ μετὰ τοῦ βασιλέως ἐν Ἱερουσαλήμ.
\VS{26}Καὶ ἦν ἡγούμενος πάντων τῶν βασιλέων ἀπὸ τοῦ ποταμοῦ καὶ ἕως γῆς ἀλλοφύλων, καὶ ἕως ὁρίων Αἰγύπτου.
\VS{27}Καὶ ἔδωκεν ὁ βασιλεὺς τὸ χρυσίον καὶ τὸ ἀργύριον ἐν Ἱερουσαλὴμ ὡς λίθους, καὶ τὰς κέρδους ὡς συκαμίνους τὰς ἐν τῇ πεδινῇ εἰς πλῆθος.
\VS{28}Καὶ ἡ ἔξοδος τῶν ἵππων ἐξ Αἰγύπτου τῷ Σαλωμὼν καὶ ἐκ πάσης τῆς γῆς.
\par }{\PP \VS{29}Καὶ οἱ κατάλοιποι λόγοι Σαλωμὼν οἱ πρῶτοι καὶ οἱ ἔσχατοι, ἰδοὺ οὗτοι γεγραμμένοι ἐπὶ τῶν λόγων Νάθαν τοῦ προφήτου, καὶ ἐπὶ τῶν λόγων Ἀχία τοῦ Σηλωνίτου, καὶ ἐν ταῖς ὁράσεσιν Ἰωὴλ τοῦ ὁρῶντος περὶ Ἱεροβοὰμ υἱοῦ Ναβάτ.
\VS{30}Καὶ ἐβασίλευσε Σαλωμὼν ἐπὶ πάντα Ἰσραὴλ τεσσαράκοντα ἔτη.
\VS{31}Καὶ ἐκοιμήθη Σαλωμὼν, καὶ ἔθαψαν αὐτὸν ἐν πόλει Δαυὶδ τοῦ πατρὸς αὐτοῦ, καὶ ἐβασιλευσε Ῥοβοὰμ υἱὸς αὐτοῦ ἀντʼ αὐτοῦ.

\par }\Chap{10}{\PP \VerseOne{1}Καὶ ἦλθε Ῥοβοὰμ εἰς Συχὲμ, ὅτι εἰς Συχὲμ ἤρχετο πᾶς Ἰσραὴλ βασιλεῦσαι αὐτόν.
\par }{\PP \VS{2}Καὶ ἐγένετο ὡς ἤκουσεν Ἱεροβοὰμ υἱὸς Ναβὰτ, καὶ αὐτὸς ἐν Αἰγύπτῳ, ὡς ἔφυγεν ἀπὸ προσώπου Σαλωμὼν τοῦ βασιλέως, καὶ κατῴκησεν Ἱεροβοὰμ ἐν Αἰγύπτῳ, καὶ ἀπέστρεψεν Ἱεροβοὰμ ἐξ Αἰγύπτου.
\VS{3}Καὶ ἀπέστειλαν καὶ ἐκάλεσαν αὐτόν· καὶ ἦλθεν Ἱεροβοὰμ καὶ πᾶσα ἡ ἐκκλησία πρὸς Ῥοβοὰμ, λέγοντες,
\VS{4}ὁ πατήρ σου ἐσκλήρυνε τὸν ζυγὸν ἡμῶν, καὶ νῦν ἄφες ἀπὸ τῆς δουλείας τοῦ πατρός σου τῆς σκληρᾶς, καὶ ἀπὸ τοῦ ζυγοῦ αὐτοῦ τοῦ βαρέος, οὗ ἔδωκεν ἐφʼ ἡμᾶς, καὶ δουλεύσομέν σοι.
\VS{5}Καὶ εἶπεν αὐτοῖς, πορεύεσθε ἕως τριῶν ἡμερῶν, καὶ ἔρχεσθε πρὸς μέ· καὶ ἀπῆλθεν ὁ λαός.
\par }{\PP \VS{6}Καὶ συνήγαγεν ὁ βασιλεὺς Ῥοβοὰμ τοὺς πρεσβυτέρους τοὺς ἑστηκότας ἐναντίον τοῦ Σαλωμὼν τοῦ πατρὸς αὐτοῦ ἐν τῷ ζῇν αὐτὸν, λέγων, πῶς ὑμεῖς βουλεύεσθε τοῦ ἀποκριθῆναι τῷ λαῷ τούτῳ λόγον;
\VS{7}Καὶ ἐλάλησαν αὐτῷ, λέγοντες, ἐὰν ἐν τῇ σήμερον γένῃ εἰς ἀγαθὸν τῷ λαῷ τούτῳ, καὶ εὐδοκήσῃς, καὶ λαλήσῃς αὐτοῖς λόγους ἀγαθοὺς, καὶ ἔσονταί σοι παῖδες πάσας τὰς ἡμέρας.
\VS{8}Καὶ κατέλιπε τὴν βουλὴν τῶν πρεσβυτέρων, οἳ συνεβουλεύσαντο αὐτῷ· καὶ συνεβουλεύσατο μετὰ τῶν παιδαρίων τῶν συνεκτραφέντων μετʼ αὐτοῦ τῶν ἑστηκότων ἐναντίον αὐτοῦ.
\VS{9}Καὶ εἶπεν αὐτοῖς, τί ὑμεῖς βουλεύεσθε, καὶ ἀποκριθήσομαι λόγον τῷ λαῷ τούτῳ, οἳ ἐλάλησαν πρὸς μὲ, λέγοντες, ἄνες ἀπὸ τοῦ ζυγοῦ οὗ ἔδωκεν ὁ πατήρ σου ἐφʼ ἡμᾶς;
\VS{10}Καὶ ἐλάλησαν αὐτῷ τὰ παιδάρια τὰ ἐκτραφέντα μετʼ αὐτοῦ, λέγοντες, οὕτως λαλήσεις τῷ λαῷ τῷ λαλήσαντι πρὸς σὲ, λέγων, ὁ πατήρ σου ἐβάρυνε τὸν ζυγὸν ἡμῶν, καὶ σὺ ἄφες ἀφʼ ἡμῶν· οὕτως ἐρεῖς, ὁ μικρὸς δάκτυλός μου παχύτερος τῆς ὀσφύος τοῦ πατρός μου.
\VS{11}Καὶ νῦν ὁ πατήρ μου ἐπαίδευσεν ὑμᾶς ζυγῷ βαρεῖ, κᾀγὼ προσθήσω ἐπὶ τὸν ζυγὸν ὑμῶν· ὁ πατήρ μου ἐπαίδευσεν ὑμᾶς ἐν μάστιξι, κᾀγὼ παιδεύσω ὑμᾶς ἐν σκορπίοις.
\par }{\PP \VS{12}Καὶ ἦλθεν Ἱεροβοὰμ καὶ πᾶς ὁ λαὸς πρὸς Ῥοβοὰμ τῇ ἡμέρᾳ τῇ τρίτῃ, ὡς ἐλάλησεν ὁ βασιλεὺς, λέγων, ἐπιστρέψατε πρὸς μὲ ἐν τῇ ἡμέρᾳ τῇ τρίτῃ.
\VS{13}Καὶ ἀπεκρίθη ὁ βασιλεὺς σκληρὰ, καὶ ἐγκατέλιπεν ὁ βασιλεὺς Ῥοβοὰμ τὴν βουλὴν τῶν πρεσβυτέρων,
\VS{14}καὶ ἐλάλησε πρὸς αὐτοὺς κατὰ τὴν βουλὴν τῶν νεωτέρων, λέγων, ὁ πατήρ μου ἐβάρυνε τὸν ζυγὸν ὑμῶν, καὶ ἐγὼ προσθήσω ἐπʼ αὐτόν· ὁ πατήρ μου ἐπαίδευσεν ὑμᾶς ἐν μάστιξι, καὶ ἐγὼ παιδεύσω ὑμᾶς ἐν σκορπίοις.
\par }{\PP \VS{15}Καὶ οὐκ ἤκουσεν ὁ βασιλεὺς τοῦ λαοῦ, ὅτι ἦν μεταστροφὴ παρὰ τοῦ Θεοῦ, λέγων, ἀνέστησε Κύριος τὸν λόγον αὐτοῦ, ὃν ἐλάλησεν ἐν χειρὶ Ἀχία τοῦ Σηλωνίτου περὶ Ἱεροβοὰμ υἱοῦ Ναβὰτ
\VS{16}καὶ παντὸς Ἰσραὴλ, ὅτι οὐκ ἤκουσεν ὁ βασιλεὺς αὐτῶν· καὶ ἀπεκρίθη ὁ λαὸς πρὸς τὸν βασιλέα, λέγων, τίς ἡμῶν ἡ μερὶς ἐν Δαυὶδ καὶ κληρονομία ἐν υἱῷ Ἰεσσαί; εἰς τὰ σκηνώματά σου, Ἰσραήλ· νῦν βλέπε τὸν οἶκόν σου, Δαυίδ. καὶ ἐπορεύθη πᾶς Ἰσραὴλ εἰς τὰ σκηνώματα αὐτοῦ.
\VS{17}Καὶ ἄνδρες Ἰσραὴλ καὶ οἱ κατοικοῦντες ἐν πόλεσιν Ἰούδα, καὶ ἐβασίλευσαν ἐπʼ αὐτῶν Ῥοβοάμ.
\par }{\PP \VS{18}Καὶ ἀπέστειλεν ἐπʼ αὐτοὺς Ῥοβοὰμ ὁ βασιλεὺς τὸν Ἀδωνιρὰμ τὸν ἐπὶ τοῦ φόρου, καὶ ἐλιθοβόλησαν αὐτὸν οἱ υἱοὶ Ἰσραὴλ λίθοις, καὶ ἀπέθανε· καὶ ὁ βασιλεὺς Ῥοβοὰμ ἔσπευσε τοῦ ἀναβῆναι εἰς τὸ ἅρμα, τοῦ φυγεῖν εἰς Ἱερουσαλήμ.
\VS{19}Καὶ ἠθέτησεν Ἰσραὴλ ἐν τῷ οἴκῳ Δαυὶδ ἕως τῆς ἡμέρας ταύτης.

\par }\Chap{11}{\PP \VerseOne{1}Καὶ ἦλθε Ῥοβοὰμ εἰς Ἱερουσαλὴμ, καὶ ἐξεκκλησίασε τὸν Ἰούδαν καὶ Βενιαμὶν ἑκατὸν ὀγδοήκοντα χιλιάδας νεανίσκων ποιούντων πόλεμον, καὶ ἐπολέμει πρὸς Ἰσραὴλ τοῦ ἐπιστρέψαι τὴν βασιλείαν τῷ Ῥοβοάμ.
\VS{2}Καὶ ἐγένετο λόγος Κυρίου πρὸς Σαμαίαν ἄνθρωπον τοῦ Θεοῦ, λέγων,
\VS{3}εἶπον πρὸς Ῥοβοὰμ τὸν τοῦ Σαλωμὼν καὶ πάντα Ἰούδαν καὶ Βενιαμὶν, λέγων,
\VS{4}τάδε λέγει Κύριος, οὐκ ἀναβήσεσθε, καὶ οὐ πολεμήσεσθε πρὸς τοὺς ἀδελφοὺς ὑμῶν· ἀποστρέφετε ἕκαστος εἰς τὸν οἶκον αὐτοῦ, ὅτι παρʼ ἐμοῦ ἐγένετο τὸ ῥῆμα τοῦτο· καὶ ἐπήκουσαν τοῦ λόγου Κυρίου, καὶ ἀπεστράφησαν τοῦ μὴ πορευθῆναι ἐπὶ Ἱεροβοάμ.
\par }{\PP \VS{5}Καὶ κατῴκησε Ῥοβοὰμ εἰς Ἱερουσαλὴμ, καὶ ᾠκοδόμησε πόλεις τειχήρεις ἐν τῇ Ἰουδαίᾳ.
\VS{6}Καὶ ᾠκοδόμησε τὴν Βηθλεὲμ, καὶ Αἰτὰν, καὶ Θεκωὲ,
\VS{7}καὶ Βαιθσουρὰ, καὶ τὴν Σοχὼθ, καὶ τὴν Ὀδολλὰμ,
\VS{8}καὶ τὴν Γὲθ, καὶ τὴν Μαρισὰν, καὶ τὴν Ζὶφ,
\VS{9}καὶ τὴν Ἀδωραὶ, καὶ Λαχὶς, καὶ τὴν Ἀζηκά,
\VS{10}καὶ τὴν Σαραὰ, καὶ τὴν Αἰλὼμ, καὶ τὴν Χεβρὼν, ἥ ἐστι τοῦ Ἰούδα καὶ Βενιαμὶν, πόλεις τειχήρεις.
\VS{11}Καὶ ὠχύρωσεν αὐτὰς τειχήρεις, καὶ ἔδωκεν ἐν αὐταῖς ἡγουμένους, καὶ παραθέσεις βρωμάτων, ἔλαιον καὶ οἶνον,
\VS{12}κατὰ πόλιν καὶ κατὰ πόλιν θυρεοὺς καὶ δόρατα, καὶ κατίσχυσεν αὐτὰς εἰς πλῆθος σφόδρα· καὶ ἦσαν αὐτῷ Ἰούδα καὶ Βενιαμίν.
\par }{\PP \VS{13}Καὶ οἱ ἱερεῖς καὶ οἱ Λευῖται οἳ ἦσαν ἐν παντὶ Ἰσραὴλ συνήχθησαν πρὸς αὐτὸν ἐκ πάντων τῶν ὁρίων·
\VS{14}Ὅτι ἐγκατέλιπον οἱ Λευῖται τὰ σκηνώματα τῆς κατασχέσεως αὐτῶν, καὶ ἐπορεύθησαν πρὸς Ἰούδα εἰς Ἱερουσαλὴμ, ὅτι ἐξέβαλεν αὐτοὺς Ἱεροβοὰμ καὶ οἱ υἱοὶ αὐτοῦ μὴ λειτουργεῖν Κυρίῳ,
\VS{15}καὶ κατέστησεν ἑαυτῷ ἱερεῖς τῶν ὑψηλῶν καὶ τοῖς εἰδώλοις καὶ τοῖς ματαίοις καὶ τοῖς μόσχοις, ἃ ἐποίησεν Ἱεροβοάμ.
\VS{16}Καὶ ἐξέβαλεν αὐτοὺς ἀπὸ φυλῶν Ἰσραὴλ, οἳ ἔδωκαν καρδίαν αὐτῶν τοῦ ζητῆσαι Κύριον Θεὸν Ἰσραήλ· καὶ ἦλθον εἰς Ἱερουσαλὴμ θῦσαι Κυρίῳ Θεῷ τῶν πατέρων αὐτῶν.
\VS{17}Καὶ κατίσχυσαν τὴν βασιλείαν Ἰούδα· καὶ κατίσχυσε Ῥοβοὰμ τὸν τοῦ Σαλωμὼν εἰς ἔτη τρία, ὅτι ἐπορεύθη ἐν ταῖς ὁδοῖς Δαυὶδ καὶ Σαλωμὼν ἔτη τρία.
\par }{\PP \VS{18}Καὶ ἔλαβεν ἑαυτῷ Ῥοβοὰμ γυναῖκα τὴν Μοολὰθ θυγατέρα Ἱεριμοὺθ υἱοῦ Δαυὶδ, καὶ Ἀβιγαίαν θυγατέρα Ἑλιὰβ τοῦ Ἰεσσαί.
\VS{19}Καὶ ἔτεκεν αὐτῷ υἱοὺς, τὸν Ἰεοὺς, καὶ τὸν Σαμαρία, καὶ τὸν Ζαάμ.
\VS{20}Καὶ μετὰ ταῦτα ἔλαβεν ἑαυτῷ τὴν Μααχὰ θυγατέρα Ἀβεσσαλὼμ, καὶ ἔτεκεν αὐτῷ τὸν Ἀβιὰ, καὶ τὸν Ἰετθὶ, καὶ τὸν Ζηζὰ, καὶ τὸν Σαλημώθ.
\VS{21}Καὶ ἠγάπησε Ῥοβοὰμ τὴν Μααχά θυγατέρα Ἀβεσσαλὼμ ὑπὲρ πάσας τὰς γυναῖκας αὐτοῦ καὶ τὰς παλλακὰς αὐτοῦ, ὅτι γυναῖκας δεκαοκτὼ εἶχε καὶ παλλακὰς ἑξήκοντα· καὶ ἐγέννησεν υἱοὺς εἴκοσι καὶ ὀκτὼ, καὶ θυγατέρας ἑξήκοντα.
\VS{22}Καὶ κατέστησεν εἰς ἄρχοντα Ἀβιὰ τὸν τῆς Μααχὰ εἰς ἡγούμενον ἐν τοῖς ἀδελφοῖς αὐτοῦ, ὅτι βασιλεῦσαι διενοεῖτο αὐτόν.
\VS{23}Καὶ ηὐξήθη παρὰ πάντας τοὺς υἱοὺς αὐτοῦ ἐν πᾶσι τοῖς ὁρίοις Ἰούδα καὶ Βενιαμὶν, καὶ ἐν ταῖς πόλεσι ταῖς ὀχυραῖς, καὶ ἔδωκεν αὐταῖς τροφὰς πλῆθος πολὺ, καὶ ᾐτήσατο πλῆθος γυναικῶν.

\par }\Chap{12}{\PP \VerseOne{1}Καὶ ἐγένετο ὡς ἡτοιμάσθη ἡ βασιλεία Ῥοβοὰμ, καὶ ὡς κατεκρατήθη, ἐγκατέλιπε τὰς ἐντολὰς Κυρίου, καὶ πᾶς Ἰσραὴλ μετʼ αὐτοῦ.
\par }{\PP \VS{2}Καὶ ἐγένετο ἐν τῷ ἔτει τῷ πέμπτῳ τῆς βασιλείας Ῥοβοὰμ, ἀνέβη Σουσακὶμ βασιλεὺς Αἰγύπτου ἐπὶ Ἱερουσαλὴμ, ὅτι ἥμαρτον ἐναντίον Κυρίου,
\VS{3}ἐν χιλίοις καὶ διακοσίοις ἅρμασι καὶ ἑξήκοντα χιλιάσιν ἵππων, καὶ οὐκ ἦν ἀριθμὸς τοῦ πλήθους τοῦ ἐλθόντος μετʼ αὐτοῦ ἐξ Αἰγύπτου, Λίβυες, Τρωγοδύται, καὶ Αἰθίοπες.
\VS{4}Καὶ κατεκράτησαν τῶν πόλεων τῶν ὀχυρῶν, αἳ ἦσαν ἐν Ἰούδα, καὶ ἦλθον εἰς Ἱερουσαλήμ.
\par }{\PP \VS{5}Καὶ Σαμαίας ὁ προφήτης ἦλθε πρὸς Ῥοβοὰμ, καὶ πρὸς τοὺς ἄρχοντας Ἰούδα τοὺς συναχθέντας εἰς Ἱερουσαλὴμ ἀπὸ προσώπου Σουσακὶμ, καὶ εἶπεν αὐτοῖς, οὕτως εἶπε Κύριος, ὑμεῖς ἐγκατελίπετέ με, καὶ ἐγὼ ἐγκαταλείψω ὑμᾶς ἐν χειρὶ Σουσακίμ.
\VS{6}Καὶ ᾐσχύνθησαν οἱ ἄρχοντες Ἰσραὴλ καὶ ὁ βασιλεὺς, καὶ εἶπαν, δίκαιος ὁ κύριος.
\VS{7}Καὶ ἐν τῷ ἰδεῖν Κύριον ὅτι ἐνετράπησαν, καὶ ἐγένετο λόγος Κυρίου πρὸς Σαμαίαν, λέγων, ἐνετράπησαν, οὐ καταφθερῶ αὐτοὺς, καὶ δώσω αὐτοὺς ὡς μικρὸν εἰς σωτηρίαν, καὶ οὐ μὴ στάξῃ ὁ θυμός μου ἐν Ἱερουσαλὴμ·
\VS{8}ὅτι ἔσονται εἰς παῖδας, καὶ γνώσονται τὴν δουλείαν μου, καὶ τὴν δουλείαν τῆς βασιλείας τῆς γῆς.
\par }{\PP \VS{9}Καὶ ἀνέβη Σουσακὶμ βασιλεὺς Αἰγύπτου ἐπὶ Ἱερουσαλὴμ, καὶ ἔλαβε τοὺς θησαυροὺς τοὺς ἐν οἴκῳ Κυρίου, καὶ τοὺς θησαυροὺς τοὺς ἐν οἴκῳ τοῦ βασιλέως, τὰ πάντα ἔλαβε· καὶ ἔλαβε τοὺς θυρεοὺς τοὺς χρυσοῦς οὓς ἐποίησε Σαλωμών.
\VS{10}Καὶ ἐποίησεν ὁ βασιλεὺς Ῥοβοὰμ θυρεοὺς χαλκοῦς ἀντʼ αὐτῶν· καὶ κατέστησεν ἐπʼ αὐτὸν Σουσακὶμ ἄρχοντας παρατρεχόντων, τοὺς φυλάσσοντας τὸν πυλῶνα τοῦ βασιλέως.
\VS{11}Καὶ ἐγένετο ἐν τῷ εἰσελθεῖν τὸν βασιλέα εἰς οἶκον Κυρίου, εἰσεπορεύοντο οἱ φυλάσσοντες, καὶ οἱ παρατρέχοντες, καὶ οἱ ἐπιστρέφοντες εἰς ἀπάντησιν τῶν παρατρεχόντων.
\VS{12}Καὶ ἐν τῷ ἐντραπῆναι αὐτὸν, ἀπεστράφη ἀπʼ αὐτοῦ ὀργὴ Κυρίου, καὶ οὐκ εἰς καταφθορὰν εἰς τέλος· καὶ γὰρ ἐν Ἰούδα ἦσαν λόγοι ἀγαθοί.
\par }{\PP \VS{13}Καὶ κατίσχυσεν ὁ βασιλεὺς Ῥοβοὰμ ἐν Ἱερουσαλὴμ, καὶ ἐβασίλευσε· καὶ τεσσαράκοντα καὶ ἑνὸς ἐτῶν Ῥοβοὰμ ἐν τῷ βασιλεῦσαι αὐτὸν, καὶ ἑπτακαίδεκα ἔτη ἐβασίλευσεν ἐν Ἱερουσαλὴμ, ἐν τῇ πόλει ᾗ ἐξελέξατο Κύριος ἐπονομάσαι τὸ ὄνομα αὐτοῦ ἐκεῖ ἐκ πασῶν φυλῶν υἱῶν Ἰσραὴλ, καὶ τὸ ὄνομα τῆς μητρὸς αὐτοῦ Νοομμὰ ἡ Ἀμμανίτις.
\VS{14}Καὶ ἐποίησε τὸ πονηρὸν, ὅτι οὐ κατεύθυνε τὴν καρδίαν αὐτοῦ ἐκζητῆσαι τὸν κύριον.
\par }{\PP \VS{15}Καὶ λόγοι Ῥοβοὰμ οἱ πρῶτοι καὶ ἔσχατοι οὐκ ἰδοὺ γεγραμμένοι ἐν τοῖς λόγοις Σαμαία τοῦ προφήτου, καὶ Ἀδδὼ τοῦ ὁρῶντος, καὶ πράξεις αὐτοῦ; καὶ ἐπολέμησε Ῥοβοὰμ τὸν Ἱεροβοὰμ πάσας τὰς ἡμέρας.
\VS{16}Καὶ ἀπέθανε Ῥοβοὰμ μετὰ τῶν πατέρων αὐτοῦ, καὶ ἐτάφη ἐν πόλει Δαυὶδ, καὶ ἐβασίλευσεν Ἀβιὰ υἱὸς αὐτοῦ ἀντʼ αὐτοῦ.

\par }\Chap{13}{\PP \VerseOne{1}Ἐν τῷ ὀκτωκαιδεκάτῳ ἔτει τῆς βασιλείας Ἱεροβοὰμ ἐβασίλευσεν Ἀβιὰ ἐπὶ Ἰούδαν.
\VS{2}Τρία ἔτη ἐβασίλευσεν ἐν Ἱερουσαλὴμ, καὶ ὄνομα τῇ μητρὶ αὐτοῦ Μααχὰ, θυγάτηρ Οὐριὴλ ἀπὸ Γαβαών.
\par }{\PP Καὶ πολεμος ἦν ἀναμέσον Ἀβιὰ καὶ ἀναμέσον Ἱεροβοάμ.
\VS{3}Καὶ παρετάξατο Ἀβιὰ ἐν δυνάμει πολεμισταῖς δυνάμεως τετρακοσίαις χιλιάσιν ἀνδρῶν δυνατῶν· καὶ Ἱεροβοὰμ παρετάξατο πρὸς αὐτὸν πόλεμον ἐν ὀκτακοσίαις χιλιάσι, δυνατοὶ πολεμισταὶ δυνάμεως.
\par }{\PP \VS{4}Καὶ ἀνέστη Ἀβιὰ ἀπὸ τοῦ ὄρους Σομόρων, ὅ ἐστιν ἐν τῷ ὄρει Ἐφραὶμ, καὶ εἶπεν, ἀκούσατε Ἱεροβοὰμ καὶ πᾶς Ἰσραήλ·
\VS{5}Οὐχ ὑμῖν γνῶναι ὅτι Κύριος ὁ Θεὸς Ἰσραὴλ ἔδωκε βασιλέα ἐπὶ τὸν Ἰσραὴλ εἰς τὸν αἰῶνα τῷ Δαυὶδ καὶ τοῖς υἱοῖς αὐτοῦ διαθήκῃ ἁλός;
\VS{6}Καὶ ἀνέστη Ἱεροβοὰμ ὁ τοῦ Ναβὰτ ὁ παῖς Σαλωμὼν τοῦ Δαυὶδ, καὶ ἀπέστη ἀπὸ τοῦ κυρίου αὐτοῦ·
\VS{7}Καὶ συνήχθησαν πρὸς αὐτὸν ἄνδρες λοιμοὶ υἱοὶ παράνομοι, καὶ ἀνέστη πρὸς Ῥοβοὰμ τὸν τοῦ Σαλωμὼν, καὶ Ῥοβοὰμ ἦν νεώτερος καὶ δειλὸς τῇ καρδίᾳ, καὶ οὐκ ἀντέστη κατὰ πρόσωπον αὐτοῦ.
\VS{8}Καὶ νῦν ὑμεῖς λέγετε ἀντιστῆναι κατὰ πρόσωπον βασιλείας Κυρίου διὰ χειρὸς υἱῶν Δαυίδ· καὶ ὑμεῖς πλῆθος πολὺ, καὶ μεθʼ ὑμῶν μόσχοι χρυσοῖ οὓς ἐποίησεν ὑμῖν Ἱεροβοὰμ εἰς θεούς.
\VS{9}Ἢ οὐκ ἐξεβάλετε τοὺς ἱερεῖς Κυρίου τοὺς υἱοὺς Ἀαρὼν καὶ τοὺς Λευίτας, καὶ ἐποιήσατε ἑαυτοῖς ἱερεῖς ἐκ τοῦ λαοῦ τῆς γῆς πάσης; ὁ προσπορευόμενος πληρῶσαι τὰς χεῖρας ἐν μόσχῳ ἐκ βοῶν καὶ κριοῖς ἑπτὰ, καὶ ἐγίνετο εἰς ἱερέα τῷ μὴ ὄντι θεῷ.
\VS{10}Καὶ ἡμεῖς Κύριον τὸν Θεὸν ἡμῶν οὐκ ἐγκατελίπομεν, καὶ οἱ ἱερεῖς αὐτοῦ λειτουργοῦσι τῷ Κυρίῳ οἱ υἱοὶ Ἀαρὼν καὶ οἱ Λευῖται, καὶ ἐν ταῖς ἐφημερίαις αὐτῶν
\VS{11}θυμιῶσι τῷ Κυρίῳ ὁλοκαύτωμα πρωῒ καὶ δείλης, καὶ θυμίαμα συνθέσεως, καὶ προθέσεις ἄρτων ἐπὶ τῆς τραπέζης τῆς καθαρᾶς, καὶ ἡ λυχνία ἡ χρυσῆ καὶ οἱ λυχνοὶ τῆς καύσεως ἀνάψαι δείλης· ὅτι φυλάσσομεν ἡμεῖς τὰς φυλακὰς Κυρίου τοῦ Θεοῦ τῶν πατέρων ἡμῶν, καὶ ὑμεῖς ἐγκατελίπετε αὐτόν.
\VS{12}Καὶ ἰδοὺ μεθʼ ἡμῶν ἐν ἀρχῇ Κύριος καὶ οἱ ἱερεῖς αὐτοῦ, καὶ αἱ σάλπιγγες τῆς σημασίας τοῦ σημαίνειν ἐφʼ ἡμᾶς. Οἱ υἱοὶ τοῦ Ἰσραὴλ, μὴ πολεμήσητε πρὸς Κύριον Θεὸν τῶν πατέρων ἡμῶν, ὅτι οὐκ εὐοδώσεται ὑμῖν.
\par }{\PP \VS{13}Καὶ Ἱεροβοὰμ ἀπέστρεψε τὸ ἔνεδρον ἐλθεῖν αὐτῷ ἐκ τῶν ὄπισθεν, καὶ ἐγένετο ἔμπροσθεν Ἰούδα, καὶ τὸ ἔνεδρον ἐκ τῶν ὄπισθεν.
\VS{14}Καὶ ἀπέστρεψεν Ἰούδας, καὶ ἰδοὺ αὐτοῖς ὁ πόλεμος ἐκ τῶν ἔμπροσθεν καὶ ἐκ τῶν ὄπισθεν, καὶ ἐβόησαν πρὸς Κύριον, καὶ οἱ ἱερεῖς ἐσάλπισαν ταῖς σάλπιγξι.
\VS{15}Καὶ ἐβόησαν ἄνδρες Ἰούδα· καὶ ἐγένετο ἐν τῷ βοᾷν ἄνδρας Ἰούδα, καὶ Κύριος ἐπάταξε τὸν Ἱεροβοὰμ καὶ τὸν Ἰσραὴλ ἐναντίον Ἀβιὰ καὶ Ἰούδα.
\VS{16}Καὶ ἔφυγον οἱ υἱοὶ Ἰσραὴλ ἀπὸ προσώπου Ἰούδα, καὶ παρέδωκεν αὐτοὺς Κύριος εἰς τὰς χεῖρας αὐτῶν.
\VS{17}Καὶ ἐπάταξεν ἐν αὐτοῖς Ἀβιὰ καὶ ὁ λαὸς αὐτοῦ πληγὴν μεγάλην· καὶ ἔπεσον τραυματίαι ἀπὸ Ἰσραὴλ πεντακόσιαι χιλιάδες ἄνδρες δυνατοί.
\VS{18}Καὶ ἐταπεινώθησαν οἱ υἱοὶ Ἰσραὴλ ἐν τῇ ἡμέρᾳ ἐκείνῃ, καὶ κατίσχυσαν οἱ υἱοὶ Ἰούδα, ὅτι ἤλπισαν ἐπὶ Κύριον Θεὸν τῶν πατέρων αὐτῶν.
\VS{19}Καὶ κατεδίωξεν Ἀβιὰ ὀπίσω Ἱεροβοὰμ, καὶ προκατελάβετο παρʼ αὐτοῦ τὰς πόλεις, τὴν Βαιθὴλ καὶ τὰς κώμας αὐτῆς, καὶ τὴν Ἰεσυνὰ καὶ τὰς κώμας αὐτῆς, καὶ τὴν Ἐφρὼν καὶ τὰς κώμας αὐτῆς.
\VS{20}Καὶ οὐκ ἔσχεν ἰσχὺν Ἱεροβοὰμ ἔτι πάσας τὰς ἡμέρας Ἀβιὰ, καὶ ἐπάταξεν αὐτὸν Κύριος, καὶ ἐτελεύτησε.
\par }{\PP \VS{21}Καὶ κατίσχυσεν Ἀβιὰ, καὶ ἔλαβεν ἑαυτῷ γυναῖκας δεκατέσσαρας, καὶ ἐγέννησεν υἱοὺς εἰκοσιδύο καὶ ἐκκαίδεκα θυγατέρας.
\par }{\PP \VS{22}Καὶ οἱ λοιποὶ λόγοι Ἀβιὰ καὶ αἱ πράξεις αὐτοῦ καὶ οἱ λόγοι αὐτοῦ γεγραμμένοι ἐπὶ βιβλίῳ τοῦ προφήτου Ἀδδώ.
\par }{\PP \VS{23}Καὶ ἀπέθανεν Ἀβιὰ μετὰ τῶν πατέρων αὐτοῦ, καὶ ἔθαψαν αὐτὸν ἐν πόλει Δαυὶδ, καὶ ἐβασίλευσεν Ἀσὰ υἱὸς αὐτοῦ ἀντʼ αὐτοῦ. ἐν ταῖς ἡμέραις Ἀσὰ ἡσύχασεν ἡ γῆ Ἰούδα δέκα ἔτη.

\par }\Chap{14}{\PP \VerseOne{1}Καὶ ἐποίησε τὸ καλὸν καὶ τὸ εὐθὲς ἐνώπιον Κυρίου τοῦ Θεοῦ αὐτοῦ.
\VS{2}Καὶ ἀπέστησε τὰ θυσιαστήρια τῶν ἀλλοτρίων καὶ τὰ ὑψηλὰ, καὶ συνέτριψε τὰς στήλας, καὶ ἐξέκοψε τὰ ἄλση,
\VS{3}καὶ εἶπε τῷ Ἰούδα ἐκζητῆσαι τὸν Κύριον Θεὸν τῶν πατέρων αὐτῶν, καὶ ποιῆσαι τὸν νόμον καὶ τὰς ἐντολάς.
\VS{4}Καὶ ἀπέστησεν ἀπὸ πασῶν πόλεων Ἰούδα τὰ θυσιαστήρια καὶ τὰ εἴδωλα, καὶ εἰρήνευσε πόλεις τειχήρεις ἐν γῇ Ἰούδα,
\VS{5}ὅτι εἰρήνευσεν ἡ γῆ, καὶ οὐκ ἦν αὐτῷ πόλεμος ἐν τοῖς ἔτεσι τούτοις, ὅτι κατέπαυσε Κύριος αὐτῷ.
\VS{6}Καὶ εἶπε τῷ Ἰούδα, οἰκοδομήσωμεν τὰς πόλεις ταύτας, καὶ ποιήσωμεν τείχη καὶ πύργους καὶ πύλας καὶ μοχλοὺς, ἐνώπιον τῆς γῆς κυριεύσομεν· ὅτι καθὼς ἐξεζητήσαμεν Κύριον τὸν Θεὸν ἡμῶν, ἐξεζήτησεν ἡμᾶς, καὶ κατέπαυσεν ἡμᾶς κυκλόθεν, καὶ εὐώδωσεν ἡμῖν.
\VS{7}Καὶ ἐγένετο δύναμις τῷ Ἀσὰ ὁπλοφόρων αἰρόντων θυρεοὺς καὶ δόρατα ἐν γῇ Ἰούδα τριακόσιαι χιλιάδες, καὶ ἐν γῇ Βενιαμὶν πελτασταὶ καὶ τοξόται διακόσιαι καὶ ὀγδοήκοντα χιλιάδες, πάντες οὗτοι πολεμισταὶ δυνάμεως.
\par }{\PP \VS{8}Καὶ ἐξῆλθεν ἐπʼ αὐτοὺς Ζαρὲ ὁ Αἰθίοψ ἐν δυνάμει ἐν χιλίαις χιλιάσι καὶ ἅρμασι τριακοσίοις, καὶ ἦλθεν ἕως Μαρησά.
\VS{9}Καὶ ἐξῆλθεν Ἀσὰ εἰς συνάντησιν αὐτῷ, καὶ παρετάξατο πόλεμον ἐν τῇ φάραγγι κατὰ Βοῤῥᾶν Μαρησά.
\VS{10}Καὶ ἐβόησεν Ἀσὰ πρὸς Κύριον Θεὸν αὐτοῦ, καὶ εἶπε, Κύριε, οὐκ ἀδυνατεῖ παρὰ σοὶ σώζειν ἐν πολλοῖς καὶ ἐν ὀλίγοις· κατίσχυσον ἡμᾶς Κύριε ὁ Θεὸς ἡμῶν, ὅτι ἐπὶ σοὶ πεποιθαμεν, καὶ ἐπὶ τῷ ὀνόματί σου ἤλθομεν ἐπὶ τὸ πλῆθος τὸ πολὺ τοῦτο· Κύριε ὁ Θεὸς ἡμῶν, μὴ κατισχυσάτω πρὸς σὲ ἄνθρωπος.
\VS{11}Καὶ ἐπάταξε Κύριος τοὺς Αἰθίοπας ἐναντίον Ἰούδα, καὶ ἔφυγον Αἰθίοπες,
\VS{12}καὶ κατεδίωξεν αὐτοὺς Ἀσὰ καὶ ὁ λαὸς αὐτοῦ ἕως Γεδώρ· καὶ ἔπεσον Αἰθίοπες ὥστε μὴ εἶναι ἐν αὐτοῖς περιποίησιν, ὅτι συνετρίβησαν ἐνώπιον Κυρίου καὶ ἐναντίον τῆς δυνάμεως αὐτοῦ, καὶ ἐσκύλευσαν σκῦλα πολλά.
\VS{13}Καὶ ἐξέκοψαν τὰς κώμας αὐτῶν κύκλῳ Γεδὼρ, ὅτι ἐγενήθη ἔκστασις Κυρίου ἐπʼ αὐτοὺς, καὶ ἐσκύλευσαν πάσας τὰς πόλεις αὐτῶν, ὅτι πολλὰ σκῦλα ἐγενήθη αὐτοῖς.
\VS{14}Καί γε σκηνὰς κτήσεων, καὶ τοὺς Ἀλιμαζονεῖς ἐξέκοψαν, καὶ ἔλαβον πρόβατα πολλὰ καὶ καμήλους, καὶ ἐπέστρεψαν εἰς Ἱερουσαλήμ.

\par }\Chap{15}{\PP \VerseOne{1}Καὶ Ἀζαρίας υἱὸς Ὡδὴδ, ἐγένετο ἐπʼ αὐτὸν πνεῦμα Κυρίου.
\VS{2}Καὶ ἐξῆλθεν εἰς ἀπάντησιν Ἀσὰ καὶ παντὶ Ἰούδα καὶ Βενιαμὶν, καὶ εἶπεν, ἀκούσατέ μου Ἀσὰ, καὶ πᾶς Ἰούδα καὶ Βενιαμίν· Κύριος μεθʼ ὑμῶν ἐν τῷ εἶναι ὑμᾶς μετʼ αὐτοῦ, καὶ ἐὰν ἐκζητήσητε αὐτὸν εὑρεθήσεται ὑμῖν, καὶ ἐὰν ἐγκαταλείπητε αὐτὸν, ἐγκαταλείψει ὑμᾶς.
\VS{3}Καὶ ἡμέραι πολλαὶ τῷ Ἰσραὴλ ἐν οὐ Θεῷ ἀληθινῷ, καὶ οὐχ ἱερέως ὑποδεικνῦντος, καὶ ἐν οὐ νόμῳ.
\VS{4}Καὶ ἐπιστρέψει αὐτοὺς ἐπὶ Κύριον Θεὸν Ἰσραὴλ, καὶ εὑρεθήσεται αὐτοῖς.
\VS{5}Καὶ ἐν ἐκείνῳ τῷ καιρῷ οὐκ ἔστιν εἰρήνη τῷ ἐκπορευομένῳ καὶ τῷ εἰσπορευομένῳ, ὅτι ἔκστασις Κυρίου ἐπὶ πάντας τοὺς κατοικοῦντας τὰς χώρας.
\VS{6}Καὶ πολεμήσει ἔθνος πρὸς ἔθνος καὶ πόλις πρὸς πόλιν, ὅτι ὁ Θεὸς ἐξέστησεν αὐτοὺς ἐν πάσῃ θλίψει.
\VS{7}Καὶ ὑμεῖς ἰσχύσατε, καὶ μὴ ἐκλυέσθωσαν αἱ χεῖρες ὑμῶν, ὅτι ἔστι μισθὸς τῇ ἐργασίᾳ ὑμῶν.
\par }{\PP \VS{8}Καὶ ἐν τῷ ἀκοῦσαι τοὺς λόγους τούτους καὶ τὴν προφητείαν Ἀδὰδ τοῦ προφήτου, καὶ κατίσχυσε καὶ ἐξέβαλε τὰ βδελύγματα ἀπὸ πάσης τῆς γῆς Ἰούδα καὶ Βενιαμὶν καὶ ἀπὸ τῶν πόλεων ὧν κατέσχεν Ἱεροβοὰμ ἐν ὄρει Ἐφραὶμ, καὶ ἐνεκαίνισε τὸ θυσιαστήριον Κυρίου, ὃ ἦν ἔμπροσθεν τοῦ ναοῦ Κυρίου.
\VS{9}Καὶ ἐξεκκλησίασε τὸν Ἰούδαν καὶ Βενιαμὶν καὶ τοὺς προσηλύτους τοὺς παροικοῦντας μετʼ αὐτοῦ ἀπὸ Ἐφραὶμ καὶ ἀπὸ Μανασσῆ καὶ ἀπὸ Συμεὼν, ὅτι προσετέθησαν πρὸς αὐτὸν πολλοὶ τοῦ Ἰσραὴλ ἐν τῷ ἰδεῖν αὐτοὺς, ὅτι Κύριος ὁ Θεὸς αὐτοῦ μετʼ αὐτοῦ.
\VS{10}Καὶ συνήχθησαν εἰς Ἱερουσαλὴμ ἐν τῷ μηνὶ τῷ τρίτῳ ἐν τῷ ἔτει τῷ πεντεκαιδεκάτῳ τῆς βασιλείας Ἀσά.
\VS{11}Καὶ ἔθυσε τῷ Κυρίῳ ἐν τῇ ἡμέρᾳ ἐκείνῃ ἀπὸ τῶν σκύλων ὧν ἤνεγκαν, μόσχους ἑπτακοσίους καὶ πρόβατα ἑπτακισχίλια.
\par }{\PP \VS{12}Καὶ διῆλθεν ἐν διαθήκῃ ζητῆσαι Κύριον Θεὸν τῶν πατέρων αὐτῶν ἐξ ὅλης τῆς καρδίας καὶ ἐξ ὅλης τῆς ψυχῆς αὐτῶν.
\VS{13}Καὶ πᾶς ὃς ἐὰν μὴ ἐκζητήσῃ τὸν Κύριον Θεὸν τοῦ Ἰσραὴλ, ἀποθανεῖται ἀπὸ νεωτέρου ἕως πρεσβυτέρου, ἀπὸ ἀνδρὸς ἕως γυναικός.
\VS{14}Καὶ ὤμοσαν ἐν Κυρίῳ ἐν φωνῇ μεγάλῃ καὶ ἐν σάλπιγξι καὶ ἐν κερατίναις.
\VS{15}Καὶ εὐφράνθησαν πᾶς Ἰούδα περὶ τοῦ ὅρκου, ὅτι ἐξ ὅλης τῆς ψυχῆς ὤμοσαν, καὶ ἐν πάσῃ θελήσει ἐζήτησαν αὐτὸν· καὶ εὑρέθη αὐτοῖς, καὶ κατέπαυσε Κύριος αὐτοῖς κυκλόθεν.
\par }{\PP \VS{16}Καὶ τὴν Μααχὰ τὴν μητέρα αὐτοῦ μετέστησε τοῦ μὴ εἶναι τῇ Ἀστάρτῃ λειτουργοῦσαν, καὶ κατέκοψε τὸ εἴδωλον, καὶ κατέκαυσεν ἐν χειμάῤῥῳ Κέδρων.
\VS{17}Πλὴν τὰ ὑψηλὰ οὐκ ἀπέστησαν, ἔτι ὑπῆρχεν ἐν τῷ Ἰσραήλ· ἀλλʼ ἡ καρδία Ἀσὰ ἐγένετο πλήρης πάσας τὰς ἡμέρας αὐτοῦ.
\VS{18}Καὶ εἰσήνεγκε τὰ ἅγια Δαυὶδ τοῦ πατρὸς αὐτοῦ, καὶ τὰ ἅγια οἴκου τοῦ Θεοῦ, ἀργύριον καὶ χρυσίον καὶ σκεύη.
\VS{19}Καὶ πόλεμος οὖκ ἦν μετʼ αὐτοῦ ἕως τοῦ πέμπτου καὶ τριακοστοῦ ἔτους τῆς βασιλείας Ἀσά.

\par }\Chap{16}{\PP \VerseOne{1}Καὶ ἐν τῷ ὀγδόῳ καὶ τριακοστῷ ἔτει τῆς βασιλείας Ἀσὰ, ἀνέβη βασιλεὺς Ἰσραὴλ ἐπὶ Ἰούδαν, καὶ ᾠκοδόμησε τὴν Ῥαμὰ τοῦ μὴ δοῦναι ἔξοδον καὶ εἴσοδον τῷ Ἀσὰ βασιλεῖ Ἰούδα.
\par }{\PP \VS{2}Καὶ ἔλαβεν Ἀσὰ ἀργύριον καὶ χρυσίον ἐκ θησαυρῶν οἴκου Κυρίου καὶ οἴκου τοῦ βασιλέως, καὶ ἀπέστειλε πρὸς τὸν υἱὸν τοῦ Ἄδερ βασιλέως Συρίας τὸν κατοικοῦντα ἐν Δαμασκῷ, λέγων,
\VS{3}διάθου διαθήκην ἀναμέσον ἐμοῦ καὶ σοῦ, καὶ ἀναμέσον τοῦ πατρός μου καὶ ἀναμέσον τοῦ πατρός σου· ἰδοὺ ἀπέσταλκά σοι χρυσίον καὶ ἀργύριον· δεῦρο καὶ διασκέδασον ἀπʼ ἐμοῦ τὸν Βαασὰ βασιλέα Ἰσραὴλ, καὶ ἀπελθέτω ἀπʼ ἐμοῦ.
\par }{\PP \VS{4}Καὶ ἤκουσεν υἱὸς Ἄδερ τοῦ βασιλέως Ἀσὰ, καὶ ἀπέστειλε τοὺς ἄρχοντας τῆς δυνάμεως αὐτοῦ ἐπὶ τὰς πόλεις Ἰσραὴλ, καὶ ἐπάταξε τὴν Ἀϊὼν, καὶ τὴν Δὰν, καὶ τὴν Ἀβελμαῒν, καὶ πάσας τὰς περιχώρους Νεφθαλί.
\par }{\PP \VS{5}Καὶ ἐγένετο ἐν τῷ ἀκοῦσαι Βαασὰ, ἀπέλιπε τοῦ μηκέτι οἰκοδομεῖν τὴν Ῥαμὰ, καὶ κατέπαυσε τὸ ἔργον αὐτοῦ.
\VS{6}Καὶ Ἀσὰ βασιλεὺς ἔλαβε πάντα τὸν Ἰούδαν, καὶ ἔλαβε τοὺς λίθους τῆς Ῥαμὰ καὶ τὰ ξύλα αὐτῆς, ἃ ᾠκοδόμησε Βαασὰ, καὶ ᾠκοδόμησεν ἐν αὐτοῖς τὴν Γαβαὲ καὶ τὴν Μασφά.
\par }{\PP \VS{7}Καὶ ἐν τῷ καιρῷ ἐκείνῳ ἦλθεν Ἀνανὶ ὁ προφήτης πρὸς Ἀσὰ βασιλέα Ἰούδα, καὶ εἶπεν αὐτῷ, ἐν τῷ πεποιθέναι σε ἐπὶ βασιλέα Συρίας, καὶ μὴ πεποιθέναι σε ἐπὶ Κύριον Θεόν σου, διὰ τοῦτο ἐσώθη ἡ δύναμις Συρίας ἀπὸ τῆς χειρός σου.
\VS{8}Οὐχ οἱ Αἰθίοπες καὶ Λίβυες ἦσαν εἰς δύναμιν πολλὴν, εἰς θάρσος, εἰς ἱππεῖς, εἰς πλῆθος σφόδρα; καὶ ἐν τῷ πεποιθέναι σε ἐπὶ Κύριον παρέδωκεν εἰς τὰς χεῖράς σου;
\VS{9}Ὅτι οἱ ὀφθαλμοὶ Κυρίου ἐπιβλέπουσιν ἐν πάσῃ τῇ γῇ, κατισχύσαι ἐν πάσῃ καρδίᾳ πλήρει πρὸς αὐτόν· ἠγνόηκας ἐπὶ τούτῳ, ἀπὸ τοῦ νῦν ἔσται μετὰ σοῦ πόλεμος.
\VS{10}Καὶ ἐθυμώθη Ἀσὰ τῷ προφήτῃ, καὶ παρέθετο αὐτὸν εἰς φυλακὴν, ὅτι ὠργίσθη ἐπὶ τούτῳ, καὶ ἐλυμῄνατο Ἀσὰ ἐν τῷ λαῷ ἐν τῷ καιρῷ ἐκείνῳ.
\par }{\PP \VS{11}Καὶ ἰδοὺ οἱ λόγοι Ἀσὰ, οἱ πρῶτοι καὶ οἱ ἔσχατοι, γεγραμμένοι ἐν βιβλίῳ βασιλέων Ἰούδα καὶ Ἰσραήλ.
\par }{\PP \VS{12}Καὶ ἐμαλακίσθη Ἀσὰ ἐν τῷ ἔτει τῷ ἐννάτῳ καὶ τριακοστῷ τῆς βασιλείας αὐτοῦ τοὺς πόδας, ἕως σφόδρα ἐμαλακίσθη· καὶ ἐν τῇ μαλακίᾳ αὐτοῦ οὐκ ἐζήτησε τὸν Κύριον, ἀλλὰ τοὺς ἰατρούς.
\VS{13}Καὶ ἐκοιμήθη Ἀσὰ μετὰ τῶν πατέρων αὐτοῦ, καὶ ἐτελεύτησεν ἐν τῷ τεσσαρακοστῷ ἔτει τῆς βασιλείας αὐτοῦ.
\VS{14}Καὶ ἔθαψαν αὐτὸν ἐν τῷ μνήματι, ᾧ ὤρυξεν ἑαυτῷ ἐν πόλει Δαυὶδ, καὶ ἐκοίμισαν αὐτὸν ἐπὶ τῆς κλίνης, καὶ ἔπλησαν ἀρωμάτων καὶ γένη μύρων μυρεψῶν, καὶ ἐποίησαν αὐτῷ ἐκφορὰν μεγάλην ἕως σφόδρα.

\par }\Chap{17}{\PP \VerseOne{1}Καὶ ἐβασίλευσεν Ἰωσαφὰτ υἱὸς αὐτοῦ ἀντʼ αὐτοῦ, καὶ κατίσχυσεν Ἰωσαφὰτ ἐπὶ τὸν Ἰσραήλ.
\VS{2}Καὶ ἔδωκε δύναμιν ἐν πάσαις ταῖς πόλεσιν Ἰούδα ταῖς ὀχυραῖς, καὶ κατέστησεν ἡγουμένους ἐν πάσαις ταῖς πόλεσιν Ἰούδα, καὶ ἐν πόλεσιν Ἐφραὶμ, ἃς προκατελάβετο Ἀσὰ ὁ πατὴρ αὐτοῦ.
\VS{3}Καὶ ἐγένετο Κύριος μετὰ Ἰωσαφὰτ, ὅτι ἐπορεύθη ἐν ὁδοῖς τοῦ πατρὸς αὐτοῦ ταῖς πρώταις, καὶ οὐκ ἐξεζήτησε τὰ εἴδωλα,
\VS{4}ἀλλὰ Κύριον τὸν Θεὸν τοῦ πατρὸς αὐτοῦ ἐξεζήτησε, καὶ ἐν ταῖς ἐντολαῖς τοῦ πατρὸς αὐτοῦ ἐπορεύθη, καὶ οὐχ ὡς τὰ ἔργα τοῦ Ἰσραήλ.
\VS{5}Καὶ κατεύθυνε Κύριος τὴν βασιλείαν ἐν χειρὶ αὐτοῦ, καὶ ἔδωκε πᾶς Ἰούδα δῶρα τῷ Ἰωσαφὰτ, καὶ ἐγένετο αὐτῷ πλοῦτος καὶ δόξα πολλή.
\VS{6}Καὶ ὑψώθη ἡ καρδία αὐτοῦ ἐν ὁδῷ Κυρίου, καὶ ἐξῇρε τὰ ὑψηλὰ καὶ τὰ ἄλση ἀπὸ τῆς γῆς Ἰούδα.
\par }{\PP \VS{7}Καὶ ἐν τῷ ἔτει τῷ τρίτῳ ἔτει τῆς βασιλείας αὐτοῦ ἀπέστειλε τοὺς ἡγουμένους αὐτοῦ καὶ τοὺς υἱοὺς τῶν δυνατῶν, τὸν Ἀβδιὰν, καὶ Ζαχαρίαν, καὶ Ναθαναὴλ, καὶ Μιχαίαν, τοῦ διδάσκειν ἐν πόλεσιν Ἰούδα.
\VS{8}Καὶ μετʼ αὐτῶν οἱ Λευῖται, Σαμαίας, καὶ Ναθανίας, καὶ Ζαβδίας, καὶ Ἀσιὴλ, καὶ Σεμιραμὼθ, καὶ Ἰωνάθαν, καὶ Ἀδωνίας, καὶ Τωβίας, καὶ Τωβαδωνίας, Λευῖται, καὶ οἱ μετʼ αὐτῶν Ἐλισαμὰ, καὶ Ἰωρὰμ, οἱ ἱερεῖς.
\VS{9}Καὶ ἐδίδασκον ἐν Ἰούδα, καὶ μετʼ αὐτῶν βίβλος νόμου Κυρίου, καὶ διῆλθον ἐν ταῖς πόλεσιν Ἰούδα, καὶ ἐδίδασκον τὸν λαόν.
\par }{\PP \VS{10}Καὶ ἐγένετο ἔκστασις Κυρίου ἐπὶ πάσαις ταῖς βασιλείαις τῆς γῆς κύκλῳ Ἰούδα, καὶ οὐκ ἐπολέμουν πρὸς Ἰωσαφάτ.
\VS{11}Καὶ ἀπὸ τῶν ἀλλοφύλων ἔφερον τῷ Ἰωσαφὰτ δῶρα καὶ ἀργύριον καὶ δόματα· καὶ οἱ Ἀραβες ἔφερον αὐτῷ κριοὺς προβάτων ἑπτακισχιλίους ἑπτακοσίους.
\VS{12}Καὶ ἦν Ἰωσαφὰτ πορευόμενος μείζων ἕως εἰς ὕψος, καὶ ᾠκοδόμησεν ἐν τῇ Ἰουδαίᾳ οἰκήσεις καὶ πόλεις ὀχυράς.
\VS{13}Καὶ ἔργα πολλὰ ἐγένετο αὐτῷ ἐν τῇ Ἰουδαίᾳ· καὶ ἄνδρες πολεμισταὶ δυνατοὶ ἰσχύοντες ἐν Ἱερουσαλήμ.
\par }{\PP \VS{14}Καὶ οὗτος ὁ ἀριθμὸς αὐτῶν κατʼ οἴκους πατριῶν αὐτῶν· καὶ τῷ Ἰούδα χιλίαρχοι, Ἔδνας ὁ ἄρχων, καὶ μετʼ αὐτοῦ υἱοὶ δυνατοὶ δυνάμεως τριακόσιαι χιλιάδες·
\VS{15}Καὶ μετʼ αὐτὸν, Ἰωανὰν ὁ ἡγούμενος, καὶ μετʼ αὐτοῦ διακόσιαι ὀγδοήκοντα χιλιάδες·
\VS{16}Καὶ μετʼ αὐτὸν Ἀμασίας ὁ τοῦ Ζαρὶ, ὁ προθυμούμενος τῷ κυρίῳ, καὶ μετʼ αὐτοῦ διακόσιαι χιλιάδες δυνατοὶ δυνάμεως.
\VS{17}Καὶ ἐκ τοῦ Βενιαμὶν δυνατὸς δυνάμεως καὶ Ἐλιαδὰ, καὶ μετʼ αὐτοῦ τοξόται καὶ πελτασταὶ διακόσιαι χιλιάδες·
\VS{18}Καὶ μετʼ αὐτὸν Ἰωζαβὰδ, καὶ μετʼ αὐτοῦ ἑκατὸν ὀγδοήκοντα χιλιάδες δυνατοὶ πολέμου.
\VS{19}Οὗτοι οἱ λειτουργοῦντες τῷ βασιλεῖ, ἐκτὸς ὧν ἔδωκεν ὁ βασιλεὺς ἐν ταῖς πόλεσι ταῖς ὀχυραῖς ἐν πάσῃ τῇ Ἰουδαίᾳ.

\par }\Chap{18}{\PP \VerseOne{1}Καὶ ἐγενήθη τῷ Ἰωσαφὰτ ἔτι πλοῦτος καὶ δόξα πολλὴ, καὶ ἐπεγαμβρεύσατο ἐν οἴκῳ Ἀχαάβ.
\VS{2}Καὶ κατέβη διὰ τέλους ἐτῶν πρὸς Ἀχαὰβ εἰς Σαμάρειαν· καὶ ἔθυσεν αὐτῷ Ἀχαὰβ πρόβατα καὶ μόσχους πολλοὺς, καὶ τῷ λαῷ τῷ μετʼ αὐτοῦ, καὶ ἠγάπα αὐτὸν τοῦ συναναβῆναι μετʼ αὐτοῦ εἰς Ῥαμὼθ τῆς Γαλααδίτιδος.
\VS{3}Καὶ εἶπεν Ἀχαὰβ βασιλεὺς Ἰσραὴλ πρὸς Ἰωσαφὰτ βασιλέα Ἰούδα, εἰ πορεύσῃ μετʼ ἐμοῦ εἰς Ῥαμὼθ τῆς Γαλααδίτιδος; καὶ εἶπεν αὐτῷ, ὡς ἐγὼ, οὕτω καὶ σύ· ὡς ὁ λαός σου, καὶ ὁ λαός μου μετὰ σοῦ εἰς πόλεμον.
\par }{\PP \VS{4}Καὶ εἶπεν Ἰωσαφὰτ πρὸς βασιλέα Ἰσραὴλ, ζήτησον δὴ σήμερον τὸν Κύριον.
\VS{5}Καὶ συνήγαγεν ὁ βασιλεὺς Ἰσραὴλ τοὺς προφήτας τετρακοσίους ἄνδρας, καὶ εἶπεν αὐτοῖς, εἰ πορευθῶ εἰς Ῥαμὼθ Γαλαὰδ εἰς πόλεμον, ἢ ἐπίοχω; καὶ εἶπαν, ἀνάβαινε, καὶ δώσει ὁ Θεὸς εἰς τὰς χεῖρας τοῦ βασιλέως.
\VS{6}Καὶ εἶπεν Ἰωσαφὰτ, οὐκ ἔστιν ὧδε προφήτης τοῦ κυρίου ἔτι, καὶ ἐπιζητήσομεν παρʼ αὐτοῦ;
\VS{7}Καὶ εἶπε βασιλεὺς Ἰσραὴλ πρὸς Ἰωσαφὰτ, ἔτι ἀνὴρ εἷς τοῦ ζητῆσαι τὸν Κύριον διʼ αὐτοῦ, καὶ ἐγὼ ἐμίσησα αὐτὸν, ὅτι οὐκ ἔστι προφητεύων περὶ ἐμοῦ εἰς ἀγαθὰ, ὅτι πᾶσαι αἱ ἡμέραι αὐτοῦ εἰς κακὰ, οὗτος Μιχαίας υἱὸς Ἰεμβλά· καὶ εἶπεν Ἰωσαφὰτ, μὴ λαλείτω ὁ βασιλεὺς οὕτως.
\par }{\PP \VS{8}Καὶ ἐκάλεσεν ὁ βασιλεὺς εὐνοῦχον ἕνα, καὶ εἶπε, τάχος Μιχαίαν υἱὸν Ἰεμβλά.
\VS{9}Καὶ βασιλεὺς Ἰσραὴλ καὶ Ἰωσαφὰτ βασιλεὺς Ἰούδα καθήμενοι ἕκαστος ἐπὶ θρόνου αὐτοῦ, καὶ ἐνδεδυμένοι στολὰς, καθήμενοι ἐν τῷ εὐρυχώρῳ θύρας πύλης Σαμαρείας, καὶ πάντες οἱ προφῆται προεφήτευον ἐναντίον αὐτῶν.
\VS{10}Καὶ ἐποίησεν ἑαυτῷ Σεδεκίας υἱὸς Χαναὰν κέρατα σιδηρᾶ, καὶ εἶπε, τάδε λέγει Κύριος, ἐν τούτοις κερατιεῖς τὴν Συρίαν ἕως ἂν συντελεσθῇ.
\VS{11}Καὶ πάντες οἱ προφῆται προεφήτευον οὕτω, λέγοντες, ἀνάβαινε εἰς Ῥαμὼθ Γαλαὰδ, καὶ εὐοδωθήσῃ, καὶ δώσει Κύριος εἰς χεῖρας τοῦ βασιλέως.
\par }{\PP \VS{12}Καὶ ὁ ἄγγελος ὁ πορευθεὶς τοῦ καλέσαι τὸν Μιχαίαν, ἐλάλησεν αὐτῷ, λέγων, ἰδοὺ ἐλάλησαν οἱ προφῆται ἐν στόματι ἑνὶ ἀγαθὰ περὶ τοῦ βασιλέως, καὶ ἔστωσαν δὴ οἱ λόγοι σου ὡς ἑνὸς αὐτῶν, καὶ λαλήσεις ἀγαθά.
\VS{13}Καὶ εἶπε Μιχαίας, ζῇ Κύριος, ὅτι ὃ ἐὰν εἴπῃ ὁ Θεὸς πρὸς μὲ, αὐτὸ λαλήσω.
\par }{\PP \VS{14}Καὶ ἦλθε πρὸς τὸν βασιλέα, καὶ εἶπεν αὐτῷ ὁ βασιλεὺς, Μιχαία, εἰ πορευθῶ εἰς Ῥαμὼθ Γαλαὰδ εἰς πόλεμον, ἢ ἐπίσχω; καὶ εἶπεν, ἀνάβαινε, καὶ εὐοδώσεις, καὶ δοθήσονται εἰς χεῖρας ὑμῶν.
\VS{15}Καὶ εἶπεν αὐτῷ ὁ βασιλεὺς, ποσάκις ὁρκίζω σε ἵνα μὴ λαλήσῃς πρὸς μὲ πλὴν τὴν ἀλήθειαν ἐν ὀνόματι Κυρίου;
\VS{16}Καὶ εἶπεν, εἶδον τὸν Ἰσραὴλ διεσπαρμένους ἐν τοῖς ὄρεσιν, ὡς πρόβατα οἷς οὐκ ἔστι ποιμήν· καὶ εἶπε Κύριος, οὐκ ἔχουσιν ἡγούμενον οὗτοι, ἀναστρεφέτωσαν ἕκαστος εἰς τὸν οἶκον αὐτοῦ ἐν εἰρήνῃ.
\par }{\PP \VS{17}Καὶ εἶπεν ὁ βασιλεὺς Ἰσραὴλ πρὸς Ἰωσαφὰτ, οὐκ εἶπόν σοι, ὅτι οὐ προφητεύει περὶ ἐμοῦ ἀγαθὰ ἀλλʼ ἢ κακά;
\VS{18}Καὶ εἶπεν, οὐχ οὕτως· ἀκούσατε λόγον Κυρίου. εἶδον τὸν Κύριον καθήμενον ἐπὶ θρόνου αὐτοῦ, καὶ πᾶσα δύναμις τοῦ οὐρανοῦ παρειστήκει ἐκ δεξιῶν αὐτοῦ καὶ ἐξ ἀριστερῶν αὐτοῦ.
\VS{19}Καὶ εἶπε Κύριος, τίς ἀπατήσει τὸν Ἀχαὰβ βασιλέα Ἰσραὴλ, καὶ ἀναβήσεται, καὶ πεσεῖται ἐν Ῥαμὼθ Γαλαάδ; καὶ οὗτος εἶπεν οὕτως, καὶ οὗτος εἶπεν οὕτως.
\VS{20}Καὶ ἐξῆλθε τὸ πνεῦμα καὶ ἔστη ἐνώπιον Κυρίου, καὶ εἶπεν, ἐγὼ ἀπατήσω αὐτόν· καὶ εἶπε Κύριος, ἐν τίνι;
\VS{21}Καὶ εἶπεν, ἐξελεύσομαι καὶ ἔσομαι πνεῦμα ψευδὲς ἐν στόματι πάντων τῶν προφητῶν αὐτοῦ. καὶ εἶπεν, ἀπατήσεις καὶ δυνήσῃ, ἔξελθε καὶ ποίησον οὕτω.
\VS{22}Καὶ νῦν ἰδοὺ ἔδωκε Κύριος πνεῦμα ψευδὲς ἐν στόματι τῶν προφητῶν σου τούτων, καὶ Κύριος ἐλάλησεν ἐπὶ σὲ κακά.
\par }{\PP \VS{23}Καὶ ἤγγισε Σεδεκίας υἱὸς Χαναὰν, καὶ ἐπάταξε τὸν Μιχαίαν ἐπὶ τὴν σιαγόνα, καὶ εἶπεν αὐτῷ, ποίᾳ τῇ ὁδῷ παρῆλθε πνεῦμα Κυρίου παρʼ ἐμοῦ τοῦ λαλῆσαι πρὸς σέ;
\VS{24}Καὶ εἶπε Μιχαίας, ἰδοὺ ὄψῃ ἐν τῇ ἡμέρᾳ ἐκείνῃ, ἐν ᾗ εἰσελεύσῃ ταμεῖον ἐκ ταμείου τοῦ κατακρυβῆναι.
\par }{\PP \VS{25}Καὶ εἶπε βασιλεὺς Ἰσραὴλ, λάβετε τὸν Μιχαίαν καὶ ἀποστρέψατε πρὸς Ἐμὴρ ἄρχοντα τῆς πόλεως, καὶ πρὸς Ἰωὰς ἄρχοντα υἱὸν τοῦ βασιλέως,
\VS{26}καὶ ἐρεῖτε, οὕτως εἶπεν ὁ βασιλεὺς, ἀπόθεσθε τοῦτον εἰς οἶκον φυλακῆς, καὶ ἐσθιέτω ἄρτον θλίψεως καὶ ὕδωρ θλίψεως, ἕως τοῦ ἐπιστρέψαι με ἐν εἰρήνῃ.
\VS{27}Καὶ εἶπε Μιχαίας, ἐὰν ἐπιστρέφων ἐπιστρέψῃς ἐν εἰρήνῃ, οὐκ ἐλάλησε Κύριος ἐν ἐμοί· καὶ εἶπεν, ἀκούσατε λαοὶ πάντες.
\par }{\PP \VS{28}Καὶ ἀνέβη βασιλεὺς Ἰσραὴλ, καὶ Ἰωσαφὰτ βασιλεὺς Ἰούδα, εἰς Ῥαμὼθ Γαλαάδ.
\VS{29}Καὶ εἶπε βασιλεὺς Ἰσραὴλ πρὸς Ἰωσαφὰτ, κατακάλυψόν με, καὶ εἰσελεύσομαι εἰς τὸν πόλεμον, καὶ σὺ ἔνδυσαι τὸν ἱματισμόν μου· καὶ συνεκαλύψατο βασιλεὺς Ἰσραὴλ, καὶ εἰσῆλθεν εἰς τὸν πόλεμον.
\VS{30}Καὶ βασιλεὺς Συρίας ἐνετείλατο τοῖς ἄρχουσι τῶν ἁρμάτων τοῖς μετʼ αὐτοῦ, λέγων, μὴ πολεμεῖτε τὸν μικρὸν καὶ τὸν μέγαν, ἀλλʼ ἢ τὸν βασιλέα Ἰσραὴλ μόνον.
\VS{31}Καὶ ἐγένετο ὡς εἶδον οἱ ἄρχοντες τῶν ἁρμάτων τὸν Ἰωσαφὰτ, καὶ αὐτοὶ εἶπαν, βασιλεὺς Ἰσραήλ ἐστι, καὶ ἐκύκλωσαν αὐτὸν τοῦ πολεμεῖν· καὶ ἐβόησεν Ἰωσαφὰτ, καὶ Κύριος ἔσωσεν αὐτὸν, καὶ ἀπέστρεψεν αὐτοὺς ὁ Θεὸς ἀπʼ αὐτοῦ.
\VS{32}Καὶ ἐγένετο ὡς εἶδον οἱ ἄρχοντες τῶν ἁρμάτων ὅτι οὐκ ἦν βασιλεὺς Ἰσραὴλ, καὶ ἀπέστρεψαν ἀπʼ αὐτοῦ.
\VS{33}Καὶ ἀνὴρ ἔτεινε τόξον εὐστόχως, καὶ ἐπάταξε τὸν βασιλέα Ἰσραὴλ ἀναμέσον τοῦ πνεύμονος καὶ ἀναμέσον τοῦ θώρακος· καὶ εἶπε τῷ ἡνίοχῳ, ἐπίστρεφε τὴν χεῖρά σου, ἐξάγαγέ με ἐκ τοῦ πολέμου, ὅτι ἐπόνεσα.
\VS{34}Καὶ ἐτροπώθη ὁ πόλεμος ἐν τῇ ἡμέρᾳ ἐκείνῃ, καὶ ὁ βασιλεὺς Ἰσραὴλ ἦν ἑστηκὼς ἐπὶ τοῦ ἅρματος ἐξεναντίας Συρίας ἕως ἑσπέρας, καὶ ἀπέθανε δύνοντος τοῦ ἡλίου.

\par }\Chap{19}{\PP \VerseOne{1}Καὶ ἐπέστρεψεν Ἰωσαφὰτ βασιλεὺς Ἰούδα εἰς τὸν οἶκον αὐτοῦ εἰς Ἱερουσαλήμ.
\VS{2}Καὶ ἐξῆλθεν εἰς ἀπάντησιν αὐτοῦ Ἰηοὺ ὁ τοῦ Ἀνανὶ ὁ προφήτης, καὶ εἶπεν αὐτῷ, βασιλεὺς Ἰωσαφὰτ, εἰ ἁμαρτωλῷ σὺ βοηθεῖς, ἢ μισουμένῳ ὑπὸ Κυρίου φιλιάζεις; διὰ τοῦτο ἐγένετο ἐπὶ σὲ ὀργὴ παρὰ Κυρίου.
\VS{3}Ὅτι ἀλλʼ ἢ λόγοι ἀγαθοὶ ηὑρέθησαν ἐν σοὶ, ὅτι ἐξῇρας τὰ ἄλση ἀπὸ τῆς γῆς Ἰούδα, καὶ κατηύθυνας τὴν καρδίαν σου ἐκζητῆσαι τὸν Κύριον.
\par }{\PP \VS{4}Καὶ κατῴκησεν Ἰωσαφὰτ εἰς Ἱερουσαλήμ· καὶ πάλιν ἐξῆλθεν εἰς τὸν λαὸν ἀπὸ Βηρσαβεὲ ἕως ὄρους Ἐφραὶμ, καὶ ἐπέστρεψεν αὐτοὺς ἐπὶ Κύριον Θεὸν τῶν πάτερων αὐτῶν.
\VS{5}Καὶ κατέστησε τοὺς κριτὰς ἐν πάσαις ταῖς πόλεσιν Ἰούδα ταῖς ὀχυραῖς, ἐν πόλει καὶ πόλει.
\VS{6}Καὶ εἶπε τοῖς κριταῖς, ἴδετε τί ὑμεῖς ποιεῖτε, ὅτι οὐκ ἀνθρώπῳ ὑμεῖς κρίνετε, ἀλλʼ ἢ τῷ Κυρίῳ, καὶ μεθʼ ὑμῶν λόγοι τῆς κρίσεως.
\VS{7}Καὶ νῦν γενέσθω φόβος Κυρίου ἐφʼ ὑμᾶς, καὶ φυλάσσετε καὶ ποιήσατε, ὅτι οὐκ ἔστι μετὰ Κυρίου Θεοῦ ἡμῶν ἀδικία, οὐδὲ θαυμάσαι πρόσωπον, οὐδὲ λαβεῖν δῶρα.
\par }{\PP \VS{8}Καί γε ἐν Ἱερουσαλὴμ κατέστησεν Ἰωσαφὰτ τῶν ἱερέων καὶ τῶν Λευιτῶν καὶ τῶν πατριαρχῶν Ἰσραὴλ εἰς κρίσιν Κυρίου, καὶ κρίνειν τοὺς κατοικοῦντας ἐν Ἱερουσαλήμ.
\VS{9}Καὶ ἐνετείλατο πρὸς αὐτοὺς, λέγων, οὕτω ποιήσετε ἐν φόβῳ Κυρίου, ἐν ἀληθείᾳ, καὶ ἐν πλήρει καρδίᾳ·
\VS{10}Πᾶς ἀνὴρ κρίσιν τὴν ἐλθοῦσαν ἐφʼ ὑμᾶς τῶν ἀδελφῶν ὑμῶν τῶν κατοικούντων ἐν ταῖς πόλεσιν αὐτῶν ἀναμέσον αἷμα αἵματος, καὶ ἀναμέσον τοῦ προστάγματος καὶ ἐντολῆς, καὶ δικαιώματα καὶ κρίματα, καὶ διαστελεῖσθε αὐτοῖς, καὶ οὐχ ἁμαρτήσονται τῷ κυρίῳ, καὶ οὐκ ἔσται ὀργὴ ἐφʼ ὑμᾶς, καὶ ἐπὶ τοὺς ἀδελφοὺς ὑμῶν· οὕτως ποιήσετε, καὶ οὐχ ἁμαρτήσεσθε.
\VS{11}Καὶ ἰδοὺ Ἀμαρίας ὁ ἱερεὺς ἡγούμενος ἐφʼ ὑμᾶς εἰς πάντα λόγον Κυρίου, καὶ Ζαβδίας υἱὸς Ἰσμαὴλ ὁ ἡγούμενος εἰς οἶκον Ἰούδα πρὸς πάντα λόγον βασιλέως, καὶ οἱ γραμματεῖς καὶ οἱ Λευῖται πρὸ προσώπου ὑμῶν· ἰσχύσατε καὶ ποιήσατε, καὶ ἔσται Κύριος μετὰ τοῦ ἀγαθοῦ.

\par }\Chap{20}{\PP \VerseOne{1}Καὶ μετὰ ταῦτα ἦλθον οἱ υἱοὶ Μωὰβ, καὶ υἱοὶ Ἀμμὼν, καὶ μετʼ αὐτῶν ἐκ τῶν Μιναίων πρὸς Ἰωσαφὰτ εἰς πόλεμον.
\VS{2}Καὶ ἦλθον καὶ ὑπέδειξαν τῷ Ἰωσαφὰτ, λέγοντες, ἥκει ἐπὶ σὲ πλῆθος πολὺ ἐκ πέραν τῆς θαλάσσης ἀπὸ Συρίας, καὶ ἰδού εἰσιν ἐν Ἀσασὰν Θαμὰρ, αὕτη ἐστὶν Ἐγγαδί.
\VS{3}Καὶ ἐφοβήθη, καὶ ἔδωκε Ἰωσαφὰτ πρόσωπον αὐτοῦ ἐκζητῆσαι τὸν Κύριον, καὶ ἐκήρυξε νηστείαν ἐν παντὶ Ἰούδα.
\VS{4}Καὶ συνήχθη Ἰούδα ἐκζητῆσαι τὸν Κύριον, καὶ ἀπὸ πασῶν τῶν πόλεων Ἰούδα ἦλθον ζητῆσαι τὸν Κύριον.
\par }{\PP \VS{5}Καὶ ἀνέστη Ἰωσαφὰτ ἐν ἐκκλησίᾳ Ἰούδα ἐν Ἱερουσαλὴμ ἐν οἴκῳ Κυρίου κατὰ πρόσωπον τῆς αὐλῆς τῆς καινῆς.
\VS{6}Καὶ εἶπε, Κύριε ὁ Θεὸς τῶν πατέρων μου, οὐχὶ σὺ εἶ Θεὸς ἐν οὐρανῷ ἄνω, καὶ σὺ κυριεύεις πασῶν τῶν βασιλειῶν τῶν ἐθνῶν; καὶ ἐν τῇ χειρί σου ἰσχὺς δυναστείας, καὶ οὐκ ἔστι πρὸς σὲ ἀντιστῆναι;
\VS{7}Οὐχὶ σὺ ὁ Κύριος ὁ ἐξολοθρεύσας τοὺς κατοικοῦντας τὴν γῆν ταύτην ἀπὸ προσώπου τοῦ λαοῦ σου Ἰσραὴλ, καὶ ἔδωκας αὐτὴν σπέρματι Ἁβραὰμ τῷ ἠγαπημένῳ σου εἰς τὸν αἰῶνα;
\VS{8}Καὶ κατῴκησαν ἐν αὐτῇ, καὶ ᾠκοδόμησαν ἐν αὐτῇ ἁγίασμα τῷ ὀνόματί σου, λέγοντες,
\VS{9}ἐὰν ἐπέλθῃ ἐφʼ ἡμᾶς κακὰ, ῥομφαία, κρίσις, θάνατος, λιμὸς, στησόμεθα ἐναντίον τοῦ οἴκου τούτου καὶ ἐναντίον σου, ὅτι τὸ ὄνομά σου ἐπὶ τῷ οἴκῳ τούτῳ, καὶ βοησόμεθα πρὸς σὲ ἀπὸ τῆς θλίψεως, καὶ ἀκούσῃ καὶ σώσεις.
\VS{10}Καὶ νῦν ἰδοὺ οἱ υἱοὶ Ἀμμὼν, καὶ Μωὰβ, καὶ ὄρος Σηείρ· εἰς οὓς οὐκ ἔδωκας τῷ Ἰσραὴλ διελθεῖν διʼ αὐτῶν, ἐξελθόντων αὐτῶν ἐκ γῆς Αἰγύπτου, ὅτι ἐξέκλιναν ἀπʼ αὐτῶν, καὶ οὐκ ἐξωλόθρευσαν αὐτούς·
\VS{11}Καὶ νῦν ἰσοὺ αὐτοὶ ἐπιχειροῦσιν ἐφʼ ἡμᾶς ἐξελθεῖν ἐκβαλεῖν ἡμᾶς ἀπὸ τῆς κληρονομίας ἡμῶν, ἧς ἔδωκας ἡμῖν.
\VS{12}Κύριε ὁ Θεὸς ἡμῶν, οὐ κρινεῖς ἐν αὐτοῖς; ὅτι οὐκ ἔστιν ἡμῖν ἰσχὺς τοῦ ἀντιστῆναι πρὸς τὸ πλῆθος τὸ πολὺ τοῦτο τὸ ἐλθὸν ἐφʼ ἡμᾶς, καὶ οὐκ οἴδαμεν τί ποιήσωμεν αὐτοῖς, ἀλλʼ ἢ ἐπὶ σοὶ οἱ ὀφθαλμοὶ ἡμῶν.
\par }{\PP \VS{13}Καὶ πᾶς Ἰούδα ἑστηκὼς ἔναντι Κυρίου, καὶ τὰ παιδία αὐτῶν, καὶ αἱ γυναῖκες αὐτῶν.
\VS{14}Καὶ τῷ Ὀζιὴλ τῷ τοῦ Ζαχαρίου, τῶν υἱῶν Βαναίου, τῶν υἱῶν Ἐλεϊὴλ τοῦ Ματθανίου τοῦ Λευίτου ἀπὸ τῶν υἱῶν Ἀσὰφ, ἐγένετο ἐπʼ αὐτὸν πνεῦμα Κυρίου ἐν τῇ ἐκκλησίᾳ.
\VS{15}Καὶ εἶπεν, ἀκούσατε πᾶς Ἰούδα καὶ οἱ κατοικοῦντες ἐν Ἱερουσαλὴμ καὶ ὁ βασιλεὺς Ἰωσαφάτ· τάδε λέγει Κύριος ὑμῖν αὐτοῖς, μὴ φοβεῖσθε, μηδὲ πτοηθῆτε ἀπὸ προσώπου τοῦ ὄχλου τοῦ πολλοῦ τούτου, ὅτι οὐχ ὑμῖν ἐστιν ἡ παράταξις, ἀλλʼ ἢ τῷ Θεῷ.
\VS{16}Αὔριον κατάβητε ἐπʼ αὐτούς· ἰδοὺ ἀναβαίνουσι κατὰ τὴν ἀνάβασιν Ἀσσεῖς, καὶ εὑρήσετε αὐτοὺς ἐπʼ ἄκρου ποταμοῦ τῆς ἐρήμου Ἰεριήλ.
\VS{17}Οὐχ ὑμῖν ἐστι πολεμῆσαι· ταῦτα σύνετε, καὶ ἴδετε τὴν σωτηρίαν Κυρίου μεθʼ ὑμῶν Ἰούδα καὶ Ἱερουσαλήμ· μὴ φοβηθῆτε, μηδὲ πτοηθῆτε αὔριον ἐξελθεῖν εἰς ἀπάντησιν αὐτοῖς, καὶ Κύριος μεθʼ ὑμῶν.
\VS{18}Καὶ κύψας Ἰωσαφὰτ ἐπὶ πρόσωπον αὐτοῦ καὶ πᾶς Ἰούδα καὶ οἱ κατοικοῦντες Ἱερουσαλὴμ, ἔπεσον ἔναντι Κυρίου προσκυνῆσαι Κυρίῳ.
\VS{19}Καὶ ἀνέστησαν οἱ Λευῖται ἀπὸ τῶν υἱῶν Καὰθ, καὶ ἀπὸ τῶν υἱῶν Κορὲ, αἰνεῖν Κυρίῳ Θεῷ Ἰσραὴλ ἐν φωνῇ μεγάλῃ εἰς ὕψος.
\par }{\PP \VS{20}Καὶ ὤρθρισαν πρωῒ, καὶ ἐξῆλθον εἰς τὴν ἔρημον Θεκωέ· καὶ ἐν τῷ ἐξελθεῖν αὐτοὺς, ἔστη Ἰωσαφὰτ καὶ ἐβόησε, καὶ εἶπεν, ἀκούσατέ μου Ἰούδα καὶ οἱ κατοικοῦντες ἐν Ἱερουσαλήμ· ἐμπιστεύσατε ἐν Κυρίῳ Θεῷ ἡμῶν, καὶ ἐμπιστευθήσεσθε· ἐμπιστεύσατε ἐν προφήτῃ αὐτοῦ, καὶ εὐοδωθήσεσθε.
\VS{21}Καὶ ἐβουλεύσατο μετὰ τοῦ λαοῦ, καὶ ἔστησε ψαλτῳδοὺς καὶ αἰνοῦντας, ἐξομολογεῖσθαι καὶ αἰνεῖν τὰ ἅγια ἐν τῷ ἐξελθεῖν ἔμπροσθεν τῆς δυνάμεως, καὶ ἔλεγον, ἐξομολογεῖσθε τῷ Κυρίῳ, ὅτι εἰς τὸν αἰῶνα τὸ ἔλεος αὐτοῦ.
\par }{\PP \VS{22}Καὶ ἐν τῷ ἄρξασθαι αὐτοὺς τῆς αἰνέσεως καὶ τῆς ἐξομολογήσεως, ἔδωκε Κύριος πολεμεῖν τοὺς υἱοὺς Ἀμμὼν ἐπὶ Μωὰβ καὶ ὄρος Σηεὶρ τοὺς ἐξελθόντας ἐπὶ Ἰούδαν, καὶ ἐτροπώθησαν.
\VS{23}Καὶ ἀνέστησαν οἱ υἱοὶ Ἀμμὼν καὶ Μωὰβ ἐπὶ τοὺς κατοικοῦντας ὄρος Σηεὶρ, ἐξολοθρεῦσαι καὶ ἐκτρίψαι αὐτούς· καὶ ὡς συνετέλεσαν τοὺς κατοικοῦντας Σηεὶρ, ἀνέστησαν εἰς ἀλλήλους τοῦ ἐξολοθρευθῆναι.
\par }{\PP \VS{24}Καὶ Ἰούδας ἦλθεν ἐπὶ τὴν σκοπιὰν τῆς ἐρήμου, καὶ ἐπέβλεψε, καὶ εἶδε τὸ πλῆθος· καὶ ἰδοὺ πάντες νεκροὶ πεπτωκότες ἐπὶ τῆς γῆς, οὐκ ἦν σωζόμενος.
\VS{25}Καὶ ἐξῆλθεν Ἰωσαφὰτ καὶ ὁ λαὸς αὐτοῦ σκυλεῦσαι τὰ σκῦλα αὐτῶν, καὶ εὗρον κτήνη πολλὰ, καὶ ἀποσκευὴν, καὶ σκῦλα, καὶ σκεύη ἐπιθυμητὰ, καὶ ἐσκύλευσαν ἐν αὐτοῖς· καὶ ἐγένοντο ἡμέραι τρεῖς σκυλευόντων αὐτῶν τὰ σκῦλα, ὅτι πολλὰ ἦν.
\VS{26}Καὶ ἐγένετο τῇ ἡμέρᾳ τῇ τετάρτῃ ἐπισυνήχθησαν εἰς τὸν αὐλῶνα τῆς εὐλογίας, ἐκεῖ γὰρ ηὐλόγησαν τὸν Κύριον· διὰ τοῦτο ἐκάλεσαν τὸ ὄνομα τοῦ τόπου ἐκείνου Κοιλὰς Εὐλογίας, ἕως τῆς ἡμέρας ταύτης.
\par }{\PP \VS{27}Καὶ ἐπέστρεψε πᾶς ἀνὴρ Ἰούδα εἰς Ἱερουσαλὴμ, καὶ Ἰωσαφὰτ ἡγούμενος αὐτῶν ἐν εὐφροσύνῃ μεγάλῃ, ὅτι εὔφραινεν αὐτοὺς Κύριος ἀπὸ τῶν ἐχθρῶν αὐτῶν.
\VS{28}Καὶ εἰσῆλθον εἰς Ἱερουσαλὴμ ἐν νάβλαις καὶ κινύραις καὶ ἐν σάλπιγξιν εἰς οἶκον Κυρίου.
\VS{29}Καὶ ἐγένετο ἔκστασις Κυρίου ἐπὶ πάσας τὰς βασιλείας τῆς γῆς, ἐν τῷ ἀκοῦσαι αὐτοὺς ὅτι Κύριος ἐπολέμησε πρὸς τοὺς ὑπεναντίους Ἰσραήλ.
\VS{30}Καὶ εἰρήνευσεν ἡ βασιλεία Ἰωσαφὰτ, καὶ κατέπαυσεν αὐτῷ ὁ Θεὸς αὐτοῦ κυκλόθεν.
\par }{\PP \VS{31}Καὶ ἐβασίλευσεν Ἰωσαφὰτ ἐπὶ τὸν Ἰούδαν, ὢν ἐτῶν τριακονταπέντε ἐν τῷ βασιλεῦσαι αὐτὸν, καὶ εἴκοσι καὶ πέντε ἔτη ἐβασίλευσεν ἐν Ἱερουσαλὴμ, καὶ ὄνομα τῇ μητρὶ αὐτοῦ Ἀζουβὰ, θυγάτηρ Σαλί.
\VS{32}Καὶ ἐπορεύθη ἐν ταῖς ὁδοῖς τοῦ πατρὸς αὐτοῦ Ἀσὰ, καὶ οὐκ ἐξέκλινε τοῦ ποιῆσαι τὸ εὐθὲς ἐνώπιον Κυρίου.
\VS{33}Ἀλλὰ καὶ τὰ ὑψηλὰ ἔτι ὑπῆρχε, καὶ ἔτι ὁ λαὸς οὐ κατεύθυνε τὴν καρδίαν αὐτῶν πρὸς Κύριον τὸν Θεὸν τῶν πατέρων αὐτῶν.
\par }{\PP \VS{34}Καὶ οἱ λοιποὶ λόγοι Ἰωσαφὰτ οἱ πρῶτοι καὶ οἱ ἔσχατοι, ἰδοὺ γεγραμμένοι ἐν λόγοις Ἰηοὺ τοῦ Ἀνανὶ, ὃς κατέγραψε βιβλίον βασιλέων Ἰσραήλ.
\par }{\PP \VS{35}Καὶ μετὰ ταῦτα ἐκοινώνησεν Ἰωσαφὰτ βασιλεὺς Ἰούδα πρὸς Ὀχοζίαν βασιλέα Ἰσραὴλ, καὶ οὗτος ἠνόμησεν·
\VS{36}ἐν τῷ ποιῆσαι καὶ πορευθῆναι πρὸς αὐτὸν, τοῦ ποιῆσαι πλοῖα τοῦ πορευθῆναι εἰς Θαρσεῖς· καὶ ἐποίησε πλοῖα ἐν Γασίων Γαβέρ.
\VS{37}Καὶ προεφήτευσεν Ἐλιέζερ ὁ τοῦ Δωδία ἀπὸ Μαρισὴς ἐπὶ Ἰωσαφὰτ, λέγων, ὡς ἐφιλίασας τῷ Ὀχοζίᾳ, ἔθραυσε Κύριος τὸ ἔργον σου, καὶ συνετρίβη τὰ πλοῖά σου· καὶ οὐκ ἐδυνάσθη πορευθῆναι εἰς Θαρσεῖς.

\par }\Chap{21}{\PP \VerseOne{1}Καὶ ἐκοιμήθη Ἰωσαφὰτ μετὰ τῶν πατέρων αὐτοῦ, καὶ ἐτάφη ἐν πόλει Δαυίδ. Καὶ ἐβασίλευσεν Ἰωρὰν υἱὸς αὐτοῦ ἀντʼ αὐτοῦ.
\VS{2}Καὶ αὐτῷ ἀδελφοὶ υἱοὶ Ἰωσαφὰτ ἓξ, Ἀζαρίας, καὶ Ἰεϊὴλ, καὶ Ζαχαρίας, καὶ Ἀζαρίας, καὶ Μιχαὴλ, καὶ Ζαφατίας· πάντες οὗτοι υἱοὶ Ἰωσαφὰτ βασιλέως Ἰούδα.
\VS{3}Καὶ ἔδωκεν αὐτοῖς ὁ πατὴρ αὐτῶν δόματα πολλὰ, ἀργύριον καὶ χρυσίον, καὶ ὅπλα μετὰ τῶν πόλεων τετειχισμένων ἐν Ἰούδα, καὶ τὴν βασιλείαν ἔδωκε τῷ Ἰωρὰμ, ὅτι οὗτος ὁ πρωτότοκος.
\VS{4}Καὶ ἀνέστη Ἰωρὰμ ἐπὶ τὴν βασιλείαν αὐτοῦ, καὶ ἐκραταιώθη, καὶ ἀπέκτεινε πάντας τοὺς ἀδελφοὺς αὐτοῦ ἐν ῥομφαίᾳ, καὶ ἀπὸ τῶν ἀρχόντων Ἰσραήλ.
\par }{\PP \VS{5}Ὄντος αὐτοῦ τριάκοντα καὶ δύο ἐτῶν, κατέστη Ἰωρὰμ ἐπὶ τὴν βασιλείαν αὐτοῦ, καὶ ὀκτὼ ἔτη ἐβασίλευσεν ἐν Ἱερουσαλήμ.
\VS{6}Καὶ ἐπορεύθη ἐν ὁδῷ βασιλέων Ἰσραὴλ, ὡς ἐποίησεν οἶκος Ἀχαὰβ, ὅτι θυγάτηρ Ἀχαὰβ ἦν αὐτοῦ γυνὴ, καὶ ἐποίησε τὸ πονηρὸν ἐναντίον Κυρίου.
\VS{7}Καὶ οὐκ ἐβούλετο Κύριος ἐξολοθρεῦσαι τὸν οἶκον Δαυὶδ, διὰ τὴν διαθήκην ἣν διέθετο τῷ Δαυὶδ, καὶ ὡς εἶπεν αὐτῷ δοῦναι αὐτῷ λύχνον καὶ τοῖς υἱοῖς αὐτοῦ πάσας τὰς ἡμέρας.
\par }{\PP \VS{8}Ἐν ταῖς ἡμέραις ἐκείναις ἀπέστη Ἐδὼμ ἀπὸ τοῦ Ἰούδα, καὶ ἐβασίλευσαν ἐφʼ ἑαυτοὺς βασιλέα.
\VS{9}Καὶ ᾤχετο Ἰωρὰμ μετὰ τῶν ἀρχόντων, καὶ πᾶσα ἡ ἵππος μετʼ αὐτοῦ· καὶ ἐγένετο καὶ ἠγέρθη νυκτὸς, καὶ ἐπάταξεν Ἐδὼμ τὸν κυκλοῦντα αὐτὸν, καὶ τοὺς ἄρχοντας τῶν ἁρμάτων, καὶ ἔφυγεν ὁ λαὸς εἰς τὰ σκηνώματα αὐτῶν.
\VS{10}Καὶ ἀπέστη ἀπὸ Ἰούδα Ἐδὼμ ἕως τῆς ἡμέρας ταύτης· τότε ἀπέστη Λομνὰ ἐν τῷ καιρῷ ἐκείνῳ ἀπὸ χειρὸς αὐτοῦ, ὅτι ἐγκατέλιπε Κύριον τὸν Θεὸν τῶν πατέρων αὐτοῦ.
\VS{11}Καὶ γὰρ αὐτὸς ἐποίησεν ὑψηλὰ ἐν ταῖς πόλεσιν Ἰούδα, καὶ ἐξεπόρνευσε τοὺς κατοικοῦντας ἐν Ἱερουσαλὴμ, καὶ ἀπεπλάνησε τὸν Ἰούδαν.
\par }{\PP \VS{12}Καὶ ἦλθεν αὐτῷ ἐν γραφῇ παρὰ Ἠλιοὺ τοῦ προφήτου, λέγων, τάδε λέγει Κύριος Θεὸς Δαυὶδ τοῦ πατρός σου, ἀνθʼ ὧν οὐκ ἐπορεύθης ἐν ὁδῷ Ἰωσαφὰτ τοῦ πατρός σου, καὶ ἐν ὁδοῖς Ἀσὰ βασιλέως Ἰούδα,
\VS{13}καὶ ἐπορεύθης ἐν ὁδοῖς βασιλέων Ἰσραὴλ, καὶ ἐξεπόρνευσας τὸν Ἰούδαν καὶ τοὺς κατοικοῦντας ἐν Ἱερουσαλὴμ, ὡς ἐξεπόρνευσεν οἶκος Ἀχαὰβ, καὶ τοὺς ἀδελφούς σου υἱοὺς τοῦ πατρός σου τοὺς ἀγαθοὺς ὑπὲρ σὲ ἀπέκτεινας,
\VS{14}ἰδοὺ Κύριος πατάξει σε πληγὴν μεγάλην ἐν τῷ λαῷ σου, καὶ ἐν τοῖς υἱοῖς σου, καὶ ἐν γυναιξί σου, καὶ ἐν πάσῃ τῇ ἀποσκευῇ σου.
\VS{15}Καὶ σὺ ἐν μαλακίᾳ πονηρᾷ, ἐν νόσῳ κοιλίας, ἕως οὗ ἐξέλθῃ ἡ κοιλία σου μετὰ τῆς μαλακίας ἐξ ἡμερῶν εἰς ἡμέρας.
\par }{\PP \VS{16}Καὶ ἐπήγειρε Κύριος ἐπὶ Ἰωρὰμ τοὺς ἀλλοφύλους, καὶ τοὺς Ἄραβας, καὶ τοὺς ὁμόρους τῶν Αἰθιόπων.
\VS{17}Καὶ ἀνέβησαν ἐπὶ Ἰούδαν, καὶ κατεδυνάστευον, καὶ ἀπέστρεψαν πᾶσαν τὴν ἀποσκευὴν ἣν εὗρον ἐν οἴκῳ τοῦ βασιλέως, καὶ τοὺς υἱοὺς αὐτοῦ, καὶ τὰς θυγατέρας αὐτοῦ, καὶ οὐ κατελείφθη αὐτῷ υἱὸς, ἀλλʼ ἢ Ὀχοζίας ὁ μικρότατος τῶν υἱῶν αὐτοῦ.
\VS{18}Καὶ μετὰ ταῦτα πάντα ἐπάταξεν αὐτὸν Κύριος εἰς τὴν κοιλίαν μαλακίαν ᾗ οὐκ ἔστιν ἰατρεία.
\VS{19}Καὶ ἐγένετο ἐξ ἡμερῶν εἰς ἡμέρας· καὶ ὡς ἦλθε καιρὸς τῶν ἡμερῶν ἡμέρας δύο, ἐξῆλθεν ἡ κοιλία αὐτοῦ μετὰ τῆς νόσου, καὶ ἀπέθανεν ἐν μαλακίᾳ πονηρᾷ· καὶ οὐκ ἐποίησεν ὁ λαὸς αὐτοῦ ἐκφορὰν, καθὼς ἐκφορὰν πατέρων αὐτοῦ.
\VS{20}Ἦν τριάκοντα καὶ δύο ἐτῶν ὅτε ἐβασίλευσε, καὶ ὀκτὼ ἔτη ἐβασίλευσεν ἐν Ἱερουσαλήμ· καὶ ἐπορεύθη οὐκ ἐν ἐπαίνῳ, καὶ ἐτάφη ἐν πόλει Δαυὶδ, καὶ οὐκ ἐν τάφοις τῶν βασιλέων.

\par }\Chap{22}{\PP \VerseOne{1}Καὶ ἐβασίλευσαν οἱ κατοικοῦντες ἐν Ἱερουσαλὴμ τὸν Ὀχοζίαν υἱὸν αὐτοῦ τὸν μικρὸν ἀντʼ αὐτοῦ, ὅτι πάντας τοὺς πρεσβυτέρους ἀπέκτεινε τὸ ἐπελθὸν ἐπʼ αὐτοὺς λῃστήριον, οἱ Ἄραβες καὶ οἱ Ἀλιμαζονεῖς· καὶ ἐβασίλευσεν Ὀχοζίας υἱὸς Ἰωρὰμ βασιλέως Ἰούδα.
\par }{\PP \VS{2}Ὢν ἐτῶν εἴκοσι Ὀχοζίας ἐβασίλευσε, καὶ ἐνιαυτὸν ἕνα ἐβασίλευσεν ἐν Ἱερουσαλὴμ, καὶ ὄνομα τῇ μητρὶ αὐτοῦ Γοθολία, θυγάτηρ Ἀμβρί.
\VS{3}Καὶ οὗτος ἐπορεύθη ἐν ὁδῷ οἴκου Ἀχαὰβ, ὅτι μήτηρ αὐτοῦ ἦν σύμβουλος τοῦ ἁμαρτάνειν.
\VS{4}Καὶ ἐποίησε τὸ πονηρὸν ἐναντίον Κυρίου ὡς οἶκος Ἀχαὰβ, ὅτι αὐτοὶ ἦσαν αὐτῷ σύμβουλοι μετὰ τὸ ἀποθανεῖν τὸν πατέρα αὐτοῦ, τοῦ ἐξολοθρεῦσαι αὐτὸν,
\VS{5}καὶ ἐν ταῖς βουλαῖς αὐτῶν ἐπορεύθη· καὶ ἐπορεύθη μετὰ Ἰωρὰμ υἱοῦ Ἀχαὰβ βασιλέως Ἰσραὴλ εἰς πόλεμον ἐπὶ Ἀζαὴλ βασιλέα Συρίας εἰς Ῥαμὼθ Γαλαάδ· καὶ ἐπάταξαν οἱ τοξόται τὸν Ἰωράμ.
\VS{6}Καὶ ἐπέστρεψεν Ἰωρὰμ τοῦ ἰατρευθῆναι εἰς Ἰεζράελ ἀπὸ τῶν πληγῶν ὧν ἐπάταξαν αὐτὸν οἱ Σύροι ἐν Ῥαμὼθ ἐν τῷ πολεμεῖν αὐτὸν πρὸς Ἀζαὴλ βασιλέα Συρίας.
\par }{\PP Καὶ Ὀχοζίας υἱὸς Ἰωρὰμ βασιλεὺς Ἰούδα κατέβη θεάσασθαι τὸν Ἰωρὰμ υἱὸν Ἀχαὰβ εἰς Ἰεζράελ, ὅτι ἠῥῥώστει.
\VS{7}Καὶ παρὰ τοῦ Θεοῦ ἐγένετο καταστροφὴ Ὀχοζία ἐλθεῖν πρὸς Ἰωράμ· καὶ ἐν τῷ ἐλθεῖν αὐτὸν, ἐξῆλθε μετʼ αὐτοῦ Ἰωρὰμ πρὸς Ἰηοὺ υἱὸν Ναμεσσεῒ χριστὸν Κυρίου εἰς τὸν οἶκον Ἀχαάβ.
\par }{\PP \VS{8}Καὶ ἐγένετο ὡς ἐξεδίκησεν Ἰηοὺ τὸν οἶκον Ἀχαὰβ, καὶ εὗρε τοὺς ἄρχοντας Ἰούδα καὶ τοὺς ἀδελφοὺς Ὀχοζίου λειτουργοῦντας τῷ Ὀχοζίᾳ, καὶ ἀπέκτεινεν αὐτούς.
\VS{9}Καὶ εἶπε τοῦ ζητῆσαι τὸν Ὀχοζίαν· καὶ κατέλαβον αὐτὸν ἰατρευόμενον ἐν Σαμαρείᾳ, καὶ ἤγαγον αὐτὸν πρὸς Ἰηοὺ, καὶ ἀπέκτεινεν αὐτὸν· καὶ ἔθαψαν αὐτὸν, ὅτι εἶπαν, υἱὸς Ἰωσαφάτ ἐστιν, ὃς ἐζήτησε τὸν Κύριον ἐν ὅλῃ τῇ καρδίᾳ αὐτοῦ.
\par }{\PP Καὶ οὐκ ἦν ἐν οἴκῳ Ὀχοζίᾳ κατισχῦσαι δύναμιν περὶ τῆς βασιλείας.
\VS{10}Καὶ Γοθολία ἡ μήτηρ Ὀχοζίου εἶδεν ὅτι τέθνηκεν ὁ υἱὸς αὐτῆς, καὶ ἠγέρθη καὶ ἀπώλεσε πᾶν τὸ σπέρμα τῆς βασιλείας ἐν οἴκῳ Ἰούδα.
\VS{11}Καὶ ἔλαβεν Ἰωσαβεὲθ θυγάτηρ τοῦ βασιλέως τὸν Ἰωὰς υἱὸν Ὀχοζίου καὶ ἔκλεψεν αὐτὸν ἐκ μέσου υἱῶν τοῦ βασιλέως τῶν θανατουμένων, καὶ ἔδωκεν αὐτὸν καὶ τὴν τροφὸν αὐτοῦ εἰς ταμεῖον τῶν κλινῶν, καὶ ἔκρυψεν αὐτὸν Ἰωσαβεὲθ θυγάτηρ τοῦ βασιλέως Ἰωρὰμ ἀδελφὴ Ὀχοζίου γυνὴ Ἰωδαὲ τοῦ ἱερέως, καὶ ἔκρυψεν αὐτὸν ἀπὸ προσώπου τῆς Γοθολίας, καὶ οὐκ ἀπέκτεινεν αὐτόν.
\VS{12}Καὶ ἦν μετʼ αὐτοῦ ἐν οἴκῳ τοῦ Θεοῦ κατακεκρυμμένος ἓξ ἔτη, καὶ Γοθολία ἐβασίλευσεν ἐπὶ τῆς γῆς.

\par }\Chap{23}{\PP \VerseOne{1}Καὶ ἐν τῷ ἔτει τῷ ὀγδόῳ ἐκραταίωσεν Ἰωδαὲ, καὶ ἔλαβε τοὺς ἑκατοντάρχους, τὸν Ἀζαρίαν υἱὸν Ἰωρὰμ, καὶ τὸν Ἰσμαὴλ υἱὸν Ἰωανὰν, καὶ τὸν Ἀζαρίαν υἱὸν Ὠβὴδ, καὶ τὸν Μαασαίαν υἱὸν Ἀδία, καὶ τὸν Ἐλισαφὰν υἱὸν Ζαχαρίου, μεθʼ ἑαυτοῦ εἰς οἶκον Κυρίου.
\VS{2}Καὶ ἐκύκλωσαν τὸν Ἰούδαν, καὶ συνήγαγον τοὺς Λευίτας ἐκ πασῶν τῶν πόλεων Ἰούδα, καὶ ἄρχοντας πατριῶν τοῦ Ἰσραὴλ, καὶ ἦλθον εἰς Ἱερουσαλήμ.
\VS{3}Καὶ διέθεντο πᾶσα ἡ ἐκκλησία Ἰούδα διαθήκην ἐν οἴκῳ τοῦ Θεοῦ μετὰ τοῦ βασιλέως· καὶ ἔδειξεν αὐτοῖς τὸν υἱὸν τοῦ βασιλέως, καὶ εἶπεν αὐτοῖς, ἰδοὺ ὁ υἱὸς τοῦ βασιλέως βασιλευσάτω, καθὼς ἐλάλησε Κύριος ἐπὶ τὸν οἶκον Δαυίδ.
\VS{4}Νῦν ὁ λόγος οὗτος, ὃν ποιήσετε· τὸ τρίτον ἐξ ὑμῶν εἰσπορευέσθωσαν τὸ σάββατον τῶν ἱερέων καὶ τῶν Λευιτῶν καὶ εἰς τὰς πύλας τῶν εἰσόδων,
\VS{5}καὶ τὸ τρίτον ἐν οἴκῳ τοῦ βασιλέως, καὶ τὸ τρίτον ἐν τῇ πύλῃ τῇ μέσῃ, καὶ πᾶς ὁ λαὸς ἐν αὐλαῖς οἴκου Κυρίου.
\VS{6}Καὶ μὴ εἰσελθέτω εἰς οἶκον Κυρίου, ἐὰν μὴ οἱ ἱερεῖς καὶ οἱ Λευῖται καὶ οἱ λειτουργοῦντες τῶν Λευιτῶν· αὐτοὶ εἰσελεύσονται ὅτι ἅγιοί εἰσι, καὶ πᾶς ὁ λαὸς φυλασσέτω φυλακὰς Κυρίου.
\VS{7}Καὶ κυκλώσουσιν οἱ Λευῖται τὸν βασιλέα κύκλῳ ἀνδρὸς σκεῦος σκεῦος ἐν χειρὶ αὐτοῦ, καὶ ὁ εἰσπορευόμενος εἰς τὸν οἶκον ἀποθανεῖται, καὶ ἔσονται μετὰ τοῦ βασιλέως ἐκπορευομένου καὶ εἰσπορευομένου αὐτοῦ.
\par }{\PP \VS{8}Καὶ ἐποίησαν οἱ Λευῖται καὶ πᾶς Ἰούδα κατὰ πάντα ὅσα ἐνετείλατο αὐτοῖς Ἰωδαὲ ὁ ἱερεύς· καὶ ἔλαβον ἕκαστος τοὺς ἄνδρας αὐτοῦ ἀπʼ ἀρχῆς τοῦ σαββάτου ἕως ἐξόδου τοῦ σαββάτου, ὅτι οὐ κατέλυσεν Ἰωδαὲ ὁ ἱερεὺς τὰς ἐφημερίας.
\VS{9}Καὶ ἔδωκεν Ἰωδαὲ τὰς μαχαίρας καὶ τοὺς θυρεοὺς καὶ τὰ ὅπλα ἃ ἦν τοῦ βασιλέως Δαυὶδ ἐν οἴκῳ τοῦ θεοῦ.
\VS{10}Καὶ ἔστησε τὸν λαὸν πάντα ἕκαστον ἐν τοῖς ὅπλοις αὐτοῦ, ἀπὸ τῆς ὠμίας τοῦ οἴκου τῆς δεξιᾶς ἕως τῆς ὠμίας τῆς ἀριστερᾶς τοῦ θυσιαστηρίου καὶ τοῦ οἴκου, ἐπὶ τὸν βασιλέα κύκλῳ.
\VS{11}Καὶ ἐξήγαγε τὸν υἱὸν τοῦ βασιλέως, καὶ ἔδωκεν ἐπʼ αὐτὸν τὸ βασίλειον καὶ τὰ μαρτύρια, καὶ ἐβασίλευσαν καὶ ἔχρισαν αὐτὸν Ἰωδαὲ ὁ ἱερεὺς καὶ οἱ υἱοὶ αὐτοῦ, καὶ εἶπαν, ζήτω ὁ βασιλεύς.
\par }{\PP \VS{12}Καὶ ἤκουσε Γοθολία τὴν φωνὴν τοῦ λαοῦ τρεχόντων, καὶ ἐξομολογουμένων, καὶ αἰνούντων τὸν βασιλέα· καὶ εἰσῆλθε πρὸς τὸν βασιλέα εἰς οἶκον Κυρίου.
\VS{13}Καὶ εἶδε, καὶ ἰδοὺ ὁ βασιλεὺς ἐπὶ τῆς στάσεως αὐτοῦ, καὶ ἐπὶ τῆς εἰσόδου οἱ ἄρχοντες καὶ οἱ σάλπιγγες· καὶ οἱ ἄρχοντες περὶ τὸν βασιλέα, καὶ πᾶς ὁ λαὸς τῆς γῆς ηὐφράνθη, καὶ ἐσάλπισαν ταῖς σάλπιγξι, καὶ οἱ ᾄδοντες ἐν τοῖς ὀργάνοις ᾠδοὶ, καὶ ὑμνοῦντες αἶνον· καὶ διέῥῥηξε Γοθολία τὴν στολὴν αὐτῆς, καὶ ἐβόησεν, ἐπιτιθέμενοι ἐπιτίθεσθε.
\VS{14}Καὶ ἐξῆλθεν Ἰωδαὲ ὁ ἱερεὺς, καὶ ἐνετείλατο Ἰωδαὲ ὁ ἱερεὺς τοῖς ἑκατοντάρχοις, καὶ τοῖς ἀρχηγοῖς τῆς δυνάμεως, καὶ εἶπεν αὐτοῖς, ἐκβάλετε αὐτὴν ἐκτὸς τοῦ οἴκου, καὶ εἰσέλθατε ὀπίσω αὐτῆς, καὶ ἀποθανέτω μαχαίρᾳ, ὅτι εἶπεν ὁ ἱερεὺς, μὴ ἀποθανέτω ἐν οἴκῳ Κυρίου.
\VS{15}Καὶ ἔδωκαν αὐτῇ ἄνεσιν, καὶ διῆλθε διὰ τῆς πύλης τῶν ἱππέων τοῦ οἴκου τοῦ βασιλέως, καὶ ἐθανάτωσαν αὐτὴν ἐκεῖ.
\par }{\PP \VS{16}Καὶ διέθετο Ἰωδαὲ διαθήκην ἀναμέσον αὐτοῦ καὶ τοῦ λαοῦ καὶ τοὺ βασιλέως, εἶναι λαὸν τῷ Κυρίῳ.
\VS{17}Καὶ εἰσῆλθε πᾶς ὁ λαὸς τῆς γῆς εἰς οἶκον Βάαλ, καὶ κατέσπασαν αὐτὸν καὶ τὰ θυσιαστήρια αὐτοῦ, καὶ τὰ εἴδωλα αὐτοῦ ἐλέπτυναν, καὶ τὸν Ματθὰν ἱερέα Βάαλ ἐθανάτωσαν ἐναντίον τῶν θυσιαστηρίων αὐτοῦ.
\VS{18}Καὶ ἐνεχείρισεν Ἰωδαὲ ὁ ἱερεὺς τὰ ἔργα οἴκου Κυρίου διὰ χειρὸς ἱερέων καὶ Λευιτῶν, καὶ ἀνέστησε τὰς ἐφημερίας τῶν ἱερέων καὶ τῶν Λευιτῶν, ἃς διέστειλε Δαυὶδ ἐπὶ τὸν οἶκον Κυρίου, καὶ ἀνενέγκαι ὁλοκαυτώματα Κυρίῳ, καθὼς γέγραπται ἐν νόμῳ Μωυσῆ, ἐν εὐφροσύνῃ καὶ ἐν ᾠδαῖς διὰ χειρὸς Δαυίδ.
\VS{19}Καὶ ἔστησαν οἱ πυλωροὶ ἐπὶ τὰς πύλας οἴκου Κυρίου, καὶ οὐκ εἰσελεύσεται ἀκάθαρτος εἰς πᾶν πρᾶγμα.
\VS{20}Καὶ ἔλαβε τοὺς πατριάρχας, καὶ τοὺς δυνατοὺς, καὶ τοὺς ἄρχοντας τοῦ λαοῦ, καὶ πάντα τὸν λαὸν τῆς γῆς, καὶ ἐπεβίβασαν τὸν βασιλέα εἰς οἶκον Κυρίου, καὶ εἰσῆλθε διὰ τῆς πύλης τῆς ἐσωτέρας εἰς τὸν οἶκον τοῦ βασιλέως, καὶ ἐκάθισαν τὸν βασιλέα ἐπὶ τοῦ θρόνου τῆς βασιλείας.
\VS{21}Καὶ ηὐφράνθη πᾶς ὁ λαὸς τῆς γῆς, καὶ ἡ πόλις ἡσύχασε, καὶ τὴν Γοθολίαν ἐθανάτωσαν.

\par }\Chap{24}{\PP \VerseOne{1}Ὢν ἐτῶν ἑπτὰ Ἰωὰς ἐν τῷ βασιλεύειν αὐτὸν, καὶ τεσσαράκοντα ἔτη ἐβασίλευσεν ἐν Ἱερουσαλὴμ. καὶ ὄνομα τῇ μητρὶ αὐτοῦ Σαβιὰ ἐκ Βηρσαβεέ.
\VS{2}Καὶ ἐποίησεν Ἰωὰς τὸ εὐθὲς ἐνώπιον Κυρίου πάσας τὰς ἡμέρας Ἰωδαὲ τοῦ ἱερέως.
\VS{3}Καὶ ἔλαβεν Ἰωδαὲ δύο γυναῖκας ἑαυτῷ, καὶ ἐγέννησαν υἱοὺς καὶ θυγατέρας.
\par }{\PP \VS{4}Καὶ ἐγένετο μετὰ ταῦτα, καὶ ἐγένετο ἐπὶ καρδίαν Ἰωὰς ἐπισκευάσαι τὸν οἶκον Κυρίου.
\VS{5}Καὶ συνήγαγε τοὺς ἱερεῖς καὶ τοὺς Λευίτας, καὶ εἶπεν αὐτοῖς, ἐξέλθατε εἰς τὰς πόλεις Ἰούδα, καὶ συναγάγετε ἀπὸ παντὸς Ἰσραὴλ ἀργύριον κατισχῦσαι τὸν οἶκον Κυρίου ἐνιαυτὸν κατʼ ἐνιαυτὸν, καὶ σπεύσατε λαλῆσαι· καὶ οὐκ ἔσπευσαν οἱ Λευῖται.
\par }{\PP \VS{6}Καὶ ἐκάλεσεν ὁ βασιλεὺς Ἰωὰς τὸν Ἰωδαὲ τὸν ἄρχοντα, καὶ εἶπεν αὐτῷ, διατί οὐκ ἐπεσκέψω περὶ τῶν Λευιτῶν τοῦ εἰσενέγκαι ἀπὸ Ἰούδα καὶ Ἱερουσαλὴμ τὸ κεκριμένον ὑπὸ Μωυσῆ ἀνθρώπου τοῦ Θεοῦ, ὅτι ἐξεκκλησίασε τὸν Ἰσραὴλ εἰς τὴν σκηνὴν τοῦ μαρτυρίου;
\VS{7}Ὅτι Γοθολία ἦν ἡ ἄνομος, καὶ οἱ υἱοὶ αὐτῆς κατέσπασαν τὸν οἶκον τοῦ Θεοῦ· καὶ γὰρ τὰ ἅγια οἴκου Κυρίου ἐποίησαν ταῖς Βααλίμ.
\par }{\PP \VS{8}Καὶ εἶπεν ὁ βασιλεὺς, γενηθήτω γλωσσόκομον, καὶ τεθήτω ἐν πύλῃ οἴκου Κυρίου ἔξω.
\VS{9}Καὶ κηρυξάτωσαν ἐν Ἰούδα καὶ ἐν Ἱερουσαλὴμ, εἰσενέγκαι Κυρίῳ καθὼς εἶπε Μωυσῆς παῖς τοῦ Θεοῦ ἐπὶ τὸν Ἰσραὴλ ἐν τῇ ἐρήμῳ.
\VS{10}Καὶ ἔδωκαν πάντες ἄρχοντες καὶ πᾶς ὁ λαὸς, καὶ εἰσέφερον καὶ ἐνέβαλον εἰς τὸ γλωσσόκομον ἕως οὗ ἐπληρώθη.
\VS{11}Καὶ ἐγένετο ὡς εἰσέφερον τὸ γλωσσόκομον πρὸς τοὺς προστάτας τοῦ βασιλέως διὰ χειρὸς τῶν Λευιτῶν, καὶ ὡς εἶδον ὅτι ἐπλεόνασε τὸ ἀργύριον, καὶ ἦλθεν ὁ γραμματεὺς τοῦ βασιλέως καὶ ὁ προστάτης τοῦ ἱερέως τοῦ μεγάλου, καὶ ἐξεκένωσεν τὸ γλωσσόκομον, καὶ κατέστησαν εἰς τὸν τόπον αὐτοῦ· οὕτως ἐποίουν ἡμέραν ἐξ ἡμέρας, καὶ συνήγαγον ἀργύριον πολύ.
\VS{12}Καὶ ἔδωκεν αὐτὸ ὁ βασιλεὺς καὶ Ἰωδαὲ ὁ ἱερεὺς τοῖς ποιοῦσι τὰ ἔργα εἰς ἐργασίαν οἴκου Κυρίου· καὶ ἐμισθοῦντο λατόμους καὶ τέκτονας ἐπισκευάσαι τὸν οἶκον Κυρίου, καὶ χαλκεῖς σιδήρου καὶ χαλκοῦ ἐπισκευάσαι τὸν οἶκον Κυρίου.
\VS{13}Καὶ ἐποίουν οἱ ποιοῦντες τὰ ἔργα, καὶ ἀνέβη μῆκος τῶν ἔργων ἐν χερσὶν αὐτῶν, καὶ ἀνέστησαν τὸν οἶκον Κυρίου ἐπὶ τὴν στάσιν αὐτοῦ, καὶ ἐνίσχυσαν.
\VS{14}Καὶ ὡς συνετέλεσαν, ἤνεγκαν πρὸς τὸν βασιλέα καὶ πρὸς Ἰωδαὲ τὸ κατάλοιπον τοῦ ἀργυρίου, καὶ ἐποίησαν σκεύη εἰς οἶκον Κυρίου, σκεύη λειτουργικὰ ὁλοκαυτωμάτων, καὶ θυΐσκας χρυσᾶς καὶ ἀργυρᾶς, καὶ ἀνήνεγκαν ὁλοκαυτώσεις ἐν οἴκῳ Κυρίου διαπαντὸς πάσας τὰς ἡμέρας Ἰωδαέ.
\par }{\PP \VS{15}Καὶ ἐγήρασεν Ἰωδαὲ πλήρης ἡμερῶν, καὶ ἐτελεύτησεν ὢν ἑκατὸν καὶ τριάκοντα ἐτῶν ἐν τῷ τελευτᾷν αὐτόν.
\VS{16}Καὶ ἔθαψαν αὐτὸν ἐν πόλει Δαυὶδ μετὰ τῶν βασιλέων, ὅτι ἐποίησεν ἀγαθωσύνην μετὰ Ἰσραὴλ καὶ μετὰ τοῦ Θεοῦ καὶ τοῦ οἴκου αὐτοῦ.
\par }{\PP \VS{17}Καὶ ἐγένετο μετὰ τὴν τελευτὴν Ἰωδαὲ εἰσῆλθον οἱ ἄρχοντες Ἰούδα, καὶ προσεκύνησαν τὸν βασιλέα· τότε ἐπήκουσεν αὐτοῖς ὁ βασιλεύς.
\VS{18}Καὶ ἐγκατέλιπον τὸν οἶκον Κυρίον Θεοῦ τῶν πατέρων αὐτῶν, καὶ ἐδούλευον ταῖς Ἀστάρταις καὶ τοῖς εἰδώλοις, καὶ ἐγένετο ὀργὴ ἐπὶ Ἰούδαν καὶ ἑπὶ Ἱερουσαλὴμ ἐν τῇ ἡμέρᾳ ταύτῃ.
\VS{19}Καὶ ἀπέστειλε πρὸς αὐτοὺς προφήτας ἐπιστρέψαι πρὸς Κύριον, καὶ οὐκ ἤκουσαν· καὶ διεμαρτύρατο αὐτοῖς, καὶ οὐχ ὑπήκουσαν.
\VS{20}Καὶ πνεῦμα Θεοῦ ἐνέδυσε τὸν Ἀζαρίαν τὸν τοῦ Ἰωδαὲ τὸν ἱερέα, καὶ ἀνέστη ἐπάνω τοῦ λαοῦ, καὶ εἶπε, τάδε λέγει Κύριος, τί παραπορεύεσθε τὰς ἐντολὰς Κυρίου; καὶ οὐκ εὐοδωθήσεσθε· ὅτι ἐγκατελίπετε τὸν Κύριον, καὶ ἐγκαταλείψει ὑμᾶς.
\VS{21}Καὶ ἐπέθεντο αὐτῷ, καὶ ἐλιθοβόλησαν αὐτὸν διʼ ἐντολῆς Ἰωὰς τοῦ βασιλέως ἐν αὐλῇ οἴκου Κυρίου.
\VS{22}Καὶ οὐκ ἐμνήσθη Ἰωὰς τοῦ ἐλέους οὗ ἐποίησεν Ἰωδαὲ ὁ πατὴρ αὐτοῦ μετʼ αὐτοῦ, καὶ ἐθανάτωσε τὸν υἱὸν αὐτοῦ· καὶ ὡς ἀπέθνησκεν, εἶπεν, ἴδοι Κύριος καὶ κρινάτω.
\par }{\PP \VS{23}Καὶ ἐγένετο μετὰ τὴν συντέλειαν τοῦ ἐνιαυτοῦ ἀνέβη ἐπʼ αὐτὸν δύναμις Συρίας, καὶ ἦλθεν ἐπὶ Ἰούδαν καὶ ἐπὶ Ἱερουσαλὴμ, καὶ κατέφθειραν πάντας τοὺς ἄρχοντας τοῦ λαοῦ ἐν τῷ λαῷ, καὶ πάντα τὰ σκῦλα αὐτῶν ἀπέστειλαν τῷ βασιλεῖ Δαμασκοῦ.
\VS{24}Ὅτι ἐν ὀλίγοις ἀνδράσιν παρεγένετο δύναμις Συρίας, καὶ ὁ Θεὸς παρέδωκεν εἰς τὰς χεῖρας αὐτῶν δύναμιν πολλὴν σφόδρα, ὅτι ἐγκατέλιπον Κύριον τὸν Θεὸν τῶν πατέρων αὐτῶν· καὶ μετὰ Ἰωὰς ἐποίησε κρίματα.
\par }{\PP \VS{25}Καὶ μετὰ τὸ ἀπελθεῖν αὐτοὺς ἀπʼ αὐτοῦ, ἐν τῷ ἐγκαταλιπεῖν αὐτὸν ἐν μαλακίαις μεγάλαις, καὶ ἐπέθεντο αὐτῷ οἱ παῖδες αὐτοῦ ἐν αἵμασιν υἱοῦ Ἰωδαὲ τοῦ ἱερέως, καὶ ἐθανάτωσαν αὐτὸν ἐπὶ τῆς κλίνης αὐτοῦ, καὶ ἀπέθανε· καὶ ἔθαψαν αὐτὸν ἐν πόλει Δαυὶδ, καὶ οὐκ ἔθαψαν αὐτὸν ἐν τῷ τάφῳ τῶν βασιλέων.
\VS{26}Καὶ οἱ ἐπιθέμενοι ἐπʼ αὐτὸν Ζαβὲδ ὁ τοῦ Σαμαὰθ ὁ Ἀμμανίτης, καὶ Ἰωζαβὲδ ὁ τοῦ Σαμαρὴθ ὁ Μωαβίτης,
\VS{27}καὶ οἱ υἱοὶ αὐτοῦ πάντες, καὶ προσῆλθον αὐτῷ οἱ πέντε· καὶ τὰ λοιπὰ ἰδοὺ γεγραμμένα ἐπὶ τὴν γραφὴν τῶν βασιλέων· καὶ ἐβασίλευσεν Ἀμασίας υἱὸς αὐτοῦ ἀντʼ αὐτοῦ.

\par }\Chap{25}{\PP \VerseOne{1}Ὢν εἴκοσι καὶ πέντε ἐτῶν ἐβασίλευσεν Ἀμασίας, καὶ εἰκοσιεννέα ἔτη ἐβασίλευσεν ἐν Ἱερουσαλὴμ, καὶ ὄνομα τῇ μητρὶ αὐτοῦ Ἰωαδαὲν ἀπὸ Ἱερουσαλήμ.
\VS{2}Καὶ ἐποίησε τὸ εὐθὲς ἐνώπιον Κυρίου, ἀλλʼ οὐκ ἐν καρδίᾳ πλήρει.
\VS{3}Καὶ ἐγένετο ὡς κατέστη ἡ βασιλεία ἐν χειρὶ αὐτοῦ, καὶ ἐθανάτωσε τοὺς παῖδας αὐτοῦ τοὺς φονεύσαντας τὸυ βασιλέα πατέρα αὐτοῦ·
\VS{4}Καὶ τοὺς υἱοὺς αὐτῶν οὐκ ἀπέκτεινε, κατὰ τὴν διαθήκην τοῦ νόμου Κύριου, καθὼς γέγραπται, ὡς ἐνετείλατο Κύριος, λέγων, οὐκ ἀποθανοῦνται πατέρες ὑπὲρ τέκνων, καὶ υἱοὶ οὐκ ἀποθανοῦνται ὑπὲρ πατέρων, ἀλλʼ ἢ ἕκαστος τῇ ἑαυτοῦ ἁμαρτίᾳ ἀποθανοῦνται.
\par }{\PP \VS{5}Καὶ συνήγαγεν Ἀμασίας τὸν οἶκον Ἰούδα, καὶ ἀνέστησεν αὐτοὺς κατʼ οἴκους πατριῶν αὐτῶν εἰς χιλιάρχους καὶ ἑκατοντάρχους ἐν παντὶ Ἰούδᾳ καὶ Ἱερουσαλήμ· καὶ ἠρίθμησεν αὐτοὺς ἀπὸ εἰκοσαετοῦς καὶ ἐπάνω, καὶ εὗρεν αὐτοὺς τριακοσίας χιλιάδας ἐξελθεῖν εἰς πόλεμον δυνατοὺς, κρατοῦντας δόρυ καὶ θυρεόν.
\VS{6}Καὶ ἐμισθώσατο ἀπὸ Ἰσραὴλ ἑκατὸν χιλιάδας δυνατοὺς ἰσχύϊ ἑκατὸν ταλάντων ἀργυρίου.
\par }{\PP \VS{7}Καὶ ἄνθρωπος τοῦ Θεοῦ ἦλθε πρὸς αὐτὸν, λέγων, βασιλεῦ, οὐ πορεύσεται μετὰ σοῦ δύναμις Ἰσραὴλ, ὅτι οὐκ ἔστι Κύριος μετὰ Ἰσραὴλ πάντων τῶν υἱῶν Ἐφραίμ.
\VS{8}Ὅτι ἐὰν ὑπολάβῃς κατισχῦσαι ἐν τούτοις, καὶ τροπώσεταί σε Κύριος ἐναντίον τῶν ἐχθρῶν, ὅτι ἐστὶ παρὰ Κυρίου καὶ ἰσχῦσαι καὶ τροπώσασθαι.
\VS{9}Καὶ εἶπεν Ἀμασίας τῷ ἀνθρώπῳ τοῦ Θεοῦ, καὶ τί ποιήσω τὰ ἑκατὸν τάλαντα ἃ ἔδωκα τῇ δυνάμει Ἰσραήλ; καὶ εἶπεν ὁ ἄνθρωπος τοῦ Θεοῦ, ἔστι τῷ Κυρίῳ δοῦναί σοι πλεῖστα τούτων.
\par }{\PP \VS{10}Καὶ διεχώρισεν Ἀμασίας τῇ δυνάμει τῇ ἐλθούσῃ πρὸς αὐτὸν ἀπὸ Ἐφραὶμ, ἀπέλθεῖν εἰς τὸν τόπον αὐτῶν· καὶ ἐθυμώθησαν σφόδρα ἐπὶ Ἰούδαν, καὶ ἐπέστρεψαν εἰς τὸν τόπον αὐτῶν ἐν ὀργῇ θυμοῦ.
\VS{11}Καὶ Ἀμασίας κατίσχυσε καὶ παρέλαβε τὸν λαὸν αὐτοῦ, καὶ ἐπορεύθη εἰς τὴν κοιλάδα τῶν ἁλῶν, καὶ ἐπάταξεν ἐκεῖ τοὺς υἱοὺς Σηεὶρ, δέκα χιλιάδας.
\VS{12}Καὶ δέκα χιλιάδας ἐζώγρησαν οἱ υἱοὶ Ἰούδα, καὶ ἔφερον αὐτοὺς ἐπὶ τὸ ἄκρον τοῦ κρημνοῦ, καὶ κατεκρήμνιζον αὐτοὺς ἀπὸ τοῦ ἄκρου τοῦ κρημνοῦ, καὶ πάντες διεῤῥήγνυντο.
\VS{13}Καὶ υἱοὶ τῆς δυνάμεως οὓς ἀπέστρεψεν Ἀμασίας τοῦ μὴ πορευθῆναι μετʼ αὐτοῦ εἰς πόλεμον, καὶ ἐπέθεντο ἐπὶ τὰς πόλεις Ἰούδα ἀπὸ Σαμαρείας ἕως Βαιθωρών· καὶ ἐπάταξαν ἐν αὐτοῖς τρεῖς χιλιάδας, καὶ ἐσκύλευσαν σκῦλα πολλά.
\par }{\PP \VS{14}Καὶ ἐγένετο μετὰ τὸ ἐλθεῖν Ἀμασίαν πατάξαντος τὴν Ἰδουμαίαν, καὶ ἤνεγκε πρὸς αὐτὸν τοὺς θεοὺς υἱῶν Σηεὶρ, καὶ ἔστησεν αὐτοὺς αὐτῷ εἰς θεοὺς, καὶ ἐναντίον αὐτῶν προσεκύνει, καὶ αὐτὸς αὐτοῖς ἔθυε.
\VS{15}Καὶ ἐγένετο ὀργὴ Κυρίου ἐπὶ Ἀμασίαν, καὶ ἀπέστειλεν αὐτῷ προφήτην, καὶ εἶπεν αὐτῷ, τί ἐζήτησας τοὺς θεοὺς τοῦ λαοῦ, οἳ οὐκ ἐξείλοντο τὸν λαὸν ἑαυτῶν ἐκ χειρός σου;
\VS{16}Καὶ ἐγένετο ἐν τῷ λαλῆσαι αὐτῷ πρὸς αὐτὸν, καὶ εἶπεν αὐτῷ, μὴ σύμβουλον τοῦ βασιλέως δέδωκά σε; πρόσεχε ἵνα μὴ μαστιγωθῇς· καὶ ἐσιώπησεν ὁ προφήτης, καὶ εἶπεν, ὅτι γινώσκω, ὅτι ἐβούλετο ἐπὶ σοὶ τοῦ καταφθεῖραί σε, ὅτι ἐποίησας τοῦτο, καὶ οὐκ ἐπήκουσας τῆς συμβουλίας μου.
\par }{\PP \VS{17}Καὶ ἐβουλεύσατο Ἀμασίας ὁ βασιλεὺς Ἰούδα, καὶ ἀπέστειλε πρὸς Ἰωὰς υἱὸν Ἰωάχαζ υἱοῦ Ἰηοὺ βασιλέα Ἰσραὴλ, λέγων, δεῦρο, καὶ ὀφθῶμεν προσώποις.
\VS{18}Καὶ ἀπέστειλεν Ἰωὰς βασιλεὺς Ἰσραὴλ πρὸς Ἀμασίαν βασιλέα Ἰούδα, λέγων, ὁ ἀκχοὺχ ὁ ἐν τῷ Λιβάνῳ ἀπέστειλε πρὸς τὴν κέδρον τὴν ἐν τῷ Λιβάνῳ, λέγων, δὸς τὴν θυγατέρα σου τῷ υἱῷ μου εἰς γυναῖκα, καὶ ἰδοὺ ἐλεύσεται τὰ θηρία τοῦ ἀγροῦ τὰ ἐν τῷ Λιβάνῳ· καὶ ἦλθον τὰ θηρία, καὶ κατεπάτησαν τὸν ἀκχούχ.
\VS{19}Εἶπας, ἰδοὺ ἐπάταξα τὴν Ἰδουμαίαν, καὶ ἐπαίρει σε ἡ καρδία σου ἡ βαρεῖα· νῦν κάθισον ἐν οἴκῳ σου, καὶ ἱνατί συμβάλλεις ἐν κακίᾳ, καὶ πεσῇ σὺ καὶ Ἰούδας μετὰ σοῦ;
\par }{\PP \VS{20}Καὶ οὐκ ἤκουσεν Ἀμασίας, ὅτι παρὰ Κυρίου ἐγένετο τοῦ παραδοῦναι αὐτὸν εἰς χεῖρας, ὅτι ἐξεζήτησε τοὺς θεοὺς τῶν Ἰδουμαίων.
\VS{21}Καὶ ἀνέβη Ἰωὰς βασιλεὺς Ἰσραὴλ, καὶ ὤφθησαν ἀλλήλοις αὐτὸς καὶ Ἀμασίας βασιλεὺς Ἰούδα ἐν Βαιθσαμὺς, ἥ ἐστι τοῦ Ἰούδα.
\VS{22}Καὶ ἐτροπώθη Ἰούδας κατὰ πρόσωπον Ἰσραὴλ, καὶ ἔφυγεν ἕκαστος εἰς τὸ σκήνωμα αὐτοῦ.
\VS{23}Καὶ τὸν Ἀμασίαν βασιλέα Ἰούδα τὸν τοῦ Ἰωὰς υἱοῦ Ἰωάχαζ κατέλαβεν Ἰωὰς βασιλεὺς Ἰσραὴλ ἐν Βαιθσαμὺς, καὶ εἰσήγαγεν αὐτὸν εἰς Ἱερουσαλὴμ· καὶ κατέσπασεν ἀπὸ τοῦ τείχους Ἱερουσαλὴμ ἀπὸ πύλης Ἐφραὶμ ἕως πύλης γωνίας τετρακοσίους πήχεις.
\VS{24}Καὶ πᾶν τὸ χρυσίον καὶ τὸ ἀργύριον, καὶ πάντα τὰ σκεύη τὰ εὑρεθέντα ἐν οἴκῳ Κυρίου καὶ παρὰ τῷ Ἀβδεδὸμ, καὶ τοὺς θησαυροὺς οἴκου τοῦ βασιλέως, καὶ τοὺς υἱοὺς τῶν συμμίξεων, καὶ ἐπέστρεψεν εἰς Σαμάρειαν.
\par }{\PP \VS{25}Καὶ ἔζησεν Ἀμασίας ὁ τοῦ Ἰωὰς βασιλεὺς Ἰούδα μετὰ τὸ ἀποθανεῖν Ἰωὰς τὸν τοῦ Ἰωάχαζ βασιλέα Ἰσραὴλ ἔτη δεκαπέντε.
\VS{26}Καὶ οἱ λοιποὶ λόγοι Ἀμασίου οἱ πρῶτοι καὶ οἱ ἔσχατοι οὐκ ἰδοὺ γεγραμμένοι ἐπὶ βιβλίου βασιλέων Ἰούδα καὶ Ἰσραήλ;
\VS{27}Καὶ ἐν τῷ καιρῷ ᾧ ἀπέστη Ἀμασίας ἀπὸ Κυρίου, καὶ ἐπέθεντο αὐτῷ ἐπίθεσιν, καὶ ἔφυγεν ἀπὸ Ἱερουσαλὴμ εἰς Λαχίς· καὶ ἀπέστειλαν κατόπισθεν αὐτοῦ εἰς Λαχίς, καὶ ἐθανάτωσαν αὐτὸν ἐκεῖ.
\VS{28}Καὶ ἀνέλαβον αὐτὸν ἐπὶ τῶν ἵππων, καὶ ἔθαψαν αὐτὸν μετὰ τῶν πατέρων αὐτοῦ ἐν πόλει Δαυίδ.

\par }\Chap{26}{\PP \VerseOne{1}Καὶ ἔλαβε πᾶς ὁ λαὸς τῆς γῆς τὸν Ὀζίαν, καὶ αὐτὸς υἱὸς ἑκκαίδεκα ἐτῶν, καὶ ἐβασίλευσαν αὐτὸν ἀντὶ τοῦ πατρὸς αὐτοῦ Ἀμασίου.
\VS{2}Αὐτὸς ᾠκοδόμησε τὴν Αἰλάθ, αὐτὸς ἐπέστρεψεν αὐτὴν τῷ Ἰούδα, μετὰ τὸ κοιμηθῆναι τὸν βασιλέα μετὰ τῶν πατέρων αὐτοῦ.
\par }{\PP \VS{3}Υἱὸς ἑκκαίδεκα ἐτῶν ἐβασίλευσεν Ὀζίας, καὶ πεντήκοντα καὶ δύο ἔτη ἐβασίλευσεν ἐν Ἱερουσαλὴμ, καὶ ὄνομα τῇ μητρὶ αὐτοῦ Ἰεχελία ἀπὸ Ἱερουσαλήμ.
\VS{4}Καὶ ἐποίησε τὸ εὐθὲς ἐνώπιον Κυρίου κατὰ πάντα ὅσα ἐποίησεν Ἀμασίας ὁ πατὴρ αὐτοῦ.
\VS{5}Καὶ ἦν ἐκζητῶν τὸν Κύριον ἐν ταῖς ἡμέραις Ζαχαρίου τοῦ συνιόντος ἐν φόβῳ Κυρίου, καὶ ἐν ταῖς ἡμέραις αὐτοῦ ἐζήτησε τὸν Κύριον, καὶ εὐώδωσεν αὐτῷ Κύριος.
\par }{\PP \VS{6}Καὶ ἐξῆλθε καὶ ἐπολέμησε πρὸς τοὺς ἀλλοφύλους, καὶ κατέσπασε τὰ τείχη Γὲθ, καὶ τὰ τείχη Ἰαβνὴρ, καὶ τὰ τείχη Ἀζώτου, καὶ ᾠκοδόμησε πόλεις Ἀζώτου, καὶ ἐν τοῖς ἀλλοφύλοις.
\VS{7}Καὶ κατίσχυσεν αὐτὸν Κύριος ἐπὶ τοὺς ἀλλοφύλους, καὶ ἐπὶ τοὺς Ἄραβας τοὺς κατοικοῦντας ἐπὶ τῆς πέτρας, καὶ ἐπὶ τοὺς Μιναίους.
\VS{8}Καὶ ἔδωκαν οἱ Μιναῖοι δῶρα τῷ Ὀζίᾳ, καὶ ἦν τὸ ὄνομα αὐτοῦ ἕως εἰσόδου Αἰγύπτου, ὅτι κατίσχυσεν ἕως ἄνω.
\par }{\PP \VS{9}Καὶ ᾠκοδόμησεν Ὀζίας πύργους ἐν Ἱερουσαλὴμ, καὶ ἐπὶ τὴν πύλην τῆς γωνίας καὶ ἐπὶ τὴν πύλην τῆς φάραγγος, καὶ ἐπὶ τῶν γενιῶν, καὶ κατίσχυσε.
\VS{10}Καὶ ᾠκοδόμησε πύργους ἐν τῇ ἐρήμῳ, καὶ ἐλατόμησε λάκκους πολλοὺς, ὅτι κτήνη πολλὰ ὑπῆρχεν αὐτῷ ἐν σεφηλᾷ· καὶ ἐν τῇ πεδινῇ, καὶ ἀμπελουργοὶ ἐν τῇ ὀρεινῇ καὶ ἐν τῷ Καρμήλῳ, ὅτι γεωργὸς ἦν.
\VS{11}Καὶ ἐγένετο τῷ Ὀζίᾳ δύναμις ποιοῦσα πόλεμον, καὶ ἐκπορευομένη εἰς παράταξιν εἰς πόλεμον καὶ εἱσπορευομένη εἰς παράταξιν εἰς ἀριθμόν· καὶ ἦν ὁ ἀριθμὸς αὐτῶν διὰ χειρὸς Ἰεϊὴλ τοῦ γραμματέως, καὶ Μαασίου τοῦ κριτοῦ, διὰ χειρὸς Ἀνανίου τοῦ διαδόχου τοῦ βασιλέως.
\VS{12}Πᾶς ὁ ἀριθμὸς τῶν πατριαρχῶν τῶν δυνατῶν εἰς πόλεμον δισχίλιοι ἑξακόσιοι,
\VS{13}καὶ μετʼ αὐτῶν δύναμις πολεμικὴ, τριακόσιαι χιλιάδες καὶ ἑπτακισχίλιοι καὶ πεντακόσιοι· οὗτοι οἱ ποιοῦντες πόλεμον ἐν δυνάμει ἰσχύος βοηθῆσαι τῷ βασιλεῖ ἐπὶ τοὺς ὑπεναντίους.
\VS{14}Καὶ ἡτοίμασεν αὐτοῖς Ὀζίας πάσῃ τῇ δυναμει θυρεοὺς καὶ δόρατα καὶ περικεφαλαίας καὶ θώρακας καὶ τόξα καὶ εἰς λίθους σφενδόνας·
\VS{15}Καὶ ἐποίησεν ἐν Ἱερουσαλὴμ μηχανὰς μεμηχανευμένας λογιστοῦ, τοῦ εἶναι ἐπὶ τῶν πύργων καὶ ἐπὶ τῶν γωνιῶν, βάλλειν βέλεσι καὶ λίθοις μεγάλοις· καὶ ἠκούσθη ἡ κατασκευὴ αὐτῶν ἕως πόῤῥω· ὅτι ἐθαυμαστώθη τοῦ βοηθῆναι ἕως οὗ κατίσχυσε.
\par }{\PP \VS{16}Καὶ ὡς κατίσχυσεν, ὑψώθη ἡ καρδία αὐτοῦ τοῦ καταφθεῖραι· καὶ ἠδίκησεν ἐν Κυρίῳ Θεῷ αὐτοῦ, καὶ εἰσῆλθεν εἰς τὸν ναὸν Κυρίου θυμιάσαι ἐπὶ τὸ θυσιαστήριον τῶν θυμιαμάτων.
\VS{17}Καὶ εἰσῆλθεν ὀπίσω αὐτοῦ Ἀζαρίας ὁ ἱερεύς, καὶ μετʼ αὐτοῦ ἱερεῖς τοῦ Κυρίου ὀγδοήκοντα υἱοὶ δυνατοί.
\VS{18}Καὶ ἔστησαν ἐπὶ Ὀζίαν τὸν βασιλέα, καὶ εἶπαν αὐτῷ, οὐ σοὶ, Ὀζία, θυμιάσαι τῷ Κυρίῳ, ἀλλʼ ἢ τοῖς ἱερεῦσιν υἱοῖς Ἀαρὼν τοῖς ἡγιασμένοις θῦσαι· ἔξελθε ἐκ τοῦ ἁγιάσματος, ὅτι ἀπέστης ἀπὸ Κυρίου· καὶ οὐκ ἔσται σοι τοῦτο εἰς δόξαν παρὰ Κυρίου Θεοῦ.
\par }{\PP \VS{19}Καὶ ἐθυμώθη Ὀζίας, καὶ ἐν τῇ χειρὶ αὐτοὺ τὸ θυμιατήριον τοῦ θυμιάσαι ἐν τῷ ναῷ· καὶ ἐν τῷ θυμωθῆναι αὐτὸν πρὸς τοὺς ἱερεῖς, καὶ ἡ λέπρα ἀνέτειλεν ἐν τῷ μετώπῳ αὐτοῦ ἐναντίον τῶν ἱερέων ἐν οἴκῳ Κυρίου ἐπάνω τοῦ θυσιαστηρίου τῶν θυμιαμάτων.
\VS{20}Καὶ ἐπέστρεψε πρὸς αὐτὸν Ἀζαρίας ὁ ἱερεὺς ὁ πρῶτος, καὶ οἱ ἱερεῖς, καὶ ἰδοὺ αὐτὸς λεπρὸς ἐν τῷ μετώπῳ, καὶ κατέσπευσαν αὐτὸν ἐκεῖθεν, καὶ γὰρ αὐτὸς ἔσπευσεν ἐξελθεῖν, ὅτι ἤλεγξεν αὐτὸν Κύριος.
\VS{21}Καὶ Ὀζίας ὁ βασιλεὺς ἦν λεπρὸς ἕως ἡμέρας τῆς τελευτῆς αὐτοῦ, καὶ ἐν οἴκῳ ἀπφουσὼθ ἐκάθητο λεπρὸς, ὅτι ἀπεσχίσθη ἀπὸ οἴκου Κυρίου· καὶ Ἰωάθαν ὁ υἱὸς αὐτοῦ ἐπὶ τῆς βασιλείας αὐτοῦ κρίνων τὸν λαὸν τῆς γῆς.
\par }{\PP \VS{22}Καὶ οἱ λοιποὶ λόγοι Ὀζίου οἱ πρῶτοι καὶ οἱ ἔσχατοι, γεγραμμένοι ὑπὸ Ἰεσσίου τοῦ προφήτου.
\VS{23}Καὶ ἐκοιμήθη Ὀζίας μετὰ τῶν πατέρων αὐτοῦ, καὶ ἔθαψαν αὐτὸν μετὰ τῶν πατέρων αὐτοῦ ἐν τῷ πεδίῳ τῆς ταφῆς τῶν βασιλέων, ὅτι εἶπαν ὅτι λεπρός ἐστι· καὶ ἐβασίλευσεν Ἰωάθαμ υἱὸς αὐτοῦ ἀντʼ αὐτοῦ.

\par }\Chap{27}{\PP \VerseOne{1}Υἱὸς εἴκοσι καὶ πέντε ἐτῶν Ἰωάθαμ ἐν τῷ βασιλεῦσαι αὐτὸν, καὶ ἑκκαίδεκα ἔτη ἐβασίλευσεν ἐν Ἱερουσαλὴμ, καὶ ὄνομα τῆς μητρὸς αὐτοῦ Ἱερουσὰ θυγάτηρ Σαδώκ.
\VS{2}Καὶ ἐποίησε τὸ εὐθὲς ἐνώπιον Κυρίου, κατὰ πάντα ἃ ἐποίησεν Ὀζίας ὁ πατὴρ αὐτοῦ, ἀλλʼ οὐκ εἰσῆλθεν εἰς τὸν ναὸν Κυρίου. Καὶ ἔτι ὁ λαὸς κατεφθείρετο.
\VS{3}Αὐτὸς ᾠκοδόμησε τὴν πύλην οἴκου Κυρίου τὴν ὑψηλὴν, καὶ ἐν τείχει Ὀπέλ ᾠκοδόμησε πολλὰ,
\VS{4}ἐν ὄρει Ἰούδα, καὶ ἐν τοῖς δρυμοῖς καὶ οἰκήσεις καὶ πύργους.
\VS{5}Αὐτὸς ἐμαχέσατο πρὸς βασιλέα υἱῶν Ἀμμὼν, καὶ κατίσχυσεν ἐπʼ αὐτόν· καὶ ἐδίδουν αὐτῷ οἱ υἱοὶ Ἀμμὼν καὶ κατʼ ἐνιαυτὸν ἑκατὸν τάλαντα ἀργυρίου, καὶ δέκα χιλιάδας κόρων πυροῦ, καὶ κριθῶν δέκα χιλιάδας· ταῦτα ἔφερεν αὐτῷ βασιλεὺς υἱῶν Ἀμμὼν κατʼ ἐνιαυτὸν ἐν τῷ πρώτῳ ἔτει καὶ ἐν τῷ δευτέρῳ καὶ τῷ τρίτῳ.
\VS{6}Κατίσχυσεν Ιωάθαμ, ὅτι ἡτοίμασεν τὰς ὁδοὺς αὐτοῦ ἐναντίον Κυρίου Θεοῦ αὐτοῦ.
\par }{\PP \VS{7}Καὶ οἱ λοιποὶ λόγοι Ἰωάθαμ καὶ ὁ πόλεμος καὶ αἱ πράξεις αὐτοῦ, ἰδοὺ γεγραμμέναι ἐπὶ βιβλίῳ βασιλέων Ἰούδα καὶ Ἰσραήλ.
\VS{9}Καὶ ἐκοιμήθη Ἰωάθαμ μετὰ τῶν πατέρων αὐτοῦ, καὶ ἐτάφη ἐν πόλει Δαυὶδ, καὶ ἐβασίλευσεν Ἄχαζ υἱὸς αὐτοῦ ἀντʼ αὐτοῦ.

\par }\Chap{28}{\PP \VerseOne{1}Υἱὸς εἴκοσι καὶ πέντε ἐτῶν ἦν Ἄχαζ ἐν τῷ βασιλεύειν αὐτὸν, καὶ ἑκκαίδεκα ἔτη ἐβασίλευσεν ἐν Ἱερουσαλήμ· καὶ οὐκ ἐποίησε τὸ εὐθὲς ἐνώπιον Κυρίου, ὡς Δαυὶδ ὁ πατὴρ αὐτοῦ.
\VS{2}Καὶ ἐπορεύθη κατὰ τὰς ὁδοὺς βασιλέων Ἰσραήλ· καὶ γὰρ γλυπτὰ ἐποίησε, καὶ τοῖς εἰδώλοις αὐτῶν
\VS{3}ἐν γὲ Βενεννόμ· καὶ διῆγε τὰ τέκνα αὐτοῦ διὰ πυρὸς κατὰ τὰ βδελύγματα τῶν ἐθνῶν, ὧν ἐξωλόθρευσε Κύριος ἀπὸ προσώπου υἱῶν Ἰσραήλ.
\VS{4}Καὶ ἐθυμία ἐπὶ τῶν ὑψηλῶν, καὶ ἐπὶ τῶν δωμάτων, καὶ ὑποκάτω παντὸς ξύλου ἀλσώδους.
\par }{\PP \VS{5}Καὶ παρέδωκεν αὐτὸν Κύριος ὁ Θεὸς αὐτοῦ διὰ χειρὸς βασιλέως Συρίας, καὶ ἐπάταξεν ἐν αὐτῷ, καὶ ᾐχμαλώτευσεν ἐξ αὐτῶν αἰχμαλωσίαν πολλὴν, καὶ ἤγαγεν εἰς Δαμασκόν· καὶ εἰς χεῖρας βασιλέως Ἰσραὴλ παρέδωκεν αὐτὸν, καὶ ἐπάταξεν ἐν αὐτῷ πληγὴν μεγάλην.
\VS{6}Καὶ ἀπέκτεινε Φακεὲ ὁ τοῦ Ῥομελία βασιλεὺς Ἰσραὴλ ἐν Ἰούδᾳ ἐν μιᾷ ἡμέρᾳ ἑκατὸν εἴκοσι χιλιάδας ἀνδρῶν δυνατῶν ἰσχύϊ, ἐν τῷ καταλιπεῖν αὐτοὺς Κύριον τὸν Θεὸν τῶν πατέρων αὐτῶν.
\VS{7}Καὶ ἀπέκτεινε Ζεχρὶ ὁ δυνατὸς τοῦ Ἐφραὶμ τὸν Μαασίαν τὸν υἱὸν τοῦ βασιλέως, καὶ τὸν Ἐζρικὰν ἡγούμενον τοῦ οἴκου αὐτοῦ, καὶ τὸν Ἐλκανὰ τὸν διάδοχον τοῦ βασιλέως.
\VS{8}Καὶ ᾐχμαλώτισαν οἱ υἱοὶ Ἰσραὴλ ἀπὸ τῶν ἀδελφῶν αὐτῶν τριακοσίας χιλιάδας, γυναῖκας καὶ υἱοὺς καὶ θυγατέρας· καὶ σκῦλα πολλὰ ἐσκύλευσαν ἐξ αὐτῶν, καὶ ἤνεγκαν τὰ σκῦλα εἰς Σαμάρειαν.
\par }{\PP \VS{9}Καὶ ἐκεῖ ἦν ὁ προφήτης τοῦ Κυρίου, Ὠδὴδ ὄνομα αὐτῷ· καὶ ἐξῆλθεν εἰς ἀπάντησιν τῆς δυνάμεως τῶν ἐρχομένων εἰς Σαμάρειαν, καὶ εἶπεν αὐτοῖς, ἰδοὺ ὀργὴ Κυρίου Θεοῦ τῶν πατέρων ὑμῶν ἐπὶ Ἰούδαν, καὶ παρέδωκεν αὐτοὺς εἰς τὰς χεῖρας ὑμῶν, καὶ ἀπεκτείνατε ἐν αὐτοῖς ἐν ὀργῇ, καὶ ἕως τῶν οὐρανῶν ἔφθακε.
\VS{10}Καὶ νῦν υἱοὺς Ἰούδα καὶ Ἱερουσαλὴμ ὑμεῖς λέγετε κατακτήσασθαι εἰς δούλους καὶ δούλας· οὐκ ἰδού εἰμι μεθʼ ὑμῶν μαρτυρῆσαι Κυρίῳ Θεῷ ὑμῶν;
\VS{11}Καὶ νῦν ἀκούσατέ μου, καὶ ἀποστρέψατε τὴν αἰχμαλωσίαν ἣν ᾐχμαλωτεύσατε τῶν ἀδελφῶν ὑμῶν, ὅτι ὀργὴ θυμοῦ Κυρίου ἐφʼ ὑμῖν.
\par }{\PP \VS{12}Καὶ ἀνέστησαν ἄρχοντες ἀπὸ τῶν υἱῶν Ἐφραὶμ, Οὐδείας ὁ τοῦ Ἰωανοῦ, καὶ Βαραχίας ὁ τοῦ Μοσολαμὼθ, καὶ Ἐζεκίας ὁ τοῦ Σελλὴμ, καὶ Ἀμασίας ὁ τοῦ Ἐλδαῒ ἐπὶ τοὺς ἐρχομένους ἀπὸ τοῦ πολέμου,
\VS{13}καὶ εἶπαν αὐτοῖς, οὐ μὴ εἰσαγάγητε τὴν αἰχμαλωσίαν ὧδε πρὸς ἡμᾶς, ὅτι εἰς τὸ ἁμαρτάνειν τῷ Κυρίῳ ἐφʼ ἡμᾶς, ὑμεῖς λέγετε προσθεῖναι ἐπὶ ταῖς ἁμαρτίαις ἡμῶν, καὶ ἐπὶ τὴν ἄγνοιαν ἡμῶν, ὅτι πολλὴ ἡ ἁμαρτία ἡμῶν, καὶ ὀργὴ θυμοῦ Κυρίου ἐπὶ τὸν Ἰσραήλ.
\VS{14}Καὶ ἀφῆκαν οἱ πολεμισταὶ τὴν αἰχμαλωσίαν καὶ τὰ σκῦλα ἐναντίον τῶν ἀρχόντων καὶ πάσης τῆς ἐκκλησίας.
\VS{15}Καὶ ἀνέστησαν ἄνδρες οἳ ἐπεκλήθησαν ἐν ὀνόματι, καὶ ἀντελάβοντο τῆς αἰχμαλωσίας, καὶ πάντας τοὺς γυμνοὺς περιέβαλον ἀπὸ τῶν σκύλων, καὶ ἐνέδυσαν αὐτοὺς καὶ ὑπέδησαν αὐτοὺς, καὶ ἔδωκαν φαγεῖν καὶ ἀλείψασθαι, καὶ ἀντελάβοντο καὶ ἐν ὑποζυγίοις παντὸς ἀσθενοῦντος, καὶ κατέστησαν αὐτοὺς εἰς Ἱεριχὼ πόλιν φοινίκων πρὸς τοὺς ἀδελφοὺς αὐτῶν, καὶ ἐπέστρεψαν εἰς Σαμάρειαν.
\par }{\PP \VS{16}Ἐν τῷ καιρῷ ἐκείνῳ ἀπέστειλεν ὁ βασιλεὺς Ἄχας πρὸς βασιλέα Ἀσσοὺρ βοηθῆσαι αὐτῷ
\VS{17}καὶ ἐν τούτῳ, ὅτι οἱ Ἰδουμαῖοι ἐπέθεντο, καὶ ἐπάταξαν ἐν Ἰούδα, καὶ ᾐχμαλώτισαν αἰχμαλωσίαν.
\VS{18}Καὶ οἱ ἀλλόφυλοι ἐπέθεντο ἐπὶ τὰς πόλεις τῆς πεδινῆς, καὶ ἀπὸ Λιβὸς τοῦ Ἰούδα, καὶ ἔλαβον τὴν Βαιθσαμὺς, καὶ τὰ ἐν οἴκῳ Κυρίου, καὶ τὰ ἐν οἴκῳ τοῦ βασιλέως καὶ τῶν ἀρχόντων, καὶ ἔδωκαν τῷ βασιλεῖ τὴν Ἀϊλὼν, καὶ τὴν Γαληρὼ, καὶ τὴν Σωχὼ καὶ τὰς κώμας αὐτῆς, καὶ τὴν Θαμνὰ καὶ τὰς κώμας αὐτῆς, καὶ τὴν Γαμζὼ καὶ τὰς κώμας αὐτῆς· καὶ κατῴκησαν ἐκεῖ.
\VS{19}Ὅτι ἐταπείνωσε Κύριος τὸν Ἰούδαν διὰ Ἄχαζ βασιλέα Ἰούδα, ὅτι ἀπέστη ἀποστάσει ἀπὸ Κυρίου.
\VS{20}Καὶ ἦλθεν ἐπʼ αὐτὸν Θαλγαφελλασὰρ βασιλεὺς Ἀσσοὺρ, καὶ ἔθλιψεν αὐτόν.
\VS{21}Καὶ ἔλαβεν Ἄχας τὰ ἐν οἴκῳ Κυρίου, καὶ τὰ ἐν οἴκῳ τοῦ βασιλέως καὶ τῶν ἀρχόντων, καὶ ἔδωκε τῷ βασιλεῖ Ἀσσούρ· καὶ οὐκ εἰς βοήθειαν αὐτῷ ἦν,
\VS{22}ἀλλʼ ἢ τῷ θλιβῆναι αὐτόν· καὶ προσέθηκε τοῦ ἀποστῆναι ἀπὸ Κυρίου, καὶ εἶπεν ὁ βασιλεὺς Ἄχαζ,
\VS{23}ἐκζητήσω τοὺς θεοὺς Δαμασκοῦ τοὺς τύπτοντάς με· καὶ εἶπεν, ὅτι θεοὶ βασιλέως Συρίας αὐτοὶ κατισχύσουσιν αὐτοὺς, αὐτοῖς τοίνυν θύσω, καὶ ἀντιλήψονταί μου· καὶ αὐτοὶ ἐγένοντο αὐτῷ εἰς σκῶλον καὶ παντὶ Ἰσραήλ.
\par }{\PP \VS{24}Καὶ ἀπέστησεν Ἄχαζ τὰ σκεύη οἴκου Κυρίου, καὶ κατέκοψεν αὐτὰ, καὶ ἔκλεισε τὰς θύρας οἴκου Κυρίου, καὶ ἐποίησεν ἑαυτῷ θυσιαστήρια ἐν πάσῃ γωνίᾳ ἐν Ἱερουσαλὴμ,
\VS{25}καὶ ἐν πάσῃ πόλει καὶ πόλει ἐν Ἰούδᾳ ἐποίησεν ὑψηλὰ θυμιᾷν θεοῖς ἀλλοτρίοις, καὶ παρώργισαν Κύριον τὸν Θεὸν τῶν πατέρων αὐτῶν.
\VS{26}Καὶ οἱ λοιποὶ λόγοι αὐτοῦ καὶ αἱ πράξεις αὐτοῦ αἱ πρῶται καὶ ἔσχαται, ἰδοὺ γεγραμμέναι ἐπὶ βιβλίῳ βασιλέων Ἰούδα καὶ Ἰσραήλ.
\VS{27}Καὶ ἐκοιμήθη Ἄχαζ μετὰ τῶν πατέρων αὐτοῦ, καὶ ἐτάφη ἐν πόλει Δαυὶδ, ὅτι οὐκ εἰσήνεγκαν αὐτὸν εἰς τοὺς τάφους τῶν βασιλέων Ἰσραὴλ, καὶ ἐβασίλευσεν Ἐζεκίας υἱὸς αὐτοῦ ἀντʼ αὐτοῦ.

\par }\Chap{29}{\PP \VerseOne{1}Καὶ Ἐζεκίας ἐβασίλευσεν ὢν εἴκοσι καὶ πέντε ἐτῶν, καὶ εἴκοσι ἐννέα ἔτη ἐβασίλευσεν ἐν Ἱερουσαλὴμ, καὶ ὄνομα τῇ μητρὶ αὐτοῦ Ἀβιὰ, θυγάτηρ Ζαχαρίου.
\VS{2}Καὶ ἐποίησε τὸ εὐθὲς ἐνώπιον Κυρίου κατὰ πάντα ὅσα ἐποίησε Δαυὶδ ὁ πατὴρ αὐτοῦ.
\par }{\PP \VS{3}Καὶ ἐγένετο ὡς ἔστη ἐπὶ τῆς βασιλείας αὐτοῦ, ἐν τῷ μηνὶ τῷ πρώτῳ ἀνέῳξε τὰς θύρας οἴκου Κυρίου καὶ ἐπεσκεύασεν αὐτάς.
\VS{4}Καὶ εἰσήγαγε τοὺς ἱερεῖς καὶ τοὺς Λευίτας, καὶ κατέστησεν αὐτοὺς εἰς τὸ κλίτος τὸ πρὸς ἀνατολὰς,
\VS{5}καὶ εἶπεν αὐτοῖς, ἀκούσατε οἱ Λευῖται· νῦν ἁγνίσθητε, καὶ ἁγνίσατε τὸν οἶκον Κυρίου Θεοῦ τῶν πατέρων ὑμῶν, καὶ ἐκβάλετε τὴν ἀκαθαρσίαν ἐκ τῶν ἁγίων·
\VS{6}Ὅτι ἀπέστησαν οἱ πατέρες ἡμῶν, καὶ ἐποίησαν τὸ πονηρὸν ἐναντίον Κυρίου Θεοῦ ἡμῶν, καὶ ἐγκατέλιπαν αὐτὸν, καὶ ἀπέστρεψαν τὸ πρόσωπον αὐτῶν ἀπὸ τῆς σκηνῆς Κυρίου, καὶ ἔδωκαν αὐχένα,
\VS{7}καὶ ἀπέκλεισαν τὰς θύρας τοῦ ναοῦ, καὶ ἔσβεσαν τοὺς λύχνους, καὶ θυμίαμα οὐκ ἐθυμίασαν, καὶ ὁλοκαυτώματα οὐ προσήνεγκαν ἐν τῷ ἁγίῳ Θεῷ Ἰσραήλ.
\VS{8}Καὶ ὠργίσθη ὀργῇ Κύριος ἐπὶ τὸν Ἰούδαν καὶ τὴν Ἱερουσαλὴμ, καὶ ἔδωκεν αὐτοὺς εἰς ἔκστασιν καὶ εἰς ἀφανισμὸν καὶ εἰς συρισμὸν ὡς ὑμεῖς ὁρᾶτε τοῖς ὀφθαλμοῖς ὑμῶν.
\VS{9}Καὶ ἰδοὺ πεπλήγασιν οἱ πατέρες ὑμῶν ἐν μαχαίρᾳ, καὶ οἱ υἱοὶ ὑμῶν καὶ αἱ θυγατέρες ὑμῶν καὶ αἱ γυναῖκες ὑμῶν ἐν αἰχμαλωσίᾳ ἐν γῇ οὐκ αὐτῶν, ὃ καὶ νῦν ἐστιν.
\VS{10}Ἐπὶ τούτοις νῦν ἐστιν ἐπὶ καρδίας διαθέσθαι διαθήκην μου, διαθήκην Κυρίου Θεοῦ Ἰσραὴλ, καὶ ἀποστρέψει τὴν ὀργὴν τοῦ θυμοῦ αὐτοῦ ἀφʼ ἡμῶν.
\VS{11}Καὶ νῦν μὴ διαλίπητε, ὅτι ἐν ὑμῖν ᾑρέτικε Κύριος στῆναι ἐναντίον αὐτοῦ λειτουργεῖν, καὶ εἶναι αὐτῷ λειτουργοῦντας καὶ θυμιῶντας.
\par }{\PP \VS{12}Καὶ ἀνέστησαν οἱ Λευῖται, Μαὰθ ὁ τοῦ Ἀμασὶ, καὶ Ἰωὴλ ὁ τοῦ Ἀζαρίου ἐκ τῶν υἱῶν Καάθ· καὶ ἐκ τῶν υἱῶν Μεραρὶ, Κὶς ὁ τοῦ Ἀβδὶ, καὶ Ἀζαρίας ὁ τοῦ Ἰλαελήλ· καὶ ἀπὸ τῶν υἱῶν Γεδσωνὶ, Ἰωδαὰδ ὁ τοῦ Ζεμμὰθ, καὶ Ἰωαδάμ· οὗτοι υἱοὶ Ἰωαχά.
\VS{13}Καὶ τῶν υἱῶν Ἐλισαφὰν, Ζαμβρὶ, καὶ Ἰεϊήλ· καὶ τῶν υἱῶν Ἀσὰφ, Ζαχαρίας καὶ Ματθανίας·
\VS{14}Καὶ τῶν υἱῶν Αἰμὰν, Ἰεϊὴλ καὶ Σεμεΐ· καὶ τῶν υἱῶν Ἰδιθοὺν, Σαμαίας, καὶ Ὀζιήλ.
\VS{15}Καὶ συνήγαγον τοὺς ἀδελφοὺς αὐτῶν, καὶ ἡγνίσθησαν κατὰ τὴν ἐντολὴν τοῦ βασιλέως διὰ προστάγματος Κυρίου, καθαρίσαι τὸν οἶκον Κυρίου.
\VS{16}Καὶ εἰσῆλθον οἱ ἱερεῖς ἔσω εἰς τὸν οἶκον Κυρίου ἁγνίσαι, καὶ ἐξέβαλον πᾶσαν τὴν ἀκαθαρσίαν τὴν εὑρεθεῖσαν ἐν τῷ οἴκῳ Κυρίου, καὶ εἰς τὴν αὐλὴν οἴκου Κυρίου· καὶ ἐδέξαντο οἱ Λευῖται ἐκβαλεῖν εἰς τὸν χειμάῤῥουν Κέδρων ἔξω.
\par }{\PP \VS{17}Καὶ ἤρξατο τῇ ἡμέρᾳ τῇ πρώτῃ νουμηνίᾳ τοῦ πρώτου μηνὸς ἁγνίσαι, καὶ τῇ ἡμέρᾳ τῇ ὀγδόῃ τοῦ μηνὸς εἰσῆλθαν εἰς τὸν ναὸν Κυρίου, καὶ ἥγνισαν τὸν οἶκον Κυρίου ἐν ἡμέραις ὀκτὼ, καὶ τῇ ἡμέρᾳ τῇ τρισκαιδεκάτῃ τοῦ μηνὸς τοῦ πρώτου συνετέλεσαν.
\par }{\PP \VS{18}Καὶ εἰσῆλθαν ἔσω πρὸς Ἐξεκίαν τὸν βασιλέα, καὶ εἶπαν, ἡγνίσαμεν πάντα τὰ ἐν οἴκῳ Κυρίου, τὸ θυσιαστήριον τῆς ὁλοκαυτώσεως καὶ τὰ σκεύη αὐτοῦ, καὶ τὴν τράπεζαν τῆς προθέσεως καὶ τὰ σκεύη αὐτῆς,
\VS{19}καὶ πάντα τὰ σκεύη ἃ ἐμίανεν ὁ βασιλεὺς Ἄχαζ ἐν τῇ βασιλείᾳ αὐτοῦ ἐν τῇ ἀποστασίᾳ αὐτοῦ, ἡτοιμάκαμεν καὶ ἡγνίσαμεν· ἰδού ἐστιν ἐναντίον τοῦ θυσιαστήριου Κυρίου.
\par }{\PP \VS{20}Καὶ ὤρθρισεν Ἐζεκίας ὁ βασιλεὺς, καὶ συνήγαγε τοὺς ἄρχοντας τῆς πόλεως, καὶ ἀνέβη εἰς οἶκον Κυρίου.
\VS{21}Καὶ ἀνήνεγκε μόσχους ἑπτὰ, κριοὺς ἑπτὰ, ἀμνοὺς ἑπτὰ, χιμάρους αἰγῶν ἑπτὰ περὶ ἁμαρτίας, περὶ τῆς βασιλείας, καὶ περὶ τῶν ἁγίων, καὶ περὶ Ἰσραήλ· καὶ εἶπε τοῖς υἱοῖς Ἀαρὼν τοῖς ἱερεῦσιν ἀναβαίνειν ἐπὶ τὸ θυσιαστήριον Κυρίου.
\VS{22}Καὶ ἔθυσαν τοὺς μόσχους, καὶ ἐδέξαντο οἱ ἱερεῖς τὸ αἷμα, καὶ προσέχεαν ἐπὶ τὸ θυσιαστήριον· καὶ ἔθυσαν τοὺς κριοὺς, καὶ προσέχεαν τὸ αἷμα ἐπὶ τὸ θυσιαστήριον· καὶ ἔθυσαν τοὺς ἀμνοὺς, καὶ περιέχεον τὸ αἷμα τῷ θυσιαστηρίῳ.
\VS{23}Καὶ προσήγαγον τοὺς χιμάρους τοὺς περὶ ἁμαρτίας ἐναντίον τοῦ βασιλέως καὶ τῆς ἐκκλησίας, καὶ ἐπέθηκαν τὰς χεῖρας αὐτῶν ἐπʼ αὐτούς.
\VS{24}Καὶ ἔθυσαν αὐτοὺς οἱ ἱερεῖς, καὶ ἐξιλάσαντο τὸ αἷμα αὐτῶν πρὸς τὸ θυσιαστήριον, καὶ ἐξιλάσαντο περὶ παντὸς Ἰσραὴλ, ὅτι εἶπεν ὁ βασιλεὺς, περὶ παντὸς Ἰσραὴλ ἡ ὁλοκαύτωσις, καὶ τὰ περὶ ἁμαρτίας.
\par }{\PP \VS{25}Καὶ ἔστησε τοὺς Λευίτας ἐν οἴκῳ Κυρίου ἐν κυμβάλοις, καὶ ἐν νάβλαις, καὶ ἐν κινύραις κατὰ τὴν ἐντολὴν Δαυὶδ τοῦ βασιλέως, καὶ Γὰδ τοῦ ὁρῶντος τῷ βασιλεῖ, καὶ Νάθαν τοῦ προφήτου, ὅτι διὰ ἐντολῆς Κυρίου τὸ πρόσταγμα ἐν χειρὶ τῶν προφητῶν.
\VS{26}Καὶ ἔστησαν οἱ Λευῖται ἐν ὀργάνοις Δαυὶδ, καὶ οἱ ἱερεῖς ταῖς σάλπιγξι.
\VS{27}Καὶ εἶπεν Ἐζεκίας ἀνένεγκαι τὴν ὁλοκαύτωσιν ἐπὶ τὸ θυσιαστήριον· καὶ ἐν τῷ ἄρξασθαι ἀναφέρειν τὴν ὁλοκαύτωσιν, ἤρξαντο ᾄδειν Κυρίῳ, καὶ σάλπιγγες πρὸς τὰ ὄργανα Δαυὶδ βασιλέως Ἰσραήλ.
\VS{28}Καὶ πᾶσα ἡ ἐκκλησία προσεκύνει, καὶ οἱ ψαλτῳδοὶ ᾄδοντες, καὶ σάλπιγγες σαλπίζουσαι ἕως οὗ συνετελέσθη ἡ ὁλοκαύτωσις.
\VS{29}Καὶ ὡς συνετέλεσαν ἀναφέροντες, ἔκαμψεν ὁ βασιλεὺς καὶ πάντες οἱ εὑρεθέντες, καὶ προσεκύνησαν.
\par }{\PP \VS{30}Καὶ εἶπεν Ἐζεκίας ὁ βασιλεὺς καὶ οἱ ἄρχοντες τοῖς Λευίταις, ὑμνεῖν τὸν Κύριον ἐν λόγοις Δαυὶδ καὶ Ἀσὰφ τοῦ προφήτου· καὶ ὕμνουν ἐν εὐφροσύνῃ, καὶ ἔπεσον καὶ προσεκύνησαν.
\par }{\PP \VS{31}Καὶ ἀπεκρίθη Ἐζεκίας, καὶ εἶπε, νῦν ἐπληρώσατε τὰς χεῖρας ὑμῶν Κυρίῳ, προσαγάγετε καὶ φέρετε θυσίας αἰνέσεως εἰς οἶκον Κυρίου· καὶ ἀνήνεγκεν ἡ ἐκκλησία θυσίας καὶ αἰνέσεις εἰς οἶκον Κυρίου, καὶ πᾶς πρόθυμος τῇ καρδίᾳ ὁλοκαυτώσεις.
\VS{32}Καὶ ἐγένετο ὁ ἀριθμὸς τῆς ὁλοκαυτώσεως ἧς ἀνήνεγκεν ἡ ἐκκλησία, μόσχοι ἑβδομήκοντα, κριοὶ ἑκατὸν, ἀμνοὶ διακόσιοι· εἰς ὁλοκαύτωσιν Κυρίῳ πάντα ταῦτα.
\VS{33}Καὶ οἱ ἠγιασμένοι μόσχοι ἑξακόσιοι, πρόβατα τρισχίλια.
\VS{34}Ἀλλʼ ἢ οἱ ἱερεῖς ἦσαν ὀλίγοι, καὶ οὐκ ἠδύναντο ἐκδεῖραι τὴν ὁλοκαύτωσιν, καὶ ἀντελάβοντο αὐτῶν οἱ ἀδελφοὶ αὐτῶν οἱ Λευῖται ἕως οὗ συνετελέσθη τὸ ἔργον, καὶ ἕως οὗ ἡγνίσθησαν οἱ ἱερεῖς· ὅτι οἱ Λευῖται προθύμως ἥγνισαν παρὰ τοὺς ἱερεῖς.
\VS{35}Καὶ ἡ ὁλοκαύτωσις πολλὴ ἐν τοῖς στέασι τῆς τελειώσεως τοῦ σωτηρίου καὶ τῶν σπονδῶν τῆς ὁλοκαυτώσεως· καὶ κατωρθώθη τὸ ἔργον ἐν οἴκῳ Κυρίου.
\par }{\PP \VS{36}Καὶ ηὐφράνθη Ἐζεκίας καὶ πᾶς ὁ λαὸς, διὰ τὸ ἡτοιμακέναι τὸν Θεὸν τῷ λαῷ, ὅτι ἐξάπινα ἐγένετο ὁ λόγος.

\par }\Chap{30}{\PP \VerseOne{1}Καὶ ἀπέστειλεν Ἐζεκίας ἐπὶ πάντα Ἰσραὴλ καὶ Ἰούδα, καὶ ἐπιστολὰς ἔγραψεν ἐπὶ τὸν Ἐφραὶμ καὶ Μανασσῆ, ἐλθεῖν εἰς οἶκον Κυρίου, εἰς Ἱερουσαλὴμ, ποιῆσαι τὸ φασὲκ τῷ Κυρίῳ Θεῷ Ἰσραήλ.
\VS{2}Καὶ ἐβουλεύσατο ὁ βασιλεὺς καὶ οἱ ἄρχοντες καὶ πᾶσα ἡ ἐκκλησία ἐν Ἱερουσαλὴμ ποιῆσαι τὸ φασὲκ τῷ μηνὶ τῷ δευτέρῳ.
\VS{3}Οὐ γὰρ ἠδυνάσθησαν ποιῆσαι αὐτὸ ἐν τῷ καιρῷ ἐκείνῳ, ὅτι οἱ ἱερεῖς οὐχ ἠγνίσθησαν ἱκανοὶ, καὶ ὁ λαὸς οὐ συνήχθη εἰς Ἱερουσαλήμ.
\VS{4}Καὶ ἤρεσεν ὁ λόγος ἐναντίον τοῦ βασιλέως καὶ ἐναντίον τῆς ἐκκλησίας.
\VS{5}Καὶ ἔστησαν λόγον διελθεῖν κήρυγμα ἐν παντὶ Ἰσραὴλ ἀπὸ Βηρσαβεὲ ἕως Δὰν, ἐλθόντας ποιῆσαι τὸ φασὲκ Κυρίῳ Θεῷ Ἰσραὴλ εἰς Ἱερουσαλὴμ, ὅτι πλῆθος οὐκ ἐποίησεν κατὰ τὴν γραφήν.
\par }{\PP \VS{6}Καὶ ἐπορεύθησαν οἱ τρέχοντες σὺν ταῖς ἐπιστολαῖς παρὰ τοῦ βασιλέως καὶ τῶν ἀρχόντων εἰς πάντα Ἰσραὴλ καὶ Ἰούδαν κατὰ τὸ πρόσταγμα τοῦ βασιλέως, λέγοντες, οἱ υἱοὶ Ἰσραὴλ ἐπιστρέψατε πρὸς Κύριον Θεὸν Ἁβραὰμ καὶ Ἰσαὰκ καὶ Ἰσραὴλ, καὶ ἐπιστρέψατε τοὺς ἀνασεσωσμένους τοὺς καταλειφθέντας ἀπὸ χειρὸς βασιλέως Ἀσσούρ.
\VS{7}Καὶ μὴ γίνεσθε καθὼς οἱ πατέρες ὑμῶν καὶ οἱ ἀδελφοὶ ὑμῶν, οἳ ἀπέστησαν ἀπὸ Κυρίου Θεοῦ πατέρων αὐτῶν, καὶ παρέδωκεν αὐτοὺς εἰς ἐρήμωσιν καθὼς ὑμεῖς ὁρᾶτε.
\VS{8}Καὶ νῦν μὴ σκληρύνητε τὰς καρδίας ὑμῶν ὡς οἱ πατέρες ὑμῶν, δότε δόξαν Κυρίῳ τῷ Θεῷ, καὶ εἰσέλθετε εἰς τὸ ἁγίασμα αὐτοῦ ὃ ἡγίασεν εἰς τὸν αἰῶνα, καὶ δουλεύσατε τῷ Κυρίῳ Θεῷ ὑμῶν, καὶ ἀποστρέψει ἀφʼ ὑμῶν θυμὸν ὀργῆς.
\VS{9}Ὅτι ἐν τῷ ἐπιστρέφειν ὑμᾶς πρὸς Κύριον, οἱ ἀδελφοὶ ὑμῶν καὶ τὰ τέκνα ὑμῶν ἔσονται ἐν οἰκτιρμοῖς ἔναντι πάντων τῶν αἰχμαλωτισάντων αὐτοὺς, καὶ ἀποστρέψει εἰς τὴν γῆν ταύτην· ὅτι ἐλεήμων καὶ οἰκτίρμων Κύριος ὁ Θεὸς ἡμῶν, καὶ οὐκ ἀποστρέψει τὸ πρόσωπον αὐτοῦ ἀφʼ ὑμῶν, ἐὰν ἐπιστρέψωμεν πρὸς αὐτόν.
\par }{\PP \VS{10}Καὶ ἦσαν οἱ τρέχοντες διαπορευόμενοι πόλιν ἐκ πόλεως ἐν τῷ ὄρει Ἐφραὶμ, καὶ Μανασσῆ, καὶ ἕως Ζαβουλών· καὶ ἐγένοντο ὡς καταγελῶντες αὐτῶν, καὶ καταμωκώμενοι.
\VS{11}Ἀλλὰ ἄνθρωποι Ἀσὴρ καὶ ἀπὸ Μανασσῆ καὶ ἀπὸ Ζαβουλὼν ἐνετράπησαν, καὶ ἦλθον εἰς Ἱερουσαλὴμ καὶ εἰς Ἰούδα.
\VS{12}Καὶ ἐγένετο χεὶρ Κυρίου δοῦναι αὐτοῖς καρδίαν μίαν ἐλθεῖν, τοῦ ποιῆσαι κατὰ τὰ προστάγματα τοῦ βασιλέως καὶ τῶν ἀρχόντων ἐν λόγῳ Κυρίου.
\par }{\PP \VS{13}Καὶ συνήχθησαν εἰς Ἱερουσαλὴμ λαὸς πολὺς τοῦ ποιῆσαι τὴν ἑορτὴν τῶν ἀζύμων ἐν τῷ μηνὶ τῷ δευτέρῳ, ἐκκλησία πολλὴ σφόδρα.
\VS{14}Καὶ ἀνέστησαν, καὶ καθεῖλαν τὰ θυσιαστήρια τὰ ἐν Ἱερουσαλὴμ, καὶ πάντα ἐν οἷς ἐθυμίων τοῖς ψευδέσι, κατέσπασαν καὶ ἔῤῥιψαν εἰς τὸν χειμάῤῥουν Κέδρων.
\VS{15}Καὶ ἔθυσαν τὸ φασὲκ τῇ τεσσαρεσκαιδεκάτῃ τοῦ μηνὸς τοῦ δευτέρου· καὶ οἱ ἱερεῖς καὶ οἱ Λευῖται ἐνετράπησαν καὶ ἥγνισαν, καὶ εἰσήνεγκαν ὁλοκαυτώματα ἐν οἴκῳ Κυρίου.
\par }{\PP \VS{16}Καὶ ἔστησαν ἐπὶ τὴν στάσιν αὐτῶν, κατὰ τὸ κρίμα αὐτῶν, κατὰ τὴν ἐντολὴν Μωυσῆ ἀνθρώπου τοῦ Θεοῦ· καὶ οἱ ἱερεῖς ἐδέχοντο τὰ αἵματα ἐκ χειρὸς τῶν Λευιτῶν.
\VS{17}Ὅτι πλῆθος τῆς ἐκκλησίας οὐχ ἡγνίσθη, καὶ οἱ Λευῖται ἦσαν τοῦ θύειν τὸ φασὲκ παντὶ τῷ μὴ δυναμένῳ ἁγνισθῆναι τῷ Κυρίῳ.
\VS{18}Ὅτι πλεῖστον τοῦ λαοῦ ἀπὸ Ἐφραὶμ, καὶ Μανασσῆ, καὶ Ἰσσάχαρ, καὶ Ζαβουλὼν, οὐκ ἥγνισαν, ἀλλʼ ἔφαγον τὸ φασὲκ παρὰ τὴν γραφήν· τοῦτο καὶ προσηύξατο Ἐζεκίας περὶ αὐτῶν, λέγων, Κύριος ἀγαθὸς ἐξιλάσθω ὑπὲρ
\VS{19}πάσης καρδίας κατευθυνούσης ἐκζητῆσαι Κύριον τὸν Θεὸν τῶν πατέρων αὐτῶν, καὶ οὐ κατὰ τὴν ἁγνείαν τῶν ἁγίων.
\VS{20}Καὶ ἐπήκουσε Κύριος τῷ Ἐζεκίᾳ, καὶ ἰάσατο τὸν λαόν.
\par }{\PP \VS{21}Καὶ ἐποίησαν οἱ υἱοὶ Ἰσραὴλ οἱ εὑρεθέντες ἐν Ἱερουσαλὴμ, τὴν ἑορτὴν τῶν ἀζύμων ἑπτὰ ἡμέρας ἐν εὐφροσύνῃ μεγάλῃ, καὶ καθυμνοῦντες τῷ Κυρίῳ ἡμέραν καθʼ ἡμέραν, καὶ οἱ ἱερεῖς καὶ οἱ Λευῖται ἐν ὀργάνοις τῷ Κυρίῳ.
\VS{22}Καὶ ἐλάλησεν Ἐζεκίας ἐπὶ πᾶσαν καρδίαν τῶν Λευιτῶν καὶ τῶν συνιόντων σύνεσιν ἀγαθὴν τῷ Κυρίῳ· καὶ συνετέλεσαν τὴν ἑορτὴν τῶν ἀζύμων ἑπτὰ ἡμέρας, θύοντες θυσίαν σωτηρίου, καὶ ἐξομολογούμενοι τῷ Κυρίῳ Θεῷ τῶν πατέρων αὐτῶν.
\par }{\PP \VS{23}Καὶ ἐβουλεύσατο ἡ ἐκκλησία ἅμα ποιῆσαι ἑπτὰ ἡμέρας ἄλλας· καὶ ἐποίησαν ἑπτὰ ἡμέρας ἐν εὐφροσύνῃ.
\VS{24}Ὅτι Ἐζεκίας ἀπήρξατο τῷ Ἰούδα τῇ ἐκκλησίᾳ χιλίους μόσχους καὶ ἑπτακισχίλια πρόβατα, καὶ οἱ ἄρχοντες ἀπήρξαντο τῷ λαῷ μόσχους χιλίους καὶ πρόβατα δέκα χιλιάδας, καὶ τὰ ἅγια τῶν ἱερέων εἰς πλῆθος.
\VS{25}Καὶ ηὐφράνθη πᾶσα ἡ ἐκκλήσια οἱ ἱερεῖς καὶ οἱ Λευῖται, καὶ πᾶσα ἡ ἐκκλησία Ἰούδα, καὶ οἱ εὑρεθέντες ἐξ Ἱερουσαλὴμ, καὶ οἱ προσήλυτοι οἱ ἐλθόντες ἀπὸ γῆς Ἰσραὴλ, καὶ οἱ κατοικοῦντες Ἰούδα.
\VS{26}Καὶ ἐγένετο εὐφροσύνη μεγάλη ἐν Ἱερουσαλήμ· ἀπὸ ἡμερῶν Σαλωμὼν υἱοῦ Δαυὶδ βασιλέως Ἰσραὴλ οὐκ ἐγένετο τοιαύτη ἑορτὴ ἐν Ἱερουσαλήμ.
\VS{27}Καὶ ἀνέστησαν οἱ ἱερεῖς οἱ Λευῖται καὶ εὐλόγησαν τὸν λαὸν· καὶ ἐπηκούσθη ἡ φωνὴ αὐτῶν, καὶ ἦλθεν ἡ προσευχὴ αὐτῶν εἰς τὸ κατοικητήριον τὸ ἅγιον αὐτοῦ εἰς τὸν οὐρανόν.

\par }\Chap{31}{\PP \VerseOne{1}Καὶ ὡς συνετελέσθη πάντα ταῦτα, ἐξῆλθε πᾶς Ἰσραὴλ οἱ εὑρεθέντες ἐν πόλεσιν Ἰούδα, καὶ συνέτριψαν τὰς στήλας, καὶ ἔκοψαν τὰ ἄλση, καὶ κατέσπασαν τὰ ὑψηλὰ καὶ τοὺς βωμοὺς ἀπὸ πάσης τῆς Ἰουδαίας καὶ Βενιαμὶν, καὶ ἐξ Ἐφραὶμ, καὶ ἀπὸ Μανασσῆ ἕως εἰς τέλος· καὶ ἐπέστρεψαν πᾶς Ἰσραὴλ ἕκαστος εἰς τὴν κληρονομίαν αὐτοῦ, καὶ εἰς τὰς πόλεις αὐτῶν.
\par }{\PP \VS{2}Καὶ ἔταξεν Ἐζεκίας τὰς ἐφημερίας τῶν ἱερέων καὶ τῶν Λευιτῶν, καὶ τὰς ἐφημερίας ἑκάστου κατὰ τὴν ἑαυτοῦ λειτουργίαν, τοῖς ἱερεῦσι καὶ τοῖς Λευίταις, εἰς τὴν ὁλοκαύτωσιν, καὶ εἰς τὴν θυσίαν τοῦ σωτηρίου, καὶ αἰνεῖν, καὶ ἐξομολογεῖσθαι, καὶ λειτουργεῖν ἐν ταῖς πύλαις ἐν ταῖς αὐλαῖς οἴκου Κυρίου.
\VS{3}Καὶ μερὶς τοῦ βασιλέως ἐκ τῶν ὑπαρχόντων αὐτοῦ εἰς τὰς ὁλοκαυτώσεις τὴν πρωϊνὴν καὶ τὴν δειλινὴν, καὶ ὁλοκαυτώσεις εἰς τὰ σάββατα, καὶ εἰς τὰς νουμηνίας, καὶ εἰς τὰς ἑορτὰς τὰς γεγραμμένας ἐν τῷ νόμῳ Κυρίου.
\par }{\PP \VS{4}Καὶ εἶπαν τῷ λαῷ τοῖς κατοικοῦσιν ἐν Ἱερουσαλὴμ, δοῦναι τὴν μεριδα τῶν ἱερέων καὶ τῶν Λευιτῶν, ὅπως κατισχύσωσιν ἐν τῇ λειτουργίᾳ οἴκου Κυρίου.
\VS{5}Καὶ ὡς προσέταξεν τὸν λόγον, ἐπλεόνασεν Ἰσραὴλ ἀπαρχὴν σίτου, καὶ οἴνου, καὶ ἐλαίου, καὶ μέλιτος, καὶ πᾶν γέννημα ἀγροῦ, καὶ ἐπιδέκατα πάντα εἰς πλῆθος ἤνεγκαν οἱ υἱοὶ Ἰσραὴλ καὶ Ἰούδα.
\VS{6}Καὶ οἱ κατοικοῦντες ἐν ταῖς πόλεσιν Ἰούδα καὶ αὐτοὶ ἤνεγκαν ἐπιδέκατα μόσχων καὶ προβάτων, καὶ ἐπιδέκατα αἰγῶν, καὶ ἡγίασαν τῷ Κυρίῳ Θεῷ αὐτῶν, καὶ εἰσήνεγκαν καὶ ἔθηκαν σωροὺς σωρούς.
\VS{7}Ἐν τῷ μηνὶ τῷ τρίτῳ ἤρξαντο οἱ σωροὶ θεμελιοῦσθαι, καὶ ἐν τῷ μηνὶ τῷ ἑβδόμῳ συνετελέσθησαν.
\VS{8}Καὶ ἦλθεν Ἐζεκίας καὶ οἱ ἄρχοντες, καὶ εἶδον τοὺς σωροὺς, καὶ ηὐλόγησαν τὸν Κύριον καὶ τὸν λαὸν αὐτοῦ Ἰσραήλ·
\VS{9}Καὶ ἐπυνθάνετο Ἐζεκίας τῶν ἱερέων καὶ τῶν Λευιτῶν ὑπὲρ τῶν σωρῶν.
\VS{10}Καὶ εἶπε πρὸς αὐτὸν Ἀζαρίας ὁ ἱερεὺς ὁ ἄρχων εἰς οἶκον Σαδὼκ, καὶ εἶπεν, ἐξ οὗ ἦρκται ἡ ἀπαρχὴ φέρεσθαι εἰς οἶκον Κυρίου, ἐφάγομεν καὶ ἐπίομεν καὶ κατελίπομεν ἕως εἰς πλῆθος, ὅτι Κύριος ηὐλόγησε τὸν λαὸν αὐτοῦ, καὶ κατελίπομεν ἐπὶ τὸ πλῆθος τοῦτο.
\par }{\PP \VS{11}Καὶ εἶπεν Ἐζεκίας ἔτι ἑτοιμάσαι παστοφόρια εἰς οἶκον Κυρίου· καὶ ἡτοίμασαν,
\VS{12}καὶ ἤνεγκαν ἐκεῖ τὰς ἀπαρχὰς καὶ τὰ ἐπιδέκατα ἐν πίστει· καὶ ἐπʼ αὐτῶν ἐπιστάτης Χωνενίας ὁ Λευίτης, καὶ Σεμεῒ ὁ ἀδελφὸς αὐτοῦ διαδεχόμενος·
\VS{13}Καὶ Ἰεϊὴλ, καὶ Ὀζίας, καὶ Ναὲθ, καὶ Ἀσαὴλ, καὶ Ἰεριμὼθ, καὶ Ἰωζαβὰδ, καὶ Ἐλιὴλ, καὶ ὁ Σαμαχία, καὶ Μαὰθ, καὶ Βαναΐας, καὶ οἱ υἱοὶ αὐτοῦ καθεσταμένοι διὰ Χωνενίου καὶ Σεμεῒ τοῦ ἀδελφοῦ αὐτοῦ, καθὼς προσέταξεν Ἐζεκίας ὁ βασιλεὺς καὶ Ἀζαρίας ὁ ἡγούμενος οἴκου Κυρίου.
\par }{\PP \VS{14}Καὶ Κορὴ ὁ τοῦ Ἰεμνὰ ὁ Λευίτης ὁ πυλωρὸς κατὰ ἀνατολὰς ἐπὶ τῶν δομάτων, δοῦναι τὰς ἀπαρχὰς Κυρίου, καὶ τὰ ἅγια τῶν ἁγίων,
\VS{15}διὰ χειρὸς Ὀδὸμ, καὶ Βενιαμὶν, καὶ Ἰησοὺς, καὶ Σεμεῒ, καὶ Ἀμαρίας, καὶ Σεχονίας, διὰ χειρὸς τῶν ἱερέων ἐν πίστει, δοῦναι τοῖς ἀδελφοῖς αὐτῶν κατὰ τὰς ἐφημερίας, κατὰ τὸν μέγαν καὶ τὸν μικρὸν,
\VS{16}ἐκτὸς τὴς ἐπιγονῆς τῶν ἀρσενικῶν ἀπὸ τριετοῦς καὶ ἐπάνω, παντὶ τῷ εἰσπορευομένῳ εἰς οἶκον Κυρίου, εἰς λόγον ἡμερῶν εἰς ἡμέραν, εἰς λειτουργείαν ἐφημερίαις διατάξεως αὐτῶν·
\VS{17}Οὗτος ὁ καταλοχισμὸς τῶν ἱερέων κατʼ οἴκους πατριῶν· καὶ οἱ Λευῖται ἐν ταῖς ἐφημερίαις αὐτῶν ἀπὸ εἰκοσαετοῦς καὶ ἐπάνω ἐν διατάξει,
\VS{18}ἐγκαταλοχίσαι ἐν πάσῃ ἐπιγονῇ υἱῶν αὐτῶν καὶ θυγατέρων αὐτῶν εἰς πᾶν πλῆθος, ὅτι ἐν πίστει ἥγνισαν τὸ ἅγιον·
\VS{19}Τοῖς υἱοῖς Ἀαρὼν τοῖς ἱερατεύουσι, καὶ οἱ ἀπὸ τῶν πόλεων αὐτῶν ἐν πάσῃ πόλει καὶ πόλει ἄνδρες οἳ ὠνομάσθησαν ἐν ὀνόματι, δοῦναι μερίδα παντὶ ἀρσενικῷ ἐν τοῖς ἱερεῦσι, καὶ παντὶ καταριθμουμένῳ ἐν τοῖς Λευίταις.
\par }{\PP \VS{20}Καὶ ἐποίησεν οὕτως Ἐζεκίας ἐν παντὶ Ἰούδα, καὶ ἐποίησε τὸ καλὸν καὶ τὸ εὐθὲς ἐναντίον τοῦ Κυρίου Θεοῦ αὐτοῦ.
\VS{21}Καὶ ἐν παντὶ ἔργῳ ᾧ ἤρξατο ἐν ἐργασίᾳ ἐν οἴκῳ Κυρίου, καὶ ἐν τῷ νόμῳ καὶ ἐν τοῖς προστάγμασιν, ἐξεζήτησε τὸν Θεὸν αὐτοῦ ἐξ ὅλης ψυχῆς αὐτοῦ, καὶ ἐποίησε καὶ εὐοδώθη.

\par }\Chap{32}{\PP \VerseOne{1}Καὶ μετὰ τοὺς λόγους τούτους καὶ τὴν ἀλήθειαν ταύτην ἦλθε Σενναχηρὶμ βασιλεὺς Ἀσσυρίων, καὶ ἦλθεν ἐπὶ Ἰούδαν, καὶ παρενέβαλεν ἐπὶ τὰς πόλεις τὰς τειχήρεις, καὶ εἶπε προκαταλαβέσθαι αὐτάς.
\par }{\PP \VS{2}Καὶ εἶδεν Ἐζεκίας ὅτι ἥκει Σενναχηρὶμ, καὶ τὸ πρόσωπον αὐτοῦ τοῦ πολεμῆσαι ἐπὶ Ἱερουσαλήμ.
\VS{3}Καὶ ἐβουλεύσατο μετὰ τῶν πρεσβυτέρων αὐτοῦ καὶ τῶν δυνατῶν, ἐμφράξαι τὰ ὕδατα τῶν πηγῶν ἃ ἦν ἔξω τῆς πόλεως, καὶ συνεπίσχυσαν αὐτῷ.
\VS{4}Καὶ συνήγαγε λαὸν πολὺν, καὶ ἐνέφραξε τὰ ὕδατα τῶν πηγῶν, καὶ τὸν ποταμὸν τὸν διορίζοντα διὰ τῆς πόλεως, λέγων, μὴ ἔλθῃ βασιλεὺς Ἀσσοὺρ, καὶ εὕρῃ ὕδωρ πολὺ, καὶ κατισχύσῃ.
\par }{\PP \VS{5}Καὶ κατίσχυσεν Ἐζεκίας, καὶ ᾠκοδόμησε πᾶν τὸ τεῖχος τὸ κατεσκαμμένον, καὶ πύργους, καὶ ἔξω προτείχισμα ἄλλο, καὶ κατίσχυσε τὸ ἀνάλημμα τῆς πόλεως Δαυὶδ, καὶ κατεσκεύασεν ὅπλα πολλά.
\VS{6}Καὶ ἔθετο ἄρχοντας τοῦ πολέμου ἐπὶ τὸν λαὸν, καὶ συνήχθησαν πρὸς αὐτὸν ἐπὶ τὴν πλατεῖαν τῆς πύλης τῆς φάραγγος, καὶ ἐλάλησεν ἐπὶ καρδίαν αὐτῶν, λέγων,
\VS{7}ἰσχύσατε καὶ ἀνδρίζεσθε, καὶ μὴ φοβηθῆτε, μηδὲ πτοηθῆτε ἀπὸ προσώπου βασιλέως Ἀσσοὺρ, καὶ ἀπὸ προσώπου παντὸς τοῦ ἔθνους τοῦ μετʼ αὐτοῦ, ὅτι μεθʼ ἡμῶν πλείονες ἢ μετʼ αὐτοῦ.
\VS{8}Μετὰ αὐτοῦ βραχίονες σάρκινοι, μεθʼ ἡμῶν δὲ Κύριος ὁ Θεὸς ἡμῶν τοῦ σώζειν καὶ τοῦ πολεμεῖν τὸν πόλεμον ἡμῶν· καὶ κατεθάρσησεν ὁ λαὸς ἐπὶ τοῖς λόγοις Ἐζεκίου βασιλέως Ἰούδα.
\par }{\PP \VS{9}Καὶ μετὰ ταῦτα ἀπέστειλε Σενναχηρὶμ βασιλεὺς Ἀσσυρίων τοὺς παῖδας ἑαυτοῦ ἐπὶ Ἱερουσαλὴμ, καὶ αὐτὸς ἐπὶ Λαχὶς, καὶ πᾶσα ἡ στρατιὰ μετʼ αὐτοῦ, καὶ ἀπέστειλε πρὸς Ἐζεκίαν βασιλέα Ἰούδα, καὶ πρὸς πάντα Ἰούδα τὸν ἐν Ἱερουσαλὴμ, λέγων,
\VS{10}οὕτως λέγει Σενναχηρὶμ βασιλεὺς Ἀσσυρίων, ἐπὶ τί ὑμεῖς πεποίθατε, καὶ καθήσεσθε ἐν τῇ περιοχῇ ἐν Ἱερουσαλήμ;
\VS{11}οὐχὶ Ἐζεκίας ἀπατᾷ ὑμᾶς τοῦ παραδοῦναι ὑμᾶς εἰς θάνατον καὶ εἰς λιμὸν καὶ εἰς δίψαν, λέγων, Κύριος ὁ Θεὸς ἡμῶν σώσει ἡμᾶς ἐκ χειρὸς βασιλέως Ἀσσούρ;
\VS{12}Οὐχ οὗτός ἐστιν Ἐζεκίας ὃς περιεῖλε τὰ θυσιαστήρια αὐτοῦ, καὶ τὰ ὑψηλὰ αὐτοῦ, καὶ εἶπε τῷ Ἰούδα καὶ τοῖς κατοικοῦσιν ἐν Ἱερουσαλὴμ, λέγων, κατέναντι τοῦ θυσιαστηρίου τούτου προσκυνήσετε, καὶ ἐπʼ αὐτῷ θυμιάσατε;
\VS{13}Οὐ γνώσεσθε ὅ, τι ἐποίησα ἐγὼ καὶ οἱ πατέρες μου πᾶσι τοῖς λαοῖς τῶν χωρῶν; μὴ δυνάμενοι ἠδύναντο θεοὶ τῶν ἐθνῶν πάσης τῆς γῆς σῶσαι τὸν λαὸν αὐτῶν ἐκ χειρός μου;
\VS{14}Τίς ἐν πᾶσι τοῖς θεοῖς τῶν ἐθνῶν τούτων οὓς ἐξωλόθρευσαν οἱ πατέρες μου; μὴ ἐδύναντο σῶσαι τὸν λαὸν αὐτῶν ἐκ χειρός μου, ὅτι δυνήσεται ὁ Θεὸς ὑμῶν σῶσαι ὑμᾶς ἐκ χειρός μου;
\VS{15}Νῦν οὖν μὴ ἀπατάτω ὑμᾶς Ἐζεκίας, καὶ μὴ πεποιθέναι ὑμᾶς ποιείτω κατὰ ταῦτα, καὶ μὴ πιστεύετε αὐτῷ, ὅτι οὐ μὴ δύνηται ὁ θεὸς παντὸς ἔθνους καὶ βασιλείας τοῦ σῶσαι τὸν λαὸν αὐτοῦ ἐκ χειρός μου καὶ ἐκ χειρὸς πατέρων μου, ὅτι ὁ Θεὸς ὑμῶν οὐ μὴ σώσει ὑμᾶς ἐκ χειρός μου.
\VS{16}Καὶ ἔτι ἐλάλησαν οἱ παῖδες αὐτοῦ ἐπὶ τὸν Κύριον Θεὸν, καὶ ἐπὶ Ἐζεκίαν παῖδα αὐτοῦ.
\par }{\PP \VS{17}Καὶ βιβλίον ἔγραψεν ὀνειδίζειν τὸν Κύριον Θεὸν Ἰσραὴλ, καὶ εἶπε περὶ αὐτοῦ, λέγων, ὡς οἱ θεοὶ τῶν ἐθνῶν τῆς γῆς οὐκ ἐξείλαντο λαοὺς αὐτῶν ἐκ χειρός μου, οὕτως οὐ μὴ ἐξέληται ὁ Θεὸς Ἐζεκίου λαὸν αὐτοῦ ἐκ χειρός μου.
\VS{18}Καὶ ἐβόησε φωνῇ μεγάλῃ Ἰουδαϊστὶ ἐπὶ τὸν λαὸν Ἱερουσαλὴμ τὸν ἐπὶ τοῦ τείχους, τοῦ βοηθῆσαι αὐτοῖς, καὶ κατασπάσαι, ὅπως προκαταλάβωνται τὴν πόλιν.
\VS{19}Καὶ ἐλάλησεν ἐπὶ Θεὸν Ἱερουσαλὴμ, ὡς καὶ ἐπὶ θεοὺς λαῶν τῆς γῆς, ἔργα χειρῶν ἀνθρώπων.
\par }{\PP \VS{20}Καὶ προσηύξατο Ἐζεκίας ὁ βασιλεὺς, καὶ Ἡσαΐας υἱὸς Ἀμὼς ὁ προφήτης περὶ τούτων, καὶ ἐβόησαν εἰς τὸν οὐρανόν.
\VS{21}Καὶ ἀπέστειλε Κύριος ἄγγελον, καὶ ἐξέτριψε πάντα δυνατὸν καὶ πολεμιστὴν καὶ ἄρχοντα καὶ στρατηγὸν ἐν τῇ παρεμβολῇ βασιλέως Ἀσσούρ· καὶ ἀπέστρεψε μετὰ αἰσχύνης προσώπου εἰς τὴν γῆν ἑαυτοῦ, καὶ ἦλθεν εἰς οἶκον θεοῦ αὐτοῦ· καὶ τῶν ἐξελθόντων ἐκ κοιλίας αὐτοῦ κατέβαλον αὐτὸν ἐν ῥομφαίᾳ.
\VS{22}Καὶ ἔσωσε Κύριος τὸν Ἐζεκίαν καὶ τοὺς κατοικοῦντας ἐν Ἱερουσαλὴμ ἐκ χειρὸς Σενναχηρὶμ βασιλέως Ἀσσοὺρ, καὶ ἐκ χειρὸς πάντων, καὶ κατέπαυσεν αὐτοὺς κυκλόθεν.
\VS{23}Καὶ πολλοὶ ἔφερον δῶρα τῷ Κυρίῳ εἰς Ἱερουσαλὴμ, καὶ δόματα τῷ Ἐζεκίᾳ βασιλεῖ Ἰούδα, καὶ ὑπερῄρθη κατʼ ὀφθαλμοὺς πάντων τῶν ἐθνῶν μετὰ ταῦτα.
\par }{\PP \VS{24}Ἐν ταῖς ἡμέραις ἐκείναις ἠῤῥώστησεν Ἐζεκίας ἕως θανάτου, καὶ προσηύξατο πρὸς Κύριον· καὶ ἐπήκουσεν αὐτῷ, καὶ σημεῖον ἔδωκεν αὐτῷ.
\VS{25}Καὶ οὐ κατὰ τὸ ἀνταπόδομα ὃ ἔδωκεν αὐτῷ ἀνταπέδωκεν Ἐζεκίας, ἀλλὰ ὑψώθη ἡ καρδία αὐτοῦ, καὶ ἐγένετο ἐπʼ αὐτὸν ὀργὴ καὶ ἐπὶ Ἰούδαν καὶ Ἱερουσαλήμ.
\VS{26}Καὶ ἐταπεινώθη Ἐζεκίας ἀπὸ τοῦ ὕψους τῆς καρδίας αὐτοῦ, αὐτὸς καὶ οἱ κατοικοῦντες Ἱερουσαλὴμ, καὶ οὐκ ἐπῆλθεν ἐπʼ αὐτοὺς ὀργὴ Κυρίου ἐν ταῖς ἡμέραις Ἐζεκίου.
\VS{27}Καὶ ἐγένετο τῷ Ἐζεκίᾳ πλοῦτος καὶ δόξα πολλὴ σφόδρα· καὶ θησαυροὺς ἐποίησεν αὐτῷ ἀργυρίου καὶ χρυσίου καὶ τοῦ λίθου τοῦ τιμίου, καὶ εἰς τὰ ἀρώματα, καὶ ὁπλοθήκας, καὶ εἰς σκεύη ἐπιθυμητὰ,
\VS{28}καὶ πόλεις εἰς τὰ γεννήματα τοῦ σίτου καὶ οἴνου καὶ ἐλαίου, καὶ κώμας καὶ φάτνας παντὸς κτήνους, καὶ μάνδρας εἰς τὰ ποίμνια,
\VS{29}καὶ πόλεις ἃς ᾠκοδόμησεν αὐτῷ, καὶ ἀποσκευὴν προβάτων καὶ βοῶν εἰς πλῆθος, ὅτι ἔδωκεν αὐτῷ Κύριος ἀποσκευὴν πολλὴν σφόδρα.
\par }{\PP \VS{30}Αὐτὸς Ἐζεκίας ἐνέφραξεν τὴν ἔξοδον τοῦ ὕδατος Γειῶν τὸ ἄνω, καὶ κατηύθυνεν αὐτὰ κάτω πρὸς Λίβα τῆς πόλεως Δαυίδ· καὶ εὐοδώθη Ἐζεκίας ἐν πᾶσι τοῖς ἔργοις αὐτοῦ.
\VS{31}Καὶ οὕτως τοῖς πρεσβευταῖς τῶν ἀρχόντων ἀπὸ Βαβυλῶνος, τοῖς ἀποσταλεῖσι πρὸς αὐτὸν πυθέσθαι παρʼ αὐτοῦ τὸ τέρας ὃ ἐγένετο ἐπὶ τῆς γῆς, ἐγκατέλιπεν αὐτὸν Κύριος τοῦ πειράσαι αὐτὸν, εἰδέναι τὰ ἐν τῇ καρδίᾳ αὐτοῦ.
\par }{\PP \VS{32}Καὶ τὰ λοιπὰ τῶν λόγων Ἐζεκίου, καὶ τὸ ἔλεος αὐτοῦ, ἰδοὺ γέγραπται ἐν τῇ προφητείᾳ Ἡσαΐου υἱοῦ Ἀμὼς τοῦ προφήτου, καὶ ἐπὶ βιβλίου βασιλέων Ἰούδα καὶ Ἰσραήλ.
\VS{33}Καὶ ἐκοιμήθη Ἐζεκίας μετὰ τῶν πατέρων αὐτοῦ, καὶ ἔθαψαν αὐτὸν ἐν ἀναβάσει τάφων υἱῶν Δαυίδ· καὶ δόξαν καὶ τιμὴν ἔδωκαν αὐτῷ ἐν τῷ θανάτῳ αὐτοῦ πᾶς Ἰούδα, καὶ οἱ κατοικοῦντες ἐν Ἱερουσαλήμ· καὶ ἐβασίλευσε Μανασσῆς υἱὸς αὐτοῦ ἀντʼ αὐτοῦ.

\par }\Chap{33}{\PP \VerseOne{1}Ὢν δεκαδύο ἐτῶν Μανασσῆς ἐν τῷ βασιλεῦσαι αὐτὸν, καὶ πεντηκονταπέντε ἔτη ἐβασίλευσεν ἐν Ἱερουσαλήμ.
\VS{2}Καὶ ἐποίησε τὸ πονηρὸν ἐναντίον Κυρίου ἀπὸ πάντων τῶν βδελυγμάτων τῶν ἐθνῶν, οὓς ἐξωλόθρευσε Κύριος ἀπὸ προσώπου τῶν υἱῶν Ἰσραήλ.
\VS{3}Καὶ ἐπέστρεψε καὶ ᾠκοδόμησε τὰ ὑψηλὰ, ἃ κατέσπασεν Ἐζεκίας ὁ πατὴρ αὐτοῦ, καὶ ἔστησε στήλας τοῖς Βααλὶμ, καὶ ἐποίησεν ἄλση, καὶ προσεκύνησεν πάσῃ τῇ στρατιᾷ τοῦ οὐρανοῦ, καὶ ἐδούλευσεν αὐτοῖς.
\VS{4}Καὶ ᾠκοδόμησε θυσιαστήρια ἐν οἴκῳ Κυρίου, οὗ εἶπε Κύριος, ἐν Ἱερουσαλὴμ ἔσται τὸ ὄνομά μου εἰς τὸν αἰῶνα.
\VS{5}Καὶ ᾠκοδόμησε θυσιαστήρια πάσῃ τῇ στρατιᾷ τοῦ οὐρανοῦ ἐν ταῖς δυσὶν αὐλαῖς οἴκου Κυρίου.
\VS{6}Καὶ αὐτὸς διήγαγε τὰ τέκνα αὐτοῦ ἐν πυρὶ ἐν γὲ Βενεννόμ· καὶ ἐκληδονίζετο, καὶ οἰωνίζετο, καὶ ἐφαρμακεύετο, καὶ ἐποίησεν ἐγγαστριμύθους καὶ ἐπᾳοιδοὺς, καὶ ἐπλήθυνε τοῦ ποιῆσαι τὸ πονηρὸν ἐναντίον Κυρίου τοῦ παροργίσαι αὐτόν.
\VS{7}Καὶ ἔθηκε τὸ γλυπτὸν, τὸ χωνευτὸν, εἰκόνα ἣν ἐποίησεν ἐν οἴκῳ Θεοῦ, οὗ εἶπε Θεὸς πρὸς Δαυὶδ καὶ πρὸς Σαλωμὼν υἱὸν αὐτοῦ, ἐν τῷ οἴκῳ τούτῳ καὶ Ἱερουσαλὴμ ἣν ἐξελεξάμην ἐκ πασῶν φυλῶν Ἰσραὴλ, θήσω τὸ ὄνομά μου εἰς τὸν αἰῶνα.
\VS{8}Καὶ οὐ προσθήσω σαλεῦσαι τὸν πόδα Ἰσραὴλ ἀπὸ τῆς γῆς ἧς ἔδωκα τοῖς πατράσιν αὐτῶν, πλὴν ἐὰν φυλάσσωνται τοῦ ποιῆσαι πάντα ἃ ἐνετειλάμην αὐτοῖς κατὰ πάντα τὸν νόμον καὶ τὰ προστάγματα καὶ τὰ κρίματα ἐν χειρὶ Μωυσῆ.
\VS{9}Καὶ ἐπλάνησε Μανασσῆς τὸν Ἰούδαν καὶ τοὺς κατοικοῦντας ἐν Ἱερουσαλὴμ, τοῦ ποιῆσαι τὸ πονηρὸν ὑπὲρ πάντα τὰ ἔθνη ἃ ἐξῇρε Κύριος ἀπὸ προσώπου υἱῶι Ἰσραήλ.
\par }{\PP \VS{10}Καὶ ἐλάλησε Κύριος ἐπὶ Μανασσῆ καὶ ἐπὶ τὸν λαὸν αὐτοῦ, καὶ οὐκ ἐπήκουσαν.
\VS{11}Καὶ ἤγαγε Κύριος ἐπʼ αὐτοὺς τοὺς ἄρχοντας τῆς δυνάμεως τοῦ βασιλέως Ἀσσοὺρ, καὶ κατέλαβον τὸν Μανασσῆ ἐν δεσμοῖς, καὶ ἔδησαν αὐτὸν ἐν πέδαις, καὶ ἤγαγον εἰς Βαβυλῶνα.
\VS{12}Καὶ ὡς ἐθλίβη, ἐζήτησε τὸ πρόσωπον Θεοῦ τοῦ Κυρίου αὐτοῦ, καὶ ἐταπεινώθη σφόδρα ἀπὸ προσώπου Θεοῦ πατέρων αὐτοῦ,
\VS{13}καὶ προσηύξατο πρὸς αὐτόν· καὶ ἐπήκουσεν αὐτοῦ καὶ ἐπήκουσε τῆς βοῆς αὐτοῦ, καὶ ἐπέστρεψεν αὐτὸν εἰς Ἱερουσαλὴμ ἐπὶ τὴν βασιλείαν αὐτοῦ, καὶ ἔγνω Μανασσῆς ὅτι Κύριος αὐτός ἐστι Θεός.
\par }{\PP \VS{14}Καὶ μετὰ ταῦτα ᾠκοδόμησε τεῖχος ἔξω τῆς πόλεως Δαυὶδ ἀπὸ Λιβὸς κατὰ Νότον ἐν τῷ χειμάῤῥῳ, καὶ κατὰ τὴν εἴσοδον τὴν διὰ τῆς πύλης τῆς ἰχθυϊκῆς ἐκπορευομένων τὴν πύλην τὴν κυκλόθεν, καὶ εἰς Ὄπελ, καὶ ὕψωσε σφόδρα, καὶ κατέστησεν ἄρχοντας τῆς δυνάμεως ἐν πάσαις ταῖς πόλεσι ταῖς τειχήρεσιν ἐν Ἰούδα.
\VS{15}Καὶ περιεῖλε τοὺς θεοὺς τοὺς ἀλλοτρίους καὶ τὸ γλυπτὸν ἐξ οἴκου Κυρίου, καὶ πάντα τὰ θυσιαστήρια, ἃ ᾠκοδόμησεν ἐν ὄρει οἴκου Κυρίου καὶ ἐν Ἱερουσαλὴμ, καὶ ἔξωθεν τῆς πόλεως·
\VS{16}Καὶ κατώρθωσε τὸ θυσιαστήριον Κυρίου, καὶ ἐθυσίασεν ἐπʼ αὐτὸ θυσίαυ σωτηρίου καὶ αἰνέσεως· καὶ εἶπε τῷ Ἰούδα, τοῦ δουλεύειν Κυρίῳ Θεῷ Ἰσραήλ.
\VS{17}Πλὴν ἔτι ὁ λαὸς ἐπὶ τῶν ὑψηλῶν ἐθυσίαζε, πλὴν Κυρίῳ Θεῷ αὐτῶν.
\par }{\PP \VS{18}Καὶ τὰ λοιπὰ τῶν λόγων Μανασσῆ, καὶ ἡ προσευχὴ αὐτοῦ ἡ πρὸς τὸν Θεὸν, καὶ λόγοι τῶν ὁρώντων τῶν λαλούντων πρὸς αὐτὸν ἐπʼ ὀνόματι Θεοῦ Ἰσραήλ,
\VS{19}ἰδοὺ ἐπὶ λόγων προσευχῆς αὐτοῦ, καὶ ἐπήκουσεν αὐτοῦ· καὶ πᾶσαι αἱ ἁμαρτίαι αὐτοῦ καὶ ἀποστάσεις αὐτοῦ, καὶ οἱ τόποι ἐφʼ οἷς ᾠκοδόμησεν ἐν αὐτοῖς τὰ ὑψηλὰ, καὶ ἔστησεν ἐκεῖ ἄλση καὶ γλυπτὰ, πρὸ τοῦ ἐπιστρέψαι, ἰδοὺ γέγραπται ἐπὶ τῶν λόγων τῶν ὁρώντων.
\VS{20}Καὶ ἐκοιμήθη Μανασσῆς μετὰ τῶν πατέρων αὐτοῦ, καὶ ἔθαψαν αὐτὸν ἐν παραδείσῳ οἴκου αὐτοῦ· καὶ ἐβασίλευσεν ἀντʼ αὐτοῦ Ἀμὼν υἱὸς αὐτοῦ.
\par }{\PP \VS{21}Ὢν ἐτῶν εἴκοσι καὶ δύο Ἀμὼν ἐν τῷ βασιλεύειν αὐτὸν, καὶ δύο ἔτη ἑβασιλευσεν ἐν Ἱερουσαλήμ.
\VS{22}Καὶ ἐποίησε τὸ πονηρὸν ἐνώπιον Κυρίου, ὡς ἐποίησε Μανασσῆς ὁ πατὴρ αὐτοῦ· καὶ πᾶσι τοῖς εἰδώλοις οἷς ἐποίησε Μανασσῆς ὁ πατὴρ αὐτοῦ, ἔθυεν Ἀμὼν καὶ ἐδούλευσεν αὐτοῖς.
\VS{23}Καὶ οὐκ ἐταπεινώθη ἐναντίον Κυρίου ὡς ἐταπεινώθη Μανασσῆς ὁ πατὴρ αὐτοῦ, ὅτι υἱὸς αὐτοῦ Ἀμὼν ἐπλήθυνε πλημμέλειαν.
\VS{24}Καὶ ἐπέθεντο αὐτῷ οἱ παῖδες αὐτοῦ, καὶ ἐπάταξαν αὐτὸν ἐν οἴκῳ αὐτοῦ.
\VS{25}Καὶ ἐπάταξεν ὁ λαὸς τῆς γῆς τοῦς ἐπιθεμένους ἐπὶ τὸν βασιλέα Ἀμὼν, καὶ ἐβασίλευσεν ὁ λαὸς τῆς γῆς τὸν Ἰωσίαν υἱὸν αὐτοῦ ἀντʼ αὐτοῦ.

\par }\Chap{34}{\PP \VerseOne{1}Ὢν ὀκτὼ ἐτῶν Ἰωσίας ἐν τῷ βασιλεῦσαι αὐτὸν, καὶ τριάκοντα καὶ ἓν ἔτος ἐβασίλευσεν ἐν Ἱερουσαλήμ.
\VS{2}Καὶ ἐποίησε τὸ εὐθὲς ἐναντίον Κυρίου, καὶ ἐπορεύθη ἐν ὁδοῖς Δαυὶδ τοῦ πατρὸς αὐτοῦ, καὶ οὐκ ἐξέκλινε δεξιὰ καὶ ἀριστερά.
\VS{3}Καὶ ἐν τῷ ὀγδόῳ ἔτει τῆς βασιλείας αὐτοῦ, καὶ αὐτὸς ἔτι παιδάριον ἤρξατο τοῦ ζητῆσαι Κύριον τὸν Θεὸν Δαυὶδ τοῦ πατρὸς αὐτοῦ· καὶ ἐν τῷ δωδεκάτῳ ἔτει τῆς βασιλείας αὐτοῦ ἤρξατο τοῦ καθαρίσαι τὸν Ἰούδαν καὶ τὴν Ἱερουσαλὴμ ἀπὸ τῶν ὑψηλῶν, καὶ τῶν ἄλσεων, καὶ ἀπὸ τῶν περιβωμίων, καὶ ἀπὸ τῶν χωνευτῶν.
\VS{4}Καὶ κατέσπασε τὰ κατὰ πρόσωπον αὐτοῦ θυσιαστήρια τῶν Βααλὶμ, καὶ τὰ ὑψηλὰ τὰ ἐπʼ αὐτῶν· καὶ ἔκοψε τὰ ἄλση καὶ τὰ γλυπτὰ, καὶ τὰ χωνευτὰ συνέτριψε, καὶ ἐλέπτυνε καὶ ἔῤῥιψεν ἐπὶ πρόσωπον τῶν μνημάτων τῶν θυσιαζόντων αὐτοῖς.
\VS{5}Καὶ ὀστᾶ ἱερέων κατέκαυσεν ἐπὶ τὰ θυσιαστήρια, καὶ ἐκαθάρισε τὸν Ἰούδαν καὶ τὴν Ἱερουσαλὴμ,
\VS{6}καὶ ἐν πόλεσι Μανασσῆ, καὶ Ἐφραὶμ, καὶ Συμεὼν, καὶ Νεφθαλὶ, καὶ τοῖς τόποις αὐτῶν κύκλῳ.
\VS{7}Καὶ κατέσπασε τὰ θυσιαστήρια, καὶ τὰ ἄλση, καὶ εἴδωλα κατέκοψε λεπτὰ, καὶ πάντα τὰ ὑψηλὰ ἔκοψεν ἀπὸ πάσης τῆς γῆς Ἰσραὴλ, καὶ ἀπέστρεψεν εἰς Ἱερουσαλήμ.
\par }{\PP \VS{8}Καὶ ἐν τῷ ἔτει τῷ ὀκτωκαιδεκάτῳ τῆς βασιλείας αὐτοῦ τοῦ καθαρίσαι τὴν γῆν καὶ τὸν οἶκον, ἀπέστειλε τὸν Σαφὰν υἱὸν Ἐσελία, καὶ τὸν Μαασὰ ἄρχοντα τῇς πόλεως, καὶ τὸν Ἰουὰχ υἱὸν Ἰωάχαζ τὸν ὑπομνηματογράφον αὐτοῦ, κραταιῶσαι τὸν οἶκον Κυρίου τοῦ Θεοῦ αὐτοῦ.
\VS{9}Καὶ ἦλθον πρὸς Χελκίαν τὸν ἱερέα τὸν μέγαν, καὶ ἔδωκαν τὸ ἀργύριον τὸ εἰσενεχθὲν εἰς οἶκον Θεοῦ, ὃ συνήγαγον οἱ Λευῖται φυλάσσοντες τὴν πύλην ἐκ χειρὸς Μανασσῆ καὶ Ἐφραὶμ, καὶ τῶν ἀρχόντων, καὶ ἀπὸ παντὸς καταλοίπου ἐν Ἰσραὴλ, καὶ υἱῶν Ἰούδα καὶ Βενιαμὶν, καὶ οἰκούντων ἐν Ἱερουσαλήμ.
\VS{10}Καὶ ἔδωκαν αὐτὸ ἐπὶ χεῖρα ποιούντων τὰ ἔργα, οἱ καθεσταμένοι ἐν οἴκῳ Κυρίου, καὶ ἔδωκαν αὐτὸ ποιοῦσι τὰ ἔργα οἳ ἐποίουν ἐν οἴκῳ Κυρίου, ἐπισκευάσαι καὶ κατισχύσαι τὸν οἶκον.
\VS{11}Καὶ ἔδωκαν τοῖς τέκτοσι καὶ τοῖς οἰκοδόμοις, ἀγοράσαι λίθους τετραπέδους καὶ ξύλα εἰς δοκοὺς στεγάσαι τοὺς οἴκους, οὓς ἐξωλόθρευσαν βασιλεῖς Ἰούδα.
\VS{12}Καὶ οἱ ἄνδρες ἐν πίστει ἐπὶ τῶν ἔργων· καὶ ἐπʼ αὐτῶν ἐπίσκοποι, Ἰὲθ καὶ Ἀβδίας οἱ Λευῖται ἐξ υἱῶν Μεραρὶ, καὶ Ζαχαρίας καὶ Μοσολλὰμ ἐκ τῶν υἱῶν Καὰθ ἐπισκοπεῖν, καὶ πᾶς Λευίτης, καὶ πᾶς συνιὼν ἐν ὀργάνοις ᾠδῶν.
\VS{13}Καὶ ἐπὶ τῶν νωτοφόρων, καὶ ἐπὶ πάντων τῶν ποιούντων τὰ ἔργα, ἐργασίᾳ καὶ ἐργασίᾳ· καὶ ἀπὸ τῶν Λευιτῶν γπαμματεῖς καὶ κριταὶ καὶ πυλωροί.
\par }{\PP \VS{14}Καὶ ἐν τῷ ἐκφέρειν αὐτοὺς τὸ ἀργύριον τὸ εἰσοδιασθὲν εἰς οἶκον Κυρίου, εὗρε Χελκίας ὁ ἱερεὺς βιβλίον νόμου Κυρίου διὰ χειρὸς Μωυσῆ.
\VS{15}Καὶ ἀπεκρίθη Χελκίας, καὶ εἶπε πρὸς Σαφὰν τὸν γραμματέα, βιβλίον νόμου εὗρον ἐν οἴκῳ Κυρίου· καὶ ἔδωκε Χελκίας τὸ βιβλίον τῷ Σαφάν.
\VS{16}Καὶ εἰσήνεγκε Σαφὰν τὸ βιβλίον πρὸς τὸν βασιλέα, καὶ ἀπέδωκεν ἔτι τῷ βασιλεῖ λόγον, πᾶν τὸ δοθὲν ἀργύριον ἐν χειρὶ τῶν παίδων σον τῶν ποιούντων.
\VS{17}Καὶ ἐχώνευσαν τὸ ἀργύριον τὸ εὑρεθὲν ἐν οἴκῳ Κυρίου, καὶ ἔδωκαν ἐπὶ χεῖρα τῶν ἐπισκόπων, καὶ ἐπὶ χεῖρα τῶν ποιούντων τὴν ἐργασίαν.
\par }{\PP \VS{18}Καὶ ἀπήγγειλε Σαφὰν ὁ γραμματεὺς τῷ βασιλεῖ λόγον, λέγων, βιβλίον δέδωκέ μοι Χελκίας ὁ ἱερεύς· καὶ ἀνέγνω αὐτὸ Σαφὰν ἐναντίον τοῦ βασιλέως.
\VS{19}Καὶ ἐγένετο ὡς ἤκουσεν ὁ βασιλεὺς τοὺς λόηους τοῦ νόμου, καὶ διέῤῥηξε τὰ ἱμάτια αὐτοῦ.
\VS{20}Καὶ ἐνετείλατο ὁ βασιλεὺς τῷ Χελκίᾳ καὶ τῷ Ἀχικὰμ υἱῷ Σαφὰν καὶ τῷ Ἀβδὸμ υἱῷ Μιχαία καὶ τῷ Σαφὰν τῷ γραμματεῖ καὶ τῷ Ἀσαΐᾳ παιδὶ τοῦ βασιλέως, λέγων,
\VS{21}πορεύθητε, ζητήσατε τὸν Κύριον περὶ ἐμοῦ καὶ περὶ παντὸς τοῦ καταλειφθέντος ἐν Ἰσραὴλ καὶ Ἰούδα περὶ τῶν λόγων τοῦ βιβλίου τοῦ εὑρεθέντος, ὅτι μέγας ὁ θυμὸς Κυρίου ἐκκέκαυται ἐν ἡμῖν, διότι οὐκ εἰσήκουσαν οἱ πατέρες ἡμῶν τῶν λόγων Κυρίου, τοῦ ποιῆσαι κατὰ πάντα τὰ γεγραμμένα ἐν τῷ βιβλίῳ τούτῳ.
\par }{\PP \VS{22}Καὶ ἐπορεύθη Χελκίας, καὶ οἷς εἶπεν ὁ βασιλεὺς, πρὸς Ὀλδὰν τὴν προφῆτιν, γυναῖκα Σελλὴμ υἱοῦ Θεκωὲ υἱοῦ Ἀρὰς, φυλάσσουσαν τὰς ἐντολὰς, καὶ αὕτη κατῴκει ἐν Ἱερουσαλὴμ ἐν μασαναὶ, καὶ ἐλάλησαν αὐτῇ κατὰ ταῦτα.
\par }{\PP \VS{23}Καὶ εἶπεν αὐτοῖς, οὕτως εἶπε Κύριος ὁ Θεὸς Ἰσραὴλ, εἴπατε τῷ ἀνδρὶ τῷ ἀποστείλαντι ὑμᾶς πρὸς μὲ,
\VS{24}οὕτω λέγει Κύριος, ἰδοὺ ἐγὼ ἐπάγω ἐπὶ τὸν τόπον τοῦτον κακὰ, τοὺς πάντας λόγους τοὺς γεγραμμένους ἐν τῷ βιβλίῳ τῷ ἀνεγνωσμένῳ ἐναντίον τοῦ βασιλέως Ἰούδα,
\VS{25}ἀνθʼ ὧν ἐγκατέλιπόν με καὶ ἐθυμίασαν θεοῖς ἀλλοτρίοις, ἵνα παροργίσωσίν με ἐν πᾶσιν τοῖς ἔργοις τῶν χειρῶν αὐτῶν· καὶ ἐξεκαύθη ὁ θυμός μου ἐν τῷ τόπῳ τούτῳ, καὶ οὐ σβεσθήσεται.
\VS{26}Καὶ ἐπὶ βασιλέα Ἰούδα τὸν ἀποστείλαντα ὑμᾶς τοῦ ζητῆσαι τὸν Κύριον, οὕτως ἐρεῖτε αὐτῷ, οὕτω λέγει Κύριος ὁ Θεὸς Ἰσραὴλ, τοὺς λόγους οὓς ἤκουσας,
\VS{27}καὶ ἐνετράπη ἡ καρδία σου, καὶ ἐταπεινώθης ἀπὸ προσώπου μου ἐν τῷ ἀκοῦσαί σε τοὺς λόγους μου ἐπὶ τὸν τόπον τοῦτον καὶ ἐπὶ τοὺς κατοικοῦντας αὐτὸν, καὶ ἐταπεινώθης ἐναντίον μου, καὶ διέῤῥηξας τὰ ἱμάτιά σου, καὶ ἔκλαυσας κατεναντίον μου, καὶ ἐγὼ ἤκουσα, φησὶ Κύριος.
\VS{28}Ἰδοὺ προστίθημί σε πρὸς τοὺς πατέρας σου, καὶ προστεθήσῃ πρὸς τὰ μνήματά σου ἐν εἰρήνῃ, καὶ οὐκ ὄψονται οἱ ὀφθαλμοί σου ἐν πᾶσι τοῖς κακοῖς οἷς ἐγὼ ἐπάγω ἐπὶ τὸν τόπον τοῦτον, καὶ ἐπὶ τοὺς κατοικοῦντας αὐτόν. Καὶ ἀπέδωκαν τῷ βασιλεῖ λόγον.
\par }{\PP \VS{29}Καὶ ἀπέστειλεν ὁ βασιλεὺς, καὶ συνήγαγε τοὺς πρεσβυτέρους Ἰούδα καὶ Ἱερουσαλήμ.
\VS{30}Καὶ ἀνέβη ὁ βασιλεὺς εἰς οἶκον Κυρίου, καὶ πᾶς Ἰούδα, καὶ οἱ κατοικοῦντες Ἱερουσαλὴμ, καὶ οἱ ἱερεῖς, καὶ οἱ Λευῖται, καὶ πᾶς ὁ λαὸς ἀπὸ μικροῦ ἕως μεγάλου, καὶ ἀνέγνω ἐν ὠσὶν αὐτῶν πάντας λόγους βιβλίου τῆς διαθήκης τοὺς εὑρεθέντας ἐν οἴκῳ Κυρίου.
\VS{31}Καὶ ἔστη ὁ βασιλεὺς ἐπὶ τὸν στύλον, καὶ διέθετο διαθήκην ἐναντίον Κυρίου, τοῦ πορευθῆναι ἐνώπιον Κυρίου, τοῦ φυλάσσειν τὰς ἐντολὰς αὐτοῦ, καὶ μαρτύρια, καὶ προστάγματα αὐτοῦ ἐν ὅλῃ καρδίᾳ, καὶ ἐν ὅλῃ ψυχῇ, ὥστε ποιεῖν τοὺς λόγους τῆς διαθήκης τοὺς γεγραμμένους ἐπὶ τῷ βιβλίῳ τούτῳ.
\VS{32}Καὶ ἔστησε πάντας τοὺς εὑρεθέντας ἐν Ἱερουσαλὴμ καὶ Βενιαμίν· καὶ ἐποίησαν οἱ κατοικοῦντες Ἱερουσαλὴμ διαθήκην ἐν οἴκῳ Κυρίου Θεοῦ πατέρων αὐτῶν.
\par }{\PP \VS{33}Καὶ περιεῖλεν Ἰωσίας τὰ πάντα βδελύγματα ἐκ πάσης τῆς γῆς, ἣ ἦν υἱῶν Ἰσραὴλ, καὶ ἐποίησε πάντας τοὺς εὑρεθέντας ἐν Ἱερουσαλὴμ καὶ ἐν Ἰσραὴλ, τοῦ δουλεύειν Κυρίῳ Θεῷ αὐτῶν πάσας τὰς ἡμέρας αὐτοῦ· οὐκ ἐξέκλινεν ἀπὸ ὄπισθε Κυρίου Θεοῦ πατέρων αὐτοῦ.

\par }\Chap{35}{\PP \VerseOne{1}Καὶ ἐποίησεν Ἰωσίας τὸ φασὲχ τῷ Κυρίῳ Θεῷ αὐτοῦ, καὶ ἔθυσε τὸ φασὲκ τῇ τεσσαρεσκαιδεκάτῃ ἡμέρᾳ τοῦ μηνὸς τοῦ πρώτου.
\VS{2}Καὶ ἔστησε τοὺς ἱερεῖς ἐπὶ τὰς φυλακὰς αὐτῶν, καὶ κατίσχυσεν αὐτοὺς εἰς τὰ ἔργα οἴκου Κυρίου.
\VS{3}Καὶ εἶπε τοῖς Λευίταις τοῖς δυνατοῖς ἐν παντὶ Ἰσραὴλ, τοῦ ἁγιασθῆναι αὐτοὺς τῷ Κυρίῳ· καὶ ἔθηκαν τὴν κιβωτὸν τὴν ἁγίαν εἰς τὸν οἶκον ὃν ᾠκοδόμησε Σαλωμὼν υἱὸς Δαυὶδ τοῦ βασιλέως Ἰσραήλ. καὶ εἶπεν ὁ βασιλεὺς, οὐκ ἔστιν ὑμῖν ἐπʼ ὤμων ἆραι οὐδέν· νῦν οὖν λειτουργήσατε τῷ Κυρίῳ Θεῷ ὑμῶν, καὶ τῷ λαῷ αὐτοῦ Ἰσραήλ.
\VS{4}Καὶ ἑτοιμάσθητε κατʼ οἴκους πατριῶν ὑμῶν, καὶ κατὰ τὰς ἐφημερίας ὑμῶν, κατὰ τὴν γραφὴν Δαυὶδ βασιλέως Ἰσραὴλ, καὶ διὰ χειρὸς Σαλωμὼν υἱοῦ αὐτοῦ.
\VS{5}Καὶ στῆτε ἐν τῷ οἴκῳ κατὰ τὰς διαιρέσεις οἴκων πατριῶν ὑμῶν τοῖς ἀδελφοῖς ὑμῶν υἱοῖς τοῦ λαοῦ, καὶ μερὶς οἴκου πατριᾶς τοῖς Λευίταις.
\VS{6}Καὶ θύσατε τὸ φασὲκ, καὶ ἑτοιμάσατε τοῖς ἀδελφοῖς ὑμῶν, τοῦ ποιῆσαι κατὰ τὸν λόγον Κυρίου διὰ χειρὸς Μωυσῆ.
\par }{\PP \VS{7}Καὶ ἀπήρξατο Ἰωσίας τοῖς υἱοῖς τοῦ λαοῦ πρόβατα, καὶ ἀμνοὺς, καὶ ἐρίφους ἀπὸ τῶν τέκνων τῶν αἰγῶν, πάντα εἰς τὸ φασὲκ, καὶ πάντας τοὺς εὑρεθέντας εἰς ἀριθμὸν τριάκοντα χιλιάδας, καὶ μόσχων τρεῖς χιλιάδας, ταῦτα ἀπὸ τῆς ὑπάρξεως τοῦ βασιλέως.
\VS{8}Καὶ οἱ ἄρχοντες αὐτοῦ ἀπήρξαντο τῷ λαῷ καὶ τοῖς ἱερεῦσι καὶ τοῖς Λευίταις· ἔδωκε δὲ Χελκίας καὶ Ζαχαρίας καὶ Ἰεϊὴλ οἱ ἄρχοντες τοῖς ἱερεῦσιν οἴκου Θεοῦ, καὶ ἔδωκαν εἰς τὸ φασὲκ πρόβατα καὶ ἀμνοὺς καὶ ἐρίφους διαχίλια ἑξακόσια, καὶ μόσχους τριακοσίους.
\VS{9}Καὶ Χωνενίας, καὶ Βαναίας, καὶ Σαμαίας, καὶ Ναθαναὴλ ἀδελφὸς αὐτοῦ, καὶ Ἀσαβίας, καὶ Ἰεϊὴλ, καὶ Ἰωζαβὰδ, ἄρχοντες τῶν Λευιτῶν, ἀπήρξαντο τοῖς Λευίταις εἰς τὸ φασὲχ πρόβατα πεντακισχίλια, καὶ μόσχους πεντακοσίους.
\par }{\PP \VS{10}Καὶ κατωρθώθη ἡ λειτουργεία, καὶ ἔστησαν οἱ ἱερεῖς ἐπὶ τὴν στάσιν αὐτῶν, καὶ οἱ Λευῖται ἐπὶ τὰς διαιρέσεις αὐτῶν κατὰ τὴν ἐντολὴν τοῦ βασιλέως.
\VS{11}Καὶ ἔθυσαν τὸ φασὲκ, καὶ προσέχεαν οἱ ἱερεῖς τὸ αἷμα ἐκ χειρὸς αὐτῶν, καὶ οἱ Λευῖται ἐξέδειραν.
\VS{12}Καὶ ἡτοίμασαν τὴν ὁλοκαύτωσιν παραδοῦναι αὐτοῖς κατὰ τὴν διαίρεσιν κατʼ οἴκους πατριῶν τοῖς υἱοῖς τοῦ λαοῦ, τοῦ προσάγειν τῷ Κυρίῳ, ὡς γέγραπται ἐν βίβλῳ Μωυσῆ· καὶ οὕτως εἰς τὸ πρωΐ.
\VS{13}Καὶ ὤπτησαν τὸ φασὲκ ἐν πυρὶ κατὰ τὴν κρίσιν, καὶ τὰ ἅγια ἥψησαν ἐν τοῖς χαλκείοις καὶ ἐν τοῖς λέβησι, καὶ εὐωδώθη, καὶ ἔδραμον πρὸς πάντας τοὺς υἱοὺς τοῦ λαοῦ.
\par }{\PP \VS{14}Καὶ μετὰ τὸ ἑτοιμάσαι αὐτοῖς καὶ τοῖς ἱερεῦσιν ὅτι οἱ ἱερεῖς ἐν τῷ ἀναφέρειν τὰ ὁλοκαυτώματα καὶ τὰ στέατα ἕως νυκτὸς, καὶ οἱ Λευῖται ἡτοίμασαν αὐτοῖς, καὶ τοῖς ἀδελφοῖς αὐτῶν υἱοῖς Ἀαρών.
\VS{15}Καὶ οἱ ψαλτῳδοὶ υἱοὶ Ἀσὰφ ἐπὶ τῆς στάσεως αὐτῶν κατὰ τὰς ἐντολὰς Δαυὶδ, καὶ Ἀσὰφ, καὶ Αἰμὰν, καὶ Ἰδιθὼμ οἱ προφῆται τοῦ βασιλέως· καὶ οἱ ἄρχοντες καὶ οἱ πυλωροὶ πύλης καὶ πύλης, οὐκ ἦν αὐτοῖς κινεῖσθαι ἀπὸ τῆς λειτουργίας τῶν ἁγίων, ὅτι οἱ ἀδελφοὶ αὐτῶν οἱ Λευῖται ἡτοίμασαν αὐτοῖς.
\VS{16}Καὶ κατωρθώθη καὶ ἡτοιμάσθη πᾶσα ἡ λειτουργία Κυρίου ἐν τῇ ἡμέρᾳ ἐκείνῃ τοῦ ποιῆσαι τὸ φασὲκ, καὶ ἐνεγκεῖν τὰ ὁλοκαυτώματα ἐπὶ τὸ θυσιαστήριον Κυρίου κατὰ τὴν ἐντολὴν τοῦ βασιλέως Ἰωσίου.
\VS{17}Καὶ ἐποίησαν οἱ υἱοὶ Ἰσραὴλ οἱ εὑρεθέντες τὸ φασὲκ ἑν τῷ καιρῷ ἐκείνῳ, καὶ τὴν ἑορτὴν τῶν ἀζύμων ἑπτὰ ἡμέρας.
\par }{\PP \VS{18}Καὶ οὐκ ἐγένετο φασὲκ ὅμοιον αὐτῷ ἐν Ἰσραὴλ, ἀπὸ ἡμερῶν Σαμουὴλ τοῦ προφήτου καὶ παντὸς βασιλέως Ἰσραὴλ· οὐκ ἐποίησαν τὸ φασὲκ ὃ ἐποίησεν Ἰωσίας, καὶ οἱ ἱερεῖς, καὶ οἱ Λευῖται, καὶ πᾶς Ἰούδα καὶ Ἰσραὴλ ὁ εὑρεθεὶς, καὶ οἱ κατοικοῦντες ἐν Ἱερουσαλὴμ, τῷ Κυρίῳ.
\VS{19}Τῷ ὀκτωκαιδεκάτῳ ἔτει τῆς βασιλείας Ἰωσίου ἐποιήθη τὸ φασὲκ τοῦτο· μετὰ ταῦτα πάντα ἃ ἔδρασεν Ἰωσίας ἐν τῷ οἴκῳ,
\VS{19a}καὶ τοὺς ἐγγαστριμύθους καὶ τοὺς γνώστας καὶ τὰ θεραφὶν καὶ τὰ εἴδωλα καὶ τὰ καρησὶμ ἃ ἦν ἐν γῇ Ἰούδα καὶ ἐν Ἱερουσαλὴμ, ἐνεπύρισεν ὁ βασιλεὺς Ἰωσίας, ἵνα στήσῃ τοὺς λόγους τοῦ νόμου τοὺς γεγραμμένους ἐπὶ τοῦ βιβλίου οὗ εὗρε Χελκίας ὁ ἱερεὺς ἐν τῷ οἴκῳ Κυρίου·
\VS{19b}ὅμοιος αὐτῷ οὐκ ἐγενήθη ἔμπροσθεν αὐτοῦ, ὃς ἐπέστρεψε πρὸς Κύριον ἐν ὅλῃ καρδίᾳ αὐτοῦ καὶ ἐν ὅλῃ ψυχῇ αὐτοῦ καὶ ἐν ὅλῃ τῇ ἰσχύϊ αὐτοῦ κατὰ πάντα τὸν νόμον Μωυσῆ, καὶ μετʼ αὐτὸν οὐκ ἀνέστη ὅμοιος·
\VS{19c}πλὴν οὐκ ἀπεστράφη Κύριος ἀπὸ ὀργῆς θυμοῦ αὐτοῦ τοῦ μεγάλου, οὗ ὠργίσθη θυμῷ Κύριος ἐν τῷ Ἰούδα, ἐπὶ πάντα τὰ παροργίσματα αὐτοῦ ἂ παρώργισε Μανασσῆς·
\VS{19d}καὶ εἶπε Κύριος, καί γε τὸν Ἰούδαν ἀποστήσω ἀπὸ προσώπου μου, καθὼς ἀπέστησα τὸν Ἰσραὴλ, καὶ ἀπωσάμην τὴν πόλιν ἣν ἐξελεξάμην τὴν Ἱερουσαλὴμ, καὶ τὸν οἶκον ὃν εἶπα, ἔσται τὸ ὄνομά μου ἐκεῖ.
\par }{\PP \VS{20}Καὶ ἀνέβη Φαραὼ Νεχαὼ βασιλεὺς Αἰγύπτου ἐπὶ τὸν βασιλέα Ἀσσυρίων ἐπὶ τὸν ποταμὸν Εὐφράτην, καὶ ἐπορεύθη βασιλεὺς Ἰωσίας εἰς συνάντησιν αὐτῷ.
\VS{21}Καὶ ἀπέστειλε πρὸς αὐτὸν ἀγγέλους, λέγων, τί ἐμοὶ καὶ σοί βασιλεῦ Ἰούδα; οὐκ ἐπὶ σὲ ἥκω σήμερον πόλεμον πολεμῆσαι· καὶ ὁ Θεὸς εἶπε τοῦ κατασπεῦσαι με· πρόσεχε ἀπὸ τοῦ Θεοῦ τοῦ μετʼ ἐμοῦ, μὴ καταφθείρῃ σε.
\VS{22}Καὶ οὐκ ἀπέστρεψεν Ἰωσίας τὸ πρόσωπον αὐτοῦ ἀπʼ αὐτοῦ, ἀλλʼ ἢ πολεμεῖν αὐτὸν ἐκραταιώθη, καὶ οὐκ ἤκουσε τῶν λόγων Νεχαὼ διὰ στόματος Θεοῦ, καὶ ἦλθε τοῦ πολεμῆσαι ἐν τῷ πεδίῳ Μαγεδδώ.
\VS{23}Καὶ ἐτόξευσαν οἱ τοξόται ἐπὶ βασιλέα Ἰωσίαν· καὶ εἶπεν ὁ βασιλεὺς τοῖς παισὶν αὐτοῦ, ἐξαγάγετέ με, ὅτι ἐπόνεσα σφόδρα.
\VS{24}Καὶ ἐξήγαγον αὐτὸν οἱ παῖδες αὐτοῦ ἀπὸ τοῦ ἅρματος, καὶ ἀνεβίβασαν αὐτὸν ἐπὶ τὸ ἅρμα τὸ δευτερεῦον ὃ ἦν αὐτῷ, καὶ ἤγαγον αὐτὸν εἰς Ἱερουσαλὴμ, καὶ ἀπέθανε, καὶ ἐτάφη μετὰ τῶν πατέρων αὐτοῦ· καὶ πᾶς Ἰούδα καὶ Ἱερουσαλὴμ ἐπένθησαν ἐπὶ Ἰωσίαν.
\VS{25}Καὶ ἐθρήνησεν Ἱερεμίας ἐπὶ Ἰωσίαν, καὶ εἶπαν πάντες οἱ ἄρχοντες καὶ αἱ ἄρχουσαι θρῆνον ἐπὶ Ἰωσίαν ἕως τῆς σήμερον· καὶ ἔδωκαν αὐτὸν εἰς πρόσταγμα ἐπὶ Ἰσραὴλ, καὶ ἰδοὺ γέγραπται ἐπὶ τῶν θρήνων.
\par }{\PP \VS{26}Καὶ ἦσαν οἱ λοιποὶ λόγοι Ἰωσίου καὶ ἡ ἐλπὶς αὐτοῦ γεγραμμένα ἐν νόμῳ Κυρίου·
\VS{27}καὶ οἱ λόγοι αὐτοῦ οἱ πρῶτοι καὶ οἱ ἔσχατοι, ἰδοὺ γεγραμμένοι ἐπὶ βιβλίῳ βασιλέων Ἰσραὴλ καὶ Ἰούδα.

\par }\Chap{36}{\PP \VerseOne{1}Καὶ ἔλαβεν ὁ λαὸς τῆς γῆς τὸν Ἰωάχαζ υἱὸν Ἰωσίου, καὶ ἔχρισαν αὐτὸν, καὶ κατέστησαν αὐτὸν ἀντὶ τοῦ πατρὸς αὐτοῦ εἰς βασιλέα ἐπὶ Ἱερουσαλήμ.
\VS{2}Υἱὸς εἴκοσι καὶ τριῶν ἐτῶν Ἰωάχαζ ἐν τῷ βασιλεύειν αὐτὸν, καὶ τρίμηνον ἐβασίλευσεν ἐν Ἱερουσαλὴμ,
\VS{2a}καὶ ὄνομα τῆς μητρὸς αὐτοῦ Ἀμιτὰλ, θυγάτηρ Ἱερεμίου ἐκ Λοβνά·
\VS{2b}καὶ ἐποίησε τὸ πονηρὸν ἐνώπιον Κυρίου κατὰ πάντα ἃ ἐποίησαν οἱ πατέρες αὐτοῦ·
\VS{2c}καὶ ἔδησεν αὐτὸν Φαραὼ Νεχαὼ ἐν Δεβλαθὰ ἐν γῇ Αἰμὰθ, τοῦ μὴ βασιλεύειν αὐτὸν ἐν Ἱερουσαλήμ.
\VS{3}Καὶ μετήγαγεν αὐτὸν ὁ βασιλεὺς εἰς Αἴγυπτον, καὶ ἐπέβαλε φόρον ἐπὶ τὴν γῆν, ἑκατὸν τάλαντα ἀργυρίου καὶ τάλαντον χρυσίου.
\VS{4}Καὶ κατέστησε Φαραὼ Νεχαὼ τὸν Ἐλιακὶμ υἱὸν Ἰωσίου βασιλέα ἐπὶ Ἰούδα ἀντὶ Ἰωσίου τοῦ πατρὸς αὐτοῦ, καὶ μετέστρεψε τὸ ὄνομα αὐτοῦ Ἰωακίμ· καὶ τὸν Ἰωαχαζ ἀδελφὸν αὐτοῦ ἔλαβε Φαραὼ Νεχαὼ, καὶ εἰσήγαγεν αὐτὸν εἰς Αἴγυπτον, καὶ ἀπέθανεν ἐκεῖ·
\VS{4a}καὶ τὸ ἀργύριον καὶ τὸ χρυσίον ἔδωκε τῷ Φαραῷ· τότε ἤρξατο ἡ γῆ φορολογεῖσθαι τοῦ δοῦναι τὸ ἀργύριον ἐπὶ στόμα Φαραώ· καὶ ἕκαστος κατὰ δύναμιν ἀπῄτει τὸ ἀργύριον καὶ τὸ χρυσίον παρὰ τοῦ λαοῦ τῆς γῆς, δοῦναι Φαραῷ Νεχαῷ.
\par }{\PP \VS{5}Ὢν εἴκοσι καὶ πέντε ἐτῶν Ἰωακὶμ ἐν τῷ βασιλεύειν αὐτὸν, καὶ ἕνδεκα ἔτη ἐβασίλευσεν ἐν Ἱερουσαλὴμ, καὶ ὄνομα τῆς μητρὸς αὐτοῦ Ζεχωρὰ, θυγάτηρ Νηρίου ἐκ Ῥαμά. καὶ ἐποίησε τὸ πονηρὸν ἐναντίον Κυρίου κατὰ πάντα ὅσα ἐποίησαν οἱ πατέρες αὐτοῦ.
\VS{5a}Ἐν ταῖς ἡμέραις αὐτοῦ ἦλθε Ναβουχοδονόσορ ὁ βασιλεὺς Βαβυλῶνος εἰς τὴν γῆν, καὶ ἦν αὐτῷ δουλεύων τρία ἔτη, καὶ ἀπέστη ἀπʼ αὐτοῦ·
\VS{5b}καὶ ἀπέστειλε Κύριος ἐπʼ αὐτοὺς τοὺς Χαλδαίους, καὶ λῃστήρια Σύρων, καὶ λῃστήρια Μωαβιτῶν, καὶ υἱῶν Ἀμμὼν, καὶ τῆς Σαμαρείας, καὶ ἀπέστησαν μετὰ τὸν λόγον τοῦτον κατὰ τὸν λόγον Κυρίου ἐν χειρὶ τῶν παίδων αὐτοῦ τῶν προφητῶν·
\VS{5c}πλὴν θυμὸς Κυρίου ἦν ἐπὶ Ἰούδαν, τοῦ ἀποστῆναι αὐτὸν ἀπὸ προσώπου αὐτοῦ διὰ τὰς ἁμαρτίας Μανασσῆ ἐν πᾶσιν οἷς ἐποίησε,
\VS{5d}καὶ ἐν αἵματι ἀθώῳ ᾧ ἐξέχεεν Ἰωακὶμ, καὶ ἔπλησε τὴν Ἱερουσαλὴμ αἵματος ἀθώου, καὶ οὐκ ἠθέλησε Κύριος ἐξολοθρεῦσαι αὐτούς.
\VS{6}Καὶ ἀνέβη ἐπʼ αὐτὸν Ναβουχοδονόσορ βασιλεὺς Βαβυλῶνος, καὶ ἔδησεν αὐτὸν ἐν χαλκαῖς πέδαις, καὶ ἀπήγαγεν αὐτὸν εἰς Βαβυλῶνα.
\VS{7}Καὶ μήρος τῶν σκευῶν σἴκου Κυρίου ἀπήνεγκεν εἰς Βαβυλῶνα, καὶ ἔθηκεν αὐτὰ ἐν τῷ ναῷ αὐτοῦ ἐν Βαβυλῶνι.
\par }{\PP \VS{8}Καὶ τὰ λοιπὰ τῶν λόγων Ἰωακὶμ καὶ πάντα ἃ ἐποίησεν, οὐκ ἰδοὺ ταῦτα γεγραμμένα ἐν βιβλίῳ λόγων τῶν ἡμερῶν τοῖς βασιλεῦσιν Ἰούδα; καὶ ἐκοιμήθη Ἰωακὶμ μετὰ τῶν πατέρων αὐτοῦ, καὶ ἐτάφη ἐν γανοζαὴ μετὰ τὼν πατέρων αὐτοῦ, καὶ ἐβασίλευσεν Ἰεχονίας υἱὸς αὐτοῦ ἀντʼ αὐτοῦ.
\par }{\PP \VS{9}Ὀκτὼ ἐτῶν Ἰεχονίας ἐν τῷ βασιλεύειν αὐτὸν, καὶ τρίμηνον καὶ δέκα ἡμέρας ἐβασίλευσεν ἐν Ἱερουσαλὴμ, καὶ ἐποίησε τὸ πονηρὸν ἐνώπιον Κυρίου.
\VS{10}Καὶ ἐπιστρέφοντος τοῦ ἐνιαυτοῦ, ἀπέστειλεν ὁ βασιλεὺς Ναβουχοδονόσορ, καὶ εἰσήνεγκεν αὐτὸν εἰς Βαβυλῶνα μετὰ τῶν σκευῶν τῶν ἐπιθυμητῶν οἴκου Κυρίου· καὶ ἐβασίλευσε Σεδεκίαν ἀδελφὸν τοῦ πατρὸς αὐτοῦ ἐπὶ Ἰούδαν καὶ Ἱερουσαλήμ.
\par }{\PP \VS{11}Ἐτῶν εἴκοσι υἱὸς καὶ ἑνὸς ἔτου Σεδεκίας ἐν τῷ βασιλεύειν αὐτὸν, καὶ ἕνδεκα ἔτη ἐβασίλευσεν ἐν Ἱερουσαλήμ.
\VS{12}Καὶ ἐποίησε τὸ πονηρὸν ἐνώπιον Κορίου Θεοῦ αὐτοῦ, οὐκ ἐνετράπη ἀπὸ προσώπου Ἱερεμίου τοῦ προφήτου καὶ ἐκ στόματος Κυρίου,
\VS{13}ἐν τῷ τὰ πρὸς τὸν βασιλέα Ναβουχοδονόσορ ἀθετῆσαι, ἃ ὥρκισεν αὐτὸν κατὰ τοῦ Θεοῦ, καὶ ἐσκλήρυνε τὸν τράχηλον αὐτοῦ καὶ τὴν καρδίαν αὐτοῦ κατίσχυσε, τοῦ μὴ ἐπιστρέψαι πρὸς Κύριον Θεὸν Ἰσραήλ.
\VS{14}Καὶ πάντες οἱ ἔνδοξοι Ἰούδα, καὶ οἱ ἱερεῖς, καὶ ὁ λαὸς τῆς γῆς ἐπλήθυναν τοῦ ἀθετῆσαι ἀθετήματα βδελυγμάτων ἐθνῶν, καὶ ἐμίαναν τὸν οἶκον Κυρίου τὸν ἐν Ἱερουσαλήμ.
\VS{15}Καὶ ἐξαπέστειλε Κύριος ὁ Θεὸς τῶν πατέρων αὐτῶν ἐν χειρὶ τῶν προθητῶν αὐτοῦ, ὀρθρίζων καὶ ἀποστέλλων τοὺς ἀγγέλους αὐτοῦ, ὅτι ἦν φειδόμενος τοῦ λαοῦ αὐτοῦ, καὶ τοῦ ἁγιάσματος αὐτοῦ.
\VS{16}Καὶ ἦσαν μυκτηρίζοντες τοὺς ἀγγέλους αὐτοῦ, καὶ ἐξουθενοῦντες τοὺς λόγους αὐτοῦ, καὶ ἐμπαίζοντες ἐν τοῖς προφήταις αὐτοῦ, ἕως ἀνέβη ὁ θυμὸς Κυρίου ἐν τῷ λαῷ αὐτοῦ, ἕως οὐκ ἦν ἴαμα.
\par }{\PP \VS{17}Καὶ ἤγαγεν ἐπʼ αὐτοὺς βασιλέα Χαλδαίων, καὶ ἀπέκτεινε τοὺς νεανίσκους αὐτῶν ἐν ῥομφαίᾳ ἐν οἴκῳ ἁγιάσματος αὐτοῦ· καὶ οὐκ ἐφείσατο τοῦ Σεδεκίου, καὶ τὰς παρθένους αὐτῶν οὐκ ἠλέησε, καὶ τοὺς πρεσβυτέρους αὐτῶν ἀπήγαγον· τὰ πάντα παρέδωκεν ἐν χερσὶν αὐτῶν.
\VS{18}Καὶ πάντα τὰ σκεύη οἴκου τοῦ Θεοῦ τὰ μεγάλα καὶ τὰ μικρὰ, καὶ τοὺς θησαυροὺς, οἴκου καὶ Κυρίου, καὶ πάντας τοὺς θησαυροὺς τοῦ βασιλέως καὶ τῶν μεγιστάνων, πάντα εἰσήνεγκεν εἰς Βαβυλῶνα.
\par }{\PP \VS{19}Καὶ ἐνέπρησε τὸν οἶκον Κυρίου, καὶ κατέσκαψε τὸ τεῖχος Ἱερουσαλὴμ, καὶ τὰς βάρεις αὐτῆς ἐνέπρησεν ἐν πυρί, καὶ πᾶν σκεῦος ὡραῖον εἰς ἀφανισμόν.
\VS{20}Καὶ ἀπῴκισε τοὺς καταλοίπους εἰς Βαβυλῶνα, καὶ ἦσαν αὐτῷ καὶ τοῖς υἱοῖς αὐτοῦ εἰς δούλους ἕως βασιλείας Μήδων,
\VS{21}τοῦ πληρωθῆναι λόγον Κυρίου διὰ στόματος Ἱερεμίου, ἕως τοῦ προσδέξασθαι τὴν γῆν τὰ σάββατα αὐτῆς σαββατίσαι, πάσας τὰς ἡμέρας ἐρημώσεως αὐτῆς σαββατίσαι εἰς συμπλήρωσιν ἐτῶν ἑβδομήκοντα.
\par }{\PP \VS{22}Ἔτους πρώτου Κύρου βασιλέως Περσῶν, μετὰ τὸ πληρωθῆναι ῥῆμα Κυρίου διὰ στὸματος Ἱερεμίου, ἐξήγειρε Κύριος τὸ πνεῦμα Κύρου βασιλέως Περσῶν, καὶ παρήγγειλε κηρύξαι ἐν πάσῃ τῇ βασιλείᾳ αὐτοῦ ἑν γραπτῷ, λέγων,
\par }{\PP \VS{23}Τάδε λέγει Κῦρος βασιλεὺς Περσῶν πάσαις ταῖς βασιλείαις τῆς γῆς, ἔδωκέ μοι Κύριος ὁ Θεὸς τοῦ οὐρανοῦ, καὶ αὐτὸς ἐνετείλατό μοι οἰκοδομῆσαι οἶκον αὐτῷ ἐν Ἱερουσαλὴμ ἐν τῇ Ἰουδαίᾳ· τίς ἐξ ὑμῶν ἐκ παντὸς τοῦ λαοῦ αὐτοῦ; ἔσται Θεὸς αὐτοῦ μετʼ αὐτοῦ, καὶ ἀναβήτω.
\par }