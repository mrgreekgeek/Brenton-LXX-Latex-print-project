\NormalFont\ShortTitle{ΣΩΣΑΝΝΑ}
{\MT ΣΩΣΑΝΝΑ

\par }\OneChap {\PP \VerseOne{1}ΚΑΙ ἦν ἀνὴρ οἰκῶν ἐν Βαβυλῶνι, καὶ ὄνομα αὐτῷ Ἰωακείμ.
\VS{2}Καὶ ἔλαβε γυναῖκα ᾗ ὄνομα Σωσάννα, θυγάτηρ Χελκείου, καλὴ σφόδρα, καὶ φοβουμένη τὸν Κύριον.
\VS{3}Καὶ οἱ γονεῖς αὐτῆς δίκαιοι, καὶ ἐδίδαξαν τὴν θυγατέρα αὐτῶν κατὰ τὸν νόμον Μωυσῆ.
\VS{4}Καὶ ἦν Ἰωακεὶμ πλούσιος σφόδρα, καὶ ἦν αὐτῷ παράδεισος γειτνιῶν τῷ οἴκῳ αὐτοῦ· καὶ πρὸς αὐτὸν προσήγοντο οἱ Ἰουδαῖοι, διὰ τὸ εἶναι αὐτὸν ἐνδοξότερον πάντων.
\par }{\PP \VS{5}Καὶ ἀπεδείχθησαν δύο πρεσβύτεροι ἐκ τοῦ λαοῦ κριταὶ ἐν τῷ ἐνιαυτῷ ἐκείνῳ, περὶ ὧν ἐλάλησεν ὁ δεσπότης, ὅτι ἐξῆλθεν ἀνομία ἐκ Βαβυλῶνος ἐκ πρεσβυτέρων κριτῶν, οἳ ἐδόκουν κυβερνᾷν τὸν λαόν.
\VS{6}Οὗτοι προσεκαρτέρουν ἐν τῇ οἰκίᾳ Ἰωακεὶμ, καὶ ἤρχοντο πρὸς αὐτοὺς πάντες οἱ κρινόμενοι.
\par }{\PP \VS{7}Καὶ ἐγένετο ἡνίκα ἀπέτρεχεν ὁ λαὸς μέσον ἡμέρας, εἰσεπορεύετο Σωσάννα, καὶ περιεπάτει ἐν τῷ παραδείσῳ τοῦ ἀνδρὸς αὐτῆς.
\VS{8}Καὶ ἐθεώρουν αὐτὴν οἱ δύο πρεσβύτεροι καθʼ ἡμέραν εἰσπορευομένην, καὶ περιπατοῦσαν, καὶ ἐγένοντο ἐν ἐπιθυμίᾳ αὐτῆς,
\VS{9}καὶ διέστρεψαν τὸν ἑαυτῶν νοῦν, καὶ ἐξέκλιναν τοὺς ὀφθαλμοὺς αὐτῶν, τοῦ μὴ βλέπειν εἰς τὸν οὐρανὸν, μηδὲ μνημονεύειν κριμάτων δικαίων.
\VS{10}Καὶ ἦσαν ἀμφότεροι κατανενυγμένοι περὶ αὐτῆς, καὶ οὐκ ἀνήγγειλαν ἀλλήλοις τὴν ὀδύνην ἑαυτῶν·
\VS{11}Ὅτι ᾐσχύνοντο ἀναγγεῖλαι τὴν ἐπιθυμίαν αὐτῶν, ὅτι ἤθελον συγγενέσθαι αὐτῇ.
\VS{12}Καὶ παρετηροῦσαν φιλοτίμως καθʼ ἡμέραν ὁρᾷν αὐτήν.
\par }{\PP \VS{13}Καὶ εἶπαν ἕτερος τῷ ἑτέρῳ, πορευθῶμεν δὴ εἰς οἶκον, ὅτι ἀρίστου ὥρα ἐστί.
\VS{14}Καὶ ἐξελθόντες διεχωρίσθησαν ἀπʼ ἀλλήλων, καὶ ἀνακάμφαντες ἦλθον ἐπιτοαυτὸ, καὶ ἀνετάζοντες ἀλλήλους τὴν αἰτίαν, ὡμολόγησαν τὴν ἐπιθυμίαν αὐτῶν· καὶ τότε κοινῇ συνετάξαντο καιρὸν, ὅτε αὐτὴν δυνήσονται εὑρεῖν μόνην.
\par }{\PP \VS{15}Καὶ ἐγένετο ἐν τῷ παρατηρεῖν αὐτοὺς ἡμέραν εὔθετον, εἰσῆλθέ ποτε καθὼς χθὲς καὶ τρίτης ἡμέρας μετὰ δύο μόνων κορασίων, καὶ ἐπεθύμησε λούσασθαι ἐν τῷ παραδείσῳ, ὅτι καῦμα ἦν.
\VS{16}Καὶ οὐκ ἧν οὐδεὶς ἐκεῖ πλὴν οἱ δύο πρεσβύτεροι κεκρυμμένοι, καὶ παρατηροῦντες αὐτήν.
\VS{17}Καὶ εἶπε τοῖς κορασίοις, ἐνέγκατε δή μοι ἔλαιον καὶ σμήγματα, καὶ τας θύρας τοῦ παραδείσου κλείσατε, ὅπως λούσωμαι.
\par }{\PP \VS{18}Καὶ ἐποίησαν καθὼς εἶπε, καὶ ἀπέκλεισαν τὰς θύρας τοῦ παραδείσου, καὶ ἐξῆλθαν κατὰ τὰς πλαγίας θύρας, ἐνέγκαι τὰ προστεταγμένα αὐταῖς, καὶ οὐκ εἴδοσαν τοὺς πρεσβυτέρους ὅτι ἦσαν κεκρυμμένοι.
\par }{\PP \VS{19}Καὶ ἐγένετο ὡς ἐξήλθοσαν τὰ κοράσια, καὶ ἀνέστησαν οἱ δύο πρεσβύται, καὶ ἐπέδραμον αὐτῇ,
\VS{20}καὶ εἶπον, ἰδοὺ αἱ θύραι τοῦ παραδείσου κέκλεινται, καὶ οὐδεὶς θεωρεῖ ἡμᾶς, καὶ ἐν ἐπιθυμίᾳ σου ἐσμέν· διὸ συγκατάθου ἡμῖν, καὶ γενοῦ μεθʼ ἡμῶν.
\VS{21}Εἰ δὲ μὴ, καταμαρτυρήσομέν σου, ὅτι ἦν μετὰ σοῦ νεανίσκος, καὶ διὰ τοῦτο ἐξαπέστειλας τὰ κοράσια ἀπὸ σοῦ.
\par }{\PP \VS{22}Καὶ ἀνεστέναξε Σωσάννα, καὶ εἶπε, στενά μοι πάντοθεν· ἐάν τε γὰρ τοῦτο πράξω, θάνατός μοι ἐστίν· ἐάν τε μὴ πράξω, οὐκ ἐκφεύξομαι τὰς χεῖρας ὑμῶν.
\VS{23}Αἱρετώτερόν μοι ἐστὶ μὴ πράξασαν ἐμπεσεῖν εἰς τὰς χεῖρας ὑμῶν, ἢ ἁμαρτεῖν ἐνώπιον Κυρίου.
\VS{24}Καὶ ἀνεβόησε φωνῇ μεγάλῃ Σωσάννα· ἐβόησαν δὲ καὶ οἱ δύο πρεσβύται κατέναντι αὐτῆς.
\par }{\PP \VS{25}Καὶ δραμων ὁ εἷς, ἤνοιξε τὰς θύρας τοῦ παραδείσου.
\VS{26}Ὡς δὲ ἤκουσαν τὴν κραυγὴν ἐν τῷ παραδεισῳ οἱ ἐκ τῆς οἰκίας, εἰσεπήδησαν διὰ τῆς πλαγίας θύρας ἰδεῖν τὸ συμβεβηκὸς αὐτῇ.
\VS{27}Ἡνίκα δὲ εἶπα οἱ πρεσβύται τοὺς λόγους αὐτῶν, κατῃσχύνθησαν οἱ δοῦλοι σφόδρα, ὅτι πώποτε οὐκ ἐῤῥήθη λόγος τοιοῦτος περὶ Σωσάννης.
\par }{\PP \VS{28}Καὶ ἐγένετο τῇ ἐπαύριον, ὡς συνῆλθεν ὁ λαὸς πρὸς τὸν ἄνδρα αὐτῆς Ἰωακεὶμ, ἦλθον οἱ δύο πρεσβύται πλήρεις τῆς ἀνόμου ἐννοίας κατὰ Σωσάννης, τοῦ θανατῶσαι αὐτὴν,
\VS{29}καὶ εἶπαν ἔμπροσθεν τοῦ λαοῦ, ἀποστείλατε ἐπὶ Σωσάνναν θυγατέρα Χελκίου, ἥ ἐστι γυνὴ Ἰωακείμ· οἱ δὲ ἀπέστειλαν.
\VS{30}Καὶ ἦλθεν αὐτὴ, καὶ οἱ γονεῖς αὐτῆς, καὶ τὰ τέκνα αὐτῆς, καὶ πάντες οἱ συγγενεῖς αὐτῆς.
\par }{\PP \VS{31}Ἡ δὲ Σωσάννα ἦν τρυφερὰ σφόδρα, καὶ καλὴ τῷ εἴδει.
\VS{32}Οἱ δὲ παράνομοι ἐκέλευσαν ἀποκαλυφθῆναι αὐτὴν, ἦν γὰρ κατακεκαλυμμένη, ὅπως ἐμπλησθῶσι τοῦ κάλλους αὐτῆς.
\VS{33}Ἔκλαιον δὲ οἱ παρʼ αὐτῆς, καὶ πάντες οἱ ἰδόντες αὐτήν.
\par }{\PP \VS{34}Ἀναστάντες δὲ οἱ δύο πρεσβύται ἐν μέσῳ τῷ λαῷ, ἔθηκαν τὰς χεῖρας ἐπὶ τὴν κεφαλὴν αὐτῆς.
\VS{35}Ἡ δὲ κλαίουσα ἀνέβλεψεν εἰς τὸν οὐρανὸν, ὅτι ἦν ἡ καρδία αὐτῆς πεποιθυῖα ἐπὶ Κυρίῳ.
\par }{\PP \VS{36}Εἶπον δὲ οἱ πρεσβύται, περιπατούντων ἡμῶν ἐν τῷ παραδείσῳ μόνων, εἰσῆλθεν αὕτη μετὰ δύο παιδισκῶν, καὶ ἀπέκλεισε τὰς θύρας τοῦ παραδείσου, καὶ ἀπέλυσε τὰς παιδίσκας.
\VS{37}Καὶ ἦλθε πρὸς αὐτὴν νεανίσκος ὃς ἦν κεκρυμμένος, καὶ ἀνέπεσε μετʼ αὐτῆς.
\VS{38}Ἡμεῖς δὲ ὄντες ἐν τῇ γωνίᾳ τοῦ παραδείσου, ἰδόντες τὴν ἀνομίαν, ἐδράμομεν ἐπʼ αὐτούς.
\par }{\PP \VS{39}Καὶ ἰδόντες συγγινομένους αὐτοὺς, ἐκείνου μὲν οὐκ ἠδυνήθημεν ἐγκρατεῖς γενέσθαι, διὰ τὸ ἰσχύειν αὐτὸν ὑπὲρ ἡμᾶς, καὶ ἀνοίξαντα τὰς θύρας ἐκπεπηδηκέναι,
\VS{40}Ταύτης δὲ ἐπιλαβόμενοι, ἐπηρωτῶμεν, τίς ἦν ὁ νεανίσκος· καὶ οὐκ ἠθέλησεν ἀγγεῖλαι ἡμῖν· ταῦτα μαρτυροῦμεν.
\VS{41}Καὶ ἐπίστευσεν αὐτοῖς ἡ συναγωγὴ ὡς πρεσβυτέροις τοῦ λαοῦ καὶ κριταῖς· καὶ κατέκριναν αὐτὴν ἀποθανεῖν.
\par }{\PP \VS{42}Ἀνεβόησε δὲ φωνῇ μεγάλῃ Σωσάννα, καὶ εἶπεν, ὁ Θεὸς ὁ αἰώνιος, ὁ τῶν κρυπτῶν γνώστης, ὁ εἰδὼς τὰ πάντα πρὶν γενέσεως αὐτῶν,
\VS{43}σὺ ἐπίστασαι ὅτι ψευδῆ μου κατεμαρτύρησαν· καὶ ἰδοὺ ἀποθνήσκω μὴ ποιήσασα μηδὲν ὧν οὗτοι ἐπονηρεύσαντο κατʼ ἐμοῦ.
\VS{44}Καὶ εἰσήκουσε Κύριος τῆς φωνῆς αὐτῆς.
\par }{\PP \VS{45}Καὶ ἀπαγομένης αὐτῆς ἀπολέσθαι, ὁ Θεὸς ἐξήγειρε τὸ πνεῦμα τὸ ἅγιον παιδαρίου νεωτέρου ᾧ ὄνομα Δανιήλ.
\VS{46}Καὶ ἐβόησε φωνῇ μεγάλῃ, ἀθῶος ἐγὼ ἀπὸ τοῦ αἵματος ταύτης.
\par }{\PP \VS{47}Ἐπέστρεψε δὲ πᾶς ὁ λαὸς πρὸς αὐτὸν, καὶ εἶπαν, τίς ὁ λόγος οὗτος, ὃν σὺ λελάληκας;
\VS{48}Ὁ δὲ στὰς ἐν μέσῳ αὐτῶν, εἶπεν, οὕτως μωροὶ οἱ υἱοὶ Ἰσραήλ; οὐκ ἀνακρίναντες, οὐδὲ τὸ σαφὲς ἐπιγνόντες, κατεκρίνατε θυγατέρα Ἰσραήλ;
\VS{49}Ἀναστρέψατε εἰς τὸ κριτήριον, ψευδῆ γὰρ οὗτοι κατεμαρτύρησαν αὐτῆς.
\par }{\PP \VS{50}Καὶ ἀνέστρεψε πᾶς ὁ λαὸς μετὰ σπουδῆς· καὶ εἶπαν αὐτῷ οἱ πρεσβύτεροι, δεῦρο κάθισον ἐν μέσῳ ἡμῶν, καὶ ἀνάγγειλον ἡμῖν, ὅτι σοὶ δέδωκεν ὁ Θεὸς τὸ πρεσβεῖον.
\VS{51}Καὶ εἶπε πρὸς αὐτοὺς Δανιὴλ, διαχωρίσατε αὐτοὺς ἀπʼ ἀλλήλων μακρὰν, καὶ ἀνακρινῶ αὐτούς.
\par }{\PP \VS{52}Ὡς δὲ διεχωρίσθησαν εἷς ἀπὸ τοῦ ἑνὸς, ἐκάλεσε τὸν ἕνα αὐτῶν, καὶ εἶπε πρὸς αὐτὸν, πεπαλαιωμένε ἡμερῶν κακῶν, νῦν ἥκασιν αἱ ἁμαρτίαι σου, ἃς ἐποίεις τὸ πρότερον,
\VS{53}κρίνων κρίσεις ἀδίκους· καὶ τοὺς μὲν ἀθώους κατακρίνων, ἀπολύων δὲ τοὺς αἰτίους, λέγοντος τοῦ Θεοῦ, ἀθῶον καὶ δίκαιον οὐκ ἀποκτενεῖς.
\VS{54}Νῦν οὖν ταύτην εἴπερ εἶδες, εἰπὸν, ὑπὸ τί δένδρον εἶδες αὐτοὺς ὁμιλοῦντας ἀλλήλοις; ὁ δὲ εἶπεν, ὑπὸ σχῖνον.
\par }{\PP \VS{55}Εἶπε δὲ Δανιὴλ, ὀρθῶς ἔψευσαι εἰς τὴν σεαυτοῦ κεφαλήν· ἤδη γὰρ ἄγγελος φάσιν Θεοῦ λαβὼν παρὰ τοῦ Θεοῦ, σχίσει σε μέσον.
\VS{56}Καὶ μεταστήσας αὐτὸν, ἐκέλευσε προσαγαγεῖν τὸν ἕτερον, καὶ εἶπεν αὐτῷ, σπέρμα Χαναὰν, καὶ οὐκ Ἰούδα, τὸ κάλλος ἐξηπάτησέ σε, καὶ ἐπιθυμία διέστρεψε τὴν καρδίαν σου.
\VS{57}Οὕτως ἑποιεῖτε θυγατράσιν Ἰσραὴλ, καὶ ἐκεῖναι φοβούμεναι ὡμίλουν ὑμῖν· ἀλλʼ οὐ θυγάτηρ Ἰούδα ὑπέμεινε τὴν ἀνομίαν ὑμῶν.
\VS{58}Νῦν οὖν λέγε μοι, ὑπὸ τί δένδρον κατέλαβες αὐτοὺς ὁμιλοῦντας ἀλλήλοις; ὁ δὲ εἶπεν, ὑπὸ πρίνον.
\par }{\PP \VS{59}Εἶπε δὲ αὐτῷ Δανιὴλ, ὀρθῶς ἔψευσαι καὶ σὺ εἰς τὴν σεαυτοῦ κεφαλὴν· μένει γὰρ ὁ ἄγγελος τοῦ Θεοῦ, τὴν ῥομφαίαν ἔχων πρίσαι σε μέσον, ὅπως ἐξολοθρεύσῃ ὑμᾶς.
\par }{\PP \VS{60}Καὶ ἀνεβόησε πᾶσα ἡ συναγωγὴ φωνῇ μεγάλῃ, καὶ εὐλόγησαν τῷ Θεῷ τῷ σώζοντι τοὺς ἐλπίζοντας ἐπʼ αὐτόν.
\VS{61}Καὶ ἀνέστησαν ἐπὶ τοὺς δύο πρεσβύτας, ὅτι συνέστησεν αὐτοὺς Δανιὴλ ἐκ τοῦ στόματος αὐτῶν ψευδομαρτυρήσαντας.
\VS{62}Καὶ ἐποίησαν αὐτοῖς ὃν τρόπον ἐπονηρεύσαντο τῷ πλησίον· ποιῆσαι κατὰ τὸν νόμον Μωυσῆ· καὶ ἀπέκτειναν αὐτοὺς, καὶ ἐσώθη αἷμα ἀναίτιον ἐν τῇ ἡμέρᾳ ἐκείνῃ.
\par }{\PP \VS{63}Χελκίας δὲ καὶ ἡ γυνὴ αὐτοῦ ᾔνεσαν περὶ τῆς θυγατρὸς αὐτῶν μετὰ Ἰωακεὶμ τοῦ ἀνδρὸς αὐτῆς καὶ τῶν συγγενῶν αὐτῶν, ὅτι οὐχ εὑρέθη ἐν αὐτῇ ἄσχημον πρᾶγμα.
\VS{64}Καὶ Δανιὴλ ἐγένετο μέγας ἐνώπιον τοῦ λαοῦ ἀπὸ τῆς ἡμέρας ἐκείνης, καὶ ἐπέκεινα.
\par }