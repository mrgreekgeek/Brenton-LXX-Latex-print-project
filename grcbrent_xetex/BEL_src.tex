\NormalFont\ShortTitle{ΒΗΛ}
{\MT ΒΗΛ ΚΑΙ ΔΡΑΚΩΝ

\par }\OneChap {\PP \VerseOne{1}ΚΑΙ ὁ βασιλεὺς Ἀστυάγης προσετέθη πρὸς τοὺς πατέρας αὐτοῦ· καὶ παρέλαβε Κύρος ὁ Πέρσης τὴν βασιλείαν αὐτοῦ.
\VS{2}Καὶ ἦν Δανιὴλ συμβιωτὴς τοῦ βασιλέως, καὶ ἔνδοξος ὑπὲρ πάντας τοὺς φίλους αὐτοῦ.
\par }{\PP \VS{3}Καὶ ἦν εἴδωλον τοῖς Βαβυλωνίοις ᾧ ὄνομα Βὴλ, καὶ ἐδαπανῶντο εἰς αὐτὸν ἑκάστης ἡμέρας σεμιδάλεως ἀρτάβαι δώδεκα, καὶ πρόβατα τεσσαράκοντα, καὶ οἴνου μετρηταὶ ἕξ.
\VS{4}Καὶ ὁ βασιλεὺς ἐσέβετο αὐτὸν, καὶ ἐπορεύετο καθʼ ἑκάστην ἡμέραν προσκυνεῖν αὐτῷ. Δανιὴλ δὲ προσεκύνει τῷ Θεῷ αὐτοῦ· καὶ εἶπεν αὐτῷ ὁ βασιλεὺς, διατί οὐ προσκυνεῖς τῷ Βήλ;
\VS{5}Ὁ δὲ εἶπεν, ὅτι οὐ σέβομαι εἴδωλα χειροποίητα, ἀλλὰ τὸν ζῶντα Θεὸν, τὸν κτίσαντα τὸν οὐρανὸν καὶ τὴν γῆν καὶ ἔχοντα πάσης σαρκὸς κυρείαν.
\par }{\PP \VS{6}Καὶ εἶπεν αὐτῷ ὁ βασιλεὺς, οὐ δοκεῖ σοι Βὴλ εἶναι ζῶν θεός; ἢ οὐχ ὁρᾷς ὅσα ἐσθίει καὶ πίνει καθʼ ἑκάστην ἡμέραν;
\VS{7}Καὶ εἶπε Δανιὴλ γελάσας, μὴ πλανῶ, βασιλεῦ, οὗτος γὰρ ἔσωθεν μέν ἐστι πηλὸς, ἔξωθεν δὲ χαλκὸς, καὶ οὐ βέβρωκεν οὐδέποτε.
\par }{\PP \VS{8}Θυμωθεὶς δὲ ὁ βασιλεὺς ἐκάλεσε τοὺς ἱερεῖς αὐτοῦ· καὶ εἶπεν αὐτοῖς ἐὰν μὴ εἴποιτέ μοι τίς ὁ κατέσθων τὴν δαπάνην ταύτην, ἀποθανεῖσθε.
\VS{9}Ἐὰν δὲ δείξητε ὅτι Βὴλ κατεσθίει αὐτὰ, ὁ Δανιὴλ ἀποθανεῖται, ὅτι ἐβλασφήμησεν εἰς τὸν Βήλ· καὶ εἶπε Δανιὴλ τῷ βασιλεῖ, γινέσθω κατὰ τὸ ῥῆμά σου.
\par }{\PP \VS{10}Καὶ ἦσαν ἱερεῖς τοῦ Βὴλ ἑβδομήκοντα ἐκτὸς γυναικῶν καὶ τέκνων· καὶ ἦλθεν ὁ βασιλεὺς μετὰ Δανιὴλ εἰς τὸν οἶκον τοῦ Βήλ.
\VS{11}Καὶ εἶπαν οἱ ἱερεῖς τοῦ Βὴλ, ἰδοὺ ἡμεῖς ἀποτρέχομεν ἔξω, σὺ δὲ, βασιλεῦ, παράθες τὰ βρώματα, καὶ τὸν οἶνον κεράσας θὲς, καὶ ἀπόκλεισον τὴν θύραν, καὶ σφράγισον τῷ δακτυλίῳ σου.
\VS{12}Καὶ ἐλθὼν πρωῒ, ἐὰν μὴ εὕρῃς πάντα βεβρωμένα ὑπὸ τοῦ Βὴλ, ἀποθανούμεθα· ἢ Δανιὴλ ὁ ψευδόμενος καθʼ ἡμῶν.
\VS{13}Αὐτοὶ δὲ κατεφρόνουν, ὅτι πεποιήκεισαν ὑπὸ τὴν τράπεζαν κεκρυμμένην εἴσοδον, καὶ διʼ αὐτῆς εἰσεπορεύοντο διόλου, καὶ ἀνήλουν αὐτά.
\par }{\PP \VS{14}Καὶ ἐγένετο ὡς ἐξήλθοσαν ἐκεῖνοι, καὶ ὁ βασιλεὺς παρέθηκε τὰ βρώματα τῷ Βήλ· καὶ ἐπέταξε Δανιὴλ τοῖς παιδαρίοις αὐτοῦ, καὶ ἤνεγκαν τέφραν· καὶ κατέσεισαν ὅλον τὸν ναὸν ἐνώπιον τοῦ βασιλέως μόνου· καὶ ἐξελθόντες ἔκλεισαν τὴν θύραν, καὶ ἐσφραγίσαντο ἐν τῷ δακτυλίῳ τοῦ βασιλέως, καὶ ἀπῆλθον.
\VS{15}Οἱ δὲ ἱερεῖς ἦλθον τὴν νύκτα κατὰ τὸ ἔθος αὐτῶν, καὶ αἱ γυναῖκες αὐτῶν, καὶ τὰ τέκνα αὐτῶν, καὶ κατέφαγον πάντα, καὶ ἐξέπιον.
\par }{\PP \VS{16}Καὶ ὤρθρισεν ὁ βασιλεὺς τὸ πρωῒ, καὶ Δανιὴλ μετʼ αὐτοῦ.
\VS{17}Καὶ εἶπε, σῶοι αἱ σφραγίδες Δανιήλ; ὁ δὲ εἶπε, σῶοι, βασιλεῦ.
\VS{18}Καὶ ἐγένετο ἅμα τῷ ἀνοῖξαι τὰς θύρας, ἐπιβλέψας ἐπὶ τὴν τράπεζαν ὁ βασιλεὺς, ἐβόησε φωνῇ μεγάλῃ, μέγας εἶ Βὴλ, καὶ οὐκ ἔστι παρὰ σοὶ δόλος οὐδὲ εἷς.
\par }{\PP \VS{19}Καὶ ἐγέλασε Δανιὴλ, καὶ ἐκράτησε τὸν βασιλέα, τοῦ μὴ εἰσελθεῖν αὐτὸν ἔσω· καὶ εἶπεν, ἴδε δὴ τὸ ἔδαφος, καὶ γνῶθι τίνος τὰ ἴχνη ταῦτα.
\VS{20}Καὶ εἶπεν ὁ βασιλεὺς, ὁρῶ τὰ ἴχνη ἀνδρῶν, καὶ γυναικῶν, καὶ παιδίων· καὶ ὀργισθεὶς ὁ βασιλεὺς τότε συνέλαβε τοὺς ἱερεῖς, καὶ τὰς γυναῖκας,
\VS{21}καὶ τὰ τέκνα αὐτῶν, καὶ ἔδειξαν αὐτῷ τὰς κρυπτὰς θύρας, διʼ ὧν εἰσεπορεύοντο, καὶ ἐδαπάνων τὰ ἐπὶ τῆς τραπέζης.
\VS{22}Καὶ ἀπέκτεινεν αὐτοὺς ὁ βασιλεὺς, καὶ ἔδωκε τὸν Βὴλ ἔκδοτον τῷ Δανιήλ· καὶ κατέστρεψεν αὐτὸν καὶ τὸ ἱερὸν αὐτοῦ.
\par }{\PP \VS{23}Καὶ ἦν Δράκων μέγας, καὶ ἐσέβοντο αὐτὸν οἱ Βαβυλώνιοι.
\VS{24}Καὶ εἶπεν ὁ βασιλεὺς τῷ Δανιὴλ, μὴ καὶ τοῦτον ἐρεῖς ὅτι χαλκοῦς ἐστιν; ἰδοὺ ζῇ, καὶ ἐσθίει, καὶ πίνει· οὐ δύνασαι εἰπεῖν, ὅτι οὐκ ἔστιν οὗτος θεὸς ζῶν· καὶ προσκύνησον αὐτῷ.
\par }{\PP \VS{25}Καὶ εἶπε Δανιὴλ, Κυρίῳ τῷ Θεῷ μου προσκυνήσω, ὅτι οὗτός ἐστι Θεὸς ζῶν.
\VS{26}Σὺ δὲ, βασιλεῦ, δός μοι ἐξουσίαν, καὶ ἀποκτενῶ τὸν δράκοντα ἄνευ μαχαίρας καὶ ῥάβδου· καὶ εἶπεν ὁ βασιλεὺς δίδωμί σοι.
\VS{27}Καὶ ἔλαβεν ὁ Δανιὴλ πίσσαν καὶ στέαρ καὶ τρίχας, καὶ ἥψησεν ἐπιτοαυτό· καὶ ἐποίησε μάζας, καὶ ἔδωκεν εἰς τὸ στόμα τοῦ δράκοντος, καὶ φαγὼν διεῤῥάγη ὁ δράκων· καὶ εἶπεν, ἴδετε τὰ σεβάσματα ὑμῶν.
\par }{\PP \VS{28}Καὶ ἐγένετο, ὡς ἤκουσαν οἱ Βαβυλώνιοι, ἠγανάκτησαν λίαν, καὶ συνεστράφησαν ἐπὶ τὸν βασιλέα, καὶ εἶπαν, Ἰουδαῖος γέγονεν ὁ βασιλεὺς, τὸν Βὴλ κατέσπασε, καὶ τὸν δράκοντα ἀπέκτεινε, καὶ τοὺς ἱερεῖς κατέσφαξε.
\VS{29}Καὶ εἶπαν ἐλθόντες πρὸς τὸν βασιλέα, παράδος ἡμῖν τὸν Δανιήλ· εἰ δὲ μὴ, ἁποκτενοῦμέν σε, καὶ τὸν οἶκόν σου.
\par }{\PP \VS{30}Καὶ εἶδεν ὁ βασιλεὺς ὅτι ἐπείγουσιν αὐτὸν σφόδρα, καὶ ἀναγκασθεὶς ὁ βασιλεὺς παρέδωκεν αὐτοῖς τὸν Δανιήλ.
\VS{31}Οἱ δὲ ἔβαλον αὐτὸν εἰς τὸν λάκκον τῶν λεόντων, καὶ ἦν ἐκεῖ ἡμέρας ἕξ.
\VS{32}Ἦσαν δὲ ἐν τῷ λάκκῳ ἑπτὰ λέοντες, καὶ ἐδίδοτο αὐτοῖς τὴν ἡμέραν δύο σώματα καὶ δύο πρόβατα· τότε δὲ οὐκ ἐδόθη αὐτοῖς, ἵνα καταφάγωσι τὸν Δανιήλ.
\par }{\PP \VS{33}Καὶ ἦν Ἀμβακοὺμ ὁ προφήτης ἐν τῇ Ἰουδαίᾳ, καὶ αὐτὸς ἥψησεν ἕψεμα, καὶ ἐνέθρυψεν ἄρτους εἰς σκάφην, καὶ ἑπορεύετο εἰς τὸ πεδίον ἀπενέγκαι τοῖς θερισταῖς.
\VS{34}Καὶ εἶπεν ὁ ἄγγελος Κυρίου τῷ Ἀμβακοὺμ, ἀπένεγκε τὸ ἄριστον ὃ ἔχεις εἰς Βαβυλῶνα τῷ Δανιὴλ εἰς τὸν λάκκον τῶν λεόντων.
\par }{\PP \VS{35}Καὶ εἶπεν Ἀμβακοὺμ, Κύριε, Βαβυλῶνα οὐχ ἑώρακα, καὶ τὸν λάκκον οὐ γινώσκω.
\VS{36}Καὶ ἐπελάβετο ὁ ἄγγελος Κυρίου τῆς κορυφῆς αὐτοῦ, καὶ βαστάσας τῆς κόμης τῆς κεφαλῆς αὐτοῦ, ἔθηκεν αὐτὸν εἰς Βαβυῶνα ἐπάνω τοῦ λάκκου, ἐν τῷ ῥοίζῳ τοῦ πνεύματος αὐτοῦ.
\VS{37}Καὶ ἐβόησεν Ἀμβακοὺμ, λέγων, Δανιὴλ, Δανιὴλ, λάβε τὸ ἄριστον ὃ ἀπέστειλέ σοι ὁ Θεός.
\par }{\PP \VS{38}Καὶ εἶπε Δανιὴλ, ἐμνήσθης γάρ μου ὁ Θεὸς, καὶ οὐκ ἐγκατέλιπες τοὺς ἀγαπῶντάς σε.
\VS{39}Καὶ ἀναστὰς δανιὴλ, ἔφαγεν· ὁ δὲ ἄγγελος τοῦ ἀπεκατέστησε τὸν Ἀμβακοὺμ παραχρῆμα εἰς τὸν τόπον αὐτοῦ.
\par }{\PP \VS{40}Ὁ δὲ βασιλεὺς ἦλθε τῇ ἑβδόμῃ πενθῆσαι τὸν Δανιὴλ, καὶ ἦλθεν ἐπὶ τὸν λάκκον, καὶ ἐνέβλεψε, καὶ ἰδοὺ, Δανιὴλ καθήμενος.
\VS{41}Καὶ ἀναβοήσας φωνῇ μεγάλῃ, εἶπε, μέγας εἰ, Κύριε ὁ Θεὸς τοῦ Δανιὴλ, καὶ οὐκ ἔστιν ἄλλος πλὴν σοῦ.
\VS{42}Καὶ ἀνέσπασεν αὐτόν· τοὺς δὲ αἰτίους τῆς ἀπωλείας αὐτοῦ ἐνέβαλεν εἰς τὸν λάκκον· καὶ κατεβρώθησαν παραχρῆμα ἐνώπιον αὐτοῦ.
\par }