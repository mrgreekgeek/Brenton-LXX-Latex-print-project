\NormalFont\ShortTitle{ΒΑΣΙΛΕΙΩΝ Γ}
{\MT ΒΑΣΙΛΕΙΩΝ Γ

\par }\ChapOne{1}{\PP \VerseOne{1}ΚΑΙ ὁ βασιλεὺς Δαυὶδ πρεσβύτερος προβεβηκὼς ἡμέραις, καὶ περιέβαλλον αὐτὸν ἱματίοις, καὶ οὐκ ἐθερμαίνετο.
\VS{2}Καὶ εἶπον οἱ παῖδες αὐτοῦ, ζητησάτωσαν τῷ βασιλεῖ παρθένον νεάνιδα, καὶ παραστήσεται τῷ βασιλεῖ, καὶ ἔσται αὐτὸν θάλπουσα, καὶ κοιμηθήσεται μετʼ αὐτοῦ, καὶ θερμανθήσεται ὁ κύριός μου ὁ βασιλεύς.
\VS{3}Καὶ ἐζήτησαν νεάνιδα καλὴν ἐκ παντὸς ὁρίου Ἰσραήλ· καὶ εὗρον τὴν Ἀβισὰγ τὴν Σωμανίτιν, καὶ ἤνεγκαν αὐτὴν πρὸς τὸν βασιλέα.
\VS{4}Καὶ ἡ νεᾶνις καλὴ ἕως σφόδρα· καὶ ἦν θάλπουσα τὸν βασιλέα, καὶ ἐλειτούργει αὐτῷ· καὶ ὁ βασιλεὺς οὐκ ἔγνω αὐτήν.
\par }{\PP \VS{5}Καὶ Ἀδωνίας υἱὸς Ἀγγὶθ ἐπῄρετο, λέγων, ἐγὼ βασιλεύσω· καὶ ἐποίησεν ἑαυτῷ ἅρματα καὶ ἱππεῖς, καὶ πεντήκοντα ἄνδρας παρατρέχειν ἔμπροσθεν αὐτοῦ.
\VS{6}Καὶ οὐκ ἀπεκώλυσεν αὐτὸν ὁ πατὴρ αὐτοῦ οὐδέποτε, λέγων, διατί σὺ ἐποίησας; καὶ γε αὐτὸς ὡραῖος τῇ ὄψει σφόδρα, καὶ αὐτὸν ἔτεκεν ὀπίσω Ἀβεσσαλώμ.
\VS{7}Καὶ ἐγένοντο οἱ λόγοι αὐτοῦ μετὰ Ἰωὰβ τοῦ υἱοῦ Σαρουίας, καὶ μετὰ Ἀβιάθαρ τοῦ ἱερέως, καὶ ἐβοήθουν ὀπίσω Ἀδωνίου.
\VS{8}Καὶ Σαδὼκ ὁ ἱερεὺς, καὶ Βαναίας υἱὸς Ἰωδαὲ, καὶ Νάθαν ὁ προφήτης, καὶ Σεμεῒ, καὶ Ῥησὶ, καὶ υἱοὶ δυνατοὶ τοῦ Δαυὶδ, οὐκ ἦσαν ὀπίσω Ἀδωνίου.
\VS{9}Καὶ ἐθυσίασεν Ἀδωνίας πρόβατα καὶ μόσχους καὶ ἄρνας μετὰ αἰθῆ τοῦ Ζωελεθὶ, ὃς ἦν ἐχόμενα τῆς Ῥωγήλ· καὶ ἐκάλεσε πάντας τοὺς ἀδελφοὺς αὐτοῦ, καὶ πάντας τοὺς ἁδροὺς Ἰούδα παῖδας τοῦ βασιλέως.
\VS{10}Καὶ Νάθαν τὸν προφήτην, καὶ Βαναίαν, καὶ τοὺς δυνατοὺς, καὶ τὸν Σαλωμὼν ἀδελφὸν αὐτοῦ, οὐκ ἐκάλεσε.
\par }{\PP \VS{11}Καὶ εἶπε Νάθαν πρὸς Βηρσαβεὲ μητέρα Σαλωμὼν, λέγων, οὐκ ἤκουσας ὅτι ἐβασίλευσεν Ἀδωνίας υἱὸς Ἀγγὶθ, καὶ ὁ κύριος ἡμῶν Δαυὶδ οὐκ ἔγνω;
\VS{12}Καὶ νῦν δεῦρο, συμβουλεύσω σοι δὴ συμβουλίαν, καὶ ἐξελοῦ τὴν ψυχήν σου, καὶ τὴν ψυχὴν τοῦ υἱοῦ σου Σαλωμών.
\VS{13}Δεῦρο εἴσελθε πρὸς τὸν βασιλέα Δαυὶδ, καὶ ἐρεῖς πρὸς αὐτὸν, λέγουσα, οὐχὶ σὺ κύριέ μου βασιλεῦ ὤμοσας τῇ δούλῃ σου, λέγων, ὅτι ὁ υἱός σου Σαλωμὼν βασιλεύσει μετʼ ἐμέ, καὶ αὐτὸς καθιεῖται ἐπὶ τοῦ θρόνου μου; καὶ τί ὅτι ἐβασίλευσεν Ἀδωνίας;
\VS{14}Καὶ ἰδοὺ ἔτι λαλούσης σου ἐκεῖ μετα τοῦ βασιλέως, καὶ ἐγὼ εἰσελεύσομαι ὀπίσω σου, καὶ πληρώσω τοὺς λόγους σου.
\par }{\PP \VS{15}Καὶ εἴσῆλθε Βηρσαβεὲ πρὸς τὸν βασιλέα εἰς τὸ ταμεῖον· καὶ ὁ βασιλεὺς πρεσβύτης σφόδρα· καὶ Ἀβισὰλ ἡ Σωμανῖτις ἦν λειτουργοῦσα τῷ βασιλεῖ.
\VS{16}Καὶ ἔκυψε Βηρσαβεὲ, καὶ προσεκύνησε τῷ βασιλεῖ· καὶ εἶπεν ὁ βασιλεύς, τί ἔστι σοί;
\VS{17}Ἡ δὲ εἶπε, κύριε, σὺ ὤμοσας ἐν Κυρίῳ τῷ Θεῷ σου τῇ δούλῃ σου, λέγων, ὅτι ὁ υἱός σου Σαλωμὼν βασιλεύσει μετʼ ἐμέ, καὶ καθήσεται ἐπὶ τοῦ θρόνου μου.
\VS{18}Καὶ νῦν ἰδοὺ Ἀδωνίας ἐβασίλευσε, καὶ σὺ κύριέ μου βασιλεῦ οὐκ ἔγνως.
\VS{19}Καὶ ἐθυσίασε μόσχους καὶ ἄρνας καὶ πρόβατα εἰς πλῆθος, καὶ ἐκάλεσε πάντας τοὺς υἱοὺς τοῦ βασιλέως, καὶ Ἀβιάθαρ τὸν ἱερέα, καὶ Ἰωὰβ τὸν ἄρχοντα τῆς δυνάμεως· καὶ τὸν Σαλωμὼν τὸν δοῦλόν σου οὐκ ἐκάλεσε.
\VS{20}Καὶ σὺ κύριέ μου βασιλεῦ, οἱ ὀφθαλμοὶ παντὸς Ἰσραὴλ πρὸς σέ, ἀπάγγειλαι αὐτοῖς τίς καθήσεται ἐπὶ τοῦ θρόνου τοῦ κυρίου μου τοῦ βασιλέως μετʼ αὐτόν.
\VS{21}Καὶ ἔσται ὡς ἂν κοιμηθῇ ὁ κύριός μου ὁ βασιλεὺς μετὰ τῶν πατέρων αὐτοῦ, καὶ ἔσομαι ἐγὼ καὶ Σαλωμὼν ὁ υἱός μου ἁμαρτωλοί.
\par }{\PP \VS{22}Καὶ ἰδοὺ ἔτι αὐτῆς λαλούσης μετὰ τοῦ βασιλέως, καὶ Νάθαν ὁ προφήτης ἦλθε.
\VS{23}Καὶ ἀνηγγέλη τῷ βασιλεῖ, ἰδοὺ Νάθαν ὁ προφήτης· καὶ εἰσῆλθε κατὰ πρόσωπον τοῦ βασιλέως, καὶ προσεκύνησε τῷ βασιλεῖ κατὰ πρόσωπον αὐτοῦ ἐπὶ τὴν γῆν.
\VS{24}Καὶ εἶπε Νὰθαν, κύριέ μου βασιλεῦ, σὺ εἶπας, Ἀδωνίας βασιλεύσει ὀπίσω μου, καὶ αὐτὸς καθήσεται ἐπὶ τοῦ θρόνου μου;
\VS{25}Ὅτι κατέβη σήμερον, καὶ ἐθυσίασε μόσχους καὶ ἄρνας καὶ πρόβατα εἰς πλῆθος, καὶ ἐκάλεσε πάντας τοὺς υἱοὺς τοῦ βασιλέως, καὶ τοὺς ἄρχοντας τῆς δυνάμεως, καὶ Ἀβιάθαρ τὸν ἱερέα· καὶ ἰδού εἰσιν ἐσθίοντες καὶ πίνοντες ἐνώπιον αὐτοῦ, καὶ εἶπαν, ζήτω ὁ βασιλεὺς Ἀδωνίας.
\VS{26}Καὶ ἐμὲ αὐτὸν τὸν δοῦλόν σου, καὶ Σαδὼκ τὸν ἱερέα, καὶ Βαναίαν υἱὸν Ἰωδαὲ, καὶ Σαλωμὼν τὸν δοῦλόν σου, οὐκ ἐκάλεσεν.
\VS{27}Εἰ διὰ τοῦ κυρίου μου τοῦ βασιλέως γέγονε τὸ ῥῆμα τοῦτο, καὶ οὐκ ἐγνώρισας τῷ δούλῳ σου τίς καθήσεται ἐπὶ τὸν θρόνον τοῦ κυρίου μου βασιλέως μετʼ αὐτόν;
\par }{\PP \VS{28}Καὶ ἀπεκρίθη ὁ βασιλεὺς Δαυὶδ, καὶ εἶπε, καλέσατέ μοι τὴν Βηρσαβεέ· καὶ εἰσῆλθεν ἐνώπιον τοῦ βασιλέως, καὶ ἔστη ἐνώπιον αὐτοῦ.
\VS{29}Καὶ ὤμοσεν ὁ βασιλεὺς, καὶ εἶπε, ζῇ Κύριος ὃς ἐλυτρώσατο τὴν ψυχήν μου ἐκ πάσης θλίψεως,
\VS{30}ὅτι καθὼς ὤμοσά σοι ἐν Κυρίῳ Θεῷ Ἰσραὴλ, λέγων, ὅτι Σαλωμὼν ὁ υἱός σου βασιλεύσει μετʼ ἐμὲ, καὶ αὐτὸς καθήσεται ἐπὶ τοῦ θρόνου μου ἀντʼ ἐμοῦ, ὅτι οὕτω ποιήσω τῇ ἡμέρᾳ ταύτῃ.
\VS{31}Καὶ ἔκυψε Βηρσαβεὲ ἐπὶ πρόσωπον ἐπὶ τὴν γῆν, καὶ προσεκύνησε τῷ βασιλεῖ, καὶ εἶπε, ζήτω ὁ κύριός μου ὁ βασιλεὺς Δαυὶδ εἰς τὸν αἰῶνα.
\par }{\PP \VS{32}Καὶ εἶπεν ὁ βασιλεὺς Δαυίδ, Καλέσατέ μοι Σαδὼκ τὸν ἱερέα, καὶ Νάθαν τὸν προφήτην, καὶ Βαναίαν υἱὸν Ἰωδαέ καὶ εἰσῆλθον ἐνώπιον τοῦ βασιλέως.
\VS{33}Καὶ εἶπεν ὁ βασιλεὺς αὐτοῖς, λάβετε τοὺς δούλους τοῦ κυρίου ὑμῶν μεθʼ ὑμῶν, καὶ ἐπιβιβάσατε τὸν υἱόν μου Σαλωμὼν ἐπὶ τὴν ἡμίονον τὴν ἐμὴν, καὶ καταγάγετε αὐτὸν εἰς τὴν Γιὼν,
\VS{34}καὶ χρισάτω αὐτὸν ἐκεῖ Σαδὼκ ὁ ἱερεὺς καὶ Νάθαν ὁ προφήτης εἰς βασιλέα ἐπὶ Ἰσραὴλ, καὶ σαλπίσατε κερατίνῃ, καὶ ἐρεῖτε, ζήτω ὁ βασιλεὺς Σαλωμών.
\VS{35}Καὶ καθήσεται ἐπὶ τοῦ θρόνου μου, καὶ βασιλεύσει ἀντʼ ἐμοῦ· καὶ ἐγὼ ἐνετειλάμην τοῦ εἶναι εἰς ἡγούμενον ἐπὶ Ἰσραὴλ καὶ Ἰούδαν.
\VS{36}Καὶ ἀπεκρίθη Βαναίας υἱὸς Ἰωδαὲ τῷ βασιλεῖ, καὶ εἶπε, γένοιτο οὕτως· πιστώσαι Κύριος ὁ Θεὸς τοῦ κυρίου μου τοῦ βασιλέως·
\VS{37}καθὼς ἦν Κύριος μετὰ τοῦ κυρίου μου τοῦ βασιλέως, οὕτως εἴη μετὰ Σαλωμὼν, καὶ μεγαλύναι τὸν θρόνον αὐτοῦ ὑπὲρ τὸν θρόνον τοῦ κυρίου μου τοῦ βασιλέως Δαυίδ.
\par }{\PP \VS{38}Καὶ κατέβη Σαδὼκ ὁ ἱερεὺς, καὶ Νάθαν ὁ προφήτης, καὶ Βαναίας υἱὸς Ἰωδαὲ, καὶ ὁ Χερεθὶ, καὶ ὁ Φελεθὶ, καὶ ἐπεκάθισαν τὸν Σαλωμὼν ἐπὶ τὴν ἡμίονον τοῦ βασιλέως Δαυὶδ, καὶ ἀπήγαγον αὐτὸν εἰς τὴν Γιών.
\VS{39}Καὶ ἔλαβε Σαδὼκ ὁ ἱερεὺς τὸ κέρας τοῦ ἐλαίου ἐκ τῆς σκηνῆς, καὶ ἔχρισε τὸν Σαλωμὼν, καὶ ἐσάλπισε τῇ κερατίνῃ· καὶ εἶπε πᾶς ὁ λαός, ζήτω ὁ βασιλεὺς Σαλωμών.
\VS{40}Καὶ ἀνέβη πᾶς ὁ λαὸς ὀπίσω αὐτοῦ, καὶ ἐχόρευον ἐν χοροῖς καὶ εὐφραινόμενοι εὐφροσύνην μεγάλην, καὶ ἐῤῥάγη ἡ γῆ ἐν τῇ φωνῇ αὐτῶν.
\par }{\PP \VS{41}Καὶ ἤκουσεν Ἀδωνίας καὶ πάντες οἱ κλητοὶ αὐτοῦ, καὶ αὐτοὶ συνετέλεσαν φαγεῖν· καὶ ἤκουσεν Ἰωὰβ τὴν φωνὴν τῆς κερατίνης, καὶ εἶπε, τίς ἡ φωνὴ τῆς πόλεως ἠχούσης;
\VS{42}Ἔτι αὐτοῦ λαλοῦντος, καὶ ἰδοὺ Ἰωνάθαν υἱὸς Ἀβιάθαρ τοῦ ἱερέως εἰσῆλθε· καὶ εἶπεν Ἀδωνίας, εἴσελθε, ὅτι ἀνὴρ δυνάμεως εἶ σὺ, καὶ ἀγαθὰ εὐαγγέλισαι.
\VS{43}Καὶ ἀπεκρίθη Ἰωνάθαν, καὶ εἶπε, καὶ μάλα ὁ κύριος ἡμῶν ὁ βασιλεὺς Δαυὶδ ἐβασίλευσε τὸν Σαλωμὼν,
\VS{44}καὶ ἀπέστειλε μετʼ αὐτοῦ ὁ βασιλεὺς τὸν Σαδὼκ τὸν ἱερέα, καὶ Νάθαν τὸν προφήτην, καὶ Βαναίαν τὸν υἱὸν Ἰωδαὲ, καὶ τὸν Χερεθὶ, καὶ τὸν Φελεθὶ, καὶ ἐπεκάθισαν αὐτὸν ἐπὶ τὴν ἡμίονον τοῦ βασιλέως·
\VS{45}Καὶ ἔχρισαν αὐτὸν Σαδὼκ ὁ ἱερεὺς καὶ Νάθαν ὁ προφήτης ἐν τῇ Γιὼν, καὶ ἀνέβησαν ἐκεῖθεν εὐφραινόμενοι, καὶ ἤχησεν ἡ πόλις· αὕτη ἡ φωνὴ ἣν ἠκούσατε.
\VS{46}Καὶ ἐκάθισε Σαλωμὼν ἐπὶ θρόνον βασιλείας.
\VS{47}Καὶ εἰσῆλθον οἱ δοῦλοι τοῦ βασιλέως εὐλογῆσαι τὸν κύριον ἡμῶν τὸν βασιλέα Δαυὶδ, λέγοντες, ἀγαθύναι ὁ Θεὸς τὸ ὄνομα Σαλωμὼν ὑπὲρ τὸ ὄνομά σου, καὶ μεγαλύναι τὸν θρόνον αὐτοῦ ὑπὲρ τὸν θρόνον σου· καὶ προσεκύνησεν ὁ βασιλεὺς ἐπὶ τὴν κοίτην.
\VS{48}Καί γε οὕτως εἶπεν ὁ βασιλεύς, εὐλογητὸς Κύριος ὁ Θεὸς Ἰσραὴλ, ὃς ἔδωκε σήμερον ἐκ τοῦ σπέρματός μου καθήμενον ἐπὶ τοῦ θρόνου μου, καὶ οἱ ὀφθαλμοί μου βλέπουσι.
\par }{\PP \VS{49}Καὶ ἐξέστησαν πάντες οἱ κλητοὶ τοῦ Ἀδωνίου, καὶ ἦλθον ἀνὴρ εἰς τὴν ὁδὸν αὐτοῦ.
\VS{50}Καὶ Ἀδωνίας ἐφοβήθη ἀπὸ προσώπου Σαλωμὼν, καὶ ἀνέστη καὶ ἀπῆλθε καὶ ἐπελάβετο τῶν κεράτων τοῦ θυσιαστηρίου.
\VS{51}Καὶ ἀνηγγέλη τῷ Σαλωμὼν, λέγοντες, ἰδοὺ Ἀδωνίας ἐφοβήθη τὸν βασιλέα Σαλωμὼν, καὶ κατέχει τῶν κεράτων τοῦ θυσιαστηρίου, λέγων, ὀμοσάτω μοι σήμερον Σαλωμὼν, εἰ οὐ θανατώσει τὸν δοῦλον αὐτοῦ ἐν ῥομφαίᾳ.
\VS{52}Καὶ εἶπε Σαλωμών, ἐὰν γένηται εἰς υἱὸν δυνάμεως, εἰ πεσεῖται τῶν τριχῶν αὐτοῦ ἐπὶ τὴν γῆν· καὶ ἐὰν κακία εὑρεθῇ ἐν αὐτῷ, θανατωθήσεται.
\VS{53}Καὶ ἀπέστειλεν ὁ βασιλεὺς Σαλωμὼν, καὶ κατήνεγκαν αὐτὸν ἀπάνωθεν τοῦ θυσιαστηρίου· καὶ εἰσῆλθε, καὶ προσεκύνησε τῷ βασιλεῖ Σαλωμών· καὶ εἶπεν αὐτῷ Σαλωμὼν, δεῦρο εἰς τὸν οἶκόν σου.

\par }\Chap{2}{\PP \VerseOne{1}Καὶ ἤγγισαν αἱ ἡμέραι Δαυὶδ ἀποθανεῖν αὐτὸν, καὶ ἀπεκρίνατο Σαλωμὼν υἱῷ αὐτοῦ, λέγων,
\VS{2}ἐγώ εἰμι πορεύομαι ἐν ὁδῷ πάσης τῆς γῆς· καὶ ἰσχύσεις, καὶ ἔσῃ εἰς ἄνδρα,
\VS{3}καὶ φυλάξεις φυλακὴν Κυρίου Θεοῦ σου τοῦ πορεύεσθαι ἐν ταῖς ὁδοῖς αὐτοῦ, φυλάσσειν τὰς ἐντολὰς αὐτοῦ καὶ τὰ δικαιώματα καὶ τὰ κρίματα τὰ γεγραμμένα ἐν τῷ νόμῳ Μωυσέως· ἵνα συνήσῃς ἃ ποιήσεις κατὰ πάντα ὅσα ἂν ἐντείλωμαί σοι·
\VS{4}Ἵνα στήσῃ Κύριος τὸν λόγον αὐτοῦ ὃν ἐλάλησε, λέγων, ἐὰν φυλάξωσιν οἱ υἱοί σου τὴν ὁδὸν αὐτῶν πορεύεσθαι ἐνώπιόν μου ἐν ἀληθείᾳ, ἐν ὅλῃ καρδίᾳ αὐτῶν, λέγων, οὐκ ἐξολοθρευθήσεταί σοι ἀνὴρ ἐπάνωθεν θρόνου Ἰσραήλ.
\VS{5}Καί γε σὺ ἔγνως ὅσα ἐποίησέ μοι Ἰωὰβ υἱὸς Σαρουίας, ὅσα ἐποίησε τοῖς δυσὶν ἄρχουσι τῶν δυνάμεων Ἰσραὴλ, τῷ Ἀβεννὴρ υἱῷ Νὴρ, καὶ τῷ Ἀμεσσαῒ υἱῷ Ἰεθὲρ, καὶ ἀπέκτεινεν αὐτοὺς, καὶ ἔταξε τὰ αἵματα πολέμου ἐν εἰρήνῃ, καὶ ἔδωκεν αἷμα ἀθῶον ἐν τῇ ζώνῃ αὐτοῦ τῇ ἐν τῇ ὀσφύϊ αὐτοῦ, καὶ ἐν τῷ ὑποδήματι αὐτοῦ τῷ ἐν τῷ ποδὶ αὐτοῦ.
\VS{6}Καὶ ποιήσεις κατὰ τὴν σοφίαν σου, καὶ σὺ κατάξεις τὴν πολιὰν αὐτοῦ ἐν εἰρήνῃ εἰς ᾅδου.
\VS{7}Καὶ τοῖς υἱοῖς Βερζελλὶ τοῦ Γαλααδίτου ποιήσεις ἔλεος, καὶ ἔσονται ἐν τοῖς ἐσθίουσιν τὴν τράπεζάν σου· ὅτι οὕτως ἤγγισάν μοι ἐν τῷ με ἀποδιδράσκειν ἀπὸ προσώπου Ἀβεσσαλὼμ τοῦ ἀδελφοῦ σου.
\VS{8}Καὶ ἰδοὺ μετὰ σοῦ Σεμεῒ υἱὸς Γηρὰ υἱὸς τοῦ Ἰεμινὶ ἐκ Βαουρὶμ, καὶ αὐτὸς κατηράσατό με κατάραν ὀδυνηρὰν τῇ ἡμέρᾳ ᾗ ἐπορευόμην εἰς παρεμβολάς· καὶ αὐτὸς κατέβη εἰς ἀπαντήν μου εἰς τὸν Ἰορδάνην, καὶ ὤμοσα αὐτῷ ἐν Κυρίῳ, λέγων, εἰ θανατώσω σε ἐν ῥομφαίᾳ.
\VS{9}Καὶ οὐ μὴ ἀθωώσῃς αὐτὸν, ὅτι ἀνὴρ σοφὸς εἶ σύ, καὶ γνώσῃ ἃ ποιήσεις αὐτῷ, καὶ κατάξεις τὴν πολιὰν αὐτοῦ ἐν αἵματι εἰς ᾅδου.
\par }{\PP \VS{10}Καὶ ἐκοιμήθη Δαυὶδ μετὰ τῶν πατέρων αὐτοῦ, καὶ ἐτάφη ἐν πόλει Δαυίδ.
\VS{11}Καὶ αἱ ἡμέραι ἃς ἐβασίλευσε Δαυὶδ ἐπὶ τὸν Ἰσραὴλ, τεσσαράκοντα ἔτη· ἐν Χεβρὼν ἐβασίλευσεν ἑπτά ἔτη, καὶ ἐν Ἱερουσαλὴμ τριάκοντα τρία ἔτη.
\par }{\PP \VS{12}Καὶ Σαλωμὼν ἐκάθισεν ἐπὶ θρόνου Δαυὶδ τοῦ πατρὸς αὐτοῦ, καὶ ἡτοιμάσθη ἡ βασιλεία αὐτοῦ σφόδρα.
\VS{13}Καὶ εἰσῆλθεν Ἀδωνίας υἱὸς Ἀγγὶθ πρὸς Βηρσαβεὲ μητέρα Σαλωμὼν, καὶ προσεκύνησεν αὐτῇ· ἡ δὲ εἶπεν, εἰρήνη ἡ εἴσοδός σου; καὶ εἶπεν, εἰρήνη·
\VS{14}λόγος μοι πρὸς σέ. Καὶ εἶπεν αὐτῷ, λάλησον.
\VS{15}Καὶ εἶπεν αὐτῇ, σὺ οἶδας, ὅτι ἐμοὶ ἦν βασιλεία, καὶ ἐπʼ ἐμὲ ἔθετο πᾶς Ἰσραὴλ τὸ πρόσωπον αὐτοῦ εἰς βασιλέα· καὶ ἐστράφη ἡ βασιλεία, καὶ ἐγένετο τῷ ἀδελφῷ μου, ὅτι παρὰ Κυρίου ἐγενήθη αὐτῷ.
\VS{16}Καὶ νῦν αἴτησιν μίαν ἐγὼ αἰτοῦμαι παρὰ σοῦ, μὴ ἀποστρέψῃς τὸ πρόσωπόν σου· καὶ εἶπεν αὐτῷ Βηρσαβεὲ, λάλει.
\VS{17}Καὶ εἶπεν αὐτῇ, εἰπον δὴ πρὸς Σαλωμὼν τὸν βασιλέα, ὅτι οὐκ ἀποστρέψει τὸ πρόσωπον αὐτοῦ ἀπὸ σοῦ, καὶ δώσει μοι τὴν Ἀβισὰγ τὴν Σωμανίτιν εἰς γυναῖκα.
\VS{18}Καὶ εἶπε Βηρσαβεὲ, καλῶς· ἐγὼ λαλήσω περὶ σοῦ τῷ βασιλεῖ.
\par }{\PP \VS{19}Καὶ εἰσῆλθε Βηρσαβεὲ πρὸς τὸν βασιλέα Σαλωμὼν αλῆσαι αὐτῷ περὶ Ἀδωνίου· καὶ ἐξανέστη ὁ βασιλεὺς εἰς ἀπαντὴν αὐτῇ, καὶ κατεφίλησεν αὐτὴν, καὶ ἐκάθισεν ἐπὶ τοῦ θρόνου· καὶ ἐτέθη θρόνος τῇ μητρὶ τοῦ βασιλέως, καὶ ἐκάθισεν ἐκ δεξιῶν αὐτοῦ.
\VS{20}Καὶ εἶπεν αὐτῷ, αἴτησιν μίαν μικρὰν ἐγὼ αἰτοῦμαι παρὰ σοῦ, μὴ ἀποστρέψῃς τὸ πρόσωπόν μου· καὶ εἶπεν αὐτῇ ὁ βασιλεύς, αἴτησαι, μήτερ ἐμὴ, καὶ οὐκ ἀποστρέψω σε.
\VS{21}Καὶ εἶπε, δοθήτω δὴ Ἀβισὰ ἡ Σωμανίτις τῷ Αδωνίᾳ τῷ ἀδελφῷ σου εἰς γυναῖκα.
\VS{22}Καὶ ἀπεκρίθη ὁ βασιλεὺς Σαλωμὼν, καὶ εἶπε τῇ μητρὶ αὐτοῦ, καὶ ἱνατί σὺ ᾔτησαι τὴν Ἀβισὰλ τῷ Ἀδωνίᾳ; καὶ αἴτησαι αὐτῷ τὴν βασιλείαν, ὅτι οὗτος ἀδελφός μου ὁ μέγας ὑπὲρ ἐμέ, καὶ αὐτῷ Ἀβιάθαρ ὁ ἱερεὺς, καὶ αὐτῷ Ἰωὰβ ὁ υἱὸς Σαρουίας ἀρχιστράτηγος ἑταῖρος.
\VS{23}Καὶ ὤμοσεν ὁ βασιλεὺς Σαλωμὼν κατὰ τοῦ Κυρίου, λέγων, τάδε ποιήσαι μοι ὁ Θεὸς καὶ τάδε παροσθείη, ὅτι κατὰ τῆς ψυχῆς αὐτοῦ ἐλάλησεν Ἀδωνίας τὸν λόγον τοῦτον.
\VS{24}Καὶ νῦν ζῇ Κύριος ὃς ἡτοίμασε με καὶ ἔθετό με ἐπὶ τὸν θρόνον Δαυὶδ τοῦ πατρός μου, καὶ αὐτὸς ἐποίησέ μοι οἶκον καθὼς ἐλάλησε Κύριος, ὅτι σήμερον θανατωθήσεται Ἀδωνίας.
\VS{25}Καὶ ἐξαπέστειλεν ὁ βασιλεὺς Σαλωμὼν ἐν χειρὶ Βαναίου υἱοῦ Ἰωδαὲ, καὶ ἀνεῖλεν αὐτὸν, καὶ ἀπέθανεν Ἀδωνίας ἐν τῇ ἡμέρᾳ ἐκείνῃ.
\par }{\PP \VS{26}Καὶ τῷ Ἀβιάθαρ τῷ ἱερεῖ εἶπεν ὁ βασιλεύς, ἀπότρεχε σὺ εἰς Ἀναθὼθ εἰς ἀγρόν σου, ὅτι ἀνὴρ θανάτου εἶ σὺ ἐν τῇ ἡμέρᾳ ταύτῃ· καὶ οὐ θανατώσω σε, ὅτι ᾖρας τὴν κιβωτὸν τῆς διαθήκης Κυρίου ἐνώπιον τοῦ πατρός μου, καὶ ὅτι ἐκακουχήθης ἐν πᾶσιν οἷς ἐκακουχήθη ὁ πατήρ μου.
\VS{27}Καὶ ἐξέβαλε Σαλωμὼν τὸν Ἀβιάθαρ τοῦ μὴ εἶναι ἱερέα τοῦ Κυρίου, πληρωθῆναι τὸ ῥῆμα Κυρίου, ὃ ἐλάλησεν ἐπὶ τὸν οἶκον Ἡλὶ ἐν Σηλώμ.
\par }{\PP \VS{28}Καὶ ἡ ἀκοὴ ἦλθεν ἕως Ἰωὰβ υἱοῦ Σαρουίας, ὅτι Ἰωὰβ ἦν κεκλικὼς ὀπίσω Ἀδωνίου, καὶ ὀπίσω Σαλωμὼν οὐκ ἔκλινε· καὶ ἔφυγεν Ἰωὰβ εἰς τὸ σκήνωμα τοῦ Κυρίου, καὶ κατέσχε τῶν κεράτων τοῦ θυσιαστηρίου.
\VS{29}Καὶ ἀπηγγέλη τῷ Σαλωμὼν, λέγοντες, ὅτι πέφευγεν Ἰωὰβ εἰς τὴν σκηνὴν τοῦ Κυρίου, καὶ ἰδοὺ κατέχει τῶν κεράτων τοῦ θυσιαστηρίου· καὶ ἀπέστειλε Σαλωμὼν ὁ βασιλεὺς πρὸς Ἰωὰβ, λέγων, τί γέγονέ σοι, ὅτι πέφευγας εἰς τὸ θυσιαστήριον; καὶ εἶπεν Ἰωὰβ, ὅτι ἐφοβήθην ἀπὸ προσώπου σου, καὶ ἔφυγον πρὸς Κύριον· καὶ ἀπέστειλε Σαλωμὼν τὸν Βαναίου υἱὸν Ἰωδαὲ, λέγων, πορεύου καὶ ἄνελε αὐτὸν, καὶ θάψον αὐτόν.
\par }{\PP \VS{30}Καὶ ἦλθε Βαναίας υἱὸς Ἰωδαὲ πρὸς Ἰωὰβ εἰς τὴν σκηνὴν τοῦ Κυρίου, καὶ εἶπεν αὐτῷ τάδε λέγει ὁ βασιλεὺς, ἔξελθε· καὶ εἶπεν Ἰωὰβ, οὐκ ἐκπορεύωμαι, ὅτι ὧδε ἀποθανοῦμαι· καὶ ἐπέστρεψε Βαναίας υἱὸς Ἰωδαὲ, καὶ εἶπε τῷ βασιλεῖ, λέγων, τάδε λελάληκεν Ἰωὰβ, καὶ τάδε ἀποκέκριταί μοι.
\VS{31}Καὶ εἶπεν αὐτῷ ὁ βασιλεύς, πορεύου, καὶ ποίησον αὐτῷ καθὼς εἴρηκε, καὶ ἄνελε αὐτὸν· καὶ θάψεις αὐτὸν, καὶ ἐξαρεῖς σήμερον τὸ αἷμα ὃ δωρεὰν ἐξέχεεν, ἀπʼ ἐμοῦ καὶ ἀπὸ τοῦ οἴκου τοῦ πατρός μου.
\VS{32}Καὶ ἐπέστρεψε Κύριος τὸ αἷμα τῆς ἀδικίας αὐτοῦ εἰς κεφαλὴν αὐτοῦ, ὡς ἀπήντησε τοῖς δυσὶν ἀνθρώποις τοῖς δικαίοις καὶ ἀγαθοῖς ὑπὲρ αὐτὸν, καὶ ἀπέκτεινεν αὐτοὺς ἐν ῥομφαίᾳ, καὶ ὁ πατήρ μου Δαυὶδ οὐκ ἔγνω τὸ αἷμα αὐτῶν, τὸν Ἀβεννὴρ υἱὸν Νὴρ ἀρχιστράτηγον Ἰσραὴλ, καὶ τὸν Ἀμεσσὰ υἱὸν Ἰεθὲρ ἀρχιστράτηγον Ἰούδα.
\VS{33}Καὶ ἐπεστράφη τὰ αἵματα αὐτῶν εἰς κεφαλὴν αὐτοῦ, καὶ εἰς κεφαλὴν τοῦ σπέρματος αὐτοῦ εἰς τὸν αἰῶνα· καὶ τῷ Δαυὶδ καὶ τῷ σπέρματι αὐτοῦ καὶ τῷ οἴκῳ αὐτοῦ καὶ τῷ θρόνῳ αὐτοῦ γένοιτο εἰρήνη ἕως αἰῶνος παρὰ Κυρίου.
\VS{34}Καὶ ἀνέβη Βαναίας υἱὸς Ἰωδαὲ, καὶ ἀπήντησεν αὐτῷ, καὶ ἐθανάτωσεν αὐτὸν, καὶ ἔθαψεν αὐτὸν ἐν τῷ οἴκῳ αὐτοῦ ἐν τῇ ἐρήμῳ.
\par }{\PP \VS{35}Καὶ ἔδωκεν ὁ βασιλεὺς τὸν Βαναίου υἱὸν Ἰωδαὲ ἀντʼ αὐτοῦ ἐπὶ τὴν στρατηγίαν· καὶ ἡ βασιλεία κατωρθοῦτο ἐν Ἰερουσαλήμ· καὶ Σαδὼκ τὸν ἱερέα ἔδωκεν αὐτὸν ὁ βασιλεὺς εἰς ἱερέα πρῶτον ἀντὶ Ἀβιάθαρ. Καὶ Σαλωμὼς υἱὸς Δαυὶδ ἐβασίλευσεν ἐπὶ Ἰσραὴλ καὶ Ἰούδα ἐν Ἱερουσαλήμ·
\VS{35a}καὶ ἔδωκε Κύριος φρόνησιν τῷ Σαλωμὼν, καὶ σοφίαν πολλὴν σφόδρα, καὶ πλάτος καρδίας, ὡς ἡ ἄμμος ἡ παρὰ τὴν θάλασσαν.
\par }{\PP \VS{35b}Καὶ ἐπληθύνθη ἡ φρόνησις Σαλωμὼν σφόδρα ὑπὲρ τὴν φρόνησιν πάντων υἱῶν ἀρχαίων, καὶ ὑπὲρ πάντας φρονίμους Αἰγύπτου·
\VS{35c}Καὶ ἔλαβε τὴν θυγατέρα Φαραὼ, καὶ εἰσήγαγεν αὐτὴν εἰς πόλιν Δαυὶδ ἕως συντελέσαι αὐτὸν οἰκοδομῆσαι τὸν οἶκον αὐτοῦ, καὶ τὸν οἶκον Κυρίου ἐν πρώτοις, καὶ τὸ τεῖχος Ἱερουσαλὴμ κυκλόθεν· ἐν ἑπτὰ ἔτεσιν ἐποίησε καὶ συνετέλεσε.
\par }{\PP \VS{35d}Καὶ ἦν τῷ Σαλωμὼν ἐβδομήκοντα χιλιάδες αἴροντες ἄρσιν, καὶ ὀγδοήκοντα χιλιάδες λατόμων ἐν τῷ ὄρει·
\VS{35e}καὶ ἐποίησε Σαλωμὼν τὴν θάλασσαν, καὶ τὰ ὑποστηρίγματα, καὶ τοὺς λουτῆρας τοὺς μεγάλους, καὶ τοὺς στύλους, καὶ τἠν κρήνην τῆς αὐλῆς, καὶ τὴν θάλασσαν τὴν χαλκῆν· καὶ ᾠκοδόμησε τὴν ἄκραν ἔπαλξιν ἐπʼ αὐτῆς, διεκόψεν τὴν πόλιν Δαυίδ.
\VS{35f}οὕτως θυγάτηρ Φαραὼ ἀνέβαινεν ἐκ τῆς πόλεως Δαυὶδ εἰς τὸν οἶκον αὐτῆς, ὃν ᾠκοδόμησεν αὐτῇ· τότε ᾠκοδόμησε τὴν ἄκραν·
\VS{35g}καὶ Σαλωμὼν ἀνέφερε τρεῖς ἐν τῷ ἐνιαυτῷ ὁλοκαυτώσεις καὶ εἰρηνικὰς ἐπὶ τὸ θυσιαστήριον ὃ ᾠκοδόμησε τῷ Κυρίῳ, καὶ ἐθυμία ἐνώπιον Κυρίου, καὶ συνετέλεσε τὸν οἶκον.
\VS{35h}Καὶ οὗτοι οἱ ἄρχοντες οἱ καθεσταμένοι ἐπὶ τὰ ἔργα τοῦ Σαλωμὼν, τρεῖς χιλιάδες καὶ ἑξακόσιοι ἐπιστάται τοῦ λαοῦ τῶν ποιούντων τὰ ἔργα·
\VS{35i}καὶ ᾠκοδόμησε τὴν Ἀσσοὺρ, καὶ τὴν Μαγδὼ, καὶ τὴν Γαζὲρ, καὶ τὴν Βαιθωρὼν ἐπάνω, καὶ τὰ Βαλλάθ·
\VS{35k}πλὴν μετὰ τὸ οἰκοδομῆσαι αὐτὸν τὸν οἶκον τοῦ Κυρίου, καὶ τὸ τεῖχος Ἱερουσαλὴμ κύκλῳ, μετὰ ταῦτα ᾠκοδόμησε τὰς πόλεις ταύτας.
\par }{\PP \VS{35l}Καὶ ἐν τῷ ἔτι Δαυὶδ ζῇν, ἐνετείλατο τῷ Σαλωμὼν, λέγων, ἰδοὺ μετὰ σοῦ Σεμεῒ υἱὸς Γηρὰ υἱὸς τοῦ σπέρματος τοῦ Ἰεμινὶ ἐκ Χεβρών·
\VS{35m}οὗτος κατηράσατό με κατάραν ὀδυνηρὰν ἐν ᾗ ἡμέρᾳ ἐπορευόμην εἰς παρεμβολάς·
\VS{35n}καὶ αὐτὸς κατέβαινεν εἰς ἀπαντήν μοι ἐπὶ τὸν Ἰορδάνην, καὶ ὤμοσα αὐτῷ κατὰ τοῦ Κυρίου, λέγων, εἰ θανατωθήσεται ἐν ῥομφαίᾳ·
\VS{35o}καὶ νῦν μὴ ἀθωώσῃς αὐτὸν, ὅτι ἀνὴρ φρόνιμος σύ· καὶ γνώσῃ ἃ ποιήσεις αὐτῷ, καὶ κατάξεις τὴν πολιὰν αὐτοῦ ἐν αἵματι εἰς ᾅδου.
\par }{\PP \VS{36}Καὶ ἐκάλεσεν ὁ βασιλεὺς τὸν Σεμεῒ, καὶ εἶπεν αὐτῷ, ὠκοδόμησον σεαυτῷ οἶκον ἐν Ἱερουσαλὴμ καὶ κάθου ἐκεῖ, καὶ οὐκ ἐξελεύσῃ ἐκεῖθεν οὐδαμοῦ.
\VS{37}Καὶ ἔσται ἐν τῇ ἡμέρᾳ τῆς ἐξόδου σου καὶ διαβήσῃ τὸν χείμαῤῥον Κέδρων, γινώσκων γνώσῃ ὅτι θανάτῳ ἀποθανῇ· τὸ αἷμά σου ἔσται ἐπὶ τὴν κεφαλήν σου· καὶ ὥρκισεν αὐτὸν ὁ βασιλεὺς ἐν τῇ ἡμέρᾳ ἐκείνῃ.
\VS{38}Καὶ εἶπε Σεμεῒ πρὸς τὸν βασιλέα, ἀγαθὸν τὸ ῥῆμα ὃ ἐλάλησας, κύριέ μου βασιλεῦ· οὕτω ποιήσει ὁ δοῦλός σου· καὶ ἐκάθισε Σεμεῒ ἐν Ἱερουσαλὴμ τρία ἔτη.
\par }{\PP \VS{39}Καὶ ἐγενήθη μετὰ τὰ τρία ἔτη, καὶ ἀπέδρασαν δύο δοῦλοι τοῦ Σεμεῒ πρὸς Ἀγχοὺς υἱὸν Μααχὰ βασιλέα Γέθ· καὶ ἀπηγγέλη τῷ Σεμεῒ, λέγοντες, ἰδοὺ οἱ δοῦλοί σου ἐν Γέθ.
\VS{40}Καὶ ἀνέστη Σεμεῒ καὶ ἐπέσαξε τὴν ὄνον αὐτοῦ, καὶ ἐπορεύθη εἰς Γὲθ πρὸς Ἀγχοὺς τοῦ ἐκζητῆσαι τοὺς δούλους αὐτοῦ· καὶ ἐπορεύθη Σεμεῒ, καὶ ἤγαγε τοὺς δούλους αὐτοῦ ἐκ Γέθ.
\VS{41}Καὶ ἀπηγγέλη τῷ Σαλωμὼν, λέγοντες, ὅτι ἐπορεύθη Σεμεῒ ἐξ Ἱερουσαλὴμ εἰς Γὲθ, καὶ ἀνέστρεψε τοὺς δούλους αὐτοῦ.
\VS{42}Καὶ ἀπέστειλεν ὁ βασιλεὺς καὶ ἐκάλεσε τὸν Σεμεῒ, καὶ εἶπε πρὸς αὐτόν, οὐχὶ ὥρκισά σε κατὰ τοῦ Κυρίου, καὶ ἐπεμαρτυράμην σοι, λέγων, ἐν ᾗ ἂν ἡμέρᾳ ἐξέλθῃς ἐξ Ἱερουσαλὴμ, καὶ πορευθῇς εἰς δεξιὰ ἢ εἰς ἀριστερά, γινώσκων γνώσῃ ὅτι θανάτῳ ἀποθανῇ;
\VS{43}Καὶ τί ὅτι οὐκ ἐφύλαξας τὸν ὅρκον Κυρίου, καὶ τὴν ἐντολὴν ἣν ἐνετειλάμην κατὰ σοῦ;
\par }{\PP \VS{44}Καὶ εἶπεν ὁ βασιλεὺς πρὸς Σεμεῒ, σὺ οἶδας πᾶσαν τὴν κακίαν σου ἣν οἶδεν ἡ καρδία σου, ἃ ἐποίησας Δαυὶδ τῷ πατρί μου, καὶ ἀνταπέδωκε Κύριος τὴν κακίαν σου εἰς κεφαλήν σου.
\VS{45}Καὶ ὁ βασιλεὺς Σαλωμὼν εὐλογημένος, καὶ ὁ θρόνος Δαυὶδ ἔσται ἕτοιμος ἐνώπιον Κυρίου εἰς τὸν αἰῶνα.
\VS{46}Καὶ ἐνετείλατο ὁ βασιλεὺς Σαλωμὼν τῷ Βαναίᾳ υἱῷ Ἰωδαὲ, καὶ ἐξῆλθε καὶ ἀνεῖλεν αὐτόν.
\par }{\PP \VS{46a}Καὶ ἦν ὁ βασιλεὺς Σαλωμὼν φρόνιμος σφόδρα καὶ σοφός· καὶ Ἰούδα καὶ Ἰσραὴλ πολλοὶ σφόδρα, ὡς ἡ ἄμμος ἡ ἐπὶ τῆς θαλάσσης εἰς πλῆθος, ἐσθίοντες καὶ πίνοντες καὶ χαίροντες·
\VS{46b}καὶ Σαλωμὼν ἦν ἄρχων ἐν πάσαις ταῖς βασιλείαις· καὶ ἦσαν προσφέροντες δῶρα, καὶ ἐδούλευον τῷ Σαλωμὼν πάσας τὰς ἡμέρας τῆς ζωῆς αὐτοῦ·
\VS{46c}καὶ Σαλωμὼν ἤρξατο ἀνοίγειν τὰ δυναστεύματα τοῦ Λιβάνου·
\VS{46d}καὶ αὐτὸς ᾠκοδόμησε τὴν Θερμαὶ ἐν τῇ ἐρήμῳ·
\VS{46e}καὶ τοῦτο τὸ ἄριστον τῷ Σαλωμών· τριάκοντα κόροι σεμιδάλεως, καὶ ἑξήκοντα κόροι ἀλεύρου κεκοπανισμένου, δέκα μόσχοι ἐκλεκτοὶ, καὶ εἴκοσι βόες νομάδες, καὶ ἑκατὸν πρόβατα, ἐκτὸς ἐλάφων καὶ δορκάδων καὶ ὀρνίθων ἐκλεκτῶν νομάδων·
\VS{46f}ὅτι ἦν ἄρχων ἐν παντὶ πέραν τοῦ ποταμοῦ ἀπὸ Ῥαφὶ ἕως Γάζης ἐν πᾶσι τοῖς βασιλεῦσι πέραν τοῦ ποταμοῦ·;
\VS{46g}καὶ ἦν αὐτῷ εἰρήνη ἐκ πάντων τῶν μερῶν αὐτοῦ κυκλόθεν· καὶ κατῴκει Ἰούδα καὶ Ἰσραὴλ πεποιθότες, ἕκαστος ὑπὸ τὴν ἄμπελον αὐτοῦ, καὶ ὑπὸ τὴν συκῆν αὐτοῦ, ἐσθίοντες καὶ πίνοντες καὶ ἑορτάζοντες ἀπὸ Δὰν καὶ ἕως Βηρσαβεὲ πάσας τὰς ἡμέρας Σαλωμών.
\par }{\PP \VS{46h}Καὶ οὗτοι οἱ ἄρχοντες τοῦ Σαλωμών. Ἀζαρίου υἱὸς Σαδὼκ τοῦ ἱερέως, καὶ Ὀρνίου υἱὸς Νάθαν ἄρχων τῶν ἐφεστηκότων· καὶ ἔδραμεν ἐπὶ τὸν οἶκον αὐτοῦ· καὶ Σουβὰ γραμματεὺς, καὶ Βασὰ υἱὸς Ἀχιθαλὰμ ἀναμιμνήσκων, καὶ Ἀβὶ υἱὸς Ἰωὰβ ἀρχιστράτηγος, καὶ Ἀχιρὲ υἱὸς Ἐδραῒ ἐπὶ τὰς ἄρσεις, καὶ Βαναίας υἱὸς Ἰωδαὲ ἐπὶ τῆς αὐλαρχίας καὶ ἐπὶ τοῦ πλινθίου, καὶ Καχοὺρ υἱὸς Νάθαν ὁ σύμβουλος.
\par }{\PP \VS{46i}Καὶ ἦσαν τῷ Σαλωμὼν τεσσαράκοντα χιλιάδες τοκάδες ἵπποι εἰς ἅρματα, καὶ δώδεκα χιλιάδες ἵππων·
\VS{46k}καὶ ἦν ἄρχων ἐν πᾶσι τοῖς βασιλεῦσιν ἀπὸ τοῦ ποταμοῦ καὶ ἕως γῆς ἀλλοφύλων καὶ ἕως ὁρίων Αἰγύπτου·
\VS{46l}καὶ Σαλωμὼν υἱὸς Δαυὶδ ἐβασίλευσεν ἐπὶ Ἰσραὴλ καὶ Ἰούδα ἐν Ἱερουσαλήμ.

\par }\Chap{3}{\PP \VS{2}Πλὴν ὁ λαὸς ἦσαν θυμιῶντες ἐπὶ τοῖς ὑψηλοῖς, ὅτι οὐκ ᾠκοδομήθη οἶκος τῷ Κυρίῳ ἕως τοῦ νῦν.
\VS{3}Καὶ ἠγάπησε Σαλωμὼν τὸν Κύριον πορεύεσθαι ἐν τοῖς προστάγμασι Δαυὶδ τοῦ πατρὸς αὐτοῦ, πλὴν ἐν τοῖς ὑψηλοῖς ἔθυε καὶ ἐθυμία.
\VS{4}Καὶ ἀνέστη καὶ ἐπορεύθη εἰς Γαβαὼν θῦσαι ἐκεῖ, ὅτι αὕτη ὑψηλοτάτη, καὶ μεγάλη· χιλίαν ὁλοκαύτωσιν ἀνήνεγκε Σαλωμὼν ἐπὶ τὸ θυσιαστήριον ἐν Γαβαών.
\par }{\PP \VS{5}Καὶ ὤφθη Κύριος τῷ Σαλωμὼν ἐν ὕπνῳ τὴν νύκτα, καὶ εἶπε Κύριος πρὸς Σαλωμὼν, αἴτησαί τι αἴτημα σεαυτῷ.
\VS{6}Καὶ εἶπε Σαλωμὼν, σὺ ἐποιήσας μετὰ τοῦ δούλου σου Δαυὶδ τοῦ πατρός μου ἔλεος μέγα, καθὼς διῆλθεν ἐνώπιόν σου ἐν ἀληθείᾳ καὶ ἐν δικαιοσύνῃ, καὶ ἐν εὐθύτητι καρδίας μετὰ σοῦ, καὶ ἐφύλαξας αὐτῷ τὸ ἔλεος τὸ μέγα τοῦτο, δοῦναι τὸν υἱὸν αὐτοῦ ἐπὶ τοῦ θρόνου αὐτοῦ, ὡς ἡ ἡμέρα αὕτη.
\VS{7}Καὶ νῦν, Κύριε ὁ Θεός μου, σὺ ἔδωκας τὸν δοῦλόν σου ἀντὶ Δαυὶδ τοῦ πατρός μου· καὶ ἐγώ εἰμι παιδάριον μικρὸν, καὶ οὐκ οἶδα τὴν ἔξοδόν μου καὶ τὴν εἴσοδόν μου.
\VS{8}Ὁ δὲ δοῦλός σου ἐν μέσῳ τοῦ λαοῦ σου, ὃν ἐξελέξω, λαὸν πολὺν, ὃς οὐκ ἀριθμηθήσεται.
\VS{9}Καὶ δώσεις τῷ δούλῳ σου καρδίαν ἀκούειν καὶ διακρίνειν τὸν λαόν σου ἐν δικαιοσύνῃ, καὶ τοῦ συνιεῖν ἀναμέσον ἀγαθοῦ καὶ κακοῦ· ὅτι τίς δυνηθήσεται κρίνειν τὸν λαόν σου τὸν βαρὺν τοῦτον;
\par }{\PP \VS{10}Καὶ ἤρεσεν ἐνώπιον Κυρίου, ὅτι ᾐτῄσατο Σαλωμὼν τὸ ῥῆμα τοῦτο.
\VS{11}Καὶ εἶπε Κύριος πρὸς αὐτὸν, ἀνθʼ ὧν ᾐτήσω παρʼ ἐμοῦ τὸ ῥῆμα τοῦτο, καὶ οὐκ ᾐτήσω σεαυτῷ ἡμέρας πολλὰς, καὶ οὐκ ᾐτήσω πλοῦτον, οὐδὲ ᾐτήσω ψυχὰς ἐχθρῶν σου, ἀλλʼ ᾐτήσω σεαυτῷ τοῦ συνιεῖν τοῦ εἰσακούειν κρίμα,
\VS{12}ἰδοὺ πεποίηκα κατὰ τὸ ῥῆμά σου· ἰδοὺ δέδωκά σοι καρδίαν φρονίμην καὶ σοφήν· ὡς σὺ οὐ γέγονεν ἔμπροσθέν σου, καὶ μετὰ σὲ οὐκ ἀναστήσεται ὅμοιός σοι.
\VS{13}Καὶ ἃ οὐκ ᾐτήσω δέδωκά σοι, καὶ πλοῦτον καὶ δόξαν, ὡς οὐ γέγονεν ἀνὴρ ὅμοιός σοι ἐν βασιλεῦσι.
\VS{14}Καὶ ἐὰν πορευθῇς ἐν τῇ ὁδῷ μου φυλάσσειν τὰς ἐντολάς μου καὶ τὰ προστάγματά μου, ὡς ἐπορεύθη Δαυὶδ ὁ πατήρ σου, καὶ πληθυνῶ τὰς ἡμέρας σου.
\VS{15}Καὶ ἐξυπνίσθη Σαλωμὼν, καὶ ἰδοὺ ἐνύπνιον· καὶ ἀνέστη καὶ παραγίνεται εἰς Ἱερουσαλήμ, καὶ ἔστη κατὰ πρόσωπον τοῦ θυσιαστηρίου τοῦ κατὰ πρόσωπον κιβωτοῦ διαθήκης Κυρίου ἐν Σιὼν, καὶ ἀνήγαγεν ὁλοκαυτώσεις, καὶ ἐποίησεν εἰρηνικὰς, καὶ ἐποίησε πότον μέγαν ἑαυτῷ καὶ πᾶσι τοῖς παισὶν αὐτοῦ.
\par }{\PP \VS{16}Τότε ὤφθησαν δύο γυναῖκες πόρναι τῷ βασιλεῖ, καὶ ἔστησαν ἐνώπιον αὐτοῦ.
\VS{17}Καὶ εἶπεν ἡ γυνὴ ἡ μία, ἐν ἐμοὶ κύριε, ἐγὼ καὶ ἡ γυνὴ αὕτη ᾠκοῦμεν ἐν οἴκῳ ἑνί, καὶ ἐτέκομεν ἐν τῷ οἴκῳ.
\VS{18}Καὶ ἐγενήθη ἐν τῇ ἡμέρᾳ τῇ τρίτῃ τεκούσης μου, ἔτεκε καὶ ἡ γυνὴ αὕτη· καὶ ἡμεῖς κατὰ τὸ αὐτό· καὶ οὐκ ἔστιν οὐθεὶς μεθʼ ἡμῶν πάρεξ ἀμφοτέρων ἡμῶν ἐν τῷ οἴκῳ.
\VS{19}Καὶ ἀπέθανεν ὁ υἱὸς τῆς γυναικὸς ταύτης τὴν νύκτα, ὡς ἐπεκοιμήθη ἐπʼ αὐτόν.
\VS{20}Καὶ ἀνέστη μέσης τῆς νυκτὸς, καὶ ἔλαβε τὸν υἱόν μου ἐκ τῶν ἀγκαλῶν μου, καὶ ἐκοίμισεν αὐτὸν ἐν τῷ κόλπῳ αὐτῆς, καὶ τὸν υἱὸν αὐτῆς τὸν τεθνηκότα ἐκοίμισεν ἐν τῷ κόλπῳ μου.
\VS{21}Καὶ ἀνέστην τοπρωῒ θηλάσαι τὸν υἱόν μου, καὶ ἐκεῖνος ἦν τεθνηκώς· καὶ ἰδοὺ κατενόησα αὐτὸν πρωῒ, καὶ ἰδοὺ οὐκ ἦν ὁ υἱός μου ὃν ἔτεκον.
\VS{22}Καὶ εἶπεν ἡ γυνὴ ἡ ἑτέρα, οὐχί, ἀλλὰ ὁ υἱός μου ὁ ζῶν, ὁ δὲ υἱός σου ὁ τεθνηκώς· καὶ ἐλάλησαν ἐνώπιον τοῦ βασιλέως.
\par }{\PP \VS{23}Καὶ εἶπεν ὁ βασιλεὺς αὐταῖς, σὺ λέγεις, οὗτος ὁ υἱός μου ὁ ζῶν, καὶ ὁ υἱὸς ταύτης ὁ τεθνηκώς· καὶ σὺ λέγεις, οὐχί, ἀλλὰ ὁ υἱός μου ὁ ζῶν, καὶ ὁ υἱός σου ὁ τεθνηκώς.
\VS{24}Καὶ εἶπεν ὁ βασιλεύς, λάβετε μάχαιραν· καὶ προσήνεγκαν τὴν μάχαιραν ἐνώπιον τοῦ βασιλέως.
\VS{25}Καὶ εἶπεν ὁ βασιλεύς, διέλετε τὸ παιδίον τὸ ζῶν τὸ θηλάζον εἰς δύο, καὶ δότε τὸ ἥμισυ αὐτοῦ ταύτῃ, καὶ τὸ ἥμισυ αὐτοῦ ταύτῃ.
\VS{26}Καὶ ἀπεκρίθη ἡ γυνὴ ἧς ἦν ὁ υἱὸς ὁ ζῶν, καὶ εἶπε πρὸς τὸν βασιλέα, ὅτι ἐταράχθη ἡ μήτρα αὐτῆς ἐπὶ τῷ υἱῷ αὐτῆς, καὶ εἶπεν, ἐν ἐμοὶ κύριε, δότε αὐτῇ τὸ παιδίον, καὶ θανάτῳ μὴ θανατώσητε αὐτό· καὶ αὕτη εἶπε, μήτε ἐμοὶ, μήτε αὐτῇ ἔστω, διέλετε.
\VS{27}Καὶ ἀπεκρίθη ὁ βασιλεὺς, καὶ εἶπε, δότε τὸ παιδίον τῇ εἰπούσῃ, δότε αὐτῇ αὐτὸ, καὶ θανάτῳ μὴ θανατώσητε αὐτὸ, αὕτη ἡ μήτηρ αὐτοῦ.
\VS{28}Καὶ ἤκουσαν πᾶς Ἰσραὴλ τὸ κρίμα τοῦτο ὃ ἔκρινεν ὁ βασιλεὺς, καὶ ἐφοβήθησαν ἀπὸ προσώπου τοῦ βασιλέως, ὅτι εἶδον ὅτι φρόνησις Θεοῦ ἐν αὐτῷ τοῦ ποιεῖν δικαίωμα.

\par }\Chap{4}{\PP \VerseOne{1}Καὶ ἦν ὁ βασιλεὺς Σαλωμὼν βασιλεύων ἐπὶ Ἰσραήλ.
\VS{2}Καὶ οὗτοι ἄρχοντες οἳ ἦσαν αὐτῷ· Ἀζαρίας υἱὸς Σαδώκ·
\VS{3}Ἐλιὰφ καὶ Ἀχιὰ υἱὸς Σηβὰ γραμματεῖς· καὶ Ἰωσαφὰτ υἱὸς Ἀχιλοὺδ ἀναμιμνήσκων·
\VS{4}Καὶ Βαναίας υἱὸν Ἰωδαὲ ἐπὶ τῆς δυνάμεως· καὶ Σαδὼκ καὶ Ἀβιάθαρ ἱερεῖς·
\VS{5}Καὶ Ὀρνία υἱὸς Νάθαν ἐπὶ τῶν καθεσταμένων· καὶ Ζαβοὺθ υἱὸς Νάθαν ἑταῖρος τοῦ βασιλέως·
\VS{6}Καὶ Ἀχισὰρ ἦν οἰκονόμος· καὶ Ἐλιὰκ ὁ οἰκονόμος· καὶ Ἐλιὰβ υἱὸς Σὰφ ἐπὶ τῆς πατριᾶς· καὶ Ἀδωνιρὰμ υἱὸς Αὐδῶν ἐπὶ τῶν φόρων.
\par }{\PP \VS{7}Καὶ τῷ Σαλωμὼν δώδεκα καθεστάμενοι ἐπὶ πάντα Ἰσραήλ, χορηγεῖν τῷ βασιλεῖ καὶ τῷ οἴκῳ αὐτοῦ· μῆνα ἐν τῷ ἐνιαυτῷ ἐγίνετο ἐπὶ τὸν ἕνα χορηγεῖν.
\VS{8}Καὶ ταῦτα τὰ ὀνόματα αὐτῶν. Βεὲν υἱὸς Ὢρ ἐν ὄρει Ἐφραὶμ εἷς.
\VS{9}Υἱὸς Δακὰρ ἐν Μακὲς, καὶ ἐν Σαλαβὶν, καὶ Βαιθσαμὺς, καὶ Ἐλὼν ἕως Βηθανὰν εἷς.
\VS{10}Υἱὸς Ἐσδὶ, ἐν Ἀραβὼθ, αὐτοῦ Σωχὼ καὶ πᾶσα ἡ γῆ Ὀφέρ.
\VS{11}Υἱοῦ Ἀμιναδὰβ πᾶσα Νεφθαδὼρ, Τεφὰθ θυγάτηρ Σαλωμὼν ἦν αὐτῷ εἰς γυναῖκα, εἷς.
\VS{12}Βανὰ υἱὸς Ἀχιλοὺθ τὴν Ἰθαανὰχ, καὶ Μαγεδδὼ, καὶ πᾶς ὁ οἶκος Σὰν ὁ παρὰ Σεσαθὰν ὑποκάτω τοῦ Ἐσραὲ, καὶ ἐκ Βηθσὰν ἕως Σαβελμαουλᾶ, ἕως Μαεβὲρ Λουκάμ, εἷς.
\VS{13}Υἱὸς Ναβὲρ ἐν Ῥαβὼθ Γαλαὰδ, τούτῳ σχοίνισμα Ἐργὰβ ἐν τῇ Βασὰν, ἑξήκοντα πόλεις μεγάλαι τειχήρεις καὶ μοχλοὶ χαλκοῖ, εἷς.
\VS{14}Ἀχιναδὰβ υἱὸς Σαδδὼ Μααναΐμ.
\VS{15}Ἀχιμαὰς ἐν Νεφθαλίμ, καὶ οὗτος ἔλαβε τὴν Βασεμμὰθ θυγατέρα Σαλωμὼν εἰς γυναῖκα, εἷς.
\VS{16}Βαανὰ υἱὸς Χουσὶ ἐν Ἀσὴρ καὶ ἐν Βααλὼθ, εἷς.
\VS{17}Σεμεῒ υἱὸς Ἠλὰ ἐν τῷ Βενιαμίν.
\VS{18}Γαβὲρ υἱὸς Ἀδαῒ ἐν τῇ γῇ Γὰδ Σηὼν βασιλέως τοῦ Ἐσεβὼν καὶ Ὢγ βασιλέως τοῦ Βασὰν, καὶ νασὲφ εἷς ἐν γῇ Ἰούδα.
\VS{19}Ἰωσαφὰτ υἱὸς Φουασοὺδ ἐν Ἱσσάχαρ.

\Chap{5}\VerseOne{1}Καὶ ἐχορήγουν οἱ καθεστάμενοι οὕτως τῷ βσαιλεῖ Σαλωμών· καὶ πάντα τὰ διαγγέλματα ἐπὶ τὴν τράπεζαν τοῦ βασιλέως ἕκαστος μῆνα αὐτοῦ, οὐ παραλλάσσουσι λόγον. Καὶ τὰς κριθὰς καὶ τὸ ἄχυρον τοῖς ἵπποις καὶ τοῖς ἅρμασιν ᾖρον εἰς τὸν τόπον οὗ ἂν ᾖ ὁ βασιλεὺς, ἕκαστος κατὰ τὴν σύνταξιν αὐτοῦ.
\par }{\PP \VS{2}Καὶ ταῦτα τὰ δέοντα τῷ Σαλωμών· ἐν ἡμέρᾳ μιᾷ τριάκοντα κόροι σεμιδάλεως, καὶ ἑξήκοντα κόροι ἀλεύρου κεκοπανισμένου,
\VS{3}καὶ δέκα μόσχοι ἐκλεκτοὶ, καὶ εἴκοσι βόες νομάδες, καὶ ἑκατὸν πρόβατα, ἐκτὸς ἐλάφων, καὶ δορκάδων ἐκλεκτῶν σιτευτά.
\VS{4}Ὅτι ἦν ἄρχων πέραν τοῦ ποταμοῦ, καὶ ἦν αὐτῷ εἰρήνη ἐκ πάντων τῶν μερῶν κυκλόθεν.
\par }{\PP \VS{9}Καὶ ἔδωκε Κύριος φρόνησιν τῷ Σαλωμὼν καὶ σοφίαν πολλὴν σφόδρα καὶ χύμα καρδίας, ὡς ἡ ἄμμος ἡ παρὰ τὴν θάλασσαν.
\VS{10}Καὶ ἐπληθύνθη Σαλωμὼν σφόδρα ὑπὲρ τὴν φρόνησιν πάντων ἀρχαίων ἀνθρώπων, καὶ ὑπὲρ πάντας φρονίμους Αἰγύπτου.
\VS{11}Καὶ ἐσοφίσατο ὑπὲρ πάντας τοὺς ἀνθρώπους· καὶ ἐσοφίσατο ὑπὲρ Γαιθὰν τὸν Ζαρίτην, καὶ τὸν Αἰνὰν, καὶ τὸν Χαλκὰδ καὶ Δαράλα υἱοὺς Μάλ.
\VS{12}Καὶ ἐλάλησε Σαλωμὼν τρισχιλίας παραβολὰς, καὶ ἦσαν ᾠδαι αὐτοῦ πεντακισχίλιαι.
\VS{13}Καὶ ἐλάλησεν ὑπὲρ τῶν ξύλων ἀπὸ τῆς κέδρου τῆς ἐν τῷ Λιβάνῳ, καὶ ἕως τῆς ὑσσώπου τῆς ἐκπορευομένης διὰ τοῦ τοίχου· καὶ ἐλάλησε περὶ τῶν κτηνῶν καὶ περὶ τῶν πετεινῶν καὶ περὶ τῶν ἑρπετῶν καὶ περὶ τῶν ἰχθύων.
\VS{14}Καὶ παρεγίνοντο πάντες οἱ λαοὶ ἀκοῦσαι τῆς σοφίας Σαλωμών· καὶ παρὰ πάντων τῶν βασιλέων τῆς γῆς, ὅσοι ἤκουον τῆς σοφίας αὐτοῦ·
\par }{\PP \VS{14a}Καὶ ἔλαβε Σαλωμὼν τὴν θυγατέρα Φαραὼ αὐτῷ εἰς γυναῖκα, καὶ εἰσήγαγεν αὐτὴν εἰς τὴν πόλιν Δαυὶδ ἕως συντελέσαι αὐτὸν τὸν οἶκον Κυρίου, καὶ τὸν οἶκον ἑαυτοῦ, καὶ τὸ τεῖχος Ἱερουσαλήμ·
\VS{14b}τότε ἀνέβη Φαραὼ βασιλεὺς Αἰγύπτου, καὶ προκατελάβετο τὴν Γαζὲρ, καὶ ἐνεπύρισεν αὐτὴν, καὶ τὸν Χανανίτην τὸν κατοικοῦντα ἐν Μεργάβ· καὶ ἔδωκεν αὐτὰς Φαραὼ ἀποστολὰς θυγατρὶ αὐτοῦ γυναικὶ Σαλωμών· καὶ Σαλωμὼν ᾠκοδόμησε τὴν Γαζέρ.
\par }{\PP \VS{15}Καὶ ἀπέστειλε Χιρὰμ βασιλεὺς Τύρου τοὺς παῖδας αὐτοῦ χρίσαι τὸν Σαλωμὼν ἀντὶ Δαυὶδ τοῦ πατρὸς αὐτοῦ, ὅτι ἀγαπῶν ἦν Χιρὰμ τὸν Δαυὶδ πάσας τὰς ἡμέρας.
\VS{16}Καὶ ἀπέστειλε Σαλωμὼν πρὸς Χιρὰμ, λέγων,
\VS{17}Σὺ οἶδας τὸν πατέρα μου Δαυὶδ, ὅτι οὐκ ἠδύνατο οἰκοδομῆσαι οἶκον τῷ ὀνόματι Κυρίου Θεοῦ μου ἀπὸ προσώπου τῶν πολέμων τῶν κυκλωσάντων αὐτὸν, ἕως τοῦ δοῦναι Κύριον αὐτοὺς ὑπὸ τὰ ἴχνη τῶν ποδῶν αὐτοῦ.
\VS{18}Καὶ νῦν ἀνέπαυσε Κύριος ὁ Θεός μου ἐμοὶ κυκλόθεν, οὐκ ἔστιν ἐπίβουλος καὶ οὐκ ἔστιν ἁμάρτημα πονηρόν.
\VS{19}Καὶ ἰδοὺ ἐγὼ λέγω οἰκοδομῆσαι οἶκον τῷ ὀνόματι Κυρίου Θεοῦ μου, καθὼς ἐλάλησε Κύριος ὁ Θεὸς πρὸς Δαυὶδ τὸν πατέρα μου, λέγων, ὁ υἱός σου ὃν δώσω ἀντὶ σοῦ ἐπὶ τὸν θρόνον σου, οὗτος οἰκοδομήσει τὸν οἶκον τῷ ὀνόματί μου.
\VS{20}Καὶ νῦν ἔντειλαι, καὶ κοψάτωσάν μοι ξύλα ἐκ τοῦ Λιβάνου· καὶ ἰδοὺ οἱ δοῦλοί μου μετὰ τῶν δούλων σου, καὶ τὸν μισθὸν δουλείας σου δώσω σοι κατὰ πάντα ὅσα ἂν εἴπῃς, ὅτι σὺ οἶδας, ὅτι οὐκ ἔστιν ἡμῖν εἰδὼς ξύλα κόπτειν καθὼς οἱ Σιδώνιοι.
\par }{\PP \VS{21}Καὶ ἐγενήθη καθὼς ἤκουσε Χιρὰμ τῶν λόγων Σαλωμὼν, ἐχάρη σφόδρα, καὶ εἶπεν, εὐλογητὸς ὁ Θεὸς σήμερον, ὃς ἔδωκε τῷ Δαυὶδ υἱὸν φρόνιμον ἐπὶ τὸν λαὸν τὸν πολὺν τοῦτον.
\VS{22}Καὶ ἀπέστειλε πρὸς Σαλωμὼν, λέγων, ἀκήκοα περὶ πάντων ὧν ἀπέσταλκας πρὸς μέ· ἐγὼ ποιήσω πᾶν θέλημά σου· ξύλα κέδρινα καὶ πεύκινα
\VS{23}οἱ δοῦλοί μου κατάξουσιν αὐτὰ ἐκ τοῦ Λιβάνου εἰς τὴν θάλασσαν, ἐγὼ θήσομαι αὐτὰ σχεδίας, ἕως τοῦ τόπου οὗ ἐὰν ἀποστείλῃς πρὸς μὲ, καὶ ἐκτινάξω αὐτὰ ἐκεῖ, καὶ σὺ ἀρεῖς· καὶ ποιήσεις τὸ θέλημά μου, τοῦ δοῦναι ἄρτους τῷ οἴκῳ μου.
\par }{\PP \VS{24}Καὶ ἦν Χιρὰμ διδοὺς τῷ Σαλωμὼν κέδρους καὶ πεύκας καὶ πᾶν θέλημα αὐτοῦ.
\VS{25}Καὶ Σαλωμὼν ἔδωκε τῷ Χιρὰμ εἴκοσι χιλιάδας κόρους πυροῦ καὶ μαχεὶρ τῷ οἴκῳ αὐτοῦ, καὶ εἴκος χιλιάδας βαὶθ ἐλαίου κεκομμένου· κατὰ τοῦτο ἐδίδου Σαλωμὼν τῷ Χιρὰμ κατʼ ἐνιαυτόν.
\VS{26}Καὶ Κύριος ἔδωκε σοφίαν τῷ Σαλωμὼν καθὼς ἐλάλησεν αὐτῷ· καὶ ἦν εἰρήνη ἀναμέσον Χιρὰμ καὶ ἀναμέσον Σαλωμὼν, καὶ διέθεντο διαθήκην ἀναμέσον αὐτῶν.
\par }{\PP \VS{27}Καὶ ἀνήνεγκεν ὁ βασιλεὺς φόρον ἐκ παντὸς Ἰσραὴλ, καὶ ἦν ὁ φόρος τριάκοντα χιλιάδες ἀνδρῶν.
\VS{28}Καὶ ἀπέστειλεν αὐτοὺς εἰς τὸν Λίβανον, δέκα χιλιάδες ἐν τῷ μηνὶ ἀλλασσόμενοι· μῆνα ἦσαν ἐν τῷ Λιβάνῳ, καὶ δύο μῆνας ἐν οἴκῳ αὐτῶν· καὶ Ἀδωνιρὰμ ἐπὶ τοῦ φόρου.
\VS{29}Καὶ ἦν τῷ Σαλωμὼν ἑβδομήκοντα χιλιάδες αἴροντες ἄρσιν, καὶ ὀγδοήκοντα χιλιάδες λατόμων ἐν τῷ ὄρει,
\VS{30}χωρὶς τῶν ἀρχόντων τῶν καθεσταμένων ἐπὶ τῶν ἔργων τῷ Σαλωμὼν, τρεῖς χιλιάδες καὶ ἑξακόσιοι ἐπιστάται οἱ ποιοῦντες τὰ ἔργα.
\VS{32}Καὶ ἡτοίμασαν τοὺς λίθους καὶ τὰ ξύλα τρία ἔτη.

\par }\Chap{6}{\PP \VerseOne{1}Καὶ ἐγενήθη ἐν τῷ τεσσαρακοστῷ καὶ τετρακοσιοστῷ ἔτει τῆς ἐξόδου υἱῶν Ἰσραὴλ ἐξ Αἰγύπτου, τῷ ἔτει τῷ τετάρτῳ ἐν μηνὶ τῷ δευτέρῳ βασιλεύοντος τοῦ βασιλέως Σαλωμὼν ἐπὶ τὸν Ἰσραὴλ,
\VS{1a}καὶ ἐνετείλατο ὁ βασιλεὺς ἵνα ἄρωσι λίθους μεγάλους τιμίους εἰς τὸν θεμέλιον τοῦ οἴκου, καὶ λίθους ἀπελεκήτους.
\VS{1b}Καὶ ἐπελέκησαν οἱ υἱοὶ Σαλωμὼν, καὶ οἱ υἱοὶ Χιρὰμ, καὶ ἔβαλαν αὐτούς.
\par }{\PP \VS{1c}Ἐν τῷ ἔτει τῷ τετάρτῷ ἐθεμελίωσε τὸν οἶκον Κυρίου ἐν μηνὶ Ζιοῦ, καὶ τῷ δευτέρῳ μηνί.
\VS{1d}Ἐν ἑνδεκάτῳ ἐνιαυτῷ, ἐν μηνὶ Βαὰλ, οὗτος ὁ μὴν ὁ ὄγδοος, συνετελέσθη ὁ οἶκος εἰς πάντα λόγον αὐτοῦ, καὶ εἰς πᾶσαν διάταξιν αὐτοῦ.
\VS{2}Καὶ ὁ οἶκος ὃν ᾠκοδόμησεν ὁ βασιλεὺς τῷ Κυρίῳ, τεσσαράκοντα ἐν πήχει μῆκος αὐτοῦ, καὶ εἴκοσι ἐν πήχει πλάτος αὐτοῦ, καὶ πέντε καὶ εἴκοσι ἐν πήχει τὸ ὕψος αὐτοῦ·
\VS{3}Καὶ τὸ αἰλὰμ κατὰ πρόσωπον τοῦ ναοῦ, εἴκοσι ἐν πήχει μῆκος αὐτοῦ εἰς τὸ πλάτος τοῦ οἴκου, κατὰ πρόσωπον τοῦ οἴκου, καὶ ᾠκοδόμησε τὸν οἶκον, καὶ συνετέλεσεν αὐτόν.
\VS{4}Καὶ ἐποίησε τῷ οἴκῳ θυρίδας παρακυπτομένας κρυπτάς.
\par }{\PP \VS{5}Καὶ ἔδωκεν ἐπὶ τὸν τοῖχον τοῦ οἴκου μέλαθρα κυκλόθεν τῷ ναῷ καὶ τῷ δαβίρ.
\VS{6}Ἡ πλευρὰ ἡ ὑποκάτω πέντε πήχεων ἐν πήχει τὸ πλάτος αὐτῆς, καὶ τὸ μέσον ἓξ, καὶ ἡ τρίτη ἑπτὰ ἐν πήχει τὸ πλάτος αὐτῆς· ὅτι διάστημα ἔδωκε τῷ οἴκῳ κυκλόθεν ἔξωθεν τοῦ οἴκου, ὅπως μὴ ἐπιλαμβάνωνται τῶν τοίχων τοῦ οἴκου.
\VS{7}Καὶ ὁ οἴκος ἐν τῷ οἰκοδομεῖσθαι αὐτὸν λίθοις ἀκροτομοις ἀργοῖς ᾠκοδομήθη· καὶ σφύρα καὶ πέλεκυς καὶ πᾶν σκεῦος σιδηροῦν οὐκ ἠκούσθη ἐν τῷ οἴκῳ ἐν τῷ οἰκοδομεῖσθαι αὐτόν.
\VS{8}Καὶ ὁ πυλὼν τῆς πλευρᾶς τῆς ὑποκάτωθεν ὑπὸ τὴν ὠμίαν τοῦ οἴκου τὴν δεξιὰν, καὶ ἑλικτὴ ἀνάβασις εἰς τὸ μέσον, καὶ ἐκ τῆς μέσης ἐπὶ τὸ τριόροφα.
\VS{9}Καὶ ᾠκοδόμησε τὸν οἶκον καὶ συνετέλεσεν αὐτόν· καὶ ἐκοιλοστάθμησε τὸν οἶκον κέδροις.
\VS{10}Καὶ ᾠκοδόμησε τοὺς ἐνδέσμους, διʼ ὅλου τοῦ οἴκου πέντε ἐν πήχει τὸ ὕψος αὐτοῦ, καὶ συνέσχε τὸν σύνδεσμον ἐν ξύλοις κεδρίνοις.
\par }{\PP \VS{15}Καὶ ᾠκοδόμησε τοὺς τοίχους τοῦ οἴκου ἔσωθεν διὰ ξύλων κεδρίνων ἀπὸ τοῦ ἐδάφους τοῦ οἴκου καὶ ἕως τῶν τοίχων καὶ ἕως τῶν δοκῶν· ἐκοιλοστάθμησε συνεχόμενα ξύλοις ἔσωθεν· καὶ περιέσχε τὸ ἔσω τοῦ οἴκου ἐν πλευραῖς πευκίναις.
\VS{16}Καὶ ᾠκοδόμησε τοὺς εἴκοσι πήχεις ἀπʼ ἄκρου τοῦ τοίχου τὸ πλευρὸν τὸ ἓν ἀπὸ τοῦ ἐδάφους ἕως τῶν δοκῶν· καὶ ἐποίησεν ἐκ τοῦ δαβὶρ εἰς τὸ ἁγιον τῶν ἁγίων.
\VS{17}Καὶ τεσσαράκοντα πήχεων ἦν ὁ ναὸς κατὰ πρόσωπον τοῦ δαβὶρ ἐν μέσῳ τοῦ οἴκου ἔσωθεν,
\VS{19}δοῦναι ἐκεῖ τὴν κιβωτὸν διαθήκης Κυρίου.
\VS{20}Εἴκοσι πήχεις μῆκος, καὶ εἴκοσι πήχεις πλάτος, καὶ εἴκοσι πήχεις τὸ ὕψος αὐτοῦ· Καὶ περιέσχεν αὐτὸ χρυσίῳ συγκεκλεισμένῳ· καὶ ἐποίησε θυσιαστήριον
\VS{21}κατὰ πρόσωπον τοῦ δαβὶρ, καὶ περιέσχεν αὐτὸ χρυσίῳ.
\VS{22}Καὶ ὅλον τὸν οἶκον περιέσχε χρυσίῳ, ἕως συντελείας παντὸς τοῦ οἴκου.
\par }{\PP \VS{23}Καὶ ἐποίησεν ἐν τῷ δαβὶρ δύο χερουβὶμ δέκα πήχεων μέγεθος ἐσταθμωμένον·
\VS{24}Καὶ πέντε πήχεων πτερύγιον τοῦ χερουβὶμ τοῦ ἑνὸς, καὶ πέντε πήχεων πτερύγιον αὐτοῦ τὸ δεύτερον, ἐν πήχει δέκα ἀπὸ μέρους πτερυγίου αὐτοῦ εἰς μέρος πτερυγίου αὐτοῦ.
\VS{25}Οὕτως τῷ χερουβὶμ τῷ δευτέρῳ, ἐν μέτρῳ ἑνὶ συντέλεια μία ἀμφοτέροις.
\VS{26}Καὶ τὸ ὕψος τοῦ χερουβὶμ τοῦ ἑνὸς δέκα ἐν πήχει· καὶ οὕτω τῷ χερουβὶμ τῷ δευτέρῳ.
\VS{27}Καὶ ἀμφότερα χερουβὶμ ἐν μέσῳ τοῦ οἴκου τοῦ ἐσωτάτου· καὶ διεπέτασε τὰς πτέρυγας αὐτῶν, καὶ ἥπτετο πτέρυξ μία τοῦ τοίχου, καὶ πτέρυξ χερουβὶμ τοῦ δευτέρου ἥπτετο τοῦ τοίχου τοῦ δευτέρου· καὶ αἱ πτέρυγες αὐτῶν ἐν μέσῳ τοῦ οἴκου ἥπτοντο πτέρυξ πτέρυγος.
\VS{28}Καὶ περιέσχε τὰ χερουβὶμ χρυσίῳ.
\par }{\PP \VS{29}Πάντας τοὺς τοίχους τοῦ οἴκου κύκλῳ ἐγκολαπτὰ ἔγραψε γραφίδι χερουβὶμ, καὶ φοίνικας τῷ ἐσωτέρῳ καὶ τῷ ἐξωτέρῳ.
\VS{30}Καὶ τὸ ἔδαφος τοῦ οἴκου περιέσχε χρυσίῳ τοῦ ἐσωτάτου καὶ τοῦ ἐξωτάτου.
\par }{\PP \VS{31}Καὶ τῷ θυρώματι τοῦ δαβὶρ ἐποίησε θύρας ξύλων ἀρκευθίνων, στοαὶ τετραπλῶς,
\VS{34}ἐν ἀμφοτέραις ταῖς θύραις ξύλα πεύκινα· δύο πτυχαὶ ἡ θύρα ἡ μία καὶ στροφεῖς αὐτῶν, καὶ δύο πτυχαὶ ἡ θύρα ἡ δευτέρα στρεφόμενα ἐγκεκολαμμένα χερουβὶμ,
\VS{35}καὶ φοίνικες, καὶ διαπεπετασμένα πέταλα, καὶ περιεχόμενα χρυσίῳ καταγομένῳ ἐπὶ τὴν ἐκτύπωσιν.
\VS{36}Καὶ ᾠκοδόμησε τὴν αὐλὴν τὴν ἐσωτάτην· τρεῖς στίχους ἀπελεκήτων, καὶ στίχος κατειργασμένης κέδρου κυκλόθεν·
\VS{36a}καὶ ᾠκοδόμησε τὸ καταπέτασμα τῆς αὐλῆς τοῦ αἰλὰμ τοῦ οἴκου τοῦ κατὰ πρόσωπον τοῦ ναοῦ.

\par }\Chap{7}{\PP \VerseOne{1}Καὶ ἀπέστειλεν ὁ βασιλεὺς Σαλωμὼν, καὶ ἔλαβε τὸν Χιρὰμ ἐκ Τύρου,
\VS{2}υἱὸν γυναικὸς χήρας, καὶ οὗτος ἀπὸ τῆς φυλῆς τῆς Νεφθαλίμ, καὶ ὁ πατὴρ αὐτοῦ ἀνὴρ Τύριος· τέκτων χαλκοῦ, καὶ πεπληρωμένος τῆς τέχνης καὶ συνέσεως καὶ ἐπιγνώσεως τοῦ ποιεῖν πᾶν ἔργον ἐν χαλκῷ· καὶ εἰσηνέχθη πρὸς τὸν βασιλέα Σαλωμών· καὶ ἐποίησε πάντα τὰ ἔργα.
\par }{\PP \VS{3}Καὶ ἐχώνευσε τοὺς δύο στύλους τῷ αἰλὰμ τοῦ οἴκου· ὀκτωκαίδεκα πήχεις ὕψος τοῦ στύλου· καὶ περίμετρον τεσσαρεσκαίδεκα πήχεις ἐκύκλου αὐτὸν τὸ πάχος τοῦ στύλου· τεσσάρων δσκτύλων τὰ κοιλώματα· καὶ οὕτως ὁ στύλος ὁ δεύτερος·
\VS{4}Καὶ δύο ἐπιθέματα ἐποίησε δοῦναι ἐπὶ τὰς κεφαλὰς τῶν στύλων χωνευτά· πέντε πήχεις τὸ ὕψος τοῦ ἐπιθέματος τοῦ ἑνὸς, καὶ πέντε πήχεις τὸ ὕψος τοῦ ἐπιθέματος τοῦ δευτέρου
\VS{5}Καὶ ἐποίησε δύο δίκτυα περικαλύψαι τὸ ἐπίθεμα τῶν στύλων· καὶ δίκτυον τῷ ἐπιθέματι τῷ ἑνὶ, καὶ δίκτυον τῷ ἐπιθέματι τῷ δευτέρῳ.
\VS{6}Καὶ ἔργον κρεμαστὸν, δύο στίχοι ῥοῶν χαλκῶν, δεδικτυωμένοι, ἔργον κρεμαστὸν, στίχος ἐπὶ στίχον· καὶ οὕτως ἐποίησε τῷ ἐπιθέματι τῷ δευτέρῳ.
\VS{7}Καὶ ἔστησε τοὺς στύλους τοῦ αἰλὰμ τοῦ ναοῦ· καὶ ἔστησε τὸν στύλον τὸν ἕνα, καὶ ἐπεκάλεσε τὸ ὄνομα αὐτοῦ Ἰαχούμ· καὶ ἔστησε τὸν στύλον τὸν δεύτερον, καὶ ἐπεκάλεσε τὸ ὄνομα αὐτοῦ Βολώζ.
\VS{8}Καὶ ἐπὶ τῶν κεφαλῶν τῶν στύλων ἔργον κρίνου κατὰ τὸ αἰλὰμ τεσσάρων πηχῶν·
\VS{9}καὶ μέλαθρον ἐπʼ ἀμφοτέρων τῶν στύλων· καὶ ἐπάνωθεν τῶν πλευρῶν ἐπίθεμα τὸ μέλαθρον τῷ πάχει.
\par }{\PP \VS{10}Καὶ ἐποίησε τὴν θάλασσαν δέκα ἐν πήχει ἀπὸ τοῦ χείλους αὐτῆς ἕως τοῦ χείλους αὐτῆς, στρογγυλόν κύκλῳ τὸ αὐτό· πέντε ἐν πήχει τὸ ὕψος αὐτῆς· καὶ συνηγμένη τρεῖς καὶ τριάκοντα ἐν πήχει.
\VS{11}Καὶ ὑποστηρίγματα ὑποκάτωθεν τοῦ χείλους αὐτῆς κυκλόθεν ἐκύκλουν αὐτὴν δέκα ἐν πήχει κυκλόθεν·
\VS{12}καὶ τὸ χεῖλος αὐτῆς ὡς ἔργον χείλους ποτηρίου βλαστὸς κρίνου· καὶ τὸ πάχος αὐτοῦ παλαιστής.
\VS{13}Καὶ δώδεκα βόες ὑποκάτω τῆς θαλάσσης, οἱ τρεῖς ἐπιβλέποντες βοῤῥὰν, καὶ οἱ τρεῖς ἐπιβλέποντες θάλασσαν, καὶ οἱ τρεῖς ἐπιβλέποντες Νότον, καὶ οἱ τρεῖς ἐπιβλέποντες ἀνατολήν· καὶ πάντα τὰ ὀπίσθια εἰς τὸν οἶκον, καὶ ἡ θάλασσα ἐπʼ αὐτῶν ἐπάνωθεν.
\par }{\PP \VS{14}Καὶ ἐποίησε δέκα μεχωνὼθ χαλκᾶς· πέντε πήχεις μῆκος τῆς μεχωνὼθ τῆς μιᾶς, καὶ τέσσαρες πήχεις τὸ πλάτος αὐτῆς, καὶ ἓξ ἐν πήχει ὕψος αὐτῆς.
\VS{15}Καὶ τοῦτο τὸ ἔργον τῶν μεχωνὼθ συγκλειστὸν αὐτοῖς, καὶ συγκλειστὸν ἀναμέσον τῶν ἐξεχομένων.
\VS{16}Καὶ ἐπὶ τὰ συγκλείσματα αὐτῶν ἀναμέσον ἐξεχομένων λέοντες καὶ βόες καὶ χερουβὶμ, καὶ ἐπὶ τῶν ἐξεχομένων, οὕτως καὶ ἐπάνωθεν, καὶ ὑποκάτωθεν τῶν λεόντων καὶ τῶν βοῶν χῶραι, ἔργον καταβάσεως.
\VS{17}Καὶ τέσσαρες τροχοὶ χαλκοῖ τῇ μεχωνὼθ τῇ μιᾷ, καὶ τὰ προσέχοντα χαλκᾶ καὶ τέσσαρα μέρη αὐτῶν, ὠμίαι ὑποκάτω τῶν λουτήρων.
\VS{18}Καὶ χεῖρες ἐν τοῖς τροχοῖς ἐν τῇ μεχωνώθ. Καὶ τὸ ὕψος τοῦ τροχοῦ τοῦ ἑνὸς πήχεος καὶ ἡμίσους.
\VS{19}Καὶ τὸ ἔργον τῶν τροχῶν ἔργον τροχῶν ἅρματος· αἱ χεῖρες αὐτῶν καὶ οἱ νῶτοι αὐτῶν καὶ ἡ πραγματεία αὐτῶν πάντα χωνευτά.
\VS{20}Αἱ τέσσαρες ὠμίαι ἐπὶ τῶν τεσσάρων γωνιῶν τῆς μεχωνὼθ τῆς μιᾶς, ἐκ τῆς μεχωνὼθ οἱ ὦμοι αὐτῆς.
\VS{21}Καὶ ἐπὶ τῆς κεφαλῆς τῆς μεχωνὼθ ἥμισυ τοῦ πήχεος μέγεθος αὐτῆς στρογγύλον κύκλῳ ἐπὶ τῆς κεφαλῆς τῆς μεχωνώθ· καὶ ἀρχὴ χειρῶν αὐτῆς καὶ τὰ συγκλείσματα αὐτῆς· καὶ ἠνοίγετο ἐπὶ τὰς ἀρχὰς τῶν χειρῶν αὐτῆς.
\VS{22}Καὶ τὰ συγκλείσματα αὐτῆς χερουβὶμ καὶ λέοντες καὶ φοίνικες ἑστῶτα, ἐχόμενον ἕκαστον κατὰ πρόσωπον ἔσω καὶ τὰ κυκλόθεν.
\VS{23}Κατʼ αὐτὴν ἐποίησε πάσας τὰς δέκα μεχωνὼθ, τάξιν μίαν καὶ μέτρον ἓν πάσαις.
\VS{24}Καὶ ἐποίησε δέκα χυτροκαύλους χαλκοῦς, τεσσαράκοντα χοεῖς χωροῦντα τὸν ἕνα χυτρόκαυλον μετρήσει τεσσάρων πήχων· χυτρόκαυλος ὁ εἷς ἐπὶ τῇ μεχωνὼθ τῇ μιᾷ ταῖς δέκα μεχωνώθ.
\VS{25}Καὶ ἔθετο τὰς πέντε μεχωνὼθ ἀπὸ τῆς ὠμίας τοῦ οἴκου ἐκ δεξιῶν, καὶ πέντε ἀπὸ τῆς ὠμίας τοῦ οἴκου ἐξ ἀριστερῶν· καὶ ἡ θάλασσα ἀπὸ τῆς ὡμίας τοῦ οἴκου ἐκ δεξιῶν κατʼ ἀνατολὰς ἀπὸ τοῦ κλίτους τοῦ Νότου.
\par }{\PP \VS{26}Καὶ ἐποίησε Χιρὰμ τοὺς λέβητας καὶ τὰς θερμαστρεῖς καὶ τὰς φιάλας· καὶ συνετέλεσε Χιρὰμ ποιῶν πάντα τὰ ἔργα ἃ ἐποίησε τῷ βασιλεῖ Σαλωμὼν ἐν οἴκῳ Κυρίου·
\VS{27}Στύλους δύο, καὶ τὰ στρεπτὰ τῶν στύλων ἐπὶ τῶν κεφαλῶν τῶν στύλων δύο· καὶ τὰ δίκτυα δύο τοῦ καλύπτειν ἀμφότερα τὰ στρεπτὰ τῶν γλυφῶν τὰ ὄντα ἐπὶ τῶν στύλων.
\VS{28}Τὰς ῥοὰς τετρακοσίας ἀμφοτέροις τοῖς δικτύοις, δύο στίχοι ῥοῶν τῷ δικτύῳ τῷ ἑνὶ, περικαλύπτειν ἀμφότερα τὰ ὄντα τὰ στρεπτὰ τῆς μεχωνὼθ ἐπʼ ἀμφοτέροις τοῖς στύλοις·
\VS{29}Καὶ τὰ μεχωνὼθ δέκα, καὶ τοὺς χυτροκαύλους δέκα ἐπὶ τῶν μεχωνώθ·
\VS{30}Καὶ τὴν θάλασσαν μίαν, καὶ τοὺς βόας δώδεκα ὑποκάτω τῆς θαλάσσης·
\VS{31}Καὶ τοὺς λέβητας καὶ τὰς θερμαστρεῖς καὶ τὰς φιάλας καὶ πάντα τὰ σκεύη, ἃ ἐποίησε Χιρὰμ τῷ βασιλεῖ Σαλωμὼν τῷ οἴκῳ Κυρίου· καὶ οἱ στύλοι τεσσαράκοντα καὶ ὀκτὼ τοῦ οἴκου τοῦ βασιλέως καὶ τοῦ οἴκου Κυρίου· πάντα τὰ ἔργα τοῦ βασιλέως ἃ ἐποίησε Χιρὰμ χαλκᾶ ἄρδην.
\VS{32}Οὐκ ἦν σταθμὸς τοῦ χαλκοῦ οὗ ἐποίησε πάντα τὰ ἔργα ταῦτα ἐκ πλήθους σφόδρα· οὐκ ἦν τέρμα τῶν σταθμῶν τοῦ χαλκοῦ.
\VS{33}Ἐν τῷ περιοίκῳ τοῦ Ἰορδάνου ἐχώνευσεν αὐτὰ ἐν τῷ πάχει τῆς γῆς ἀναμέσον Σοκχὼθ καὶ ἀναμέσον Σειρά.
\par }{\PP \VS{34}Καὶ ἔλαβεν ὁ βασιλεὺς Σαλωμὼν τὰ σκεύη ἃ ἐποίησεν ἐν οἴκῳ Κυρίου, τὸ θυσιαστήριον τὸ χρυσοῦν, καὶ τὴν τράπεζαν ἐφʼ ἧς οἱ ἄρτοι τῆς προσφορᾶς, χρυσῆν,
\VS{35}καὶ τὰς λυχνίας πέντε ἐξ ἀριστερῶν, καὶ πέντε ἐκ δεξιῶν κατὰ πρόσωπον τοῦ δαβὶρ χρυσᾶς συγκλειομένας, καὶ τὰ λαμπάδια, καὶ τοὺς λύχνους, καὶ τὰς ἐπαρύστρις χρυσᾶς.
\VS{36}Καὶ τὰ πρόθυρα, καὶ οἱ ἧλοι, καὶ αἱ φιάλαι, καὶ τὰ τρυβλία, καὶ αἱ θυΐσκαι χρυσαῖ, συγκλειστά· καὶ τὰ θυρώματα τῶν θυρῶν τοῦ οἴκου τοῦ ἐσωτάτου ἁγίου τῶν ἁγίων, καὶ τὰς θύρας τοῦ ναοῦ χρυσᾶς.
\par }{\PP \VS{37}Καὶ ἀνεπληρώθη τὸ ἔργον ὃ ἐποίησε Σαλωμὼν οἴκου Κυρίου· καὶ εἰσήνεγκε Σαλωμὼν τὰ ἅγια Δαυὶδ τοῦ πατρὸς αὐτοῦ, καὶ πάντα τὰ ἅγια Σαλωμὼν, τὸ ἀργύριον καὶ τὸ χρυσίον καὶ τὰ σκεύη ἔδωκεν εἰς τοὺς θησαυροὺς οἴκου Κυρίου.
\par }{\PP \VS{38}Καὶ τὸν οἶκον ἑαυτῷ ᾠκοδόμησε Σαλωμὼν τρισκαίδεκα ἔτεσι·
\VS{39}Καὶ ᾠκοδόμησε τὸν οἶκον δρυμῷ τοῦ Λιβάνου· ἑκατὸν πήχεις μῆκος αὐτοῦ, καὶ πεντήκοντα πήχεις πλάτος αὐτοῦ, καὶ τριάκοντα πηχῶν ὕψος αὐτοῦ· καὶ τριῶν στύχων στύλων κεδρίνων, καὶ ὠμίαι κέδριναι τοῖς στύλοις.
\VS{40}Καὶ ἐφάτνωσε τὸν οἶκον ἄνωθεν ἐπὶ τῶν πλευρῶν τῶν στύλων· καὶ ἀριθμὸς τῶν στύλων τεσσαράκοντα καὶ πέντε ὁ στίχος,
\VS{41}καὶ μέλαθρα τρία, καὶ χῶρα ἐπὶ χώραν τρισσῶς.
\VS{42}Καὶ πάντα τὰ θυρώματα, καὶ αἱ χῶραι τετράγωνοι μεμελαθρωμέναι· καὶ ἀπὸ τοῦ θυρώματος ἐπὶ θύραν τρισσῶς.
\VS{43}Καὶ τὸ αἰλὰμ τῶν στύλων, πεντήκοντα μῆκος, καὶ πεντήκοντα ἐν πλάτει ἐζυγωμένα αἰλὰμ ἐπὶ πρόσωπον αὐτῶν· καὶ στύλοι καὶ πάχος ἐπὶ πρόσωπον αὐτῆς τοῖς αἰλαμίν.
\VS{44}Καὶ τὸ αἰλὰμ τῶν θρόνων οὗ κρινεῖ ἐκεῖ, αἰλὰμ τοῦ κριτηρίου.
\par }{\PP \VS{45}Καὶ ὁ οἶκος αὐτῶν ἐν ᾧ καθήσεται ἐκεῖ, αὐλὴ μία ἐξελισσομένη τούτοις κατὰ τὸ ἔργον τοῦτο· Καὶ οἶκον τῇ θυγατρὶ Φαραὼ ἣν ἔλαβε Σαλωμὼν, κατὰ τὸ αἰλὰμ τοῦτο.
\par }{\PP \VS{46}Πάντα ταῦτα ἐκ λίθων τιμίων κεκολαμμένα ἐκ διαστήματος ἔσωθεν καὶ ἐκ τοῦ θεμελίου ἕως τῶν γεισῶν· καὶ ἔξωθεν εἰς τὴν αὐλὴν τὴν μεγάλην,
\VS{47}τὴν τεθεμελιωμένην ἐν τιμίοις λίθοις μεγάλοις, λίθοις δεκαπήχεσι καὶ τοῖς ὀκταπήχεσι·
\VS{48}Καὶ ἐπάνωθεν τιμίοις κατὰ τὸ μέτρον ἀπελεκήτων, καὶ κέδροις.
\VS{49}Τῆς αὐλῆς τῆς μεγάλης κύκλῳ τρεῖς στίχοι ἀπελεκήτων, καὶ στίχος κεκολαμμένης κέδρου·
\VS{50}καὶ συνετέλεσε Σαλωμὼν ὅλον τὸν οἶκον αὐτοῦ.

\par }\Chap{8}{\PP \VerseOne{1}Καὶ ἐγένετο ὡς συνετέλεσε Σαλωμὼν τοῦ οἰκοδομῆσαι τὸν οἶκον Κυρίου καὶ τὸν οἶκον αὐτοῦ μετὰ εἴκοσι ἔτη, τότε ἐξεκκλησίασεν ὁ βασιλεὺς Σαλωμὼν πάντας τοὺς πρεσβυτέρους Ἰσραὴλ ἐν Σιὼν, τοῦ ἐνεγκεῖν τὴν κιβωτὸν διαθήκης Κυρίου ἐκ πόλεως Δαυὶδ, αὕτη ἐστὶ Σιὼν,
\VS{2}ἐν μηνὶ Ἀθανίν.
\par }{\PP \VS{3}Καὶ ᾖραν οἱ ἱερεῖς τὴν κιβωτὸν
\VS{4}καὶ τὸ σκήνωμα τοῦ μαρτυρίου καὶ τὰ σκεύη τὰ ἅγια τὰ ἐν τῷ σκηνώματι τοῦ μαρτυρίου.
\VS{5}Καὶ ὁ βασιλεὺς καὶ πὰς Ἰσραὴλ ἔμπροσθεν τῆς κιβωτοῦ θύοντες πρόβατα, βόας, ἀναρίθμητα·
\VS{6}Καὶ εἰσφέρουσιν οἱ ἱερεῖς τὴν κιβωτὸν εἰς τὸν τόπον αὐτῆς, εἰς τὸ δαβὶρ τοῦ οἴκου, εἰς τὰ ἅγια τῶν ἁγίων, ὑπὸ τὰς πτέρυγας τῶν χερουβίν.
\VS{7}Ὅτι τὰ χερουβὶμ διαπεπετασμένα ταῖς πτέρυξιν ἐπὶ τὸν τόπον τῆς κιβωτοῦ· καὶ περιεκάλυπτον τὰ χερουβὶμ ἐπὶ τὴν κιβωτὸν καὶ ἐπὶ τὰ ἅγια αὐτῆς ἐπάνωθεν.
\VS{8}Καὶ ὑπερεῖχον τὰ ἡγιασμένα· καὶ ἐνεβλέποντο αἱ κεφαλαὶ τῶν ἡγιασμένων ἐκ τῶν ἁγίων εἰς πρόσωπον τοῦ δαβὶρ, καὶ οὐκ ὠπτάνοντο ἔξω.
\VS{9}Οὐκ ἦν ἐν τῇ κιβωτῷ πλὴν δύο πλάκες λίθιναι, πλάκες τῆς διαθήκης ἃς ἔθηκε Μωυσῆς ἐν Χωρὴβ, ἃς διέθετο Κύριος μετὰ τῶν υἱῶν Ἰσραὴλ ἐν τῷ ἐκπορεύεσθαι αὐτοὺς ἐκ γῆς Αἰγύπτου.
\par }{\PP \VS{10}Καὶ ἐγένετο ὡς ἐξῆλθον οἱ ἱερεῖς ἐκ τοῦ ἁγίου, καὶ ἡ νεφέλη ἔπλησε τὸν οἶκον.
\VS{11}Καὶ οὐκ ἠδύναντο οἱ ἱερεῖς στήκειν λειτουργεῖν ἀπὸ προσώπου τῆς νεφέλης, ὅτι ἔπλησε δόξα Κυρίου τὸν οἶκον.
\par }{\PP \VS{14}Καὶ ἀπέστρεψεν ὁ βασιλεὺς τὸ πρόσωπον αὐτοῦ, καὶ εὐλόγησεν ὁ βασιλεὺς πάντα Ἰσραήλ· καὶ πᾶσα ἐκκλησία Ἰσραὴλ εἱστήκει·
\VS{15}Καὶ εἶπεν, εὐλογητὸς Κύριος ὁ Θεὸς Ἰσραὴλ σημερον, ὃς ἐλάλησεν ἐν τῷ στόματι αὐτοῦ περὶ Δαυὶδ τοῦ πατρός μου καὶ ἐν ταῖς χερσὶν αὐτοῦ ἐπλήρωσε, λέγον,
\VS{16}ἀφʼ ἧς ἡμέρας ἐξήγαγον τὸν λαόν μου τὸν Ἰσραὴλ ἐξ Αἰγύπτου, οὐκ ἐξελεξάμην ἐν πόλει ἐν ἑνὶ σκήπτρῳ Ἰσραὴλ τοῦ οἰκοδομῆσαι οἶκον τοῦ εἶναι τὸ ὄνομά μου ἐκεῖ· καὶ ἐξελεξάμην ἐν Ἱερουσαλὴμ εἶναι τὸ ὄνομά μου ἐκεῖ· καὶ ἐξελεξάμην τὸν Δαυὶδ τοῦ εἶναι ἐπὶ τὸν λαόν μου τὸν Ἰσραήλ.
\VS{17}Καὶ ἐγένετο ἐπὶ τῆς καρδίας τοῦ πατρός μου οἰκοδομῆσαι οἶκον τῷ ὀνόματι Κυρίου Θεοῦ Ἰσραήλ.
\VS{18}Καὶ εἶπε Κύριος πρὸς Δαυὶδ τὸν πατέρα μου, ἀνθʼ ὧν ἦλθεν ἐπὶ τὴν καρδίαν σου τοῦ οἰκοδομῆσαι οἶκον τῷ ὀνόματί μου, καλῶς ἐποίησας ὅτι ἐγενήθη ἐπὶ τὴν καρδίαν σου.
\VS{19}Πλὴν σὺ οὐκ οἰκοδομήσεις τὸν οἶκον, ἀλλʼ ἢ ὁ υἱός σου ὁ ἐξελθὼν ἐκ τῶν πλευρῶν σου, οὗτος οἰκοδομήσει τὸν οἶκον τῷ ὀνόματί μου.
\VS{20}Καὶ ἀνέστησε Κύριος τὸ ῥῆμα αὐτοῦ ὃ ἐλάλησε· καὶ ἀνέστην ἀντὶ Δαυὶδ τοῦ πατρός μου, καὶ ἐκάθισα ἐπὶ τοῦ θρόνου Ἰσραὴλ, καθὼς ἐλάλησε Κύριος, καὶ ᾠκοδόμησα τὸν οἶκον τῷ ὀνόματι Κυρίου Θεοῦ Ἰσραήλ.
\VS{21}Καὶ ἐθέμην ἐκεῖ τόπον τῇ κιβωτῷ, ἐν ᾗ ἐστιν ἐκεῖ διαθήκη Κυρίου ἣν διέθετο Κύριος μετὰ τῶν πατέρων ἡμῶν ἐν τῷ ἐξαγαγεῖν αὐτὸν αὐτοὺς ἐκ γῆς Αἰγύπτου.
\par }{\PP \VS{22}Καὶ ἀνέστη Σαλωμὼν κατὰ πρόσωπον τοῦ θυσιαστηρίου Κυρίου ἐνώπιον πάσης ἐκκλησίας Ἰσραήλ· καὶ διεπέτασε τὰς χεῖρας αὐτοῦ εἰς τὸν οὐρανὸν,
\VS{23}καὶ εἶπε, Κύριε ὁ Θεὸς Ἰσραὴλ, οὐκ ἔστιν ὡς σὺ Θεὸς ἐν τῷ οὐρανῷ ἄνω καὶ ἐπὶ τῆς γῆς κάτω, φυλάσσων διαθήκην καὶ ἔλεος τῷ δούλῳ σου τῷ πορευομένῳ ἐνώπιόν σου ἐν ὅλῃ τῇ καρδίᾳ αὐτοῦ,
\VS{24}ἃ ἐφύλαξας τῷ δούλῳ σου Δαυὶδ τῷ πατρί μου· καὶ γὰρ ἐλάλησας ἐν τῷ στόματί σου, καὶ ἐν χερσί σου ἐπλήρωσας, ὡς ἡ ἡμέρα αὕτη.
\VS{25}Καὶ νῦν Κύριε ὁ Θεὸς Ἰσραὴλ, φύλαξον τῷ δούλῳ σου Δαυὶδ τῷ πατρί μου ἃ ἐλάλησας αὐτῷ, λέγων, οὐκ ἐξαρθήσεταί σου ἀνὴρ ἐκ προσώπου μου καθήμενος ἐπὶ θρόνου Ἰσραὴλ, πλὴν ἐὰν φυλάξωνται τὰ τέκνα σου τὰς ὁδοὺς αὐτῶν τοῦ πορεύεσθαι ἐνώπιόν μου καθὼς ἐπορεύθης ἐνώπιον ἐμοῦ.
\VS{26}Καὶ νῦν, Κύριε ὁ Θεὸς Ἰσραὴλ, πιστωθήτω δὴ τὸ ῥῆμά σου τῷ Δαυὶδ τῷ πατρί μου.
\par }{\PP \VS{27}Ὅτι εἰ ἀληθῶς κατοικήσει ὁ Θεὸς μετὰ ἀνθρώπων ἐπὶ τῆς γῆς; εἰ ὁ οὐρανὸς καὶ ὁ οὐρανὸς τοῦ οὐρανοῦ οὐκ ἀρκέσουσί σοι, πλὴν καὶ ὁ οἶκος οὗτος ὃν ᾠκοδόμησα τῷ ὀνόματί σου;
\VS{28}Καὶ ἐπιβλέψῃ ἐπὶ τὴν δέησίν μου Κύριε ὁ Θεὸς Ἰσραὴλ, ἀκούειν τῆς προσευχῆς ἧς ὁ δοῦλός σου προσεύχεται ἐνώπιόν σου πρὸς σὲ σήμερον,
\VS{29}τοῦ εἶναι τοὺς ὀφθαλμούς σου ἠνεῳγμένους εἰς τὸν οἶκον τοῦτον ἡμέρας καὶ νυκτὸς, εἰς τὸν τὸπον ὃν εἶπας, ἔσται τὸ ὄνομά μου ἐκεῖ, τοῦ εἰσακούειν τῆς προσευχῆς ἧς προσεύχεται ὁ δοῦλός σου εἰς τὸν τόπον τοῦτον ἡμέρας καὶ νυκτός.
\VS{30}Καὶ εἰσακούσῃ τῆς δεήσεως τοῦ δούλου σου καὶ τοῦ· λαοῦ σου Ἰσραὴλ ἃ ἂν προσεύξωνται εἰς τὸν τόπον τοῦτον· καὶ σὺ εἰσακούσῃ ἐν τῷ τόπῳ τῆς κατοικήσεώς σου ἐν οὐρανῷ· καὶ ποιήσεις καὶ ἵλεως ἔσῃ.
\par }{\PP \VS{31}Ὅσα ἂν ἁμάρτῃ ἕκαστος τῷ πλησίον αὐτοῦ, καὶ ἐὰν λάβῃ ἐπʼ αὐτὸν ἀρὰν τοῦ ἀράσασθαι αὐτὸν, καὶ ἔλθῃ καὶ ἐξαγορεύσῃ κατὰ πρόσωπον τοῦ θυσιαστηρίου σου ἐν τῷ οἴκῳ τούτῳ,
\VS{32}καὶ σὺ εἰσακούσῃ ἐκ τοῦ οὐρανοῦ καὶ ποιήσεις· καὶ κρινεῖς τὸν λαόν σου Ἰσραὴλ, ἀνομηθῆναι ἄνομον, δοῦναι τὴν ὁδὸν αὐτοῦ εἰς κεφαλὴν αὐτοῦ, καὶ τοῦ δικαιῶσαι δίκαιον, δοῦναι αὐτῷ κατὰ τὴν δικαιοσύνην αὐτοῦ.
\par }{\PP \VS{33}Ἐν τῷ πταῖσαι τὸν λαόν σου Ἰσραὴλ ἐνώπιον ἐχθρῶν, ὅτι ἁμαρτήσονταί σοι, καὶ ἐπιστρέψουσι καὶ ἐξομολογήσονται τῷ ὀνόματί σου, καὶ προσεύξονται καὶ δεηθήσονται ἐν τῷ οἴκῳ τούτῳ,
\VS{34}καὶ σὺ εἰσακούσῃ ἐκ τοῦ οὐρανοῦ, καὶ ἵλεως ἔσῃ ταῖς ἁμαρτίαις τοῦ λαοῦ σου Ἰσραὴλ, καὶ ἐπιστρέψεις αὐτοὺς εἰς τὴν γῆν ἣν ἔδωκας τοῖς πατράσιν αὐτῶν.
\par }{\PP \VS{35}Ἐν τῷ συσχεθῆναι τὸν οὐρανὸν καὶ μὴ γενέσθαι ὑετὸν, ὅτι ἁμαρτήσονταί σοι, καὶ προσεύξονται εἰς τὸν τόπον τοῦτον, καὶ ἐξομολογήσονται τῷ ὀνόματί σου, καὶ ἀπὸ τῶν ἁμαρτιῶν αὐτῶν ἀποστρέψουσιν ὅταν ταπεινώσῃς αὐτοὺς,
\VS{36}καὶ εἰσακούσῃ ἐκ τοῦ οὐρανοῦ, καὶ ἵλεως ἔσῃ ταῖς ἁμαρτίαις τοῦ δούλου σου καὶ τοῦ λαοῦ σου Ἰσραήλ· ὅτι δηλώσεις αὐτοῖς τὴν ὁδὸν τὴν ἀγαθὴν πορεύεσθαι ἐν αὐτῇ, καὶ δώσεις ὑετὸν ἐπὶ τὴν γῆν ἣν ἔδωκας τῷ λαῳ σου ἐν κληρονομίᾳ.
\par }{\PP \VS{37}Λιμὸς ἐὰν γένηται, θάνατος ἐὰν γένηται, ὅτι ἔσται ἐμπυρισμὸς, βροῦχος, ἐρυσίβη ἐὰν γένηται, καὶ ἐὰν θλίψῃ αὐτὸν ὁ ἐχθρὸς αὐτοῦ ἐν μιᾷ τῶν πόλεων αὐτοῦ, πᾶν συνάντημα, πᾶν πόνον,
\VS{38}πᾶσαν προσευχὴν, πᾶσαν δέησιν ἐὰν γένηται παντὶ ἀνθρώπῳ, ὡς ἂν γνῶσιν ἕκαστος ἁφὴν καρδίας αὐτοῦ, καὶ διαπετάσῃ τὰς χεῖρας αὐτοῦ εἰς τὸν οἶκον τοῦτον,
\VS{39}καὶ σὺ εἰσακούσῃ ἐκ τοῦ οὐρανοῦ ἐξ ἑτοίμου κατοικητηρίου σου, καὶ ἵλεως ἔσῃ, καὶ ποιήσεις καὶ δώσεις ἀνδρὶ κατὰ τὰς ὁδοὺς αὐτοῦ, καθὼς ἂν γνῷς τὴν καρδίαν αὐτοῦ, ὅτι σὺ μονώτατος οἶδας τὴν καρδίαν πάντων υἱῶν ἀνθρώπων,
\VS{40}ὅπως φοβῶνταί σε πάσας τὰς ἡμέρας ὅσας αὐτοὶ ζῶσιν ἐπὶ τῆς γῆς, ἧς ἔδωκας τοῖς πατράσιν ἡμῶν.
\par }{\PP \VS{41}Καὶ τῷ ἀλλοτρίῳ ὃς οὐκ ἔστιν ἀπὸ λαοῦ σοῦ οὗτος,
\VS{42}καὶ ἥξουσι καὶ προσεύξονται εἰς τὸν τόπον τοῦτον,
\VS{43}καὶ σὺ εἰσακούσῃ ἐκ τοῦ οὐρανοῦ ἐξ ἑτοίμου κατοικητηρίου σου, καὶ ποιήσεις κατὰ πάντα ὅσα ἂν ἐπικαλέσηταί σε ὁ ἀλλότριος, ὅπως γνῶσι πάντες οἱ λαοὶ τὸ ὄνομά σου, καὶ φοβῶνταί σε, καθὼς ὁ λαός σου Ἰσραὴλ, καὶ γνῶσιν ὅτι τὸ ὄνομά σου ἐπικέκληται ἐπὶ τὸν οἶκον τοῦτον ὃν ᾠκοδόμησα.
\par }{\PP \VS{44}Ὅτι ἐξελεύσεται ὁ λαός σου εἰς πόλεμον ἐπὶ τοὺς ἐχθροὺς αὐτοῦ ἐν ὁδῷ ᾗ ἐπιστρέψεις αὐτοὺς, καὶ προσεύξονται ἐν ὀνόματι Κυρίου ὁδὸν τῆς πόλεως ἧς ἑξελέξω ἐν αὐτῇ, καὶ τοῦ οἴκου οὗ ᾠκοδόμησα τῷ ὀνόματί σου,
\VS{45}καὶ σὺ εἰσακούσῃ ἐκ τοῦ οὐρανοῦ τῆς δεήσεως αὐτῶν, καὶ τῆς προσευχῆς αὐτῶν, καὶ ποιήσεις τὸ δικαίωμα αὐτοῖς.
\par }{\PP \VS{46}Ὅτι ἁμαρτήσονταί σοι, ὅτι οὐκ ἔστιν ἄνθρωπος ὃς οὐχ ἁμαρτήσεται, καὶ ἐπάξεις αὐτοὺς καὶ παραδώσεις αὐτοὺς ἐνώπιον ἐχθρῶν, καὶ αἰχμαλωτιοῦσιν οἱ αἰχμαλωτίζοντες εἰς γῆν μακρὰν ἢ ἐγγὺς,
\VS{47}καὶ ἐπιστρέψουσι καρδίας αὐτῶν ἐν τῇ γῇ οὗ μετήχθησαν ἐκεῖ, καὶ ἐπιστρέψωσιν ἐν γῇ μετοικίας αὐτῶν, καὶ δεηθῶσί σου, λέγοντες, ἡμάρτομεν, ἠδικήσαμεν, ἠνομήσαμεν,
\VS{48}καὶ ἐπιστρέψωσι πρὸς σὲ ἐν ὅλῃ καρδίᾳ αὐτῶν καὶ ἐν ὅλῃ ψυχῇ αὐτῶν ἐν τῇ γῇ ἐχθρῶν αὐτῶν οὗ μετήγαγες αὐτοὺς, καὶ προσεύξονται πρὸς σὲ ὁδὸν γῆς αὐτῶν ἧς ἔδωκας τοῖς πατράσιν αὐτῶν, καὶ τῆς πόλεως ἧς ἐξελέξω, καὶ τοῦ οἴκου οὗ ᾠκοδόμηκα τῷ ὀνόματί σου,
\VS{49}καὶ εἰσακούσῃ ἐκ τοῦ οὐρανοῦ ἐξ ἑτοίμου κατοικητηρίου σου,
\VS{50}καὶ ἵλεως ἔσῃ ταῖς ἀδικίας αὐτῶν αἷς ἥμαρτόν σοι, καὶ κατὰ πάντα τὰ ἀθετήματα αὐτῶν ἃ ἠθέτησάν σοι, καὶ δώσεις αὐτοὺς εἰς οἰκτιρμοὺς ἐνώπιον αἰχμαλωτευόντων αὐτοὺς, καὶ οἰκτειρήσουσιν εἰς αὐτοὺς,
\VS{51}ὅτι λαός σου καὶ κληρονομία σου, οὓς ἐξήγαγες ἐκ γῆς Αἰγύπτου ἐκ μέσου χωνευτηρίου σιδήρου.
\VS{52}Καὶ ἔστωσαν οἱ ὀφθαλμοί σου καὶ τὰ ὦτά σου ἠνεῳγμένα εἰς τὴν δέησιν τοῦ δούλου σου, καὶ εἰς τὴν δέησιν τοῦ λαοῦ σου Ἰσραὴλ, εἰσακούειν αὐτῶν ἐν πᾶσιν οἷς ἂν ἐπικαλέσωνταί σε.
\VS{53}Ὅτι σὺ διέστειλας αὐτοὺς σεαυτῷ εἰς κληρονομίαν ἐκ πάντων τῶν λαῶν τῆς γῆς, καθὼς ἐλάλησας ἐν χειρὶ δούλου σου Μωυσῆ, ἐν τῷ ἐξαγαγεῖν σε τοὺς πατέρας ἡμῶν ἐκ γῆς Αἰγύπτου, Κύριε Κύριε.
\par }{\PP \VS{53a}Τότε ἐλάλησε Σαλωμὼν ὑπὲρ τοῦ οἴκου, ὡς συνετέλεσε τοῦ οἰκοδομῆσαι αὐτὸν, Ἥλιον ἐγνώρισεν ἐν οὐρανῷ· Κύριος εἶπε τοῦ κατοικεῖν ἐν γνόφῳ· οἰκοδόμησον οἶκόν μου, οἶκον εὐπρεπῆ σεαυτῷ τοῦ κατοικεῖν ἐπὶ καινότητος· οὐκ ἰδοὺ αὕτη γέγραπται ἐν βιβλίῳ τῆς ᾠδῆς;
\par }{\PP \VS{54}Καὶ ἐγένετο ὡς συνετέλεσε Σαλωμὼν προσευχόμενος πρὸς Κύριον ὅλην τὴν προσευχὴν καὶ τὴν δέησιν ταύτην, καὶ ἀνέστη ἀπὸ προσώπου τοῦ θυσιαστηρίου Κυρίου ὀκλακὼς ἐπὶ τὰ γόνατα αὐτοῦ, καὶ αἱ χεῖρες αὐτοῦ διαπεπετασμέναι εἰς τὸν οὐρανόν.
\par }{\PP \VS{55}Καὶ ἔστη, καὶ εὐλόγησε πᾶσαν ἐκκλησίαν Ἰσραὴλ φωνῇ μεγάλῃ, λέγων,
\VS{56}εὐλογητὸς Κύριος σήμερον ὃς ἔδωκε κατάπαυσιν τῷ λαῷ αὐτοῦ Ἰσραὴλ, κατὰ πάντα ὅσα ἐλάλησεν· οὐ διεφώνησε λόγος εἷς ἐν πᾶσι τοῖς λόγοις αὐτοῦ τοῖς ἀγαθοῖς οἷς ἐλάλησεν ἐν χειρὶ δούλου αὐτοῦ Μωυσῆ.
\VS{57}Γένοιτο Κύριος ὁ Θεὸς ἡμῶν μεθʼ ἡμῶν, καθὼς ἦν μετὰ τῶν πατέρων ἡμῶν· μὴ ἐγκαταλοίποιτο ἡμᾶς μηδὲ ἀποστρέψοιτο ἡμᾶς,
\VS{58}Γένοιτο Κύριος ὁ Θεὸς ἡμῶν μεθʼ ἡμῶν, καθὼς ἦν μετὰ τῶν πατέρων ἡμῶν· μὴ ἐγκαταλοίποιτο ἡμᾶς μηδὲ ἀποστρέψοιτο ἡμᾶς, ἐπικλῖναι καρδίας ἡμῶν ἐπʼ αὐτὸν τοῦ πορεύεσθαι ἐν πάσαις ὁδοῖς αὐτοῦ, καὶ φυλάσσειν πάσας ἐντολὰς αὐτοῦ, καὶ τὰ προστάγματα αὐτοῦ, ἃ ἐνετείλατο τοῖς πατράσιν ἡμῶν.
\VS{59}Καὶ ἔστωσαν οἱ λόγοι οὗτοι ὡς δεδέημαι ἐνώπιον Κυρίου Θεοῦ ἡμῶν, ἐγγίζοντες πρὸς Κύριον Θεὸν ἡμῶν ἡμέρας καὶ νυκτὸς, τοῦ ποιεῖν τὸ δικαίωμα τοῦ δοῦλου σου, καὶ τὸ δικαίωμα λαοῦ Ἰσραὴλ ῥῆμα ἡμέρας ἐν ἡμέρᾳ ἐνιαυτοῦ·
\VS{60}ὅπως γνῶσι πάντες οἱ λαοὶ τῆς γῆς, ὅτι Κύριος ὁ Θεὸς, αὐτὸς Θεὸς, καὶ οὐκ ἔστιν ἔτι.
\VS{61}Καὶ ἔστωσαν αἱ καρδίαι ἡμῶν τέλειαι πρὸς Κύριον Θεὸν ἡμῶν, καὶ ὁσίως πορεύεσθαι ἐν τοῖς προστάγμασιν αὐτοῦ, καὶ φυλάσσειν ἐντολὰς αὐτοῦ, ὡς ἡ ἡμέρα αὕτη.
\par }{\PP \VS{62}Καὶ ὁ βασιλεὺς καὶ πάντες οἱ υἱοὶ Ἰσραὴλ ἔθυσαν θυσίαν ἐνώπιον Κυρίου.
\VS{63}Καὶ ἔθυσεν ὁ βασιλεὺς Σαλωμὼν τὰς θυσίας τῶν εἰρηνικῶν ἃς ἔθυσε τῷ Κυρίῳ, βοῶν δύο καὶ εἴκοσι χιλιάδας, προβάτων ἑκατὸν καὶ εἴκοσι χιλιάδας· καὶ ἐνεκαίνισε τὸν οἶκον Κυρίου ὁ βασιλεὺς καὶ πάντες οἱ υἱοὶ Ἰσραήλ.
\VS{64}Τῇ ἡμέρᾳ ἐκείνῃ ἡγίασεν ὁ βασιλεὺς τὸ μέσον τῆς αὐλῆς τὸ κατὰ πρόσωπον τοῦ οἴκου Κυρίου· ὅτι ἐποίησεν ἐκεῖ τὴν ὁλοκαύτωσιν καὶ τὰς θυσίας καὶ τὰ στέατα τῶν εἰρηνικῶν, ὅτι τὸ θυσιαστήριον τὸ χαλκοῦν τὸ ἐνώπιον Κυρίου μικρὸν τοῦ μὴ δύνασθαι τὴν ὁλοκαύτωσιν καὶ τὰς θυσίας τῶν εἰρηνικῶν ὑπενεγκεῖν.
\par }{\PP \VS{65}Καὶ ἑποίησε Σαλωμὼν τὴν ἑορτὴν ἐν τῇ ἡμέρᾳ ἐκείνῃ, καὶ πᾶς Ἰσραὴλ μετʼ αὐτοῦ, ἐκκλησία μεγάλη ἀπὸ τῆς εἰσόδου Ἡμὰθ ἕως ποταμοῦ Αἰγύπτου, ἐνώπιον Κυρίου Θεοῦ ἡμῶν ἐν τῷ οἴκῳ ᾧ ᾠκοδόμησεν, ἐσθίων καὶ πίνων καὶ εὐφραινόμενος ἐνώπιον Κυρίου Θεοῦ ἡμῶν ἑπτὰ ἡμέρας.
\VS{66}Καὶ ἐν τῇ ἡμέρᾳ τῇ ὀγδόῃ ἐξαπέστειλε τὸν λαόν· καὶ εὐλόγησαν τὸν βασιλέα, καὶ ἀπῆλθεν ἕκαστος εἰς τὰ σκηνώματα αὐτοῦ χαίροντες· καὶ ἀγαθὴ ἡ καρδία ἐπὶ τοῖς ἀγαθοῖς οἷς ἐποίησε Κύριος τῷ Δαυὶδ δούλῳ αὐτοῦ, καὶ τῷ Ἰσραὴλ λαῷ αὐτοῦ.

\par }\Chap{9}{\PP \VerseOne{1}Καὶ ἐγενήθη ὡς συνετέλεσε Σαλωμὼν οἰκοδομεῖν τὸν οἶκον Κυρίου, καὶ τὸν οἶκον τοῦ βασιλέως, καὶ πᾶσαν τὴν πραγματείαν Σαλωμὼν, ὅσα ἠθέλησε ποιῆσαι,
\VS{2}καὶ ὤφθη Κύριος τῷ Σαλωμὼν δεύτερον, καθὼς ὤφθη ἐν Γαβαών.
\par }{\PP \VS{3}Καὶ εἶπε πρὸς αὐτὸν Κύριος, ἤκουσα τῆς φωνῆς τῆς προσευχῆς σου, καὶ τῆς δεήσεώς σου ἧς ἐδεήθης ἐνώπιόν μου· πεποίηκά σοι κατὰ πᾶσαν τὴν προσευχήν σου· ἡγίακα τὸν οἶκον τοῦτον ὃν ᾠκοδόμησας τοῦ θέσθαι τὸ ὄνομά μου ἐκεῖ εἰς τὸν αἰῶνα, καὶ ἔσονται οἱ ὀφθαλμοί μου ἐκεῖ καὶ ἡ καρδία μου πάσας τὰς ἡμέρας.
\VS{4}Καὶ σὺ ἐὰν πορευθῇς ἐνώπιον ἐμοῦ, καθὼς ἐπορεύθη Δαυὶδ ὁ πατήρ σου, ἐν ὁσιότητι καρδίας καὶ ἐν εὐθύτητι, καὶ τοῦ ποιεῖν κατὰ πάντα ἃ ἐνετειλάμην αὐτῷ, καὶ τὰ προστάγματά μου καὶ τὰς ἐντολάς μου φυλάξῃς,
\VS{5}καὶ ἀναστήσω τὸν θρόνον τῆς βασιλείας σου ἐν Ἰσραὴλ εἰς τὸν αἰῶνα, καθὼς ἐλάλησα Δαυὶδ πατρί σου, λέγων, οὐκ ἐξαρθήσεταί σοι ἀνὴρ ἡγούμενος ἐν Ἰσραήλ.
\VS{6}Ἐὰν δὲ ἀποστραφέντες ἀποστραφῆτε ὑμεῖς καὶ τὰ τέκνα ὑμῶν ἀπʼ ἐμοῦ, καὶ μὴ φυλάξητε τὰς ἐντολάς μου καὶ τὰ προστάγματά μου ἃ ἔδωκε Μωυσῆς ἐνώπιον ὑμῶν, καὶ πορευθῆτε καὶ δουλεύσητε θεοῖς ἑτέροις καὶ προσκυνήσητε αὐτοῖς,
\VS{7}καὶ ἐξαρῶ τὸν Ἰσραὴλ ἀπὸ τῆς γῆς ἧς ἔδωκα αὐτοῖς, καὶ τὸν οἶκον τοῦτον ὃν ἡγίασα τῷ ὀνόματί μου ἀποῤῥίψω ἐκ προσώπου μου· καὶ ἔσται Ἰσραὴλ εἰς ἀφανισμὸν καὶ εἰς λάλημα εἰς πάντας τοὺς λαούς.
\VS{8}Καὶ ὁ οἶκος οὗτος ἔσται ὁ ὑψηλὸς, πᾶς ὁ διαπορευόμενος διʼ αὐτοῦ ἐκστήσεται καὶ συριεῖ, καὶ ἐροῦσιν, ἕνεκεν τίνος ἐποίησε Κύριος οὕτως τῇ γῇ ταύτῃ καὶ τῷ οἴκῳ τούτῳ;
\VS{9}Καὶ ἐροῦσιν, ἀνθʼ ὧν ἐγκατέλιπον Κύριον Θεὸν αὐτῶν, ὃς ἐξήγαγε τοὺς πατέρας αὐτῶν ἐξ Αἰγύπτου, ἐξ οἴκου δουλείας, καὶ ἀντελάβοντο θεῶν ἀλλοτρίων καὶ προσεκύνησαν αὐτοῖς καὶ ἐδούλευσαν αὐτοῖς, διὰ τοῦτο ἐπήγαγε Κύριος ἐπʼ αὐτοὺς τὴν κακίαν ταύτην.
\par }{\PP \VS{9a}Τότε ἀνήγαγε Σαλωμὼν τὴν θυγατέρα Φαραὼ ἐκ πόλεως Δαυὶδ εἰς οἶκον αὐτοῦ, ὃν ᾠκοδόμησεν ἐαυτῷ ἐν ταῖς ἡμέραις ἐκείναις.
\par }{\PP \VS{10}Εἴκοσι ἔτη ἐν οἷς ᾠκοδόμησε Σαλωμὼν τοὺς δύο οἴκους, τὸν οἶκον Κυρίου καὶ τὸν οἶκον τοῦ βασιλέως,
\VS{11}Χιρὰμ βασιλεὺς Τύρου ἀντελάβετο τοῦ Σαλωμὼν ἐν ξύλοις κεδρίνοις, καὶ ἐν ξύλοις πευκίνοις, καὶ ἐν χρυσίῳ, καὶ ἐν χρυσίῳ, καὶ ἐν παντὶ θελήματι αὐτοῦ· τότε ἔδωκεν ὁ βασιλεὺς τῷ Χιρὰμ εἴκοσι πόλεις ἐν τῇ γῇ τῇ Γαλιλαίᾳ.
\VS{12}Καὶ ἐξήλθε Χιρὰμ ἐκ Τύρου, καὶ ἐπορεύθη εἰς τὴν Γαλιλαίαν τοῦ ἰδεῖν τὰς πόλεις ἃς ἔδωκεν αὐτῷ Σαλωμών· καὶ οὐκ ἤρεσαν αὐτῷ.
\VS{13}Καὶ εἶπε, τί αἱ πόλεις αὗται ἃς ἔδωκάς μοι ἀδελφέ; καὶ ἐκάλεσεν αὐτὰς Ὅριον ἕως τῆς ἡμέρας ταύτης.
\VS{14}Καὶ ἤνεγκε Χιρὰμ τῷ Σαλωμὼν ἑκατὸν καὶ εἴκοσι τάλαντα χρυσίου.
\VS{26}Καὶ ναῦν ὑπὲρ οὗ ἐποίησεν ὁ βασιλεὺς Σαλωμὼν ἐν Γασίων Γαβὲρ τὴν οὖσαν ἐχομένην Αἰλὰθ ἐπὶ τοῦ χείλους τῆς ἐσχάτης θαλάσσης ἐν γῇ Ἐδώμ.
\VS{27}Καὶ ἀπέστειλε Χιρὰμ ἐν τῇ νηῒ τῶν παίδων αὐτοῦ ἄνδρας ναυτικοὺς ἐλαύνειν εἰδότας θάλασσαν μετὰ τῶν παίδων Σαλωμών.
\VS{28}Καὶ ἦλθον εἰς Σωφιρὰ, καὶ ἔλαβον ἐκεῖθεν χρυσίου ἑκατὸν καὶ εἴκοσι τάλαντα, καὶ ἤνεγκαν τῷ βασιλεῖ Σαλωμών.

\par }\Chap{10}{\PP \VerseOne{1}Καὶ βασίλισσα Σαβὰ ἤκουσε τὸ ὄνομα Σαλωμὼν καὶ τὸ ὄνομα Κυρίου, καὶ ἦλθε πειράσαι αὐτὸν ἐν αἰνίγμασι.
\VS{2}Καὶ ἦλθεν εἰς Ἰερουσαλὴμ ἐν δυνάμει βαρείᾳ σφόδρα· καὶ κάμηλοι αἴρουσαι ἡδύσματα καὶ χρυσὸν πολὺν σφόδρα καὶ λίθον τίμιον· καὶ εἰσῆλθε πρὸς Σαλωμὼν, καὶ ἐλάλησεν αὐτῷ πάντα ὅσα ἦν ἐν τῇ καρδίᾳ αὐτῆς.
\VS{3}Καὶ ἀπήγγειλεν αὐτῇ Σαλωμὼν πάντας τοὺς λόγους αὐτῆς· οὐκ ἦν λόγος παρεωραμένος παρὰ τοῦ βασιλέως, ὃν οὐκ ἀπήγγειλεν αὐτῇ.
\VS{4}Καὶ εἶδε βασίλισσα Σαβὰ πᾶσαν τὴν φρόνησιν Σαλωμὼν, καὶ τὸν οἶκον ὃν ᾠκοδόμησε,
\VS{5}καὶ τὰ βρώματα Σαλωμὼν, καὶ τὴν καθέδραν παίδων αὐτοῦ, καὶ τὴν στάσιν λειτουργῶν αὐτοῦ, καὶ τὸν ἱματισμὸν αὐτοῦ, καὶ τοὺς οἰνοχόους αὐτοῦ, καὶ τὴν ὁλοκαύτωσιν αὐτοῦ ἣν ἀνέφερεν ἐν οἴκῳ Κυρίου, καὶ ἐξ ἑαυτῆς ἐγένετο·
\VS{6}Καὶ εἶπε πρὸς τὸν βασιλέα Σαλωμὼν, ἀληθινὸς ὁ λόγος ὃν ἤκουσα ἐν τῇ γῇ μου περὶ τοῦ λόγου σου καὶ περὶ τῆς φρονήσεώς σου.
\VS{7}Καὶ οὐκ ἐπίστευσα τοῖς λαλοῦσί μοι, ἕως ὅτου παρεγενόμην καὶ ἑωράκασιν οἱ ὀφθαλμοί μου· καὶ ἰδοὺ οὐκ εἰσὶ τὸ ἥμισυ καθὼς ἀπήγγειλάν μοι· προστέθεικας ἀγαθὰ πρὸς αὐτὰ ἐπὶ πᾶσαν τὴν ἀκοὴν ἣν ἤκουσα ἐν τῇ γῇ μου.
\VS{8}Μακάριαι αἱ γυναῖκές σου, μακάριοι οἱ παῖδές σου οὗτοι οἱ παρεστηκότες ἐνώπιόν σου διόλου, οἱ ἀκούοντες πᾶσαν τὴν φρόνησίν σου.
\VS{9}Γένοιτο Κύριος ὁ Θεός σου εὐλογημένος, ὃς ἠθέλησεν ἐν σοὶ δοῦναί σε ἐπὶ θρόνου Ἰσραὴλ, διὰ τὸ ἀγαπᾷν Κύριον τὸν Ἰσραὴλ στῆσαι εἰς τὸν αἰῶνα· καὶ ἔθετό σε βασιλέα ἐπʼ αὐτοὺς, τοῦ ποιεῖν κρίμα ἐν δικαιοσύνῃ καὶ ἐν κρίμασιν αὐτῶν.
\par }{\PP \VS{10}Καὶ ἔδωκε τῷ Σαλωμὼν ἑκατὸν εἴκοσι τάλαντα χρυσίου, καὶ ἡδύσματα πολλὰ σφόδρα, καὶ λίθον τίμιον· οὐκ ἐληλύθει κατὰ τὰ ἡδύσματα ἐκεῖνα ἔτι εἰς πλῆθος, ἃ ἔδωκε βασίλισσα Σαβὰ τῷ βασιλεῖ Σαλωμών.
\par }{\PP \VS{11}Καὶ ἡ ναῦς Χιρὰμ ἡ αἴρουσα τὸ χρυσίον ἐκ Σουφὶρ, ἤνεγκε ξύλα πελεκητὰ πολλὰ σφόδρα καὶ λίθον τίμιον.
\VS{12}Καὶ ἐποίησεν ὁ βασιλεὺς τὰ ξύλα τὰ πελεκητὰ ὑποστηρίγματα τοῦ οἴκου Κυρίου καὶ τοῦ οἴκου τοῦ βασιλέως, καὶ νάβλας καὶ κινύρας τοῖς ᾠδοῖς· οὐκ ἐληλύθει τοιαῦτα ξύλα ἀπελέκητα ἐπὶ τῆς γῆς, οὐδὲ ὤφθησάν που ἕως τῆς ἡμέρας ταύτης.
\VS{13}Καὶ ὁ βασιλεὺς Σαλωμὼν ἔδωκε τῇ βασιλίσσῃ Σαβὰ πάντα ὅσα ἠθέλησεν, ὅσα ᾐτήσατο, ἐκτὸς πάντων ὧν ἐδεδώκει αὐτῇ διὰ χειρὸς τοῦ βασιλέως Σαλωμών· καὶ ἀπεστράφη, καὶ ἦλθεν εἰς τὴν γῆν αὐτῆς αὐτὴ, καὶ πάντες οἱ παῖδες αὐτῆς.
\par }{\PP \VS{14}Καὶ ἦν ὁ σταθμὸς τοῦ χρυσίου τοῦ ἐληλυθότος τῷ Σαλωμὼν ἐν ἐνιαυτῷ ἑνὶ, ἑξακόσια καὶ ἑξηκονταὲξ τάλαντα χρυσίου,
\VS{15}χωρὶς τῶν φόρων τῶν ὑποτεταγμένων καὶ τῶν ἐμπόρων καὶ πάντων τῶν βασιλέων τοῦ πέραν καὶ τῶν σατραπῶν τῆς γῆς.
\par }{\PP \VS{16}Καὶ ἐποίησε Σαλωμὼν τριακόσια δόρατα χρυσᾶ ἐλατά· τριακόσιοι χρυσοῖ ἐπῆσαν ἐπὶ τὸ δόρυ τὸ ἕν·
\VS{17}Καὶ τριακόσια ὅπλα χρυσᾶ ἐλατά· καὶ τρεῖς μναῖ ἐνῆσαν χρυσοῦ εἰς τὸ ὅπλον τὸ ἕν· καὶ ἔδωκεν αὐτὰ ὁ βασιλεὺς εἰς οἶκον δρυμοῦ τοῦ Λιβάνου.
\par }{\PP \VS{18}Καὶ ἐποίησεν ὁ βασιλεὺς θρόνον ἐλεφάντινον μέγαν, καὶ περιεχρύσωσεν αὐτὸν χρυσίῳ δοκίμῳ.
\VS{19}Ἓξ ἀναβαθμοὶ τῷ θρόνῳ, καὶ προτομαὶ μόσχων τῷ θρόνῳ ἐκ τῶν ὀπίσω αὐτοῦ, καὶ χεῖρες ἔνθεν καὶ ἔνθεν ἐπὶ τοῦ τόπου τῆς καθέδρας, καὶ δύο λέοντες ἑστηκότες παρὰ τὰς χεῖρας,
\VS{20}καὶ δώδεκα λέοντες ἑστῶτες ἐκεῖ ἐπὶ τῶν ἓξ ἀναβαθμῶν ἔνθεν καὶ ἔνθεν· οὐ γέγονεν οὕτως πάσῃ βασιλείᾳ.
\VS{21}Καὶ πάντα τὰ σκεύη τὰ ὑπὸ τοῦ Σαλωμὼν γεγονότα χρυσᾶ, καὶ λουτῆρες χρυσοῖ, καὶ πάντα τὰ σκεύη οἴκου δρυμοῦ τοῦ Λιβάνου χρυσίῳ συγκεκλεισμένα. οὐκ ἦν ἀργύριον, ὅτι οὐκ ἦν λογιζόμενον ἐν ταῖς ἡμέραις Σαλωμών·
\VS{22}Ὅτι ναῦς Θαρσὶς τῷ βασιλεῖ Σαλωμὼν ἐν τῇ θαλάσσῃ μετὰ τῶν νηῶν Χιράμ· μία διὰ τριῶν ἐτῶν ἤρχετο τῷ βασιλεῖ ναῦς ἐκ Θαρσὶς χρυσίου καὶ ἀργυρίου καὶ λίθων τορευτῶν καὶ πελεκητῶν.
\par }{\PP \VS{22a}Αὕτη ἦν ἡ πραγματεία τῆς προνομῆς ἧς ἀνήνεγκαν ὁ βασιλεὺς Σαλωμὼν οἰκοδομῆσαι τὸν οἶκον Κυρίου, καὶ τὸν οἶκον τοῦ βασιλέως, καὶ τὸ τεῖχος Ἱερουσαλὴμ, καὶ τὴν ἄκραν, τοῦ περιφράξαι τὸν φραγμὸν τῆς πόλεως Δαυὶδ, καὶ τὴν Ἀσσοὺρ, καὶ τὴν Μαγδὰν, καὶ τὴν Γαζὲρ, καὶ τὴν Βαιθωρὼν τὴν ἀνωτέρω, καὶ τὴν Ἰεθερμὰθ, καὶ πάσας τὰς πόλεις τῶν ἁρμάτων, καὶ πάσας τὰς πόλεις τῶν ἱππέων, καὶ τὴν πραγματείαν Σαλωμὼν, ἣν ἐπραγματεύσατο οἰκοδομῆσαι ἐν Ἱερουσαλὴμ καὶ ἐν πάσῃ τῇ γῇ, τοῦ μὴ κατάρξαι αὐτοῦ
\VS{22b}πάντα τὸν λαὸν τὸν ὑπολελειμμένον ὑπὸ τοῦ Χετταίου καὶ τοῦ Ἀμοῤῥαίου καὶ τοῦ Φερεζαίου καὶ τοῦ Χαναναίου καὶ τοῦ Εὐαίου καὶ τοῦ Ἰεβουσαίου καὶ τοῦ Γεργεσαίου, τῶν μὴ ἐκ τῶν υἱῶν Ἰσραὴλ ὄντων, τὰ τέκνα αὐτῶν τὰ ὑπολελειμμένα μετʼ αὐτοῦ ἐν τῇ γῇ, οὓς οὐκ ἐδύναντο οἱ υἱοὶ Ἰσραὴλ ἐξολοθρεῦσαι αὐτοὺς, καὶ ἀνήγαγεν αὐτοὺς Σαλωμὼν εἰς φόρον ἕως τῆς ἡμέρας ταύτης·
\VS{22c}καὶ ἐκ τῶν υἱῶν Ἰσραὴλ οὐκ ἔδωκε Σαλωμὼν πρᾶγμα, ὅτι αὐτοὶ ἦσαν ἄνδρες οἱ πολεμισταὶ, καὶ παῖδες αὐτοῦ καὶ ἄρχοντες καὶ τρισσοὶ αὐτοῦ, καὶ ἄρχοντες τῶν ἁρμάτων αὐτοῦ, καὶ ἱππεῖς αὐτοῦ.
\par }{\PP \VS{23}Καὶ ἐμεγαλύνθη Σαλωμὼν ὑπὲρ πάντας τοὺς βασιλεῖς τῆς γῆς πλούτῳ καὶ φρονήσει.
\VS{24}Καὶ πάντες βασιλεῖς τῆς γῆς ἐζήτουν τὸ πρόσωπον Σαλωμὼν, τοῦ ἀκοῦσαι τῆς φρονήσεως αὐτοῦ ἧς ἔδωκε Κύριος τῇ καρδίᾳ αὐτοῦ.
\VS{25}Καὶ αὐτοὶ ἔφερον ἕκαστος τὰ δῶρα, σκεύη χρυσᾶ, καὶ ἱματισμὸν, στακτὴν, καὶ ἡδύσματα, καὶ ἵππους, καὶ ἡμιόνους τὸ κατʼ ἐνιαυτὸν ἐνιαυτῷ.
\VS{26}Καὶ ἦσαν τῷ Σαλωμὼν τέσσαρες χιλιάδες θήλειαι ἵπποι εἰς ἅρματα, καὶ δώδεκα χιλιάδες ἱππέων· καὶ ἔθετο αὐτὰς ἐν ταῖς πόλεσι τῶν ἁρμάτων καὶ μετὰ τοῦ βασιλέως ἐν Ἱερουσαλήμ·
\VS{26a}καὶ ἦν ἡγούμενος πάντων τῶν βασιλέων ἀπὸ τοῦ ποταμοῦ καὶ ἕως γῆς ἀλλοφύλων καὶ ἕως ὁρίων Αἰγύπτου.
\par }{\PP \VS{27}Καὶ ἔδωκεν ὁ βασιλεὺς τὸ χρυσίον καὶ τὸ ἀργύριον ἐν Ἱερουσαλὴμ ὡς λίθους, καὶ τὰς κέδρους ἔδωκεν ὡς συκαμίνους τὰς ἐν τῇ πεδινῇ εἰς πλῆθος.
\VS{28}Καὶ ἡ ἔξοδος Σαλωμῶν τῶν ἱππέων καὶ ἐξ Αἰγύπτου, καὶ ἐκ Θεκουὲ ἔμποροι τοῦ βασιλέως· καὶ ἐλάμβανον ἐκ Θεκουὲ ἐν ἀλλάγματι.
\VS{29}Καὶ ἀνέβαινεν ἡ ἔξοδος ἐξ Αἰγύπτου ἅρμα ἀντὶ ἑκατὸν ἀργυρίου, καὶ ἵππος ἀντὶ πεντήκοντα ἀργυρίου· καὶ οὕτως πᾶσι τοῖς βασιλεῦσι Χεττιῒν, καὶ βασιλεῦσι Συρίας κατὰ θάλασσαν ἐξεπορεύοντο.

\par }\Chap{11}{\PP \VerseOne{1}Καὶ ὁ βασιλεὺς Σαλωμὼν ἦν φιλογύνης. Καὶ ἦσαν αὐτῷ γυναῖκες ἄρχουσαι ἑπτακόσιαι, καὶ παλλακαὶ τριακόσιαι. Καὶ ἔλαβε γυναῖκας ἀλλοτρίας, καὶ τὴν θυγατέρα Φαραὼ, Μωαβίτιδας, Ἀμμανίτιδας, Σύρας, καὶ Ἰδουμαίας, Χετταίας, καὶ Ἀμοῤῥαίας,
\VS{2}ἐκ τῶν ἐθνῶν ὧν ἀπεῖπε Κύριος τοῖς υἱοῖς Ἰσραὴλ, οὐκ εἰσελεύσεσθε εἰς αὐτοὺς, καὶ αὐτοὶ οὐκ εἰσελεύσονται εἰς ὑμᾶς, μὴ ἐκκλίνωσι τὰς καρδίας ὑμῶν ὀπίσω εἰδώλων αὐτῶν· εἰς αὐτοὺς ἐκολλήθη Σαλωμὼν τοῦ ἀγαπῆσαι.
\VS{4}Καὶ ἐγενήθη ἐν καιρῷ γήρους Σαλωμὼν, καὶ οὖκ ἦν ἡ καρδία αὐτοῦ τελεία μετὰ Κυρίου Θεοῦ αὐτοῦ, καθὼς ἡ καρδία Δαυὶδ τοῦ πατρὸς αὐτοῦ. Καὶ ἐξέκλιναν γυναῖκες αἱ ἀλλότριαι τὴν καρδίαν αὐτοῦ ὀπίσω θεῶν αὐτῶν.
\VS{5}Τότε ᾠκοδόμησε Σαλωμὼν ὑψηλὸν τῷ Χαμὼς εἰδώλῳ Μωὰβ, καὶ τῷ βασιλεῖ αὐτῶν εἰδώλῳ υἱῶν Ἀμμὼν,
\VS{6}καὶ τῇ Ἀστάρτῃ βδελύγματι Σιδωνίων.
\VS{7}Καὶ οὕτως ἐποίησε πάσαις ταῖς γυναιξὶν αὐτοῦ ταῖς ἀλλοτρίαις, αἳ ἐθυμίων καὶ ἔθυον τοῖς εἰδώλοις αὐτῶν,
\VS{8}καὶ ἐποίησε Σαλωμὼν τὸ πονηρὸν ἐνώπιον Κυρίου· οὐκ ἐπορεύθη ὀπίσω Κυρίου, ὡς Δαυὶδ ὁ πατὴρ αὐτοῦ.
\par }{\PP \VS{9}Καὶ ὠργίσθη Κύριος ἐπὶ Σαλωμὼν, ὅτι ἐξέκλινε καρδίαν αὐτοῦ ἀπὸ Κυρίου Θεοῦ Ἰσραὴλ, τοῦ ὀφθέντος αὐτῷ δὶς,
\VS{10}καὶ ἐντειλαμένου αὐτῷ ὑπὲρ τοῦ λόγου τούτου, τὸ παράπαν μὴ πορευθῆναι ὀπίσω θεῶν ἑτέρων, καὶ φυλάξασθαι ποιῆσαι ἃ ἐνετείλατο αὐτῷ Κύριος ὁ Θεός· οὐδʼ ἦν ἡ καρδία αὐτοῦ τελεία μετὰ Κυρίου, κατὰ τὴν καρδίαν Δαυὶδ τοῦ πατρὸς αὐτοῦ.
\VS{11}Καὶ εἶπε Κύριος πρὸς Σαλωμὼν, ἀνθʼ ὧν ἐγένετο ταῦτα μετὰ σοῦ, καὶ οὐκ ἐφύλαξας τὰς ἐντολάς μου καὶ τὰ προστάγματά μου ἃ ἐνετειλάμην σοι, διαῤῥήσσων διαῤῥήξω τὴν βασιλείαν σου ἐκ χειρός σου, καὶ δώσω αὐτὴν τῷ δούλῳ σου.
\VS{12}Πλὴν ἐν ταῖς ἡμέραις σου οὐ ποιήσω αὐτὰ διὰ Δαυὶδ τὸν πατέρα σου· ἐκ χειρὸς υἱοῦ σου λήψομαι αὐτήν.
\VS{13}Πλὴν ὅλην τὴν βασιλείαν οὐ μὴ λάβω· σκῆπτρον ἓν δώσω τῷ υἱῷ σου διὰ Δαυὶδ τὸν δοῦλόν μου, καὶ διὰ Ἱερουσαλὴμ τὴν πόλιν ἣν ἐξελεξάμην.
\par }{\PP \VS{14}Καὶ ἤγειρε Κύριος σατὰν τῷ Σαλωμὼν τὸν Ἄδερ τὸν Ἰδουμαῖον, καὶ τὸν Ἐσρὼμ υἱὸν Ἐλιαδαὲ τὸν ἐν Ρααμὰ, Ἀδαδέζερ βασιλέα Σουβὰ κύριον αὐτοῦ· καὶ συνηθροίσθησαν ἐπʼ αὐτὸν ἄνδρες, καὶ ἦν ἄρχων συστρέμματος, καὶ προκατελάβετο τὴν Δαμασέκ· καὶ ἦσαν σατὰν τῷ Ἰσραὴλ πάσας τὰς ἡμέρας Σαλωμών· καὶ Ἄδερ ὁ Ἰδουμαῖος ἐκ τοῦ σπέρματος τῆς βασιλείας ἐν Ἰδουμαίᾳ.
\VS{15}Καὶ ἐγένετο ἐν τῷ ἐξολοθρεῦσαι Δαυὶδ τὸν Ἐδὼμ ἐν τῷ πορευθῆναι Ἰωὰβ ἄρχοντα τῆς στρατιᾶς θάπτειν τοὺς τραυματίας, καὶ ἔκοψαν πᾶν ἀρσενικὸν ἐν τῇ Ἰδουμαίᾳ·
\VS{16}ὅτι ἓξ μῆνας ἐνεκάθητο ἐκεῖ Ἰωὰβ καὶ πᾶς Ἰσραὴλ ἐν τῇ Ἰδουμαίᾳ, ἕως ὅτου ἐξωλόθρευσε πᾶν ἀρσενικὸν ἐν τῇ Ἰδουμαίᾳ·
\VS{17}Καὶ ἀπέδρα Ἄδερ αὐτὸς καὶ πάντες ἄνδρες Ἰδουμαῖοι τῶν παίδων τοῦ πατρὸς αὐτοῦ μετʼ αὐτοῦ, καὶ εἰσῆλθον εἰς Αἴγυπτον· καὶ Ἄδερ παιδάριον μικρόν.
\VS{18}Καὶ ἀνίστανται ἄνδρες ἐκ τῆς πόλεως Μαδιὰμ, καὶ ἄρχονται εἰς Φαρὰν, καὶ λαμβάνουσιν ἄνδρας μεθʼ αὑτῶν, καὶ ἔρχονται πρὸς Φαραὼ βασιλέα Αἰγύπτου· καὶ εἰσῆλθεν Ἄδερ πρὸς Φαραὼ, καὶ ἔδωκεν αὐτῷ οἶκον, καὶ ἄρτους διέταξεν αὐτῷ.
\VS{19}Καὶ εὗρεν Ἄδερ χάριν ἐναντίον Φαραὼ σφόδρα, καὶ ἔδωκεν αὐτῷ γυναῖκα ἀδελφὴν τῆς γυναικὸς αὐτοῦ, ἀδελφὴν Θεκεμίνας μείζω. Καὶ ἔτεκεν αὐτῷ ἡ ἀδελφὴ Θεκεμίνας τῷ Ἄδερ τὸν Γανηβὰθ υἱὸν αὐτῆς·
\VS{20}καὶ ἐξέθρεψεν αὐτὸν Θεκεμίνα ἐν μέσῳ υἱῶν Φαραώ· καὶ ἦν Γανηβὰθ ἐν μέσῳ υἱῶν Φαραώ.
\par }{\PP \VS{21}Καὶ Ἄδερ ἤκουσεν ἐν Αἰγύπτῳ ὅτι κεκοίμηται Δαυὶδ μετὰ τῶν πατέρων αὐτοῦ, καὶ ὅτι τέθνηκεν Ἰωὰβ ὁ ἄρχων τῆς στρατιᾶς, καὶ εἶπεν Ἄδερ πρὸς Φαραὼ, ἐξαπόστειλόν με, καὶ ἀποστρέψω εἰς τὴν γῆν μου.
\VS{22}Καὶ εἶπε Φαραὼ τῷ Ἄδερ, τίνι σὺ ἐλαττονῇ μετʼ ἐμοῦ; καὶ ἰδοὺ σὺ ζητεῖς ἀπελθεῖν εἰς τὴν γῆν σου; καὶ εἶπεν αὐτῷ Ἄδερ, ὅτι ἐξαποστέλλων ἐξαποστελεῖς με· καὶ ἀνέστρεψεν Ἄδερ εἰς τὴν γῆν αὐτοῦ·
\VS{25}αὕτη ἡ κακία ἣν ἐποίησεν Ἄδερ· καὶ ἐβαρυθύμησεν Ἰσραὴλ, καὶ ἐβασίλευσεν ἐν τῇ Ἐδώμ.
\par }{\PP \VS{26}Καὶ Ἱεροβοὰμ υἱὸς Ναβὰτ ὁ Ἐφραθὶ ἐκ τῆς Σαριρὰ, υἱὸς γυναικὸς χήρας, δοῦλος Σαλωμών.
\VS{27}Καὶ τοῦτο τὸ πρᾶγμα ὡς ἐπῇρατο χεῖρας ἐπὶ βασιλέα Σαλωμών· καὶ ὁ βασιλεὺς Σαλωμὼν ᾠκοδόμησε τὴν ἄκραν, συνέκλεισε τὸν φραγμὸν τῆς πόλεως Δαυὶδ τοῦ πατρὸς αὐτοῦ.
\VS{28}Καὶ ὁ ἄνθρωπος Ἱεροβοὰμ ἰσχυρὸς δυνάμει· καὶ εἶδε Σαλωμὼν τὸ παιδάριον ὅτι ἀνὴρ ἔργων ἐστὶ, καὶ κατέστησεν αὐτὸν ἐπὶ τὰς ἄρσεις οἴκου Ἰωσήφ.
\par }{\PP \VS{29}Καὶ ἐγενήθη ἐν τῷ καιρῷ ἐκείνῳ, καὶ Ἱεροβοὰμ ἐξῆλθεν ἐξ Ἱερουσαλὴμ, καὶ εὗρεν αὐτὸν Ἀχιὰ ὁ Σηλωνίτης ὁ προφήτης ἐν τῇ ὁδῷ, καὶ ἀπέστησεν αὐτὸν ἐκ τῆς ὁδοῦ· καὶ Ἀχιὰ περιβεβλημένος ἱματίῳ καινῷ, καὶ ἀμφότεροι μόνοι ἐν τῷ πεδίῳ.
\VS{30}Καὶ ἐπελάβετο Ἀχιὰ τοῦ ἱματίου αὐτοῦ τοῦ καινοῦ τοῦ ἐπʼ αὐτῷ, καὶ διέῤῥηξεν αὐτὸ δώδεκα ῥήγματα,
\VS{31}καὶ εἶπε τῷ Ἱεροβοὰμ, λάβε σεαυτῷ δέκα ῥήγματα, ὅτι τάδε λέγει Κύριος ὁ Θεὸς Ἰσραὴλ, ἰδοὺ ἐγὼ ῥήσσω τὴν βασιλείαν ἐκ χειρὸς Σαλωμὼν, καὶ δώσω σοι δέκα σκῆπτρα.
\VS{32}Καὶ δύο σκῆπτρα ἔσονται αὐτῷ διὰ τὸν δοῦλόν μου Δαυὶδ, καὶ διὰ Ἱερουσαλὴμ τὴν πόλιν ἣν ἐξελεξάμην ἐν αὐτῇ ἐκ πασῶν φυλῶν Ἰσραήλ.
\VS{33}Ἀνθʼ ὧν ἐγκατέλιπέ με, καὶ ἐποίησε τῇ Ἀστάρτῃ βδελύγματι Σιδωνίων, καὶ τῷ Χαμὼς, καὶ τοῖς εἰδώλοις Μωὰβ, καὶ τῷ βασιλεῖ αὐτῶν προσοχθίσματι υἱῶν Ἀμμὼν, καὶ οὐκ ἐπορεύθη ἐν ταῖς ὁδοῖς μου τοῦ ποιῆσαι τὸ εὐθὲς ἐνώπιον ἐμοῦ, ὡς Δαυὶδ ὁ πατὴρ αὐτοῦ.
\VS{34}Καὶ οὐ μὴ λάβω τὴν βασιλείαν ὅλην ἐκ χειρὸς αὐτοῦ, διότι ἀντιτασσόμενος ἀντιτάξομαι αὐτῳ πάσας τὰς ἡμέρας τῆς ζωῆς αὐτοῦ, διὰ τὸν Δαυὶδ τὸν δοῦλόν μου ὃν ἐξελεξάμην αὐτόν.
\VS{35}Καὶ λήψομαι τὴν βασιλείαν ἐκ χειρὸς τοῦ υἱοῦ αὐτοῦ, καὶ δώσω σοι τὰ δέκα σκῆπτρα.
\VS{36}Τῷ δὲ υἱῷ αὐτοῦ δώσω τὰ δύο σκῆπτρα, ὅπως ᾖ θέσις τῷ δούλῳ μου Δαυὶδ πάσας τὰς ἡμέρας ἐνώπιον ἐμοῦ ἐν Ἱερουσαλὴμ τῇ πόλει, ἣν ἐξελεξάμην ἐμαυτῷ τοῦ θέσθαι τὸ ὄνομά μου ἐκεῖ.
\VS{37}Καὶ σὲ λήψομαι, καὶ βασιλεύσεις ἐν οἷς ἐπιθυμεῖ ἡ ψυχή σου, καὶ σὺ ἔσῃ βασιλεὺς ἐπὶ τὸν Ἰσραήλ.
\VS{38}Καὶ ἔσται ἐὰν φυλάξῃς πάντα ὅσα ἂν ἐντείλωμαί σοι, καὶ πορευθῇς ἐν ταῖς ὁδοῖς μου, καὶ ποιήσῃς τὸ εὐθὲς ἐνώπιον ἐμοῦ, τοῦ φυλάξασθαι τὰ προστάγματά μου καὶ τὰς ἐντολάς μου, καθὼς ἐποίησε Δαυὶδ ὁ δοῦλός μου, καὶ ἔσομαι μετὰ σοῦ καὶ οἰκοδομήσω σοι οἶκον πιστὸν, καθὼς ᾠκοδόμησα τῷ Δαυίδ.
\par }{\PP \VS{40}Καὶ ἐζήτησε Σαλωμὼν θανατῶσαι τὸν Ἱεροβοάμ· καὶ ἀνέστη καὶ ἀπέδρα εἰς Αἴγυπτον πρὸς Σουσακὶμ βασιλέα Αἰγύπτου, καὶ ἦν ἐν Αἰγύπτῳ ἕως οὗ ἀπέθανε Σαλωμών.
\par }{\PP \VS{41}Καὶ τὰ λοιπὰ τῶν λόγων Σαλωμὼν, καὶ πάντα ὅσα ἐποίησε, καὶ πᾶσαν τὴν φρόνησιν αὐτοῦ, οὐκ ἰδοὺ ταῦτα γέγραπται ἐν βιβλίῳ ῥημάτων Σαλωμών;
\VS{42}Καὶ αἱ ἡμέραι ἃς ἐβασίλευε Σαλωμὼν ἐν Ἱερουσαλὴμ ἐπὶ πάντα Ἰσραὴλ τεσσαράκοντα ἔτη.
\VS{43}Καὶ ἐκοιμήθη Σαλωμὼν μετὰ τῶν πατέρων αὐτοῦ, καὶ ἔθαψαν αὐτὸν ἐν πόλει Δαυὶδ τοῦ πατρὸς αὐτοῦ· καὶ ἐγενήθη ὡς ἤκουσεν Ἱεροβοὰμ υἱὸς Ναβὰτ, καὶ αὐτοῦ ἔτι ὄντος ἐν Αἰγύπτῳ ὡς ἔφυγεν ἐκ προσώπου Σαλωμὼν καὶ ἐκάθητο ἐν Αἰγύπτῳ, κατευθύνει καὶ ἔρχεται εἰς τὴν πόλιν αὐτοῦ εἰς τὴν γῆν Σαριρὰ τὴν ἐν ὄρει Ἐφραίμ. Καὶ ὁ βασιλεὺς Σαλωμὼν ἐκοιμήθη μετὰ τῶν πατέρων αὐτοῦ, καὶ ἐβασίλευσε Ῥοβοὰμ ὁ υἱὸς αὐτοῦ ἀντʼ αὐτοῦ.

\par }\Chap{12}{\PP \VerseOne{1}Καὶ πορεύεται βασιλεὺς Ῥοβοὰμ εἰς Σίκιμα, ὅτι εἰς Σίκιμα ἤρχοντο πᾶς Ἰσραὴλ βασιλεῦσαι αὐτόν.
\VS{3}Καὶ ἐλάλησεν ὁ λαὸς πρὸς τὸν βασιλέα Ῥοβοὰμ, λέγοντες,
\VS{4}Ὁ πατήρ σου ἐβάρυνε τὸν κλοιὸν ἡμῶν, καὶ σὺ νῦν κούφισον ἀπὸ τῆς δουλείας τοῦ πατρός σου τῆς σκληρᾶς, καὶ ἀπὸ τοῦ κλοιοῦ αὐτοῦ τοῦ βαρέως, οὗ ἔδωκεν ἐφʼ ἡμᾶς, καὶ δουλεύσομέν σοι.
\VS{5}Καὶ εἶπε πρὸς αὐτοὺς, ἀπέλθετε ἕως ἡμερῶν τριῶν, καὶ ἀναστρέψατε πρὸς μέ· καὶ ἀπῆλθον.
\par }{\PP \VS{6}Καὶ ἀπήγγειλεν ὁ βασιλεὺς τοῖς πρεσβυτέροις, οἳ ἦσαν παρεστῶτες ἐνώπιον Σαλωμὼν τοῦ πατρὸς αὐτοῦ ἔτι ζῶντος αὐτοῦ, λέγων, πῶς ὑμεῖς βουλεύεσθε καὶ ἀποκριθῶ τῷ λαῷ τούτῳ λόγον;
\VS{7}Καὶ ἐλάλησαν πρὸς αὐτὸν, λέγοντες, εἰ ἐν τῇ ἡμέρᾳ ταύτῃ ἔσῃ δοῦλος τῷ λαῷ τούτῳ, καὶ δουλεύσεις αὐτοῖς, καὶ λαλήσεις πρὸς αὐτοὺς λόγους ἀγαθοὺς, καὶ ἔσονταί σοι δοῦλοι πάσας τὰς ἡμέρας.
\par }{\PP \VS{8}Καὶ ἐγκατέλιπε τὴν βουλὴν τῶν πρεσβυτέρων ἃ συνεβουλεύσαντο αὐτῷ, καὶ συνεβουλεύσατο μετὰ τῶν παιδαρίων τῶν ἐκτραφέντων μετʼ αὐτοῦ τῶν παρεστηκότων πρὸ προσώπου αὐτοῦ.
\VS{9}Καὶ εἶπεν αὐτοῖς, τί ὑμεῖς συμβουλεύετε; καὶ τί ἀποκριθῶ τῷ λαῷ τούτῳ τοῖς λέγουσι πρὸς μὲ, λεγόντων, κούφισον ἀπὸ τοῦ κλοιοῦ οὗ ἔδωκεν ὁ πατήρ σου ἐφʼ ἡμᾶς;
\par }{\PP \VS{10}Καὶ ἐλάλησαν πρὸς αὐτὸν τὰ παιδάρια τὰ ἐκτραφέντα μετʼ αὐτοῦ οἱ παρεστηκότες πρὸ προσώπου αὐτοῦ, λέγοντες, τάδε λαλήσεις τῷ λαῷ τούτῳ τοῖς λαλήσασι πρὸς σὲ, λέγοντες, ὁ πατήρ σου ἐβάρυνε τὸν κλοιὸν ἡμῶν, καὶ σὺ νῦν κούφισον ἀφʼ ἡμῶν· τάδε λαλήσεις πρὸς αὐτοὺς, ἡ μικρότης μου παχυτέρα τῆς ὀσφύος τοῦ πατρός μου.
\VS{11}Καὶ νῦν ὁ πατήρ μου ἐπεσάσσετο ὑμᾶς κλοιῷ βαρεῖ, κᾀλὼ προσθήσω ἐπὶ τὸν κλοιὸν ὑμῶν· ὁ πατήρ μου ἐπαίδευσεν ὑμᾶς ἐν μάστιξιν, ἐγὼ δὲ παιδεύσω ὑμᾶς ἐν σκορπίοις.
\par }{\PP \VS{12}Καὶ παρεγένοντο πᾶς Ἰσραὴλ πρὸς τὸν βασιλέα Ῥοβοὰμ ἐν τῇ ἡμέρᾳ τῇ τρίτῃ, καθότι ἐλάλησεν αὐτοῖς ὁ βασιλεὺς, λέγων, ἀναστράφητε πρὸς μὲ τῇ ἡμέρᾳ τῇ τρίτῃ.
\VS{13}Καὶ ἀπεκρίθη ὁ βασιλεὺς πρὸς τὸν λαὸν σκληρά· καὶ ἐγκατέλιπε Ῥοβοὰμ τὴν βουλὴν τῶν πρεσβυτέρων ἃ συνεβουλεύσαντο αὐτῷ,
\VS{14}καὶ ἐλάλησε πρὸς αὐτοὺς κατὰ τὴν βουλὴν τῶν παιδαρίων, λέγων, ὁ πατήρ μου ἐβάρυνε τὸν κλοιὸν ὑμῶν, κᾀγὼ προσθήσω ἐπὶ τὸν κλοιὸν ὑμῶν· ὁ πατήρ μου ἐπαίδευσεν ὑμᾶς ἐν μάστιξι, κᾀγὼ παιδεύσω ὑμᾶς ἐν σκορπίοις.
\par }{\PP \VS{15}Καὶ οὐκ ἤκουσεν ὁ βασιλεὺς τοῦ λαοῦ, ὅτι ἦν μεταστροφὴ παρὰ Κυρίου, ὅπως στήσῃ τὸ ῥῆμα αὐτοῦ ὃ ἐλάλησεν ἐν χειρὶ Ἀχιὰ τοῦ Σηλωνίτου περὶ Ἱεροβοὰμ υἱοῦ Ναβάτ.
\VS{16}Καὶ εἶδον πᾶς Ἰσραὴλ, ὅτι οὐκ ἤκουσεν ὁ βασιλεὺς αὐτῶν· καὶ ἀπεκρίθη ὁ λαὸς τῷ βασιλεῖ, λέγων, τίς ἡμῖν μερὶς ἐν Δαυίδ; καὶ οὐκ ἔστιν ἡμῖν κληρονομία ἐν υἱῷ Ἰεσσαί· ἀπότρεχε Ἰσραὴλ, εἰς τὰ σκηνώματά σου· νύν βόσκε τὸν οἶκόν σου, Δαυίδ· καὶ ἀπῆλθεν Ἰσραὴλ εἰς τὰ σκηνώματα αὐτοῦ.
\par }{\PP \VS{18}Καὶ ἀπέστειλεν ὁ βασιλεὺς τὸν Ἀδωνιρὰμ τὸν ἐπὶ τοῦ φόρου, καὶ ἐλιθοβόλησαν αὐτὸν ἐν λίθοις καὶ ἀπέθανε· καὶ ὁ βασιλεὺς Ῥοβοὰμ ἔφθασεν ἀναβῆναι τοῦ φυγεῖν εἰς Ἱερουσαλήμ.
\par }{\PP \VS{19}Καὶ ἠθέτησεν Ἰσραὴλ εἰς τὸν οἶκον Δαυὶδ ἕως τῆς ἡμέρας ταύτης.
\VS{20}Καὶ ἐγένετο ὡς ἤκουσε πᾶς Ἰσραὴλ ὅτι ἀνέκαμψεν Ἱεροβοὰμ ἐξ Αἰγύπτου, καὶ ἀπέστειλαν καὶ ἐκάλεσαν αὐτὸν εἰς τὴν συναγωγὴν, καὶ ἐβασίλευσαν αὐτὸν ἐπὶ Ἰσραήλ· καὶ οὐκ ἦν ὀπίσω οἴκου Δαυὶδ πάρεξ σκήπτρου Ἰούδα καὶ Βενιαμὶν μόνοι.
\par }{\PP \VS{21}Καὶ Ῥοβοὰμ εἰσῆλθεν εἰς Ἱερουσαλὴμ, καὶ ἐξεκκλησίασε τὴν συναγωγὴν Ἰούδα καὶ σκῆπτρον Βενιαμὶν ἑκατὸν καὶ εἴκοσι χιλιάδας νεανιῶν ποιούντων πόλεμον, τοῦ πολεμεῖν πρὸς οἶκον Ἰσραὴλ, ἐπιστρέψαι τὴν βασιλείαν Ῥοβοὰμ υἱῷ Σαλωμών.
\VS{22}Καὶ ἐγένετο λόγος Κυρίου πρὸς Σαμαίαν ἄνθρωπον τοῦ Θεοῦ, λέγων,
\VS{23}εἶπον τῷ Ῥοβοὰμ υἱῷ Σαλωμὼν βασιλεῖ Ἰούδα, καὶ πρὸς πάντα οἶκον Ἰούδα καὶ Βενιαμὶν, καὶ τῷ καταλοίπῳ τοῦ λαοῦ, λέγων,
\VS{24}τάδε λέγει Κύριος, οὐκ ἀναβήσεσθε οὐδὲ πολεμήσετε μετὰ τῶν ἀδελφῶν ὑμῶν υἱῶν Ἰσραήλ· ἀποστρεφέτω ἕκαστος εἰς τὸν οἶκον ἑαυτοῦ, ὅτι παρʼ ἐμοῦ γέγονε τὸ ῥῆμα τοῦτο· καὶ ἤκουσαν τοῦ λόγου Κυρίου, καὶ κατέπαυσαν τοῦ πορευθῆναι κατὰ τὸ ῥῆμα Κυρίου.
\par }{\PP \VS{24a}Καὶ ὁ βασιλεὺς Σαλωμὼν κοιμᾶται μετὰ τῶν πατέρων αὐτοῦ, καὶ θάπτεται μετὰ τῶν πατέρων αὐτοῦ, ἐν πόλει Δαυίδ· καὶ ἐβασίλευσε Ῥοβοὰμ υἱὸς αὐτοῦ ἀντʼ αὐτοῦ ἐν Ἱερουσαλὴμ, υἱὸς ὢν ἑκκαίδεκα ἐτῶν ἐν τῷ βασιλεύειν αὐτὸν, καὶ δώδεκα ἔτη ἐβασίλευσεν ἐν Ἱερουσαλήμ· καὶ ὄνομα τῆς μητρὸς αὐτοῦ Ναανὰν, θυγάτηρ Ἄνα υἱοῦ Ναὰς βασιλέως υἱῶν Ἀμμών· καὶ ἐποίησε τὸ πονηρὸν ἐνώπιον Κυρίου, καὶ οὐκ ἐπορεύθη ἐν ὁδῷ Δαυὶδ τοῦ πατρὸς αὐτοῦ.
\par }{\PP \VS{24b}Καὶ ἦν ἄνθρωπος ἐξ ὄρους Ἐφραὶμ δοῦλος τῷ Σαλωμὼν, καὶ ὄνομα αὐτῷ Ἱεροβοὰμ, καὶ ὄνομα τῆς μητρὸς αὐτοῦ Σαριρὰ, γυνὴ πόρνη· καὶ ἔδωκεν αὐτὸν Σαλωμὼν εἰς ἄρχοντα σκυτάλης ἐπὶ ἄρσεις οἴκου Ἰωσήφ· καὶ ᾠκοδόμησε τῷ Σαλωμὼν τὴν Σαριρὰ τὴν ἐν ὄρει Ἐφραίμ· καὶ ἦσαν αὐτῷ τριακόσια ἅρματα ἵππων· οὗτος ᾠκοδόμησε τὴν ἄκραν ἐν ταῖς ἄρσεσιν οἴκου Ἐφραὶμ, οὗτος συνέκλεισε τὴν πόλιν Δαυὶδ, καὶ ἦν ἐπαιρόμενος ἐπὶ τὴν βασιλείαν·
\VS{24c}καὶ ἐζήτει Σαλωμὼν θανατῶσαι αὐτόν· καὶ ἐφοβήθη, καὶ ἀπέδρα αὐτὸς πρὸς Σουσακὶμ βασιλέα Αἰγύπτου, καὶ ἦν μετʼ αὐτοῦ ἕως ἀπέθανε Σαλωμών·
\par }{\PP \VS{24d}Καὶ ἤκουσεν Ἱεροβοὰμ ἐν Αἰγύπτῳ ὅτι τέθνηκε Σαλωμὼν, καὶ ἐλάλησεν εἰς τὰ ὦτα Σουσακὶμ βασιλέως Αἰγύπτου, λέγων, ἐξαπόστειλόν με, καὶ ἀπελεύσομαι ἐγὼ εἰς τὴν γῆν μου· καὶ εἶπεν αὐτῷ Σουσακὶμ, αἴτησαί τι αἴτημα, καὶ δώσω σοι·
\VS{24e}καὶ Σουσακὶμ ἔδωκε τῷ Ἱεροβοὰμ τὴν Ἀνὼ ἀδελφὴν Θεκεμίνας τὴν πρεσβυτέραν τῆς γυναικὸς αὐτοῦ αὐτῷ εἰς γυναῖκα· αὕτη ἦν μεγάλη ἐν μέσῳ τῶν θυγατέρων τοῦ βασιλέως, καὶ ἔτεκε τῷ Ἱεροβοὰμ τὸν Ἀβιὰ υἱὸν αὐτοῦ·
\VS{24f}καὶ εἶπεν Ἱεροβοὰμ πρὸς Σουσακὶμ, ὄντως ἐξαπόστειλόν με, καὶ ἀπελεύσομαι·
\par }{\PP Καὶ ἐξῆλθεν Ἱεροβοὰμ ἐξ Αἰγύπτου, καὶ ἦλθεν εἰς γῆν Σαριρὰ τὴν ἐν ὄρει Ἐφραίμ· καὶ συνάγεται ἐκεῖ πᾶν σκῆπτρον Ἐφραίμ· καὶ ᾠκοδόμησεν ἐκεῖ Ἱεροβοὰμ χάρακα·
\par }{\PP \VS{24g}Καὶ ἠῤῥώστησε τὸ παιδάριον αὐτοῦ ἀῤῥωστίᾳ κραταιᾷ σφόδρα· καὶ ἐπορεύθη Ἱεροβοὰμ ἐρωτῆσαι περὶ τοῦ παιδαρίου· καὶ εἶπε πρὸς Ἀνὼ τὴν γυναῖκα αὐτοῦ, ἀνάστηθι, πορεύου, ἐπερώτησον τὸν Θεὸν περὶ τοῦ παιδαρίου, εἰ ζήσεται ἐκ τῆς ἀῤῥωστίας αὐτοῦ·
\VS{24h}καὶ ἄνθρωπος ἦν ἐν Σηλὼμ, καὶ ὄνομα αὐτῷ Ἀχιὰ, καὶ οὗτος ἦν υἱὸς ἑξήκοντα ἐτῶν, καὶ ῥῆμα Κυρίου μετʼ αὐτοῦ· καὶ εἶπεν Ἱεροβοὰμ πρὸς τὴν γυναῖκα αὐτοῦ, ἀνάστηθι, καὶ λάβε εἰς τὴν χεῖρά σου τῷ ἀνθρώπῳ τοῦ Θεοῦ ἄρτους, καὶ κολλύρια τοῖς τέκνοις αὐτοῦ, καὶ σταφυλὴν, καὶ στάμνον μέλιτος·
\VS{24i}καὶ ἀνέστη ἡ γυνὴ, καὶ ἔλαβεν εἰς τὴν χεῖρα αὐτῆς ἄρτους, καὶ δύο κολλύρια, καὶ σταφυλὴν, καὶ στάμνον μέλιτος τῷ Ἀχιά· καὶ ὁ ἄνθρωπος πρεσβύτερος, καὶ οἱ ὀφθαλμοὶ αὐτοῦ ἠμβλυώπουν τοῦ ἰδεῖν·
\VS{24k}καὶ ἀνέστη ἐκ Σαριρὰ καὶ πορεύεται· καὶ ἐγένετο ἐλθούσης αὐτὴς εἰς τὴν πόλιν πρὸς Ἀχιὰ τὸν Σηλωνίτην, καὶ εἶπεν Ἀχιὰ τῷ παιδαρίῳ αὐτοῦ, ἔξελθε δὴ εἰς ἀπαντὴν Ἀνὼ τῇ γυναικὶ Ἱεροβοὰμ, καὶ ἐρεῖς αὐτῇ, εἴσελθε, καὶ μὴ στῇς, ὅτι τάδε λέγει Κύριος, σκληρὰ ἐγὼ ἐπαποστέλλω ἐπὶ σέ·
\VS{24l}καὶ εἰσῆλθεν Ἀνὼ πρὸς τὸν ἄνθρωπον τοῦ Θεοῦ, καὶ εἴπεν αὐτῇ Ἀχιὰ, ἱνατί ἐνήνοχάς μοι ἄρτους, καὶ σταφυλὴν, καὶ κολλύρια, καὶ στάμνον μέλιτος; τάδε λέγει Κύριος, ἰδοὺ σὺ ἀπελεύσῃ ἀπʼ ἐμοῦ, καὶ ἔσται εἰσελθούσης σου τὴν πόλιν εἰς Σαριρὰ, καὶ τὰ κοράσιά σου ἐξλεύσονταί σοι εἰς συνάντησιν, καὶ ἐροῦσί σοι, τὸ παιδάριον τέθνηκεν·
\VS{24m}ὅτι τάδε λέγει Κύριος, ἰδοὺ ἐγὼ ἐξολοθρεύσω τοῦ Ἱεροβοὰμ οὐροῦντα πρὸς τοῖχον, καὶ ἔσονται οἱ τεθνηκότες τοῦ Ἱεροβοὰμ ἐν τῇ πόλει, καταφάγονται οἱ κύνες, καὶ τὸν τεθνηκότα ἐν τῷ ἀγρῷ καταφάγεται τὰ πετεινὰ τοῦ οὐρανοῦ, καὶ τὸ παιδάριον κόψεται, οὐαὶ κύριε, ὅτι εὑρέθη ἐν αὐτῷ ῥῆμα καλὸν περὶ τοῦ Κυρίου·
\par }{\PP \VS{24n}Καὶ ἀπῆλθεν ἡ γυνὴ, ὡς ἤκουσε· καὶ ἐγένετο ὡς εἰσῆλθεν εἰς τὴν Σαριρὰ, καὶ τὸ παιδάριον ἀπέθανε· καὶ ἐξῆλθεν ἡ κραυγὴ εἰς ἀπαντήν· καὶ ἐπορεύθη Ἱεροβοὰμ εἰς Σίκιμα τὴν ἐν ὄρει Ἐφραὶμ, καὶ συνήθροισεν ἐκεῖ τὰς φυλὰς τοῦ Ἰσραὴλ, καὶ ἀνέβη ἐκεῖ Ῥοβοὰμ υἱὸς Σαλωμών·
\VS{24o}καὶ λόγος Κυρίου ἐγένετο πρὸς Σαμαίαν τὸν Ἐνλαμὶ, λέγων, λάβε σεαυτῷ ἱμάτιον καινὸν τὸ οὐκ εἰσεληλυθὸς εἰς ὕδωρ, καὶ ῥῆξον αὐτὸ δώδεκα ῥήγματα, καὶ δώσεις τῷ Ἱεροβοὰμ, καὶ ἐρεῖς αὐτῷ, τάδε λέγει Κύριος, λάβε σεαυτῷ δέκα ῥήγματα τοῦ περιβαλέσθαι σε· καὶ ἔλαβεν Ἱεροβοάμ· καὶ εἶπε Σαμαίας, τάδε λέγει Κύριος ἐπὶ τὰς δέκα φυλὰς τοῦ Ἰσραήλ.
\par }{\PP \VS{24p}Καὶ εἶπεν ὁ λαὸς πρὸς Ῥοβοὰμ υἱὸν Σαλωμὼν, ὁ πατήρ σου ἐβάρυνε τὸν κλοιὸν αὐτοῦ ἐφʼ ἡμᾶς, καὶ ἐβάρυνε τὰ βρώματα τῆς τραπέζης αὐτοῦ· καὶ νῦν κουφιεῖς ἐφʼ ἡμᾶς, καὶ δουλεύσομέν σοι·
\VS{24q}καὶ εἶπε Ῥοβοὰμ πρὸς τὸν λαὸν, ἔτι τριῶν ἡμερῶν, καὶ ἀποκριθήσομαι ὑμῖν ῥῆμα· καὶ εἶπε Ῥοβοὰμ, εἰσαγάγετέ μοι τοὺς πρεσβυτέρους, καὶ συμβουλεύσομαι μετʼ αὐτῶν τί ἀποκριθῶ τῷ λαῷ ῥῆμα ἐν τῇ ἡμέρᾳ τῇ τρίτῃ· Καὶ ἐλάλησε Ῥοβοὰμ εἰς τὰ ὦτα αὐτῶν, καθὼς ἀπέστειλεν ὁ λαὸς πρὸς αὐτόν· καὶ εἶπον οἱ πρεσβύτεροι τοῦ λαοῦ, οὕτως ἐλάλησε πρὸς σὲ ὁ λαός·
\par }{\PP \VS{24r}Καὶ διεσκέδασε Ῥοβοὰμ τὴν βουλὴν αὐτῶν, καὶ οὐκ ἤρεσεν ἐνώπιον αὐτοῦ· καὶ ἀπέστειλε, καὶ εἰσήγαγε τοὺς συντρόφους αὐτοῦ, καὶ ἐλάλησεν αὐτοῖς, ταῦτα καὶ ταῦτα ἀπέσταλκεν ὁ λαὸς πρὸς μὲ, λέγων· καὶ εἶπαν οἱ σύντροφοι αὐτοῦ, οὕτως λαλήσεις πρὸς τὸν λαὸν, λέγων, ἡ μικρότης μου παχυτέρα ὑπὲρ τὴν ὀσφῦν τοῦ πατρός μου· ὁ πατήρ μου ἐμαστίγου ὑμᾶς μάστιξιν, ἐγὼ δὲ κατάρξω ὑμᾶς ἐν σκορπίοις.
\par }{\PP \VS{24s}Καὶ ἤρεσε τὸ ῥῆμα ἐνώπιον Ῥοβοάμ· καὶ ἀπεκρίθη τῷ λαῷ, καθὼς συνεβούλευσαν αὐτῷ οἱ σύντροφοι αὐτοῦ τὰ παιδάρια·
\VS{24t}καὶ εἶπε πᾶς ὁ λαὸς ὡς ἀνὴρ εἷς ἕκαστος τῷ πλησίον αὐτοῦ, καὶ ἀνέκραξαν ἅπαντες, λέγοντες, οὐ μερὶς ἡμῖν ἐν Δαυὶδ, οὐδὲ κληρονομία ἐν υἱῷ Ἰεσσαί· ἕκαστος εἰς τὰ σκηνώματά σου Ἰσραὴλ, ὅτι ὁ ἄνθρωπος οὗτος οὐκ εἰς ἄρχοντα οὐδὲ εἰς ἡγούμενον·
\VS{24u}καὶ διεσπάρη πᾶς ὁ λαὸς ἐκ Σικίμων, καὶ ἀπῆλθον ἕκαστος εἰς τὸ σκήνωμα αὐτοῦ· Καὶ κατεκράτησε Ῥοβοὰμ, καὶ ἀπῆλθε, καὶ ἀνέβη ἐπὶ τὸ ἅρμα αὐτοῦ, καὶ εἰσῆλθεν εἰς Ἱερουσαλήμ· καὶ πορεύονται ὀπίσω αὐτοῦ πᾶν σκῆπτρον Ἰούδα, καὶ πᾶν σκῆπτρον Βενιαμίν.
\VS{24x}Καὶ ἐγένετο ἐνισταμένου τοῦ ἐνιαυτοῦ, καὶ συνήθροισε Ῥοβοὰμ πάντα ἄνδρα Ἰούδα καὶ Βενιαμὶν, καὶ ἀνέβη τοῦ πολεμεῖν πρὸς Ἱεροβοὰμ εἰς Σίκιμα·
\VS{24y}καὶ ἐγένετο ῥῆμα Κυρίου πρὸς Σαμαίαν ἄνθρωπον τοῦ Θεοῦ, λέγων, εἶπον τῷ Ῥοβοὰμ βασιλεῖ Ἰούδα, καὶ πρὸς πάντα οἶκον Ἰούδα καὶ Βενιαμὶν, καὶ πρὸς τὸ κατάλειμμα τοῦ λαοῦ, λέγων, τάδε λέγει Κύριος, οὐκ ἀναβήσεσθε οὐδὲ πολεμήσετε πρὸς τοὺς ἀδελφοὺς ὑμῶν υἱοὺς Ἰσραὴλ, ἀναστρέφετε ἕκαστος εἰς τὸν οἶκον αὐτοῦ, ὅτι παρʼ ἐμοῦ γέγονε τὸ ῥῆμα τοῦτο·
\VS{24z}καὶ ἤκουσαν τοῦ λόγου Κυρίου, καὶ ἀνέσχον μὴ πορευθῆναι κατὰ τὸ ῥῆμα Κυρίου.
\par }{\PP \VS{25}Καὶ ᾠκοδόμησεν Ἱεροβοὰμ τὴν Σίκιμα τὴν ἐν ὄρει Ἐφραὶμ, καὶ κατῴκει ἐν αὐτῇ, καὶ ἐξῆλθεν ἐκεῖθεν καὶ ᾠκοδόμησε τὴν Φανουήλ.
\VS{26}Καὶ εἶπεν Ἱεροβοὰμ ἐν τῇ καρδίᾳ αὐτοῦ, ἰδοὺ νῦν ἐπιστρέψει ἡ βασιλεία εἰς οἴκον Δαυίδ·
\VS{27}Ἐὰν ἀναβῇ ὁ λαὸς οὗτος ἀναφέρειν θυσίαν ἐν οἴκῳ Κυρίου εἰς Ἱερουσαλὴμ, καὶ ἐπιστραφήσεται καρδία τοῦ λαοῦ πρὸς Κύριον καὶ κύριον αὐτῶν, πρὸς Ῥοβοὰμ βασιλέα Ἰούδα, καὶ ἀποκτενοῦσί με.
\VS{28}Καὶ ἐβουλεύσατο ὁ βασιλεὺς, καὶ ἐπορεύθη, καὶ ἐποίησε δύο δαμάλεις χρυσᾶς, καὶ εἶπε πρὸς τὸν λαὸν, ἱκανούσθω ὑμῖν ἀναβαίνειν εἰς Ἱερουσαλήμ· ἰδοὺ θεοί σου Ἰσραὴλ οἱ ἀναγαγόντες σε ἐκ γῆς Αἰγύπτου.
\VS{29}Καὶ ἔθετο τὴν μίαν ἐν Βαιθὴλ, καὶ τὴν μίαν ἔδωκεν ἐν Δάν.
\VS{30}Καὶ ἐγένετο ὁ λόγος οὗτος εἰς ἁμαρτίαν· καὶ ἐπορεύετο ὁ λαὸς πρὸ προσώπου τῆς μιᾶς ἕως Δὰν, καὶ εἴασαν τὸν οἶκον Κυρίου
\VS{31}Καὶ ἐποίησεν οἴκους ἐφʼ ὑψηλῶν, καὶ ἐποίησεν ἱερεῖς μέρος τι ἐκ τοῦ λαοῦ, οἳ οὐκ ἦσαν ἐκ τῶν υἱῶν Λευί.
\par }{\PP \VS{32}Καὶ ἐποίησεν Ἱεροβοὰμ ἑορτὴν ἐν τῷ μηνὶ τῷ ὀγδόῳ ἐν τῇ πεντεκαιδεκάτῃ ἡμέρᾳ τοῦ μηνὸς κατὰ τὴν ἑορτὴν τὴν ἐν γῇ Ἰούδα, καὶ ἀνέβη ἐπὶ τὸ θυσιαστήριον ὃ ἐποίησεν ἐν Βαιθὴλ τοῦ θύειν ταῖς δαμάλεσιν αἷς ἐποίησε, καὶ παρέστησεν ἐν Βαιθὴλ τοὺς ἱερεῖς τῶν ὑψηλῶν ὧν ἐποίησε.
\VS{33}Καὶ ἀνέβη ἐπὶ τὸ θυσιαστήριον ὃ ἐποίησε, τῇ πεντεκαιδεκάτῃ ἡμέρᾳ ἐν τῷ μηνὶ τῷ ὀγδόῳ ἐν τῇ ἑορτῇ ᾗ ἐπλάσατο ἀπὸ καρδίας αὐτοῦ· καὶ ἐποίησεν ἑορτὴν τοῖς υἱοῖς Ἰσραὴλ, καὶ ἀνέβη ἐπὶ τὸ θυσιαστήριον τοῦ ἐπιθῦσαι.

\par }\Chap{13}{\PP \VerseOne{1}Καὶ ἰδοὺ ἄνθρωπος τοῦ Θεοῦ ἐξ Ἰούδα παρεγένετο ἐν λόγῳ Κυρίου εἰς Βαιθὴλ, καὶ Ἱεροβοὰμ εἱστήκει ἐπὶ τὸ θυσιαστήριον ἐπιθῦσαι.
\VS{2}Καὶ ἐπεκάλεσε πρὸς τὸ θυσιαστήριον ἐν λόγῳ Κυρίου, καὶ εἶπε, θυσιαστήριον, θυσιαστήριον, τάδε λέγει Κύριος, ἰδοὺ υἱὸς τίκτεται τῷ οἴκῳ Δαυὶδ, Ἰωσίας ὄνομα αὐτῷ, καὶ θύσει ἐπὶ σὲ τοὺς ἱερεῖς τῶν ὑψηλῶν τῶν ἐπιθυόντων ἐπὶ σὲ, καὶ ὀστᾶ ἀνθρώπων καύσει ἐπὶ σέ.
\VS{3}Καὶ δώσει ἐν τῇ ἡμέρᾳ ἐκείνῃ τέρας, λέγων, τοῦτο τὸ ῥῆμα ὃ ἐλάλησε Κύριος, λέγων, ἰδοὺ τὸ θυσιαστήριον ῥήγνυται, καὶ ἐκχυθήσεται ἡ πιότης ἡ ἐπʼ αὐτῷ.
\par }{\PP \VS{4}Καὶ ἐγένετο ὡς ἤκουσεν ὁ βασιλεὺς Ἱεροβοὰμ τῶν λόγων τοῦ ἀνθρώπου τοῦ Θεοῦ τοῦ ἐπικαλεσαμένου ἐπὶ τὸ θυσιαστήριον τὸ ἐν Βαιθὴλ, καὶ ἐξέτεινεν ὁ βασιλεὺς τὴν χεῖρα αὐτοῦ ἀπὸ τοῦ θυσιαστηρίου, λέγων, συλλάβετε αὐτόν· καὶ ἰδοὺ ἐξηράνθη ἡ χεὶρ αὐτοῦ, ἣν ἐξέτεινεν ἐπʼ αὐτὸν, καὶ οὐκ ἐδυνήθη ἐπιστρέψαι αὐτὴν πρὸς αὐτόν.
\VS{5}Καὶ τὸ θυσιαστήριον ἐῤῥάγη, καὶ ἐξεχύθη ἡ πιότης ἀπὸ τοῦ θυσιαστηρίου, κατὰ τὸ τέρας ὃ ἔδωκεν ὁ ἄνθρωπος τοῦ Θεοῦ ἐν λόγῳ Κυρίου.
\VS{6}Καὶ εἶπεν ὁ βασιλεὺς Ἱεροβοὰμ τῷ ἀνθρώπῳ τοῦ Θεοῦ, δεήθητι τοῦ πρόσώπου Κυρίου τοῦ Θεοῦ σου, καὶ ἐπιστρεψάτω ἡ χείρ μου πρὸς ἐμέ· καὶ ἐδεήθη ὁ ἄνθρωπος τοῦ Θεοῦ τοῦ προσώπου Κυρίου, καὶ ἐπέστρεψε τὴν χεῖρα τοῦ βασιλέως πρὸς αὐτὸν, καὶ ἐγένετο καθὼς τὸ πρότερον.
\par }{\PP \VS{7}Καὶ ἐλάλησεν ὁ βασιλεὺς πρὸς τὸν ἄνθρωπον τοῦ Θεοῦ, εἴσελθε μετʼ ἐμοῦ εἰς οἶκον, καὶ ἀρίστησον, καὶ δώσω σοι δόμα.
\VS{8}Καὶ εἶπεν ὁ ἄνθρωπος τοῦ Θεοῦ πρὸς τὸν βασιλέα, ἐὰν δῷς μοι τὸ ἥμισυ τοῦ οἴκου σου, οὐκ εἰσελεύσομαι μετὰ σοῦ, οὐδὲ μὴ φάγω ἄρτον, οὐδὲ μὴ πίω ὕδωρ ἐν τῷ τόπῳ τούτῳ·
\VS{9}ὅτι οὕτως ἐνετείλατό μοι Κύριος ἐν λόγῳ, λέγων, μὴ φάγῃς ἄρτον καὶ μὴ πίῃς ὕδωρ καὶ μὴ ἐπιστρέψῃς ἐν τῇ ὁδῷ ᾗ ἐπορεύθης ἐν αὐτῇ.
\VS{10}Καὶ ἀπῆλθεν ἐν ὁδῷ ἄλλῃ, καὶ οὐκ ἀνέστρεψεν ἐν τῇ ὁδῷ ᾗ ἦλθεν ἐν αὐτῇ εἰς Βαιθήλ.
\par }{\PP \VS{11}Καὶ προφήτης εἷς πρεσβύτης κατῴκη ἐν Βαιθὴλ, καὶ ἔρχονται οἱ υἱοὶ αὐτοῦ καὶ διηγήσαντο αὐτῷ πάντα τὰ ἔργα ἃ ἐποίησεν ὁ ἄνθρωπος τοῦ Θεοῦ ἐν τῇ ἡμέρᾳ ἐκείνῃ ἐν Βαιθὴλ, καὶ τοὺς λόγους οὓς ἐλάλησε τῷ βασιλεῖ, καὶ ἐπέστρεψαν τὸ πρόσωπον τοῦ πατρὸς αὐτῶν.
\VS{12}Καὶ ἐλάλησε πρὸς αὐτοὺς ὁ πατὴρ αὐτῶν, λέγων, ποίᾳ ὁδῷ πεπόρευται; καὶ δεικνύουσιν αὐτῷ οἱ υἱοὶ αὐτοῦ τὴν ὁδὸν ἐν ᾗ ἀνῆλθεν ὁ ἄνθρωπος τοῦ Θεοῦ ὁ ἐλθὼν ἐξ Ἰούδα.
\VS{13}Καὶ εἶπε τοῖς υἱοῖς αὐτοῦ, ἐπισάξατέ μοι τὸν ὄνον· καὶ ἐπέσαξαν αὐτῷ τὸν ὄνον, καὶ ἐπέβη ἐπʼ αὐτὸν,
\VS{14}καὶ ἐπορεύθη κατόπισθεν τοῦ ἀνθρώπου τοῦ Θεοῦ, καὶ εὗρεν αὐτὸν καθήμενον ὑπὸ δρῦν, καὶ εἶπεν αὐτῷ, εἰ σὺ εἶ ὁ ἄνθρωπος τοῦ Θεοῦ ὁ ἐληλυθὼς ἐξ Ἰούδα; καὶ εἶπεν αὐτῷ, ἐγώ.
\VS{15}Καὶ εἶπεν αὐτῷ, δεῦρο μετʼ ἐμοῦ, καὶ φάγε ἄρτον.
\VS{16}Καὶ εἶπεν, οὐ μὴ δύνωμαι τοῦ ἐπιστρέψαι μετὰ σοῦ, οὐδὲ μὴ φάγομαι ἄρτον, οὐδὲ πίομαι ὕδωρ ἐν τῷ τόπῳ τούτῳ·
\VS{17}Ὅτι οὕτως ἐντέταλταί μοι ἐν λόγῳ Κύριος, λέγων, μὴ φάγῃς ἄρτον ἐκεῖ καὶ μὴ πίῃς ὕδωρ καὶ μὴ ἐπιστρέψῃς ἐκεῖ ἐν τῇ ὁδῷ ᾗ ἐπορεύθης ἐν αὐτῇ.
\par }{\PP \VS{18}Καὶ εἶπε πρὸς αὐτὸν, κᾀγὼ προφήτης εἰμὶ καθὼς σὺ, καὶ ἄγγελος λελάληκε πρὸς μὲ ἐν ῥήματι Κυρίου, λέγων, ἐπίστρεψον αὐτὸν πρὸς σεαυτὸν εἰς τὸν οἶκόν σου, καὶ φαγέτω ἄρτον, καὶ πιέτω ὕδωρ· καὶ ἐψεύσατο αὐτῷ·
\VS{19}Καὶ ἐπέστρεψεν αὐτὸν, καὶ ἔφαγεν ἄρτον καὶ ἔπιεν ὕδωρ ἐν τῴ οἴκῳ αὐτοῦ.
\par }{\PP \VS{20}Καὶ ἐγένετο αὐτῶν καθημένων ἐπὶ τῆς τραπέζης, καὶ ἐγένετο λόγος Κυρίου πρὸς τὸν προφήτην τὸν ἐπιστρέψαντα αὐτόν·
\VS{21}καὶ εἶπε πρὸς τὸν ἄνθρωπον τοῦ Θεοῦ τὸν ἥκοντα ἐξ Ἰούδα, λέγων, τάδε λέγει Κύριος, ἀνθʼ ὧν παρεπίκρανας τὸ ῥῆμα Κυρίου, καὶ οὐκ ἐφύλαξας τὴν ἐντολὴν ἣν ἐνετείλατό σοι Κύριος ὁ Θεός σου,
\VS{22}καὶ ἐπέστρεψας, καὶ ἔφαγες ἄρτον καὶ ἔπιες ὕδωρ ἐν τῷ τόπῳ τούτῳ ᾧ ἐλάλησε πρὸς σὲ, λέγων, οὐ μὴ φάγῃς ἄρτον καὶ μὴ πίῃς ὕδωρ, οὐ μὴ εἰσέλθῃ τὸ σῶμά σου εἰς τὸν τάφον τῶν πατέρων σου.
\par }{\PP \VS{23}Καὶ ἐγένετο μετὰ τὸ φαγεῖν αὐτὸν ἄρτον καὶ πιεῖν ὕδωρ, καὶ ἐπέσαξεν αὐτῷ τὸν ὄνον, καὶ ἐπέστρεψε, καὶ ἀπῆλθε.
\VS{24}Καὶ εὗρεν αὐτὸν λέων ἐν τῇ ὁδῷ, καὶ ἐθανάτωσεν αὐτόν· καὶ ἦν τὸ σῶμα αὐτοῦ ἐῥῥιμμένον ἐν τῇ ὁδῷ, καὶ ὁ ὄνος εἱστήκει παρʼ αὐτὸ, καὶ ὁ λέων εἱστήκει παρὰ τὸ σῶμα.
\VS{25}Καὶ ἰδοὺ ἄνδρες παραπορευόμενοι καὶ εἶδον τὸ θνησιμαῖον ἐῤῥιμμένον ἐν τῇ ὁδῷ, καὶ ὁ λέων εἱστήκει ἐχόμενα τοῦ θνησιμαίου· καὶ εἰσῆλθον, καὶ ἐλάλησαν ἐν τῇ πόλει οὗ ὁ προφήτης ὁ πρεσβύτης κατῴκει ἐν αὐτῄ.
\VS{26}Καὶ ἤκουσεν ὁ ἐπιστρέψας αὐτὸν ἐκ τῆς ὁδοῦ, καὶ εἶπεν, ὁ ἄνθρωπος τοῦ Θεοῦ οὗτός ἐστιν ὃς παρεπίκρανε τὸ ῥῆμα Κυρίου·
\VS{28}Καὶ ἐπορεύθη καὶ εὗρε τὸ σῶμα αὐτοῦ ἐῤῥιμμένον ἐν τῇ ὁδῷ, καὶ ὁ ὄνος, καὶ ὁ λέων εἱστήκεισαν παρὰ τὸ σῶμα· καὶ οὐκ ἔφαγεν ὁ λέων τὸ σῶμα τοῦ ἀνθρώπου τοῦ Θεοῦ, καὶ οὐ συνέτριψε τὸν ὄνον.
\par }{\PP \VS{29}Καὶ ᾖρεν ὁ προφήτης τὸ σῶμα τοῦ ἀνθρώπου τοῦ Θεοῦ, καὶ ἐπέθηκεν αὐτὸ ἐπὶ τὸν ὄνον, καὶ ἐπέστρεψεν αὐτὸν εἰς τὴν πόλιν
\VS{30}ὁ προφήτης, τοῦ θάψαι αὐτὸν ἐν τῷ τάφῳ ἑαυτοῦ, καὶ ἐκόψαντο αὐτὸν, οὐαὶ ἀδελφέ.
\VS{31}Καὶ ἐγένετο μετὰ τὸ κόψασθαι αὐτὸν, καὶ εἶπε τοῖς υἱοῖς αὐτοῦ, λέγων, ἐὰν ἀποθάνω, θάψατέ με ἐν τῷ τάφῳ τούτῳ οὗ ὁ ἄνθρωπος τοῦ Θεοῦ τέθαπται ἐν αὐτῷ, παρὰ τὰ ὀστᾶ αὐτοῦ θέτε με, ἵνα σωθῶσι τὰ ὀστᾶ μου μετὰ τῶν ὀστῶν αὐτοῦ.
\VS{32}Ὅτι γινόμενον ἔσται τὸ ῥῆμα ὃ ἐλάλησεν ἐν λόγῳ Κυρίου ἐπὶ τὸ θυσιαστήριον ἐν Βαιθὴλ καὶ ἐπὶ τοὺς οἴκους τοὺς ὑψηλοὺς τοὺς ἐν Σαμαρείᾳ.
\par }{\PP \VS{33}Καὶ μετὰ τὸ ῥῆμα τοῦτο οὐκ ἐπέστρεψεν Ἱεροβοὰμ ἀπὸ τῆς κακίας αὐτοῦ, καὶ ἐπέστρεψε καὶ ἐποίησεν ἐκ μέρους τοῦ λαοῦ ἱερεῖς ὑψηλῶν· ὁ βουλόμενος ἐπλήρου τὴν χεῖρα αὐτοῦ, καὶ ἐγένετο ἱερεὺς εἰς τὰ ὑψηλά.
\VS{34}Καὶ ἐγένετο τὸ ῥῆμα τοῦτο εἰς ἁμαρτίαν τῷ οἴκῳ Ἱεροβοὰμ, καὶ εἰς ὄλεθρον, καὶ εἰς ἀφανισμὸν ἀπὸ προσώπου τῆς γῆς.

\par }\Chap{14}{\PP \VS{21}Καὶ Ῥοβοὰμ υἱὸς Σαλωμὼν ἐβασίλευσεν ἐπὶ Ἰούδαν· υἱὸς τεσσαράκοντα καὶ ἑνὸς ἐνιαυτῶν Ῥοβοὰμ ἐν τῷ βασιλεύειν αὐτόν· καὶ ἑπτακαίδεκα ἔτη ἐβασίλευσεν ἐν Ἱερουσαλὴμ τῇ πόλει, ἣν ἐξελέξατο Κύριος θέσθαι τὸ ὄνομα αὐτοῦ ἐκεῖ ἐκ πασῶν φυλῶν τοῦ Ἰσραήλ· καὶ τὸ ὄνομα τῆς μητρὸς αὐτοῦ Νααμὰ ἡ Ἀμμωνίτις.
\VS{22}Καὶ ἐποίησε Ῥοβοὰμ τὸ πονηρὸν ἐνώπιον Κυρίου· καὶ παρεζήλωσεν αὐτὸν ἐν πᾶσιν οἷς ἐποίησαν οἱ πατέρες αὐτῶν ἐν ταῖς ἁμαρτίαις αὐτῶν αἷς ἥμαρτον.
\VS{23}Καὶ ᾠκοδόμησαν ἑαυτοῖς ὑψηλὰ καὶ στήλας καὶ ἄλση ἐπὶ πάντα βουνὸν ὑψηλὸν, καὶ ὑποκάτω παντὸς ξύλου συσκίου.
\VS{24}Καὶ σύνδεσμος ἐγενήθη ἐν τῇ γῇ, καὶ ἐποίησαν ἀπὸ πάντων τῶν βδελυγμάτων τῶν ἐθνῶν ὧν ἐξῇρε Κύριος ἀπὸ προσώπου υἱῶν Ἰσραήλ.
\par }{\PP \VS{25}Καὶ ἐγένετο ἐν τῷ ἐνιαυτῷ τῷ πέμπτῳ βασιλεύοντος Ῥοβοὰμ, ἀνέβη Σουσακεὶμ βασιλεὺς Αἰγύπτου ἐπὶ Ἱερουσαλὴμ,
\VS{26}καὶ ἔλαβε πάντας τοὺς θησαυροὺς οἴκου Κυρίου καὶ τοὺς θησαυροὺς οἴκου τοῦ βασιλέως, καὶ τὰ δόρατα τὰ χρυσᾶ ἃ ἔλαβε Δαυὶδ ἐκ χειρὸς τῶν παίδων Ἀδραζαὰρ βασιλέως Σουβὰ, καὶ εἰσήνεγκεν αὐτὰ εἰς Ἱερουσαλὴμ τὰ πάντα ἃ ἔλαβεν, ὅπλα τὰ χρυσᾶ ὅσα ἐποίησε Σαλωμὼν, καὶ ἀπήνεγκεν αὐτὰ εἰς Αἴγυπτον.
\VS{27}Καὶ ἐποίησε Ῥοβοὰμ ὁ βασιλεὺς ὅπλα χαλκᾶ ἀντʼ αὐτῶν· καὶ ἐπέθεντο ἐπʼ αὐτὸν οἱ ἡγούμενοι τῶν παρατρεχόντων οἱ φυλάσσοντες τὸν πυλῶνα οἴκου βασιλέως.
\VS{28}Καὶ ἐγένετο ὅτε εἰσεπορεύετο ὁ βασιλεὺς εἰς οἶκον Κυρίου, καὶ ᾖρον αὐτὰ οἱ παρατρέχοντες καὶ ἀπηρείδοντο αὐτὰ εἰς τὸ θεὲ τῶν παρατρεχόντων.
\par }{\PP \VS{29}Καὶ τὰ λοιπὰ τῶν λόγων Ῥοβοὰμ καὶ πάντα ἃ ἐποίησεν, οὐκ ἰδοὺ ταῦτα γεγραμμένα ἐν βιβλίῳ λόγων τῶν ἡμερῶν τοῖς βασιλεῦσιν Ἰούδα;
\VS{30}Καὶ πόλεμος ἦν ἀναμέσον Ῥοβοὰμ καὶ ἀναμέσον Ἱεροβοὰμ πάσας τὰς ἡμέρας.
\VS{31}Καὶ ἐκοιμήθη Ῥοβοὰμ μετὰ τῶν πατέρων αὐτοῦ, καὶ θάπτεται μετὰ τῶν πατέρων αὐτοῦ, ἐν πόλει Δαυίδ· καὶ ἐβασίλευσεν Ἀβιοὺ ὁ υἱὸς αὐτοῦ ἀντʼ αὐτοῦ.

\par }\Chap{15}{\PP \VerseOne{1}Καὶ ἐν τῷ ὀκτωκαιδεκάτῳ ἔτει βασιλεύοντος Ἱεροβοὰμ υἱοῦ Ναβὰτ, βασιλεύει Ἀβιοὺ υἱὸς Ῥοβοὰμ ἐπὶ Ἰούδαν·
\VS{2}Καὶ τρία ἔτη ἐβασίλευσεν ἐπὶ Ἱερουσαλήμ· καὶ ὄνομα τῆς μητρὸς αὐτοῦ Μααχὰ, θυγάτηρ Ἀβεσσαλώμ.
\VS{3}Καὶ ἐπορεύθη ἐν ταῖς ἁμαρτίαις τοῦ πατρὸς αὐτοῦ αἷς ἐποίησεν ἐνώπιον αὐτοῦ, καὶ οὐκ ἦν ἡ καρδία αὐτοῦ τελεία μετὰ Κυρίου Θεοῦ αὐτοῦ, ὡς ἡ καρδία τοῦ πατρὸς αὐτοῦ.
\VS{4}Ὅτι διὰ Δαυὶδ ἔδωκεν αὐτῷ Κύριος κατάλειμμα, ἵνα στήσῃ τὰ τέκνα αὐτοῦ μετʼ αὐτὸν, καὶ στήσῃ τὴν Ἱερουσαλὴμ·
\VS{5}ὡς ἐποίησε Δαυὶδ τὸ εὐθὲς ἐνώπιον Κυρίου, οὐκ ἐξέκλινεν ἀπὸ πάντων ὧν ἐνετείλατο αὐτῷ πάσας τὰς ἡμέρας τῆς ζωῆς αὐτοῦ.
\par }{\PP \VS{7}Καὶ τὰ λοιπὰ τῶν λόγων Ἀβιοὺ καὶ πάντα ἃ ἐποίησεν, οὐκ ἰδοὺ ταῦτα γεγραμμένα ἐπὶ βιβλίῳ λόγων τῶν ἡμερῶν τοῖς βασιλεῦσιν Ἰούδα; καὶ πόλεμος ἦν ἀναμέσον Ἀβιοὺ καὶ ἀναμέσον Ἱεροβοάμ.
\VS{8}Καὶ ἐκοιμήθη Ἀβιοὺ μετὰ τῶν πατέρων αὐτοῦ ἐν τῷ εἰκοστῷ καὶ τετάρτῳ ἔτει τοῦ Ἱεροβοὰμ, καὶ θάπτεται μετὰ τῶν πατέρων αὐτοῦ ἐν πόλει Δαυίδ· καὶ βασιλεύει Ἀσὰ υἱὸς αὐτοῦ ἀντʼ αὐτοῦ.
\par }{\PP \VS{9}Ἐν τῷ ἐνιαυτῷ τετάρτῳ καὶ εἰκοστῷ τοῦ Ἱεροβοὰμ βασιλέως Ἰσραὴλ, βασιλεύει Ἀσὰ ἐπὶ Ἰούδαν·
\VS{10}Καὶ τεσσαράκοντα καὶ ἓν ἔτος ἐβασίλευσεν ἐν Ἱερουσαλήμ· καὶ ὄνομα τῆς μητρὸς αὐτοῦ Ἀνὰ, θυγάτηρ Ἀβεσσαλώμ.
\VS{11}Καὶ ἐποίησεν Ἀσὰ τὸ εὐθὲς ἐνώπιον Κυρίου, ὡς Δαυὶδ ὁ πατὴρ αὐτοῦ.
\VS{12}Καὶ ἀφεῖλε τὰς τελετὰς ἀπὸ τῆς γῆς, καὶ ἐξαπέστειλε πάντα τὰ ἐπιτηδεύματα ἃ ἐποίησαν οἱ πατέρες αὐτοῦ.
\VS{13}Καὶ τὴν Ἀνὰ τὴν μητέρα ἑαυτοῦ μετέστησε τοῦ μὴ εἶναι ἡγουμένην, καθὼς ἐποίησε σύνοδον ἐν τῷ ἄλσει αὐτῆς· καὶ ἐξέκοψεν Ἀσὰ τὰς καταδύσεις αὐτῆς, καὶ ἐνέπρησε πυρὶ ἐν τῷ χειμάῤῥῳ τῶν Κέδρων.
\VS{14}Τὰ δὲ ὑψηλὰ οὐκ ἐξῇρε· πλὴν ἡ καρδία Ἀσὰ ἦν τελεία μετὰ Κυρίου πάσας τὰς ἡμέρας αὐτοῦ.
\VS{15}Καὶ εἰσήνεγκε τοὺς κίονας τοῦ πατρὸς αὐτοῦ, καὶ τοὺς κίονας αὐτοῦ εἰσήνεγκεν εἰς τὸν οἶκον Κυρίου ἀργυροῦς καὶ χρυσοῦς, καὶ σκεύη.
\par }{\PP \VS{16}Καὶ πόλεμος ἦν ἀναμέσον Ἀσὰ καὶ ἀναμέσον Βαασὰ βασιλέως Ἰσραὴλ πάσας τὰς ἡμέρας αὐτῶν.
\VS{17}Καὶ ἀνέβη Βαασὰ βασιλεὺς Ἰσραὴλ ἐπὶ Ἰούδαν, καὶ ᾠκοδόμησε τὴν Ῥαμὰ, τοῦ μὴ εἶναι ἐκπορευόμενον καὶ εἰσπορευόμενον τῷ Ἀσὰ βασιλεῖ Ἰούδα.
\par }{\PP \VS{18}Καὶ ἔλαβεν Ἀσὰ σύμπαν τὸ ἀργύριον καὶ τὸ χρυσίον τὸ εὑρεθὲν ἐν τοῖς θησαυροῖς οἴκου Κυρίου καὶ ἐν τοῖς θησαυροῖς τοῦ οἴκου τοῦ βασιλέως, καὶ ἔδωκεν αὐτὰ εἰς χεῖρας παίδων αὐτοῦ· καὶ ἐξαπέστειλεν αὐτοὺς ὁ βασιλεὺς Ἀσὰ πρὸς υἱὸν Ἄδερ υἱὸν Ταβερεμὰ υἱοῦ Ἀζὶν βασιλέως Συρίας τοῦ κατοικοῦντος ἐν Δαμασκῷ, λέγων,
\VS{19}διάθου διαθήκην ἀναμέσον ἐμοῦ καὶ ἀναμέσον σοῦ, καὶ ἀναμέσον τοῦ πατρός μου καὶ τοῦ πατρός σου· ἰδοὺ ἐξαπέσταλκά σοι δῶρα ἀργύριον καὶ χρυσίον· δεῦρο, διασκέδασον τὴν διαθήκην σου τὴν πρὸς Βαασὰ βασιλέα Ἰσραὴλ, καὶ ἀναβήσεται ἀπʼ ἐμοῦ.
\VS{20}Καὶ ἤκουσεν υἱὸς Ἄδερ τοῦ βασιλέως Ἀσά, καὶ ἀπέστειλε τοὺς ἄρχοντας τῶν δυνάμεων αὐτοῦ ταῖς πόλεσι τοῦ Ἰσραὴλ, καὶ ἐπάταξαν τὴν Ἀῒν, τὴν Δὰν, καὶ τὴν Ἀβὲλ οἴκου Μααχὰ, καὶ πᾶσαν τὴν Χεννερὲθ, ἕως πάσης τῆς γῆς Νεφθαλί.
\VS{21}Καὶ ἐγένετο ὡς ἤκουσε Βαασὰ, καὶ διέλιπε τοῦ οἰκοδομεῖν τὴν Ῥαμὰ, καὶ ἀνέστρεψεν εἰς Θερσά.
\par }{\PP \VS{22}Καὶ ὁ βασιλεὺς Ἀσὰ παρήγγειλε παντὶ Ἰούδα εἰς ἐνακὶμ, καὶ αἴρουσι τοὺς λίθους τῆς Ῥαμὰ, καὶ τὰ ξύλα αὐτῆς ἃ ᾠκοδόμησε Βαασά· καὶ ᾠκοδόμησεν ἐν αὐτοῖς ὁ βασιλεὺς Ἀσὰ πᾶν βουνὸν Βενιαμὶν καὶ τὴν σκοπιάν.
\par }{\PP \VS{23}Καὶ τὰ λοιπὰ τῶν λόγων Ἀσὰ, καὶ πᾶσα ἡ δυναστεία αὐτοῦ ἣν ἐποίησε, καὶ τὰς πόλεις ἃς ᾠκοδόμησεν, οὐκ ἰδοὺ ταῦτα γεγραμμένα ἐστὶν ἐπὶ βιβλίῳ λόγων τῶν ἡμερῶν τοῖς βασιλεῦσιν Ἰούδα; πλὴν ἐν τῷ καιρῷ τοῦ γήρως αὐτοῦ ἐπόνεσε τοὺς πόδας αὐτοῦ.
\VS{24}Καὶ ἐκοιμήθη Ἀσὰ μετὰ τῶν πατέρων αὐτοῦ, καὶ θάπτεται μετὰ τῶν πατέρων αὐτοῦ ἐν πόλει Δαυίδ πατρὸς αὐτοῦ· καὶ βασιλεύει Ἰωσαφὰτ υἱὸς αὐτοῦ ἀντʼ αὐτοῦ.
\par }{\PP \VS{25}Καὶ Ναβὰτ υἱὸς Ἱεροβοὰμ βασιλεύει ἐπὶ Ἰσραὴλ ἐν ἔτει δευτέρῳ τοῦ Ἀσὰ βασιλέως Ἰούδα, καὶ ἐβασίλευσεν ἐν Ἰσραὴλ ἔτη δύο.
\VS{26}Καὶ ἐποίησε τὸ πονηρὸν ἐνώπιον Κυρίου, καὶ ἐπορεύθη ἐν ὁδῷ τοῦ πατρὸς αὐτοῦ καὶ ἐν ταῖς ἁμαρτίαις αὐτοῦ αἷς ἐξήμαρτε τὸν Ἰσραήλ.
\par }{\PP \VS{27}Καὶ περιεκάθισεν αὐτὸν Βαασὰ υἱὸς Ἀχιὰ ἐπὶ τὸν οἶκον Βελαὰν υἱοῦ Ἀχιὰ, καὶ ἐχάραξεν αὐτὸν ἐν Γαβαθὼν τῇ τῶν ἀλλοφύλων· καὶ Ναβὰτ καὶ πᾶς Ἰσραὴλ περιεκάθητο ἐπὶ Γαβαθών.
\VS{28}Καὶ ἐθανάτωσεν αὐτὸν Βαασὰ ἐν ἔτει τρίτῳ τοῦ Ἀσὰ υἱοῦ Ἀσὰ βασιλέως Ἰούδα, καὶ ἐβασίλευσεν.
\VS{29}Καὶ ἐγένετο ὡς ἐβασίλευσε, καὶ ἐπάταξεν ὅλον τὸν οἶκον Ἱεροβοὰμ, καὶ οὐχ ὑπελείπετο πᾶσαν πνοὴν τοῦ Ἱεροβοὰμ ἕως τοῦ ἐξολοθρεῦσαι αὐτὸν, κατὰ τὸ ῥῆμα Κυρίου ὃ ἐλάλησεν ἐν χειρὶ δούλου αὐτοῦ Ἀχιὰ τοῦ Σηλωνίτου·
\VS{30}περὶ τῶν ἁμαρτιῶν Ἱεροβοὰμ, ὅς ἐξήμαρτε τὸν Ἰσραήλ, καὶ ἐν τῷ παροργισμῷ αὐτοῦ ᾧ παρώργισε τὸν Κύριον Θεὸν τοῦ Ἰσραήλ.
\VS{31}Καὶ τὰ λοιπὰ τῶν λόγων Ναβὰτ καὶ πάντα ἃ ἐποίησεν, οὐκ ἰδοὺ ταῦτα γεγραμμένα ἐστὶν ἐν βιβλίῳ λόγων τῶν ἡμερῶν τοῖς βασιλεῦσιν Ἰσραήλ;
\par }{\PP \VS{33}Καὶ ἐν τῷ ἔτει τῷ τρίτῳ τοῦ Ἀσὰ βασιλέως Ἰούδα βασιλεύει Βαασὰ υἱὸς Ἀχιὰ ἐπὶ Ἰσραὴλ ἐν Θερσᾷ εἴκοσι καὶ τέσσαρα ἔτη.
\VS{34}Καὶ ἐποίησε τὸ πονηρὸν ἐνώπιον Κυρίου, καὶ ἐπορεύθη ἐν ὁδῷ Ἱεροβοὰμ υἱοῦ Ναβὰτ καὶ ἐν ταῖς ἁμαρτίαις αὐτοῦ ὡς ἐξήμαρτε τὸν Ἰσραήλ.

\par }\Chap{16}{\PP \VerseOne{1}Καὶ ἐγένετο λόγος Κυρίου ἐν χειρὶ Ἰοὺ υἱοῦ Ἀνανὶ πρὸς Βαασά.
\VS{2}Ἀνθʼ ὧν ὕψωσά σε ἀπὸ τῆς γῆς, καὶ ἔδωκά σε ἡγούμενον ἐπὶ τὸν λαόν μου Ἰσραήλ, καὶ ἐπορεύθης ἐν τῇ ὁδῷ Ἱεροβοὰμ, καὶ ἐξήμαρτες τὸν λαόν μου τὸν Ἰσραὴλ, τοῦ παροργίσαι με ἐν τοῖς ματαίοις αὐτῶν,
\VS{3}ἰδοὺ ἐγὼ ἐξεγείρω ὀπίσω Βαασὰ, καὶ ὄπισθεν τοῦ οἴκου αὐτοῦ, καὶ δώσω τὸν οἶκόν σου ὡς τὸν οἶκον Ἱεροβοὰμ υἱοῦ Ναβάτ.
\VS{4}Τὸν τεθνηκότα τοῦ Βαασὰ ἐν τῇ πόλει καταφάγονται αὐτὸν οἱ κύνες, καὶ τὸν τεθνηκότα αὐτοῦ ἐν τῷ πεδίῳ καταφάγονται αὐτὸν τὰ πετεινὰ τοῦ οὐρανοῦ.
\par }{\PP \VS{5}Καὶ τὰ λοιπὰ τῶν λόγων Βαασὰ καὶ πάντα ἃ ἐποίησε, καὶ αἱ δυναστεῖαι αὐτοῦ, οὐκ ἰδοὺ ταῦτα γεγραμμένα ἐν βιβλίῳ λόγων τῶν ἡμερῶν τῶν βασιλέων Ἰσραήλ;
\VS{6}Καὶ ἐκοιμήθη Βαασὰ μετὰ τῶν πατέρων αὐτοῦ, καὶ θάπτεται ἐν Θερσᾷ, καὶ βασιλεύει Ἠλὰ υἱὸς αὐτοῦ ἀντʼ αὐτοῦ.
\par }{\PP \VS{7}Καὶ ἐν χειρὶ Ἰοὺ υἱοῦ Ἀνανὶ ἐλάλησε Κύριος ἐπὶ Βαασὰ καὶ ἐπὶ τὸν οἶκον αὐτοῦ, πᾶσαν τὴν κακίαν ἣν ἐποίησεν ἐνώπιον Κυρίου τοῦ παροργίσαι αὐτὸν ἐν τοῖς ἔργοις τῶν χειρῶν αὐτου, τοῦ εἶναι κατὰ τὸν οἶκον Ἱεροβοὰμ, καὶ ὑπὲρ τοῦ πατάξαι αὐτόν.
\par }{\PP \VS{8}Καὶ Ἠλὰ υἱὸς Βαασὰ ἐβασίλευσεν ἐπὶ Ἰσραὴλ δύο ἔτη ἐν Θερσᾷ.
\VS{9}Καὶ συνέστρεψεν ἐπʼ αὐτὸν Ζαμβρὶ ὁ ἄρχων τῆς ἡμίσους τῆς ἵππου, καὶ αὐτὸς ἦν ἐν Θερσᾷ πίνων μεθύων ἐν τῷ οἴκῳ Ὠσᾶ τοῦ οἰκονόμου ἐν Θερσᾷ.
\VS{10}Καὶ εἰσῆλθε Σαμβρὶ καὶ ἐπάταξεν αὐτὸν καὶ ἐθανάτωσεν αὐτὸν, καὶ ἐβασίλευσεν ἀντʼ αὐτοῦ.
\VS{11}Καὶ ἐγενήθη ἐν τῷ βασιλεῦσαι αὐτὸν ἐν τῷ καθίσαι αὐτὸν ἐπὶ τοῦ θρόνου αὐτοῦ, καὶ ἐπάταξεν ὅλον τὸν οἶκον Βαασὰ,
\VS{12}κατὰ τὸ ῥῆμα ὃ ἐλάλησε Κύριος ἐπὶ τὸν οἶκον Βαασὰ, καὶ πρὸς Ἰοὺ τὸν προφήτην
\VS{13}περὶ πασῶν τῶν ἁμαρτιῶν Βαασὰ καὶ Ἠλὰ τοῦ υἱοῦ αὐτοῦ, ὡς ἐξήμαρτε τὸν Ἰσραὴλ, τοῦ παροργίσαι Κύριον τὸν Θεὸν Ἰσραὴλ ἐν τοῖς ματαίοις αὐτῶν.
\VS{14}Καὶ τὰ λοιπὰ τῶν λόγων Ἠλὰ ἃ ἐποίησεν, οὐκ ἰδοὺ ταῦτα γεγραμμένα ἐν βιβλίῳ λόγων τῶν ἡμερῶν τῶν βασιλέων Ἰσραήλ;
\par }{\PP \VS{15}Καὶ Ζαμβρὶ ἐβασίλευσεν ἐν Θερσᾷ ἡμέρας ἑπτά· καὶ ἡ παρεμβολὴ Ἰσραὴλ ἐπὶ Γαβαθὼν τὴν τῶν ἀλλοφύλων.
\VS{16}Καὶ ἤκουσεν ὁ λαὸς ἐν τῇ παρεμβολῇ, λεγόντων, συνεστράφη Ζαμβρὶ καὶ ἔπαισν τὸν βασιλέα· καὶ ἐβασίλευσαν ἐν Ἰσραὴλ τὸν Ἀμβρὶ τὸν ἡγούμενον τῆς στρατιᾶς ἐπὶ Ἰσραὴλ ἐν τῇ ἡμέρᾳ ἐκείνῃ ἐν τῇ παρεμβολῇ.
\VS{17}Καὶ ἀνέβη Ἀμβρὶ καὶ πᾶς Ἰσραὴλ μετʼ αὐτοῦ ἐκ Γαβαθὼν, καὶ περιεκάθισαν ἐπὶ Θερσᾷ.
\VS{18}Καὶ ἐγενήθη ὡς εἶδε Ζαμβρὶ ὅτι προκατείληπται αὐτοῦ ἡ πόλις, καὶ πορεύεται εἰς ἄντρον τοῦ οἴκου τοῦ βασιλέως, καὶ ἐνεπύρισεν ἐπʼ αὐτὸν τὸν οἶκον τοῦ βασιλέως, καὶ ἀπέθανεν
\VS{19}ὑπὲρ τῶν ἁμαρτιῶν αὐτοῦ ὧν ἐποίησε, τοῦ ποιῆσαι τὸ πονηρὸν ἐνώπιον Κυρίου πορευθῆναι ἐν ὁδῷ Ἱεροβοὰμ υἱοῦ Ναβὰτ, καὶ ἐν ταῖς ἁμαρτίαις αὐτοῦ ὡς ἐξήμαρτε τὸν Ἰσραήλ.
\VS{20}Καὶ τὰ λοιπὰ τῶν λόγων Ζαμβρὶ καὶ τὰς συνάψεις αὐτοῦ ἃς συνῆψεν, οὐκ ἰδοὺ ταῦτα γεγραμμένα ἐν βιβλίῳ λόγων τῶν ἡμερῶν τῶν βασιλέων Ἰσραήλ;
\par }{\PP \VS{21}Τότε μερίζεται ὁ λαὸς Ἰσραήλ· ἥμισυ τοῦ λαοῦ γίνεται ὀπίσω Θαμνὶ υἱοῦ Γωνὰθ τοῦ βασιλεῦσαι αὐτὸν, καὶ τὸ ἥμισυ τοῦ λαοῦ γίνεται ὀπίσω Ἀμβρί.
\VS{22}Ὁ λαὸς ὁ ὢν ὀπίσω Ἀμβρὶ ὑπερεκράτησε τὸν λαὸν τὸν ὀπίσω Θαμνὶ υἱοῦ Γωνάθ· καὶ ἀπέθανε Θαμνὶ καὶ Ἰωρὰμ ὁ ἀδελφὸς αὐτοῦ ἐν τῷ καιρῷ ἐκείνῳ, καὶ ἐβασίλευσεν Ἀμβρὶ μετὰ Θαμνί.
\par }{\PP \VS{23}Ἐν τῷ ἔτει τῷ τριακοστῷ καὶ πρώτῳ τοῦ βασιλέως Ἀσὰ βασιλεύει Ἀμβρὶ ἐπὶ Ἰσραὴλ δώδεκα ἔτη· ἐν Θερσᾷ βασιλεύει ἓξ ἔτη.
\VS{24}Καὶ ἐκτήσατο Ἀμβρὶ τὸ ὄρος τὸ Σεμερὼν παρὰ Σεμὴρ τοῦ κυρίου τοῦ ὄρους ἐν δύο ταλάντων ἀργυρίου· καὶ ᾠκοδόμησε τὸ ὄρος, καὶ ἐπεκάλεσαν τὸ ὄνομα τοῦ ὄρους οὗ ᾠκοδόμησαν ἐπὶε τῷ ὀνόματι Σεμὴρ τοῦ κυρίου τοῦ ὄρους, Σεμηρών.
\VS{25}Καὶ ἐποίησεν Ἀμβρὶ τὸ πονηρὸν ἐνώπιον Κυρίου, καὶ ἐπονηρεύσατο ὑπὲρ πάντας τοὺς γενομένους ἔμπροσθεν αὐτοῦ.
\VS{26}Καὶ ἐπορεύθη ἐν πάσῃ ὁδῷ Ἱεροβοὰμ υἱοῦ Ναβὰτ, καὶ ἐν ταῖς ἁμαρτίαις αὐτοῦ αἷς ἐξήμαρτε τὸν Ἰσραὴλ, τοῦ παροργίσαι τὸν Κύριον Θεὸν Ἰσραὴλ ἐν τοῖς ματαίοις αὐτῶν.
\VS{27}Καὶ τὰ λοιπὰ τῶν λόγων Ἀμβρὶ καὶ πάντα ἃ ἐποίησε, καὶ πᾶσα ἡ δυναστεία αὐτοῦ, οὐκ ἰδοὺ ταῦτα γεγραμμένα ἐν βιβλίῳ λόγων τῶν ἡμερῶν τῶν βασιλέων Ἰσραήλ;
\par }{\PP \VS{28}Καὶ ἐκοιμήθη Ἀμβρὶ μετὰ τῶν πατέρων αὐτοῦ, καὶ θάπτεται ἐν Σαμαρείᾳ, καὶ βασιλεύει Ἀχαὰβ ὁ υἱὸς αὐτοῦ ἀντʼ αὐτοῦ.
\par }{\PP \VS{28a}Καὶ ἐν τῷ ἐνιαυτῷ τῷ ἑνδεκάτῳ ἔτει τοῦ Ἀμβρὶ βασιλεύει Ἰωσαφὰτ υἱὸς Ἀσὰ ἐτῶν τριάκοντα καὶ πέντε ἐν τῇ βασιλείᾳ αὐτοῦ, καὶ εἰκοσιπέντε ἔτη ἐβασίλευσεν ἐν Ἱερουσαλήμ· καὶ ὄνομα τῆς μητρὸς αὐτοὐ Γαζουβὰ, θυγάτηρ Σελί·
\VS{28b}καὶ ἐπορεύθη ἐν τῇ ὁδῷ Ἀσὰ τοῦ πατρὸς αὐτοῦ, καὶ οὐκ ἐξέκλινεν ἀπʼ αὐτῆς τοῦ ποιεῖν τὸ εὐθὲς ἐνώπιον Κυρίου· πλὴν τῶν ὑψηλῶν οὐκ ἐξῇραν· ἔθυον ἐν τοῖς ὑψηλοῖς, καὶ ἐθυμίων·
\VS{28c}καὶ ἃ συνέθετο Ἰωσαφὰτ μετὰ βασιλέως Ἰσραὴλ, καὶ πᾶσα ἡ δυναστεία ἣν ἐποίησε, καὶ οὓς ἐπολέμησεν, οὐκ ἰδοὺ ταῦτα γεγραμμένα ἐν βιβλίῳ λόγων τῶν ἡμερῶν τῶν βασιλέων Ἰούδα;
\VS{28d}καὶ τὰ λοιπὰ τῶν συμπλοκῶν ἃς ἐπέθεντο ἐν ταῖς ἡμέραις Ἀσὰ τοῦ πατρὸς αὐτοῦ ἐξῇρεν ἀπὸ τῆς γῆς·
\VS{28e}καὶ βασιλεὺς οὐκ ἦν ἐν Συρίᾳ· Νασίβ.
\par }{\PP \VS{28f}Καὶ ὁ βασιλεὺς Ἰωσαφὰτ ἐποίησε ναῦν εἰς Θαρσὶς πορεύεσθαι εἰς Σωφὶρ ἐπὶ τὸ χρυσίον· καὶ οὐκ ἐπορεύθη, ὅτι συνετρίβη ἡ ναῦς ἐν Γασιὼν Γαβέρ·
\VS{28g}τότε εἶπεν βασιλεὺς Ἰσραὴλ πρὸς Ἰωσαφὰτ, ἐξαποστελῶ τοὺς παῖδάς σου καὶ τὰ παιδάριά μου ἐν τῇ νηΐ· καὶ οὐκ ἐβούλετο Ἰωσαφάτ·
\VS{28h}καὶ ἐκοιμήθη Ἰωσαφὰτ μετὰ τῶν πατέρων αὐτοῦ, καὶ θάπτεται μετὰ τῶν πατέρων αὐτοῦ ἐν πόλει Δαυίδ· καὶ ἐβασίλευσεν Ἰωρὰμ υἱὸς αὐτοῦ ἀντʼ αὐτοῦ.
\par }{\PP \VS{29}Ἐν ἔτει δευτέρῳ τοῦ Ἰωσαφὰτ βασιλέως Ἰούδα, Ἀχαὰβ υἱὸς Ἀμβρὶ ἐβασίλευσεν ἐπὶ Ἰσραὴλ ἐν Σαμαρείᾳ εἴκοσι καὶ δύο ἔτη.
\VS{30}Καὶ ἐποίησεν Ἀχαὰβ τὸ πονηρὸν ἐνώπιον Κυρίου, καὶ ἐπονηρεύσατο ὑπὲρ πάντας τοὺς ἔμπροσθεν αὐτοῦ.
\VS{31}Καὶ οὐκ ἦν αὐτῷ ἱκανὸν τοῦ πορεύεσθαι ἐν ταῖς ἁμαρτίαις Ἱεροβοὰμ υἱοῦ Ναβὰτ, καὶ ἔλαβε γυναῖκα τὴν Ἰεζάβελ θυγατέρα Ἰεθεβαὰλ βασιλέως Σιδωνίων· καὶ ἐπορεύθη καὶ ἐδούλευσε τῷ Βάαλ, καὶ προσεκύνησεν αὐτῷ.
\VS{32}Καὶ ἔστησε θυσιαστήριον τῷ Βάαλ ἐν οἴκῳ τῶν προσοχθισμάτων αὐτοῦ, ὃν ᾠκοδόμησεν ἐν Σαμαρείᾳ.
\VS{33}Καὶ ἐποίησεν Ἀχαὰβ ἄλσος· καὶ προσέθηκεν Ἀχαὰβ τοῦ ποιῆσαι παροργίσματα, τοῦ παροργίσαι τὸν Κύριον Θεὸν τοῦ Ἰσραὴλ, καὶ τὴν ψυχὴν αὐτοῦ ἐξολοθρευθῆναι, ἐκακοποίησεν ὑπὲρ πάντας τοὺς βασιλεῖς Ἰσραὴλ τοὺς γενομένους ἔμπροσθεν αὐτοῦ.
\par }{\PP \VS{34}Καὶ ἐν ταῖς ἡμέραις αὐτοῦ ᾠκοδόμησεν Ἀχιὴλ ὁ Βαιθηλίτης τὴν Ἱεριχώ· ἐν τῷ Ἀβιρὼν πρωτοτόκῳ αὐτοῦ ἐθεμελίωσεν αὐτὴν, καὶ τῷ Σεγοὺβ τῷ νεωτέρῳ αὐτοῦ ἐπέστησε θύρας αὐτῆς, κατὰ τὸ ῥῆμα Κυρίου, ὃ ἐλάλησεν ἐν χειρὶ Ἰησοῦ υἱοῦ Ναυῆ.

\par }\Chap{17}{\PP \VerseOne{1}Καὶ εἶπεν Ἠλιοὺ ὁ προφήτης Θεσβίτης ὁ ἐκ Θεσβῶν τῆς Γαλαὰδ πρὸς Ἀχαὰβ, ζῇ Κύριος ὁ Θεὸς τῶν δυνάμεων ὁ Θεὸς Ἰσραὴλ, ᾧ παρέστην ἐνώπιον αὐτοῦ, εἰ ἔσται τὰ ἔτη ταῦτα δρόσος καὶ ὑετὸς, ὅτι εἰ μὴ διὰ στόματος λόγου μου.
\par }{\PP \VS{2}Καὶ ἐγένετο ῥῆμα Κυρίου πρὸς Ἠλιοὺ,
\VS{3}πορεύου ἐντεῦθεν κατὰ ἀνατολὰς, καὶ κρύβηθι ἐν τῷ χειμάῤῥῳ Χοῤῥαθ τοῦ ἐπὶ προσώπου τοῦ Ἰορδάνου.
\VS{4}Καὶ ἔσται ἐκ τοῦ χειμάῤῥου πίεσαι ὕδωρ, καὶ τοῖς κόραξιν ἐντελοῦμαι διατρέφειν σε ἐκεῖ.
\VS{5}Καὶ ἐποίησεν Ἠλιοὺ κατὰ τὸ ῥῆμα Κυρίου, καὶ ἐκάθισεν ἐν τῷ χειμάῤῥῳ Χοῤῥὰθ ἐπὶ προσώπου τοῦ Ἰορδάνου.
\VS{6}Καὶ οἱ κόρακες ἔφερον αὐτῷ ἄρτους τοπρωῒ, καὶ κρέα τοδείλης, καὶ ἐκ τοῦ χειμάῤῥου ἔπινεν ὕδωρ.
\VS{7}Καὶ ἐγένετο μεθʼ ἡμέρας, καὶ ἐξηράνθη ὁ χειμάῤῥους, ὅτι οὐκ ἐγένετο ὑετὸς ἐπὶ τῆς γῆς.
\par }{\PP \VS{8}Καὶ ἐγένετο ῥῆμα Κυρίου πρὸς Ἠλιοὺ,
\VS{9}ἀνάστηθι, καὶ πορεύου εἰς Σαρεπτὰ τῆς Σιδωνίας· ἰδοὺ ἐντέταλμαι ἐκεῖ γυναικὶ χήρᾳ τοῦ διατρέφειν σε.
\VS{10}Καὶ ἀνέστη καὶ ἐπορεύθη εἰς Σαρεπτὰ, καὶ ἦλθεν εἰς τὸν πυλῶνα τῆς πόλεως· καὶ ἰδοὺ ἐκεῖ γυνὴ χήρα συνέλεγε ξύλα· καὶ ἐβόησεν ὀπίσω αὐτῆς Ἠλιοὺ, καὶ εἶπεν αὐτῇ, λάβε δή μοι ὀλίγον ὕδωρ εἰς ἄγγος, καὶ πίομαι.
\VS{11}Καὶ ἐπορεύθη λαβεῖν, καὶ ἐβόησεν ὀπίσω αὐτῆς Ἠλιοὺ, καὶ εἶπε, λήψῃ δή μοι ψωμὸν ἄρτου τοῦ ἐν τῇ χειρί σου.
\VS{12}Καὶ εἶπεν ἡ γυνὴ, ζῇ Κύριος ὁ Θεός σου, εἰ ἔστι μοι ἐγκρυφίας, ἀλλʼ ἢ ὅσον δρὰξ ἀλεύρου ἐν τῇ ὑδρίᾳ, καὶ ὀλίγον ἔλαιον ἐν τῷ καψάκῃ· καὶ ἰδοὺ ἐγὼ συλλέξω δύο ξυλάρια, καὶ εἰσελεύσομαι καὶ ποιήσω αὐτὸ ἐμαυτῇ καὶ τοῖς τέκνοις μου, καὶ φαγόμεθα, καὶ ἀποθανούμεθα.
\par }{\PP \VS{13}Καὶ εἶπε πρὸς αὐτὴν Ἠλιοὺ, θάρσει, εἴσελθε καὶ ποίησον κατὰ τὸ ῥῆμά σου· ἀλλὰ ποίησόν μοι ἐκεῖθεν ἐγκρυφίαν μικρὸν, καὶ ἐξοίσεις μοι ἐν πρώτοις, σαυτῇ δὲ καὶ τοῖς τέκνοις σου ποιήσεις ἐπʼ ἐσχάτῳ·
\VS{14}Ὅτι τάδε λέγει Κύριος, ἡ ὑδρία τοῦ ἀλεύρου οὐκ ἐκλείψει, καὶ ὁ καψάκης τοῦ ἐλαίου οὐκ ἐλαττονήσει, ἕως ἡμέρας τοῦ δοῦναι Κύριον τὸν ὑετὸν ἐπὶ τῆς γῆς.
\VS{15}Καὶ ἐπορεύθη ἡ γυνὴ, καὶ ἐποίησε, καὶ ἤσθιεν αὐτὴ καὶ αὐτὸς καὶ τὰ τέκνα αὐτῆς.
\VS{16}Καὶ ἡ ὑδρία τοῦ ἀλεύρου οὐκ ἐξέλιπε, καὶ ὁ καψάκης τοῦ ἐλαίου οὐκ ἠλαττονήθη, κατὰ τὸ ῥῆμα Κυρίου ὃ ἐλάλησεν ἐν χειρὶ Ἠλιού.
\par }{\PP \VS{17}Καὶ ἐγένετο μετὰ ταῦτα, καὶ ἠῤῥώστησεν ὁ υἱὸς τῆς γυναικὸς τῆς κυρίας τοῦ οἴκου· καὶ ἦν ἡ ἀῤῥωστία αὐτοῦ κραταιὰ σφόδρα ἕως οὐχ ὑπελείφθη ἐν αὐτῷ πνεῦμα.
\VS{18}Καὶ εἶπε πρὸς Ἠλιοὺ, τί ἐμοὶ καὶ σοὶ ἄνθρωπε τοῦ Θεοῦ; εἰσῆλθες πρὸς μὲ τοῦ ἀναμνῆσαι ἀδικίας μου, καὶ θανατῶσαι τὸν υἱόν μου;
\par }{\PP \VS{19}Καὶ εἶπεν Ἠλιοὺ πρὸς τὴν γυναῖκα, δός μοι τὸν υἱόν σου· καὶ ἔλαβεν αὐτὸν ἐκ τοῦ κόλπου αὐτῆς, καὶ ἀνήνεγκεν αὐτὸν εἰς τὸ ὑπερῷον ἐν ᾧ αὐτὸς ἐκάθητο ἐκεῖ, καὶ ἐκοίμισεν αὐτὸν ἐπὶ τῆς κλίνης.
\VS{20}Καὶ ἀνεβόησεν Ἠλιοὺ, καὶ εἶπεν, οἵ μοι Κύριε, ὁ μάρτυς τῆς χήρας μεθʼ ἧς ἐγὼ κατοικῶ μετʼ αὐτῆς, σὺ κεκάκωκας τοῦ θανατῶσαι τὸν υἱὸν αὐτῆς.
\VS{21}Καὶ ἐνεφύσησε τῷ παιδαρίῳ τρὶς, καὶ ἐπεκαλέσατο τὸν Κύριον, καὶ εἶπε, Κύριε ὁ Θεός μου, ἐπιστραφήτω δὴ ἡ ψυχὴ τοῦ παιδαρίου τούτου εἰς αὐτόν.
\VS{22}Καὶ ἐγένετο οὕτως· καὶ ἀνεβόησε τὸ παιδάριον.
\VS{23}Καὶ κατήγαγεν αυτὸ ἀπὸ τοῦ ὑπερῴου εἰς τὸν οἶκον, καὶ ἔδωκεν αὐτὸ τῇ μητρὶ αὐτοῦ· καὶ εἶπεν Ἠλιοὺ, βλέπε, ζῇ ὁ υἱός σου.
\VS{24}Καὶ εἶπεν ἡ γυνὴ πρὸς Ἠλιοὺ, ἰδοὺ ἔγνωκα ὅτι σὺ ἄνθρωπος Θεοῦ, καὶ ῥῆμα Κυρίου ἐν τῷ στόματί σου ἀληθινόν.

\par }\Chap{18}{\PP \VerseOne{1}Καὶ ἐγένετο μεθʼ ἡμέρας πολλὰς, καὶ ῥῆμα Κυρίου ἐγένετο πρὸς Ἠλιοὺ ἐν τῷ ἐνιαυτῷ τῷ τρίτῳ, λέγων, πορεύθητι, καὶ ὄφθητι τῷ Ἀχαὰβ, καὶ δώσω ὑετὸν ἐπὶ πρόσωπον τῆς γῆς.
\VS{2}Καὶ ἐπορεύθη Ἠλιοὺ τοῦ ὀφθῆναι τῷ Ἀχαὰβ, καὶ ἡ λιμὸς κραταιὰ ἐν Σαμαρείᾳ.
\par }{\PP \VS{3}Καὶ ἐκάλεσεν Ἀχαὰβ τὸν Ἀβδιοὺ τὸν οἰκονόμον· καὶ Ἀβδιοὺ ἦν φοβούμενος τὸν Κύριον σφόδρα.
\VS{4}Καὶ ἐγένετο ἐν τῷ τύπτειν τὴν Ἰεζάβελ τοὺς προφήτας Κυρίου, καὶ ἔλαβεν Ἀβδιοὺ ἑκατὸν ἄνδρας προφήτας καὶ κατέκρυψεν αὐτοὺς κατὰ πεντήκοντα ἐν σπηλαίῳ, καὶ διέτρεφεν αὐτοὺς ἐν ἄρτῳ καὶ ὕδατι.
\VS{5}Καὶ εἶπεν Ἀχαὰβ πρὸς Ἀβδιοὺ, δεῦρο, καὶ διέλθωμεν ἐπὶ τὴν γῆν καὶ ἐπὶ πηγὰς τῶν ὑδάτων καὶ ἐπὶ χειμάῤῥους, ἐὰν πῶς εὕρωμεν βοτάνην, καὶ περιποιησώμεθα ἵππους καὶ ἡμιόνους, καὶ οὐκ ἐξολοθρευθήσονται ἀπὸ τῶν σκηνῶν.
\VS{6}Καὶ ἐμέρισαν ἑαυτοῖς τὴν ὁδὸν τοῦ διελθεῖν αὐτήν· Ἀχαὰβ ἐπορεύθη ἐν ὁδῷ μιᾷ, καὶ Ἀβδιοὺ ἐπορεύθη ἐν ὁδῷ ἄλλῃ μόνος.
\VS{7}Καὶ ἦν Ἀβδιοὺ ἐν τῇ ὁδῷ μόνος· καὶ ἦλθεν Ἠλιοὺ εἰς συνάντησιν αὐτοῦ μόνος· καὶ Ἀβδιοὺ ἔσπευσε καὶ ἔπεσεν ἐπὶ πρόσωπον αὐτοῦ, καὶ εἶπεν, εἰ σὺ εἶ αὐτὸς, κύριέ μου Ἠλιού;
\VS{8}Καὶ εἶπεν Ἠλιοὺ αὐτῷ, ἑγώ· πορεύου, λέγε τῷ κυρίῳ σου, ἰδοὺ Ἠλιού.
\VS{9}Καὶ εἶπεν Ἀβδιοὺ, τί ἡμάρτηκα, ὅτι δίδως τὸν δοῦλόν σου εἰς χεῖρα Ἀχαὰβ τοῦ θανατῶσαί με;
\VS{10}Ζῇ Κύριος ὁ Θεός σου, εἰ ἔστιν ἔθνος ἢ βασιλεία, οὗ οὐκ ἀπέστειλεν ὁ κύριός μου ζητεῖν σε· καὶ εἰ εἶπον, οὐκ ἔστι, καὶ ἐνέπρησε τὴν βασιλείαν καὶ τὰς χώρας αὐτῆς, ὅτι οὐχ εὕρηκέ σε.
\VS{11}Καὶ νῦν σὺ λέγεις, πορεύου, ἀνάγγελε τῷ κυρίῳ σου, ἰδοὺ Ἠλιού.
\VS{12}Καὶ ἔσται ἐὰν ἐγὼ ἀπέλθω ἀπὸ σοῦ, καὶ πνεῦμα Κυρίου ἀρεῖ σε εἰς τὴν γῆν ἣν οὐκ οἶδα, καὶ εἰσελεύσομαι ἀπαγγεῖλαι τῷ Ἀχαὰβ, καὶ οὐχ εὑρήσει σε, καὶ ἀποκτενεῖ με· καὶ ὁ δοῦλός σου ἐστὶ φοβούμενος τὸν Κύριον ἐκ νεότητος αὐτοῦ.
\VS{13}Ἢ οὐκ ἀπηγγέλη σοι τῷ κυρίῳ μου, οἷα πεποίηκα ἐν τῷ ἀποκτείνειν τὴν Ἰεζάβελ τοὺς προφήτας Κυρίου, καὶ ἔκρυψα ἀπὸ τῶν προφητῶν Κυρίου ἑκατὸν ἄνδρας, ἀνὰ πεντήκοντα ἐν σπηλαίῳ, καὶ ἔθρεψα ἐν ἄρτοις καὶ ὕδατι;
\VS{14}Καὶ νῦν σὺ λέγεις μοι, πορεύου, λέγε τῷ κυρίῳ σου, ἰδοὺ Ἠλιού· καὶ ἀποκτενεῖ με.
\VS{15}Καὶ εἶπεν Ἠλιοὺ, ζῇ Κύριος τῶν δυνάμεων ᾧ παρέστην ἐνώπιον αὐτοῦ, ὅτι σήμερον ὀφθήσομαι αὐτῷ.
\par }{\PP \VS{16}Καὶ ἐπορεύθη Ἀβδιοὺ εἰς συναντὴν τῷ Ἀχαὰβ, καὶ ἀπήγγειλεν αὐτῷ· καὶ ἐξέδραμεν Ἀχαὰβ, καὶ ἐπορεύθη εἰς συνάντησιν Ἠλιού.
\VS{17}Καὶ ἐγένετο ὡς εἶδεν Ἀχαὰβ τὸν Ἠλιού, καὶ εἶπεν Ἀχαὰβ πρὸς Ἠλιοὺ, εἰ σὺ εἶ αὐτὸς ὁ διαστρέφων τὸν Ἰσραήλ;
\VS{18}Καὶ εἶπεν Ἠλιοὺ, οὐ διαστρέφω τὸν Ἰσραὴλ, ὅτι ἀλλʼ ἢ σὺ καὶ οἶκος τοῦ πατρός σου, ἐν τῷ καταλιμπάνειν ὑμᾶς τὸν Κύριον Θεὸν ὑμῶν, καὶ ἐπορεύθης ὀπίσω τῶν Βααλίμ.
\VS{19}Καὶ νῦν ἀπόστειλον, συνάθροισον πρὸς μὲ πάντα Ἰσραὴλ εἰς ὄρος τὸ Καρμήλιον, καὶ τοὺς προφήτας τῆς αἰσχύνης τετρακοσίους καὶ πεντήκοντα, καὶ τοὺς προφήτας τῶν ἀλσῶν τετρακοσίους ἐσθίοντας τράπεζαν Ἰεζάβελ.
\par }{\PP \VS{20}Καὶ ἀπέστειλεν Ἀχαὰβ εἰς πάντα Ἰσραὴλ, καὶ ἐπισυνήγαγεπάντας τοὺς προφήτας εἰς ὄρος τὸ Καρμήλιον.
\par }{\PP \VS{21}Καὶ προσήγαγεν Ἠλιοὺ πρὸς πάντας· καὶ εἶπεν αὐτοῖς Ἠλιοὺ, ἕως πότε ὑμεῖς χωλανεῖτε ἐπʼ ἀμφοτέραις ταῖς ἰγνύαις· εἰ ἔστι Κύριος ὁ Θεὸς, πορεύεσθε ὀπίσω αὐτοῦ· εἰ δὲ Βάαλ, πορεύεσθε ὀπίσω αὐτοῦ· καὶ οὐκ ἀπεκρίθη ὁ λαὸς λόγον.
\VS{22}Καὶ εἶπεν Ἠλιοὺ πρὸς τὸν λαόν, ἐγὼ ὑπολέλειμμαι προφήτης τοῦ Κυρίου μονώτατος· καὶ οἱ προφῆται τοῦ Βάαλ τετρακόσιοι καὶ πεντήκοντα ἄνδρες, καὶ οἱ προφῆται τοῦ ἄλσους τετρακόσιοι·
\VS{23}Δότωσαν ἡμῖν δύο βόας, καὶ ἐκλεξάσθωσαν ἑαυτοῖς τὸν ἕνα, καὶ μελισάτωσαν, καὶ ἐπιθέτωσαν ἐπὶ τῶν ξύλων, καὶ πῦρ μὴ ἐπιθέτωσαν· καὶ ἐγὼ ποιήσω τὸν βοῦν τὸν ἄλλον, καὶ πῦρ οὐ μὴ ἐπιθῶ.
\VS{24}Καὶ βοᾶτε ἐν ὀνόματι θεῶν ὑμῶν, καὶ ἐγὼ ἐπικαλέσομαι ἐν τῷ ὀνόματι Κυρίου τοῦ Θεοῦ μου· καὶ ἔσται ὁ θεὸς ὃς ἂν ἐπακούσῃ ἐν πυρὶ, οὗτος Θεός· καὶ ἀπεκρίθησαν πᾶς ὁ λαὸς, καὶ εἶπον, καλὸν τὸ ῥῆμα ὃ ἐλάλησας.
\par }{\PP \VS{25}Καὶ εἶπεν Ἠλιοὺ τοῖς προφήταις τῆς αἰσχύνης, ἐκλέξασθε ἑαυτοῖς τὸν μόσχον τὸν ἕνα, καὶ ποιήσατε πρῶτοι, ὅτι πολλοὶ ὑμεῖς· καὶ ἐπικαλέσασθε ἐν ὀνόματι θεοῦ ὑμῶν, καὶ πῦρ μὴ ἐπιθῆτε.
\VS{26}Καὶ ἔλαβον τὸν μόσχον καὶ ἐποίησαν, καὶ ἐπεκαλοῦντο ἐν ὀνόματι τοῦ Βάαλ ἐκ πρωΐθεν ἕως μεσημβρίας, καὶ εἶπον, ἐπάκουσον ἡμῶν ὁ Βάαλ, ἐπάκουσον ἡμῶν· καὶ οὐκ ἦν φωνὴ, καὶ οὐκ ἦν ἀκρόασις· καὶ διέτρεχον ἐπὶ τοῦ θυσιαστηρίου οὗ ἐποίησαν.
\VS{27}Καὶ ἐγένετο μεσημβρία, καὶ ἐμυκτήρισεν αὐτοὺς Ἠλιοὺ ὁ Θεσβίτης, καὶ εἶπεν, ἐπικαλεῖσθε ἐν φωνῇ μεγάλῃ, ὅτι θεός ἐστιν· ὅτι ἀδολεσχία αὐτῷ ἐστι, καὶ ἅμα μή ποτε χρηματίζει αὐτὸς, ἢ μή ποτε καθεύδει αὐτὸς, καὶ ἐξαναστήσεται.
\VS{28}Καὶ ἐπεκαλοῦντο ἐν φωνῇ μεγάλῃ, καὶ κατετέμνοντο κατὰ τὸν ἐθισμὸν αὐτῶν ἐν μαχαίραις καὶ σειρομάσταις ἕως ἐκχύσεως αἵματος ἐπʼ αὐτοὺς,
\VS{29}καὶ προεφήτευον ἕως οὗ παρῆλθε τὸ δειλινόν· καὶ ἐγένετο ὡς ὁ καιρὸς τοὺ ἀναβῆναι τὴν θυσίαν, καὶ ἐλάλησεν Ἠλιοὺ ὁ Θεσβίτης πρὸς τοὺς προφήτας τῶν προσοχθισμάτων, λέγων, μετάστητε ἀπὸ τοῦ νῦν, καὶ ἐγὼ ποιήσω τὸ ὁλοκαύτωμά μου· καὶ μετέστησαν, καὶ ἀπῆλθον.
\par }{\PP \VS{30}Καὶ εἶπεν Ἠλιοὺ πρὸς τὸν λαὸν, προσαγάγετε πρὸς μέ· καὶ προσήγαγε πᾶς ὁ λαὸς πρὸς αὐτόν.
\VS{31}Καὶ ἔλαβεν Ἡλιοὺ δώδεκα λίθους κατὰ ἀριθμὸν φυλῶν τοῦ Ἰσραὴλ, ὡς ἐλάλησε Κύριος πρὸς αὐτὸν, λέγων, Ἰσραὴλ ἔσται τὸ ὄνομά σου.
\VS{32}Καὶ ᾠκοδόμησε τοὺς λίθους ἐν ὀνόματι Κυρίου, καὶ ἰάσατο τὸ θυσιαστήριον τὸ κατεσκαμμένον· καὶ ἐποίησε θάλασσαν χωροῦσαν δύο μετρητὰς σπέρματος κυκλόθεν τοῦ θυσιαστηρίου·
\VS{33}Καὶ ἐστοίβασε τὰς σχίδακας ἐπὶ τὸ θυσιαστήριον ὃ ἐποίησε, καὶ ἐμέλισε τὸ ὁλοκαύτωμα καὶ ἐπέθηκεν ἐπὶ τὰς σχίδακας, καὶ ἐστοίβασεν ἐπὶ τὸ θυσιαστήριον,
\VS{34}καὶ εἶπε, λάβετέ μοι τέσσαρας ὑδρίας ὕδατος, καὶ ἐπιχέετε ἐπὶ τὸ ὁλοκαύτωμα καὶ ἐπὶ τὰς σχίδακας· καὶ ἐποίησαν οὕτως. Καὶ εἶπε, δευτερώσατε· καὶ ἐδευτέρωσαν· καὶ εἶπε, τρισσώσατε· καὶ ἐτρίσσευσαν.
\VS{35}Καὶ διεπορεύετο τὸ ὕδωρ κύκλῳ τοῦ θυσιαστηρίου, καὶ τὴν θάλασσαν ἔπλησαν ὕδατος.
\par }{\PP \VS{36}Καὶ ἀνεβόησεν Ἠλιοὺ εἰς τὸν οὐρανὸν, καὶ εἶπε, Κύριε ὁ Θεὸς Ἁβραὰμ καὶ Ἰσαὰκ καὶ Ἰσραήλ, ἐπάκουσόν μου Κύριε, ἐπάκουσόν μου σήμερον ἐν πυρί, καὶ γνώτωσαν πᾶς ὁ λαὸς οὗτος, ὅτι σὺ εἶ Κύριος ὁ Θεὸς Ἰσραὴλ, καὶ ἐγὼ δοῦλός σου, καὶ διὰ σὲ πεποίηκα τὰ ἔργα ταῦτα.
\VS{37}Ἐπάκουσόν μου Κύριε, ἐπάκουσόν μου, καὶ γνώτω ὁ λαὸς οὗτος, ὅτι σὺ εἶ Κύριος ὁ Θεὸς, καὶ σὺ ἔστρεψας τὴν καρδίαν τοῦ λαοῦ τούτου ὀπίσω.
\VS{38}Καὶ ἔπεσε πῦρ παρὰ Κυρίου ἐκ τοῦ οὐρανοῦ, καὶ κατέφαγε τὰ ὁλοκαυτώματα καὶ τὰς σχίδακας καὶ τὸ ὕδωρ τὸ ἐν τῇ θαλάσσῃ, καὶ τοὺς λίθους καὶ τὸν χοῦν ἐξέλειξε τὸ πῦρ.
\par }{\PP \VS{39}Καὶ ἔπεσε πᾶς ὁ λαὸς ἐπὶ πρόσωπον αὐτῶν, καὶ εἶπον, ἀληθῶς Κύριος ὁ Θεὸς αὐτὸς ὁ Θεός.
\VS{40}Καὶ εἶπεν Ἠλιοὺ πρὸς τὸν λαὸν, συλλάβετε τοὺς προφήτας τοῦ Βάαλ, μηδεὶς σωθήτω ἐξ αὐτῶν· καὶ συνελαβον αὐτοὺς, καὶ κατάγει αὐτοὺς Ἠλιοὺ εἰς τὸν χειμάῤῥουν Κισσῶν, καὶ ἔσφαξεν αὐτοὺς ἐκεῖ.
\par }{\PP \VS{41}Καὶ εἶπεν Ἠλιοὺ τῷ Ἀχαὰβ, ἀνάβηθι, καὶ φάγε καὶ πίε, ὅτι φωνὴ τῶν ποδῶν τοῦ ὑετοῦ.
\VS{42}Καὶ ἀνέβη Ἀχαὰβ τοῦ φαγεῖν καὶ πιεῖν· καὶ Ἠλιοὺ ἀνέβη ἐπὶ τὴν Κάρμηλον καὶ ἔκυψεν ἐπὶ τὴν γῆν, καὶ ἔθηκε τὸ πρόσωπον αὐτοῦ ἀναμέσον τῶν γονάτων αὐτοῦ,
\VS{43}καὶ εἶπε τῷ παιδαρίῳ αὐτοῦ, ἀνάβηθι, καὶ ἐπίβλεψον ὁδὸν τῆς θαλάσσης· καὶ ἐπέβλεψε τὸ παιδάριον, καὶ εἶπεν, οὐκ ἔστιν οὐθέν· καὶ εἶπεν Ἠλιοὺ, καὶ σὺ ἐπίστρεψον ἑπτάκις.
\VS{44}Καὶ ἐπέστρεψε τὸ παιδάριον ἑπτάκις· καὶ ἐγένετο ἐν τῷ ἑβδόμῳ, καὶ ἰδοὺ νεφέλη μικρὰ ὡς ἴχνος ἀνδρὸς ἀνάγουσα ὕδωρ· καὶ εἶπεν, ἀνάβηθι, καὶ εἶπον Ἀχαὰβ, ζεῦξον τὸ ἅρμα σου καὶ κατάβηθι, μὴ καταλάβῃ σε ὁ ὑετός.
\VS{45}Καὶ ἐγένετο ἕως ὧδε καὶ ὧδε, καὶ ὁ οὐρανὸς συνεσκότασε νεφέλαις καὶ πνεύματι, καὶ ἐγένετο ὑετὸς μέγας· καὶ ἔκλαιε καὶ ἐπορεύετο Ἀχαὰβ ἕως Ἰεζράελ.
\VS{46}Καὶ χεὶρ Κυρίου ἐπὶ τὸν Ἠλιοὺ, καὶ συνέσφιγξε τὴν ὀσφὺν αὐτοῦ, καὶ ἔτρεχεν ἔμπροσθεν Ἀχαὰβ εἰς Ἰεζράελ.

\par }\Chap{19}{\PP \VerseOne{1}Καὶ ἀνήγγειλεν Ἀχαὰβ τῇ Ἰεζάβελ γυναικὶ αὐτοῦ πάντα ἃ ἐποίησεν Ἠλιοὺ, καὶ ὡς ἀπέκτεινε τοὺς προφήτας ἐν ῥομφαίᾳ.
\VS{2}Καὶ ἀπέστειλεν Ἰεζάβελ πρὸς Ἠλιοὺ, καὶ εἶπεν, εἰ σὺ εἶ Ἠλιοὺ καὶ ἐγὼ Ἰεζάβελ, τάδε ποιήσαι μοι ὁ θεὸς καὶ τάδε προσθείη, ὅτι ταύτην τὴν ὥραν αὔριον θήσομαι τὴν ψυχήν σου καθὼς ψυχὴν ἑνὸς ἐξ αὐτῶν.
\VS{3}Καὶ ἐφοβήθη Ἠλιοὺ, καὶ ἀνέστη καὶ ἀπῆλθε κατὰ τὴν ψυχὴν αὐτοῦ, καὶ ἔρχεται εἰς Βηρσαβεὲ γῆν Ἰούδα, καὶ ἀφῆκε τὸ παιδάριον αὐτοῦ ἐκεῖ.
\par }{\PP \VS{4}Καὶ αὐτὸς ἐπορεύθη ἐν τῇ ἐρήμῳ ὁδὸν ἡμέρας, καὶ ἦλθε καὶ ἐκάθισεν ὑποκάτω ῥαθμὲν, καὶ ᾐτήσατο τὴν ψυχὴν αὐτοῦ ἀποθανεῖν· καὶ εἶπεν, ἱκανούσθω νῦν, λάβε δὴ τὴν ψυχήν μου ἀπʼ ἐμοῦ Κύριε, ὅτι οὐ κρείσσων ἐγώ εἰμι ὑπὲρ τοὺς πατέρας μου.
\VS{5}Καὶ ἐκοιμήθη καὶ ὕπνωσεν ἐκεῖ ὑπὸ φυτόν· καὶ ἰδού τις ἥψατο αὐτοῦ, καὶ εἶπεν αὐτῷ, ἀνάστηθι καὶ φάγε.
\VS{6}Καὶ ἐπέβλεψεν Ἠλιού· καὶ ἰδοὺ πρὸς κεφαλῆς αὐτοῦ ἐγκρυφίας ὀλυρίτης καὶ καψάκης ὕδατος· καὶ ἀνέστη καὶ ἔφαγε καὶ ἔπιε, καὶ ἐπιστρέψας ἐκοιμήθη.
\VS{7}Καὶ ἐπέστρεψεν ὁ ἄγγελος Κυρίου ἐκ δευτέρου, καὶ ἥψατο αὐτοῦ, καὶ εἶπεν αὐτῷ, ἀνάστα, φάγε, ὅτι πολλὴ ἀπὸ σοῦ ἡ ὁδός.
\VS{8}Καὶ ἀνέστη, καὶ ἔφαγε, καὶ ἔπιε· καὶ ἐπορεύθη ἐν ἰσχύϊ τῆς ρώσεως ἐκείνης τεσσαράκοντα ἡμέρας καὶ τεσσαράκοντα νύκτας ἕως ὄρους Χωρήβ.
\par }{\PP \VS{9}Καὶ εἰσῆλθεν ἐκεῖ εἰς τὸ σπήλαιον, καὶ κατέλυσεν ἐκεῖ· καὶ ἰδοὺ ῥῆμα Κυρίου πρὸς αὐτὸν, καὶ εἶπε, τί σὺ ἐνταῦθα Ἠλιού;
\VS{10}Καὶ εἶπεν Ἠλιοὺ, ζηλῶν ἐζήλωκα τῷ Κυρίῳ παντοκράτορι, ὅτι ἐγκατέλιπόν σε οἱ υἱοὶ Ἰσραήλ· τὰ θυσιαστήριά σου κατέσκαψαν, καὶ τοὺς προφήτας σου ἀπέκτειναν ἐν ῥομφαίᾳ καὶ ὑπολέλειμμαι ἐγὼ μονώτατος, καὶ ζητοῦσι τὴν ψυχήν μου λαβεῖν αὐτήν.
\VS{11}Καὶ εἶπεν, ἐξελεύσῃ αὔριον, καὶ στήσῃ ἐνώπιον Κυρίου ἐν τῷ ὄρει· ἰδοὺ παρελεύσεται Κύριος. Καὶλἰδοὺ πνεῦμα μέγα κραταιὸν διαλύον ὄρη καὶ συντρίβον πέτρας ἐνώπιον Κυρίου, οὐκ ἐν τῷ πνεύματι Κύριος· καὶ μετὰ τό πνεῦμα συσσεισμὸς, οὐκ ἐν τῷ συσσεισμῷ Κύριος·
\VS{12}Καὶ μετὰ τὸν συσσεισμὸν πῦρ, οὐκ ἐν τῷ πυρὶ Κύριος· καὶ μετὰ τὸ πῦρ φωνὴ αὔρας λεπτῆς.
\par }{\PP \VS{13}Καὶ ἐγένετο ὡς ἤκουσεν Ἠλιοὺ, καὶ ἐπεκάλυψε τὸ πρόσωπον αὐτοῦ ἐν τῇ μηλωτῇ αὐτοῦ, καὶ ἐξῆλθε καὶ ἔστη ὑπὸ σπήλαιον· καὶ ἰδοὺ πρὸς αὐτὸν φωνὴ, καὶ εἶπε, τί σὺ ἐνταῦθα Ἠλιού;
\VS{14}Καὶ εἶπεν Ἠλιού, ζηλῶν ἐζήλωκα τῷ Κυρίῳ παντοκράτορι, ὅτι ἐγκατέλιπον τὴν διαθήκην σου οἱ υἱοὶ Ἰσραήλ· καὶ τὰ θυσιαστήριά σου καθεῖλαν, καὶ τοὺς προφήτας σου ἀπέκτειναν ἐν ῥομφαίᾳ, καὶ ὑπολέλειμμαι ἐγὼ μονώτατος, καὶ ζητοῦσι τὴν ψυχήν μου λαβεῖν αὐτήν.
\VS{15}Καὶ εἶπε Κύριος πρὸς αὐτὸν, πορεύου, ἀνάστρεφε εἰς τὴν ὁδόν σου, καὶ ἥξεις εἰς τὴν ὁδὸν ἐρήμου Δαμασκοῦ· καὶ ἥξεις καὶ χρίσεις τὸν Ἀζαὴλ εἰς βασιλέα τῆς Συρίας·
\VS{16}Καὶ τὸν Ἰοὺ υἱὸν Ναμεσσὶ χρίσεις εἰς βασιλέα ἐπὶ Ἰσραήλ· καὶ τὸν Ἑλισαιὲ υἱὸν Σαφὰτ χρίσεις εἰς προφήτην ἀντὶ σοῦ.
\VS{17}Καὶ ἔσται τὸν σωζόμενον ἐκ ῥομφαίας Ἀζαὴλ, θανατώσει Ἰού· καὶ τὸν σωζόμενον ἐκ ῥομφαίας Ἰοὺ, θανατώσει Ἑλισαιέ.
\VS{18}Καὶ καταλείψεις ἐν Ἰσραὴλ ἑπτὰ χιλιάδας ἀνδρῶν, πάντα γόνατα ἃ οὐκ ὤκλασαν γόνυ τῷ Βάαλ, καὶ πᾶν στόμα ὃ οὐ προσεκύνησεν αὐτῷ.
\par }{\PP \VS{19}Καὶ ἀπῆλθεν ἐκεῖθεν καὶ εὑρίσκει τὸν Ἑλισαιὲ υἱὸν Σαφὰτ, καὶ αὐτὸς ἠροτρία ἐν βουσί· δώδεκα ζεύγη ἐνώπιον αὐτοῦ, καὶ αὐτὸς ἐν τοῖς δώδεκα· καὶ ἀπῆλθεν ἐπʼ αὐτὸν, καὶ ἐπέῥῥιψε τὴν μηλωτὴν αὐτοῦ ἐπʼ αὐτόν.
\VS{20}Καὶ κατέλιπεν Ἑλισαιὲ τὰς βόας, καὶ κατέδραμεν ὀπίσω Ἠλιοὺ, καὶ εἶπε, καταφιλήσω τὸν πατέρα μου, καὶ ἀκολουθήσω ὀπίσω σοῦ· καὶ εἶπεν Ἠλιοὺ, ἀνάστρεφε, ὅτι πεποίηκά σοι.
\VS{21}Καὶ ἀνέστρεψεν ἐξ ὄπισθεν αὐτοῦ· καὶ ἔλαβε τὰ ζεύγη τῶν βοῶν, καὶ ἔθυσε καὶ ἥψησεν αὐτὰ ἐν τοῖς σκεύεσι τῶν βοῶν, καὶ ἔδωκε τῷ λαῷ, καὶ ἔφαγον· καὶ ἀνέστη καὶ ἐπορεύθη ὀπίσω Ἠλιοὺ, καὶ ἐλειτούργει αὐτῷ.

\par }\Chap{20}{\PP \VerseOne{1}Καὶ ἀμπελὼν εἷς ἦν τῷ Ναβουθαὶ τῷ Ἰεζραηλίτῃ παρὰ τῇ ἅλῳ Ἀχαὰβ βασιλέως Σαμαρείας.
\VS{2}Καὶ ἐλάλησεν Ἀχαὰβ πρὸς Ναβουθαὶ, λέγων, δός μοι τὸν ἀμπελῶνά σου, καὶ ἔσται μοι εἰς κῆπον λαχάνων, ὅτι ἐγγίζων οὗτος τῷ οἴκῳ μου, καὶ δώσω σοι ἀμπελῶνα ἄλλον ἀγαθὸν ὑπὲρ αὐτόν· εἰ δὲ ἀρέσκει ἐνώπιόν σου, δώσω σοι ἀργύριον ἄλλαγμα ἀμπελῶνός σου τούτου, καὶ ἔσται μοι εἰς κῆπον λαχάνων.
\VS{3}Καὶ εἶπε Ναβουθαὶ πρὸς Ἀχαάβ, μὴ γένοιτό μοι παρὰ Θεοῦ μου δοῦναι κληρονομίαν πατέρων μου σοί.
\par }{\PP \VS{4}Καὶ ἐγένετο τὸ πνεῦμα Ἀχαὰβ τεταραγμένον, καὶ ἐκοιμήθη ἐπὶ τῆς κλίυης αὐτοῦ, καὶ συνεκάλυψε τὸ πρόσωπον αὐτοῦ, καὶ οὐκ ἔφαγεν ἄρτον.
\VS{5}Καὶ εἰσῆλθεν Ἰεζάβελ ἡ γυνὴ αὐτοῦ πρὸς αὐτὸν, καὶ ἐλάλησε πρὸς αὐτὸν, τί τὸ πνεῦμά σου τεταραγμένον, καὶ οὐκ εἶ σὺ ἐσθίων ἄρτον;
\VS{6}Καὶ εἶπε πρὸς αὐτὴν, ὅτι ἐλάλησα πρὸς Ναβουθαὶ τὸν Ἰεζραηλίτην, λέγων, δός μοι τὸν ἀμπελῶνά σου ἀργυρίου· εἰ δὲ βούλῃ, δώσω σοι ἀμπελῶνα ἄλλον ἀντʼ αὐτοῦ· καὶ εἶπεν οὐ δώσω σοι κληρονομίαν πατέρων μου.
\VS{7}Καὶ εἶπε πρὸς αὐτὸν Ἰεζάβελ ἡ γυνὴ αὐτοῦ, σὺ νῦν οὕτω ποιεῖς βασιλέα ἐπὶ Ἰσραήλ; ἀνάστηθι καὶ φάγε ἄρτον καὶ σαυτοῦ γενοῦ, ἐγὼ δὲ δώσω σοι τὸν ἀμπελῶνα Ναβουθαὶ τοῦ Ἰεζραηλίτου.
\par }{\PP \VS{8}Καὶ ἔγραψε βιβλίον ἐπὶ τῷ ὀνόματι Ἀχαὰβ, καὶ ἐσφραγίσατο τῇ σφραγίδι αὐτοῦ· καὶ ἀπέστειλε τὸ βιβλίον πρὸς τοὺς πρεσβυτέρους καὶ τοὺς ἐλευθέρους τοὺς κατοικοῦντας μετὰ Ναβουθαί.
\VS{9}Καὶ ἐγέγραπτο ἐν τοῖς βιβλίοις, λέγων, νηστεύσατε νηστείαν, καὶ καθίσατε τὸν Ναβουθαὶ ἐν ἀρχῇ τοῦ λαοῦ·
\VS{10}Καὶ ἐγκαθίσατε δύο ἄνδρας υἱοὺς παρανόμων ἐξεναντίας αὐτοῦ, καὶ καταμαρτυρησάτωσαν αὐτοῦ, λέγοντες, εὐλόγησε Θεὸν καὶ βασιλέα· καὶ ἐξαγαγέτωσαν αὐτὸν, καὶ λιθοβολησάτωσαν αὐτὸν, καὶ ἀποθανέτω.
\par }{\PP \VS{11}Καὶ ἐποίησαν οἱ ἄνδρες τῆς πόλεως αὐτοῦ οἱ πρεσβύτεροι καὶ οἱ ἐλεύθεροι οἱ κατοικοῦντες ἐν τῇ πόλει αὐτοῦ, καθὼς ἀπέστειλε πρὸς αὐτοὺς Ἰεζάβελ, καὶ καθὰ ἐγέγραπτο ἐν τοῖς βιβλίοις οἷς ἀπέστειλε πρὸς αὐτούς.
\VS{12}Καὶ ἐκάλεσαν νηστείαν, καὶ ἐκάθισαν τὸν Ναβουθαὶ ἐν ἀρχῇ τοῦ λαοῦ.
\VS{13}Καὶ εἰσῆλθον δύο ἄνδρες υἱοὶ παρανόμων, καὶ ἐκάθισαν ἐξεναντίας αὐτοῦ, καὶ κατεμαρτύρησαν αὐτοῦ, λέγοντες, εὐλόγηκας Θεὸν καὶ βασιλέα· καὶ ἐξήγαγον αὐτὸν ἔξω τῆς πόλεως, καὶ ἐλιθοβόλησαν αὐτὸν ἐν λίθοις, καὶ ἀπέθανε.
\VS{14}Καὶ ἀπέστειλαν πρὸς Ἰεζάβελ, λέγοντες, λελιθοβόληται Ναβουθαὶ, καὶ τέθνηκε.
\par }{\PP \VS{15}Καὶ ἐγένετο ὡς ἤκουσεν Ἰεζάβελ, καὶ εἶπε πρὸς Ἀχαὰβ, ἀνάστα, κληρονόμει τὸν ἀμπελῶνα Ναβουθαὶ τοῦ Ἰεζραηλίτου, ὃς οὐκ ἔδωκέ σοι ἀργυρίου, ὅτι οὐκ ἔστι Ναβουθαὶ ζῶν, ὅτι τέθνηκε.
\VS{16}Καὶ ἐγένετο ὡς ἤκουσεν Ἀχαὰβ ὅτι τέθνηκε Ναβουθαὶ ὁ Ἰεζραηλίτης, καὶ διέῤῥηξε τὰ ἱμάτια αὐτοῦ, καὶ περιεβάλετο σάκκον· καὶ ἐγένετο μετὰ ταῦτα, καὶ ἀνέστη καὶ κατέβη Ἀχαὰβ εἰς τὸν ἀμπελῶνα Ναβουθαὶ τοῦ Ἰεζραηλίτου κληρονομῆσαι αὐτόν.
\par }{\PP \VS{17}Καὶ εἶπε Κύριος πρὸς Ἠλιοὺ τὸν Θεσβίτην, λέγων,
\VS{18}ἀνάστηθι καὶ κατάβηθι εἰς ἀπαντὴν Ἀχαὰβ βασιλέως Ἰσραὴλ τοῦ ἐν Σαμαρείᾳ, ὅτι οὗτος ἐν ἀμπελῶνι Ναβουθαὶ, ὅτι καταβέβηκεν ἐκεῖ κληρονομῆσαι αὐτόν.
\VS{19}Καὶ λαλήσεις πρὸς αὐτὸν, λέγων, τάδε λέγει Κύριος, ὡς σὺ ἐφόνευσας καὶ ἐκληρονόμησας, διὰ τοῦτο τάδε λέγει Κύριος, ἐν παντὶ τόπῳ ᾧ ἔλειξαν αἱ ὗες καὶ οἱ κύνες τὸ αἷμα Ναβουθαὶ, ἐκεῖ λείξουσιν οἱ κύνες τὸ αἷμά σου, καὶ αἱ πόρναι λούσονται ἐν τῷ αἵματί σου.
\VS{20}Καὶ εἶπεν Ἀχαὰβ πρὸς Ἠλιοὺ, εἰ εὕρηκάς με ὁ ἐχθρός μου; καὶ εἶπεν, εὕρηκα· διότι μάτην πέπρασαι ποιῆσαι τὸ πονηρὸν ἐνώπιον Κυρίου, παροργίσαι αὐτόν.
\VS{21}Ἰδοὺ ἐγὼ ἐπάγω ἐπὶ σὲ κακά· καὶ ἐκκαύσω ὀπίσω σου, καὶ ἐξολοθρεύσω τοῦ Ἀχαὰβ οὐροῦντα πρὸς τοῖχον, καὶ συνεχόμενον καὶ ἐγκαταλελειμμένον ἐν Ἰσραήλ.
\VS{22}Καὶ δώσω τὸν οἶκόν σου ὡς τὸν οἶκον Ἱεροβοὰμ υἱοῦ Ναβὰτ, καὶ ὡς τὸν οἶκον Βαασὰ υἱοῦ Ἀχιὰ, περὶ τῶν παροργισμάτων ὧν παρώργισας καὶ ἐξήμαρτες τὸν Ἰσραὴλ.
\VS{23}Καὶ τῇ Ἰεζάβελ ἐλάλησε Κύριος, λέγων, οἱ κύνες καταφάγονται αὐτὴν ἐν τῷ προτειχίσματι τοῦ Ἰεζράελ·
\VS{24}Τὸν τεθνηκότα τοῦ Ἀχαὰβ ἐν τῇ πόλει φάγονται οἱ κύνες, καὶ τὸν τεθνηκότα αὐτοῦ ἐν τῷ πεδίῳ φάγονται τὰ πετεινὰ τοῦ οὐρανοῦ.
\par }{\PP \VS{25}Πλὴν ματαίως Ἀχαὰβ, ὃς ἐπράθη ποιῆσαι τὸ πονηρὸν ἐνώπιον Κυρίου, ὡς μετέθηκεν αὐτὸν Ἰεζάβελ ἡ γυνὴ αὐτοῦ.
\VS{26}Καὶ ἐβδελύχθη σφόδρα πορεύεσθαι ὀπίσω τῶν βδελυγμάτων, κατὰ πάντα ἃ ἐποίησεν ὁ Ἀμοῤῥαῖος, ὃν ἐξωλόθρευσε Κύριος ἀπὸ προσώπου υἱῶν Ἰσραήλ.
\par }{\PP \VS{27}Καὶ ὑπὲρ τοῦ λόγου ὡς κατενύγη Ἀχαὰβ ἀπὸ προσώπου τοῦ Κυρίου, καὶ ἐπορεύετο κλαίων, καὶ διέῤῥηξε τὸν χιτῶνα αὐτοῦ, καὶ ἐζώσατο σάκκον ἐπὶ τὸ σῶμα αὐτοῦ, καὶ ἐνήστευσε· καὶ περιεβάλετο σάκκον ἐν τῇ ἡμέρᾳ ᾗ ἐπάταξε Ναβουθαὶ τὸν Ἰεζραηλίτην, καὶ ἐπορεύθη.
\VS{28}Καὶ ἐγένετο ῥῆμα Κυρίου ἐν χειρὶ δούλου αὐτοῦ Ἠλιοὺ περὶ Ἀχαὰβ, καὶ εἶπε Κύριος,
\VS{29}ἑώρακας ὡς κατενύγη Ἀχαὰβ ἀπὸ προσώπου μου; οὐκ ἐπάξω τὴν κακίαν ἐν ταῖς ἡμέραις αὐτοῦ, ἀλλʼ ἐν ταῖς ἡμέραις τοῦ υἱοῦ αὐτοῦ ἐπάξω τὴν κακίαν.

\par }\Chap{21}{\PP \VerseOne{1}Καὶ συνήθροισεν υἱὸς Ἄδερ πᾶσαν τὴν δύναμιν αὐτοῦ, καὶ ἀνέβη καὶ περιεκάθισεν ἐπὶ Σαμάρειαν, καὶ τριακονταδύο βασιλεῖς μετʼ αὐτοῦ, καὶ πᾶς ἵππος καὶ ἅρμα· καὶ ἀνέβησαν καὶ περιεκάθισαν ἐπὶ Σαμάρειαν, καὶ ἐπολέμησαν ἐπʼ αὐτήν.
\VS{2}Καὶ ἀπέστειλε πρὸς Ἀχαὰβ βασιλέα Ἰσραὴλ εἰς τὴν πόλιν, καὶ εἶπε πρὸς αὐτὸν, τάδε λέγει υἱὸς Ἄδερ,
\VS{3}τὸ ἀργύριόν σου καὶ τὸ χρυσίον σου ἐμόν ἐστι, καὶ αἱ γυναῖκές σου καὶ τὰ τέκνα σου ἐμά ἐστι.
\VS{4}Καὶ ἀπεκρίθη βασιλεὺς Ἰσραὴλ, καὶ εἶπε, καθὼς ἐλάλησας κύριέ μου βασιλεῦ, σὸς ἐγώ εἰμι καὶ πάντα τὰ ἐμά.
\par }{\PP \VS{5}Καὶ ἀνέστρεψαν οἱ ἄγγελοι, καὶ εἶπαν, τάδε λέγει ὁ υἱὸς Ἄδερ, ἐγὼ ἀπέστειλα πρὸς σὲ, λέγων, τὸ ἀργύριόν σου καὶ τὸ χρυσίον σου καὶ τὰς γυναῖκας καὶ τὰ τέκνα σου δώσεις ἐμοὶ,
\VS{6}ὅτι ταύτην τὴν ὥραν αὔριον ἀποστελῶ τοὺς παῖδάς μου πρὸς σὲ, καὶ ἐρευνήσουσι τὸν οἶκόν σου καὶ τοὺς οἴκους τῶν παίδων σου, καὶ ἔσται πάντα τὰ ἐπιθυμήματα τῶν ὀφθαλμῶν αὐτῶν ἐφʼ ἃ ἂν ἐπιβάλωσι τὰς χεῖρας αὐτῶν, καὶ λήψονται.
\VS{7}Καὶ ἐκάλεσεν ὁ βασιλεὺς Ἰσραὴλ πάντας τοὺς πρεσβυτέρους τῆς γῆς, καὶ εἶπε, γνῶτε δὴ καὶ ἴδετε ὅτι κακίαν οὗτος ζητεῖ, ὅτι ἀπέσταλκε πρὸς μὲ περὶ τῶν γυναικῶν μου, καὶ περὶ τῶν υἱῶν μου, καὶ περὶ τῶν θυγατέρων μου· τὸ ἀργύριόν μου καὶ τὸ χρυσίον μου οὐκ ἀπεκώλυσα ἀπʼ αὐτοῦ.
\VS{8}Καὶ εἶπαν αὐτῷ οἱ πρεσβύτεροι καὶ πᾶς ὁ λαὸς, μὴ ἀκούσῃς, καὶ μὴ θελήσῃς.
\VS{9}Καὶ εἶπε τοῖς ἀγγέλοις υἱοῦ Ἄδερ, λέγετε τῷ κυρίῳ ὑμῶν, πάντα ὅσα ἀπέσταλκας πρὸς τὸν δοῦλόν σου ἐν πρώτοις ποιήσω, τὸ δὲ ῥῆμα τοῦτο οὐ δυνήσομαι ποιῆσαι· καὶ ἀπῇραν οἱ ἄνδρες, καὶ ἐπέστρεψαν αὐτῷ λόγον.
\par }{\PP \VS{10}Καὶ ἀπέστειλε πρὸς αὐτὸν υἱὸς Ἄδερ, λέγων, τάδε ποιήσαι μοι ὁ Θεὸς καὶ τάδε προσθείη, εἰ ἐκποιήσει ὁ χοῦς Σαμαρείας ταῖς ἀλώπεξι παντὶ τῷ λαῷ τοῖς πεζοῖς μου.
\VS{11}Καὶ ἀπεκρίθη ὁ βασιλεὺς Ἰσραὴλ, καὶ εἶπεν, ἱκανούσθω· μὴ καυχάσθω ὁ κυρτὸς, ὡς ὁ ὀρθός.
\VS{12}Καὶ ἐγένετο ὅτε ἀπεκρίθη αὐτῷ τὸν λόγον τοῦτον, πίνων ἦν αὐτὸς καὶ πάντες οἱ βασιλεῖς οἱ μετʼ αὐτοῦ ἐν σκηναῖς· καὶ εἶπε τοῖς παισὶν αὐτοῦ, οἰκοδομήσατε χάρακα· καὶ ἔθεντο χάρακα ἐπὶ τὴν πόλιν.
\par }{\PP \VS{13}Καὶ ἰδοὺ προφήτης εἷς προσῆλθε τῷ Ἀχαὰβ βασιλεῖ Ἰσραὴλ, καὶ εἶπε, τάδε λέγει Κύριος, εἰ ἑώρακας τὸν ὄχλον τὸν μέγαν τοῦτον; ἰδοὺ ἐγὼ δίδωμι αὐτὸν σήμερον εἰς χεῖρας σὰς, καὶ γνώσῃ ὅτι ἐγὼ Κύριος.
\VS{14}Καὶ εἶπεν Ἀχαὰβ, ἐν τίνι; καὶ εἶπε, τάδε λέγει Κύριος, ἐν τοῖς παιδαρίοις τῶν ἀρχόντων τῶν χωρῶν· καὶ εἶπεν Ἀχαὰβ, τίς συνάψει τὸν πόλεμον; καὶ εἶπε, σύ.
\par }{\PP \VS{15}Καὶ ἐπεσκέψατο Ἀχαὰβ τοὺς ἄρχοντας τὰ παιδάρια τῶν χωρῶν, καὶ ἐγένοντο διακόσια τριάκοντα· καὶ μετὰ ταῦτα ἐπεσκέψατο τὸν λαὸν πάντα υἱὸν δυνάμεως, ἑπτὰ χιλιάδας.
\VS{16}Καὶ ἐξῆλθε μεσημβρίας, καὶ υἱὸς Ἄδερ πίνων μεθύων ἐν Σοκχὼθ αὐτὸς καὶ οἱ βασιλεῖς, τριάκοντα καὶ δύο βασιλεῖς συμβοηθοὶ αὐτοῦ.
\VS{17}Καὶ ἐξῆλθον ἄρχοντες παιδάρια τῶν χωρῶν ἐν πρώτοις· καὶ ἀποστέλλουσι καὶ ἀπαγγέλλουσι τῷ βασιλεῖ Συρίας, λέγοντες, ἄνδρες ἐξεληλύθασιν ἐκ Σαμαρείας.
\VS{18}Καὶ εἶπεν αὐτοῖς, εἰ εἰς εἰρήνην ἐκπορεύονται, συλλαβεῖν αὐτοὺς ζῶντας· καὶ εἰ εἰς πόλεμον, ζῶντας συλλαβεῖν αὐτούς·
\VS{19}καὶ μὴ ἐξελθάτωσαν ἐκ τῆς πόλεως ἄρχοντα τὰ παιδάρια τῶν χωρῶν. Καὶ ἡ δύναμις ὀπίσω αὐτῶν
\VS{20}ἐπάταξεν ἕκαστος τὸν παρʼ αὐτοῦ· καὶ ἐδευτέρωσιν ἕκαστος τὸν παρʼ αὐτοῦ· καὶ ἔφυγε Συρία· καὶ κατεδίωξεν αὐτοὺς Ἰσραήλ· καὶ σώζεται υἱὸς Ἄδερ βασιλεὺς Συρίας ἐφʼ ἵππου ἱππέως.
\VS{21}Καὶ ἐξῆλθεν βασιλεὺς Ἰσραὴλ, καὶ ἔλαβε πάντας τοὺς ἵππους καὶ τὰ ἅρματα, καὶ ἐπάταξε πληγὴν μεγάλην ἐν Συρίᾳ.
\VS{22}Καὶ προσῆλθεν ὁ προφήτης πρὸς βασιλέα Ἰσραὴλ, καὶ εἶπε, κραταιοῦ καὶ γνῶθι καὶ ἴδε τί ποιήσεις, ὅτι ἐπιστρέφοντος τοῦ ἐνιαυτοῦ υἱὸς Ἄδερ βασιλεὺς Συρίας ἀναβαίνει ἐπὶ σὲ.
\par }{\PP \VS{23}Καὶ οἱ παῖδες βασιλέως Συρίας· καὶ εἶπον, θεὸς ὀρέων Θεὸς Ἰσραὴλ καὶ οὐ θεὸς κοιλάδων, διὰ τοῦτο ἐκραταίωσεν ὑπὲρ ἡμᾶς· ἐὰν δὲ πολεμήσωμεν αὐτοὺς κατʼ εὐθὺ, εἰ μὴν κραταιώσωμεν ὑπὲρ αὐτούς.
\VS{24}Καὶ τὸ ῥῆμα τοῦτο ποίησον· ἀπόστησον τοὺς βασιλεῖς ἕκαστον εἰς τὸν τόπον αὐτῶν, καὶ θοῦ ἀντʼ αὐτῶν σατράπας,
\VS{25}καὶ ἀλλάξομέν σοι δύναμιν κατὰ τὴν δύναμιν τὴν πεσοῦσαν, καὶ ἵππον κατὰ τὴν ἵππον, καὶ ἅρματα κατὰ τὰ ἅρματα, καὶ πολεμήσομεν πρὸς αὐτοὺς κατʼ εὐθὺ, καὶ κραταιώσομεν ὑπὲρ αὐτούς· καὶ ἤκουσε τῆς φωνῆς αὐτοῦ, καὶ ἐποίησεν οὕτως.
\par }{\PP \VS{26}Καὶ ἐγένετο ἐπιστρέψαντος τοῦ ἐνιαυτοῦ, καὶ ἐπεσκέψατο υἱὸς Ἄδερ τὴν Συρίαν, καὶ ἀνέβη εἰς Ἀφεκὰ εἰς πόλεμον ἐπὶ Ἰσραήλ.
\VS{27}Καὶ οἱ υἱοὶ Ἰσραὴλ ἐπεσκέπησαν, καὶ παρεγένοντο εἰς ἀπαντὴν αὐτῶν· καὶ παρενέβαλεν Ἰσραὴλ ἐξεναντίας αὐτῶν ὡσεὶ δύο ποίμνια αἰγῶν· καὶ Συρία ἔπλησε τὴν γῆν.
\par }{\PP \VS{28}Καὶ προσῆλθεν ὁ ἄνθρωπος τοῦ Θεοῦ, καὶ εἶπε τῷ βασιλεῖ Ἰσραὴλ, τάδε λέγει Κύριος, ἀνθʼ ὧν εἶπε Συρία, θεὸς ὀρέων Κύριος ὁ Θεὸς Ἰσραὴλ καὶ οὐ θεὸς κοιλάδων αὐτὸς, καὶ δώσω τὴν δύναμιν τὴν μεγάλην ταύτην εἰς χεῖρα σὴν, καὶ γνώσῃ ὅτι ἐγὼ Κύριος.
\VS{29}Καὶ παρεμβάλλουσιν οὗτοι ἀπέναντι τούτων ἑπτὰ ἡμέρας· καὶ ἐγένετο ἐν τῇ ἡμέρᾳ τῇ ἑβδόμῃ, καὶ προσήγαγεν ὁ πόλεμος, καὶ ἐπάταξεν Ἰσραὴλ τὴν Συρίαν ἑκατὸν χιλιάδας πεζῶν μιᾷ ἡμέρᾳ.
\VS{30}Καὶ ἔφυγον οἱ κατάλοιποι εἰς Ἀφεκὰ εἰς τὴν πόλιν, καὶ ἔπεσε τὸ τεῖχος ἐπὶ εἴκοσι καὶ ἑπτὰ χιλιάδας ἀνδρῶν τῶν καταλοίπων· καὶ υἱὸς Ἄδερ ἔφυγε καὶ εἰσῆλθεν εἰς τὸν οἶκον τοῦ κοιτῶνος, εἰς τὸ ταμιεῖον.
\par }{\PP \VS{31}Καὶ εἶπε τοῖς παισὶν αὐτοῦ, οἶδα ὅτι βασιλεῖς Ἰσραὴλ βασιλεῖς ἐλέους εἰσίν· ἐπιθώμεθα δὴ σάκκους ἐπὶ τὰς ὀσφύας ἡμῶν, καὶ σχοινία ἐπὶ τὰς κεφαλὰς ἡμῶν, καὶ ἐξέλθωμεν πρὸς βασιλεα Ἰσραὴλ, εἴπως ζωογονήσει τὰς ψυχὰς ἡμῶν.
\VS{32}Καὶ περιεζώσαντο σάκκους ἐπὶ τὰς ὀσφύας αὐτῶν, καὶ ἔθεσαν σχοινία ἐπὶ τὰς κεφαλὰς αὐτῶν, καὶ εἶπον τῷ βασιλεῖ Ισραὴλ, δοῦλός σου υἱὸς Ἄδερ λέγει, ζησάτω δὴ ἡ ψυχὴ ἡμῶν· καὶ εἶπεν, εἰ ἔτι ζῇ, ἀδελφός μου ἐστί.
\VS{33}Καὶ οἱ ἄνδρες οἰωνίσαντο, καὶ ἐσπείσαντο· καὶ ἀνελέξαντο τὸν λόγον ἐκ τοῦ στόματος αὐτοῦ, καὶ εἶπον, ἀδελφός σου υἱὸς Ἄδερ· καὶ εἶπεν, εἰσέλθατε καὶ λάβετε αὐτόν· καὶ ἐξῆλθε πρὸς αὐτὸν υἱὸς Ἄδερ, καὶ ἀναβιβάζουσιν αὐτὸν πρὸς αὐτὸν ἐπὶ τὸ ἅρμα.
\VS{34}Καὶ εἶπε πρὸς αὐτὸν, τὰς πόλεις ἃς ἔλαβεν ὁ πατήρ μου παρὰ τοῦ πατρός σου ἀποδώσω σοι· καὶ ἐξόδους θήσεις σεαυτῷ ἐν Δαμασκῷ, καθὼς ἔθετο ὁ πατήρ μου ἐν Σαμαρείᾳ· καὶ ἐγὼ ἐν διαθήκῃ ἐξαποστελῶ σε. Καὶ διέθετο αὐτῷ διαθήκην, καὶ ἐξαπέστειλεν αὐτόν.
\par }{\PP \VS{35}Καὶ ἄνθρωπος εἷς ἐκ τῶν υἱῶν τῶν προφητῶν εἶπε πρὸς τὸν πλησίον αὐτοῦ ἐν λόγῳ Κυρίου, πάταξον δή με· καὶ οὐκ ἠθέλησεν ὁ ἄνθρωπος πατάξαι αὐτόν.
\VS{36}Καὶ εἶπε πρὸς αὐτὸν, ἀνθʼ ὧν οὐκ ἤκουσας τῆς φωνῆς Κυρίου, καὶ ἰδοὺ σὺ ἀποτρέχεις ἀπʼ ἐμοῦ, καὶ πατάξει σε λέων· καὶ ἀπῆλθεν ἀπʼ αὐτοῦ, καὶ εὑρίσκει αὐτὸν λέων, καὶ ἐπάταξεν αὐτόν.
\VS{37}Καὶ εὑρίσκει ἄνθρωπον ἄλλον, καὶ εἶπε, πάταξόν με δή· καὶ ἐπάταξεν αὐτὸν ὁ ἄνθρωπος, πατάξας καὶ συνέτριψε.
\par }{\PP \VS{38}Καὶ ἐπορεύθη ὁ προφήτης καὶ ἔστη τῷ βασιλεῖ Ἰσραὴλ ἐπὶ τῆς ὁδοῦ, καὶ κατεδήσατο ἐν τελαμῶνι τοὺς ὀφθαλμοὺς αὐτοῦ.
\VS{39}Καὶ ἐγένετο ὡς παρεπορεύετο ὁ βασιλεὺς, καὶ οὗτος ἐβόα πρὸς τὸν βασιλέα, καὶ εἶπεν, ὁ δοῦλός σου ἐξῆλθεν ἐπὶ τὴν στρατιὰν τοῦ πολέμου, καὶ ἰδοὺ ἀνὴρ εἰσήγαγε πρὸς μὲ ἄνδρα, καὶ εἶπε πρὸς μὲ, φύλαξον τοῦτον τὸν ἄνδρα· ἐὰν δὲ ἐκπηδῶν ἐκπηδήσῃ, καὶ ἔσται ἡ ψυχή σου ἀντὶ τῆς ψυχῆς αὐτοῦ, ἢ τάλαντον ἀργυρίου στήσεις.
\VS{40}Καὶ ἐγενήθη, περιεβλέψατο ὁ δοῦλός σου ὧδε καὶ ὧδε, καὶ οὗτος οὐκ ἦν· καὶ εἶπε πρὸς αὐτὸν ὁ βασιλεὺς Ἰσραὴλ, ἰδοὺ καὶ τὰ ἔνεδρα παρʼ ἐμοὶ ἐφόνευσας·
\VS{41}Καὶ ἔσπευσε καὶ ἀφεῖλε τὸν τελαμῶνα ἀπὸ τῶν ὀφθαλμῶν αὐτοῦ· καὶ ἐπέγνω αὐτὸν ὁ βασιλεὺς Ἰσραὴλ, ὅτι ἐκ τῶν προφητῶν οὗτος.
\VS{42}Καὶ εἶπε πρὸς αὐτὸν, τάδε λέγει Κύριος, διότι ἐξήνεγκας σὺ ἄνδρα ὀλέθριον ἐκ τῆς χειρός σου, καὶ ἔσται ἡ ψυχή σου ἀντὶ τῆς ψυχῆς αὐτοῦ, καὶ ὁ λαός σου ἀντὶ τοῦ λαοῦ αὐτοῦ.
\VS{43}Καὶ ἀπῆλθεν ὁ βασιλεὺς Ἰσραὴλ συγκεχυμένος καὶ ἐκλελυμένος, καὶ ἔρχεται εἰς Σαμάρειαν.

\par }\Chap{22}{\PP \VerseOne{1}Καὶ ἐκάθισε τὰ τρία ἔτη, καὶ οὐκ ἦν πόλεμος ἀναμέσον Συρίας καὶ ἀναμέσον Ἰσραήλ.
\VS{2}Καὶ ἐγενήθη ἐν τῷ ἐνιαυτῷ τῷ τρίτῳ, καὶ κατέβη Ἰωσαφὰτ βασιλεὺς Ἰούδα πρὸς βασιλέα Ἰσραήλ.
\VS{3}Καὶ εἶπε βασιλεὺς Ἰσραὴλ πρὸς τοὺς παῖδας αὐτοῦ, εἰ οἴδατε ὅτι ἡμῖν Ῥεμμὰθ Γαλαὰδ, καὶ ἡμεῖς σιωπῶμεν λαβεῖν αὐτὴν ἐκ χειρὸς βασιλέως Συρίας;
\VS{4}Καὶ εἶπε βασιλεὺς Ἰσραὴλ πρὸς Ἰωσαφὰτ, ἀναβήσῃ μεθʼ ἡμῶν εἰς Ῥεμμὰθ Γαλαὰδ εἰς πόλεμον;
\VS{5}Καὶ εἶπεν Ἰωσαφὰτ, καθὼς ἐγὼ, καὶ σὺ οὕτως· καθὼς ὁ λαός μου, ὁ λαός σου· καθὼς οἱ ἵπποι μου, οἱ ἵπποι σου.
\par }{\PP Καὶ εἶπεν Ἰωσαφὰτ βασιλεὺς Ἰούδα πρὸς βασιλέα Ἰσραὴλ, ἐπερωτήσατε δὴ σήμερον τὸν Κύριον.
\VS{6}Καὶ συνήθροισεν ὁ βασιλεὺς Ἰσραὴλ πάντας τοὺς προφήτας ὡς τετρακοσίους ἄνδρας, καὶ εἶπεν αὐτοῖς ὁ βασιλεὺς, εἰ πορευθῶ εἰς Ῥεμμὰθ Γαλαὰδ εἰς πόλεμον ἢ ἐπισχῶ; καὶ εἶπον, ἀνάβαινε, καὶ διδοὺς δώσει Κύριος εἰς χεῖρας τοῦ βασιλέως.
\par }{\PP \VS{7}Καὶ εἶπεν Ἰωσαφὰτ πρὸς βασιλέα Ἰσραὴλ, οὐκ ἔστιν ὧδε προφήτης τοῦ κυρίου, καὶ ἐπερωτήσομεν τὸν κύριον διʼ αὐτοῦ;
\VS{8}Καὶ εἶπεν ὁ βασιλεὺς Ἰσραὴλ πρὸς Ἰωσαφὰτ, εἷς ἐστιν ἀνὴρ εἰς τὸ ἐπερωτῆσαι δὀ αὐτοῦ τὸν Κύριον, καὶ ἐγὼ μεμίσηκα αὐτὸν, ὄτι οὐ λαλεῖ περὶ ἐμοῦ καλὰ, ἀλλʼ ἢ κακὰ, Μιχαίας υἱὸς Ἰεμβλαά· καὶ εἶπεν Ἰωσαφὰτ βασιλεὺς Ἰούδα, μὴ λεγέτω ὁ βασιλεὺς οὕτως.
\par }{\PP \VS{9}Καὶ ἐκάλεσεν ὁ βασιλεὺς Ἰσραὴλ εὐνοῦχον ἕνα, καὶ εἶπε, τοτάχος Μιχαίαν υἱὸν Ἰεμβλαά.
\VS{10}Καὶ ὁ βασιλεὺς Ἰσραὴλ καὶ Ἰωσαφὰτ βασιλεὺς Ἰούδα ἐκάθηντο ἀνὴρ ἐπὶ τοῦ θρόνου αὐτοῦ ἔνοπλοι ἐν ταῖς πύλαις Σαμαρείας· καὶ πάντες οἱ προφῆται ἐπροφήτευον ἐνώπιον αὐτῶν.
\VS{11}Καὶ ἐποίησεν ἑαυτῷ Σεδεκίας υἱὸς Χαναὰν κέρατα σιδηρᾶ, καὶ εἶπε, τάδε λέγει Κύριος, ἐν τούτοις κερατιεῖς τὴν Συρίαν ἕως συντελεσθῇ.
\VS{12}Καὶ πάντες οἱ προφῆται ἐπροφήτευον οὕτως, λέγοντες, ἀνάβαινε εἰς Ῥεμμὰθ Γαλαὰδ, καὶ εὐοδώσει, καὶ δώσει Κύριος εἰς χεῖράς σου καὶ τὸν βασιλέα Συρίας.
\par }{\PP \VS{13}Καὶ ὁ ἄγγελος ὁ πορευθεὶς καλέσαι τὸν Μιχαίαν, ἐλάλησεν αὐτῷ, λέγων, ἰδοὺ δὴ λαλοῦσι πάντες οἱ προφῆται ἐν στόματι ἐνὶ καλὰ περὶ τοῦ βασιλέως, γίνου δὴ καὶ σὺ εἰς τοὺς λόγους σου κατὰ τοὺς λόγους ἑνὸς τούτων, καὶ λάλησον καλά.
\VS{14}Καὶ εἶπε Μιχαίας, ζῇ Κύριος, ὅτι ἃ ἐὰν εἴπῃ Κύριος πρὸς μὲ, ταῦτα λαλήσω.
\par }{\PP \VS{15}Καὶ ἦλθε πρὸς τὸν βασιλέα· καὶ εἶπεν αὐτῷ ὁ βασιλεὺς, Μιχαία, εἰ ἀναβῶ εἰς Ῥεμμὰθ Γαλαὰδ εἰς πόλεμον, ἢ ἐπισχῶ; καὶ εἶπεν, ἀνάβαινε, καὶ εὐοδώσει Κύριος εἰς χεῖρα τοῦ βασιλέως.
\VS{16}Καὶ εἶπεν αὐτῷ ὁ βασιλεὺς, ποσάκις ἐγὼ ὁρκίζω σε, ὃπως λαλήσῃς πρὸς μὲ ἀλήθειαν ἐν ὀνόματι Κυρίου;
\VS{17}Καὶ εἶπεν, οὐχ οὕτως· ἑώρακα πάντα τὸν Ἰσραὴλ διεσπαρμένον ἐν τοῖς ὄρεσιν ὡς ποίμνιον ᾧ οὐκ ἔστι ποιμήν· καὶ εἶπε Κύριος, οὐ κύριος τούτοις Θεός; ἕκαστος εἰς τὸν οἶκον αὐτοῦ ἐν εἰρήνῃ ἀναστρεφέτω.
\par }{\PP \VS{18}Καὶ εἶπε βασιλεὺς Ἰσραὴλ πρὸς Ἰωσαφὰτ βασιλέα Ἰούδα, οὐκ εἶπα πρὸς σε, ὅτι οὐ προφητεύει οὗτός μοι καλὰ, διότι ἀλλʼ ἢ κακά;
\VS{19}Καὶ εἶπε Μιχαίας, οὐχ οὕτως· οὐκ ἐγώ· ἄκουε ῥῆμα Κυρίου· οὐχ οὕτως. Εἶδον Θεὸν Ἰσραὴλ καθήμενον ἐπὶ θρόνου αὐτοῦ, καὶ πᾶσα ἡ στρατιὰ τοῦ οὐρανοῦ εἱστήκει περὶ αὐτὸν ἐκ δεξιῶν αὐτοῦ καὶ ἐξ εὐωνύμων αὐτοῦ.
\VS{20}Καὶ εἶπε Κύριος, τίς ἀπατήσει τὸν Ἀχαὰβ βασιλέα Ἰσραὴλ, καὶ ἀναβήσεται, καὶ πεσεῖται ἐν Ῥεμμὰθ Γαλαάδ; καὶ εἶπεν οὗτος οὕτως, καὶ οὗτος οὕτως.
\VS{21}Καὶ ἐξῆλθε πνεῦμα καὶ ἔστη ἐνώπιον Κυρίου, καὶ εἶπεν, ἐγὼ ἀπατήσω αὐτόν.
\VS{22}Καὶ εἶπε πρὸς αὐτὸν Κύριος, ἐν τίνι; καὶ εἶπεν, ἐξελεύσομαι, καὶ ἔσομαι πνεῦμα ψευδὲς εἰς τὸ στόμα πάντων τῶν προφητῶν αὐτοῦ· καὶ εἶπεν, ἀπατήσεις, καί γε δυνήσῃ· ἔξελθε καὶ ποίησον οὕτως.
\VS{23}Καὶ νῦν ἰδοὺ ἔδωκε Κύριος πνεῦμα ψευδὲς ἐν στόματι πάντων τῶν προφητῶν σου τούτων, καὶ Κύριος ἐλάλησεν ἐπὶ σὲ κακά.
\par }{\PP \VS{24}Καὶ προσῆλθε Σεδεκίας υἱὸς Χαναὰν, καὶ ἐπάταξε τὸν Μιχαίαν ἐπὶ τὴν σιαγόνα, καὶ εἶπε, ποῖον πνεῦμα Κύριου τὸ λαλῆσαν ἐν σοί;
\VS{25}Καὶ εἶπε Μιχαίας, ἰδοὺ σὺ ὄψῃ τῇ ἡμέρᾳ ἐκείνῃ, ὅταν εἰσέλθῃς ταμεῖον τοῦ ταμείου τοῦ κρυβῆναι ἐκεῖ.
\VS{26}Καὶ εἶπεν ὁ βασιλεὺς Ἰσραὴλ, λάβετε τὸν Μιχαίαν, καὶ ἀποστρέψατε αὐτὸν πρὸς Σεμὴρ τὸν βασιλέα τῆς πόλεως· καὶ τῷ Ἰωὰς υἱῷ τοῦ βασιλέως
\VS{27}εἶπον θέσθαι τοῦτον ἐν φυλακῇ, καὶ ἐσθίειν αὐτὸν ἄρτον θλίψεως καὶ ὕδωρ θλίψεως ἕως τοῦ ἐπιστρέψαι με ἐν εἰρήνῃ.
\VS{28}Καὶ εἶπε Μιχαίας, ἐὰν ἐπιστρέφων ἐπιστρέψῃς ἐν εἰρήνῃ, οὐ λελάληκε Κύριος ἐν ἐμοί.
\par }{\PP \VS{29}Καὶ ἀνέβη βασιλεὺς Ἰσραὴλ καὶ Ἰωσαφὰτ βασιλεὺς Ἰούδα μετʼ αὐτοῦ εἰς Ῥεμμὰθ Γαλαάδ.
\VS{30}Καὶ εἶπε βασιλεὺς Ἰσραὴλ πρὸς Ἰωσαφὰτ βασιλέα Ἰούδα, συγκαλύψομαι καὶ εἰσελεύσομαι εἰς τὸν πόλεμον, καὶ σὺ ἔνδυσαι τὸν ἱματισμόν μου· καὶ συνεκαλύψατο βασιλεὺς Ἰσραὴλ, καὶ εἰσῆλθεν εἰς τὸν πόλεμον.
\VS{31}Καὶ βασιλεὺς Συρίας ἐνετείλατο τοῖς ἄρχουσι τῶν ἁρμάτων αὐτοῦ τριάκοντα καὶ δυσὶ, λέγων, μὴ πολεμεῖτε μικρὸν καὶ μέγαν, ἀλλʼ ἢ τὸν βασιλέα Ἰσραὴλ μονώτατον.
\VS{32}Καὶ ἐγένετο ὡς εἶδον οἱ ἄρχοντες τῶν ἁρμάτων τὸν Ἰωσαφὰτ βασιλέα Ἰούδα, καὶ αὐτοὶ εἶπαν, φαίνεται βασιλεὺς Ἰσραήλ οὗτος, καὶ ἐκύκλωσαν αὐτὸν πολεμῆσαι· καὶ ἀνέκραξεν Ἰωσαφάτ.
\VS{33}Καὶ ἐγένετο ὡς εἶδον οἱ ἄρχοντες τῶν ἁρμάτων ὅτι οὐκ ἔστι βασιλεὺς Ἰσραὴλ οὗτος, καὶ ἀπέστρεψαν ἀπʼ αὐτοῦ.
\par }{\PP \VS{34}Καὶ ἐπέτεινεν εἷς τὸ τόξον εὐστόχως, καὶ ἐπάταξε τὸν βασιλέα Ἰσραὴλ ἀναμέσον τοῦ πνεύμονος καὶ ἀναμέσον τοῦ θώρακος· καὶ εἶπε τῷ ἡνιόχῳ αὐτοῦ, ἐπίστρεψον τὰς χεῖράς σου καὶ ἐξάγαγέ με ἐκ τοῦ πολέμου, ὅτι τέτρωμαι.
\VS{35}Καὶ ἐτροπώθη ὁ πόλεμος ἐν τῇ ἡμέρᾳ ἐκείνῃ, καὶ ὁ βασιλεὺς ἦν ἑστηκὼς ἐπὶ τοῦ ἅρματος ἐξεναντίας Συρίας ἀπὸ πρωῒ ἕως ἑσπέρας, καὶ ἀπέχυνε τὸ αἷμα ἀπὸ τῆς πληγῆς εἰς τὸν κόλπον τοῦ ἅρματος, καὶ ἀπέθανεν ἑσπέρας, καὶ ἐξεπορεύετο τὸ αἷμα τῆς τροπῆς ἕως τοῦ κόλπου τοῦ ἅρματος.
\VS{36}Καὶ ἔστη ὁ στρατοκῆρυξ δύνοντος τοῦ ἡλίου, λέγων, ἕκαστος εἰς τὴν ἑαυτοῦ πόλιν καὶ εἰς τὴν ἑαυτοῦ γῆν,
\VS{37}ὅτι τέθνηκεν ὁ βασιλεύς· καὶ ἦλθον εἰς Σαμάρειαν, καὶ ἔθαψαν τὸν βασιλέα ἐν Σαμαρείᾳ.
\VS{38}Καὶ ἀπένιψαν τὸ ἅρμα ἐπὶ τὴν κρήνην Σαμαρείας· καὶ ἐξέλιξαν αἱ ὗες καὶ οἱ κύνες τὸ αἷμα, καὶ αἱ πόρναι ἐλούσαντο ἐν τῷ αἵματι, κατὰ τὸ ῥῆμα Κυρίου ὃ ἐλάλησε.
\par }{\PP \VS{39}Καὶ τὰ λοιπὰ τῶν λόγων Ἀχαὰβ καὶ πάντα ἃ ἐποίησε, καὶ οἴκον ἐλεφάντινον ὃν ᾠκοδόμησε, καὶ πάσας τὰς πόλεις ἃς ἐποίησεν, οὐκ ἰδοὺ ταῦτα γέγραπται ἐν βιβλίῳ λόγων τῶν ἡμερῶν τῶν βασιλέων Ἰσραήλ;
\VS{40}Καὶ ἐκοιμήθη Ἀχαὰβ μετὰ τῶν πατέρων αὐτοῦ, καὶ ἐβασίλευσεν Ὀχοζίας υἱὸς αὐτοῦ ἀντʼ αὐτοῦ.
\par }{\PP \VS{41}Καὶ Ἰωσαφὰτ υἱὸς Ἀσὰ ἐβασίλευσεν ἐπὶ Ἰούδαν· ἐν ἔτει τετάρτῷ τοῦ Ἀχαὰβ βασιλέως Ἰσραὴλ ἐβασίλευσεν
\VS{42}Ἰωσαφὰτ, Υἱὸς τρίακοντα καὶ πέντε ἐτῶν ἐν τῷ βασιλεύειν αὐτόν, καὶ εἴκοσι καὶ πέντε ἔτη ἐβασίλευσεν ἐν Ἰερουσαλὴμ· καὶ ὄνομα τῇ μητρὶ αὐτοῦ Ἀζουβὰ θυγάτηρ Σαλαΐ.
\VS{43}Καὶ ἐπορεύθη ἐν πάση ὁδῷ Ἀσὰ τοῦ πατρὸς αὐτοῦ, οὐκ ἐξέκλινεν ἀπʼ αὐτῆς τοῦ ποιῆσαι τὸ εὐθὲς ἐν ὀφθαλμοῖς Κυρίου.
\VS{44}Πλὴν τῶν ὑψηλῶν οὐκ ἐξῇρεν· ἔτι ὁ λαὸς ἐθυσίαζε καὶ ἐθυμίων ἐν τοῖς ὑψηλοῖς.
\VS{45}Καὶ εἰρήνευσεν Ἰωσαφὰτ μετὰ βασιλέως Ἰσραήλ.
\par }{\PP \VS{46}Καὶ τὰ λοιπὰ τῶν λόγων Ἰωσαφὰτ, καὶ αἱ δυναστεῖαι αὐτοῦ ὅσα ἐποίησεν, οὐκ ἰδοὺ ταῦτα γεγραμμένα ἐν βιβλίῳ λόγων τῶν ἡμερῶν βασιλέων Ἰούδα;
\VS{51}Καὶ ἐκοιμήθη Ἰωσαφὰτ μετὰ τῶν πατέρων αὐτοῦ, καὶ ἐτάφη παρὰ τοῖς πατράσιν αὐτοῦ ἐν πόλει Δαυὶδ τοῦ πατρὸς αὐτοῦ, καὶ ἐβασίλευσεν Ἰωρὰμ υἱὸς αὐτοῦ ἀντʼ αὐτοῦ.
\par }{\PP \VS{52}Καὶ Ὀχοζίας υἱὸς Ἀχαὰβ ἐβασίλευσεν ἐπὶ Ἰσραὴλ ἐν Σαμαρείᾳ· ἐν ἔτει ἑπτακαιδεκάτῳ Ἰωσαφὰτ βασιλέως Ἰούδα, Ὀχοζίας υἱὸς Ἀχαὰβ ἐβασίλευσεν ἐν Ἰσραὴλ ἐν Σαμαρείᾳ δύο ἔτη.
\VS{53}Καὶ ἐποίησε τὸ πονηρὸν ἐναντίον Κυρίου, καὶ ἐπορεύθη ἐν ὁδῶ Ἀχαὰβ τοῦ πατρὸς αὐτοῦ καὶ ἐν ὁδῷ Ἰεζάβελ τῆς μητρὸς αὐτοῦ, καὶ ἐν ταῖς ἁμαρτίαις οἴκου Ἱεροβοὰμ υἱοῦ Ναβὰτ ὃς ἐξήμαρτε τὸν Ἰσραήλ·
\VS{54}Καὶ ἐδούλευσε τοῖς Βααλὶμ καὶ προσεκύνησεν αὐτοῖς, καὶ παρώργισε τὸν Κύριον Θεὸν Ἰσραὴλ, κατὰ πάντα τὰ γενόμενα ἔμπροσθεν αὐτοῦ.
\par }