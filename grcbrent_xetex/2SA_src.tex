\NormalFont\ShortTitle{ΒΑΣΙΛΕΙΩΝ Β}
{\MT ΒΑΣΙΛΕΙΩΝ Β

\par }\ChapOne{1}{\PP \VerseOne{1}ΚΑΙ ἐγένετο μετὰ τὸ ἀποθανεῖν Σαοὺλ, καὶ Δαυὶδ ἀνέστρεψε τύπτων τὸν Ἀμαλὴκ, καὶ ἐκάθισε Δαυὶδ ἐν Σεκελὰκ ἡμέρας δύο.
\VS{2}Καὶ ἐγενήθη τῇ ἡμέρᾳ τῇ τρίτῃ, καὶ ἰδοὺ ἀνὴρ ἦλθεν ἐκ τῆς παρεμβολῆς ἐκ τοῦ λαοῦ Σαοὺλ, καὶ τὰ ἱμάτια αὐτοῦ διεῤῥωγότα, καὶ γῆ ἐπὶ τῆς κεφαλῆς αὐτοῦ· καὶ ἐγένετο ἐν τῷ εἰσελθεῖν αὐτὸν πρὸς Δαυὶδ, καὶ ἔπεσεν ἐπὶ τὴν γῆν καὶ προσεκύνησεν αὐτῷ.
\par }{\PP \VS{3}Καὶ εἶπεν αὐτῷ Δαυὶδ, πόθεν σὺ παραγίνῃ; καὶ εἶπεν πρὸς αὐτὸν, ἐκ τῆς παρεμβολῆς Ἰσραὴλ ἐγὼ διασέσωσμαι.
\VS{4}Καὶ εἶπεν αὐτῷ Δαυὶδ, τίς ὁ λόγος οὗτος; ἀπάγγειλόν μοι· καὶ εἶπεν, ὅτι ἔφυγεν ὁ λαὸς ἐκ τοῦ πολέμου, καὶ πεπτώκασι πολλοὶ ἐκ τοῦ λαοῦ καὶ ἀπέθανον, καὶ Σαοὺλ καὶ Ἰωνάθαν ὁ υἱὸς αὐτοῦ ἀπέθανε.
\par }{\PP \VS{5}Καὶ εἶπε Δαυὶδ τῷ παιδαρίῳ τῷ ἀπαγγελλοντι αὐτῷ, πῶς οἶδας ὅτι τέθνηκε Σαοὺλ καὶ Ἰωνάθαν ὁ υἱὸς αὐτοῦ;
\VS{6}Καὶ εἶπε τὸ παιδάριον τὸ ἀπαγγέλλον αὐτῷ, περιπτώματι περιέπεσον ἐν τῷ ὄρει τῷ Γελβουὲ, καὶ ἰδοὺ Σαοὺλ ἐπεστήρικτο ἐπὶ τὸ δόρυ αὐτοῦ, καὶ ἰδοὺ τὰ ἅρματα καὶ οἱ ἱππάρχαι συνῆψαν αὐτῷ.
\VS{7}Καὶ ἐπέβλεψεν ἐπὶ τὰ ὀπίσω αὐτοῦ, καὶ εἶδέ με, καὶ ἐκάλεσέ με· καὶ εἶπα, ἰδοὺ ἐγώ.
\VS{8}Καὶ εἶπέ μοι, τίς εἶ σύ; καὶ εἶπα, Ἀμαληκίτης ἐγώ εἰμι.
\VS{9}Καὶ εἶπε πρὸς μὲ, στῆθι δὴ ἐπάνω μου καὶ θανάτωσόν με, ὅτι κατέσχε με σκότος δεινὸν, ὅτι πᾶσα ἡ ψυχή μου ἐν ἐμοί.
\VS{10}Καὶ ἐπέστην ἐπʼ αὐτὸν καὶ ἐθανάτωσα αὐτὸν, ὅτι ᾔδειν ὅτι οὐ ζήσεται μετὰ τὸ πεσεῖν αὐτόν· καὶ ἔλαβον τὸ βασίλειον τὸ ἐπὶ τὴν κεφαλὴν αὐτοῦ, καὶ τὸν χλιδόνα τὸν ἐπὶ τοῦ βραχίονος αὐτοῦ, καὶ ἐνήνοχα αὐτὰ τῷ κυρίῳ μου ὧδε.
\par }{\PP \VS{11}Καὶ ἐκράτησε Δαυὶδ τῶν ἱματίων αὐτοῦ, καὶ διέῤῥηξεν αὐτά· καὶ πάντες οἱ ἄνδρες οἱ μετʼ αὐτοῦ διέῤῥηξαν τὰ ἱμάτια αὐτῶν.
\VS{12}Καὶ ἐκόψαντο καὶ ἔκλαυσαν καὶ ἐνήστευσαν ἕως δείλης ἐπὶ Σαοὺλ καὶ ἐπὶ Ἰωνάθαν τὸν υἱὸν αὐτοῦ καὶ ἐπὶ τὸν λαὸν Ἰούδα καὶ ἐπὶ τὸν οἶκον Ἰσραὴλ, ὅτι ἐπλήγησαν ἐν ῥομφαίᾳ.
\par }{\PP \VS{13}Καὶ εἶπε Δαυὶδ τῷ παιδαρίῳ τῷ ἀπαγγέλλοντι αὐτῷ, πόθεν εἶ σύ; καὶ εἶπεν, υἱὸς ανδρὸς παροίκου Ἀμαληκίτου ἐγώ εἰμι.
\par }{\PP \VS{14}Καὶ εἶπεν αὐτῷ Δαυὶδ, πῶς οὐκ ἐφοβήθης ἐπενεγκεῖν χεῖρά σου διαφθεῖραι τὸν χριστὸν Κυρίου;
\VS{15}Καὶ ἐκάλεσε Δαυὶδ ἓν τῶν παιδαρίων αὐτοῦ, καὶ εἶπε, προσελθὼν ἀπάντησον αὐτῷ· καὶ ἐπάταξεν αὐτὸν, καὶ ἀπέθανε.
\VS{16}Καὶ εἶπε πρὸς αὐτὸν Δαυὶδ, τὸ αἷμά σου ἐπὶ τὴν κεφαλήν σου, ὅτι τὸ στόμα σου ἀπεκρίθη κατὰ σοῦ, λέγον, ὅτι ἐγὼ ἐθανάτωσα τὸν χριστὸν Κυρίου.
\par }{\PP \VS{17}Καὶ ἐθρήνησε Δαυὶδ τὸν θρῆνον τοῦτον ἐπὶ Σαοὺλ καὶ ἐπὶ Ἰωνάθαν τὸν υἱὸν αὐτοῦ.
\VS{18}Καὶ εἶπε τοῦ διδάξαι τοὺς υἱοὺς Ἰούδα· ἰδοὺ γέγραπται ἐπὶ βιβλίου τοῦ εὐθοῦς.
\par }{\PP \VS{19}Στήλωσον Ἰσραὴλ ὑπὲρ τῶν τεθνηκότων ἐπὶ τὰ ὕψη σου τραυματιῶν· πῶς ἔπεσαν δυνατοί;
\VS{20}Μὴ ἀναγγείλητε ἐν Γὲθ, καὶ μὴ εὐαγγελίσησθε ἐν ταῖς ἐξόδοις Ἀσκάλωνος, μή ποτε εὐφρανθῶσι θυγατέρες ἀλλοφύλων, μή ποτε ἀγαλλιάσωνται θυγατέρες τῶν ἀπεριτμήτων.
\VS{21}Ὄρη τὰ ἐν Γελβουὲ μὴ καταβάτω δρόσος καὶ μὴ ὑετὸς ἐφʼ ὑμᾶς, καὶ ἀγροὶ ἀπαρχῶν, ὅτι ἐκεῖ προσωχθίσθη θυρεὸς δυνατῶν· θυρεὸς Σαοὺλ οὐκ ἐχρίσθη ἐν ἐλαίῳ.
\VS{22}Ἀφʼ αἵματος τραυματιῶν καὶ ἀπὸ στέατος δυνατῶν τόξον Ἰωνάθαν οὐκ ἀπεστράφη κενὸν εἰς τὰ ὀπίσω, καὶ ῥομφαία Σαοὺλ οὐκ ἀνέκαμψε κενή.
\VS{23}Σαοὺλ καὶ Ἰωνάθαν οἱ ἠγαπημένοι καὶ ὡραῖοι οὐ διακεχωρισμένοι, εὐπρεπεῖς ἐν τῇ ζωῇ αὐτῶν, καὶ ἐν τῷ θανάτῳ αὐτῶν οὐ διεχωρίσθησαν· ὑπὲρ ἀετοὺς κοῦφοι, καὶ ὑπὲρ λέοντας ἐκραταιώθησαν.
\VS{24}Θυγατέρες Ἰσραὴλ ἐπὶ Σαοὺλ κλαύσατε, τὸν ἐνδιδύσκοντα ὑμᾶς κόκκινα μετὰ κόσμου ὑμῶν, τὸν ἀναφέροντα κόσμον χρυσοῦν ἐπὶ τὰ ἐνδύματα ὑμῶν.
\VS{25}Πῶς ἔπεσαν δυνατοὶ ἐν μέσῳ τοῦ πολέμου, Ἰωνάθαν ἐπὶ τὰ ὕψη σου τραυματίαι;
\VS{26}Ἀλγῶ ἐπὶ σοὶ, ἀδελφέ μου Ἰωνάθαν, ὡραιώθης μοι σφόδρα, ἐθαυμαστώθη ἡ ἀγάπησίς σου ἐμοὶ ὑπὲρ ἀγάπησιν γυναικῶν.
\VS{27}Πῶς ἔπεσαν δυνατοὶ, καὶ ἀπώλοντο σκεύη πολεμικά;

\par }\Chap{2}{\PP \VerseOne{1}Καὶ ἐγένετο μετὰ ταῦτα καὶ ἐπηρώτησε Δαυὶδ ἐν Κυρίῳ, λέγων, εἰ ἀναβῶ εἰς μίαν τῶν πόλεων Ἰούδα; καὶ εἶπε Κύριος πρὸς αὐτὸν, ἀνάβηθι· καὶ εἶπε Δαυὶδ, ποῦ ἀναβῶ; καὶ εἶπεν, εἰς Χεβρών.
\VS{2}Καὶ ἀνέβη ἐκεῖ Δαυὶδ εἰς Χεβρὼν, καὶ ἀμφότεραι αἱ γυναῖκες αὐτοῦ, Ἀχινάαμ ἡ Ἰεζραηλίτις, καὶ Ἀβιγαία ἡ γυνὴ Νάβαλ τοῦ Καρμηλίου,
\VS{3}καὶ οἱ ἄνδρες οἱ μετʼ αὐτοῦ ἕκαστος, καὶ ὁ οἶκος αὐτοῦ, καὶ κατῴκουν ἐν ταῖς πόλεσι Χεβρών.
\par }{\PP \VS{4}Καὶ ἔρχονται ἄνδρες τῆς Ἰουδαίας, καὶ χρίουσι τὸν Δαυὶδ ἐκεῖ τοῦ βασιλεύειν ἐπὶ τὸν οἶκον Ἰούδα· Καὶ ἀπήγγειλαν τῷ Δαυὶδ, λέγοντες, ὅτι οἱ ἄνδρες Ἰαβὶς τῆς Γαλααδίτιδος ἔθαψαν τὸν Σαούλ.
\VS{5}Καὶ ἀπέστειλε Δαυὶδ ἀγγέλους πρὸς τοὺς ἡγουμένους Ἰαβὶς τῆς Γαλααδίτιδος, καὶ εἶπε πρὸς αὐτοὺς Δαυὶδ, εὐλογημένοι ὑμεῖς τῷ Κυρίῳ, ὅτι ἐποιήσατε τὸ ἔλεος τοῦτο ἐπὶ τὸν κύριον ὑμῶν, ἐπὶ Σαοὺλ τὸν χριστὸν Κυρίου, καὶ ἐθάψατε αὐτὸν καὶ Ἰωνάθαν τὸν υἱὸν αὐτοῦ.
\VS{6}Καὶ νῦν ποιῆσαι Κύριος μεθʼ ὑμῶν ἔλεος καὶ ἀλήθειαν· καί γε ἐγὼ ποιήσω μεθʼ ὑμῶν τὸ ἀγαθὸν τοῦτο, ὅτι ἐποιήσατε τὸ ῥῆμα τοῦτο.
\VS{7}Καὶ νῦν κραταιούσθωσαν αἱ χεῖρες ὑμῶν, καὶ γίνεσθε εἰς υἱοὺς δυνατοὺς, ὅτι τέθνηκεν ὁ κύριος ὑμῶν Σαοὺλ, καί γε ἐμὲ κέχρικεν ὁ οἶκος Ἰούδα ἐφʼ ἑαυτὸν εἰς βασιλέα.
\par }{\PP \VS{8}Καὶ Ἀβεννὴρ υἱὸς Νὴρ ἀρχιστράτηγος τοῦ Σαοὺλ ἔλαβε τὸν Ἰεβοσθὲ υἱὸν Σαοὺλ, καὶ ἀνεβίβασεν αὐτὸν ἐκ τῆς παρεμβολῆς εἰς Μαναὲμ,
\VS{9}καὶ ἐβασίλευσεν αὐτὸν ἐπὶ τὴν Γαλααδίτιν, καὶ ἐπὶ τὸν Θασιρὶ, καὶ ἐπὶ τὸν Ἰεζραὴλ, καὶ ἐπὶ τὸν Ἐφραὶμ, καὶ ἐπὶ τὸν Βενιαμὶν, καὶ ἐπὶ πάντα Ἰσραήλ.
\VS{10}Τεσσαράκοντα ἐτῶν Ἰεβοσθὲ υἱὸς Σαοὺλ, ὅτε ἐβασίλευσεν ἐπὶ τὸν Ἰσραὴλ· καὶ δύο ἔτη ἐβασίλευσε, πλὴν τοῦ οἴκου Ἰούδα, οἳ ἦσαν ὀπίσω Δαυίδ.
\par }{\PP \VS{11}Καὶ ἐγένοντο αἱ ἡμέραι ἃς Δαυὶδ ἐβασίλευσεν ἐν Χεβρὼν ἐπὶ τὸν οἶκον Ἰούδα, ἑπτὰ ἔτη καὶ μῆνας ἕξ.
\par }{\PP \VS{12}Καὶ ἐξῆλθεν Ἀβεννὴρ υἱὸς Νὴρ, καὶ οἱ παῖδες Ἰεβοσθὲ υἱοῦ Σαοὺλ ἐκ Μαναὲμ εἰς Γαβαών·
\VS{13}Καὶ Ἰωὰβ υἱὸς Σαρουία, καὶ οἱ παῖδες Δαυὶδ ἐξῆλθον ἐκ Χεβρὼν, καὶ συναντῶσιν αὐτοῖς ἐπὶ τὴν κρήνην τὴν Γαβαὼν ἐπὶ τὸ αὐτὸ, καὶ ἐκάθισαν οὗτοι ἐπὶ τὴν κρήνην ἐντεῦθεν, καὶ οὗτοι ἐπὶ τὴν κρήνην ἐντεῦθεν.
\VS{14}Καὶ εἶπεν Ἀβεννὴρ πρὸς Ἰωὰβ, ἀναστήτωσαν δὴ τὰ παιδάρια, καὶ παιξάτωσαν ἐνώπιον ὑμῶν· καὶ εἶπεν Ἰωὰβ, ἀναστήτωσαν.
\VS{15}Καὶ ἀνέστησαν καὶ παρῆλθον ἐν ἀριθμῷ τῶν παίδων Βενιαμὶν δώδεκα τῶν Ἰεβοσθὲ υἱοῦ Σαοὺλ, καὶ δώδεκα ἐκ τῶν παίδων Δαυίδ
\VS{16}Καὶ ἐκράτησαν ἕκαστος τῇ χειρὶ τὴν κεφαλὴν τοῦ πλησίον αὐτοῦ, καὶ μάχαιρα αὐτοῦ εἰς πλευρὰν τοῦ πλησίον αὐτοῦ, καὶ πίπτουσι κατὰ τὸ αὐτό· καὶ ἐκλήθη τὸ ὄνομα τοῦ τόπου ἐκείνου, μερὶς τῶν ἐπιβούλων, ἥ ἐστιν ἐν Γαβαών.
\VS{17}Καὶ ἐγένετο ὁ πόλεμος σκληρὸς ὥστε λίαν ἐν τῇ ἡμέρᾳ ἐκείνῃ· καὶ ἔπταισεν Ἀβεννὴρ καὶ ἄνδρες Ἰσραὴλ ἐνώπιον παίδων Δαυίδ.
\VS{18}Καὶ ἐγένοντο ἐκεῖ τρεῖς υἱοὶ Σαρουία, Ἰωὰβ, καὶ Ἀβεσσὰ, καὶ Ἀσαήλ· καὶ Ἀσαὴλ κοῦφος τοῖς ποσὶν αὐτοῦ· ὡσεὶ μία δορκὰς ἐν ἀγρῷ.
\par }{\PP \VS{19}Καὶ κατεδίωξεν Ἀσαὴλ ὀπίσω Ἀβεννὴρ, καὶ οὐκ ἐξέκλινε τοῦ πορεύεσθαι εἰς δεξιὰ οὐδὲ εἰς ἀριστερὰ κατόπισθεν Ἀβεννήρ.
\VS{20}Καὶ ἐπέβλεψεν Ἀβεννὴρ εἰς τὰ ὀπίσω αὐτοῦ, καὶ εἶπεν, εἰ σὺ εἶ αὐτὸς Ἀσαήλ; καὶ εἶπεν, ἐγώ εἰμι.
\VS{21}Καὶ εἶπεν αὐτῷ Ἀβεννὴρ, ἔκκλινον σὺ εἰς τὰ δεξιὰ ἢ εἰς τὰ ἀριστερὰ, καὶ κατάσχε σεαυτῷ ἓν τῶν παιδαρίων, καὶ λάβε σεαυτῷ τὴν πανοπλίαν αὐτοῦ· καὶ οὐκ ἠθέλησεν Ἀσαὴλ ἐκκλῖναι ἐκ τῶν ὄπισθεν αὐτοῦ.
\VS{22}Καὶ προσέθετο ἔτι Ἀβεννὴρ λέγων τῷ Ἀσαὴλ, ἀπόστηθι ἀπʼ ἐμοῦ, ἵνα μὴ πατάξω σε εἰς τὴν γῆν· καὶ πῶς ἀρῶ τὸ πρόσωπόν μου πρὸς Ἰωάβ;
\VS{23}Καὶ ποῦ ἐστι ταῦτα; ἐπίστρεφε πρὸς Ἰωὰβ τὸν ἀδελφόν σου. Καὶ οὐκ ἐβούλετο τοῦ ἀποστῆναι· καὶ τύπτει αὐτὸν Ἀβεννὴρ ἐν τῷ ὀπίσω τοῦ δόρατος ἐπὶ τὴν ψόαν, καὶ διεξῆλθε τὸ δόρυ ἐκ τῶν ὀπίσω αὐτοῦ, καὶ πίπτει ἐκεῖ καὶ ἀποθνήσκει ὑποκάτω αὐτοῦ· καὶ ἐγένετο πᾶς ὁ ἐρχόμενος ἕως τοῦ τόπου οὗ ἔπεσεν ἐκεῖ Ἀσαὴλ καὶ ἀπέθανε, καὶ ὑφίστατο.
\VS{24}Καὶ κατεδίωξεν Ἰωὰβ καὶ Ἀβεσσὰ ὀπίσω Ἀβεννὴρ, καὶ ὁ ἥλιος ἔδυνε· καὶ αὐτοὶ εἰσῆλθον ἕως τοῦ βουνοῦ Ἀμμὰν, ὅ ἐστιν ἐπὶ προσώπου Γαῒ, ὁδὸν ἔρημον Γαβαών.
\par }{\PP \VS{25}Καὶ συναθροίζονται οἱ υἱοὶ Βενιαμὶν οἱ ὀπίσω Ἀβεννὴρ, καὶ ἐγενήθησαν εἰς συνάντησιν μίαν, καὶ ἔστησαν ἐπὶ κεφαλὴν βουνοῦ ἑνός.
\VS{26}Καὶ ἐκάλεσεν Ἀβεννὴρ Ἰωὰβ, καὶ εἶπε, μὴ εἰς νῖκος καταφάγεται ἡ ῥομφαία; ἢ οὐκ οἴδας ὅτι πικρὰ ἔσται εἰς τὰ ἔσχατα; καὶ ἕως πότε οὐ μὴ εἴπῃς τῷ λαῷ ἀποστρέφειν ἀπὸ ὄπισθε τῶν ἀδελφῶν ἡμῶν;
\VS{27}Καὶ εἶπεν Ἰωὰβ, ζῇ Κύριος, ὅτι εἰ μὴ ἐλάλησας, διότι τότε ἐκ πρωϊόθεν ἀνέβη ἂν ὁ λαὸς ἕκαστος κατόπισθε τοῦ ἀδελφοῦ αὐτοῦ.
\VS{28}Καὶ ἐσάλπισεν Ἰωὰβ τῇ σάλπιγγι, καὶ ἀπέστησαν πᾶς ὁ λαὸς, καὶ οὐ κατεδίωξαν ὀπίσω τοῦ Ἰσραὴλ, καὶ οὐ προσέθεντο ἔτι τοῦ πολεμεῖν.
\par }{\PP \VS{29}Καὶ Ἀβεννὴρ καὶ οἱ ἄνδρες αὐτοῦ ἀπῆλθον εἰς δυσμὰς ὅλην τὴν νύκτα ἐκείνην, καὶ διέβαινον τὸν Ἰορδάνην, καὶ ἐπορεύθησαν ὅλην τὴν παρατείνουσαν, καὶ ἔρχονται εἰς τὴν παρεμβολήν.
\VS{30}Καὶ Ἰωὰβ ἀνέστρεψεν ὄπισθεν ἀπὸ τοῦ Ἀβεννὴρ, καὶ συνήθροισε πάντα τὸν λαὸν, καὶ ἐπεσκέπησαν τῶν παίδων Δαυὶδ ἐννεακαίδεκα ἄνδρες,
\VS{31}καὶ Ἀσαήλ. Καὶ οἱ παῖδες Δαυὶδ ἐπάταξαν τῶν υἱῶν Βενιαμὶν τῶν ἀνδρῶν Ἀβεννὴρ τριακοσίους ἑξήκοντα ἄνδρας παρʼ αὐτοῦ.
\par }{\PP \VS{32}Καὶ αἴρουσι τὸν Ἀσαὴλ, καὶ θάπτουσιν αὐτὸν ἐν τῷ τάφῳ τοῦ πατρὸς αὐτοῦ ἐν Βηθλεέμ· καὶ ἐπορεύθη Ἰωὰβ καὶ οἱ ἄνδρες οἱ μετʼ αὐτοῦ ὅλην τὴν νύκτα, καὶ διέφαυσεν αὐτοῖς ἐν Χεβρών.

\par }\Chap{3}{\PP \VerseOne{1}Καὶ ἐγένετο ὁ πόλεμος ἐπὶ πολὺ ἀναμέσον τοῦ οἴκου Σαοὺλ καὶ ἀναμέσον τοῦ οἴκου Δαυίδ· καὶ ὁ οἶκος Δαυὶδ ἐπορεύετο καὶ ἐκραταιοῦτο, καὶ ὁ οἶκος Σαοὺλ ἐπορεύετο καὶ ἠσθένει.
\VS{2}Καὶ ἐτέχθησαν τῷ Δαυὶδ υἱοὶ ἐν Χεβρών· καὶ ἦν ὁ πρωτότοκος αὐτοῦ Ἀμνὼν τῆς Ἀχινόομ τῆς Ἰεζραηλίτιδος.
\VS{3}Καὶ ὁ δεύτερος αὐτοῦ Δαλουΐα τῆς Ἀβιγαίας τῆς Καρμηλίας, καὶ ὁ τρίος, Ἀβεσσαλὼμ υἱὸς Μααχὰ θυγατρὸς Θολμὶ βασιλέως Γεσσὶρ,
\VS{4}καὶ ὁ τέταρτος Ὀρνία υἱὸς Ἀγγὶθ, καὶ ὁ πέμπτος Σαφατία τῆς Ἀβιτὰλ,
\VS{5}καὶ ὁ ἕκτος Ἰεθεραὰμ τῆς Αἰγὰλ γυναικὸς Δαυίδ· οὗτοι ἐτέχθησαν τῷ Δαυὶδ ἐν Χεβρών.
\par }{\PP \VS{6}Καὶ ἐγένετο ἐν τῷ εἶναι τὸν πόλεμον ἀναμέσον τοῦ οἴκου Σαοὺλ καὶ ἀναμέσον τοῦ οἴκου Δαυὶδ, καὶ Ἀβεννὴρ ἦν κρατῶν τοῦ οἴκου Σαούλ.
\VS{7}Καὶ τῷ Σαοὺλ παλλακὴ Ῥεσφὰ θυγάτηρ Ἰώλ· καὶ εἶπεν Ἰεβοσθὲ υἱὸς Σαοὺλ πρὸς Ἀβεννὴρ, τί ὅτι εἰσῆλθες πρὸς τὴν παλλακὴν τοῦ πατρός μου;
\VS{8}Καὶ ἐθυμώθη σφόδρα Ἀβεννὴρ περὶ τοῦ λόγου τούτου τῷ Ἰεβοσθέ. καὶ εἶπεν Ἀβεννὴρ πρὸς αὐτὸν, μὴ κεφαλὴ κυνὸς ἐγώ εἰμι; ἐποίησα σήμερον ἔλεος μετὰ τοῦ οἴκου Σαοὺλ τοῦ πατρός σου, καὶ περὶ ἀδελφῶν καὶ περὶ γνωρίμων, καὶ οὐκ ηὐτομόλησα εἰς τὸν οἶκον Δαυὶδ, καὶ ἐπιζητεῖς ἐπʼ ἐμὲ σὺ ὑπὲρ ἀδικίας γυναικὸς σήμερον;
\VS{9}Τάδε ποιήσαι ὁ Θεὸς τῷ Ἀβεννὴρ καὶ τάδε προσθείη αὐτῷ, ὅτι καθὼς ὤμοσε Κύριος τῷ Δαυὶδ, ὅτι οὕτως ποιήσω αὐτῷ ἐν τῇ ἡμέρᾳ ταύτῃ,
\VS{10}περιελεῖν τὴν βασιλείαν ἀπὸ τοῦ οἴκου Σαοὺλ, καὶ τοῦ ἀναστῆσαι τὸν θρόνον Δαυὶδ ἐπὶ Ἰσραὴλ καὶ ἐπὶ τὸν Ἰούδαν ἀπὸ Δὰν ἕως Βηρσαβεέ.
\VS{11}Καὶ οὐκ ἠδυνάσθη ἔτι Ἰεβοσθὲ ἀποκριθῆναι τῷ Ἀβεννὴρ ῥῆμα, ἀπὸ τοῦ φοβεῖσθαι αὐτόν.
\par }{\PP \VS{12}Καὶ ἀπέστειλεν Ἀβεννὴρ ἀγγέλους πρὸς Δαυὶδ εἰς Θαιλὰμ οὗ ἦν, παραχρῆμα, λέγων, διάθου διαθήκην σου μετʼ ἐμοῦ, καὶ ἰδοὺ ἡ χείρ μου μετὰ σοῦ ἐπιστρέψαι πρὸς σὲ πάντα τὸν οἶκον Ἰσραήλ.
\VS{13}Καὶ εἶπε Δαυὶδ, καλῶς ἐγὼ διαθήσομαι πρὸς σὲ διαθήκην· πλὴν λόγον ἕνα ἐγὼ αἰτοῦμαι παρὰ σοῦ, λέγων, οὐκ ὄψει τὸ πρόσωπόν μου, ἐὰν μὴ ἀγάγῃς τὴν Μελχὸλ θυγατέρα Σαοὺλ παραγινομένου σου ἰδεῖν τὸ πρόσωπόν μου.
\VS{14}Καὶ ἐξαπέστειλε Δαυὶδ πρὸς Ἰεβοσθὲ υἱον Σαοὺλ ἀγγέλους, λέγων, ἀπόδος μοι τῆν γυναῖκά μου τὴν Μελχὸλ, ἣν ἔλαβον ἐν ἑκατὸν ἀκροβυστίαις ἀλλοφύλων.
\VS{15}Καὶ ἀπέστειλεν Ἰεβοσθὲ, καὶ ἔλαβεν αὐτὴν παρὰ τοῦ ἀνδρὸς αὐτῆς παρὰ Φαλτιὴλ υἱοῦ Σελλῆς.
\VS{16}Καὶ ἐπορεύετο ὁ ἀνὴρ αὐτῆς μετʼ αὐτῆς κλαίων ὀπίσω αὐτῆς ἕως Βαρακίμ· καὶ εἶπε πρὸς αὐτὸν Ἀβεννὴρ, πορεύου, ἀνάστρεφε· καὶ ἀνέστρεψε.
\par }{\PP \VS{17}Καὶ εἶπεν Ἀβεννὴρ πρὸς τοὺς πρεσβυτέρους Ἰσραὴλ, λέγων, χθὲς καὶ τρίτην ἐζητεῖτε τὸν Δαυὶδ βασιλεύειν ἐφʼ ὑμᾶς.
\VS{18}Καὶ νῦν ποιήσατε, ὅτι Κύριος ἐλάλησε περὶ Δαυὶδ, λέγων, ἐν χειρὶ τοῦ δούλου μου Δαυὶδ σώσω τὸν Ἰσραὴλ ἐκ χειρὸς ἀλλοφύλων, καὶ ἐκ χειρὸς πάντων τῶν ἐχθρῶν αὐτῶν.
\VS{19}Καὶ ἐλάλησεν Ἀβεννὴρ ἐν τοῖς ὠσὶ Βενιαμίν· καὶ ἐπορεύθη Ἀβεννὴρ τοῦ λαλῆσαι εἰς τὰ ὦτα τοῦ Δαυὶδ εἰς Χεβρὼν πάντα ὅσα ἤρεσεν ἐν ὀφθαλμοῖς Ἰσραὴλ καὶ ἐν ὀφθαλμοῖς οἴκου Βενιαμίν.
\VS{20}Καὶ ἦλθεν Ἀβεννὴρ πρὸς Δαυὶδ εἰς Χεβρὼν, καὶ μετʼ αὐτοῦ εἴκοσι ἄνδρες· καὶ ἐποίησε Δαυὶδ τῷ Ἀβεννὴρ καὶ τοῖς ἀνδράσι τοῖς μετʼ αὐτοῦ πότον.
\VS{21}Καὶ εἶπεν Ἀβεννὴρ πρὸς Δαυὶδ, ἀναστήσομαι δὴ καὶ πορεύσομαι καὶ συναθροίσω πρὸς κύριόν μου τὸν βασιλέα πάντα Ἰσραὴλ· καὶ διαθήσομαι μετʼ αὐτοῦ διαθήκην, καὶ βασιλεύσεις ἐπὶ πᾶσιν οἷς ἐπιθυμεῖ ἡ ψυχή σου. Καὶ ἀπέστειλε Δαυὶδ τὸν Ἀβεννὴρ, καὶ ἐπορεύθη ἐν εἰρήνῃ.
\par }{\PP \VS{22}Καὶ ἰδοὺ οἱ παῖδες Δαυὶδ καὶ Ἰωὰβ παρεγένοντο ἐκ τῆς ἐξοδίας, καὶ σκῦλα πολλὰ ἔφερον μεθʼ ἑαυτῶν· καὶ Ἀβεννὴρ οὐκ ἦν μετὰ Δαυὶδ εἰς Χεβρὼν, ὅτι ἀπεστάλκει αὐτὸν, καὶ ἀπεληλύθει ἐν εἰρήνῃ.
\VS{23}Καὶ Ἰωὰβ καὶ πᾶσα ἡ στρατιὰ αὐτοῦ ἤλθοσαν, καὶ ἀπηγγέλη τῷ Ἰωὰβ, λέγοντες, ἥκει Ἀβεννὴρ υἱὸς Νὴρ πρὸς Δαυὶδ, καὶ ἀπέσταλκεν αὐτὸν, καὶ ἀπῆλθεν ἐν εἰρήνῃ.
\VS{24}Καὶ εἰσῆλθεν Ἰωὰβ πρὸς τὸν βασιλέα, καὶ εἶπε, τί τοῦτο ἐποίησας; ἰδοὺ ἦλθεν Ἀβεννὴρ πρὸς σὲ, καὶ ἱνατί ἐξαπέσταλκας αὐτὸν, καὶ ἀπελήλυθεν ἐν εἰρήνῃ;
\VS{25}Ἢ οὐκ οἶδας τὴν κακίαν Ἀβεννὴρ υἱοῦ Νὴρ, ὅτι ἀπατῆσαί σε παρεγένετο, καὶ γνῶναι τὴν ἔξοδόν σου καὶ τὴν εἴσοδόν σου, καὶ γνῶναι ἅπαντα ὅσα σὺ ποιεῖς;
\par }{\PP \VS{26}Καὶ ἀνέστρεψεν Ἰωὰβ ἀπὸ τοῦ Δαυὶδ, καὶ ἀπέστειλεν ἀγγέλους πρὸς Ἀβεννὴρ ὀπίσω, καὶ ἐπιστρέφουσιν αὐτὸν ἀπὸ τοῦ φρέατος τοῦ Σεειρὰμ· καὶ Δαυὶδ οὐκ ᾔδει.
\VS{27}Καὶ ἐπέστρεψε τὸν Ἀβεννὴρ εἰς Χεβρὼν, καὶ ἐξέκλινεν αὐτὸν Ἰωὰβ ἐκ πλαγίων τῆς πύλης λαλῆσαι πρὸς αὐτὸν, ἐνεδρεύων· καὶ ἐπάταξεν αὐτὸν ἐκεῖ εἰς τὴν ψόαν, καὶ ἀπέθανεν ἐν τῷ αἵματι Ἀσαὴλ τοῦ ἀδελφοῦ Ἰωάβ.
\par }{\PP \VS{28}Καὶ ἤκουσε Δαυὶδ μετὰ ταῦτα, καὶ εἶπεν, ἀθῶός εἰμι ἐγὼ καὶ ἡ βασιλεία μου ἀπὸ Κυρίου καὶ ἕως αἰῶνος ἀπὸ τῶν αἱμάτων Ἀβεννὴρ υἱοῦ Νήρ.
\VS{29}Καταντησάτωσαν ἐπὶ κεφαλὴν Ἰωὰβ καὶ ἐπὶ πάντα τὸν οἶκον τοῦ πατρὸς αὐτοῦ, καὶ μὴ ἐκλείποι ἐκ τοῦ οἴκου Ἰωὰβ γονοῤῥυὴς, καὶ λεπρὸς, καὶ κρατῶν σκυτάλης, καὶ πίπτων ἐν ῥομφαίᾳ, καὶ ἐλασσούμενος ἄρτοις.
\VS{30}Ἰωὰβ δὲ καὶ Ἀβεσσὰ ὁ ἀδελφὸς αὐτοῦ διαπαρετηροῦντο τὸν Ἀβεννὴρ, ἀνθʼ ὧν ἐθανάτωσε τὸν Ἀσαὴλ τὸν ἀδελφὸν αὐτῶν ἐν Γαβαὼν, ἐν τῷ πολέμῳ.
\par }{\PP \VS{31}Καὶ εἶπε Δαυὶδ πρὸς Ἰωὰβ καὶ πρὸς πάντα τὸν λαὸν τὸν μετʼ αὐτοῦ, διαῤῥήξατε τὰ ἱμάτια ὑμῶν, καὶ περιζώσασθε σάκκους, καὶ κόπτεσθε ἐνώπιον Ἀβεννήρ· καὶ ὁ βασιλεὺς Δαυὶδ ἐπορεύετο ὀπίσω τῆς κλίνης.
\VS{32}Καὶ θάπτουσι τὸν Ἀβεννὴρ ἐν Χεβρών· καὶ ᾖρεν ὁ βασιλεὺς τὴν φωνὴν αὐτοῦ καὶ ἔκλαυσεν ἐπὶ τοῦ τάφου αὐτοῦ, καὶ ἔκλαυσε πᾶς ὁ λαὸς ἐπὶ Ἀβεννήρ.
\par }{\PP \VS{33}Καὶ ἐθρήνησεν ὁ βασιλεὺς ἐπὶ Ἀβεννὴρ, καὶ εἶπεν, εἰ κατὰ τὸν θάνατον Νάβαλ ἀποθανεῖται Ἀβεννήρ;
\VS{34}Αἱ χεῖρές σου οὐκ ἐδέθησαν, οἱ πόδες σου οὐκ ἐν πέδαις· οὐ προσήγαγεν ὡς Νάβαλ, ἐνώπιον υἱῶν ἀδικίας ἔπεσας· καὶ συνήχθη πᾶς ὁ λαὸς τοῦ κλαῦσαι αὐτόν.
\VS{35}Καὶ ἦλθε πᾶς ὁ λαὸς περιδειπνῆσαι τὸν Δαυὶδ ἄρτοις ἔτι οὔσης ἡμέρας· καὶ ὤμοσε Δαυὶδ, λέγων, τάδε ποιήσαι μοι ὁ Θεὸς καὶ τάδε προσθείη, ὅτι ἐὰν μὴ δύῃ ὁ ἥλιος, οὐ μὴ γεύσομαι ἄρτου ἢ ἀπὸ παντὸς τινός.
\VS{36}Καὶ ἔγνω πᾶς ὁ λαὸς, καὶ ἤρεσεν ἐνώπιον αὐτῶν πάντα ὅσα ἐποίησεν ὁ βασιλεὺς ἐνώπιον τοῦ λαοῦ.
\VS{37}Καὶ ἔγνω πᾶς ὁ λαὸς καὶ πᾶς Ἰσραὴλ ἐν τῇ ἡμέρᾳ ἐκείνῃ, ὅτι οὐκ ἐγένετο παρὰ τοῦ βασιλέως θανατῶσαι τὸν Ἀβεννὴρ υἱὸν Νήρ.
\par }{\PP \VS{38}Καὶ εἶπεν ὁ βασιλεὺς πρὸς τοὺς παῖδας αὐτοῦ, οὐκ οἴδατε, ὅτι ἡγούμενος μέγας πέπτωκεν ἐν τῇ ἡμέρᾳ ταύτῃ ἐν τῷ Ἰσραήλ;
\VS{39}Καὶ ὅτι ἐγώ εἰμι συγγενὴς σήμερον, καὶ καθεσταμένος ὑπὸ βασιλέως; οἱ δὲ ἄνδρες οὗτοι υἱοὶ Σαρουίας σκληρότεροί μου εἰσίν· ἀποδῷ Κύριος τῷ ποιοῦντι τὰ πονηρὰ κατὰ τὴν κακίαν αὐτοῦ.

\par }\Chap{4}{\PP \VerseOne{1}Καὶ ἤκουσεν Ἰεβοσθὲ υἱὸς Σαοὺλ, ὅτι τέθνηκεν Ἀβεννὴρ υἱὸς Νὴρ ἐν Χεβρὼν, καὶ ἐξελύθησαν αἱ χεῖρες αὐτοῦ, καὶ πάντες οἱ ἄνδρες Ἰσραὴλ παρείθησαν.
\VS{2}Καὶ δύο ἄνδρες ἡγούμενοι συστρεμμάτων τῷ Ἰεβοσθὲ υἱῷ Σαούλ· ὄνομα τῷ ἑνὶ Βαανὰ, καὶ ὄνομα τῷ δευτέρῳ Ῥηχὰβ, υἱοὶ Ῥεμμὼν τοῦ Βηρωθαίου ἐκ τῶν υἱῶν Βενιαμίν· ὅτι Βηρὼθ ἐλογίζετο τοῖς υἱοῖς Βενιαμίν·
\VS{3}Καὶ ἀπέδρασαν οἱ Βηρωθαῖοι εἰς Γεθαὶμ, καὶ ἦσαν ἐκεῖ παροικοῦντες ἕως τῆς ἡμέρας ταύτης.
\par }{\PP \VS{4}Καὶ τῷ Ἰωνάθαν υἱῷ Σαοὺλ υἱὸς πεπληγὼς τοὺς πόδας υἱὸς ἐτῶν πέντε, καὶ οὗτος ἐν τῷ ἐλθεῖν τὴν ἀγγελίαν Σαοὺλ καὶ Ἰωνάθαν τοῦ υἱοῦ αὐτοῦ ἐξ Ἰεζραὴλ, καὶ ᾖρεν αὐτὸν ἡ τιθηνὸς αὐτοῦ, καὶ ἔφυγε· καὶ ἐγένετο ἐν τῷ σπεύδειν αὐτὸν καὶ ἀναχωρεῖν, καὶ ἔπεσε καὶ ἐχωλάνθη, καὶ ὄνομα αὐτῷ Μεμφιβοσθέ.
\par }{\PP \VS{5}Καὶ ἐπορεύθησαν υἱοὶ Ῥεμμὼν τοῦ Βηρωθαίου Ῥηχὰβ καὶ Βαανὰ, καὶ εἰσῆλθον εν τῷ καύματι τῆς ἡμέρας εἰς οἶκον Ἰεβοσθὲ, καὶ αὐτὸς ἐκάθευδεν ἐν τῇ κοίτῃ τῆς μεσημβρίας.
\VS{6}Καὶ ἰδοὺ ἡ θυρωρὸς τοῦ οἴκου ἐκάθαιρε πυροὺς, καὶ ἐνύσταξε καὶ ἐκάθευδε· καὶ Ῥηχὰβ καὶ Βαανὰ οἱ ἀδελφοὶ διέλαθον,
\VS{7}καὶ εἰσῆλθον εἰς τὸν οἶκον· καὶ Ἰεβοσθὲ ἐκάθευδεν ἐπὶ τῆς κλίνης αὐτοῦ ἐν τῷ κοιτῶνι αὐτοῦ· καὶ τύπτουσιν αὐτὸν, καὶ θανατοῦσιν αὐτὸν, καὶ ἀφαιροῦσι τὴν κεφαλὴν αὐτοῦ· καὶ ἔλαβον τὴν κεφαλὴν αὐτοῦ, καὶ ἀπῆλθον ὁδὸν τὴν κατὰ δυσμὰς ὅλην τὴν νύκτα.
\par }{\PP \VS{8}Καὶ ἤνεγκαν τὴν κεφαλὴν Ἰεβοσθὲ τῷ Δαυὶδ εἰς Χεβρὼν, καὶ εἶπον πρὸς τὸν βασιλέα, ἰδοὺ ἡ κεφαλὴ Ἰεβοσθὲ υἱοῦ Σαοὺλ τοῦ ἐχθροῦ σου, ὃς ἐζήτει τὴν ψυχήν σου, καὶ ἔδωκε Κύριος τῷ κυρίῳ βασιλεῖ ἐκδίκησιν τῶν ἐχθρῶν αὐτοῦ, ὡς ἡ ἡμέρα αὕτη, ἐκ Σαοὺλ τοῦ ἐχθροῦ σου καὶ ἐκ τοῦ σπέρματος αὐτοῦ.
\par }{\PP \VS{9}Καὶ ἀπεκρίθη Δαυὶδ τῷ Ῥηχὰβ καὶ τῷ Βαανὰ ἀδελφῷ αὐτοῦ υἱοῖς Ῥεμμὼν τοῦ Βηρωθαίου, καὶ εἶπεν αὐτοῖς, ζῇ Κύριος, ὃς ἐλυτρώσατο τὴν ψυχήν μου ἐκ πάσης θλίψεως, ὅτι ὁ ἀπαγγείλας μοι ὅτι τέθνηκε Σαοὺλ,
\VS{10}καὶ αὐτὸς ἦν ὡς εὐαγγελιζόμενος ἐνώπιόν μου, καὶ κατέσχον αὐτὸν καὶ ἀπέκτεινα αὐτὸν ἐν Σεκελὰκ, ᾧ ἔδει με δοῦναι εὐαγγέλια.
\VS{11}Καὶ νῦν ἄνδρες πονηροὶ ἀπεκτάγκασιν ἄνδρα δίκαιον ἐν τῷ οἴκῳ αὐτοῦ ἐπὶ τῆς κοίτης αὐτοῦ· καὶ νῦν ἐκζητήσω τὸ αἷμα αὐτοῦ ἐκ χειρὸς ὑμῶν, καὶ ἐξολοθρεύσω ὑμᾶς ἐκ τῆς γῆς.
\VS{12}Καὶ ἐνετείλατο Δαυὶδ τοῖς παιδαρίοις αὐτοῦ, καὶ ἀποκτείνουσιν αὐτούς, καὶ κολοβοῦσι τὰς χεῖρας αὐτῶν καὶ τοὺς πόδας αὐτῶν, καὶ ἐκρέμασαν αὐτοὺς ἐπι τῆς κρήνης ἐν Χεβρὼν, καὶ τὴν κεφαλὴν Ἰεβοσθὲ ἔθαψαν ἐν τῷ τάφῳ Ἀβεννὴρ υἱοῦ Νήρ.

\par }\Chap{5}{\PP \VerseOne{1}Καὶ παραγίνονται πᾶσαι αἱ φυλαὶ Ἰσραὴλ πρὸς Δαυὶδ εἰς Χεβρὼν, καὶ εἶπαν αὐτῷ, ἰδοὺ ὀστᾶ σου, καὶ σάρκες σου ἡμεῖς.
\VS{2}Καὶ ἐχθὲς καὶ τρίτην ὄντος Σαοὺλ βασιλέως ἐφʼ ἡμῖν, σὺ ἦσθα ὁ ἐξάγων καὶ εἰσάγων τὸν Ἰσραήλ· καὶ εἶπε Κύριος πρὸς σὲ, σὺ ποιμανεῖς τὸν λαόν μου τὸν Ἰσραὴλ, καὶ σὺ ἔσῃ εἰς ἡγούμενον ἐπὶ τὸν λαόν μου Ἰσραήλ.
\VS{3}Καὶ ἔρχονται πάντες οἱ πρεσβύτεροι Ἰσραὴλ πρὸς τὸν βασιλέα εἰς Χεβρὼν, καὶ διέθετο αὐτοῖς ὁ βασιλεὺς Δαυὶδ διαθήκην ἐν Χεβρὼν ἐνώπιον Κυρίου· καὶ χρίουσι τὸν Δαυὶδ εἰς βασιλέα ἐπὶ πάντα Ἰσραήλ.
\VS{4}Υἱὸς τριάκοντα ἐτῶν Δαυὶδ ἐν τῷ βασιλεῦσαι αὐτόν, καὶ τεσσαράκοντα ἔτη ἐβασίλευσεν.
\VS{5}Ἑπτὰ ἔτη καὶ μῆνας ἓξ ἐβασίλευσεν ἐν Χεβρὼν ἐπὶ τὸν Ἰούδαν, καὶ τριάκοντα τρία ἔτη ἐβασίλευσεν ἐπὶ πάντα Ἰσραὴλ καὶ Ἰούδαν ἐν Ἱερουσαλήμ.
\par }{\PP \VS{6}Καὶ ἀπῆλθε Δαυὶδ καὶ οἱ ἄνδρες αὐτοῦ εἰς Ἰερουσαλὴμ πρὸς τὸν Ἱεβουσαῖον τὸν κατοικοῦντα τὴν γῆν· καὶ ἐῤῥέθη τῷ Δαυὶδ, οὐκ εἰσελεύσῃ ὧδε, ὅτι ἀντέστησαν οἱ τυφλοὶ καὶ οἱ χωλοὶ, λέγοντες, ὅτι οὐκ εἰσελεύσεται Δαυὶδ ὧδε.
\VS{7}Καὶ προκατελάβετο Δαυὶδ τὴν περιοχὴν Σιών· αὕτη ἡ πόλις τοῦ Δαυίδ.
\VS{8}Καὶ εἶπε Δαυὶδ τῇ ἡμέρᾳ ἐκείνῃ, πᾶς τύπτων Ἰεβουσαῖον, ἁπτέσθω ἐν παραξιφίδι καὶ τοὺς χωλοὺς καὶ τοὺς τυφλοὺς καὶ τοὺς μισοῦντας τὴν ψυχὴν Δαυίδ. Διὰ τοῦτο ἐροῦσι, τυφλοὶ καὶ χωλοὶ οὐκ εἰσελεύσοντάι εἰς οἶκον Κυρίου.
\VS{9}Καὶ ἐκάθισε Δαυὶδ ἐν τῇ περιοχῇ, καὶ ἐκλήθη αὕτη ἡ πόλις Δαυίδ· καὶ ᾠκοδόμησεν αὐτὴν πόλιν κύκλῳ ἀπὸ τῆς ἄκρας, καὶ τὸν οἶκον αὐτοῦ.
\VS{10}Καὶ διεπορεύετο Δαυὶδ πορευόμενος καὶ μεγαλυνόμενος, καὶ Κύριος παντοκράτωρ μετʼ αὐτοῦ.
\par }{\PP \VS{11}Καὶ ἀπέστειλε Χειρὰμ βασιλεὺς Τύρου ἀγγέλους πρὸς Δαυὶδ, καὶ ξύλα κέδρινα, καὶ τέκτονας ξύλων, καὶ τέκτονας λίθων, καὶ ᾠκοδόμησαν οἶκον τῷ Δαυίδ.
\VS{12}Καὶ ἔγνω Δαυὶδ, ὅτι ἡτοίμασεν αὐτὸν Κύριος εἰς βασιλέα ἐπὶ Ἰσραὴλ, καὶ ὅτι ἐπῄρθη ἡ βασιλεία αὐτοῦ διὰ τὸν λαὸν αὐτοῦ Ἰσραήλ.
\par }{\PP \VS{13}Καὶ ἔλαβε Δαυὶδ ἔτι γυναῖκας καὶ παλλακὰς ἐξ Ἰερουσαλὴμ, μετὰ τὸ ἐλθεῖν αὐτὸν ἐκ Χεβρών· καὶ ἐγένοντο τῷ Δαυὶδ ἔτι υἱοὶ καὶ θυγατέρες.
\VS{14}Καὶ ταῦτα τὰ ὀνόματα τῶν γεννηθέντων αὐτῷ ἐν Ἱερουσαλήμ· Σαμμοὺς, καὶ Σωβὰβ, καὶ Νάθαν, καὶ Σαλωμὼν,
\VS{15}καὶ Ἐβεὰρ, καὶ Ἐλισουὲ, καὶ Ναφὲκ, καὶ Ἰεφιὲς,
\VS{16}καὶ Ἐλισαμὰ, καὶ Ἐλιδαὲ, καὶ Ἐλιφαλὰθ,
\VS{16a}Σαμαὲ, Ἰεσσιβὰθ, Νάθαν, Γαλαμαὰν, Ἰεβαὰρ, Θεησοῦς, Ἐλιφαλὰτ, Ναγὲδ, Ναφὲκ, Ἰανάθαν, Λεασαμὺς, Βααλιμὰθ, Ἐλιφαάθ.
\par }{\PP \VS{17}Καὶ ἤκουσαν οἱ ἀλλόφυλοι ὅτι κέχρισται Δαυὶδ βασιλεὺς ἐπὶ Ἰσραὴλ, καὶ ἀνέβησαν πάντες οἱ ἀλλόφυλοι ζητεῖν τὸν Δαυίδ· καὶ ἤκουσε Δαυὶδ, καὶ κατέβη εἰς τὴν περιοχήν.
\VS{18}Καὶ οἱ ἀλλόφυλοι παραγίνονται, καὶ συνέπεσαν εἰς τὴν κοιλάδα τῶν Τιτάνων.
\par }{\PP \VS{19}Καὶ ἠρώτησε Δαυὶδ διὰ Κυρίου, λέγων, εἰ ἀναβῶ πρὸς τοὺς ἀλλοφύλους, καὶ παραδώσεις αὐτοὺς εἰς τὰς χεῖράς μου; καὶ εἶπε Κύριος πρὸς Δαυὶδ, ἀνάβαινε, ὅτι παραδιδοὺς παραδώσω τοὺς ἀλλοφύλους εἰς τὰς χεῖράς σου.
\VS{20}Καὶ ἦλθε Δαυὶδ ἐκ τῶν ἐπάνω διακοπῶν, καὶ ἔκοψε τοὺς ἀλλοφύλους ἐκεῖ· καὶ εἶπε Δαυὶδ, διέκοψε Κύριος τοὺς ἐχθροὺς ἀλλοφύλους ἐνώπιον ἐμοῦ, ὡς διακόπτεται ὕδατα· διὰ τοῦτο ἐκλήθη τὸ ὄνομα τοῦ τόπου ἐκείνου, Ἐπάνω διακοπῶν.
\VS{21}Καὶ καταλιμπάνουσιν ἐκεῖ τοὺς θεοὺς αὐτῶν, καὶ ἐλάβοσαν αὐτοὺς Δαυὶδ καὶ οἱ ἄνδρες οἱ μετʼ αὐτοῦ.
\par }{\PP \VS{22}Καὶ προσέθεντο ἔτι ἀλλόφυλοι τοῦ ἀναβῆναι, καὶ συνέπεσαν ἐν τῇ κοιλάδι τῶν Τιτάνων.
\VS{23}Καὶ ἐπηρώτησε Δαυὶδ διὰ Κυρίου· καὶ εἶπε Κύριος, οὐκ ἀναβήσῃ εἰς συνάντησιν αὐτῶν, ἀποστρέφου ἀπʼ αὐτῶν, καὶ παρέσῃ αὐτοῖς πλησίον τοῦ κλαυθμῶνος.
\VS{24}Καὶ ἔσται ἐν τῷ ἀκοῦσαί σε τὴν φωνὴν τοῦ συγκλεισμοῦ ἀπὸ τοῦ ἄλσους τοῦ κλαυθμῶνος, τότε καταβήσῃ πρὸς αὐτούς, ὅτι τότε ἐξελεύσεται Κύριος ἔμπροσθέν σου κόπτειν ἐν τῷ πολέμῳ τῶν ἀλλοφύλων.
\VS{25}Καὶ ἐποιήσε Δαυὶδ καθὼς ἐνετείλατο αὐτῷ Κύριος, καὶ ἐπάταξε τοὺς ἀλλοφύλους ἀπὸ Γαβαὼν ἕως τῆς γῆς Γαζηρά.

\par }\Chap{6}{\PP \VerseOne{1}Καὶ συνήγαγεν ἔτι Δαυὶδ πάντα νεανίαν ἐξ Ἰσραὴλ, ὡς ἑβδομήκοντα χιλιάδας.
\VS{2}Καὶ ἀνέστη καὶ ἐπορεύθη Δαυὶδ καὶ πᾶς ὁ λαὸς ὁ μετʼ αὐτοῦ καὶ ἀπὸ τῶν ἀρχόντων Ἰούδα ἐν ἀναβάσει τοῦ ἀναγαγεῖν ἐκεῖθεν τὴν κιβωτὸν τοῦ Θεοῦ, ἐφʼ ἣν ἐπεκλήθη τὸ ὄνομα τοῦ Κυρίου τῶν δυνάμεων καθημένου ἐπὶ τῶν χερουβὶν ἐπʼ αὐτῆς.
\par }{\PP \VS{3}Καὶ ἐπεβίβασεν τὴν κιβωτὸν Κυρίου ἐφʼ ἅμαξαν καινὴν, καὶ ᾖραν αὐτὴν ἐξ οἴκου Ἀμιναδὰβ τοῦ ἐν τῷ βουνῷ· καὶ Ὀζὰ καὶ οἱ ἀδελφοὶ αὐτοῦ υἱοὶ Ἀμιναδὰβ ἦγαν τὴν ἅμαξαν σὺν τῇ κιβωτῷ.
\VS{4}Καὶ οἱ ἀδελφοὶ αὐτοῦ ἐπορεύοντο ἔμπροσθεν τῆς κιβωτοῦ.
\VS{5}Καὶ Δαυὶδ καὶ υἱοὶ Ἰσραὴλ παίζοντες ἐνώπιον Κυρίου ἐν ὀργάνοις ἡρμοσμένοις ἐν ἰσχύϊ, καὶ ἐν ᾠδαῖς, καὶ ἐν κινύραις, καὶ ἐν νάβλαις, καὶ ἐν τυμπάνοις, καὶ ἐν κυμβάλοις, καὶ ἐν αὐλοῖς.
\par }{\PP \VS{6}Καὶ παραγίνονται ἕως ἅλω Ναχώρ· καὶ ἐξέτεινεν Ὀζὰ τὴν χεῖρα αὐτοῦ ἐπὶ τὴν κιβωτὸν τοῦ Θεοῦ κατασχεῖν αὐτὴν, καὶ ἐκράτησεν αὐτὴν, ὅτι περιέσπασεν αὐτὴν ὁ μόσχος.
\VS{7}Καὶ ἐθυμώθη ὀργῇ Κύριος τῷ Ὀζᾷ, καὶ ἔπαισεν αὐτὸν ἐκεῖ ὁ Θεός, καὶ ἀπέθανεν ἐκεῖ παρὰ τὴν κιβωτὸν τοῦ Κυρίου ἐνώπιον τοῦ Θεοῦ.
\VS{8}Καὶ ἠθύμησε Δαυὶδ ὑπὲρ οὗ διέκοψε Κύριος διακοπὴν ἐν τῷ Ὀζᾷ, καὶ ἐκλήθη ὁ τόπος ἐκεῖνος, διακοπὴ Ὀζᾶ, ἕως τῆς ἡμέρας ταύτης.
\VS{9}Καὶ ἐφοβήθη Δαυὶδ τὸν Κύριον ἐν τῇ ἡμέρᾳ ἐκείνῃ, λέγων, πῶς εἰσελεύσεται πρὸς μὲ ἡ κιβωτὸς Κυρίου;
\VS{10}Καὶ οὐκ ἐβούλετο Δαυὶδ τοῦ ἐκκλῖναι πρὸς αὐτὸν τὴν κιβωτὸν διαθήκης Κυρίου εἰς τὴν πόλιν Δαυίδ· καὶ ἀπέκλινεν αὐτὴν Δαυὶδ εἰς οἶκον Ἀβεδδαρὰ τοῦ Γεθαίου.
\VS{11}Καὶ ἐκάθισεν ἡ κιβωτὸς τοῦ Κυρίου εἰς οἶκον Ἀβεδδαρὰ τοῦ Γεθαίου μῆνας τρεῖς· καὶ εὐλόγησε Κύριος ὅλον τὸν οἶκον Ἀβεδδαρὰ, καὶ πάντα τὰ αὐτοῦ.
\par }{\PP \VS{12}Καὶ ἀπηγγέλη τῷ βασιλεῖ Δαυὶδ, λέγοντες, εὐλόγησε Κύριος τὸν οἶκον Ἀβεδδαρὰ, καὶ πάντα τὰ αὐτοῦ, ἕνεκα τῆς κιβωτοῦ τοῦ Θεοῦ· καὶ ἐπορεύθη Δαυὶδ, καὶ ἀνήγαγε τὴν κιβωτὸν τοῦ Κυρίου ἐκ τοῦ οἴκου Ἀβεδδαρὰ εἰς τὴν πόλιν Δαυὶδ ἐν εὐφροσύνῃ.
\VS{13}Καὶ ἦσαν μετʼ αὐτοῦ αἴροντες τὴν κιβωτὸν ἑπτὰ χοροὶ, καὶ θύμα μόσχος καὶ ἄρνες.
\VS{14}Καὶ Δαυὶδ ἀνεκρούετο ἐν ὀργάνοις ἡρμοσμένοις ἐνώπιον Κυρίου, καὶ ὁ Δαυὶδ ἐνδεδυκὼς στολὴν ἔξαλλον.
\VS{15}Καὶ Δαυὶδ καὶ πᾶς ὁ οἶκος Ἰσραὴλ ἀνήγαγον τὴν κιβωτὸν Κυρίου μετὰ κραυγῆς καὶ μετὰ φωνῆς σάλπιγγος.
\par }{\PP \VS{16}Καὶ ἐγένετο τῆς κιβωτοῦ παραγινομένης ἕως πόλεως Δαυίδ, καὶ Μελχὸλ ἡ θυγάτηρ Σαοὺλ διέκυπτε διὰ τῆς θυρίδος, καὶ εἶδε τὸν βασιλέα Δαυὶδ ὀρχούμενον, καὶ ἀνακρουόμενον ἐνώπιον Κυρίου, καὶ ἐξουδένωσεν αὐτὸν ἐν τῇ καρδίᾳ αὐτῆς.
\par }{\PP \VS{17}Καὶ φέρουσι τὴν κιβωτὸν τοῦ Κυρίου, καὶ ἀνέθηκαν αὐτὴν εἰς τὸν τόπον αὐτῆς εἰς μέσον τῆς σκηνῆς, ἧς ἔπηξεν αὐτῇ Δαυὶδ· καὶ ἀνήνεγκε Δαυιδ ὁλοκαυτώματα ἐνώπιον Κυρίου, εἰρηνικάς.
\VS{18}Καὶ συνετέλεσε Δαυὶδ συναναφέρων τὰς ὁλοκαυτώσεις καὶ τὰς εἰρηνικὰς, καὶ εὐλόγησε τὸν λαὸν ἐν ὀνόματι Κυρίου τῶν δυνάμεων.
\VS{19}Καὶ διεμέρισε παντὶ τῷ λαῷ εἰς πᾶσαν τὴν δύναμιν τοῦ Ἰσραὴλ ἀπὸ Δὰν ἕως Βηρσαβεὲ, καὶ ἀπὸ ἀνδρὸς ἕως γυναικὸς, ἑκάστῳ κολλυρίδα ἄρτου, καὶ ἐσχαρίτην, καὶ λάγανον ἀπὸ τηγάνου· καὶ ἀπῆλθε πᾶς ὁ λαὸς ἕκαστος εἰς τὸν οἶκον αὐτοῦ.
\par }{\PP \VS{20}Καὶ ἐπέστρεψε Δαυὶδ εὐλογῆσαι τὸν οἶκον αὐτοῦ· καὶ ἐξῆλθε Μελχὸλ ἡ θυγάτηρ Σαοὺλ εἰς ἀπάντησιν Δαυὶδ, καὶ εὐλόγησεν αὐτὸν, καὶ εἶπε, τί δεδόξασται σήμερον ὁ βασιλεὺς Ἰσραὴλ, ὃς ἀπεκαλύφθη σήμερον ἐν ὀφθαλμοῖς παιδισκῶν τῶν δούλων ἑαυτοῦ, καθὼς ἀποκαλύπτεται ἀποκαλυφθεὶς εἷς τῶν ὀρχουμένων;
\VS{21}Καὶ εἶπε Δαυὶδ πρὸς Μελχόλ, ἐνώπιον Κυρίου ὀρχήσομαι· εὐλογητὸς Κύριος ὃς ἐξελέξατό με ὑπὲρ τὸν πατέρα σου καὶ ὑπὲρ πάντα τὸν οἶκον αὐτοῦ, του καταστῆσαί με εἰς ἡγούμενον ἐπὶ τὸν λαὸν αὐτοῦ ἐπὶ τὸν Ἰσραὴλ·
\VS{22}καὶ παίξομαι καὶ ὀρχήσομαι ἐνώπιον Κυρίου, καὶ ἀποκαλυφθήσομαι ἔτι οὕτως, καὶ ἔσομαι ἀχρεῖος ἐν ὀφθαλμοῖς σου, καὶ μετὰ τῶν παιδισκῶν, ὧν εἶπάς με μὴ δοξασθῆναι.
\VS{23}Καὶ τῇ Μελχὸλ θυγατρὶ Σαοὺλ οὐκ ἐγένετο παιδίον ἕως τῆς ἡμέρας τοῦ ἀποθανεῖν αὐτήν.

\par }\Chap{7}{\PP \VerseOne{1}Καὶ ἐγένετο ὅτε ἐκάθισεν ὁ βασιλεὺς ἐν τῷ οἴκῳ αὐτοῦ, καὶ Κύριος κατεκληρονόμησεν αὐτὸν κύκλῳ ἀπὸ πάντων τῶν ἐχθρῶν αὐτοῦ τῶν κύκλῳ,
\VS{2}καὶ εἶπεν ὁ βασιλεὺς πρὸς Νάθαν τὸν προφήτην, ἰδοὺ δὴ ἐγὼ κατοικῶ ἐν οἴκῳ κεδρίνῳ, καὶ ἡ κιβωτὸς τοῦ Θεοῦ κάθηται ἐν μέσῳ τῆς σκηνῆς.
\VS{3}Καὶ εἶπε Νάθαν πρὸς τὸν βασιλέα, πάντα ὅσα ἂν ἐν τῇ καρδίᾳ σου, βάδιζε καὶ ποίει, ὅτι Κύριος μετὰ σοῦ.
\par }{\PP \VS{4}Καὶ ἐγένετο τῇ νυκτὶ ἐκείνῃ, καὶ ἐγένετο ῥῆμα Κυρίου πρὸς Νάθαν, λέγων, πορεύου,
\VS{5}καὶ εἶπον πρὸς τὸν δοῦλόν μου Δαυὶδ, τάδε λέγει Κύριος, οὐ σὺ οἰκοδομήσεις μοι οἶκον τοῦ κατοικῆσαί με.
\VS{6}Ὅτι οὐ κατῴκηκα ἐν οἴκῳ ἀφʼ ἧς ἡμέρας ἀνήγαγον τοὺς υἱοὺς Ἰσραὴλ ἐξ Αἰγύπτου ἕως τῆς ἡμέρας ταύτης, καὶ ἤμην ἐυπεριπατῶν ἐν καταλύματι καὶ ἐν σκηνῇ,
\VS{7}ἐν πᾶσιν οἷς διῆλθον ἐν παντὶ Ἰσραήλ· εἰ λαλῶν ἐλάλησα πρὸς μίαν φυλὴν τοῦ Ἰσραὴλ, ᾧ ἐνετειλάμην ποιμαίνειν τὸν λαόν μου Ἰσραὴλ, λέγων, ἱνατί οὐκ ᾠκοδομήκατέ μοι οἶκον κέδρινον;
\par }{\PP \VS{8}Καὶ νῦν τάδε ἐρεῖς τῷ δούλῳ μου Δαυὶδ, τάδε λέγει Κύριος παντοκράτωρ, ἔλαβόν σε ἐκ τῆς μάνδρας τῶν προβάτων, τοῦ εἶναί σε εἰς ἡγούμενον ἐπὶ τὸν λαόν μου ἐπὶ τὸν Ἰσραὴλ,
\VS{9}καὶ ἤμην μετὰ σοῦ ἐν πᾶσιν οἷς ἐπορεύου, καὶ ἐξωλόθρευσα πάντας τοὺς ἐχθρούς σου ἀπὸ προσώπου σου, καὶ ἐποίησά σε ὀνομαστὸν κατὰ τὸ ὄνομα τῶν μεγάλων τῶν ἐπὶ τῆς γῆς.
\VS{10}Καὶ θήσομαι τόπον τῷ λαῷ μου τῷ Ἰσραὴλ, καὶ καταφυτεύσω αὐτὸν, καὶ κατασκηνώσει καθʼ ἑαυτὸν, καὶ οὐ μεριμνήσει οὐκέτι· καὶ οὐ προσθήσει υἱὸς ἀδικίας τοῦ ταπεινῶσαι αὐτὸν, καθὼς ἀπʼ ἀρχῆς,
\VS{11}ἀπὸ τῶν ἡμερῶν ὧν ἔταξα κριτὰς ἐπὶ τὸν λαόν μου Ἰσραήλ· καὶ ἀναπαύσω σε ἀπὸ πάντων τῶν ἐχθρῶν σου· καὶ ἀπαγγελεῖ σοι Κύριος, ὅτι οἶκον οἰκοδομήσεις αὐτῷ.
\VS{12}Καὶ ἔσται ἐὰν πληρωθῶσιν αἱ ἡμέραι σου, καὶ κοιμηθήσῃ μετὰ τῶν πατέρων σου, καὶ ἀναστήσω τὸ σπέρμα σου μετὰ σὲ, ὃς ἔσται ἐκ τῆς κοιλίας σου, καὶ ἑτοιμάσω τὴν βασιλείαν αὐτοῦ.
\VS{13}Αὐτὸς οἰκοδομήσει μοι οἶκον τῷ ὀνόματί μου, καὶ ἀνορθώσω τὸν θρόνον αὐτοῦ ἕως εἰς τὸν αἰῶνα.
\VS{14}Ἐγὼ ἔσομαι αὐτῷ εἰς πατέρα, καὶ αὐτὸς ἔσται μοι εἰς υἱόν· καὶ ἐὰν ἔλθῃ ἡ ἀδικία αὐτοῦ, καὶ ἐλέγξω αὐτὸν ἐν ῥάβδῳ ἀνδρῶν, καὶ ἐν ἁφαῖς υἱῶν ἀνθρώπων·
\VS{15}Τὸ δὲ ἔλεός μου οὐκ ἀποστήσω ἀπʼ αὐτοῦ, καθὼς ἀπέστησα ἀφʼ ὧν ἀπέστησα ἐκ προσώπου μου.
\VS{16}Καὶ πιστωθήσεται ὁ οἶκος αὐτοῦ, καὶ ἡ βασιλεία αὐτοῦ ἕως αἰῶνος ἐνώπιόν μου· καὶ ὁ θρόνος αὐτοῦ ἔσται ἀνωρθωμένος εἰς τον αἰῶνα.
\par }{\PP \VS{17}Κατὰ πάντας τοὺς λόγους τούτους, καὶ κατὰ πᾶσαν τὴν ὅρασιν ταύτην, οὕτως ἐλάλησε Νάθαν πρὸς Δαυίδ.
\par }{\PP \VS{18}Καὶ εἰσῆλθεν ὁ βασιλεὺς Δαυὶδ, καὶ ἐνάθισεν ἐνώπιον Κυρίου, καὶ εἶπε, τίς εἰμι ἐγὼ, Κύριέ μου Κύριε, καὶ τίς ὁ οἶκός μου, ὅτι ἠγάπησάς με ἕως τούτων;
\VS{19}Καὶ κατεσμικρύνθην μικρὸν ἐνώπίον σου, Κύριέ μου Κύριε, καὶ ἐλάλησας ὑπὲρ τοῦ οἴκου τοῦ δούλου σου εἰς μακράν· οὗτος δὲ ὁ νόμος τοῦ ἀνθρώπου, Κύριέ μου Κύριε;
\VS{20}Καὶ τί προσθήσει Δαυὶδ ἔτι τοῦ λαλῆσαι πρὸς σέ; καὶ νῦν σὺ οἶδας τὸν δοῦλόν σου, Κύριέ μου Κύριε,
\VS{21}καὶ διὰ τὸν δοῦλόν σου πεποίηκας, καὶ κατὰ τὴν καρδίαν σου ἐποίησας πᾶσαν τὴν μεγαλωσύνην ταύτην, γνωρίσαι τῷ δούλῳ σου, ἕνεκεν τοῦ μεγαλύναι σε
\VS{22}Κύριέ μου· ὅτι οὐκ ἔστιν ὡς σὺ, καὶ οὐκ ἔστι Θεὸς πλὴν σοῦ ἐν πᾶσιν οἷς ἠκούσαμεν ἐν τοῖς ὠσὶν ἡμῶν.
\VS{23}Καὶ τίς ὡς ὁ λαός σου Ἰσραὴλ ἔθνος ἄλλο ἐν τῇ γῇ; ὡς ὡδήγησεν αὐτὸν ὁ Θεὸς τοῦ λυτρώσασθαι αὐτῷ λαὸν, τοῦ θέσθαι σε ὄνομα, τοῦ ποιῆσαι μεγαλωσύνην καὶ ἐπιφάνειαν, τοῦ ἐκβαλεῖν σε ἐκ προσώπου τοῦ λαοῦ σου, οὓς ἐλυτρώσω σεαυτῷ ἐξ Αἰγύπτου, ἔθνη καὶ σκηνώματα;
\VS{24}Καὶ ἡτοίμασας σεαυτῷ τὸν λαόν σου Ἰσραὴλ εἰς λαὸν ἕως αἰῶνος, καὶ σὺ Κύριε ἐγένου αὐτοῖς εἰς Θεόν.
\VS{25}Καὶ νῦν, κύριέ μου, ῥῆμα ὃ ἐλάλησας περὶ τοῦ δούλου σου καὶ τοῦ οἴκου αὐτοῦ, πίστωσον ἕως τοῦ αἰῶνος, Κύριε παντοκράτωρ θεὲ τοῦ Ἰσραήλ· καὶ νῦν καθὼς ἐλάλησας,
\VS{26}Μεγαλυνθείη τὸ ὄνομά σου ἕως αἰῶνος.
\VS{27}Κύριε παντοκράτωρ Θεὸς Ἰσραὴλ, ἀπεκάλυψας τὸ ὠτίον τοῦ δούλου σου, λέγων, οἶκον οἰκοδομήσω σοι· διὰ τοῦτο εὗρεν ὁ δοῦλός σου τὴν καρδίαν ἑαυτοῦ τοῦ προσεύξασθαι πρὸς σὲ τὴν προσευχὴν ταύτην.
\VS{28}Καὶ νῦν, Κύριέ μου Κύριε, σὺ εἶ Θεὸς, καὶ οἱ λόγοι σου ἔσονται ἀληθιναὶ, καὶ ἐλάλησας ὑπὲρ τοῦ δούλου σου τὰ ἀγαθὰ ταῦτα.
\VS{29}Καὶ νῦν ἄρξαι καὶ εὐλόγησον τὸν οἶκον τοῦ δούλου σου, τοῦ εἶναι εἰς τὸν αἰῶνα ἐνώπιόν σου· ὅτι σὺ Κύριέ μου Κύριε ἐλάλησας, καὶ ἀπὸ τῆς εὐλογίας σου εὐλογηθήσεται ὁ οἶκος τοῦ δούλου σου τοῦ εἶναι εἰς τὸν αἰῶνα.

\par }\Chap{8}{\PP \VerseOne{1}Καὶ ἐγένετο μετὰ ταῦτα, καὶ ἐπάταξε Δαυὶδ τοὺς ἀλλοφύλους, καὶ ἐτροπώσατο αὐτούς· καὶ ἔλαβε Δαυὶδ τὴν ἀφωρισμένην ἐκ χειρὸς τῶν ἀλλοφύλων.
\par }{\PP \VS{2}Καὶ ἐπάταξε Δαυὶδ τὴν Μωὰβ, καὶ διεμέτρησεν αὐτοὺς ἐν σχοινίοις, κοιμίσας αὐτοὺς ἐπὶ τὴν γῆν· καὶ ἐγένετο τὰ δύο σχοινίσματα τοῦ θανατῶσαι, καὶ τὰ δύο σχοινίσματα ἐζώγρησε· καὶ ἐγένετο Μωὰβ τῷ Δαυὶδ εἰς δούλους φέροντας ξένια.
\par }{\PP \VS{3}Καὶ ἐπάταξε Δαυὶδ τὸν Ἁδρααζὰρ υἱὸν Ῥαὰβ, βασιλέα Σουβὰ, πορευομένου αὐτοῦ ἐπιστῆσαι τὴν χεῖρα αὐτοῦ ἐπὶ τὸν ποταμὸν Εὐφράτην.
\VS{4}Καὶ προκατελάβετο Δαυὶδ τῶν αὐτοῦ χίλια ἅρματα, καὶ ἑπτὰ χιλιάδας ἱππέων, καὶ εἴκοσι χιλιάδας ἀνδρῶν πεζῶν· καὶ παρέλυσε Δαυὶδ πάντα τὰ ἅρματα, καὶ ὑπελείπετο ἑαυτῷ ἑκατὸν ἅρματα.
\VS{5}Καὶ παραγίνεται Συρία Δαμασκοῦ βοηθῆσαι τῷ Ἀδρααζὰρ βασιλεῖ Σουβὰ, καὶ ἐπάταξε Δαυὶδ ἐν τῷ Σύρῳ εἴκοσι δύο χιλιάδας ἀνδρῶν.
\VS{6}Καὶ ἔθετο Δαυὶδ φρουρὰν ἐν Συρίᾳ τῇ κατὰ Δαμασκὸν, καὶ ἐγένετο ὁ Σύρος τῷ Δαυὶδ εἰς δούλους φέροντας ξένια· καὶ ἔσωσε Κύριος τὸν Δαυὶδ ἐν πᾶσιν οἷς ἐπορεύετο.
\VS{7}Καὶ ἔλαβε Δαυὶδ τοὺς χλιδῶνας τοὺς χρυσοῦς, οἳ ἦσαν ἐπὶ τῶν παίδων τῶν Ἀδρααζὰρ βασιλέως Σουβὰ, καὶ ἤνεγκεν αὐτὰ εἰς Ἱερουσαλήμ· καὶ ἔλαβεν αὐτὰ Σουσακὶμ βασιλεὺς Αἰγύπτου, ἐν τῷ ἀναβῆναι αὐτὸν εἰς Ἱερουσαλὴμ ἐν ἡμέραις Ῥοβοὰμ υἱοῦ Σαλωμῶντος.
\VS{8}Καὶ ἐκ τῆς Μετεβὰκ καὶ ἐκ τῶν ἐκλεκτῶν πόλεων τοῦ Ἀδρααζὰρ ἔλαβεν ὁ βασιλεὺς Δαυὶδ χαλκὸν πολὺν σφόδρα· ἐν αὐτῷ ἐποίησε Σαλωμὼν τὴν θάλασσαν τὴν χαλκῆν, καὶ τοὺς στύλους, καὶ τοὺς λουτῆρας, καὶ πάντα τὰ σκεύη.
\par }{\PP \VS{9}Καὶ ἤκουσε Θοοὺ ὁ βασιλεὺς Ἡμὰθ, ὅτι ἐπάταξε Δαυὶδ πᾶσαν τὴν δύναμιν Ἀδρααζὰρ,
\VS{10}καὶ ἀπέστειλε Θοοὺ Ἰεδδουρὰμ τὸν υἱὸν αὐτοῦ πρὸς βασιλέα Δαυὶδ ἐρωτῆσαι αὐτὸν τὰ εἰς εἰρήνην, καὶ εὐλογῆσαι αὐτὸν ὑπὲρ οὗ ἐπολέμησε τὸν Ἀδρααζὰρ, καὶ ἐπάταξεν αὐτὸν, ὅτι ἀντικείμενος ἦν τῷ Ἀδρααζάρ· καὶ ἐν ταῖς χερσὶν αὐτοῦ ἦσαν σκεύη ἀργυρᾶ, καὶ σκεύη χρυσᾶ, καὶ σκεύη χαλκᾶ.
\VS{11}Καὶ ταῦτα ἡγίασεν ὁ βασιλεὺς Δαυὶδ τῷ Κυρίῳ, μετὰ τοῦ ἀργυρίου καὶ μετὰ τοῦ χρυσίου οὗ ἡγίασεν ἐκ πασῶν τῶν πόλεων ὧν κατεδυνάστευσεν,
\VS{12}ἐκ τῆς Ἰδουμαίας, καὶ ἐκ τῆς Μωὰβ, καὶ ἐκ τῶν υἱῶν Ἀμμὼν, καὶ ἐκ τῶν ἀλλοφύλων, καὶ ἐξ Ἀμαλὴκ, καὶ ἐκ τῶν σκύλων Ἀδρααζὰρ υἱοῦ Ῥαὰβ βασιλέως Σουβά.
\par }{\PP \VS{13}Καὶ ἐποίησε Δαυὶδ ὄνομα· καὶ ἐν τῷ ἀνακάμπτειν αὐτὸν ἐπάταξε τὴν Ἰδουμαίαν ἐν Γεβελὲμ εἰς ὀκτωκαίδεκα χιλιάδας.
\VS{14}Καὶ ἔθετο ἐν τῇ Ἰδουμαίᾳ φρουρὰν, ἐν πάσῃ τῇ Ἰδουμαίᾳ· καὶ ἐγένοντο πάντες οἱ Ἰδουμαῖοι δοῦλοι τῷ βασιλεῖ· καὶ ἔσωσε Κύριος τὸν Δαυὶδ ἐν πᾶσιν οἷς ἐπορεύετο.
\par }{\PP \VS{15}Καὶ ἐβασίλευσε Δαυὶδ ἐπὶ πάντα Ἰσραήλ· καὶ ἦν Δαυὶδ ποιῶν κρίμα καὶ δικαιοσύνην ἐπὶ πάντα τὸν λαὸν αὐτοῦ.
\VS{16}Καὶ Ἰωὰβ υἱὸς Σαρουίας ἐπὶ τῆς στρατιᾶς· καὶ Ἰωσαφὰτ υἱὸς Ἀχιλοὺδ ἐπὶ τῶν ὑπομνημάτων·
\VS{17}καὶ Σαδὼκ υἱὸς Ἀχιτὼβ καὶ Ἀχιμέλεχ υἱὸς Ἀβιάθαρ ἱερεῖς· καὶ Σασὰ ὁ γραμματεύς·
\VS{18}καὶ Βαναίας υἱὸς Ἰωδαὲ σύμβουλος· καὶ ὁ Χελεθὶ, καὶ ὁ Φελετὶ, καὶ οἱ υἱοὶ Δαυὶδ αὐλάρχαι ἦσαν.

\par }\Chap{9}{\PP \VerseOne{1}Καὶ εἶπε Δαυὶδ, εἰ ἔστιν ἔτι ὑπολελειμμένος ἐν τῷ οἴκῳ Σαοὺλ, καὶ ποιήσω μετʼ αὐτοῦ ἔλεος ἕνεκεν Ἰωνάθαν;
\VS{2}Καὶ ἐκ τοῦ οἴκου Σαοὺλ ἦν παῖς, καὶ ὄνομα αὐτῷ Σιβά· καὶ καλοῦσιν αὐτὸν πρὸς Δαυίδ· καὶ εἶπε πρὸς αὐτὸν ὁ βασιλεὺς, σὺ εἶ Σιβά; καὶ εἶπεν, ἐγὼ δοῦλος σός.
\VS{3}Καὶ εἶπεν ὁ βασιλεὺς, εἰ ὑπολέλειπται ἐκ τοῦ οἴκου Σαοὺλ ἔτι ἀνὴρ, καὶ ποιήσω μετʼ αὐτοῦ ἔλεος Θεοῦ; καὶ εἶπε Σιβὰ πρὸς τὸν βασιλέα, ἔτι ἐστὶν υἱὸς τῷ Ἰωνάθαν πεπληγὼς τοὺς πόδας.
\VS{4}Καὶ εἶπεν ὁ βασιλεὺς, ποῦ οὗτος; καὶ εἶπε Σιβὰ πρὸς τὸν βασιλέα, ἰδοὺ ἐν οἴκῳ Μαχεὶρ υἱοῦ Ἀμιὴλ ἐκ τῆς Λοδάβαρ.
\VS{5}Καὶ ἀπέστειλεν ὁ βασιλεὺς Δαυὶδ, καὶ ἔλαβεν αὐτὸν ἐκ τοῦ οἴκου Μαχὶρ υἱοῦ Ἀμιὴλ ἐκ τῆς Λαδάβαρ.
\par }{\PP \VS{6}Καὶ παραγίνεται Μεμφιβοσθὲ υἱὸς Ἰωνάθαν υἱοῦ Σαοὺλ πρὸς τὸν βασιλέα Δαυίδ, καὶ ἔπεσεν ἐπὶ πρόσωπον αὐτοῦ, καὶ προσεκύνησεν αὐτῷ· καὶ εἶπεν αὐτῷ Δαυίδ, Μεμφιβοσθέ; καὶ εἶπεν, ἰδοὺ ὁ δοῦλός σου.
\VS{7}Καὶ εἶπεν αὐτῷ Δαυίδ, μὴ φοβοῦ, ὅτι ποιῶν ποιήσω μετὰ σοῦ ἔλεος διὰ Ἰωνάθαν τὸν πατέρα σου, καὶ ἀποκαταστήσω σοι πάντα ἀγρὸν Σαοὺλ πατρὸς τοῦ πατρός σου, καὶ σὺ φαγῃ ἄρτον ἐπὶ τῆς τραπέζης μου διαπαντός.
\VS{8}Καὶ προσεκύνησε Μεμφιβοσθὲ, καὶ εἶπε, τίς εἰμι ὁ δοῦλός σου, ὅτι ἐπέβλεψας ἐπὶ τὸν κύνα τὸν τεθνηκότα τὸν ὅμοιον ἐμοί;
\par }{\PP \VS{9}Καὶ ἐκάλεσεν ὁ βασιλεὺς Σιβὰ τὸ παιδάριον Σαοὺλ, καὶ εἶπε πρὸς αὐτὸν, πάντα ὅσα ἐστὶ τῷ Σαοὺλ καὶ ὅλῳ τῷ οἴκῳ αὐτοῦ δέδωκα τῷ υἱῷ τοῦ κυρίου σου.
\VS{10}Καὶ ἐργᾷ αὐτῷ τὴν γῆν σὺ, καὶ οἱ υἱοί σου, καὶ οἱ δοῦλοί σου, καὶ εἰσοίσεις τῷ υἱῷ τοῦ κυρίου σου ἄρτους, καὶ ἔδεται ἄρτους· καὶ Μεμφιβοσθὲ υἱὸς τοῦ κυρίου σου φάγεται διαπαντὸς ἄρτον ἐπὶ τῆς τραπέζης μου· καὶ τῷ Σιβᾷ ἦσαν πεντεκαίδεκα υἱοὶ, καὶ εἴκοσι δοῦλοι.
\VS{11}Καὶ εἶπε Σιβὰ πρὸς τὸν βασιλέα, κατὰ πάντα ὅσα ἐντέταλται ὁ κύριός μου ὁ βασιλεὺς τῷ δούλῳ αὐτοῦ, οὕτως ποιήσει ὁ δοῦλός σου· καὶ Μεμφιβοσθὲ ἤσθιεν ἐπὶ τῆς τραπέζης Δαυὶδ καθὼς εἷς τῶν υἱῶν αὐτοῦ τοῦ βασιλέως.
\VS{12}Καὶ τῷ Μεμφιβοσθὲ υἱὸς μικρὸς ἦν, καὶ ὄνομα αὐτῷ Μιχά· καὶ πᾶσα ἡ κατοίκησις τοῦ οἴκου Σιβὰ δοῦλοι τοῦ Μεμφιβοσθέ.
\VS{13}Καὶ Μεμφιβοσθὲ κατῴκει ἐν Ἱερουσαλὴμ, ὅτι ἐπὶ τῆς τραπέζης τοῦ βασιλέως αὐτὸς διαπαντὸς ἤσθιε, καὶ αὐτὸς ἦν χωλὸς ἀμφοτέροις τοῖς ποσὶν αὐτοῦ.

\par }\Chap{10}{\PP \VerseOne{1}Καὶ ἐγένετο μετὰ ταῦτα καὶ ἀπέθανε βασιλεὺς υἱῶν Ἀμμὼν, καὶ ἐβασίλευσεν Ἀννὼν υἱὸς αὐτοῦ ἀντʼ αὐτοῦ.
\VS{2}Καὶ εἶπε Δαυὶδ, ποιήσω ἔλεος μετὰ Ἀννὼν υἱοῦ Ναὰς, ὃν τρόπον ἐποίησεν ὁ πατὴρ αὐτοῦ μετʼ ἐμοῦ ἔλεος. Καὶ ἀπέστειλεν Δαυὶδ παρακαλέσαι αὐτὸν ἐν χειρὶ τῶν δούλων αὐτοῦ περὶ τοῦ πατρὸς αὐτοῦ· καὶ παρεγένοντο οἱ παῖδες Δαυὶδ εἰς τὴν γῆν υἱῶν Ἀμμών.
\VS{3}Καὶ εἶπον οἱ ἄρχοντες υἱῶν Ἀμμὼν πρὸς Ἀννῶν τὸν κύριον αὐτῶν, μὴ παρὰ τὸ δοξάζειν Δαυὶδ τὸν πατέρα σου ἐνώπιόν σου, ὅτι ἀπέστειλέ σοι παρακαλοῦντας; ἀλλʼ ὅπως οὐχὶ ἐρευνήσωσι τὴν πόλιν καὶ κατασκοπήσωσιν αὐτὴν καὶ τοῦ κατασκέψασθαι αὐτὴν ἀπέστειλε Δαυὶδ τοὺς παῖδας αὐτοῦ πρὸς σέ;
\VS{4}Καὶ ἔλαβεν Ἀννὼν τοὺς παῖδας Δαυὶδ, καὶ ἐξύρησε τοὺς πώγωνας αὐτῶν, καὶ ἀπέκοψε τοὺς μανδύας αὐτῶν ἐν τῷ ἡμίσει ἕως τῶν ἰσχίων αὐτῶν, καὶ ἐξαπέστειλεν αὐτούς.
\par }{\PP \VS{5}Καὶ ἀπήγγειλαν τῷ Δαυὶδ ὑπὲρ τῶν ἀνδρῶν, καὶ ἀπέστειλεν εἰς ἀπαντὴν αὐτῶν, ὅτι ἦσαν οἱ ἄνδρες ἠτιμασμένοι σφόδρα· καὶ εἶπεν ὁ βασιλεὺς, καθίσατε ἐν Ἱεριχὼ ἕως τοῦ ἀνατεῖλαι τοὺς πώγωνας ὑμῶν, καὶ ἐπιστραφήσεσθε.
\par }{\PP \VS{6}Καὶ εἶδον οἱ υἱοὶ Ἀμμὼν ὅτι κατῃσχύνθησαν ὁ λαὸς Δαυίδ· καὶ ἀπέστειλαν οἱ υἱοὶ Ἀμμὼν, καὶ ἐμισθώσαντο τὴν Συρίαν Βαιθραὰμ, καὶ τὴν Συρίαν Σουβὰ, καὶ Ῥοὼβ, εἴκοσι χιλιάδας πεζῶν, καὶ τὸν βασιλέα Ἀμαλὴκ χιλίους ἄνδρας, καὶ Ἰστὼβ δώδεκα χιλιάδας ἀνδρῶν.
\par }{\PP \VS{7}Καὶ ἤκουσε Δαυὶδ, καὶ ἀπέστειλε τὸν Ἰωὰβ καὶ πᾶσαν τὴν δύναμιν τοὺς δυνατούς.
\VS{8}Καὶ ἐξῆλθον οἱ υἱοὶ Ἀμμὼν καὶ παρετάξαντο πόλεμον παρὰ τῇ θύρᾳ τῇς πύλης, Συρίας Σουβὰ καὶ Ῥοὼβ καὶ Ἰστὼβ καὶ Ἀμαλὴκ μόνοι ἐν ἀγρῷ.
\VS{9}Καὶ εἶδεν Ἰωὰβ ὅτι ἐγενήθη πρὸς αὐτὸν ἀντιπρόσωπον τοῦ πολέμου ἐκ τοῦ κατὰ πρόσωπον ἐξεναντίας καὶ ἐκ τοῦ ὄπισθεν, καὶ ἐπελέξατο ἐκ πάντων τῶν νεανιῶν Ἰσραὴλ, καὶ παρετάξαντο ἐξ ἐναντίας Συρίας.
\VS{10}Καὶ τὸ κατάλοιπον τοῦ λαοῦ ἔδωκεν ἐν χειρὶ Ἀβεσσὰ τοῦ ἀδελφοῦ αὐτοῦ, καὶ παρετάξαντο ἐξεναντίας υἱῶν Ἀμμών.
\VS{11}Καὶ εἶπεν, εὰν κραταιωθῇ Συρία ὑπὲρ ἐμὲ, καὶ ἔσεσθέ μοι εἰς σωτηρίαν· καὶ ἐὰν κραταιωθῶσιν υἱοὶ Ἀμμὼν ὑπὲρ σὲ, καὶ ἐσόμεθα τοῦ σῶσαί σε.
\VS{12}Ἀνδρίζου καὶ κραταιωθῶμεν ὑπὲρ τοῦ λαοῦ ἡμῶν καὶ περὶ τῶν πόλεων τοῦ Θεοῦ ἡμῶν, καὶ Κύριος ποιήσει τὸ ἀγαθὸν ἐν ὀφθαλμοῖς αὐτοῦ.
\par }{\PP \VS{13}Καὶ προσῆλθεν Ἰωὰβ καὶ ὁ λαὸς αὐτοῦ μετʼ αὐτοῦ εἰς πόλεμον πρὸς Συρίαν, καὶ ἔφυγαν ἀπὸ προσώπου αὐτοῦ.
\VS{14}Καὶ οἱ υἱοὶ Ἀμμὼν εἶδαν ὅτι ἔφυγε Συρία, καὶ ἔφυγαν ἀπὸ προσώπου Ἀβεσσὰ, καὶ εἰσῆλθον εἰς τὴν πόλιν· καὶ ἀνέστρεψεν Ἰωὰβ ἀπὸ τῶν υἱῶν Ἀμμὼν, καὶ παρεγένετο εἰς Ἱερουσαλήμ.
\par }{\PP \VS{15}Καὶ εἶδε Συρία ὅτι ἔπτάισεν ἔμπροσθεν Ἰσραὴλ, καὶ συνήχθησαν ἐπὶ τὸ αὐτό.
\VS{16}Καὶ ἀπέστειλεν Ἀδρααζὰρ, καὶ συνήγαγε τὴν Συρίαν τὴν ἐκ τοῦ πέραν τοῦ ποταμοῦ Χαλαμὰκ, καὶ παρεγένοντο Αἱλάμ· καὶ Σωβὰκ ἄρχων τῆς δυνάμεως Ἀδρααζὰρ ἔμπροσθεν αὐτῶν.
\par }{\PP \VS{17}Καὶ ἀπηγγέλη τῷ Δαυὶδ, καὶ συνήγαγε τὸν πάντα Ἰσραὴλ, καὶ διέβη τὸν Ἰορδάνην, καὶ παρεγένετο εἰς Αἰλάμ· καὶ παρετάξατο Συρία ἀπέναντι Δαυὶδ, καὶ ἐπολέμησαν μετʼ αὐτοῦ.
\VS{18}Καὶ ἔφυγε Συρία ἀπὸ πρόσωπου Ἰσραήλ· καὶ ἀνεῖλε Δαυὶδ ἐκ τῆς Συρίας ἑπτακόσια ἅρματα, καὶ τεσσαράκοντα χιλιάδας ἱππέων, καὶ τὸν Σωβὰκ τὸν ἄρχοντα τῆς δυνάμεως αὐτοῦ ἐπάταξε, καὶ ἀπέθανεν ἐκεῖ.
\VS{19}Καὶ εἶδαν πάντες οἱ βασιλεῖς οἱ δοῦλοι Ἀδρααζὰρ ὅτι ἔπταισαν ἔμπροσθεν Ἰσραὴλ, καὶ ηὐτομόλησαν μετὰ Ἰσραὴλ, καὶ ἐδούλευσαν αὐτοῖς· καὶ ἐφοβήθη Συρία τοῦ σῶσαι ἔτι τοὺς υἱοὺς Ἀμμών.

\par }\Chap{11}{\PP \VerseOne{1}Καὶ ἐγένετο, ἐπιστρέψαντος τοῦ ἐνιαυτοῦ εἰς τὸν καιρὸν τῆς ἐξοδίας τῶν βασιλέων, καὶ ἀπέστειλε Δαυὶδ τὸν Ἰωὰβ, καὶ τοὺς παῖδας αὐτοῦ μετʼ αὐτοῦ, καὶ τὸν πάντα Ἰσραὴλ, καὶ διέφθειραν τοὺς υἱοὺς Ἀμμών· καὶ διεκάθισαν ἐπὶ Ῥαββάθ· καὶ Δαυὶδ ἐκάθισεν ἐν Ἱερουσαλήμ.
\par }{\PP \VS{2}Καὶ ἐγένετο πρὸς ἑσπέραν, καὶ ἀνέστη Δαυὶδ ἀπὸ τῆς κοίτης αὐτοῦ, καὶ περιεπάτει ἐπὶ τοῦ δώματος τοῦ οἴκου τοῦ βασιλέως, καὶ εἶδε γυναῖκα λουομένην ἀπὸ τοῦ δώματος, καὶ ἡ γυνὴ καλὴ τῷ εἴδει σφόδρα.
\VS{3}Καὶ ἀπέστειλε Δαυὶδ, καὶ ἐζήτησε τὴν γυναῖκα, καὶ εἶπεν, οὐχὶ αὕτη Βηρσαβεὲ θυγάτηρ Ἐλιὰβ γυνὴ Οὐρίου τοῦ Χετταίου;
\par }{\PP \VS{4}Καὶ ἀπέστειλεν Δαυὶδ ἀγγέλους, καὶ ἔλαβεν αὐτὴν, καὶ εἰσῆλθε πρὸς αὐτὴν, καὶ ἐκοιμήθη μετʼ αὐτῆς· καὶ αὕτη ἁγιαζομένη ἀπὸ ἀκαθαρσίας αὐτῆς, καὶ ἀπέστρεψεν εἰς τὸν οἶκον αὐτῆς.
\VS{5}Καὶ ἐν γαστρὶ ἔλαβεν ἡ γυνή· καὶ ἀποστείλασα ἀπήγγειλε τῷ Δαυὶδ, καὶ εἶπεν, ἐγώ εἰμι ἐν γαστρὶ ἔχω.
\VS{6}Καὶ ἀπέστειλε Δαυὶδ πρὸς Ἰωὰβ, λέγων, ἀπόστειλον πρὸς μὲ τὸν Οὐρίαν τὸν Χετταῖον· καὶ ἀπέστειλεν Ἰωὰβ τὸν Οὐρίαν πρὸς Δαυίδ.
\par }{\PP \VS{7}Καὶ παραγίνεται Οὐρίας καὶ εἰσῆλθε πρὸς αὐτὸν, καὶ ἐπηρώτησε Δαυὶδ εἰς εἰρήνην Ἰωὰβ, καὶ εἰς εἰρήνην τοῦ λαοῦ, καὶ εἰς εἰρήνην τοῦ πολέμου.
\VS{8}Καὶ εἶπε Δαυὶδ τῷ Οὐρίᾳ, Κατάβηθι εἰς τὸν οἶκόν σου, καὶ νίψαι τοὺς πόδας σου· καὶ ἐξῆλθεν Οὐρίας ἐξ οἴκου τοῦ βασιλέως, καὶ ἐξῆλθεν ὀπίσω αὐτοῦ ἄρσις τοῦ βασιλέως.
\VS{9}Καὶ ἐκοιμήθη Οὐρίας παρὰ τῇ θύρᾳ τοῦ βασιλέως μετὰ τῶν δούλων τοῦ κυρίου αὐτοῦ, καὶ οὐ κατέβη εἰς τὸν οἶκον αὐτοῦ.
\VS{10}Καὶ ἀνήγγειλαν τῷ Δαυὶδ, λέγοντες, ὅτι Οὐ κατέβη Οὐρίας εἰς τὸν οἶκον αὐτοῦ· καὶ εἶπε Δαυὶδ πρὸς Οὐρίαν, οὐχὶ ἐξ ὁδοῦ σὺ ἔρχῃ; τί ὅτι οὐ κατέβης εἰς τὸν οἶκόν σου;
\VS{11}Καὶ εἶπεν Οὐρίας πρὸς Δαυὶδ, ἡ κιβωτὸς, καὶ Ἰσραὴλ, καὶ Ἰούδας κατοικοῦσιν ἐν σκηναῖς, καὶ ὁ κύριός μου Ἰωὰβ, καὶ οἱ δοῦλοι τοῦ κυρίου μου ἐπὶ πρόσωπον τοῦ ἀγροῦ παρεμβάλλουσι, καὶ ἐγὼ εἰσελεύσομαι εἰς τὸν οἶκόν μου τοῦ φαγεῖν, καὶ πιεῖν, καὶ κοιμηθῆναι μετὰ τῆς γυναικός μου; πῶς; ζῇ ἡ ψυχή σου, εἰ ποιήσω τὸ ῥῆμα τοῦτο.
\VS{12}Καὶ εἶπε Δαυὶδ πρὸς Οὐρίαν, κάθισον ἐνταῦθα καί γε σήμερον, καὶ αὔριον ἐξαποστελῶ σε· καὶ ἐκάθισεν Οὐρίας ἐν Ἱερουσαλὴμ ἐν τῇ ἡμέρᾳ ἐκείνῃ καὶ τῇ ἐπαύριον.
\par }{\PP \VS{13}Καὶ ἐκάλεσεν αὐτὸν Δαυὶδ, καὶ ἔφαγεν ἐνώπιον αὐτοῦ, καὶ ἔπιε, καὶ ἐμέθυσεν αὐτὸν, καὶ ἐξῆλθεν ἑσπέρας τοῦ κοιμηθῆναι ἐπὶ τῆς κοίτης αὐτοῦ μετὰ τῶν δούλων τοῦ κυρίου αὐτοῦ, καὶ εἰς τὸν οἶκον αὐτοῦ οὐ κατέβη.
\par }{\PP \VS{14}Καὶ ἐγένετο πρωῒ, καὶ ἔγραψε Δαυὶδ βιβλίον πρὸς Ἰωὰβ, καὶ ἀπέστειλεν ἐν χειρὶ Οὐρίου.
\VS{15}Καὶ ἔγραψεν ἐν βιβλίῳ, λέγων, εἰσάγαγε τὸν Οὐρίαν ἐξεναντίας τοῦ πολέμου τοῦ κραταιοῦ, καὶ ἀποστραφήσεσθε ἀπὸ ὄπισθεν αὐτοῦ, καὶ πληγήσεται καὶ ἀποθανεῖται.
\par }{\PP \VS{16}Καὶ ἐγενήθη ἐν τῷ φυλάσσειν Ἰωὰβ ἐπὶ τὴν πόλιν, καὶ ἔθηκε τὸν Οὐρίαν εἰς τὸν τόπον οὗ ᾔδει ὅτι ἄνδρες δυνάμεως ἐκεῖ.
\VS{17}Καὶ ἐξῆλθον οἱ ἄνδρες τῆς πόλεως, καὶ ἐπολέμουν μετὰ Ἰωάβ· καὶ ἔπεσαν ἐκ τοῦ λαοῦ ἐκ τῶν δούλων Δαυὶδ, καὶ ἀπέθανε καί γε Οὐρίας ὁ Χετταῖος.
\par }{\PP \VS{18}Καὶ ἀπέστειλεν Ἰωὰβ, καὶ ἀπήγγειλε τῷ Δαυὶδ πάντας τοὺς λόγους τοῦ πολέμου λαλῆσαι πρὸς τὸν βασιλέα·
\VS{19}καὶ ἐνετείλατο τῷ ἀγγέλῳ, λέγων, ἐν τῷ συντελέσαι πάντας τοὺς λόγους τοῦ πολέμου λαλῆσαι πρὸς τὸν βασιλέα,
\VS{20}καὶ ἔσται ἐὰν ἀναβῇ ὁ θυμὸς τοῦ βασιλέως, καὶ εἴπῃ σοι, τί ὅτι ἠγγίσατε πρὸς τὴν πόλιν πολεμῆσαι; οὐκ ᾔδειτε ὅτι τοξεύσουσιν ἀπάνωθεν τοῦ τείχους;
\VS{21}Τίς ἐπάταξε τὸν Ἀβιμέλεχ υἱὸν Ἱεροβάαλ υἱοῦ Νήρ; οὐχὶ γυνὴ ἔῤῥιψε κλάσμα μύλου ἐπʼ αὐτὸν ἀπὸ ἄνωθεν τοῦ τείχους, καὶ ἀπέθανεν ἐν Θαμασί; ἱνατί προσηγάγετε πρὸς τὸ τεῖχος; καὶ ἐρεῖς, καί γε ὁ δοῦλός σου Οὐρίας ὁ Χετταῖος ἀπέθανε.
\par }{\PP \VS{22}Καὶ ἐπορεύθη ὁ ἄγγελος Ἰωὰβ πρὸς τὸν βασιλέα εἰς Ἱερουσαλὴμ, καὶ παρεγένετο καὶ ἀπήγγειλε τῷ Δαυὶδ πάντα ὅσα ἀπήγγειλεν αὐτῷ Ἰωὰβ, πάντα τὰ ῥήματα τοῦ πολέμου· καὶ ἐθυμώθη Δαυὶδ πρὸς Ἰωὰβ, καὶ εἶπε πρὸς τὸν ἄγγελον, ἱνατί προσηγάγετε πρὸς τὴν πόλιν τοῦ πολεμῆσαι; οὐκ ᾔδειτε ὅτι πληγήσεσθε ἀπὸ τοῦ τείχους; τίς ἐπάταξε τὸν Ἀβιμέλεχ υἱὸν Ἱεροβάαλ; οὐχὶ γυνὴ ἔῤῥιψεν ἐπʼ αὐτὸν κλάσμα μύλου ἀπὸ τοῦ τείχους, καὶ ἀπέθανεν ἐν Θαμασί; ἱνατί προσηγάγετε πρὸς τὸ τεῖχος;
\VS{23}Καὶ εἶπεν ὁ ἄγγελος πρὸς Δαυὶδ, ὅτι ἐκραταίωσαν ἐφʼ ἡμᾶς οἱ ἄνδρες, καὶ ἐξῆλθον ἐφʼ ἡμᾶς εἰς τὸν ἀγρὸν, καὶ ἐγενήθημεν ἐπʼ αὐτοὺς ἕως τῆς θύρας τῆς πύλης.
\VS{24}Καὶ ἐτόξευσαν οἱ τοξεύοντες πρὸς τοὺς παῖδάς σου ἀπάνωθεν τοῦ τείχους, καὶ ἀπέθανον τῶν παίδων τοῦ βασιλέως, καί γε ὁ δοῦλός σου Οὐρίας ὁ Χετταῖος ἀπέθανε.
\VS{25}Καὶ εἶπε Δαυὶδ πρὸς τὸν ἄγγελον, τάδε ἐρεῖς πρὸς Ἰωάβ, μὴ πονηρὸν ἔστω ἐν ὀφθαλμοῖς σου τὸ ῥῆμα τοῦτο, ὅτι ποτὲ μὲν οὕτως καὶ ποτὲ οὕτως φάγεται ἡ μάχαιρα· κραταίωσον τὸν πόλεμόν σου εἰν τὴν πόλιν, καὶ κατάσπασον αὐτὴν, καὶ κραταίωσον αὐτήν.
\par }{\PP \VS{26}Καὶ ἤκουσεν ἡ γυνὴ Οὐρίου ὅτι ἀπέθανεν Οὐρίας ὁ ἀνὴρ αὐτῆς, καὶ ἐκόψατο τὸν ἄνδρα αὐτῆς.
\VS{27}Καὶ διῆλθε τὸ πένθος, καὶ ἀπέστειλεν Δαυὶδ, καὶ συνήγαγεν αὐτὴν εἰς τὸν οἶκον αὐτοῦ, καὶ ἐγενήθη αὐτῷ εἰς γυναῖκα, καὶ ἔτεκεν αὐτῷ υἱόν· καὶ πονηρὸν ἐφάνη τὸ ῥῆμα ὃ ἐποίησε Δαυὶδ ἐν ὀφθαλμοῖς Κυρίου.

\par }\Chap{12}{\PP \VerseOne{1}Καὶ ἀπέστειλε Κύριος τὸν Νάθαν τὸν προφήτην πρὸς Δαυίδ· καὶ εἰσῆλθε πρὸς αὐτὸν, καὶ εἶπεν αὐτῷ, δύο ἦσαν ἄνδρες ἐν πόλει μιᾷ, εἷς πλούσιος, καὶ εἷς πένης.
\VS{2}Καὶ τῷ πλουσίῳ ἦν ποίμνια καὶ βουκόλια πολλὰ σφόδρα.
\VS{3}Καὶ τῷ πένητι οὐδὲν ἀλλʼ ἢ ἀμνὰς μία μικρὰ, ἣν ἐκτήσατο καὶ περιεποιήσατο, καὶ ἐξέθρεψεν αὐτὴν, καὶ ἡδρύνθη μετʼ αὐτοῦ καὶ μετὰ τῶν υἱῶν αὐτοῦ ἐπὶ τὸ αὐτὸ, ἐκ τοῦ ἄρτου αὐτοῦ ἤσθιε, καὶ ἐκ τοῦ ποτηρίου αὐτοῦ ἔπινε, καὶ ἐν τῷ κόλπῷ αὐτοῦ ἐκάθευδε, καὶ ἦν αὐτῷ ὡς θυγάτηρ.
\VS{4}Καὶ ἦλθε πάροδος τῷ ἀνδρὶ τῷ πλουσίῳ, καὶ ἐφείσατο λαβεῖν ἐκ τῶν ποιμνίων αὐτοῦ, καὶ ἐκ τῶν βουκολίων αὐτοῦ, τοῦ ποιῆσαι τῷ ξένῳ ὁδοιπόρῳ τῷ ἐλθόντι πρὸς αὐτὸν, καὶ ἔλαβε τὴν ἀμνάδα τοῦ πένητος. καὶ ἐποίησεν αὐτὴν τῷ ἀνδρὶ τῷ ἐλθόντι πρὸς αὐτόν.
\VS{5}Καὶ ἐθυμώθη ὀργῇ Δαυὶδ σφόδρα τῷ ἀνδρὶ, καὶ εἶπε Δαυὶδ πρὸς Νάθαν, ζῇ Κύριος, ὅτι υἱὸς θανάτου ὁ ἀνὴρ ὁ ποιήσας τοῦτο·
\VS{6}Καὶ τὴν ἀμνάδα ἀποτίσει ἑπταπλασίονα, ἀνθʼ ὧν ὅτι ἐποίησε τὸ ῥῆμα τοῦτο, καὶ περὶ οὗ οὐκ ἐφείσατο.
\par }{\PP \VS{7}Καὶ εἶπε Νάθαν πρὸς Δαυὶδ, σὺ εἶ ὁ ἀνὴρ ὁ ποιήσας τοῦτο· τάδε λέγει Κύριος ὁ Θεὸς Ἰσραὴλ, ἐγώ εἰμι ὁ χρίσας σε εἰς βασιλέα ἐπὶ Ἰσραὴλ, καὶ ἐγώ εἰμι ἐῤῥυσάμην σε ἐκ χειρὸς Σαοὺλ,
\VS{8}καὶ ἔδωκά σοι τὸν οἶκον τοῦ κυρίου σου, καὶ τὰς γυναῖκας τοῦ κυρίου σου ἐν τῷ κόλπῳ σου, καὶ ἔδωκά σοι τὸν οἶκον Ἰσραὴλ καὶ Ἰούδα· καὶ εἰ μικρόν ἐστι, προσθήσω σοι κατὰ ταῦτα.
\VS{9}Τί ὅτι ἐφαύλισας τὸν λόγον Κυρίου, τοῦ ποιῆσαι τὸ πονηρὸν ἐν ὀφθαλμοῖς αὐτοῦ; τὸν Οὐρίαν τὸν Χετταῖον ἐπάταξας ἐν ῥομφαίᾳ, καὶ τὴν γυναῖκα αὐτοῦ ἔλαβες σεαυτῷ εἰς γυναῖκα, καὶ αὐτὸν ἀπέκτεινας ἐν ῥομφαίᾳ υἱῶν Ἀμμών.
\VS{10}Καὶ νὺν οὐκ ἀποστήσεται ῥομφαία ἐκ τοῦ οἴκου σου ἕως αἰῶνος, ἀνθʼ ὧν ὅτι ἐξουδένωσάς με, καὶ ἔλαβες τὴν γυναῖκα τοῦ Οὐρίου τοῦ Χετταίου, τοῦ εἶναί σοι εἰς γυναῖκα.
\VS{11}Τάδε λέγει Κύριος, ἰδοὺ ἐγὼ ἐξεγείρω ἐπὶ σὲ κακὰ ἐκ τοῦ οἴκου σου, καὶ λήψομαι τὰς γυναῖκάς σου κατʼ ὀφθαλμούς σου, καὶ δώσω τῷ πλησίον σου, καὶ κοιμηθήσεται μετὰ τῶν γυναικῶν σου ἐναντίον τοῦ ἡλίου τούτου.
\VS{12}Ὅτι σὺ ἐποίησας κρυβῇ, κᾀγὼ ποιήσω τὸ ῥῆμα τοῦτο ἐναντίον παντὸς Ἰσραὴλ, καὶ ἀπέναντι τοῦ ἡλίου τούτου.
\par }{\PP \VS{13}Καὶ εἶπε Δανὶδ τῷ Νάθαν, ἡμάρτηκα τῷ Κυρίῳ· καὶ εἶπε Νάθαν πρὸς Δαυὶδ, καὶ Κύριος παρεβίβασε τὸ ἁμάρτημά σου· οὐ μὴ ἀποθάνῃς.
\VS{14}Πλὴν ὅτι παροργίζων παρώργισας τοὺς ἐχθροὺς Κυρίου ἐν τῷ ῥήματι τούτῳ, καί γε ὁ υἱός σου ὁ τεχθείς σοι θανάτῳ ἀποθανεῖται.
\par }{\PP \VS{15}Καὶ ἀπῆλθε Νάθαν εἰς τὸν οἶκον αὐτοῦ· καὶ ἔθραυσε Κύριος τὸ παιδίον ὃ ἔτεκεν ἡ γυνὴ Οὐρίου τοῦ Χετταίου τῷ Δαυὶδ, καὶ ἠῤῥώστησε.
\VS{16}Καὶ ἐζήτησε Δαυὶδ τὸν Θεὸν περὶ τοῦ παιδαρίου, καὶ ἐνήστευσε Δαυὶδ νηστείαν, καὶ εἰσῆλθε καὶ ηὐλίσθη ἐπὶ τῆς γῆς.
\VS{17}Καὶ ἀνέστησαν ἐπʼ αὐτὸν οἱ πρεσβύτεροι τοῦ οἴκου αὐτοῦ ἐγεῖραι αὐτὸν ἀπὸ τῆς γῆς, καὶ οὐκ ἠθέλησε, καὶ οὐ συνέφαγεν αὐτοῖς ἄρτον.
\par }{\PP \VS{18}Καὶ ἐγένετο ἐν τῇ ἡμέρᾳ τῇ ἑβδόμῃ, καὶ ἀπέθανε τὸ παιδάριον· καὶ ἐφοβήθησαν οἱ δοῦλοι Δαυὶδ ἀναγγεῖλαι αὐτῷ, ὅτι τέθνηκε τὸ παιδάριον, ὅτι εἶπον, ἰδοὺ ἐν τῷ τὸ παιδάριον ἔτι ζῇν ἐλαλήσαμεν πρὸς αὐτὸν, καὶ οὐκ εἰσήκουσε τῆς φωνῆς ἡμῶν· καὶ πῶς εἴπωμεν πρὸς αὐτὸν ὅτι τέθνηκε τὸ παιδάριον, καὶ ποιήσει κακά;
\VS{19}Καὶ συνῆκε Δαυὶδ, ὅτι οἱ παῖδες αὐτοῦ ψιθυρίζουσι, καὶ ἐνόησε Δαυὶδ ὅτι τέθνηκε τὸ παιδάριον· καὶ εἶπε Δαυὶδ πρὸς τοὺς παῖδας αὐτοῦ, εἰ τέθνηκε τὸ παιδάριον; καὶ εἶπαν, τέθνηκε.
\VS{20}Καὶ ἀνέστη Δαυὶδ ἐκ τῆς γῆς, καὶ ἐλούσατο, καὶ ἠλείψατο, καὶ ἤλλαξε τὰ ἱμάτια αὐτοῦ, καὶ εἰσῆλθεν εἰς τὸν οἶκον τοῦ Θεοῦ, καὶ προσεκύνησεν αὐτῷ, καὶ εἰσῆλθεν εἰς τὸν οἶκον αὐτοῦ, καὶ ᾔτησεν ἄρτον φαγεῖν, καὶ παρέθηκαν αὐτῷ ἄρτον, καὶ ἔφαγε.
\VS{21}Καὶ εἶπαν οἱ παῖδες αὐτοῦ πρὸς αὐτόν, τί τὸ ῥῆμα τοῦτο ὃ ἐποίησας ἕνεκα τοῦ παιδαρίου; ἔτι ζῶντος ἐνήστευες καὶ ἔκλαιες καὶ ἠγρύπνεις, καὶ ἡνίκα ἀπέθανε τὸ παιδάριον, ἀνέστης, καὶ ἔφαγες ἄρτον, καὶ πέπωκας;
\VS{22}Καὶ εἶπε Δαυὶδ, ἐν τῷ τὸ παιδάριον ἔτι ζῇν ἐνήστευσα καὶ ἔκλαυσα, ὅτι εἶπα, τίς οἶδεν εἰ ἐλεήσει με Κύριος, καὶ ζήσεται τὸ παιδάριον;
\VS{23}Καὶ νῦν τέθνηκεν, ἱνατί τοῦτο ἐγὼ νηστεύω; μὴ δυνήσομαι ἐπιστρέψαι αὐτὸν ἔτι; ἐγὼ πορεύσομαι πρὸς αὐτόν, καὶ αὐτὸς οὐκ ἀναστρέψει πρὸς μέ.
\par }{\PP \VS{24}Καὶ παρεκάλεσε Δαυὶδ Βηρσαβεὲ τὴν γυναῖκα αὐτοῦ, καὶ εἰσῆλθε πρὸς αὐτὴν, καὶ ἐκοιμήθη μετʼ αὐτῆς, καὶ συνέλαβε καὶ ἔτεκεν υἱὸν, καὶ ἐκάλεσε τὸ ὄνομα αὐτοῦ Σαλωμὼν, καὶ Κύριος ἠγάπησεν αὐτόν.
\VS{25}Καὶ ἀπέστειλεν ἐν χειρὶ Νάθαν τοῦ προφήτου, καὶ ἐκάλεσε τὸ ὄνομα αὐτοῦ Ἰεδδεδὶ, ἕνεκεν Κυρίου.
\par }{\PP \VS{26}Καὶ ἐπολέμησεν Ἰωὰβ ἐν Ῥαββὰθ υἱῶν Ἀμμὼν, καὶ κατέλαβε τὴν πόλιν τῆς βασιλείας.
\VS{27}Καὶ ἀπέστειλεν Ἰωὰβ ἀγγέλους πρὸς Δαυὶδ, καὶ εἶπεν, ἐπολέμησα ἐν Ῥαββὰθ, καὶ κατελαβόμην τὴν πόλιν τῶν ὑδάτων.
\VS{28}Καὶ νῦν συνάγαγε τὸ κατάλοιπον τοῦ λαοῦ, καὶ παρέμβαλε ἐπὶ τὴν πόλιν, καὶ προκαταλαβοῦ αὐτὴν, ἵνα μὴ προκαταλάβωμαι ἐγὼ τὴν πόλιν, καὶ κληθῇ τὸ ὄνομά μου ἐπʼ αὐτήν.
\par }{\PP \VS{29}Καὶ συνήγαγε Δαυὶδ πάντα τὸν λαὸν, καὶ ἐπορεύθη εἰς Ῥαββὰθ, καὶ ἐπολέμησεν ἐν αὐτῇ, καὶ κατελάβετο αὐτήν.
\VS{30}Καὶ ἔλαβε τὸν στέφανον Μολχὸμ τοῦ βασιλέως αὐτῶν ἀπὸ τῆς κεφαλῆς αὐτοῦ, καὶ ὁ σταθμὸς αὐτοῦ τάλαντον χρυσίου, καὶ λίθου τιμίου, καὶ ἦν ἐπὶ τῆς κεφαλῆς Δαυὶδ, καὶ σκῦλα τῆς πόλεως ἐξήνεγκε πολλὰ σφόδρα.
\VS{31}Καὶ τὸν λαὸν τὸν ὄντα ἐν αὐτῇ ἐξήγαγε, καὶ ἔθηκεν ἐν τῷ πρίονι, καὶ ἐν τοῖς τριβόλοις τοῖς σιδηροῖς, καὶ ὑποτομεῦσι σιδηροῖς, καὶ διήγαγεν αὐτοὺς διὰ τοῦ πλινθίου· καὶ οὕτως ἐποίησε πάσαις ταῖς πόλεσιν υἱῶν Ἀμμών· καὶ ἐπέστρεψε Δαυὶδ καὶ πᾶς ὁ λαὸς εἰς Ἰερουσαλήμ.

\par }\Chap{13}{\PP \VerseOne{1}Καὶ ἐγενήθη μετὰ ταῦτα καὶ τῷ Ἀβεσσαλὼμ υἱῷ Δαυὶδ ἀδελφὴ καλὴ τῷ εἴδει σφόδρα, καὶ ὄνομα αὐτῇ Θημὰρ, καὶ ἠγάπησεν αὐτὴν Ἀμνὼν υἱὸς Δαυίδ.
\VS{2}Καὶ ἐθλίβετο Ἀμνὼν ὥστε ἀῤῥωστεῖν διὰ Θημὰρ τὴν ἀδελφὴν αὐτοῦ, ὅτι παρθένος ἦν αὕτη, καὶ ὑπέρογκον ἐν ὀφθαλμοῖς Ἀμνὼν τοῦ ποιῆσαί τι αὐτῇ.
\VS{3}Καὶ ἦν τῷ Ἀμνὼν ἑταῖρος, καὶ ὄνομα αὐτῷ Ἰωναδὰβ, υἱὸς Σαμαὰ τοῦ ἀδελφοῦ Δαυίδ· καὶ Ἰωναδὰβ ἀνὴρ σοφὸς σφόδρα,
\VS{4}καὶ εἶπεν αὐτῷ, τί σοι ὅτι σὺ οὕτως ἀσθενὴς, υἱὲ τοῦ βασιλέως, τὸ πρωῒ πρωΐ; οὐκ ἀπαγγέλλεις μοι; καὶ εἶπεν αὐτῷ Ἀμνὼν, Θημὰρ τὴν ἀδελφὴν Ἀβεσσαλὼμ τοῦ ἀδελφοῦ μου ἐγὼ ἀγαπῶ.
\VS{5}Καὶ εἶπεν αὐτῷ Ἰωναδὰβ, κοιμήθητι ἐπὶ τῆς κοίτης σου καὶ μαλακίσθητι, καὶ εἰσελεύσεται ὁ πατήρ σου τοῦ ἰδεῖν σε, καὶ ἐρεῖς πρὸς αὐτὸν, ἐλθέτω δὴ Θημὰρ ἡ ἀδελφή μου, καὶ ψωμισάτω με, καὶ ποιησάτω κατʼ ὀφθαλμούς μου βρῶμα, ὅπως ἴδω καὶ φάγω ἐκ τῶν χειρῶν αὐτῆς.
\VS{6}Καὶ ἐκοιμήθη Ἀμνὼν καὶ ἠῤῥώστησε· καὶ εἰσῆλθεν ὁ βασιλεὺς ἰδεῖν αὐτόν· καὶ εἶπεν Ἀμνὼν πρὸς τὸν βασιλέα, ἐλθέτω δὴ Θημὰρ ἡ ἀδελφή μου πρὸς μὲ, καὶ κολλυρισάτω ἐν ὀφθαλμοῖς μου δύο κολλυρίδας, καὶ φάγομαι ἐκ τῆς χειρὸς αὐτῆς.
\par }{\PP \VS{7}Καὶ ἀπέστειλε Δαυὶδ πρὸς Θημὰρ εἰς τὸν οἶκον, λέγων, πορεύθητι δὴ εἰς τὸν οἶκον τοῦ ἀδελφοῦ σου, καὶ ποίησον αὐτῷ βρῶμα.
\VS{8}Καὶ ἐπορεύθη Θημὰρ εἰς τὸν οἶκον Ἀμνὼν ἀδελφοῦ αὐτῆς, καὶ αὐτὸς κοιμώμενος· καὶ ἔλαβε τὸ σταῖς καὶ ἐφύρασε, καὶ ἐκολλύρισε κατʼ ὀφθαλμοὺς αὐτοῦ, καὶ ἥψησε τὰς κολλυρίδας.
\VS{9}Καὶ ἔλαβε τὸ τήγανον καὶ κατεκένωσεν ἐνώπιον αὐτοῦ, καὶ οὐκ ἠθέλησε φαγεῖν· καὶ εἶπεν Ἀμνὼν, ἐξαγάγετε πάντα ἄνδρα ἐπάνωθέν μου· καὶ ἐξήγαγον πάντα ἄνδρα ἀπὸ ἐπάνωθεν αὐτοῦ.
\VS{10}Καὶ εἶπεν Ἀμνὼν πρὸς Θημὰρ, εἰσένεγκε τὸ βρῶμα εἰς τὸ ταμιεῖον, καὶ φάγομαι ἐκ τῆς χειρός σου· καὶ ἔλαβε Θημὰρ τὰς κολλυρίδας ἃς ἐποίησε, καὶ εἰσήνεγκε τῷ Ἀμνὼν ἀδελφῷ αὐτῆς εἰς τὸν κοιτῶνα.
\VS{11}Καὶ προσήγαγεν αὐτῷ τοῦ φαγεῖν, καὶ ἐπελάβετο αὐτῆς, καὶ εἶπεν αὐτῇ, δεῦρο, κοιμήθητι μετʼ ἐμοῦ, ἀδελφή μου.
\VS{12}Καὶ εἶπεν αὐτῷ, μὴ ἀδελφέ μου, μὴ ταπεινώσῃς με, διότι οὐ ποιηθήσεται οὕτως ἐν Ἰσραὴλ, μὴ ποιήσῃς τὴν ἀφροσύνην ταύτην.
\VS{13}Καὶ ἐγὼ ποῦ ἀποίσω τὸ ὄνειδός μου; καὶ σὺ ἔσῃ ὡς εἷς τῶν ἀφρόνων ἐν Ἰσραήλ· καὶ νῦν λάλησον δὴ πρὸς τὸν βασιλέα, ὅτι οὐ μὴ κωλύσῃ με ἀπὸ σοῦ.
\VS{14}Καὶ οὐκ ἠθέλησεν Ἀμνὼν τοῦ ἀκοῦσαι τῆς φωνῆς αὐτῆς· καὶ ἐκραταίωσεν ὑπὲρ αὐτὴν, καὶ ἐταπείνωσεν αὐτήν, καὶ ἐκοιμήθη μετʼ αὐτῆς.
\par }{\PP \VS{15}Καὶ ἐμίσησεν αὐτὴν Ἀμνὼν μῖσος μέγα σφόδρα, ὅτι μέγα τὸ μῖσος ὃ ἐμίσησεν αὐτὴν μπὲρ τὴν ἀγάπην ἣν ἠγάπησεν αὐτὴν, ὅτι μείζων ἡ κακία ἡ ἐσχάτη ἢ ἡ πρώτη· καὶ εἶπεν αὐτῇ Ἀμνὼν, ἀνάστηθι, καὶ πορεύου.
\VS{16}Καὶ εἶπεν αὐτῷ Θημὰρ περὶ τῆς κακίας τῆς μεγάλης ταύτης ὑπὲρ ἑτέραν ἣν ἐποίησας μετʼ ἐμοῦ, τοῦ ἐξαποστεῖλαί με· καὶ οὐκ ἐθέλησεν Ἀμνὼν ἀκοῦσαι τῆς φωνῆς αὐτῆς.
\VS{17}Καὶ ἐκάλεσε τὸ παιδάριον αὐτοῦ τὸν προεστηκότα τοῦ οἴκου, καὶ εἶπεν αὐτῷ, ἐξαποστείλατε δὴ ταύτην ἀπʼ ἐμοῦ ἔξω, καὶ ἀπόκλεισον τὴν θύραν ὀπίσω αὐτῆς.
\VS{18}Καὶ ἐπʼ αὐτῆς ἦν χιτὼν καρπωτὸς, ὅτι οὕτως ἐνεδιδύσκοντο αἱ θυγατέρες τοῦ βασιλέως αἱ παρθένοι τοὺς ἐπενδύτας αὐτῶν· καὶ ἐξήγαγεν αὐτὴν ὁ λειτουργὸς αὐτοῦ ἔξω, καὶ ἀπέκλεισε τὴν θύραν ὀπίσω αὐτῆς.
\par }{\PP \VS{19}Καὶ ἔλαβε Θημὰρ σποδὸν, καὶ ἐπέθηκεν ἐπὶ τὴν κεφαλὴν αὐτῆς· καὶ τὸν χιτῶνα τὸν καρπωτὸν τὸν ἐπʼ αὐτῆς διέῤῥηξε· καὶ ἐπέθηκε τὰς χεῖρας αὐτῆς ἐπὶ τὴν κεφαλὴν αὐτῆς, καὶ ἐπορεύθη πορευομένη καὶ κράζουσα.
\VS{20}Καὶ εἶπε πρὸς αὐτὴν Ἀβεσσαλὼμ ὁ ἀδελφὸς αὐτῆς, μὴ Ἀμνὼν ὁ ἀδελφός σου ἐγένετο μετὰ σοῦ; καὶ νῦν ἀδελφή μου κώφευσον, ὅτι ἀδελφός σου ἐστί· μὴ θῇς τὴν καρδίαν σου τοῦ λαλῆσαι τὸ ῥῆμα τοῦτο· καὶ ἐκάθισε Θημὰρ χηρεύουσα ἐν τῷ οἴκῳ Ἀβεσσαλὼμ τοῦ ἀδελφοῦ αὐτῆς.
\par }{\PP \VS{21}Καὶ ἤκουσεν ὁ βασιλεὺς Δαυὶδ πάντας τοὺς λόγους τούτους, καὶ ἐθυμώθη σφόδρα, καὶ οὐκ ἐλύπησε τὸ πνεῦμα Ἀμνὼν τοῦ υἱοῦ αὐτοῦ, ὅτι ἠγάπα αὐτὸν, ὅτι πρωτότοκος αὐτοῦ ἦν.
\VS{22}Καὶ οὐκ ἐλάλησεν Ἀβεσσαλὼμ μετὰ Ἀμνὼν ἀπὸ πονηροῦ ἕως ἀγαθοῦ, ὅτι ἐμίσει Ἀβεσσαλὼμ τὸν Ἀμνὼν ἐπὶ λόγου οὗ ἐταπείνωσε Θημὰρ τὴν ἀδελφὴν αὐτοῦ.
\VS{23}Καὶ ἐγένετο εἰς διετηρίδα ἡμερῶν, καὶ ἦσαν κείροντες τῷ Ἀβεσσαλὼμ ἐν Βελασὼρ τῇ ἐχόμενα Ἐφραὶμ, καὶ ἐκάλεσεν Ἀβεσσαλὼμ πάντας τοὺς υἱοὺς τοῦ βασιλέως.
\VS{24}Καὶ ἦλθεν Ἀβεσσαλὼμ πρὸς τὸν βασιλέα, καὶ εἶπεν, ἰδοὺ δὴ κείρουσι τῷ δούλῳ σου, πορευθήτω δὴ ὁ βασιλεὺς καὶ οἱ παῖδες αὐτοῦ μετὰ τοῦ δούλου σου.
\VS{25}Καὶ εἶπεν ὁ βασιλεὺς πρὸς Ἀβεσσαλὼμ, μὴ δὴ υἱέ μου, μὴ πορεύθῶμεν πάντες ἡμεῖς, καὶ οὐ μὴ καταβαρυνθῶμεν ἐπὶ σέ· καὶ ἐβιάσατο αὐτόν, καὶ οὐκ ἠθέλησε τοῦ πορευθῆναι, καὶ εὐλόγησεν αὐτόν.
\VS{26}Καὶ εἶπεν Ἀβεσσαλὼμ πρὸς αὐτὸν, καὶ εἰ μή, πορευθήτω δὴ μεθʼ ἡμῶν Ἀμνὼν ὁ ἀδελφός μου· καὶ εἶπεν αὐτῷ ὁ βασιλεὺς, ἱνατί πορευθῇ μετὰ σοῦ;
\VS{27}Καὶ ἐβιάσατο αὐτὸν Ἀβεσσαλώμ, καὶ ἀπέστειλε μετʼ αὐτοῦ τὸν Ἀμνὼν καὶ πάντας τοὺς υἱοὺς τοῦ βασιλέως· καὶ ἐποίησεν Ἀβεσσαλὼμ πότον κατὰ τὸν πότον τοῦ βασιλέως.
\par }{\PP \VS{28}Καὶ ἐνετείλατο Ἀβεσσαλὼμ τοῖς παιδαρίοις αὐτοῦ, λέγων, ἴδετε ὡς ἂν ἀγαθυνθῇ ἡ καρδία Ἀμνὼν ἐν τῷ οἴνῳ, καὶ εἴπω πρὸς ὑμᾶς, πατάξατε τὸν Ἀμνὼν, καὶ θανατώσατε αὐτόν· μὴ φοβηθῆτε, ὅτι οὐχὶ ἐγώ εἰμι ὁ ἐντελλόμενος ὑμῖν; ἀνδρίζεσθε καὶ γίνεσθε εἰς υἱοὺς δυνάμεως.
\VS{29}Καὶ ἐποίησαν τὰ παιδάρια Ἀβεσσαλὼμ τῷ Ἀμνὼν, καθὰ ἐνετείλατο αὐτοῖς Ἀβεσσαλώμ· καὶ ἀνέστησαν πάντες οἱ υἱοὶ τοῦ βασιλέως, καὶ ἐπεκάθισαν ἀνὴρ ἐπὶ τὴν ἡμιονον αὐτοῦ, καὶ ἔφυγον.
\par }{\PP \VS{30}Καὶ ἐγένετο, αὐτῶν ὄντων ἐν τῇ ὁδῷ, καὶ ἡ ἀκοὴ ἦλθε πρὸς Δαυὶδ, λέγων, ἐπάταξεν Ἀβεσσαλὼμ πάντας τοὺς υἱοὺς τοῦ βασιλέως, καὶ οὐ κατελείφθη ἐξ αὐτῶν οὐδὲ εἷς.
\VS{31}Καὶ ἀνέστη ὁ βασιλεὺς καὶ διέῤῥηξε τὰ ἱμάτια αὐτοῦ καὶ ἐκοιμήθη ἐπὶ τὴν γῆν, καὶ πάντες οἱ παῖδες αὐτοῦ οἱ περιεστῶτες αὐτῷ διέῤῥηξαν τὰ ἱμάτια αὐτῶν.
\VS{32}Καὶ ἀπεκρίθη Ἰωναδὰβ υἱὸς Σαμαὰ ἀδελφοῦ Δαυὶδ, καὶ εἶπε, μὴ εἰπάτω ὁ κύριός μου ὁ βασιλεὺς ὅτι πάντα τὰ παιδάρια τοὺς υἱοὺς τοῦ βασιλέως ἐθανάτωσεν, ὅτι Ἀμνὼν μονώτατος ἀπέθανεν, ὅτι ἐπὶ στόματος Ἀβεσσαλὼμ ἦν κείμενος ἀπὸ τῆς ἡμέρας ἧς ἐταπείνωσε Θημὰρ τὴν ἀδελφὴν αὐτοῦ.
\VS{33}Καὶ νῦν μὴ θέσθω ὁ κύριός μου ὁ βασιλεὺς ἐπὶ τὴν καρδίαν αὐτοῦ ῥῆμα, λέγων, πάντες οἱ υἱοὶ τοῦ βασιλέως ἀπέθανον· ὅτι ἀλλʼ ἢ Ἀμνὼν μονώτατος ἀπέθανε.
\VS{34}Καὶ ἀπέδρα Ἀβεσσαλώμ.
\par }{\PP Καὶ ᾖρε τὸ παιδάριον ὁ σκοπὸς τοὺς ὀφθαλμοὺς αὐτοῦ, καὶ εἶδε· καὶ ἰδοὺ λαὸς πολὺς πορευόμενος ἐν τῇ ὁδῷ ὄπισθεν αὐτοῦ ἐκ πλευρᾶς τοῦ ὄρους ἐν τῇ καταβάσει· καὶ παρεγένετο ὁ σκοπὸς καὶ ἀπήγγειλε τῷ βασιλεῖ, καὶ εἶπεν, ἄνδρας ἑώρακα ἐκ τῆς ὁδοῦ τὴς Ὠρωνῆν ἐκ μέρους τοῦ ὄρους.
\VS{35}Καὶ εἶπεν Ἰωναδὰβ πρὸς τὸν βασιλέα, ἰδοὺ οἱ υἱοὶ τοῦ βασιλέως πάρεισι· κατὰ τὸν λόγον τοῦ δούλου σου, οὕτως ἐγένετο.
\VS{36}Καὶ ἐγένετο ἡνίκα συνετέλεσε λαλῶν, καὶ ἰδοὺ οἱ υἱοὶ τοῦ βασιλέως ἦλθον, καὶ ἐπῇραν τὴν φωνὴν αὐτῶν καὶ ἔκλαυσαν· καί γε ὁ βασιλεὺς καὶ πάντες οἱ παῖδες αὐτοῦ ἔκλαυσαν κλαυθμὸν μέγαν σφόδρα.
\par }{\PP \VS{37}Καὶ Ἀβεσσαλὼμ ἔφυγε, καὶ ἐπορεύθη, πρὸς Θολμὶ υἱὸν Ἐμιοὺδ βασιλέα Γεδσοὺρ εἰς γῆν Χαμααχάδ· καὶ ἐπένθησεν ὁ βασιλεὺς Δαυὶδ ἐπὶ τὸν υἱὸν αὐτοῦ πάσας τὰς ἡμέρας.
\VS{38}Καὶ Ἀβεσσαλὼμ ἀπέδρα, καὶ ἐπορεύθη εἰς Γεδσοὺρ, καὶ ἦν ἐκεῖ ἔτη τρία.
\VS{39}Καὶ ἐκόπασεν ὁ βασιλεὺς Δαυὶδ τοῦ ἐξελθεῖν πρὸς Ἀβεσσαλὼμ, ὅτι παρεκλήθη ἐπὶ Ἀμνὼν, ὅτι ἀπέθανε.

\par }\Chap{14}{\PP \VerseOne{1}Καὶ ἔγνω Ἰωὰβ υἱὸς Σαρουίας ὅτι ἡ καρδία τοῦ βασιλέως ἐπὶ Ἀβεσσαλώμ.
\VS{2}Καὶ ἀπέστειλεν Ἰωὰβ εἰς Θεκωὲ, καὶ ἔλαβεν ἐκεῖθεν γυναῖκα σοφὴν, καὶ εἶπε πρὸς αὐτὴν, πένθησον δὴ, καὶ ἔνδυσαι ἱμάτια πενθικὰ, καὶ μὴ ἀλείψῃ ἔλαιον, καὶ ἔσῃ ὡς γυνὴ πενθοῦσα ἐπὶ τεθνηκότι τοῦτο ἡμέρας πολλὰς,
\VS{3}καὶ ἐλεύσῃ πρὸς τὸν βασιλέα, καὶ λαλήσεις πρὸς αὐτὸν κατὰ τὸ ῥῆμα τοῦτο· καὶ ἔθηκεν Ἰωὰβ τοὺς λόγους ἐν τῷ στόματι αὐτῆς.
\par }{\PP \VS{4}Καὶ εἰσῆλθεν ἡ γυνὴ ἡ Θεκωῖτις πρὸς τὸν βασιλέα, καὶ ἔπεσεν ἐπὶ πρόσωπον αὐτῆς εἰς τὴν γῆν, καὶ προσεκύνησεν αὐτῷ, καὶ εἶπε, σῶσον βασιλεῦ, σῶσον.
\VS{5}Καὶ εἶπε πρὸς αὐτὴν ὁ βασιλεὺς, τί ἐστι σοι;
\par }{\PP \VS{6}Ἡ δὲ εἶπε, καὶ μάλα γυνὴ χήρα ἐγώ εἰμι, καὶ ἀπέθανεν ὁ ἀνήρ μου. Καί γε τῇ δούλῃ σου δύο υἱοὶ, καὶ ἐμαχέσαντο ἀμφότεροι ἐν τῷ ἀγρῷ, καὶ οὐκ ἦν ὁ ἐξαιρούμενος ἀναμέσον αὐτῶν· καὶ ἔπαισεν ὁ εἷς τὸν ἕνα ἀδελφὸν αὐτοῦ, καὶ ἐθανάτωσεν αὐτόν.
\VS{7}Καὶ ἰδοὺ ἐπανέστη ὅλη ἡ πατριὰ πρὸς τὴν δούλην σου, καὶ εἶπαν, δὸς τὸν παίσαντα τὸν ἀδελφὸν αὐτοῦ, καὶ θανατώσομεν αὐτὸν ἀντὶ τῆς ψυχῆς τοῦ ἀδελφοῦ αὐτοῦ οὗ ἀπέκτεινε, καὶ ἐξαροῦμεν καί γε τὸν κληρονόμον ὑμῶν· καὶ σβέσουσι τὸν ἄνθρακά μου τὸν καταλειφθέντα ὥστε μὴ θέσθαι τῷ ἀνδρί μου κατάλειμμα καὶ ὄνομα ἐπὶ προσώπου τῆς γῆς.
\par }{\PP \VS{8}Καὶ εἶπεν ὁ βασιλεὺς πρὸς τὴν γυναῖκα, ὑγιαίνουσα βάδιζε εἰς τὸν οἶκόν σου, κᾀγὼ ἐντελοῦμαι περὶ σοῦ.
\VS{9}Καὶ εἶπεν ἡ γυνὴ ἡ Θεκωΐτις πρὸς τὸν βασιλέα, ἐπʼ ἐμὲ, κύριέ μου βασιλεῦ, ἡ ἀνομία καὶ ἐπὶ τὸν οἶκον τοῦ πατρός μου, καὶ ὁ βασιλεὺς καὶ ὁ θρόνος αὐτοῦ ἀθῶος.
\VS{10}Καὶ εἶπεν ὁ βασιλεύς, τίς ὁ λαλῶν πρὸς σὲ, καὶ ἄξεις αὐτὸν πρὸς ἐμὲ, καὶ οὐ προσθήσει ἔτι ἅψασθαι αὐτοῦ;
\VS{11}Καὶ εἶπε, μνημονευσάτω δὴ ὁ βασιλεὺς τὸν Κύριον Θεὸν αὐτοῦ πληθυνθῆναι ἀγχιστέα τοῦ αἵματος τοῦ διαφθεῖραι, καὶ οὐ μὴ ἐξάρωσι τὸν υἱόν μου· καὶ εἶπε, ζῇ Κύριος, εἶ πεσεῖται ἀπὸ τῆς τριχὸς τοῦ υἱοῦ σου ἐπὶ τὴν γῆν.
\par }{\PP \VS{12}Καὶ εἶπεν ἡ γυνὴ, λαλησάτω δὴ ἡ δούλη σου πρὸς τὸν κύριόν μου βασιλέα ῥῆμα· καὶ εἶπε, λάλησον.
\VS{13}Καὶ εἶπεν ἡ γυνὴ, ἱνατί ἐλογίσω τοιοῦτο ἐπὶ λαὸν Θεοῦ; ἢ ἐκ στόματος τοῦ βασιλέως ὁ λόγος οὗτος ὡς πλημμέλεια, τοῦ μὴ ἐπιστρέψαι τὸν βασιλέα τὸν ἐξωσμένον αὐτοῦ;
\VS{14}Ὅτι θανάτῳ ἀποθανούμεθα, καὶ ὥσπερ τὸ ὕδωρ τὸ καταφερόμενον ἐπὶ τῆς γῆς ὃ οὐ συναχθήσεται, καὶ λήψεται ὁ Θεὸς ψυχὴν, καὶ λογιζόμενος τοῦ ἐξῶσαι ἀπʼ αὐτοῦ ἐξεωσμένον.
\VS{15}Καὶ νῦν ὃ ἦλθον λαλῆσαι πρὸς τὸν βασιλέα τὸν κύριόν μου τὸ ῥῆμα τοῦτο, ὅτι ὄψεταί με ὁ λαὸς, καὶ ἐρεῖ ἡ δούλη σου, λαλησάτω δὴ πρὸς τὸν κύριόν μον τὸν βασιλέα, εἴπως ποιήσει ὁ βασιλεὺς τὸ ῥῆμα τῆς δούλης αὐτοῦ
\VS{16}ὅτι ἀκούσει ὁ βασιλεύς· ῥυσάσθω τὴν δούλην αὐτοῦ ἐκ χειρὸς τοῦ ἀνδρὸς τοῦ ζητοῦντος ἐξᾶραί με καὶ τὸν υἱόν μου ἀπὸ κληρονομίας Θεοῦ.
\VS{17}Καὶ εἶπεν ἡ γυνή, εἰ ἤδη ὁ λόγος τοῦ κυρίου μου τοῦ βασιλέως εἰς θυσίας· ὅτι καθὼς ἄγγελος Θεοῦ, οὕτως ὁ κύριός μου ὁ βασιλεὺς, τοῦ ἀκούειν τὸ ἀγαθὸν καὶ τὸ πονηρόν· καὶ Κύριος ὁ Θεός σου ἔσται μετὰ σοῦ.
\par }{\PP \VS{18}Καὶ ἀπεκρίθη ὁ βασιλεὺς, καὶ εἶπε πρὸς τὴν γυναῖκα, μὴ δὴ κρύψῃς ἀπʼ ἐμοῦ ῥῆμα, ὃ ἐγὼ ἐπερωτῶ σε· καὶ εἶπεν ἡ γυνή, λαλησάτω δὴ ὁ κύριός μου ὁ βασιλεύς.
\VS{19}Καὶ εἶπεν ὁ βασιλεὺς, μὴ ἡ χεὶρ Ἰωὰβ ἐν παντὶ τούτῳ μετὰ σοῦ; καὶ εἶπεν ἡ γυνὴ τῷ βασιλεῖ, ζῇ ἡ ψυχή σου, κύριέ μου βασιλεῦ, εἰ ἔστιν εἰς τὰ δεξιὰ ἢ εἰς τὰ ἀριστερὰ ἐκ πάντων ὧν ἐλάλησεν ὁ κύριός μου ὁ βασιλεὺς, ὅτι ὁ δοῦλός σου Ἰωὰβ αὐτὸς ἐνετείλατό μοι, καὶ αὐτὸς ἔθετο ἐν τῷ στόματι τῆς δούλης σου πάντας τοὺς λόγους τούτους.
\VS{20}Ἕνεκεν τοῦ περιελθεῖν τὸ πρόσωπον τοῦ ῥήματος τούτου, ὃ ἐποίησεν ὁ δοῦλός σου Ἰωὰβ τὸν λόγον τοῦτον· καὶ ὁ κύριός μου σοφὸς καθὼς σοφία ἀγγέλου τοῦ Θεοῦ, τοῦ γνῶναι πάντα τὰ ἐν τῇ γῇ.
\par }{\PP \VS{21}Καὶ εἶπεν ὁ βασιλεὺς πρὸς Ἰωὰβ, ἰδοὺ δὴ ἐποίησά σοι κατὰ τὸν λόγον σου τοῦτον· πορεύου, ἐπιστρέψον τὸ παιδάριον τὸν Ἀβεσσαλώμ.
\VS{22}Καὶ ἔπεσεν Ἰωὰβ ἐπὶ πρόσωπον αὐτοῦ ἐπὶ τὴν γῆν, καὶ προσεκύνησε, καὶ εὐλόγησε τὸν βασιλέα· καὶ εἶπεν Ἰωὰβ, σήμερον ἔγνω ὁ δοῦλός σου ὅτι εὗρον χάριν ἐν ὀφθαλμοῖς σου, κύριέ μου βασιλεῦ, ὅτι ἐποίησεν ὁ κύριός μου ὁ βασιλεὺς τὸν λόγον τοῦ δούλου αὐτοῦ.
\VS{23}Καὶ ἀνέστη Ἰωὰβ, καὶ ἐπορεύθη εἰς Γεδσούρ, καὶ ἤγαγε τὸν Ἀβεσσαλὼμ εἰς Ἰερουσαλήμ.
\VS{24}Καὶ εἶπεν ὁ βασιλεὺς, ἀποστραφήτω εἰς τὸν οἶκον αὐτοῦ, καὶ τὸ πρόσωπόν μου μὴ βλεπέτω· καὶ ἀπέστρεψεν Ἀβεσσαλὼμ εἰς τὸν οἶκον αὐτοῦ, καὶ τὸ πρόσωπον τοῦ βασιλέως οὐκ εἶδε.
\par }{\PP \VS{25}Καὶ ὡς Ἀβεσσαλὼμ οὐκ ἦν ἀνὴρ ἐν παντὶ Ἰσραὴλ αἰνετὸς σφόδρα, ἀπὸ ἴχνους ποδὸς αὐτοῦ καὶ ἕως κορυφῆς αὐτοῦ οὐκ ἦν ἐν αὐτῷ μῶμος.
\VS{26}Καὶ ἐν τῷ κείρεσθαι αὐτὸν τὴν κεφαλὴν αὐτοῦ, καὶ ἐγένετο ἀπʼ ἀρχῆς ἡμερῶν εἰς ἡμέρας ὡς ἂν ἐκείρετο, ὅτι κατεβαρύνετο ἐπʼ αὐτὸν, καὶ κειρόμενος αὐτὴν ἔστησε τὴν τρίχα τῆς κεφαλῆς αὐτοῦ διακοσίους σίκλους ἐν τῷ σίκλῳ τῷ βασιλικῷ.
\VS{27}Καὶ ἐτέχθησαν τῷ Ἀβεσσαλὼμ τρεῖς υἱοὶ, καὶ θυγάτηρ μία, καὶ ὄνομα αὐτῇ Θημάρ· αὕτη ἦν γυνὴ καλὴ σφόδρα· καὶ γίνεται γυνὴ Ῥοβοὰμ υἱῷ Σαλωμὼν, καὶ τίκτει αὐτῷ τὸν Ἀβία.
\par }{\PP \VS{28}Καὶ ἐκάθισεν Ἀβεσσαλὼμ ἐν Ἱερουσαλὴμ δύο ἔτη ἡμερῶν, καὶ τὸ πρόσωπον τοῦ βασιλέως οὐκ εἶδε.
\VS{29}Καὶ ἀπέστειλεν Ἀβεσσαλὼμ πρὸς Ἰωὰβ ἀποστεῖλαι αὐτὸν πρὸς τὸν βασιλέα, καὶ οὐκ ἠθέλησεν ἐλθεῖν πρὸς αὐτόν· καὶ ἀπέστειλεν ἐκ δευτέρου πρὸς αὐτὸν, καὶ οὐκ ἠθέλησε παραγενέσθαι.
\VS{30}Καὶ εἶπεν Ἀβεσσαλὼμ πρὸς τοὺς παῖδας αὐτοῦ, ἴδετε, ἡ μερὶς ἐν ἀγρῷ τοῦ Ἰωὰβ ἐχόμενά μου, καὶ αὐτῷ ἐκεῖ κριθαὶ, πορεύεσθε καὶ ἐμπρήσατε αὐτὴν ἐν πυρί· καὶ ἐνέπρησαν οἱ παῖδες Ἀβεσσαλὼμ τὴν μερίδα· καὶ παραγίνονται οἱ δοῦλοι Ἰωὰβ πρὸς αὐτὸν διεῤῥηχότες τὰ ἱμάτια αὐτῶν, καὶ εἶπον, ἐνεπύρισαν οἱ δοῦλοι Ἀβεσσαλὼμ τὴν μερίδα ἐν πυρί.
\VS{31}Καὶ ἀνέστη Ἰωὰβ, καὶ ἦλθε πρὸς Ἀβεσσαλὼμ εἰς τὸν οἶκον, καὶ εἶπε πρὸς αὐτὸν, ἱνατί ἐνεπύρισαν οἱ παῖδές σου τὴν μερίδα τὴν ἐμὴν ἐν πυρί;
\VS{32}Καὶ εἶπεν Ἀβεσσαλὼμ πρὸς Ἰωὰβ, ἰδοὺ ἀπέστειλα πρὸς σὲ, λέγων, ἧκε ὧδε, καὶ ἀποστελῶ σε πρὸς τὸν βασιλέα, λέγων, ἱνατί ἦλθον ἐκ Γεδσούρ; ἀγαθόν μοι ἦν εἶναι ἐκεῖ· καὶ νῦν ἰδοὺ τὸ πρόσωπον τοῦ βασιλέως οὐκ εἶδον· εἰ δέ ἐστιν ἐν ἐμοὶ ἀδικία, καὶ θανάτωσόν με.
\par }{\PP \VS{33}Καὶ εἰσῆλθεν Ἰωὰβ πρὸς τὸν βασιλέα, καὶ ἀπήγγειλεν αὐτῷ· καὶ ἐκάλεσε τὸν Ἀβεσσαλώμ· καὶ εἰσῆλθε πρὸς τὸν βασιλέα, καὶ προσεκύνησεν αὐτῷ, καὶ ἔπεσεν ἐπὶ πρόσωπον αὐτοῦ ἐπὶ τὴν γῆν, καὶ κατὰ πρόσωπον τοῦ βασιλέως· καὶ κατεφίλησεν ὁ βασιλεὺς τὸν Ἀβεσσαλώμ.

\par }\Chap{15}{\PP \VerseOne{1}Καὶ ἐγένετο μετὰ ταῦτα καὶ ἐποίησεν ἑαυτῷ Ἀβεσσαλὼμ ἅρματα, καὶ ἵππους, καὶ πεντήκοντα ἄνδρας παρατρέχειν ἔμπροσθεν αὐτοῦ.
\VS{2}Καὶ ὤρθρισεν Ἀβεσσαλὼμ, καὶ ἔστη ἀνὰ χεῖρα τῆς ὁδοῦ τῆς πύλης· καὶ ἐγένετο πᾶς ἀνὴρ ᾧ ἐγένετο κρίσις, ἦλθε πρὸς τὸν βασιλέα εἰς κρίσιν, καὶ ἐβόησε πρὸς αὐτὸν Ἀβεσσαλὼμ, καὶ ἔλεγεν αὐτῷ, ἐκ ποίας πόλεως σὺ εἶ; καὶ εἶπεν, ἐκ μιᾶς φυλῶν Ἰσραὴλ ὁ δοῦλός σου.
\VS{3}Καὶ εἶπε πρὸς αὐτὸν ὁ Ἀβεσσαλὼμ, ἰδοὺ οἱ λόγοι σου ἀγαθοὶ καὶ εὔκολοι, καὶ ὁ ἀκούων οὐκ ἔστι σοι παρὰ τοῦ βασιλέως.
\VS{4}Καὶ εἶπεν Ἀβεσσαλὼμ, τίς με καταστήσει κριτὴν ἐν τῇ γῇ, καὶ ἐπʼ ἐμὲ ἐλεύσεται πᾶς ἀνὴρ ᾧ ἐὰν ᾖ ἀντιλογία καὶ κρίσις, καὶ δικαιώσω αὐτόν;
\VS{5}Καὶ ἐγένετο ἐν τῷ ἐγγίζειν ἄνδρα τοῦ προσκυνῆσαι αὐτῷ, καὶ ἐξέτεινε τὴν χεῖρα αὐτοῦ, καὶ ἐπελαμβάνετο αὐτοῦ, καὶ κατεφίλησεν αὐτόν.
\VS{6}Καὶ ἐποίησεν Ἀβεσσαλὼμ κατὰ τὸ ῥῆμα τοῦτο παντὶ Ἰσραὴλ τοῖς παραγινομένοις εἰς κρίσιν πρὸς τὸν βασιλέα, καὶ ἰδιοποιεῖτο Ἀβεσσαλὼμ τὴν καρδίαν ἀνδρῶν Ἰσραήλ.
\par }{\PP \VS{7}Καὶ ἐγένετο ἀπὸ τέλους τεσσαράκοντα ἐτῶν, καὶ εἶπεν Ἀβεσσαλὼμ πρὸς τὸν πατέρα αὐτοῦ, πορεύσομαι δὴ, καὶ ἀποτίσω τὰς εὐχάς μου, ἃς ηὐξάμην τῷ Κυρίῳ ἐν Χεβρών·
\VS{8}Ὅτι εὐχὴν ηὔξατο ὁ δοῦλός σου ἐν τῷ οἰκεῖν με ἐν Γεδσοὺρ ἐν Συρίᾳ, λέγων, ἐὰν ἐπιστρέφων ἐπιστρέψῃ με Κύριος εἰς Ἰερουσαλὴμ, καὶ λατρεύσω τῷ Κυρίῳ.
\VS{9}Καὶ εἶπεν αὐτῷ ὁ βασιλεὺς, Βάδιζε εἰς εἰρήνην· καὶ ἀναστὰς ἐπορεύθη εἰς Χεβρών.
\par }{\PP \VS{10}Καὶ ἀπέστειλεν Ἀβεσσαλὼμ κατασκόπους ἐν πάσαις φυλαῖς Ἰσραὴλ, λέγων, ἐν τῷ ἀκοῦσαι ὑμᾶς τὴν φωνὴν τῆς κερατίνης, καὶ ἐρεῖτε, βεβασίλευκε βασιλεὺς Ἀβεσσαλὼμ ἐν Χεβρών.
\VS{11}Καὶ μετὰ Ἀβεσσαλὼμ ἐπορεύθησαν διακόσιοι ἄνδρες ἐξ Ἰερουσαλὴμ κλητοί· καὶ πορευόμενοι τῇ ἁπλότητι αὐτῶν, καὶ οὐκ ἔγνωσαν πᾶν ῥῆμα.
\VS{12}Καὶ ἀπέστειλεν Ἀβεσσαλὼμ τῷ Ἀχειτόφελ τῷ Θεκωνεὶ, σύμβουλον Δαυὶδ, ἐκ τῆς πόλεως αὐτοῦ εἰς Γωλὰ, ἐν τῷ θυσιάζειν αὐτόν· καὶ ἐγένετο σύντριμμα ἰσχυρόν· καὶ ὁ λαὸς ὁ πορευόμενος καὶ πολὺς μετὰ Ἀβεσσαλώμ.
\par }{\PP \VS{13}Καὶ παρεγένετο ἀπαγγέλων πρὸς Δαυὶδ, λέγων, ἐγενήθη ἡ καρδία ἀνδρῶν Ἰσραὴλ ὀπίσω Ἀβεσσαλώμ.
\VS{14}Καὶ εἶπε Δαυὶδ πᾶσι τοῖς παισὶν αὐτοῦ τοῖς μετʼ αὐτοῦ τοῖς ἐν Ἰερουσαλὴμ, ἀνάστητε καὶ φύγωμεν, ὅτι οὐκ ἔστιν ἡμῖν σωτηρία ἀπὸ προσώπου Ἀβεσσαλώμ· ταχύνατε τοῦ πορευθῆναι, ἵνα μὴ ταχύνῃ καὶ καταλάβῃ ἡμᾶς, καὶ ἐξώσῃ ἐφʼ ἡμᾶς τὴν κακίαν, καὶ πατάξῃ τὴν πόλιν ἐν στόματι μαχαίρης.
\VS{15}Καὶ εἶπον οἱ παῖδες τοῦ βασιλέως πρὸς τὸν βασιλέα, κατὰ πάντα ὅσα αἱρεῖται ὁ κύριος ἡμῶν ὁ βασιλεὺς, ἰδοὺ οἱ παῖδές σου.
\par }{\PP \VS{16}Καὶ ἐξῆλθεν ὁ βασιλεὺς καὶ πᾶς ὁ οἶκος αὐτοῦ τοῖς ποσὶν αὐτῶν· καὶ ἀφῆκεν ὁ βασιλεὺς δέκα γυναῖκας τῶν παλλακῶν αὐτοῦ φυλάσσειν τὸν οἶκον.
\VS{17}Καὶ ἐξῆλθεν ὁ βασιλεὺς καὶ πάντες οἱ παῖδες αὐτοῦ πεζῇ, καὶ ἔστησαν ἐν οἴκῳ τῷ μακράν.
\VS{18}Καὶ πάντες οἱ παῖδες αὐτοῦ ἀνὰ χεῖρα αὐτοῦ παρῆγον, καὶ πᾶς Χελεθὶ καὶ πᾶς ὁ Φελεθὶ, καὶ ἔστησαν ἐπὶ τῆς ἐλαίας ἐν τῇ ἐρήμῳ· καὶ πᾶς ὁ λαὸς παρεπορεύετο ἐχόμενος αὐτοῦ, καὶ πάντες οἱ περὶ αὐτὸν, καὶ πάντες οἱ ἁδροὶ, καὶ πάντες οἱ μαχηταὶ ἑξακόσιοι ἄνδρες· καὶ παρῆσαν ἐπὶ χεῖρα αὐτοῦ· καὶ πᾶς ὁ Χελεθι, καὶ πᾶς ὁ Φελεθὶ, καὶ πάντες οἱ Γεθαῖοι οἱ ἑξακόσιοι ἄνδρες οἱ ἐλθόντες τοῖς ποσὶν αὐτῶν ἐκ Γὲθ, καὶ πορευόμενοι ἐπὶ πρόσωπον τοῦ βασιλέως.
\par }{\PP \VS{19}Καὶ εἶπεν ὁ βασιλεὺς πρὸς Ἐθὶ τὸν Γεθαῖον, ἱνατί πορεύῃ καὶ σὺ μεθʼ ἡμῶν; ἐπίστρεφε, καὶ οἴκει μετὰ τοῦ βασιλέως, ὅτι ξένος εἶ σὺ, καὶ ὅτι μετῴκηκας σὺ ἐκ τοῦ τόπου σου.
\VS{20}Εἰ ἐχθὲς παραγέγονας, καὶ σήμερον κινήσω σε μεθʼ ἡμῶν; καί γε μεταναστήσεις τὸν τόπον σου· χθὲς ἡ ἐξέλευσίς σου, καὶ σήμερον μετακινήσω σε μεθʼ ἡμῶν τοῦ πορευθῆναι; καὶ ἐγὼ πορεύσομαι οὗ ἐὰν ἐγὼ πορευθῶ· ἐπιστρέφου καὶ ἐπίστρεψον τοὺς ἀδελφούς σου μετὰ σοῦ, καὶ Κύριος ποιήσει μετὰ σοῦ ἔλεος καὶ ἀλήθειαν.
\VS{21}Καὶ ἀπεκρίθη Ἐθὶ τῷ βασιλεῖ, καὶ εἶπε, ζῇ Κύριος καὶ ζῇ ὁ κύριός μου ὁ βασιλεὺς, ὅτι εἰς τὸν τόπον οὗ ἐὰν ᾖ ὁ κύριός μου, καὶ ἐὰν εἰς θάνατον καὶ ἐὰν εἰς ζωὴν, ὅτι ἐκεῖ ἔσται ὁ δοῦλός σου.
\VS{22}Καὶ εἶπεν ὁ βασιλεὺς πρὸς Ἐθὶ, δεῦρο, καὶ διάβαινε μετʼ ἐμοῦ· καὶ παρῆλθεν Ἐθὶ ὁ Γεθαῖος καὶ ὁ βασιλεὺς, καὶ πάντες οἱ παῖδες αὐτοῦ καὶ πᾶς ὁ ὄχλος ὁ μετʼ αὐτοῦ.
\par }{\PP \VS{23}Καὶ πᾶσα ἡ γῆ ἔκλαιε φωνῇ μεγάλῃ· καὶ πᾶς ὁ λαὸς παρεπορεύοντο ἐν τῷ χειμάῤῥῳ τῶν Κέδρων· καὶ ὁ βασιλεὺς διέβη τὸν χειμάῤῥουν Κέδρων· καὶ πᾶς ὁ λαὸς καὶ ὁ βασιλεὺς παρεπορεύοντο ἐπὶ πρόσωπον ὁδοῦ τὴν ἔρημον.
\par }{\PP \VS{24}Καὶ ἰδοὺ καί γε Σαδὼκ καὶ πάντες οἱ Λευῖται μετʼ αὐτοῦ, αἴροντες τὴν κιβωτὸν διαθήκης Κυρίου ἀπὸ Βαιθάρ· καὶ ἔστησαν τὴν κιβωτὸν τοῦ Θεοῦ· καὶ ἀνέβη Ἀβιάθαρ ἕως ἐπαύσατο πᾶς ὁ λαὸς παρελθεῖν ἐκ τῆς πόλεως.
\VS{25}Καὶ εἶπεν ὁ βασιλεὺς πρὸς τὸν Σαδὼκ, ἀπόστρεψον τὴν κιβωτὸν τοῦ Θεοῦ εἰς τὴν πόλιν· ἐὰν εὕρω χάριν ἐν ὀφθαλμοῖς Κυρίου, καὶ ἐπιστρέψει με, καὶ δείξει μοι αὐτὴν καὶ τὴν εὐπρέπειαν αὐτῆς.
\VS{26}Καὶ ἐὰν εἴπῃ οὕτως, οὐκ ἠθέληκα ἐν σοί· ἰδοὺ ἐγώ εἰμι, ποιείτω μοι κατὰ τὸ ἀγαθὸν ἐν ὀφθαλμοῖς αὐτοῦ.
\par }{\PP \VS{27}Καὶ εἶπεν ὁ βασιλεὺς τῷ Σαδὼκ τῷ ἱερεῖ, ἴδετε, σὺ ἐπιστρέφεις εἰς τὴν πόλιν ἐν εἰρήνῃ, καὶ Ἀχιμάας ὁ υἱός σου, καὶ Ἰωνάθαν ὁ υἱὸς Ἀβιάθαρ οἱ δύο υἱοὶ ὑμῶν μεθʼ ὑμῶν.
\VS{28}Ἴδετε, ἐγώ εἰμι στρατεύομαι ἐν Ἀραβὼθ τῆς ἐρήμου, ἕως τοῦ ἐλθεῖν ῥῆμα παρʼ ὑμῶν τοῦ ἀπαγγεῖλαί μοι.
\VS{29}Καὶ ἀπέστρεψε Σαδὼκ καὶ Ἀβιάθαρ τὴν κιβωτὸν τοῦ Θεοῦ εἰς Ἰερουσαλὴμ, καὶ ἐκάθισεν ἐκεῖ.
\par }{\PP \VS{30}Καὶ Δαυὶδ ἀνέβαινεν ἐν τῇ ἀναβάσει τῶν ἐλαιῶν ἀναβαίνων καὶ κλαίων, καὶ τὴν κεφαλὴν ἐπικεκαλυμμένος, καὶ αὐτὸς ἐπορεύετο ἀνυπόδετος· καὶ πὰς ὁ λαὸς ὁ μετʼ αὐτοῦ ἐπεκάλυψεν ἀνὴρ τὴν κεφαλὴν αὐτοῦ· καὶ ἀνέβαινον ἀναβαίνοντες καὶ κλαίοντες.
\VS{31}Καὶ ἀνηγγέλη Δαυὶδ, λέγοντες, καὶ Ἀχιτόφελ ἐν τοῖς συστρεφομένοις μετὰ Ἀβεσσαλώμ· καὶ εἶπε Δαυὶδ, διασκέδασον δὴ τὴν βουλὴν Ἀχιτόφελ, Κύριε ὁ Θεός μου.
\par }{\PP \VS{32}Καὶ ἦν Δαυὶδ ἐρχόμενος ἕως τοῦ Ῥὼς, οὗ προσεκύνησεν ἐκεῖ τῷ Θεῷ· καὶ ἰδοὺ εἰς ἀπαντὴν αὐτῷ Χουσὶ ὁ ἀρχιεταῖρος Δαυίδ διεῤῥηχὼς τὸν χιτῶνα αὐτοῦ, καὶ γῆ ἐπὶ τῆς κεφαλῆς αὐτοῦ.
\VS{33}Καὶ εἶπεν αὐτῷ Δαυὶδ ἐὰν μὲν διαβῇς μετʼ ἐμοῦ, καὶ ἔσῃ ἐπʼ ἐμὲ εἰς βάσταγμα·
\VS{34}καὶ ἐὰν ἐπιστρέψῃς ἐπὶ τὴν πόλιν, καὶ ἐρεῖς τῷ Ἀβεσσαλὼμ, διεληλύθασιν οἱ ἀδελφοί σου, καὶ ὁ βασιλεὺς κατόπισθέ μου διελήλυθεν ὁ πατήρ σου· καὶ νῦν παῖς σού εἰμὶ, βασιλεῦ, ἔασόν με ζῆσαι· παῖς τοῦ πατρὸς σου ἤμην τότε καὶ ἀρτίως, καὶ νῦν ἐγὼ δοῦλος σός· καὶ διασκεδάσεις μοι τὴν βουλὴς Ἀχιτόφελ.
\VS{35}Καὶ ἰδοὺ ἐκεῖ μετὰ σοῦ Σαδὼκ καὶ Ἀβιάθαρ οἱ ἱερεῖς· καὶ ἔσται πὰν ῥῆμα ὁ ἐὰν ἀκούσῃς ἐξ οἴκου τοῦ βασιλέως, καὶ ἀπαγγελεῖς τῷ Σαδὼκ καὶ τῷ Ἀβιάθαρ τοῖς ἱερεῦσιν.
\VS{36}Ἰδοὺ ἐκεῖ μετʼ αὐτῶν δύο υἱοὶ αὐτῶν, Ἀχιμάας υἱὸς τῷ Σαδὼκ, καὶ Ἰωνάθαν υἱὸς τῷ Ἀβιάθαρ· καὶ ἀποστελεῖτε ἐν χειρὶ αὐτῶν πρὸς μὲ πὰν ῥῆμα ὃ ἐὰν ἀκούσητε.
\VS{37}Καὶ εἰσῆλθε Χουσὶ ὁ ἐταῖρος Δαυὶδ εἰς τὴν πόλιν, καὶ Ἀβεσσαλὼμ ἄρτι εἰσεπορεύετο εἰς Ἰερουσαλήμ.

\par }\Chap{16}{\PP \VerseOne{1}Καὶ Δαυὶδ παρῆλθε βραχύ τι ἀπὸ τῆς Ῥὼς, καὶ ἰδοὺ Σιβὰ τὸ παιδάριον Μεμφιβοσθὲ εἰς ἀπαντὴν αὐτοῦ· καὶ ζεῦγος ὄνων ἐπισεσαγμένων, καὶ ἐπʼ αὐτοῖς διακόσιοι ἄρτοι, καὶ ἑκατὸν σταφίδες, καὶ ἑκατὸν φοίνικες καὶ νέβελ οἴνου.
\VS{2}Καὶ εἶπεν ὁ βασιλεὺς πρὸς Σιβὰ, τί ταῦτά σοι; καὶ εἶπε Σιβὰ, τὰ ὑποζύγια τῇ οἰκίᾳ τοῦ βασιλέως τοῦ ἐπικαθῆσθαι, καὶ οἱ ἄρτοι καὶ οἱ φοίνικες εἰς βρῶσιν τοῖς παιδαρίοις, καὶ ὁ οἶνος πιεῖν τοῖς ἐκλελυμένοις ἐν τῇ ἐρήμῳ.
\VS{3}Καὶ εἶπεν ὁ βασιλεὺς, καὶ ποῦ ὁ υἱὸς τοῦ κυρίου σου; καὶ εἶπε Σιβὰ πρὸς τὸν βασιλέα, ἰδοὺ κάθηται ἐν Ἱερουσαλὴμ, ὅτι εἶπε, σήμερον ἐπιστρέψουσί μοι οἶκος Ἰσραὴλ τὴν βασιλείαν τοῦ πατρός μου.
\VS{4}Καὶ εἶπεν ὁ βασιλεὺς τῷ Σιβᾷ, ἰδού σοι πάντα ὅσα ἐστὶ Μεμφιβοσθέ· καὶ εἶπε Σιβὰ προσκυνήσας, εὕροιμι χάριν ἐν ὀφθαλμοῖς σου κύριέ μου βασιλεῦ.
\par }{\PP \VS{5}Καὶ ἦλθεν ὁ βασιλεὺς Δαυὶδ ἕως Βαουρίμ· καὶ ἰδοὺ ἐκεῖθεν ἀνὴρ ἐξεπορεύετο ἐκ συγγενείας οἴκου Σαοὺλ, καὶ ὄνομα αὐτῷ Σεμεῒ υἱὸς Γηρά· ἐξῆλθε ἐκπορευόμενος καὶ καταρώμενος,
\VS{6}καὶ λιθάζων ἐν λίθοις τὸν Δαυὶδ, καὶ πάντας τοὺς παῖδας τοῦ βασιλέως Δαυίδ· καὶ πᾶς ὁ λαὸς ἦν, καὶ πάντες οἱ δυνατοὶ, ἐκ δεξιῶν καὶ ἐξ εὐωνύμων τοῦ βασιλέως.
\VS{7}Καὶ οὕτως ἔλεγε Σεμεῒ ἐν τῷ καταρᾶσθαι αὐτὸν, ἔξελθε ἔξελθε ἀνὴρ αἱμάτων καὶ ἀνὴρ ὁ παράνομος.
\VS{8}Ἐπέστρεψεν ἐπὶ σὲ Κύριος πάντα τὰ αἵματα τοῦ οἴκου Σαοὺλ, ὅτι ἐβασίλευσας ἀντʼ αὐτοῦ· καὶ ἔδωκε Κύριος τὴν βασιλείαν ἐν χειρὶ Ἀβεσσαλὼμ τοῦ υἱοῦ σου· καὶ ἰδοὺ σὺ ἐν τῇ κακίᾳ σου, ὅτι ἀνὴρ αἱμάτων σύ.
\par }{\PP \VS{9}Καὶ εἶπεν Ἀβεσσὰ υἱὸς Σαρουίας πρὸς τὸν βασιλέα, ἱνατί καταρᾶται ὁ κύων ὁ τεθνηκὼς οὗτος τὸν κύριόν μου τὸν βασιλέα; διαβήσομαι δὴ καὶ ἀφελῶ τὴν κεφαλὴν αὐτοῦ.
\VS{10}Καὶ εἶπεν ὁ βασιλεὺς, τί ἐμοὶ καὶ ὑμῖν, υἱοὶ Σαρουίας; καὶ ἄφετε αὐτὸν, καὶ οὕτως καταράσθω, ὅτι Κύριος εἶπεν αὐτῷ καταρᾶσθαι τὸν Δαυίδ· καὶ τίς ἐρεῖ, ὡς τί ἐποίησας οὕτως;
\VS{11}Καὶ εἶπε Δαυὶδ πρὸς Ἀβεσσὰ καὶ πρὸς πάντας τοὺς παῖδας αὐτοῦ, ἰδοὺ ὁ υἱός μου ὁ ἐξελθὼν ἐκ τῆς κοιλίας μου ζητεῖ τὴν ψυχήν μου, καὶ προσέτι νῦν ὁ υἱὸς τοῦ Ἰεμινί· ἄφετε αὐτὸν καταρᾶσθαι, ὅτι εἶπεν αὐτῷ Κύριος.
\VS{12}Εἴπως ἴδοι Κύριος ἐν τῇ ταπεινώσει μου, καὶ ἐπιστρέψει μοι ἀγαθὰ ἀντὶ τῆς κατάρας αὐτοῦ τῇ ἡμέρᾳ ταύτῃ.
\par }{\PP \VS{13}Καὶ ἐπορεύθη Δαυὶδ καὶ πάντες οἱ ἄνδρες αὐτοῦ ἐν τῇ ὁδῷ· καὶ Σεμεῒ ἐπορεύετο ἐκ πλευρᾶς τοῦ ὄρους ἐχόμενα αὐτοῦ πορευόμενος καὶ καταρώμενος καὶ λιθάζων ἐν λίθοις ἐκ πλαγίων αὐτοῦ καὶ τῷ χοῒ πάσσων.
\VS{14}Καὶ ἦλθεν ὁ βασιλεὺς καὶ πᾶς ὁ λαὸς μετʼ αὐτοῦ ἐκλελυμένοι, καὶ ἀνέψυξαν ἐκεῖ.
\par }{\PP \VS{15}Καὶ Ἀβεσσαλὼμ καὶ πᾶς ἀνὴρ Ἰσραὴλ εἰσῆλθον εἰς Ἰερουσαλὴμ, καὶ Ἀχιτόφελ μετʼ αὐτοῦ.
\VS{16}Καὶ ἐγενήθη ἡνίκα ἦλθε Χουσὶ ὁ ἀρχιεταῖρος Δαυὶδ πρὸς Ἀβεσσαλὼμ, καὶ εἶπε Χουσὶ πρὸς Ἀβεσσαλὼμ, ζήτω ὁ βασιλεύς.
\VS{17}Καὶ εἶπεν Ἀβεσσαλὼμ πρὸς Χουσὶ, τοῦτο τὸ ἔλεός σου μετὰ τοῦ ἑταίρου σου; ἱνατί οὐκ ἀπῆλθες μετὰ τοῦ ἑταίρου σου;
\VS{18}Καὶ εἶπε Χουσὶ πρὸς Ἀβεσσαλὼμ, οὐχὶ, ἀλλὰ κατόπισθεν οὗ ἐξελέξατο Κύριος καὶ ὁ λαὸς οὗτος καὶ πᾶς ἀνὴρ Ἰσραὴλ, αὐτῷ ἔσομαι, καὶ μετὰ αὐτοῦ καθήσομαι.
\VS{19}Καὶ τὸ δεύτερον, τίνι ἐγὼ δουλεύσω; οὐχὶ ἐνώπιον τοῦ υἱοῦ αὐτοῦ; καθάπερ ἐδούλευσα ἐνώπιον τοῦ πατρός σου, οὕτως ἔσομαι ἐνώπιόν σου.
\par }{\PP \VS{20}Καὶ εἶπεν Ἀβεσσαλὼμ πρὸς Ἀχιτόφελ, φέρετε ἑαυτοῖς βουλὴν τί ποιήσωμεν.
\VS{21}Καὶ εἶπεν Ἀχιτόφελ πρὸς Ἀβεσσαλὼμ, εἴσελθε πρὸς τὰς παλλακὰς τοῦ πατρός σου, ἃς κατέλιπε φυλάσσειν τὸν οἶκον αὐτοῦ, καὶ ἀκούσεται πᾶς Ἰσραὴλ, ὅτι κατῄσχυνα, τὸν πατέρα σοῦ, καὶ ἐνισχύσουσιν αἱ χεῖρες πάντων τῶν μετὰ σοῦ.
\VS{22}Καὶ ἔπηξαν τὴν σκηνὴν τῷ Ἀβεσσαλὼμ ἐπὶ τὸ δῶμα, καὶ εἰσῆλθεν Ἀβεσσαλὼμ πρὸς τὰς παλλακὰς τοῦ πατρὸς αὐτοῦ κατʼ ὀφθαλμοὺς παντὸς Ἰσραήλ.
\VS{23}Καὶ ἡ βουλὴ Ἀχιτόφελ, ἣν ἐβουλεύσατο ἐν ταῖς ἡμέραις ταῖς πρώταις, ὃν τρόπον ἐπερωτήσῃ τις ἐν λόγῳ τοῦ Θεοῦ· οὕτως πᾶσα ἡ βουλὴ τοῦ Ἀχιτόφελ καί γε τῷ Δαυὶδ καί γε τῷ Ἀβεσσαλώμ.

\par }\Chap{17}{\PP \VerseOne{1}Καὶ εἶπεν Ἀχιτόφελ πρὸς Ἀβεσσαλὼμ, ἐπιλέξω δὴ ἐμαυτῷ δώδεκα χιλιάδας ἀνδρῶν, καὶ ἀναστήσομαι καὶ καταδιώξω ὀπίσω Δαυὶδ τὴν νύκτα.
\VS{2}Καὶ ἐπελεύσομαι ἐπʼ αὐτὸν, καὶ αὐτὸς κοπιῶν καὶ ἐκλελυμένος χερσὶ, καὶ ἐκστήσω αὐτὸν, καὶ φεύξεται πᾶς ὁ λαὸς ὁ μετʼ αὐτοῦ, καὶ πατάξω τὸν βασιλέα μονώτατον·
\VS{3}Καὶ ἐπιστρέψω πάντα τὸν λαὸν πρὸς σὲ, ὃν τρόπον ἐπιστρέφει ἡ νύμφη πρὸς τὸν ἄνδρα αὐτῆς· πλὴν ψυχὴν ἀνδρὸς ἑνὸς σὺ ζητεῖς, καὶ παντὶ τῷ λαῷ ἔσται εἰρήνη.
\VS{4}Καὶ εὐθὺς ὁ λόγος ἐν ὀφθαλμοῖς Ἀβεσσαλὼμ, καὶ ἐν ὀφθαλμοῖς πάντων τῶν πρεσβυτέρων Ἰσραήλ.
\par }{\PP \VS{5}Καὶ εἶπεν Ἀβεσσαλὼμ, καλέσατε δὴ καί γε τὸν Χουσὶ τὸν Ἀραχὶ, καὶ ἀκούσωμεν τί ἐν τῷ στόματι αὐτοῦ, καί γε αὐτοῦ.
\VS{6}Καὶ εἰσῆλθε Χουσὶ πρὸς Ἀβεσσαλὼμ, καὶ εἶπεν Ἀβεσσαλὼμ πρὸς αὐτὸν, λέγων, κατὰ τὸ ῥῆμα τοῦτο ἐλάλησεν Ἀχιτόφελ· ποιήσομεν κατὰ τὸν λόγον αὐτοῦ; εἰ δὲ μὴ, σὺ λάλησον.
\par }{\PP \VS{7}Καὶ εἶπε Χουσὶ πρὸς Ἀβεσσαλὼμ, οὐκ ἀγαθὴ αὕτη ἡ βουλὴ ἣν ἐβουλεύσατο Ἀχιτόφελ τὸ ἅπαξ τοῦτο.
\VS{8}Καὶ εἶπε Χουσὶ, σὺ οἶδας τὸν πατέρα σου καὶ τοὺς ἄνδρας αὐτοῦ, ὅτι δυνατοί εἰσι σφόδρα καὶ κατάπικροι τῇ ψυχῇ αὐτῶν, ὡς ἄρκος ἠτεκνωμένη ἐν ἀγρῷ, καὶ ὡς ὗς τραχεῖα ἐν τῇ πεδίῳ· ὁ πατήρ σου ἀνὴρ πολεμιστὴς, καὶ οὐ μὴ καταλύσῃ τὸν λαόν.
\VS{9}Ἰδοὺ γὰρ αὐτὸς νῦν κέκρυπται ἐν ἑνὶ τῶν βουνῶν ἢ ἐν ἑνὶ τῶν τόπων· καὶ ἔσται ἐν τῷ ἐπιπεσεῖν αὐτοῖς ἐν ἀρχῇ, καὶ ἀκούσῃ ἀκούων, καὶ εἴπῃ, ἐγενήθη θραῦσις ἐν τῷ λαῷ τῷ ὀπίσω Ἀβεσσαλώμ·
\VS{10}Καί γε αὐτὸς υἱὸς δυνάμεως, οὗ ἡ καρδία καθὼς ἡ καρδία τοῦ λέοντος, τηκομένη τακήσεται· ὅτι· οἶδε πᾶς Ἰσραὴλ, ὅτι δυνατὸς ὁ πατήρ σου, καὶ υἱοὶ δυνάμεως οἱ μετʼ αὐτοῦ.
\VS{11}Ὅτι οὕτως συμβουλεύων ἐγὼ συνεβούλευσα, καὶ συναγόμενος συναχθήσεται ἐπὶ σὲ πᾶς Ἰσραὴλ ἀπὸ Δὰν καὶ ἕως Βηρσαβεὲ, ὡς ἡ ἄμμος ἡ ἐπὶ τῆς θαλάσσης εἰς πλῆθος· καὶ τὸ πρόσωπόν σου πορευόμενον ἐν μέσῳ αὐτῶν.
\VS{12}Καὶ ἥξομεν πρὸς αὐτὸν εἰς ἕνα τῶν τόπων οὗ ἐὰν εὕρωμεν αὐτὸν ἐκεῖ, καὶ παρεμβαλοῦμεν ἐπʼ αὐτὸν, ὡς πίπτει δρόσος ἐπὶ τὴν γῆν, καὶ οὐχ ὑπολειψόμεθα ἐν αὐτῷ καὶ τοῖς ἀνδράσιν αὐτοῦ τοῖς μετʼ αὐτοῦ, καί γε ἕνα.
\VS{13}Καὶ ἐὰν εἰς τὴν πόλιν συναχθῇ, καὶ λήψεται πᾶς Ἰσραὴλ πρὸς τὴν πόλιν ἐκείνην σχοινία, καὶ συροῦμεν αὐτὴν ἕως εἰς τὸν χειμάῤῥουν, ὅπως μὴ καταλειφθῇ ἐκεῖ μηδὲ λίθος.
\par }{\PP \VS{14}Καὶ εἶπεν Ἀβεσσαλὼμ, καὶ πᾶς ἀνὴρ Ἰσραὴλ, ἀγαθὴ ἡ βουλὴ Χουσὶ τοῦ Ἀραχὶ ὑπὲρ τὴν βουλὴν Ἀχιτόφελ· καὶ Κύριος ἐνετείλατο διασκεδάσαι τὴν βουλὴν τοῦ Ἀχιτόφελ τὴν ἀγαθὴν, ὅπως ἂν ἐπαγάγῃ Κύριος ἐπὶ Ἀβεσσαλὼμ τὰ κακὰ πάντα.
\par }{\PP \VS{15}Καὶ εἶπε Χουσὶ ὁ τοῦ Ἀραχὶ πρὸς Σαδὼκ καὶ Ἀβιάθαρ τοὺς ἱερεῖς, οὕτως καὶ οὕτως συνεβούλευσεν Ἀχιτόφελ τῷ Ἀβεσσαλὼμ καὶ τοῖς πρεσβυτέροις Ἰσραήλ· καὶ οὕτως καὶ οὕτως συνεβούλευσα ἐγώ.
\VS{16}Καὶ νῦν ἀποστείλατε ταχὺ καὶ ἀναγγείλατε τῷ Δαυὶδ, λέγοντες, μὴ αὐλισθῇς τὴν νύκτα ἐν Ἀραβὼθ τῆς ἐρήμου, καί γε διαβαίνων σπεῦσον, μήποτε καταπείσῃ τὸν βασιλέα, καὶ πάντα τὸν λαὸν τὸν μετʼ αὐτοῦ.
\par }{\PP \VS{17}Καὶ Ἰωνάθαν καὶ Ἀχιμάας εἱστήκεισαν ἐν τῇ πηγῇ Ῥωγὴλ, καὶ ἐπορεύθη ἡ παιδίσκη, καὶ ἀνήγγειλεν αὐτοῖς, καὶ αὐτοὶ πορεύονται καὶ ἀναγγέλλουσι τῷ βασιλεῖ Δαυίδ· ὅτι οὐκ ἠδύναντο ὀφθῆναι τοῦ εἰσελθεῖν τἰς τὴν πόλιν.
\VS{18}Καὶ εἶδεν αὐτοὺς παιδάριον, καὶ ἀνήγγειλε τῷ Ἀβεσσαλώμ· καὶ ἐπορεύθησαν οἱ δύο ταχέως, καὶ εἰσῆλθαν εἰς οἰκίαν ἀνδρὸς ἐν Βαουρίμ· καὶ αὐτῷ λάκκος ἐν τῇ αὐλῇ, καὶ κατέβησαν ἐκεῖ.
\VS{19}Καὶ ἔλαβεν ἡ γυνὴ, καὶ διεπέτασε τὸ ἐπικάλυμμα ἐπὶ πρόσωπον τοῦ λάκκου, καὶ ἔψυξεν ἐπʼ αὐτῷ ἀραφὼθ, καὶ οὐκ ἐγνώσθη ῥῆμα.
\VS{20}Καὶ ἦλθον οἱ παῖδες Ἀβεσσαλὼμ πρὸς τὴν γυναῖκα εἰς τὴν οἰκίαν, καὶ εἶπαν, ποῦ Ἀχιμάας καὶ Ἰωνάθαν; καὶ εἶπεν αὐτοῖς ἡ γυνὴ, παρῆλθαν μικρὸν το ὕδατος· καὶ ἐζήτησαν, καὶ οὐχ εὗραν, καὶ ἀνέστρεψαν εἰς Ἰερουσαλήμ.
\VS{21}Ἐγένετο δὲ μετὰ τὸ ἀπελθεῖν αὐτοὺς, καὶ ἀνέβησαν ἐκ τοῦ λάκκου, καὶ ἐπορεύθησαν· καὶ ἀπήγγειλαν τῷ βασιλεῖ Δαυὶδ, καὶ εἶπαν πρὸς Δαυὶδ, ἀνάστητε καὶ διάβητε ταχέως τὸ ὕδωρ, ὅτι οὕτως ἐβουλεύσατο περὶ ὑμῶν Ἀχιτόφελ.
\par }{\PP \VS{22}Καὶ ἀνέστη Δαυὶδ καὶ πᾶς ὁ λαὸς ὁ μετʼ αὐτοῦ, καὶ διέβησαν τὸν Ἰορδάνην ἕως τοῦ φωτὸς τοῦ πρωῒ, ἕως ἑνὸς οὐκ ἔλαθεν ὃς οὐ διῆλθε τὸν Ἰορδάνην.
\par }{\PP \VS{23}Καὶ Ἀχιτόφελ εἶδεν ὅτι οὐκ ἐγενήθη ἡ βουλὴ αὐτοῦ, καὶ ἐπέσαξε τὴν ὄνον αὐτοῦ, καὶ ἀνέστη καὶ ἀπῆλθεν εἰς τὸν οἶκον αὐτοῦ εἰς τὴν πόλιν αὐτοῦ· καὶ ἐνετείλατο τῷ οἴκῳ αὐτοῦ, καὶ ἀπήγξατο καὶ ἀπέθανεν καὶ ἐτάφη ἐν τῷ τάφῳ τοῦ πατρὸς αὐτοῦ.
\par }{\PP \VS{24}Καὶ Δαυεὶδ διῆλθεν εἰς Μαναΐμ καὶ Ἀβεσσαλὼμ διέβη τὸν Ἰορδάνην αὐτὸς καὶ πᾶς ἀνὴρ Ἰσραὴλ μετʼ αὐτοῦ.
\VS{25}Καὶ τὸν Ἀμεσσαῒ κατέστησεν Ἀβεσσαλὼμ ἀντὶ Ἰωὰβ ἐπὶ τῆς δυνάμεως. Καὶ Ἀμεσσαῒ υἱὸς ἀνδρὸς, καὶ ὄνομα αὐτῷ Ἰεθὲρ ὁ Ἰεζραηλίτης· οὗτος εἰσῆλθε πρὸς Ἀβιγαίαν θυγατέρα Νάας ἀδελφὴν Σαρουίας μητρὸς Ἰωάβ.
\VS{26}Καὶ παρενέβαλε πᾶς Ἰσραὴλ καὶ Ἀβεσσαλὼμ εἰς τὴν γῆν Γαλαάδ.
\par }{\PP \VS{27}Καὶ ἐγένετο ἡνίκα ἦλθε Δαυὶδ εἰς Μαναῒμ, καὶ Οὐεσβὶ υἱὸς Νάας ἐκ Ῥαββὰθ υἱῶν Ἀμμὼν, καὶ Μαχὶρ υἱὸς Ἀμιὴλ ἐκ Λωδαβὰρ, καὶ Βερζελλὶ ὁ Γαλααδίτης ἐκ Ῥωγελλὶμ,
\VS{28}ἤνεγκαν δέκα κοίτας ἀμφιτάπους, καὶ λέβητας δέκα, καὶ σκεύη κεράμου, καὶ πυροὺς, καὶ κριθὰς, καὶ ἄλευρον, καὶ ἄλφιτον, καὶ κύαμον, καὶ φακὸν,
\VS{29}καὶ μέλι, καὶ βούτυρον καὶ πρόβατα καὶ σαφὼθ βοῶν· καὶ προσήνεγκαν τῷ Δαυὶδ, καὶ τῷ λαῷ τῷ μετʼ αὐτοῦ φαγεῖν· ὅτι εἶπεν, ὁ λαὸς πεινῶν καὶ ἐκλελυμένος καὶ διψῶν ἐν τῇ ἐρήμῳ.

\par }\Chap{18}{\PP \VerseOne{1}Καὶ ἐπεσκέψατο Δαυὶδ τὸν λαὸν τὸν μετʼ αὐτοῦ, καὶ κατέστησεν ἐπʼ αὐτῶν χιλιάρχους καὶ ἑκατοντάρχους.
\VS{2}Καὶ ἀπέστειλε Δαυὶδ τὸν λαὸν τὸ τρίτον ἐν χειρὶ Ἰωὰβ, καὶ τὸ τρίτον ἐν χειρὶ Ἀβεσσὰ υἱοῦ Σαρουίας ἀδελφοῦ Ἰωὰβ, καὶ τὸ τρίτον ἐν χειρὶ Ἐθὶ τοῦ Γεθαίου· καὶ εἶπε Δαυὶδ πρὸς τὸν λαὸν, ἐξελθὼν ἐξελεύσομαι καί γε ἐγὼ μεθʼ ὑμῶν.
\VS{3}Καὶ εἶπαν, οὐκ ἐξελεύσῃ, ὅτι ἐὰν φυγῇ φύγωμεν, οὐ θήσουσιν ἐφʼ ἡμᾶς καρδίαν· καὶ ἐὰν ἀποθάνωμεν τὸ ἥμισυ ἡμῶν, οὐ θήσουσιν ἐφʼ ἡμᾶς καρδίαν, ὅτι σὺ ὡς ἡμεῖς δέκα χιλιάδες· καὶ νῦν ἀγαθὸν, ὅτι ἔσῃ ἡμῖν ἐν τῇ πόλει βοήθεια τοῦ βοηθεῖν.
\VS{4}Καὶ εἶπε πρὸς αὐτοὺς ὁ βασιλεὺς, ὃ ἐὰν ἀρέσῃ ἐν ὀφθαλμοῖς ὑμῶν, ποιήσω· καὶ ἔστη ὁ βασιλεὺς ἀνὰ χεῖρα τῆς πύλης· καὶ πᾶς ὁ λαὸς ἐξεπορεύετο εἰς ἑκατοντάδας καὶ εἰς χιλιάδας.
\par }{\PP \VS{5}Καὶ ἐνετείλατο ὁ βασιλεὺς τῷ Ἰωὰβ καὶ τῷ Ἀβεσσὰ καὶ τῷ Ἐθὶ, λέγων, φείσασθέ μου τοῦ παιδαρίου τοῦ Ἀβεσσαλώμ· καὶ πᾶς ὁ λαὸς ἤκουσεν ἐντελλομένου τοῦ βασιλέως πᾶσι τοῖς ἄρχουσιν ὑπὲρ Ἀβεσσαλώμ.
\par }{\PP \VS{6}Καὶ ἐξῆλθε πᾶς ὁ λαὸς εἰς τὸν δρυμὸν ἐξεναντίας Ἰσραήλ· καὶ ἐγένετο ὁ πόλεμος ἐν τῷ δρυμῷ Ἐφραίμ.
\VS{7}Καὶ ἔπταισεν ἐκεῖ ὁ λαὸς Ἰσραὴλ ἐνώπιον τῶν παίδων Δαυὶδ, καὶ ἐγένετο ἡ θραῦσις μεγάλη ἐν τῇ ἡμέρᾳ ἐκείνῃ, εἴκοσι χιλιάδες ἀνδρῶν.
\VS{8}Καὶ ἐγένετο ἐκεῖ ὁ πόλεμος διεσπαρμένος ἐπὶ πρόσωπον πάσης τῆς γῆς· καὶ ἐπλεόνασεν ὁ δρυμὸς τοῦ καταφαγεῖν ἐκ τοῦ λαοῦ, ὑπὲρ οὓς κατέφαγεν ἐν τῷ λαῷ ἡ μάχαιρα τῇ ἡμέρᾳ ἐκαίνῃ.
\VS{9}Καὶ συνήντησεν Ἀβεσσαλὼμ ἐνώπιον τῶν παίδων Δαυίδ· καὶ Ἀβεσσαλὼμ ἦν ἐπιβεβηκὼς ἐπὶ τοῦ ἡμιόνου αὐτοῦ, καὶ εἰσῆλθεν ὁ ἡμίονος ὑπὸ τὸ δάσος τῆς δρυὸς τῆς μεγάλης, καὶ περιεπλάκη ἡ κεφαλὴ αὐτοῦ ἐν τῇ δρυῒ, καὶ ἐκρεμάσθη ἀναμέσον τοῦ οὐρανοῦ καὶ ἀναμέσον τῆς γῆς, καὶ ὁ ἡμίονος ὑποκάτω αὐτοῦ παρῆλθε.
\par }{\PP \VS{10}Καὶ εἶδεν ἀνὴρ εἷς, καὶ ἀνήγγειλε τῷ Ἰωὰβ, καὶ εἶπεν, ἰδοὺ ἑώρακα τὸν Ἀβεσσαλὼμ κρεμάμενον ἐν τῇ δρυΐ.
\VS{11}Καὶ εἶπεν Ἰωὰβ τῷ ἀνδρὶ τῇ ἀναγγέλλοντι αὐτῷ, καὶ ἰδοὺ ἑώρακας· τί ὅτι οὐκ ἐπάταξας αὐτὸν ἐκεῖ εἰς τὴν γῆν; καὶ ἐγὼ ἂν δεδώκειν σοι δέκα ἀργυρίου καὶ παραζώνην μίαν.
\VS{12}Εἶπε δὲ ὁ ἀνὴρ πρὸς Ἰωὰβ, καὶ ἐγώ εἰμι ἵστημι ἐπὶ τὰς χεῖράς μου χιλίους σίκλους ἀργυρίου, οὐ μὴ ἐπιβάλω τὴν χεῖρά μου ἐπὶ τὸν υἱὸν τοῦ βασιλέως· ὅτι ἐν τοῖς ὠσὶν ἡμῶν ἐνετείλατο ὁ βασιλεύς σοι καὶ τῷ Ἀβεσσὰ καὶ τῷ Ἐθὶ, λέγων, φυλάξατέ μοι τὸ παιδάριον τὸν Ἀβεσσαλὼμ,
\VS{13}μὴ ποιῆσαι ἐν τῇ ψυχῇ αὐτοῦ ἄδικον· καὶ πᾶς ὁ λόγος οὐ λήσεται ἀπὸ τοῦ βασιλέως, καὶ σὺ στήσῃ ἐξεναντίας.
\VS{14}Καὶ εἶπεν Ἰωὰβ, τοῦτο ἐγὼ ἄρξομαι, οὐχ οὕτως μενῶ ἐνώπιόν σου· καὶ ἔλαβεν Ἰωὰβ τρία βέλη ἐν τῇ χειρὶ αὐτοῦ, καὶ ἐνέπηξεν αὐτὰ ἐν τῇ καρδίᾳ Ἀβεσσαλὼμ, ἔτι αὐτοῦ ζῶντος ἐν τῇ καρδίᾳ τῆς δρυός.
\VS{15}Καὶ ἐκύκλωσαν δέκα παιδάρια αἴροντα τὰ σκεύη Ἰωὰβ, καὶ ἐπάταξαν τὸν Ἀβεσσαλὼμ, καὶ ἐθανάτωσαν αὐτόν.
\par }{\PP \VS{16}Καὶ ἐσάλπισεν Ἰωὰβ ἐν κερατίνῃ, καὶ ἀπέστρεψεν ὁ λαὸς τοῦ μὴ διώκειν ὀπίσω Ἰσραὴλ, ὅτι ἐφείδετο Ἰωὰβ τοῦ λαοῦ.
\VS{17}Καὶ ἔλαβε τὸν Ἀβεσσαλὼμ, καὶ ἔῤῥιψεν αὐτὸν εἰς χάσμα μέγα ἐν τῷ δρυμῷ εἰς τὸν βόθυνον τὸν μέγαν, καὶ ἐστήλωσεν ἐπʼ αὐτὸν σωρὸν λίθων μέγαν σφόδρα· καὶ πᾶς Ἰσραὴλ ἔφυγεν ἀνὴρ εἰς τὸ σκήνωμα αὐτοῦ.
\VS{18}Καὶ Ἀβεσσαλὼμ ἔτι ζῶν ἔλαβε καὶ ἔστησεν ἑαυτῷ τὴν στήλην ἐν ᾗ ἐλήφθη, καὶ ἐστήλωσεν αὐτὴν λαβεῖν τὴν στήλην τὴν ἐν τῇ βασιλέως, ὅτι εἶπεν, ὅτι οὐκ ἔστιν αὐτῷ υἱὸς ἕνεκν τοῦ ἀναμνῆσαι τὸ ὄνομα αὐτοῦ· καὶ ἐκάλεσε τὴν στήλην, Χεὶρ Ἀβεσσαλὼμ, ἕως τῆς ἡμέρας ταύτης.
\par }{\PP \VS{19}Καὶ Ἀχιμάας υἱὸς Σαδὼκ εἶπε, δράμω δὴ καὶ εὐαγγελιῶ τῷ βασιλεῖ, ὅτι ἔκρινε Κύριος ἐκ χειρὸς τῶν ἐχθρῶν αὐτοῦ.
\VS{20}Καὶ εἶπεν αὐτῷ Ἰωὰβ, οὐκ ἀνὴρ εὐαγγελίας σὺ ἐν τῇ ἡμέρᾳ ταύτῃ, καὶ εὐαγγελιῇ ἐν ἡμέρᾳ ἄλλῃ· ἐν δὲ τῇ ἡμέρᾳ ταύτῃ οὐκ εὐαγγελιῇ, οὗ εἵνεκεν ὁ υἱὸς τοῦ βασιλέως ἀπέθανε.
\VS{21}Καὶ εἶπεν Ἰωὰβ τῷ Χουσὶ, Βαδίσας ἀνάγγειλον τῷ βασιλεῖ ὅσα εἶδες· καὶ προσεκύνησε Χουσεὶ τῷ Ἰωὰβ, καὶ ἐξῆλθε.
\VS{22}Καὶ προσέθετο ἕτι Ἀχιμάας υἱὸς Σαδὼκ, καὶ εἶπε πρὸς Ἰωὰβ, καὶ ἔστω, ὅτι δράμω καί γε ἐγὼ ὀπίσω τοῦ Χουσί· καὶ εἶπεν Ἰωὰβ, ἱνατί σὺ τοῦτο τρέχεις, υἱέ μου; δεῦρο, οὐκ ἔστι σοι εὐαγγέλια εἰς ὠφέλειαν πορευομένῳ.
\VS{23}Καὶ εἶπε, τί γὰρ ἐὰν δράμω; καὶ εἶπεν αὐτῷ Ἰωὰβ, δράμε· καὶ ἔδραμεν Ἀχιμάας τὴν ὁδὸν τὴν τοῦ Κεχὰρ, καὶ ὑπερέβη τὸν Χουσί.
\par }{\PP \VS{24}Καὶ Δαυὶδ ἐκάθητο ἀναμέσον τῶν δύο πυλῶν· καὶ ἐπορεύθη ὁ σκοπὸς εἰς τὸ δῶμα τῆς πύλης πρὸς τὸ τεῖχος, καὶ ἐπῇρε τοὺς ὀφθαλμοὺς αὐτοῦ, καὶ εἶδε. καὶ ἰδοὺ ἀνὴρ τρέχων μόνος ἐνώπιον αὐτοῦ.
\VS{25}Καὶ ἀνεβόησεν ὁ σκοπὸς καὶ ἀπήγγειλε τῷ βασιλεῖ· καὶ εἶπεν ὁ βασιλεὺς, εἰ μόνος ἐστὶν, εὐαγγέλια ἐν τῷ στόματι αὐτοῦ· καὶ ἐπορεύετο πορευόμενος καὶ ἐγγίζων.
\VS{26}Καὶ εἶδεν ὁ σκοπὸς ἄνδρα ἕτερον τρέχοντα· καὶ ἐβόησεν ὁ σκοπὸς πρὸς τῇ πύλῃ, καὶ εἶπε, καὶ ἰδοὺ ἀνὴρ ἕτερος τρέχων μόνος· καὶ εἶπεν ὁ βασιλεὺς, καί γε οὗτος εὐαγγελιζόμενος.
\VS{27}Καὶ εἶπεν ὁ σκοπός, ἐγὼ ὁρῶ τὸν δρόμον τοῦ πρώτου ὡς δρόμον Ἀχιμάας υἱοῦ Σαδώκ· καὶ εἶπεν ὁ βασιλεὺς, ἀνὴρ ἀγαθὸς οὗτος, καί γε εἰς εὐαγγελίαν ἀγαθὴν ἐλεύσεται.
\par }{\PP \VS{28}Καὶ ἐβόησεν Ἀχιμάας, καὶ εἶπε πρὸς τὸν βασιλέα, εἰρήνη· καὶ προσεκύνησε τῷ βασιλεῖ ἐπὶ πρόσωπον αὐτοῦ ἐπὶ τὴν γῆν, καὶ εἶπεν, εὐλογητὸς Κύριος ὁ Θεός σου, ὃς ἀπέκλεισε τοὺς ἄνδρας τοὺς ἐπαραμένους τὴν χεῖρα αὐτῶν ἐν τῷ κυρίῳ μου τῷ βασιλεῖ.
\VS{29}Καὶ εἶπεν ὁ βασιλεὺς, εἰρήνη τῷ παιδαρίῳ τῷ Ἀβεσσαλώμ; καὶ εἶπεν Ἀχιμάας, εἶδον τὸ πλῆθος τὸ μέγα τοῦ ἀποστεῖλαι τὸν δοῦλον τοῦ βασιλέως Ἰωὰβ καὶ τὸν δοῦλόν σου, καὶ οὐκ ἔγνων τί ἐκεῖ.
\VS{30}Καὶ εἶπεν ὁ βασιλεὺς, ἐπίστρεψον, στηλώθητι ὧδε· καὶ ἐπεστράφη, καὶ ἔστη.
\par }{\PP \VS{31}Καὶ ἰδοὺ ὁ Χουσὶ παρεγένετο, καὶ εἶπε τῷ βασιλεῖ, εὐαγγελισθήτω ὁ κύριός μου ὁ βασιλεὺς, ὅτι ἔκρινέ σοι Κύριος σήμερον ἐκ χειρὸς πάντων τῶν ἐπεγειρομένων ἐπὶ σέ.
\VS{32}Καὶ εἶπεν ὁ βασιλεὺς πρὸς τὸν Χουσὶ, εἰ εἰρήνη τῷ παιδαρίῳ τῷ Ἀβεσσαλώμ; καὶ εἶπεν ὁ Χουσὶ, γένοιντο ὡς τὸ παιδάριον οἱ ἐχθροὶ τοῦ κυρίου μου τοῦ βασιλέως, καὶ πάντες ὅσοι ἐπανέστησαν ἐπʼ αὐτὸν εἰς κακά.

\Chap{19}\VerseOne{1}Καὶ ἐταράχθη ὁ βασιλεὺς, καὶ ἀνέβη εἰς τὸ ὑπερῷον τῆς πύλης, καὶ ἔκλαυσε· καὶ οὕτως εἶπεν ἐν τῷ πορεύεσθαι αὐτὸν, υἱέ μου Ἀβεσσαλὼμ, υἱέ μου, υἱέ μου Ἀβεσαλώμ· τίς δῴη τὸν θάνατόν μου ἀντὶ σοῦ; ἐγὼ ἀντὶ σοῦ Ἀβεσσαλὼμ, υἱέ μου, υἱέ μου.
\par }{\PP \VS{2}Καὶ ἀνηγγέλη τῷ Ἰωὰβ, λέγοντες, ἰδοῦ ὁ βασιλεὺς κλαίει καὶ πενθεῖ ἐπὶ Ἀβεσσαλώμ.
\VS{3}Καὶ ἐγένετο ἡ σωτηρία ἐν τῇ ἡμέρᾳ ἐκείνῃ εἰς πένθος παντὶ τῷ λαῷ, ὅτι ἤκουσεν ὁ λαὸς ἐν τῇ ἡμέρᾳ ἐκείνῃ, λέγων, ὅτι λυπεῖται ὁ βασιλεὺς ἐπὶ τῷ υἱῷ αὐτοῦ.
\VS{4}Καὶ διεκλέπτετο ὁ λαὸς ἐν τῇ ἡμέρᾳ ἐκείνῃ τοῦ εἰσελθεῖν εἰς τὴν πόλιν, καθὼς διακλέπτεται ὁ λαὸς οἱ αἰσχυνόμενοι ἐν τῷ αὐτοὺς φεύγειν ἐν τῷ πολέμῳ.
\VS{5}Καὶ ὁ βασιλεὺς ἔκρυψε τὸ πρόσωπον αὐτοῦ· καὶ ἔκραξεν ὁ βασιλεὺς φωνῇ μεγάλῃ, λέγων, υἱέ μου Ἀβεσσαλὼμ, Ἀβεσσαλὼμ υἱὲ μου.
\par }{\PP \VS{6}Καὶ εἰσῆλθεν Ἰωὰβ πρὸς τὸν βασιλέα εἰς τὸν οἶκον, καὶ εἶπε, κατῄσχυνας σήμερον τὰ πρόσωπα πάντων τῶν δούλων σου τῶν ἐξαιρουμένων σε σήμερον, καὶ τὴν ψυχὴν τῶν υἱῶν σου, καὶ τῶν θυγατέρων σου, καὶ τὴν ψυχὴν τῶν γυναικῶν σου, καὶ τῶν παλλακῶν σου,
\VS{7}τοῦ ἀγαπᾷν τοὺς μισοῦντάς σε, καὶ μισεῖν τοὺς ἀγαπῶντάς σε· καὶ ἀνήγγειλας σήμερον, ὅτι οὐκ εἰσὶν οἱ ἄρχοντές σου, οὐδὲ παῖδες· ὅτι ἔγνωκα σήμερον, ὅτι εἰ Ἀβεσσαλὼμ ἔζη, πάντες ἡμεῖς σήμερον νεκροί, ὅτι τότε τὸ εὐθὲς ἦν ἐν ὀφθαλμοῖς σου.
\VS{8}Καὶ νῦν ἀναστὰς ἔξελθε καὶ λάλησον εἰς τὴν καρδίαν τῶν δούλων σου, ὅτι ἐν Κυρίῳ ὤμοσα, ὅτι εἰ μὴ ἐκπορεύσῃ σήμερον, εἰ αὐλισθήσεται ἀνὴρ μετὰ σοῦ τὴν νύκτα ταύτην· καὶ ἐπίγνωθι σεαυτῷ, καὶ κακόν σοι τοῦτο ὑπὲρ πᾶν τὸ κακὸν τὸ ἐπελθόν σοι ἐκ νεότητός σου ἕως τοῦ νῦν.
\VS{9}Καὶ ἀνέστη ὁ βασιλεὺς καὶ ἐκάθισεν ἐν τῇ πύλῃ· καὶ πᾶς ὁ λαὸς ἀνήγγειλαν, λέγοντες, ἰδοὺ ὁ βασιλεὺς κάθηται ἐν τῇ πύλῃ· καὶ εἰσῆλθε πᾶς ὁ λαὸς κατὰ πρόσωπον τοῦ βασιλέως ἐπὶ τὴν πύλην· καὶ Ἰσραὴλ ἔφυγεν ἀνὴρ εἰς τὰ σκηνώματα αὐτοῦ.
\par }{\PP \VS{10}Καὶ ἦν πᾶς ὁ λαὸς κρινόμενος ἐν πάσαις φυλαῖς Ἰσραὴλ, λέγοντες, ὁ βασιλεὺς Δαυὶδ ἐῤῥύσατο ἡμᾶς ἀπὸ πάντων τῶν ἐχθρῶν ἡμῶν, καὶ αὐτὸς ἐξείλετο ἡμᾶς ἐκ χειρὸς ἀλλοφύλων· καὶ νῦν πέφευγεν ἀπὸ τῆς γῆς, καὶ ἀπὸ τῆς βασιλείας αὐτοῦ, καὶ ἀπὸ Ἀβεσσαλώμ.
\VS{11}Καὶ Ἀβεσσαλὼμ, ὃν ἐχρίσαμεν ἐφʼ ἡμῶν, ἀπέθανεν ἐν τῷ πολέμῳ· καὶ νῦν ἱνατί ὑμεῖς κωφεύετε τοῦ ἐπιστρέψαι τὸν βασιλέα; καὶ τὸ ῥῆμα παντὸς Ἰσραὴλ ἦλθε πρὸς τὸν βασιλέα.
\par }{\PP \VS{12}Καὶ ὁ βασιλεὺς Δαυὶδ ἀπέστειλε πρὸς Σαδὼκ καὶ πρὸς Ἀβιάθαρ τοῦς ἱερεῖς, λέγων, λαλήσατε πρὸς τοὺς πρεσβυτέρους Ἰούδα, λέγοντες, ἱνατί γίνεσθε ἔσχατοι τοῦ ἐπιστρέψαι τὸν βασιλέα εἰς τὸν οἶκον αὐτοῦ; καὶ λόγος παντὸς Ἰσραὴλ ἦλθε πρὸς τὸν βασιλέα εἰς τὸν οἶκον αὐτοῦ.
\VS{13}Ἀδελφοί μου ὑμεῖς, ὀστᾶ μου καὶ σάρκες μου ὑμεῖς· ἱνατί γίνεσθε ἔσχατοι τοῦ ἐπιστρέψαι τὸν βασιλέα εἰς τὸν οἶκον αὐτοῦ;
\VS{14}καὶ τῷ Ἀμεσσαῒ; ἐρεῖτε, οὐχὶ ὀστοῦν μου καὶ σάρξ μου σύ; καὶ νῦν τάδε ποιήσαι μοι ὁ Θεὸς, καὶ τάδε προσθείη, εἰ μὴ ἄρχων δυνάμεως ἔσῃ ἐνώπιον ἐμοῦ πάσας τὰς ἡμέρας ἀντὶ Ἰωάβ.
\VS{15}Καὶ ἔκλινε τὴν καρδίαν παντὸς ἀνδρὸς Ἰούδα ὡς ἀνδρὸς ἑνός· καὶ ἀπέστειλαν πρὸς τὸν βασιλεα, λέγοντες, ἐπιστράφηθι σὺ καὶ πάντες οἱ δοῦλοί σου.
\VS{16}Καὶ ἐπέστρεψεν ὁ βασιλεὺς, καὶ ἦλθεν ἕως τοῦ Ἰορδάνου· καὶ ἄνδρες Ἰούδα ἦλθαν εἰς Γάλγαλα τοῦ πορεύεσθαι εἰς ἀπαντὴν τοῦ βασιλέως, διαβιβάσαι τὸν βασιλέα τὸν Ἰορδάνην.
\par }{\PP \VS{17}Καὶ ἐτάχυνε Σεμεῒ υἱὸς Γηρὰ υἱοῦ τοῦ Ἰεμινὶ ἐκ Βαουρὶμ, καὶ κατέβη μετὰ ἀνδρὸς Ἰούδα εἰς ἀπαντὴν τοῦ βασιλέως Δαυίδ,
\VS{18}καὶ χίλιοι ἄνδρες μετʼ αὐτοῦ ἐκ τοῦ Βενιαμὶν, καὶ Σιβὰ τὸ παιδάριον τοῦ οἴκου Σαοὺλ, καὶ πεντεκαίδεκα υἱοὶ αὐτοῦ μετʼ αὐτοῦ, καὶ εἴκοσι δοῦλοι αὐτοῦ μετʼ αὐτοῦ· καὶ κατεύθυναν τὸν Ἰορδάνην ἔμπροσθεν τοῦ βασιλέως,
\VS{19}καὶ ἐλειτούργησαν τὴν λειτουργίαν τοῦ διαβιβάσαι τὸν βασιλέα· καὶ διέβη ἡ διάβασις τοῦ ἐξεγεῖραι τὸν οἶκον τοῦ βασιλέως, καὶ τοῦ ποιῆσαι τὸ εὐθὲς ἐν ὀφθαλμοῖς αὐτοῦ. Καὶ Σεμεῒ υἱὸς Γηρὰ ἔπεσεν ἐπὶ πρόσωπον αὐτοῦ ἐνώπιον τοῦ βασιλέως, διαβαίνοντος αὐτοῦ τὸν Ἰορδάνην,
\VS{20}καὶ εἶπε πρὸς τὸν βασιλέα, μὴ δὴ λογισάσθω ὁ κύριός μου ἀνομίαν, καὶ μὴ μνησθῇς ὅσα ἠδίκησεν ὁ παῖς σου ἐν τῇ ἡμέρᾳ ᾗ ὁ κύριός μου ἐξεπορεύετο ἐξ Ἰερουσαλὴμ, τοῦ θέσθαι τὸν βασιλέα εἰς τὴν καρδίαν αὐτοῦ.
\VS{21}Ὅτι ἔγνω ὁ δοῦλός σου ὅτι ἐγὼ ἥμαρτον, καὶ ἰδοὺ ἐγὼ ἦλθον σήμερον πρότερος παντὸς Ἰσραὴλ καὶ οἴκου Ἰωσὴφ, τοῦ καταβῆναί με εἰς ἀπαντὴν τοῦ κυρίου μου τοῦ βασιλέως.
\par }{\PP \VS{22}Καὶ ἀπεκρίθη Ἀβεσσαὲ υἱὸς Σαρουίας, καὶ εἶπε, μὴ ἀντὶ τούτου οὐ θανατωθήσεται Σεμεῒ, ὅτι κατηράσατο τὸν χριστὸν Κυρίου;
\VS{23}Καὶ εἶπε Δαυίδ, τί ἐμοὶ καὶ ὑμῖν, υἱοὶ Σαρουίας, ὅτι γίνεσθέ μοι σήμερον εἰς ἐπίβουλον; σήμερον οὐ θανατωθήσεταί τις ἀνὴρ ἐξ Ἰσραήλ· ὅτι οὐκ οἴδα εἰ σήμερον βασιλεύω ἐγὼ ἐπὶ τὸν Ἰσραήλ.
\VS{24}Καὶ εἶπεν ὁ βασιλεὺς πρὸς Σεμεῒ, οὐ μὴ ἀποθάνῃς· καὶ ὤμοσεν αὐτῷ ὁ βασιλεύς.
\par }{\PP \VS{25}Καὶ Μεμφιβοσθὲ υἱὸς υἱοῦ Σαοὺλ κατέβη εἰς ἀπαντὴν τοῦ βασιλέως, καὶ οὐκ ἐθεράπευσε τοὺς πόδας αὐτοῦ, οὐδὲ ὠνυχίσατο, οὐδὲ ἐποίησε τὸν μύστακα αὐτοῦ, καὶ τὰ ἱμάτια αὐτοῦ οὐκ ἀπέπλυνεν, ἀπὸ τῆς ἡμέρας ἧς ἀπῆλθεν ὁ βασιλεὺς, ἕως τῆς ἡμέρας ἧς αὐτὸς παρεγένετο ἐν εἰρήνῃ.
\par }{\PP \VS{26}Καὶ ἐγένετο ὅτε εἰσῆλθεν εἰς Ἱερουσαλὴμ εἰς ἀπάντησιν τοῦ βασιλέως, καὶ εἶπεν αὐτῷ ὁ βασιλεὺς, τί ὅτι οὐκ ἐπορεύθης μετʼ ἐμοῦ, Μεμφιβοσθέ;
\VS{27}Καὶ εἶπε πρὸς αὐτὸν Μεμφιβοσθὲ, κύριέ μου βασιλεῦ, ὁ δοῦλός μου παρελογίσατό με, ὅτι εἶπεν ὁ παῖς σου αὐτῷ, ἐπίσαξόν μοι τὴν ὄνον, καὶ ἐπιβῶ ἐπʼ αὐτὴν, καὶ πορεύσομαι μετὰ τοῦ βασιλέως, ὅτι χωλὸς ὁ δοῦλός σου.
\VS{28}Καὶ μεθώδευσεν ἐν τῷ δούλῳ σου πρὸς τὸν κύριόν μου τὸν βασιλέα· καὶ ὁ κύριός μου ὁ βασιλεὺς ὡς ἄγγελος τοῦ Θεοῦ, καὶ ποίησον τὸ ἀγαθὸν ἐν ὀφθαλμοῖς σου.
\VS{29}Ὅτι οὐκ ἦν πᾶς ὁ οἶκος τοῦ πατρός μου, ἀλλʼ ἢ ὅτι ἄνδρες θανάτου τῷ κυρίῳ μου τῷ βασιλεῖ, καὶ ἔθηκας τὸν δοῦλόν σου ἐν τοῖς ἐσθίουσι τὴν τράπεζάν σου· καὶ τί ἐστι μοὶ ἔτι δικαίωμα, καὶ τοῦ κεκραγέναι με ἔτι πρὸς τὸν βασιλέα;
\par }{\PP \VS{30}Καὶ εἶπεν αὐτῷ ὁ βασιλεὺς, ἱνατί λαλεῖς ἔτι τοὺς λόγους σου; εἶπον, σὺ καὶ Σιβὰ διελεῖσθε τὸν ἀγρόν.
\VS{31}Καὶ εἶπε Μεμφιβοσθὲ πρὸς τὸν βασιλέα, καί γε τὰ πάντα λαβέτω, μετὰ τὸ παραγενέσθαι τὸν κύριόν μου τὸν βασιλέα ἐν εἰρήνῃ εἰς τὸν οἶκον αὐτοῦ.
\par }{\PP \VS{32}Καὶ Βερζελλὶ ὁ Γαλααδίτης κατέβη ἐκ Ῥωγελλὶμ, καὶ διέβη μετὰ τοῦ βασιλέως τὸν Ἰορδάνην ἐκπέμψαι αὐτὸν τὸν Ἰορδάνην.
\VS{33}Καὶ Βερζελλὶ ἀνὴρ πρεσβύτερος σφόδρα, υἱὸς ὀγδοήκοντα ἐτῶν, καὶ αὐτὸς διέθρεψε τὸν βασιλέα ἐν τῷ οἰκεῖν αὐτὸν ἐν Μαναῒμ, ὅτι ἀνὴρ μέγας ἦν σφόδρα.
\VS{34}Καὶ εἶπεν ὁ βασιλεὺς πρὸς Βερζελλὶ, σὺ διαβήσῃ μετʼ ἐμοῦ, καὶ διαθρέψω τὸ γῆράς σου μετʼ ἐμοῦ ἐν Ἱερουσαλήμ.
\VS{35}Καὶ εἶπε Βερζελλὶ πρὸς τὸν βασιλέα, πόσαι ἡμέραι ἐτῶν ζωῆς μου, ὅτι ἀναβήσομαι μετὰ τοῦ βασιλέως εἰς Ἱερουσαλήμ;
\VS{36}Υἱὸς ὀγδοήκοντα ἐτῶν ἐγώ εἰμι σήμερον· εἰ μὴν γνώσομαι ἀναμέσον ἀγαθοῦ καὶ κακοῦ; εἰ γεύσεται ὁ δοῦλός σου ἔτι ὃ φάγομαι ἢ πίομαι; ἢ ἀκούσομαι ἔτι φωνὴν ᾀδόντων καὶ ᾀδουσῶν; καὶ ἱνατί ἔσται ἔτι ὁ δοῦλός σου εἰς φορτίον ἐπὶ τὸν κύριόν μου τὸν βασιλέα;
\VS{37}Ὡς βραχὺ διαβήσεται ὁ δοῦλός σου τὸν Ἰορδάνην μετὰ τοῦ βασιλέως· καὶ ἱνατί ἀνταποδίδωσί μοι ὁ βασιλεὺς τὴν ἀνταπόδοσιν ταύτην;
\VS{38}Καθισάτω δὴ ὁ δοῦλός σου, καὶ ἀποθανοῦμαι ἐν τῇ πόλει μου παρὰ τῷ τάφῳ τοῦ πατρός μου καὶ τῆς μητρός μου· καὶ ἰδοὺ ὁ δοῦλός σου Χαμαὰμ διαβήσεται μετὰ τοῦ κυρίου μου τοῦ βασιλέως· καὶ ποίησον αὐτῷ τὸ ἀγαθὸν ἐν ὀφθαλμοῖς σου.
\VS{39}Καὶ εἶπεν ὁ βασιλεὺς, μετʼ ἐμοῦ διαβήτω Χαμαὰμ, κᾀγὼ ποιήσω αὐτῷ τὸ ἀγαθὸν ἐν ὀφθαλμοῖς μου, καὶ πάντα ὅσα ἂν ἐκλέξῃ ἐπʼ ἐμοὶ, ποιήσω σοι.
\par }{\PP \VS{40}Καὶ διέβη πᾶς ὁ λαὸς τὸν Ἰορδάνην, καὶ ὁ βασιλεὺς διέβη, καὶ κατεφίλησεν ὁ βασιλεὺς τὸν Βερζελλὶ, καὶ εὐλόγησεν αὐτὸν, καὶ ἐπέστρεψεν εἰς τὸν τόπον αὐτοῦ.
\VS{41}Καὶ διέβη ὁ βασιλεὺς εἰς Γάλγαλα, καὶ Χαμαὰμ διέβη μετʼ αὐτοῦ· καὶ πᾶς ὁ λαὸς Ἰούδα διαβαίνοντες μετὰ τοῦ βασιλέως, καί γε τὸ ἥμισυ τοῦ λαοῦ Ἰσραήλ.
\par }{\PP \VS{42}Καὶ ἰδοὺ πᾶς ἀνὴρ Ἰσραὴλ παρεγένοντο πρὸς τὸν βασιλέα, καὶ εἶπε πρὸς τὸν βασιλέα, τί ὅτι ἔκλεψάν σε οἱ ἀδελφοὶ ἡμῶν ἀνὴρ Ἰούδα, καὶ διεβίβασαν τὸν βασιλέα καὶ τὸν οἶκον αὐτοῦ τὸν Ἰορδάνην, καὶ πάντες ἄνδρες Δαυὶδ μετʼ αὐτοῦ;
\VS{43}Καὶ ἀπεκρίθη πᾶς ἀνὴρ Ἰούδα πρὸς ἄνδρα Ἰσραὴλ, καὶ εἶπαν, διότι ἐγγίζει πρὸς μὲ ὁ βασιλεύς· καὶ ἱνατί οὕτως ἐθυμώθης περὶ τοῦ λόγου τούτου; μὴ βρώσει ἐφάγαμεν ἐκ τοῦ βασιλέως, ἢ δόμα ἔδωκεν, ἢ ἄρσιν ᾖρεν ἡμῖν;
\VS{44}Καὶ ἀπεκρίθη ἀνὴρ Ἰσραὴλ τῷ ἀνδρὶ Ἰούδα, καὶ εἶπε, δέκα χεῖρές μοι ἐν τῷ βασιλεῖ, καὶ πρωτότοκος ἐγὼ ἢ σὺ, καί γε ἐν τῷ Δαυίδ εἰμι ὑπὲρ σέ· καὶ ἱνατί τοῦτο ὕβρισάς με, καὶ οὐκ ἐλογίσθη ὁ λόγος μου πρῶτός μοι τοῦ Ἰούδα ἐπιστρέψαι τὸν βασιλέα ἐμοί; καὶ ἐσκληρύνθη ὁ λόγος ἀνδρὸς Ἰούδα ὑπὲρ τὸν λόγον ἀνδρὸς Ἰσραήλ.

\par }\Chap{20}{\PP \VerseOne{1}Καὶ ἐκεῖ ἐπικαλούμενος υἱὸς παράνομος, καὶ ὄνομα αὐτῷ Σαβεὲ, υἱὸς Βοχορί ἀνὴρ ὁ Ἰεμινὶ, καὶ ἐσάλπισε τῇ κερατίνῃ, καὶ εἶπεν, οὐκ ἔστιν ἡμῖν μερὶς ἐν Δαυὶδ, οὐδὲ κληρονομία ἡμῖν ἐν τῷ υἱῷ Ἰεσσαί· ἀνὴρ εἰς τὰ σκηνώματά σου, Ἰσραήλ.
\VS{2}Καὶ ἀνέβη πᾶς ἀνὴρ Ἰσραὴλ ἀπὸ ὄπισθεν Δαυὶδ ὀπίσω Σαβεὲ υἱοῦ Βοχορί· καὶ ἀνὴρ Ἰούδα ἐκολλήθη τῷ βασιλεῖ αὐτῶν, ἀπὸ τοῦ Ἰορδάνου καὶ ἕως Ἰερουσαλήμ.
\par }{\PP \VS{3}Καὶ εἰσῆλθε Δαυὶδ εἰς τὸν οἶκον αὐτοῦ εἰς Ἱερουσαλήμ· καὶ ἔλαβεν ὁ βασιλεὺς τὰς δέκα γυναῖκας τὰς παλλακὰς αὐτοῦ, ἃς ἀφῆκε φυλάσσειν τὸν οἶκον, καὶ ἔδωκεν αὐτὰς ἐν οἴκῳ φυλακῆς, καὶ διέθρεψεν αὐτὰς, καὶ πρὸς αὐτὰς οὐκ εἰσῆλθε· καὶ ἦσαν συνεχόμεναι ἕως ἡμέρας θανάτου αὐτῶν χῆραι ζῶσαι.
\par }{\PP \VS{4}Καὶ εἶπεν ὁ βασιλεὺς πρὸς Ἀμεσσαῒ, βόησόν μοι τὸν ἄνδρα Ἰούδα τρεῖς ἡμέρας, σὺ δὲ αὐτοῦ στῆθι.
\VS{5}Καὶ ἐπορεύθη Ἀμεσσαῒ τοῦ βοῆσαι τὸν Ἰούδαν, καὶ ἐχρόνισεν ἀπὸ τοῦ καιροῦ οὗ ἐτάξατο αὐτῷ Δαυίδ.
\VS{6}Καὶ εἶπε Δαυὶδ πρὸς Ἀμεσσαῒ, νῦν κακοποιήσει ἡμᾶς Σαβεὲ υἱὸς Βοχορὶ ὑπὲρ Ἀβεσσαλώμ· καὶ νῦν σὺ λάβε μετὰ σεαυτοῦ τοὺς παῖδας τοῦ κυρίου σου, καὶ καταδίωξον ὀπίσω αὐτοῦ, μή ποτε ἑαυτῷ εὕρῃ πόλεις ὀχυρὰς, καὶ σκιάσει τοὺς ὀφθαλμοὺς ἡμῶν.
\VS{7}Καὶ ἐξῆλθον ὀπίσω αὐτοὺ Ἀβεσσαῒ καὶ οἱ ἄνδρες Ἰωὰβ, καὶ ὁ Χερεθὶ, καὶ ὁ Φελεθὶ, καὶ πάντες οἱ δυνατοὶ, καὶ ἐξῆλθον ἐξ Ἱερουσαλὴμ διῶξαι ὀπίσω Σαβεὲ υἱοῦ Βοχορί.
\par }{\PP \VS{8}Καὶ αὐτοὶ παρὰ τῷ λίθῳ τῷ μεγάλῳ τῷ ἐν Γαβαών· καὶ Ἀμεσσαῒ εἰσῆλθεν ἔμπροσθεν αὐτῶν· καὶ Ἰωὰβ περιεζωσμένος μανδύαν τὸ ἔνδυμα αὐτοῦ, καὶ ἐπʼ αὐτῷ ἐζωσμένος μάχαιραν ἐζευγμένην ἐπὶ τῆς ὀσφύος αὐτοῦ ἐν κολεῷ αὐτῆς· καὶ ἡ μάχαιρα ἐξῆλθε· καὶ αὐτὴ ἐξῆλθε καὶ ἔπεσε.
\par }{\PP \VS{9}Καὶ εἶπεν Ἰωὰβ τῷ Ἀμεσσαῒ, εἰ ὑγιαίνεις σὺ, ἀδελφέ; καὶ ἐκράτησεν ἡ χεὶρ ἡ δεξιὰ Ἰωὰβ τοῦ πώγωνος Ἀμεσσαῒ τοῦ καταφιλῆσαι αὐτόν.
\VS{10}Καὶ Ἀμεσσαῒ οὐκ ἐφυλάξατο τὴν μάχαιραν τὴν ἐν τῇ χειρὶ Ἰωάβ· καὶ ἔπαισεν αὐτὸν ἐν αὐτῇ Ἰωὰβ εἰς τὴν ψόαν, καὶ ἐξεχύθη ἡ κοιλία αὐτοῦ εἰς τὴν γῆν, καὶ οὐκ ἐδευτέρωσεν αὐτῷ, καὶ ἀπέθανε· καὶ Ἰωὰβ καὶ Ἀβεσσαῒ ὁ ἀδελφὸς αὐτοῦ ἐδίωξεν ὀπίσω Σαβεὲ υἱοῦ Βοχορί.
\VS{11}Καὶ ἀνὴρ ἔστη ἐπʼ αὐτὸν τῶν παιδαρίων Ἰωὰβ, καὶ εἶπε, τίς ὁ βουλόμενος Ἰωὰβ, καὶ τίς τοῦ Δαυὶδ, ὀπίσω Ἰωάβ;
\VS{12}Καὶ Ἀμεσσαῒ πεφυρμένος ἐν τῷ αἵματι ἐν μέσὼ τῆς τρίβου· καὶ εἶδεν ἀνὴρ, ὅτι εἱστήκει πᾶς ὁ λαὸς, καὶ ἀπέστρεψε τὸν Ἀμεσσαῒ ἐκ τῆς τρίβου εἰς ἀγρόν· καὶ ἐπέῤῥιψεν ἐπʼ αὐτὸν ἱμάτιον, καθʼ ὅτι εἶδε πάντα τὸν ἐρχόμενον ἐπʼ αὐτὸν ἑστηκότα.
\VS{13}Ἡνίκα δὲ ἔφθασεν ἐκ τῆς τρίβου, παρῆλθε πᾶς ἀνὴρ Ἰσραὴλ ὀπίσω Ἰωὰβ τοῦ διῶξαι ὀπίσω Σαβεὲ υἱοῦ Βοχορί.
\par }{\PP \VS{14}Καὶ διῆλθεν ἐν πάσαις φυλαῖς Ἰσραὴλ εἰς Ἀβὲλ καὶ εἰς Βαθμαχά· καὶ πάντες ἐν Χαῤῥὶ καὶ ἐξεκκλησιάσθησαν, καὶ ἦλθον κατόπισθεν αὐτοῦ.
\VS{15}Καὶ παρεγενήθησαν καὶ ἐπολιόρκουν ἐπʼ αὐτὸν ἐν Ἀβὲλ καὶ Φερμαχά· καὶ ἐξέχεαν πρόσχωμα πρὸς τὴν πόλιν, καὶ ἔστη ἐν τῷ προτειχίσματι· καὶ πᾶς ὁ λαὸς ὁ μετὰ Ἰωὰβ ἐνοοῦσαν καταβαλεῖν τὸ τεῖχος.
\VS{16}Καὶ ἐβόησε γυνὴ σοφὴ ἐκ τοῦ τείχους, καὶ εἶπεν, ἀκούσατε ἀκούσατε, εἴπατε δὴ πρὸς Ἰωὰβ, ἔγγισον ἕως ὧδε, καὶ λαλήσω πρὸς αὐτόν.
\par }{\PP \VS{17}Καὶ προσήγγισε πρὸς αὐτήν· καὶ εἶπεν ἡ γυνὴ, εἰ σὺ εἶ Ἰωάβ; ὁ δὲ εἶπεν, ἐγώ. εἶπε δὲ αὐτῷ, ἄκουσον τοὺς λόγους τῆς δούλης σου· καὶ εἶπεν Ἰωὰβ, ἀκούω ἐγώ εἰμι.
\VS{18}Καὶ εἶπε λέγουσα, λόγον ἐλάλησαν ἐν πρώτοις, λέγοντες, ἠρωτημένος ἠρωτήθη ἐν τῇ Ἀβὲλ καὶ ἐν Δὰν εἰ ἐξέλιπον ἃ ἔθεντο οἱ πιστοὶ τοῦ Ἰσραήλ· ἐρωτῶντες ἐπερωτήσουσιν ἐν Ἀβὲλ, καὶ οὕτως εἰ ἐξέλιπον.
\VS{19}Ἐγώ εἰμι εἰρηνικὰ τῶν στηριγμάτων Ἰσραήλ· σὺ δὲ ζητεῖς θανατῶσαι πόλιν καὶ μητρόπολιν ἐν Ἰσραήλ· ἱνατί καταποντίζεις κληρονομίαν Κυρίου;
\VS{20}Καὶ ἀπεκρίθη Ἰωὰβ, καὶ εἶπεν, ἵλεώς μοι ἵλεώς μοι, εἰ καταποντιῶ καὶ εἰ φθερῶ.
\VS{21}Οὐχ οὕτως ὁ λόγος, ὅτι ἀνὴρ ἐξ ὄρους Ἐφραὶμ, Σαβεὲ υἱὸς Βοχορὶ ὄνομα αὐτοῦ, καὶ ἐπῇρε τὴν χεῖρα αὐτοῦ ἐπὶ τὸν βασιλέα Δαυίδ. δότε αὐτόν μοι μόνον, καὶ ἀπελεύσομαι ἀπάνωθεν τῆς πόλεως. Καὶ εἶπεν ἡ γυνὴ πρὸς Ἰωὰβ, ἰδοὺ ἡ κεφαλὴ αὐτοῦ ῥιφήσεται πρὸς σὲ διὰ τοῦ τείχους.
\par }{\PP \VS{22}Καὶ εἰσῆλθεν ἡ γυνὴ πρὸς πάντα τὸν λαὸν, καὶ ἐλάλησε πρὸς πᾶσαν τὴν πόλιν ἐν τῇ σοφίᾳ αὐτῆς· καὶ ἀφεῖλε τὴν κεφαλὴν Σαβεὲ υἱοῦ Βοχορί· καὶ ἀφεῖλε καὶ ἔβαλε πρὸς Ἰωάβ· καὶ ἐσάλπισεν ἐν κερατίνῃ, καὶ διεσπάρησαν ἀπὸ τῆς πόλεως ἀπʼ αὐτοῦ ἀνὴρ εἰς τὰ σκηνώματα αὐτοῦ· καὶ Ἰωὰβ ἀπέστρεψεν εἰς Ἰερουσαλὴμ πρὸς τὸν βασιλέα.
\par }{\PP \VS{23}Καὶ ὁ Ἰωὰβ πρὸς πάσῃ τῇ δυνάμει Ἰσραήλ· καὶ Βαναίας υἱὸς Ἰωδαὲ ἐπὶ τοῦ Χερεθὶ, καὶ ἐπὶ τοῦ Φελεθί·
\VS{24}Καὶ Ἀδωνιρὰμ ἐπὶ τοῦ φόρου· καὶ Ἰωσαφὰθ υἱὸς Ἀχιλοὺθ ἀναμιμνήσκων.
\VS{25}Καὶ Σουσὰ γραμματεύς· καὶ Σαδὼκ καὶ Ἀβιάθαρ ἱερεῖς·
\VS{26}Καί γε Ἰρὰς ὁ Ἰαρὶν ἦν ἱερεὺς τῷ Δαυίδ.

\par }\Chap{21}{\PP \VerseOne{1}Καὶ ἐγένετο λιμὸς ἐν ταῖς ἡμέραις Δαυὶδ τρία ἔτη, ἐνιαυτὸς ὁ ἐχόμενος ἐνιαυτοῦ· καὶ ἐζήτησε Δαυὶδ τὸ πρόσωπον Κυρίου· καὶ εἶπε Κύριος, ἐπὶ Σαοὺλ καὶ ἐπὶ τὸν οἶκον αὐτοῦ ἀδικία ἐν θανάτῳ αἱμάτων αὐτοῦ, περὶ οὗ ἐθανάτωσε τοὺς Γαβαωνίτας.
\VS{2}Καὶ ἐκάλεσεν ὁ βασιλεὺς Δαυὶδ τοὺς Γαβαωνίτας, καὶ εἶπε πρὸς αὐτούς· καὶ οἱ Γαβαωνίται οὐχ υἱοὶ Ἰσραήλ εἰσιν, ὅτι ἀλλʼ ἢ ἐκ τοῦ ἐλλείμματος τοῦ Ἀμοῤῥαίου, καὶ οἱ υἱοὶ Ἰσραὴλ ὤμοσαν αὐτοῖς· καὶ ἐζήτησε Σαοὺλ πατάξαι αὐτοὺς ἐν τῷ ζηλῶσαι αὐτὸν τοὺς υἱοὺς Ἰσραὴλ καὶ Ἰούδα.
\par }{\PP \VS{3}Καὶ εἶπε Δαυὶδ πρὸς τοὺς Γαβαωνίτας, τί ποιήσω ὑμῖν, καὶ ἐν τίνι ἐξιλάσωμαι, καὶ εὐλογήσετε τὴν κληρονομίαν Κυρίου;
\VS{4}Καὶ εἶπαν αὐτῷ οἱ Γαβαωνίται, οὐκ ἔστιν ἡμῖν ἀργύριον ἢ χρυσίον μετὰ Σαοὺλ καὶ μετὰ τοῦ οἴκου αὐτοῦ, καὶ οὐκ ἔστιν ἡμῖν ἀνὴρ θανατῶσαι ἐν Ἰσραήλ. Καὶ εἶπε, τί ὑμεῖς λέγετε, καὶ ποιήσω ὑμῖν;
\VS{5}καὶ εἶπαν πρὸς τὸν βασιλέα, ὁ ἀνὴρ ὃς συνετέλεσεν ἐφʼ ἡμᾶς καὶ ἐδίωξεν ἡμᾶς, ὃς παρελογίσατο ἐξολοθρεῦσαι ἡμᾶς, ἀφανίσωμεν αὐτὸν, τοῦ μὴ ἑστάναι αὐτὸν ἐν παντὶ ὁρίῳ Ἰσραήλ.
\VS{6}Δότω ἡμῖν ἑπτὰ ἄνδρας ἐκ τῶν υἱῶν αὐτοῦ, καὶ ἐξηλιάσωμεν αὐτοὺς τῷ Κυρίῳ ἐν τῷ Γαβαὼν Σαοὺλ ἐκλεκτοὺς Κυρίου· καὶ εἶπεν ὁ βασιλεὺς, ἐγὼ δώσω.
\par }{\PP \VS{7}Καὶ ἐφείσατο ὁ βασιλεὺς ἐπὶ Μεμφιβοσθὲ υἱὸν Ἰωνάθαν υἱοῦ Σαοὺλ διὰ τὸν ὅρκον Κυρίου τὸν ἀναμέσον αὐτῶν, καὶ ἀναμέσον Δαυὶδ, καὶ ἀναμέσον Ἰωνάθαν υἱοῦ Σαούλ.
\par }{\PP \VS{8}Καὶ ἔλαβεν ὁ βασιλεὺς τοὺς δύο υἱοὺς Ῥεσφὰ θυγατρὸς Ἁϊᾶ, οὓς ἔτεκε τῷ Σαοὺλ, τὸν Ἐρμωνοῒ καὶ τὸν Μεμφιβοσθὲ, καὶ τοὺς πέντε υἱοὺς τῆς Μιχὸλ θυγατρὸς Σαοὺλ, οὓς ἔτεκε τῷ Ἐσδριὴλ υἱῷ Βερζελλὶ τῷ Μωουλαθί·
\VS{9}Καὶ ἔδωκεν αὐτοὺς ἐν χειρὶ τῶν Γαβαωνιτῶν, καὶ ἐξηλίασαν αὐτοὺς ἐν τῷ ὄρει ἔναντι Κυρίου· καὶ ἔπεσαν οἱ ἑπτὰ αὐτοὶ ἐπὶ τὸ αὐτό· καὶ αὐτοὶ δὲ ἐθανατώθησαν ἐν ἡμέραις θερισμοῦ ἐν πρώτοις, ἐν ἀρχῇ θερισμοῦ κριθῶν.
\VS{10}Καὶ ἔλαβε Ῥεσφὰ θυγάτηρ Ἁϊᾶ τὸν σάκκον, καὶ ἔπηξεν αὐτῇ πρὸς τὴν πέτραν ἐν ἀρχῇ θερισμοῦ κριθῶν, ἕως ἔσταξεν ἐπʼ αὐτοὺς ὕδωρ ἐκ τοῦ οὐρανοῦ· καὶ οὐκ ἔδωκε τὰ πετεινὰ τοῦ οὐρανοῦ καταπαῦσαι ἐπʼ αὐτοὺς ἡμέρας, καὶ τὰ θηρία τοῦ ἀγροῦ νυκτός.
\par }{\PP \VS{11}Καὶ ἀπηγγέλη τῷ Δαυὶδ ὅσα ἐποίησε Ῥεσφὰ θυγάτηρ Ἁϊᾶ παλλακὴ Σαούλ· καὶ ἐξελύθησαν, καὶ κατέλαβεν αὐτοὺς Δὰν υἱὸς Ἰωὰ ἐκ τῶν ἀπογόνων τῶν γιγάντων.
\VS{12}Καὶ ἐπορεύθη Δαυὶδ καὶ ἔλαβε τὰ ὀστᾶ Σαοὺλ, καὶ τὰ ὀστᾶ Ἰωνάθαν τοῦ υἱοῦ αὐτοῦ, παρὰ τῶν ἀνδρῶν υἱῶν Ἰαβὶς Γαλαὰδ, οἳ ἔκλεψαν αὐτοὺς ἐκ τῆς πλατείας Βαιθσὰν, ὅτι ἔστησαν αὐτοὺς ἐκεῖ οἱ ἀλλόφυλοι ἐν τῇ ἡμέρᾳ ᾗ ἐπάταξαν οἱ ἀλλόφυλοι τὸν Σαοὺλ ἐν Γελβουέ.
\VS{13}Καὶ ἁνένεγκεν ἐκεῖθεν τὰ ὀστᾶ Σαοὺλ καὶ τὰ ὀστᾶ Ἰωνάθαν τοῦ υἱοῦ αὐτοῦ, καὶ συνήγαγε τὰ ὀστᾶ τῶν ἐξηλιασμένων.
\VS{14}Καὶ ἔθαψαν τὰ ὀστᾶ Σαοὺλ καὶ τὰ ὀστᾶ Ἰωνάθαν τοῦ υἱοῦ αὐτοῦ καὶ τὰ ὀστᾶ τῶν ἡλιασθένων ἐν γῇ Βενιαμὶν ἐν τῇ πλευρᾷ ἐν τῷ τάφῳ Κὶς τοῦ πατρὸς αὐτοῦ· καὶ ἐποίησαν πάντα ὅσα ἐνετείλατο ὁ βασιλεύς· καὶ ἐπήκουσεν ὁ Θεὸς τῇ γῇ μετὰ ταῦτα.
\par }{\PP \VS{15}Καὶ ἐγενήθη ἔτι πόλεμος τοῖς ἀλλοφύλοις μετὰ Ἰσραήλ· καὶ κατέβη Δαυὶδ καὶ οἱ παῖδες αὐτοῦ μετʼ αὐτοῦ, καὶ ἐπολέμησαν μετὰ τῶν ἀλλοφύλων· καὶ ἐπορεύθη Δαυίδ.
\VS{16}Καὶ Ἰεσβὶ, ὃς ἦν ἐν τοῖς ἐκγόνοις τοῦ Ῥαφά, καὶ ὁ σταθμὸς τοῦ δόρατος αὐτοῦ, τριακοσίων σίκλων ὁλκῇ χαλκοῦ, καὶ αὐτὸς περιεζωσμένος κορύνην, καὶ διενοεῖτο τοῦ πατάξαι τὸν Δαυίδ.
\VS{17}Καὶ ἐβοήθησεν αὐτῷ Ἀβεσσὰ υἱὸς Σαρουίας, καὶ ἐπάταξε τὸν ἀλλόφυλον καὶ ἐθανάτωσεν αὐτόν· τότε ὤμοσαν οἱ ἄνδρες Δαυὶδ, λέγοντες, οὐκ ἐξελεύσῃ ἔτι μεθʼ ἡμῶν εἰς πόλεμον, καὶ οὐ μὴ σβέσῃς τὸν λύχνον Ἰσραήλ.
\par }{\PP \VS{18}Καὶ ἐγενήθη μετὰ ταῦτα ἔτι πόλεμος ἐν Γὲθ μετὰ τῶν ἀλλοφύλων· τότε ἐπάταξε Σεβοχὰ ὁ Ἀστατωθὶ τὸν Σὲφ ἐν τοῖς ἐγγόνοις τοῦ Ῥαφά.
\par }{\PP \VS{19}Καὶ ἐγένετο ὁ πόλεμος ἐν Ῥὸμ μετὰ τῶν ἀλλοφύλων· καὶ ἐπάταξεν Ἐλεανὰν υἱὸς Ἀριωργὶμ ὁ Βαιθλεεμίτης τὸν Γολιὰθ τὸν Γεθαῖον· καὶ τὸ ξύλον τοῦ δόρατος αὐτοῦ ὡς ἀντίον ὑφαινόντων.
\VS{20}Καὶ ἐγένετο ἔτι πόλεμος ἐν Γέθ· καὶ ἦν ἀνὴρ μαδὼν, καὶ οἱ δάκτυλοι τῶν χειρῶν αὐτοῦ, καὶ οἱ δάκτυλοι τῶν ποδῶν αὐτοῦ ἓξ καὶ ἓξ, εἰκοσιτέσσαρες ἀριθμῷ· καί γε αὐτὸς ἐτέχθη τῷ Ῥαφᾴ.
\VS{21}Καὶ ὠνείδισε τὸν Ἰσραὴλ, καὶ ἐπάταξεν αὐτὸν Ἰωνάθαν υἱὸς Σεμεῒ ἀδελφοῦ Δαυίδ.
\par }{\PP \VS{22}Οἱ τέσσαρες οὗτοι ἐτέχθησαν ἀπόγονοι τῶν γιγάντων ἐν Γὲθ τῷ Ῥαφὰ οἶκος, καὶ ἔπεσαν ἐν χειρὶ Δαυὶδ, καὶ ἐν χειρὶ τῶν δούλων αὐτοῦ.

\par }\Chap{22}{\PP \VerseOne{1}Καὶ ἐλάλησε Δαυὶδ τῷ Κυρίῳ τοὺς λόγους τῆς ᾠδῆς ταύτης, ἐν ᾗ ἡμέρᾳ ἐξείλετο αὐτὸν Κύριος ἐκ χειρὸς πάντων τῶν ἐχθρῶν αὐτοῦ, καὶ ἐκ χειρὸς Σαούλ.
\VS{2}Καὶ εἶπεν ᾠδή·
\par }{\PP Κύριε πέτρα μου, καὶ ὀχύρωμά μου, καὶ ἐξαιρούμενός με ἐμοί,
\VS{3}ὁ Θεός μου, φύλαξ μου ἔσται μοι, πεποιθὼς ἔσομαι ἐπʼ αὐτῷ· ὑπερασπιστής μου, καὶ κέρας σωτηρίας μου, ἀντιλήπτωρ μου, καὶ καταφυγή μου σωτηρίας μου, ἐξ ἀδίκου σώσεις με.
\par }{\PP \VS{4}Αἰνετὸν ἐπικαλέσομαι Κύριον, καὶ ἐκ τῶν ἐχθρῶν μου σωθήσομαι.
\VS{5}Ὅτι περιέσχον με συντριμμοὶ θανάτου, χείμαῤῥοι ἀνομίας ἐθάμβησάν με.
\VS{6}Ὠδῖνες θανάτου ἐκύκλωσάν με, προέφθασάν με σκληρότητες θανάτου.
\VS{7}Ἐν τῷ θλίβεσθαί με ἐπικαλέσομαι τὸν Κύριον, καὶ πρὸς τὸν Θεόν μου βοήσομαι, καὶ ἐπακούσεται ἐκ ναοῦ αὐτοῦ φωνῆς μου, καὶ ἡ κραυγή μου ἐν τοῖς ὠσὶν αὐτοῦ.
\par }{\PP \VS{8}Καὶ ἐταράχθη καὶ ἐσείσθη ἡ γῆ, καὶ τὰ θεμέλια τοῦ οὐρανοῦ συνεταράχθησαν καὶ ἐσπαράχθησαν, ὃτι ἐθυμώθη Κύριος αὐτοῖς.
\VS{9}Ἀνέβη καπνὸς ἐν τῇ ὀργῇ αὐτοῦ, καὶ πῦρ ἐκ στόματος αὐτοῦ κατέδεται· ἄνθρακες ἐξεκαύθησαν ἀπʼ αὐτοῦ.
\VS{10}Καὶ ἔκλινεν οὐρανοὺς καὶ κατέβη, καὶ γνόφος ὑποκάτω τῶν ποδῶν αὐτοῦ.
\VS{11}Καὶ ἐπεκάθισεν ἐπὶ τῷ χερουβὶν καὶ ἐπετάσθη, καὶ ὤφθη ἐπὶ πτερύγων ἀνέμου.
\VS{12}Καὶ ἔθετο σκότος ἀποκρυφὴν αὐτοῦ· κύκλῳ αὐτοῦ ἡ σκηνὴ αὐτοῦ σκότος ὑδάτων, ἐπάχυνεν ἐν νεφέλαις ἀέρος.
\VS{13}Ἀπὸ τοῦ φέγγους ἐναντίον αὐτοῦ ἐξεκαύθησαν ἄνθρακες πυρός.
\VS{14}Ἐβρόντησεν ἐξ οὐρανοῦ Κύριος, καὶ ὁ ὕψιστος ἔδωκε φωνὴν αὐτοῦ.
\VS{15}Καὶ ἀπέστειλε βέλη, καὶ ἐσκόρπισεν αὐτούς· καὶ ἢστραψεν ἀστραπὴν, καὶ ἐξέστησεν αὐτούς.
\VS{16}Καὶ ὤφθησαν ἀφέσεις θαλάσσης, καὶ ἀπεκαλύφθη θεμέλια τῆς οἰκουμένης ἐν τῇ ἐπιτιμήσει Κυρίου, ἀπὸ πνοῆς πνεύματος θυμοῦ αὐτοῦ.
\VS{17}Ἀπέστειλεν ἐξ ὕξ ὕψους καὶ ἔλαβέ με, εἵλκυσέ με ἐξ ὑδάτων πολλῶν.
\VS{18}Ἐῤῥύσατό με ἐξ ἐχθρῶν μου ἰσχύος, ἐκ τῶν μισούντων με, ὅτι ἐκραταιώθησαν ὑπὲρ ἐμέ.
\par }{\PP \VS{19}Προέφθασάν με ἡμέραι θλίψεώς μου· καὶ ἐγένετο Κύριος ἐπιστήριγμά μου;
\VS{20}καὶ ἐξήγαγέ με εἰς πλατυσμὸν, καὶ ἐξείλετό με, ὅτι ηὐδόκησες ἐν ἐμοί.
\VS{21}Καὶ ἀνταπέδωκέ μοι Κύριος κατὰ τὴν δικαιοσύνην μου, καὶ κατὰ τὴν καθαριότητα τῶν χειρῶν μου ἀνταπέδωκέ μοι.
\VS{22}Ὅτι ἐφύλαξα ὁδοὺς Κυρίου, καὶ οὐκ ἠσέβησα ἀπὸ τοῦ Θεοῦ μου.
\VS{23}Ὅτι πάντα τὰ κρίματα αὐτοῦ κατεναντίον μου καὶ τὰ δικαιώματα αὐτοῦ, οὐκ ἀπέστην ἀπʼ αὐτῶν.
\VS{24}Καὶ ἔσομαι ἄμωμος αὐτῷ, καὶ προφυλάξομαι ἀπὸ τῆς ἀνομίας μου.
\VS{25}Καὶ ἀποδώσει μοι Κύριος κατὰ τὴν δικαιοσύνην μου, καὶ κατὰ τὴν καθαριότητα τῶν χειρῶν μου ἐνώπιον τῶν ὀφθαλμῶν αὐτοῦ.
\par }{\PP \VS{26}Μετὰ ὁσίου ὁσιωθήσῃ, καὶ μετὰ ἀνδρὸς τελείου τελειωθήσῃ·
\VS{27}Καὶ μετὰ ἐκλεκτοῦ ἐκλεκτὸς ἔσῃ, καὶ μετὰ στρεβλοῦ στρεβλωθήσῃ.
\VS{28}Καὶ τὸν λαὸν τὸν πτωχὸν σώσεις, καὶ ὀφθαλμοὺς ἐπὶ μετεώρων ταπεινώσεις.
\VS{29}Ὅτι σὺ ὁ λύχνος μου Κύριε, καὶ Κύριος ἐκλάμψει μοι τὸ σκότος μου.
\VS{30}Ὅτι ἐν σοὶ δραμοῦμαι μονόζωνος, καὶ ἐν τῷ Θεῷ μου ὑπερβήσομαι τεῖχος.
\par }{\PP \VS{31}Ὁ ἰσχυρὸς ἄμωμος ἡ ὁδὸς αὐτοῦ· τὸ ῥῆμα Κυρίου κραταιὸν πεπυρωμένον· ὑπερασπιστής ἐστι πᾶσι τοῖς πεποιθόσιν ἐπʼ αὐτόν.
\VS{32}Τίς ἰσχυρὸς πλὴν Κυρίου; καὶ τίς κτίστης ἔσται πλὴν τοῦ Θεοῦ ἡμῶν;
\VS{33}Ὁ ἰσχυρὸς ὁ κραταιῶν με δυνάμει, καὶ ἐξετίναξεν ἄμωμον τὴν ὁδόν μου.
\VS{34}Τιθεὶς τοὺς πόδας μου ὡς ἐλάφων, καὶ ἐπὶ τὰ ὕψη ἱστῶν με.
\VS{35}Διδάσκων χεῖράς μου εἰς πόλεμον, καὶ κατάξας τόξον χαλκοῦν ἐν βραχίονί μου.
\VS{36}Καὶ ἔδωκάς μοι ὑπερασπισμὸν σωτηρίας μου, καὶ ἡ ὑπακοή σου ἐπλήθυνέ με εἰς πλατυσμὸν εἰς τὰ διαβήματά μου ὑποκάτω μου,
\VS{37}καὶ οὐκ ἐσαλεύθησαν τὰ σκέλη μου.
\par }{\PP \VS{38}Διώξω ἐχθρούς μου, καὶ ἀφανιῶ αὐτοὺς, καὶ οὐκ ἀναστρέψω ἕως ἂν συντελέσω αὐτούς.
\VS{39}Καὶ θλάσω αὐτοὺς καὶ οὐκ ἀναστήσονται, καὶ πεσοῦνται ὑπὸ τοὺς πόδας μου.
\VS{40}Καὶ ἐνισχύσεις με δυνάμει εἰς πόλεμον, κάμψεις τοὺς ἐπιστανομένους μοι ὑποκάτω μου.
\VS{41}Καὶ τοὺς ἐχθρούς μου ἔδωκάς μοι νῶτον, τοὺς μισοῦντάς με, καὶ ἐθανάτωσας αὐτούς.
\VS{42}Βοήσονται, καὶ οὐκ ἔστι βοηθὸς, πρὸς Κύριον, καὶ οὐκ ἐπήκουσεν αὐτῶν.
\VS{43}Καὶ ἐλέανα αὐτοὺς ὡς χοῦν γῆς, ὡς πηλὸν ἐξόδων ἐλέπτυνα αὐτούς.
\VS{44}Καὶ ῥύσῃ με ἐκ μάχης λαῶν, φυλάξεις με εἰς κεφαλὴν ἐθνῶν· λαὸς ὃν οὐκ ἔγνω ἐδούλευσάν μοι.
\VS{45}Υἱοὶ ἀλλότριοι ἐψεύσαντό μοι, εἰς ἀκοὴν ὠτίου ἤκουσάν μου.
\VS{46}Υἱοὶ ἀλλότριοι ἀποῤῥιφήσονται, καὶ σφαλοῦσιν ἐκ τῶν συγκλεισμῶν αὐτῶν.
\par }{\PP \VS{47}Ζῇ Κύριος, καὶ εὐλογητὸς ὁ φύλαξ μου, καὶ ὑψωθήσεται ὁ Θεός μου ὁ φύλαξ τῆς σωτηρίας μου.
\VS{48}Ἰσχυρὸς Κύριος ὁ διδοὺς ἐκδικήσεις ἐμοὶ, παιδεύων λαοὺς ὑποκάτω μου,
\VS{49}καὶ ἐξάγων με ἐξ ἐχθρῶν μου· καὶ ἐκ τῶν ἐπεγειρομένων μοι ὑψώσεις με, ἐξ ἀνδρὸς ἀδικημάτων ῥύσῃ με.
\VS{50}Διὰ τοῦτο ἐξομολογήσομαί σοι Κύριε ἐν τοῖς ἔθνεσι, καὶ ἐν τῷ ὀνόματί σου ψαλῶ.
\VS{51}Μεγαλύνων τὰς σωτηρίας βασιλέως αὐτοῦ, καὶ ποιῶν ἔλεος τῷ χριστῷ αὐτοῦ τῷ Δαυὶδ, καὶ τῷ σπέρματι αὐτοῦ ἕως αἰῶνος.
\par }{\PP Καὶ οὗτοι οἱ λόγοι Δαυὶδ οἱ ἔσχατοι·

\par }\Chap{23}{\PP \VerseOne{1}Πιστὸς Δαυὶδ υἱὸς Ἰεσσαί, καὶ πιστὸς ἀνὴρ ὃν ἀνέστησε Κύριος ἐπὶ χριστὸν Θεοῦ Ἰακὼβ, καὶ εὐπρεπεῖς ψαλμοὶ Ἰσραήλ.
\par }{\PP \VS{2}Πνεῦμα Κυρίου ἐλάλησεν ἐν ἐμοὶ, καὶ ὁ λόγος αὐτοῦ ἐπὶ γλώσσης μου.
\VS{3}Λέγει ὁ Θεὸς Ἰσραὴλ, ἐμοὶ ἐλάλησε φύλαξ ἐξ Ἰσραὴλ παραβολήν· εἶπὸν ἐν ἀνθρώπῳ, πῶς κραταιώσητε φόβον χριστοῦ;
\VS{4}Καὶ ἐν φωτὶ Θεοῦ πρωΐας, ἀνατείλαι ἥλιος τοπρωῒ, οὗ Κύριος παρῆλθεν ἐκ φέγγους, καὶ ὡς ἐξ ὑετοῦ χλόης ἀπὸ γῆς.
\VS{5}Οὐ γὰρ οὕτως ὁ οἶκός μου μετὰ ἰσχυροῦ, διαθήκην γὰρ αἰώνιον ἔθετό μοι ἑτοίμην ἐν παντὶ καιρῷ πεφυλαγμένην· ὅτι πᾶσα σωτηρία μου καὶ πᾶν θέλημα, ὅτι οὐ μὴ βλαστήσῃ ὁ παράνομος.
\VS{6}Ὥσπερ ἄκανθα ἐξωσμένη πάντες οὗτοι, ὅτι οὐ χειρὶ ληφθήσονται,
\VS{7}καὶ ἀνὴρ οὐ κοπιάσει ἐν αὐτοῖς· καὶ πλῆρες σιδήρου, καὶ ξύλον δόρατος, καὶ ἐν πυρὶ καύσει, καὶ καυθήσονται αἰσχύνην αὐτῶν.
\par }{\PP \VS{8}Ταῦτα τὰ ὀνόματα τῶν δυνατῶν Δαυίδ· Ἰεβοσθὲ ὁ Χαναναῖος ἄρχων τοῦ τρίτου ἐστίν· Ἀδινὼν ὁ Ἀσωναῖος, οὗτος ἐσπάσατο τὴν ῥομφαίαν αὐτοῦ ἐπὶ ὀκτακοσίους στρατιώτας εἰσάπαξ.
\VS{9}Καὶ μετʼ αὐτὸν Ἐλεανὰν υἱὸς πατραδέλφου αὐτοῦ υἱὸς Δουδὶ τοῦ ἐν τοῖς τρισὶ δυνατοῖς μετὰ Δαυίδ· καὶ ἐν τῷ ὀνειδίσαι αὐτὸν ἐν τοῖς ἀλλοφύλοις, συνήχθησαν ἐκεῖ εἰς πόλεμον, καὶ ἀνέβησεν ἀνὴρ Ἰσραήλ.
\VS{10}Αὐτὸς ἀνέστη καὶ ἐπάταξεν ἐν τοῖς ἀλλοφύλοις, ἕως οὗ ἐκοπίασεν ἡ χεὶρ αὐτοῦ, καὶ προσεκολλήθη ἡ χεὶρ αὐτοῦ πρὸς τὴν μάχαιραν· καὶ ἐποίησε Κύριος σωτηρίαν μεγάλην ἐν τῇ ἡμέρᾳ ἐκείνῃ· καὶ ὁ λαὸς ἐκάθητο ὀπίσω αὐτοῦ πλὴν ἐκδιδύσκειν.
\par }{\PP \VS{11}Καὶ μετʼ αὐτὸν Σαμαΐα υἱὸς Ἄσα ὁ Ἀρουχαῖος· καὶ συνήχθησαν οἱ ἀλλόφυλοι εἰς Θηρία· καὶ ἦν ἐκεῖ μερὶς τοῦ ἀγροῦ πλήρης φακοῦ· καὶ ὁ λαὸς ἔφυγεν ἐκ προσώπου ἀλλοφύλων.
\VS{12}Καὶ ἐστηλώθη ἐν μέσῳ τῆς μερίδος, καὶ ἐξείλατο αὐτὴν, καὶ ἐπάταξε τοὺς ἀλλοφύλους· καὶ ἐποίησε Κύριος σωτηρίαν μεγάλην.
\par }{\PP \VS{13}Καὶ κατέβησαν τρεῖς ἀπὸ τῶν τριάκοντα, καὶ κατέβησαν εἰς Κασὼν πρὸς Δαυὶδ, εἰς τὸ σπήλαιον Ὀδολλάμ· καὶ τάγμα τῶν ἀλλοφύλων, καὶ παρενέβαλον ἐν τῇ κοιλάδι Ῥαφαΐν.
\VS{14}Καὶ Δαυὶδ τότε ἐν τῇ περιοχῇ, καὶ τὸ ὑπόστεμα τῶν ἀλλοφύλων τότε ἐν Βηθλεέμ.
\VS{15}Καὶ ἐπεθύμησε Δαυὶδ καὶ εἶπε, τίς ποτιεῖ με ὕδωρ ἐκ τοῦ λάκκου τοῦ ἐν Βηθλεὲμ τοῦ ἐν τῇ πύλῃ; τὸ δὲ σύστεμα τῶν ἀλλοφύλων τότε ἐν Βηθλεέμ.
\VS{16}Καὶ διέῤῥηξαν οἱ τρεῖς δυνατοὶ ἐν τῇ παρεμβολῇ τῶν ἀλλοφύλων, καὶ ὑδρεύσαντο ὕδωρ ἐκ τοῦ λάκκου τοῦ ἐν Βηθλεὲμ τοῦ ἐν τῇ πύλῃ· καὶ ἔλαβαν, καὶ παρεγένοντο πρὸς Δαυίδ, καὶ οὐκ ἡθέλησε πιεῖν αὐτό· καὶ ἔσπεισεν αὐτὸ τῷ Κυρίῳ.
\VS{17}Καὶ εἶπεν, ἵλεώς μοι Κύριε τοῦ ποιῆσαι τοῦτο, εἰ αἷμα τῶν ἀνδρῶν τῶν πορευθέντων ἐν ταῖς ψυχαῖς αὐτῶν πίομαι· καὶ οὐκ ἠθέλησε πιεῖν αὐτό. Ταῦτα ἐποίησαν οἱ τρεῖς δυνατοί.
\par }{\PP \VS{18}Καὶ Ἀβεσσὰ ὁ ἀδελφὸς Ἰωάβ υἱὸς Σαρουίας αὐτὸς ἄρχων ἐν τοῖς τρισὶ, καὶ αὐτὸς ἐξήγειρε τὸ δόρυ αὐτοῦ ἐπὶ τριακοσίους τραυματίας· καὶ αὐτῷ ὄνομα ἐν τοῖς τρισὶν.
\VS{19}Ἐκ τῶν τριῶν ἐκείνων ἔνδοξος, καὶ ἐγένετο αὐτοῖς εἰς ἄρχοντα, καὶ ἕως τῶν τριῶν οὐκ ἦλθε.
\par }{\PP \VS{20}Καὶ Βαναίας υἱὸς Ἰωδαὲ ἀνὴρ αὐτὸς πολλοστὸς ἔργοις, ἀπὸ Καβεσεήλ, καὶ αὐτὸς ἐπάταξε τοὺς δύο υἱοὺς Ἀριὴλ τοῦ Μωάβ· καὶ αὐτός κατέβη καὶ ἐπάταξε τὸν λέοντα ἐν μέσῳ τοῦ λάκκου ἐν τῇ ἡμέρᾳ τῆς χιόνος.
\VS{21}Αὐτὸς ἐπάταξε τὸν ἄνδρα τὸν Αἰγύπτιον, ἄνδρα ὁρατὸν, ἐν δὲ τῇ χειρὶ τοῦ Αἰγυπτίου δόρυ ὡς ξύλον διαβάθρας· καὶ κατέβη πρὸς αὐτὸν ἐν ῥἀβδῳ, καὶ ἥρπασε τὸ δόρυ ἐκ τῆς χειρὸς τοῦ Αἰγυπτίου, καὶ ἀπέκτεινεν αὐτὸν ἐν τῷ δόρατι αὐτοῦ.
\VS{22}Ταῦτα ἐποίησε Βαναίας υἱὸν Ἰωδαὲ, καὶ αὐτῷ ὄνομα ἐν τοῖς τρισὶ τοῖς δυνατοῖς,
\VS{23}ἐκ τῶν τριῶν ἔνδοξος, καὶ πρὸς τοὺς τρεῖς οὐκ ἦλθε· καὶ ἔταξεν αὐτὸν Δαυὶδ πρὸς τὰς ἀκοὰς αὐτοῦ.
\par }{\PP Καὶ ταῦτα τὰ ὀνόματα τῶν δυνατῶν Δαυὶδ τοῦ βασιλέως.
\VS{24}Ἀσαὴλ ἀδελφὸς Ἰωάβ· οὗτος ἐν τοῖς τριάκοντα· Ἐλεανὰν υἱὸς Δουδὶ πατραδέλφου αὐτοῦ ἐν Βηθλεέμ·
\VS{25}Σαιμὰ ὁ Ῥουδαῖος·
\VS{26}Σελλὴς ὁ Κελωθί· Ἴρας υἱὸς Ἴσκα ὁ Θεκωΐτης·
\VS{27}Ἀβιέζερ ὁ Ἀνωθίτης, ἐκ τῶν υἱῶν τοῦ Ἀνωθίτου·
\VS{28}Ἐλλὼν ὁ Ἀωΐτης· Νοερὲ ὁ Νετωφατίτης·
\VS{29}Ἐσθαὶ υἱὸς Ῥιβὰ ἐκ Γαβαὲθ υἱὸς Βενιαμὶν τοῦ Ἐφραθαίου· Ἀσμὼθ ὁ Βαρδιαμίτης·
\VS{32}Ἐμασοὺ ὁ Σαλαβωνίτης· υἱοὶ Ἀσάν, Ἰωνάθαν·
\VS{33}Σαμνὰν ὁ Ἁρωδίτης· Ἀμνὰν υἱὸς Ἀραῒ Σαραουρίτης·
\VS{34}Ἀλιφαλὲθ υἱὸς τοῦ Ἀσβίτου, υἱὸς τοῦ Μαχαχαχί· Ἐλιὰβ υἱὸς Ἀχιτόφελ τοῦ Γελωνίτου·
\VS{35}Ἀσαραῒ ὁ Καρμήλιος τοῦ Οὐραιοερχί·
\VS{36}Γάαλ υἱὸς Ναθανά· πολυδυνάμεως υἱὸς Γαλααδδί·
\VS{37}Ἐλιὲ ὁ Ἀμμανίτης·
\VS{37a}Ἁδροὶ ἀπὸ χειμάῤῥων·
\VS{37b}Γαδαβιὴλ υἱὸς τοῦ Ἀραβωθαίου·
\VS{37c}Γελωρὲ ὁ Βηθωραῖος αἴρων τὰ σκεύη· Ἰωὰβ υἱὸς Σαρουίας·
\VS{38}Ἴρας ὁ Ἐθιραῖος· Γηρὰβ ὁ Ἐθεναῖος·
\VS{39}Οὐρίας ὁ Χετταῖος· οἱ πάντες τριάκοντα καὶ ἑπτά.

\par }\Chap{24}{\PP \VerseOne{1}Καὶ προσέθετο ὀργὴν Κύριος ἐκκαῆναι ἐν Ἰσραήλ, καὶ ἐπέσεισε τὸν Δαυὶδ ἐν αὐτοῖς, λέγων, βάδιζε, ἀρίθμησον τὸν Ἰσραὴλ καὶ τὸν Ἰούδαν.
\VS{2}Καὶ εἶπεν ὁ βασιλεὺς πρὸς Ἰωὰβ ἄρχοντα τῆς ἰσχύος τὸν μετʼ αὐτοῦ, διέλθε δὴ πάσας φυλὰς Ἰσραὴλ καὶ Ἰούδα, ἀπὸ Δὰν καὶ ἕως Βηρσαβεέ, καὶ ἐπίσκεψαι τὸν λαὸν, καὶ γνώσομαι τὸν ἀριθμὸν τοῦ λαοῦ.
\VS{3}Καὶ εἶπεν Ἰωὰβ πρὸς τὸν βασιλέα, καὶ προσθείη Κύριος ὁ Θεὸς πρὸς τὸν λαὸν ὥσπερ αὐτοὺς καὶ ὥσπερ αὐτοὺς ἑκατονταπλασίονα, καὶ ὀφθαλμοὶ τοῦ κυρίου μου τοῦ βασιλέως ὁρῶντες· καὶ ὁ κύριός μου ὁ βασιλεὺς ἱνατί βούλεται ἐν τῷ λόγῳ τούτῳ;
\VS{4}Καὶ ὑπερίσχυσεν ὁ λόγος τοῦ βασιλέως πρὸς Ἰωὰβ καὶ εἰς τοὺς ἄρχοντας τῆς δυνάμεως·
\par }{\PP Καὶ ἐξῆλθεν Ἰωὰβ καὶ οἱ ἄρχοντες τῆς ἰσχύος ἐνώπιον τοῦ βασιλέως ἐπισκέψασθαι τὸν λαὸν τὸν Ἰσραήλ.
\VS{5}Καὶ διέβησαν τὸν Ἰορδάνην, καὶ παρενέβαλον ἐν Ἀροὴρ ἐκ δεξιῶν τῆς πόλεως τῆς ἐν μέσῳ τῆς φάραγγος Γὰδ καὶ Ἐλιέζερ.
\VS{6}Καὶ ἦλθον εἰς Γαλαὰδ καὶ εἰς γῆν Θαβασὼν, ἥ ἐστιν Ἀδασαὶ, καὶ παρεγένοντο εἰς Δανιδὰν καὶ Οὐδὰν, καὶ ἐκύκλωσαν Σιδῶνα.
\VS{7}Καὶ ἦλθον εἰς Μάψαρ Τύρου, καὶ εἰς πάσας τὰς πόλεις τοῦ Εὐαίου καὶ τοῦ Χαναναίου· καὶ ἦλθαν κατὰ Νότον Ἰούδα εἰς Βηρσαβεὲ,
\VS{8}καὶ περιώδευσαν ἐν πάσῃ τῇ γῇ· καὶ παρεγένοντο ἀπὸ τέλους ἐννέα μηνῶν καὶ εἴκοσι ἡμερῶν εἰς Ἰερουσαλήμ.
\VS{9}Καὶ ἔδωκεν Ἰωὰβ τὸν ἀριθμὸν τῆς ἐπισκέψεως τοῦ λαοῦ πρὸς τὸν βασιλέα· καὶ ἐγένετο Ἰσραὴλ, ὀκτακόσιαι χιλιάδες ἀνδρῶν δυνάμεως σπωμένων ῥομφαίαν· καὶ ἀνὴρ Ἰούδα, πεντακόσιαι χιλιάδες ἀνδρῶν μαχητῶν.
\par }{\PP \VS{10}Καὶ ἐπάταξε καρδία Δαυὶδ αὐτὸν μετὰ τὸ ἀριθμῆσαι τὸν λαόν· καὶ εἶπε Δαυὶδ πρὸς Κύριον, ἥμαρτον σφόδρα ὃ ἐποίησα νῦν Κύριε· παραβίβασον δὴ τὴν ἀνομίαν τοῦ δούλου σου, ὅτι ἐμωράνθην σφόδρα.
\par }{\PP \VS{11}Καὶ ἀνέστη Δαυὶδ τοπρωΐ· καὶ λόγος Κυρίου ἐγένετο πρὸς Γὰδ τὸν προφήτην τὸν ὁρῶντα, λέγων,
\VS{12}πορεύθητι, καὶ λάλησον πρὸς Δαυὶδ, λέγων, τάδε λέγει Κύριος, τρία ἐγώ εἰμι αἴρω ἐπὶ σὲ, καὶ ἔκλεξαι σεαυτῷ ἓν ἐξ αὐτῶν, καὶ ποιήσω σοι.
\VS{13}Καὶ εἰσῆλθε Γὰδ πρὸς Δαυὶδ, καὶ ἀνήγγειλε, καὶ εἶπεν αὐτῷ, ἔκλεξαι σεαυτῷ γενέσθαι, εἰ ἔλθῃ σοι τρία ἔτη λιμὸς ἐν τῇ γῇ σου, ἢ τρεῖς μῆνας φεύγειν σε ἔμπροσθεν τῶν ἐχθρῶν σου, καὶ ἔσονται διώκοντές σε, ἢ γενέσθαι τρεῖς ἡμέρας θάνατον ἐν τῇ γῇ σου· νῦν οὖν γνῶθι καὶ ἴδε τί ἀποκριθῶ τῷ ἀποστείλαντί με ῥῆμα.
\VS{14}Καὶ εἶπε Δαυὶδ πρὸς Γὰδ, στενά μοι πάντοθεν σφόδρα ἐστίν· ἐμπεσοῦμαι δὴ εἰς χεῖρας Κυρίου, ὅτι πολλοὶ οἱ οἰκτιρμοὶ αὐτοῦ σφόδρα· εἰς δὲ χεῖρας ἀνθρώπου οὐ μὴ ἐμπέσω.
\par }{\PP \VS{15}Καὶ ἐξελέξατο ἑαυτῷ Δαυὶδ τὸν θάνατον· καὶ ἡμέραι θερισμοῦ πυρῶν· καὶ ἔδωκε Κύριος θάνατον ἐν Ἰσραὴλ ἀπὸ πρωΐθεν ἕως ὥρας ἀρίστου, καὶ ἤρξατο ἡ θραῦσις ἐν τῷ λαῷ· καὶ ἀπέθανεν ἐκ τοῦ λαοῦ ἀπὸ Δὰν καὶ ἕως Βηρσαβεὲ, ἑβδομήκοντα χιλιάδες ἀνδρῶν.
\VS{16}Καὶ ἐξέτεινεν ὁ ἄγγελος τοῦ Θεοῦ τὴν χεῖρα αὐτοῦ εἰς Ἱερουσαλὴμ τοῦ διαφθεῖραι αὐτὴν, καὶ παρεκλήθη Κύριος ἐπὶ τῇ κακίᾳ, καὶ εἶπε τῷ ἀγγέλῳ τῷ διαφθείροντι ἐν τῷ λαῷ, πολὺ νῦν, ἄνες τὴν χεῖρά σου· καὶ ὁ ἄγγελος Κυρίου ἦν παρὰ τῇ ἅλῳ Ὀρνὰ τοῦ Ἰεβουσαίου.
\VS{17}Καὶ εἶπε Δαυὶδ πρὸς Κύριον, ἐν τῷ ἰδεῖν αὐτὸν τὸν ἄγγελον τὸν τύπτοντα ἐν τῷ λαῷ, καὶ εἶπεν, ἰδοὺ, ἐγώ εἰμι ἠδίκησα· καὶ οὗτοι τὰ πρόβατα τί ἐποίησαν; γενέσθω δὴ ἡ χείρ σου ἐν ἐμοὶ, καὶ ἐν τῷ οἴκῳ τοῦ πατρός μου.
\par }{\PP \VS{18}Καὶ ἦλθε Γὰδ πρὸς Δαυὶδ ἐν τῇ ἡμέρᾳ ἐκείνῃ, καὶ εἶπεν αὐτῷ, ἀνάβηθι, καὶ στῆσον τῷ Κυρίῳ θυσιαστήριον ἐν τῷ ἅλωνι Ὀρνὰ τοῦ Ἰεβουσαίου.
\VS{19}Καὶ ἀνέβη Δαυὶδ κατὰ τὸν λόγον Γὰδ, καθʼ ὃν τρόπον ἐνετείλατο αὐτῷ Κύριος.
\VS{20}Καὶ διέκυψεν Ὀρνὰ, καὶ εἶδε τὸν βασιλέα καὶ τοὺς παῖδας αὐτοῦ παραπορευομένους ἐπάνω αὐτοῦ· καὶ ἐξῆλθεν Ὀρνὰ, καὶ προσεκύνησε τῷ βασιλεῖ ἐπὶ πρόσωπον αὐτοῦ ἐπὶ τὴν γῆν.
\VS{21}Καὶ εἶπεν Ὀρνὰ, τί ὅτι ἦλθεν ὁ κύριός μου ὁ βασιλεὺς πρὸς τὸν δοῦλον αὐτοῦ; καὶ εἶπε Δαυὶδ, κτήσασθαι παρὰ σοῦ τὸν ἅλωνα τοῦ οἰκοδομῆσαι θυσιαστήριον τῷ Κυρίῳ, καὶ συσχεθῇ ἡ θραῦσις ἐπάνω τοῦ λαοῦ.
\VS{22}Καὶ εἶπεν Ὀρνὰ πρὸς Δαυὶδ, λαβέτω καὶ ἀνενεγκάτω ὁ κύριός μου ὁ βασιλεὺς τῷ Κυρίῳ τὸ ἀγαθὸν ἐν ὀφθαλμοῖς αὐτοῦ· ἰδοὺ οἱ βόες εἰς ὁλοκαύτωμα, καὶ οἱ τροχοὶ καὶ τὰ σκεύη τῶν βοῶν εἰς ξύλα.
\VS{23}Τὰ πάντα ἔδωκεν Ὀρνὰ τῷ βασιλεῖ· καὶ εἶπεν Ὀρνὰ πρὸς τὸν βασιλέα, Κύριος ὁ Θεός σου εὐλογήσαι σε.
\VS{24}Καὶ εἶπεν ὁ βασιλεὺς πρὸς Ὀρνὰ, οὐχὶ, ὅτι ἀλλὰ κτώμενος κτήσομαι παρὰ σοῦ ἐν ἀναλλάγματι, καὶ οὐκ ἀνοίσω τῷ Κυρίῳ μου Θεῷ ὁλοκαύτωμα δωρεάν. Καὶ ἐκτήσατο Δαυὶδ τὸν ἅλωνα καὶ τοὺς βόας ἐν ἀργυρίῳ σίκλων πεντήκοντα.
\VS{25}Καὶ ᾠκοδόμησεν ἐκεῖ Δαυὶδ θυσιαστήριον Κυρίῳ, καὶ ἀνήνεγκεν ὁλοκαυτώσεις καὶ εἰρηνικάς· καὶ προσέθηκε Σαλωμὼν ἐπὶ τὸ θυσιαστήριον ἐπʼ ἐσχάτῳ, ὅτι μικρὸν ἦν ἐν πρώτοις· καὶ ἐπήκουσε Κύριος τῇ γῇ, καὶ συνεσχέθη ἡ θραῦσις ἐπάνωθεν Ἰσραήλ.
\par }