\NormalFont\ShortTitle{ΗΣΑΙΑΣ}
{\MT ΗΣΑΙΑΣ

\par }\ChapOne{1}{\PP \VerseOne{1}ὍΡΑΣΙΣ ἣν εἶδεν Ἡσαΐας υἱὸς Ἀμὼς, ἣν εἶδε κατὰ τῆς Ἰουδαίας καὶ κατὰ Ἱερουσαλὴμ, ἐν βασιλείᾳ Ὀζίου, καὶ Ἰωάθαμ, καὶ Ἄχαζ, καὶ Ἐζεκίου οἳ ἐβασίλευσαν τῆς Ἰουδαίας.
\par }{\PP \VS{2}Ἄκουε οὐρανὲ, καὶ ἐνωτίζου γῆ, ὅτι Κύριος ἐλάλησεν, υἱοὺς ἐγέννησα καὶ ὕψωσα, αὐτοὶ δέ με ἠθέτησαν.
\VS{3}Ἔγνω βοῦς τὸν κτησάμενον, καὶ ὄνος τὴν φάτνην τοῦ κυρίου αὐτοῦ· Ἰσραὴλ δέ με οὐκ ἔγνω, καὶ ὁ λαός με οὐ συνῆκεν.
\par }{\PP \VS{4}Οὐαὶ ἔθνος ἁμαρτωλὸν, λαὸς πλήρης ἁμαρτιῶν, σπέρμα πονηρὸν, υἱοὶ ἄνομοι· ἐγκατελίπατε τὸν Κύριον, καὶ παρωργίσατε τὸν ἅγιον τοῦ Ἰσραήλ.
\VS{5}Τί ἔτι πληγῆτε προστιθέντες ἀνομίαν; πᾶσα κεφαλὴ εἰς πόνον, καὶ πᾶσα καρδία εἰς λύπην·
\VS{6}Ἀπὸ ποδῶν ἕως κεφαλῆς, οὐκ ἔστιν ἐν αὐτῷ ὁλοκληρία, οὔτε τραῦμα, οὔτε μώλωψ, οὔτε πληγὴ φλεγμαίνουσα· οὐκ ἔστι μάλαγμα ἐπιθεῖναι, οὔτε ἔλαιον, οὔτε καταδέσμους.
\VS{7}Ἡ γῆ ὑμῶν ἔρημος, αἱ πόλεις ὑμῶν πυρίκαυστοι· τὴν χώραν ὑμῶν ἐνώπιον ὑμῶν ἀλλότριοι κατεσθίουσιν αὐτὴν, καὶ ἠρήμωται κατεστραμμένη ὑπὸ λαῶν ἀλλοτρίων.
\VS{8}Ἐγκαταλειφθήσεται ἡ θυγάτηρ Σιὼν, ὡς σκηνὴ, ἐν ἀμπελῶνι, καὶ ὡς ὀπωροφυλάκιον ἐν σικυηράτῳ, ὡς πόλις πολιορκουμένη.
\VS{9}Καὶ εἰ μὴ Κύριος σαβαὼθ ἐγκατέλιπεν ἡμῖν σπέρμα, ὡς Σόδομα ἂν ἐγενήθημεν, καὶ ὡς Γόμοῤῥα ἂν ὁμοίωθημεν.
\par }{\PP \VS{10}Ἀκούσατε λόγον Κυρίου, ἄρχοντες Σοδόμων· προσέχετε νόμον Θεοῦ, λαὸς Γομόῤῥας.
\VS{11}Τί μοι πλῆθος τῶν θυσιῶν ὑμῶν; λέγει Κύριος· πλήρης εἰμὶ ὁλοκαυτωμάτων κριῶν, καὶ στέαρ ἀρνῶν καὶ αἷμα ταύρων καὶ τράγων οὐ βούλομαι,
\VS{12}οὐδʼ ἂν ἔρχησθε ὀφθῆναί μοι· τίς γὰρ ἐξεζήτησε ταῦτα ἐκ τῶν χειρῶν ὑμῶν; πατεῖν τὴν αὐλήν μου οὐ προσθήσεθε.
\VS{13}Ἐὰν φέρητε σεμίδαλιν, μάταιον θυμίαμα, βδέλυγμά μοι ἐστι· τὰς νουμηνίας ὑμῶν, καὶ τὰ σάββατα, καὶ ἡμέραν μεγάλην οὐκ ἀνέχομαι· νηστείαν, καὶ ἀργίαν,
\VS{14}καὶ τὰς νουμηνίας ὑμῶν, καὶ τὰς ἑορτὰς ὑμῶν μισεῖ ἡ ψυχή μου· ἐγενήθητέ μοι εἰς πλησμονὴν, οὐκέτι ἀνήσω τὰς ἁμαρτίας ὑμῶν.
\VS{15}Ὅταν ἐκτείνητε τὰς χεῖρας, ἀποστρέψω τοὺς ὀφθαλμούς μου ἀφʼ ὑμῶν· καὶ ἐὰν πληθύνητε τὴν δέησιν, οὐκ εἰσακούσομαι ὑμῶν· αἱ γὰρ χεῖρες ὑμῶν αἵματος πλήρεις.
\par }{\PP \VS{16}Λούσασθε, καθαροὶ γένεσθε, ἀφέλετε τὰς πονηρίας ἀπὸ τῶν ψυχῶν ὑμῶν, ἀπέναντι τῶν ὀφθαλμῶν μου· παύσασθε ἀπὸ τῶν πονηριῶν ὑμῶν,
\VS{17}μάθετε καλὸν ποιεῖν, ἐκζητήσατε κρίσιν, ῥύσασθε ἀδικούμενον, κρίνατε ὀρφανῷ, καὶ δικαιώσατε χήραν.
\par }{\PP \VS{18}Καὶ δεῦτε, διελεγχθῶμεν, λέγει Κύριος· καὶ ἐὰν ὦσιν αἱ ἁμαρτίαι ὑμῶν ὡς φοινικοῦν, ὡς χιόνα λευκανῶ· ἐὰν δὲ ὦσιν ὡς κόκκινον, ὡς ἔριον λευκανῶ.
\VS{19}Καὶ ἐὰν θέλητε, καὶ εἰσακούσητέ μου, τὰ ἀγαθὰ τῆς γῆς φάγεσθε.
\VS{20}Ἐὰν δὲ μὴ θέλητε, μηδὲ εἰσακούσητέ μου, μάχαιρα ὑμᾶς κατέδεται· τὸ γὰρ στόμα Κυρίου ἐλάλησε ταῦτα.
\par }{\PP \VS{21}Πῶς ἐγένετο πόρνη πόλις πιστὴ Σιὼν πλήρης κρίσεως; ἐν ᾗ δικαιοσύνη ἐκοιμήθη ἐν αὐτῇ, νῦν δὲ φονευταί.
\VS{22}Τὸ ἀργύριον ὑμῶν ἀδόκιμον· οἱ κάπηλοί σου μίσγουσι τὸν οἶνον ὕδατι.
\VS{23}Οἱ ἄρχοντές σου ἀπειθοῦσι, κοινωνοὶ κλεπτῶν, ἀγαπῶντες δῶρα, διώκοντες ἀνταπόδομα, ὀρφανοῖς οὐ κρίνοντες, καὶ κρίσιν χηρῶν οὐ προσέχοντες.
\par }{\PP \VS{24}Διατοῦτο τάδε λέγει Κύριος ὁ δεσπότης σαβαὼθ, οὐαὶ οἱ ἰσχύοντες Ἰσραήλ· οὐ παύσεται γάρ μου ὁ θυμὸς ἐν τοῖς ὑπεναντίοις, καὶ κρίσιν ἐκ τῶν ἐχθρῶν μου ποιήσω.
\VS{25}Καὶ ἐπάξω τὴν χεῖρά μου ἐπὶ σὲ, καὶ πυρώσω εἰς καθαρόν, τοὺς δὲ ἀπειθοῦντας ἀπολέσω, καὶ ἀφελῶ πάντας ἀνόμους ἀπὸ σοῦ·
\VS{26}Καὶ ἐπιστήσω τοὺς κριτάς σου ὡς τὸ πρότερον, καὶ τοὺς συμβούλους σου ὡς τὸ ἀπʼ ἀρχῆς· καὶ μετὰ ταῦτα κληθήσῃ πόλις δικαιοσύνης, μητρόπολις πιστὴ Σιών·
\VS{27}μετὰ γὰρ κρίματος σωθήσεται ἡ αἰχμαλωσία αὐτῆς, καὶ μετὰ ἐλεημοσύνης.
\VS{28}Καὶ συντριβήσονται οἱ ἄνομοι καὶ οἱ ἁμαρτωλοὶ ἅμα, καὶ οἱ ἐγκαταλιπόντες τὸν Κύριον συντελεσθήσονται.
\VS{29}Διότι αἰσχυνθήσονται ἀπὸ τῶν εἰδώλων αὐτῶν ἃ αὐτοὶ ἠβούλοντο, καὶ ᾐσχύνθησαν ἐπὶ τοῖς κήποις, ἃ ἐπεθύμησαν.
\VS{30}Ἔσονται γὰρ ὡς τερέβινθος ἀποβεβληκυῖα τὰ φύλλα, καὶ ὡς παράδεισος ὕδωρ μὴ ἔχων.
\VS{31}Καὶ ἔσται ἡ ἰσχὺς αὐτῶν ὡς καλάμη στιππύου, καὶ αἱ ἐργασίαι αὐτῶν ὡς σπινθῆρες, καὶ κατακαυθήσονται οἱ ἄνομοι καὶ οἱ ἁμαρτωλοὶ ἅμα, καὶ οὐκ ἔσται ὁ σβέσων.

\par }\Chap{2}{\PP \VerseOne{1}Ὁ λόγος ὁ γενόμενος πρὸς Ἡσαΐαν υἱὸν Ἀμὼς περὶ τῆς Ἰουδαίας, καὶ περὶ Ἱερουσαλήμ.
\par }{\PP \VS{2}Ὅτι ἔσται ἐν ταῖς ἐσχάταις ἡμέραις ἐμφανὲς τὸ ὄρος Κυρίου, καὶ ὁ οἶκος τοῦ Θεοῦ ἐπʼ ἄκρου τῶν ὀρέων, καὶ ὑψωθήσεται ὑπεράνω τῶν βουνῶν, καὶ ἥξουσιν ἐπʼ αὐτὸ πάντα τὰ ἔθνη.
\VS{3}Καὶ πορεύσονται ἔθνη πολλὰ, καὶ ἐροῦσι, δεῦτε καὶ ἀναβῶμεν εἰς τὸ ὄρος Κυρίου, καὶ εἰς τὸν οἶκον τοῦ Θεοῦ Ἰακὼβ, καὶ ἀναγγελεῖ ἡμῖν τὴν ὁδὸν αὐτοῦ, καὶ πορευσόμεθα ἐν αὐτῇ· ἐκ γὰρ Σιὼν ἐξελεύσεται νόμος, καὶ λόγος Κυρίου ἐξ Ἱερουσαλήμ.
\VS{4}Καὶ κρινεῖ ἀναμέσον τῶν ἐθνῶν, καὶ ἐξελέγξει λαὸν πολύν· καὶ συγκόψουσι τὰς μαχαίρας αὐτῶν εἰς ἄροτρα, καὶ τὰς ζιβύνας αὐτῶν εἰς δρέπανα· καὶ οὐ λήψεται ἔθνος ἐπʼ ἔθνος μάχαιραν, καὶ οὐ μὴ μάθωσιν ἔτι πολεμεῖν.
\par }{\PP \VS{5}Καὶ νῦν ὁ οἶκος Ἰακὼβ, δεῦτε πορευθῶμεν τῷ φωτὶ Κυρίου.
\VS{6}Ἀνῆκεν γὰρ τὸν λαὸν αὐτοῦ τὸν οἶκον τοῦ Ἰσραήλ· ὅτι ἐνεπλήσθη ὡς τὸ ἀπʼ ἀρχῆς ἡ χώρα αὐτῶν κληδονισμῶν, ὡς ἡ τῶν ἀλλοφύλων, καὶ τέκνα πολλὰ ἀλλόφυλα ἐγενήθη αὐτοῖς.
\VS{7}Ἐνεπλήσθη γὰρ ἡ χώρα αὐτῶν ἀργυρίου καὶ χρυσίου, καὶ οὐκ ἦν ἀριθμὸς τῶν θησαυρῶν αὐτῶν· καὶ ἐνεπλήσθη ἡ γῆ ἵππων, καὶ οὐκ ἦν ἀριθμὸς τῶν ἁρμάτων αὐτῶν.
\VS{8}Καὶ ἐνεπλήσθη ἡ γῆ βδελυγμάτων τῶν ἔργων τῶν χειρῶν αὐτῶν, καὶ προσεκύνησαν οἷς ἐποίησαν οἱ δάκτυλοι αὐτῶν.
\VS{9}Καὶ ἔκυψεν ἄνθρωπος, καὶ ἐταπεινώθη ἀνήρ, καὶ οὐ μὴ ἀνήσω αὐτούς.
\par }{\PP \VS{10}Καὶ νῦν εἰσέλθετε εἰς τὰς πέτρας, καὶ κρύπτεσθε εἰς τὴν γῆν ἀπὸ προσώπου τοῦ φόβου Κυρίου, καὶ ἀπὸ τῆς δόξης τῆς ἰσχύος αὐτοῦ, ὅταν ἀναστῇ θραῦσαι τὴν γῆν.
\VS{11}Οἱ γὰρ ὀφθαλμοὶ Κυρίου ὑψηλοί, ὁ δὲ ἄνθρωπος ταπεινός· καὶ ταπεινωθήσεται τὸ ὕψος τῶν ἀνθρώπων, καὶ ὑψωθήσεται Κύριος μόνος ἐν τῇ ἡμέρᾳ ἐκείνῃ.
\par }{\PP \VS{12}Ἡμέρα γὰρ Κυρίου σαβαὼθ ἐπὶ πάντα ὑβριστὴν καὶ ὑπερήφανον καὶ ἐπὶ πάντα ὑψηλὸν καὶ μετέωρον, καὶ ταπεινωθήσονται.
\VS{13}Καὶ ἐπὶ πᾶσαν κέδρον τοῦ Λιβάνου τῶν ὑψηλῶν καὶ μετεώρων, καὶ ἐπὶ πᾶν δένδρον βαλάνου Βασάν,
\VS{14}καὶ ἐπὶ πᾶν ὑψηλὸν ὄρος, καὶ ἐπὶ πάντα βουνὸν ὑψηλόν,
\VS{15}καὶ ἐπὶ πάντα πύργον ὑψηλὸν, καὶ ἐπὶ πᾶν τεῖχος ὑψηλόν,
\VS{16}καὶ ἐπὶ πᾶν πλοῖον θαλάσσης, καὶ ἐπὶ πᾶσαν θέαν πλοίων κάλλους.
\VS{17}Καὶ ταπεινωθήσεται πᾶς ἄνθρωπος, καὶ πεσεῖται ὕβρις τῶν ἀνθρώπων, καὶ ὑψωθήσεται Κύριος μόνος ἐν τῇ ἡμέρᾳ ἐκείνῃ.
\VS{18}Καὶ τὰ χειροποίητα πάντα κατακρύψουσιν,
\VS{19}εἰσενέγκαντες εἰς τὰ σπήλαια, καὶ εἰς τὰς σχισμὰς τῶν πετρῶν, καὶ εἰς τὰς τρώγλας τῆς γῆς, ἀπὸ προσώπου τοῦ φόβου Κυρίου, καὶ ἀπὸ τῆς δόξης τῆς ἰσχύος αὐτοῦ, ὅταν ἀναστῇ θραῦσαι τὴν γῆν.
\VS{20}τῇ γὰρ ἡμέρᾳ ἐκείνῃ ἐκβαλεῖ ἄνθρωπος τὰ βδελύγματα αὐτοῦ τὰ ἀργυρᾶ καὶ τὰ χρυσᾶ, ἃ ἐποίησαν προσκυνεῖν τοῖς ματαίοις καὶ ταῖς νυκτερίσι,
\VS{21}τοῦ εἰσελθεῖν εἰς τὰς τρώγλας τῆς στερεᾶς πέτρας, καὶ εἰς τὰς σχισμὰς τῶν πετρῶν, ἀπὸ προσώπου τοῦ φόβου Κυρίου, καὶ ἀπὸ τῆς δόξης τῆς ἰσχύος αὐτοῦ, ὅταν ἀναστῇ θραῦσαι τὴν γῆν.

\par }\Chap{3}{\PP \VerseOne{1}Ἰδοὺ δὴ ὁ δεσπότης Κύριος σαβαὼθ ἀφελεῖ ἀπὸ Ἱερουσαλὴμ, καὶ ἀπὸ τῆς Ἰουδαίας, ἰσχύοντα καὶ ἰσχύουσαν, ἰσχὺν ἄρτου καὶ ἰσχὺν ὕδατος,
\VS{2}γίγαντα καὶ ἰσχύοντα, καὶ ἄνθρωπον πολεμιστὴν, καὶ δικαστὴν, καὶ προφήτην, καὶ στοχαστὴν, καὶ πρεσβύτερον,
\VS{3}καὶ πεντηκόνταρχον, καὶ θαυμαστὸν σύμβουλον, καὶ σοφὸν ἀρχιτέκτονα, καὶ συνετὸν ἀκροατήν.
\VS{4}Καὶ ἐπιστήσω νεανίσκους ἄρχοντας αὐτῶν, καὶ ἐμπαίκται κυριεύσουσιν αὐτῶν.
\VS{5}Καὶ συμπεσεῖται ὁ λαὸς, ἄνθρωπος πρὸς ἄνθρωπον, καὶ ἄνθρωπος πρὸς τὸν πλησίον αὐτοῦ· προσκόψει τὸ παιδίον πρὸς τὸν πρεσβύτην, ὁ ἄτιμος πρὸς τὸν ἔντιμον.
\VS{6}Ὅτι ἐπιλήψεται ἄνθρωπος τοῦ ἀδελφοῦ αὐτοῦ, ἢ τοῦ οἰκείου τοῦ πατρὸς αὐτοῦ, λέγων, ἱμάτιον ἔχεις, ἀρχηγὸς γενοῦ ἡμῶν, καὶ τὸ βρῶμα τὸ ἐμὸν ὑπὸ σὲ ἔστω.
\VS{7}Καὶ ἀποκριθεὶς ἐν τῇ ἡμέρᾳ ἐκείνῃ ἐρεῖ, οὐκ ἔσομαί σου ἀρχηγός· οὐ γάρ ἐστιν ἐν τῷ οἴκῳ μου ἄρτος, οὐδὲ ἱμάτιον· οὐκ ἔσομαι ἀρχηγὸς τοῦ λαοῦ τούτου.
\VS{8}Ὅτι ἀνεῖται Ἱερουσαλὴμ, καὶ ἡ Ἰουδαία συμπέπτωκε, καὶ αἱ γλῶσσαι αὐτῶν μετὰ ἀνομίας, τὰ πρὸς Κύριον ἀπειθοῦντες. Διότι νῦν ἐταπεινώθη ἡ δόξα αὐτῶν,
\VS{9}καὶ ἡ αἰσχύνη τοῦ προσώπου αὐτῶν ἀντέστη αὐτοῖς· τὴν δὲ ἁμαρτίαν αὐτῶν ὡς Σοδόμων ἀνήγγειλαν καὶ ἐνεφάνισαν· οὐαὶ τῇ ψυχῇ αὐτῶν, διότι βεβούλευνται βουλὴν πονηρὰν, καθʼ ἑαυτῶν
\VS{10}εἰπόντες, δήσωμεν τὸν δίκαιον, ὅτι δύσχρηστος ἡμῖν ἐστι· τοίνυν τὰ γεννήματα τῶν ἔργων αὐτῶν φάγονται.
\VS{11}Οὐαὶ τῷ ἀνόμῳ, πονηρὰ κατὰ τὰ ἔργα τῶν χειρῶν αὐτοῦ συμβήσεται αὐτῷ.
\VS{12}Λαός μου, οἱ πράκτορες ὑμῶν καλαμῶνται ὑμᾶς, καὶ οἱ ἀπαιτοῦντες κυριεύουσιν ὑμῶν· λαός μου, οἱ μακαρίζοντες ὑμᾶς πλανῶσιν ὑμᾶς, καὶ τὸν τρίβον τῶν ποδῶν ὑμῶν ταράσσουσιν.
\par }{\PP \VS{13}Ἀλλὰ νῦν καταστήσεται εἰς κρίσιν Κύριος, καὶ στήσει εἰς κρίσιν τὸν λαὸν αὐτοῦ.
\VS{14}Αὐτὸς Κύριος εἰς κρίσιν ἥξει μετὰ τῶν πρεσβυτέρων τοῦ λαοῦ, καὶ μετὰ τῶν ἀρχόντων αὐτοῦ· ὑμεῖς δὲ τί ἐνεπυρίσατε τὸν ἀμπελῶνά μου, καὶ ἡ ἁρπαγὴ τοῦ πτωχοῦ ἐν τοῖς οἴκοις ὑμῶν;
\VS{15}Τί ὑμεῖς ἀδικεῖτε τὸν λαόν μου, καὶ τὸ πρόσωπον τῶν πτωχῶν καταισχύνετε;
\par }{\PP \VS{16}Τάδε λέγει Κύριος, ἀνθʼ ὧν ὑψώθησαν αἱ θυγατέρες Σιών, καὶ ἐπορεύθησαν ὑψηλῷ τραχήλῳ καὶ ἐν νεύμασιν ὀφθαλμῶν, καὶ τῇ πορείᾳ τῶν ποδῶν ἅμα σύρουσαι τοὺς χιτῶνας, καὶ τοῖς ποσὶν ἅμα παίζουσαι·
\VS{17}Καὶ ταπεινώσει ὁ Θεὸς ἀρχούσας θυγατέρας Σιών· καὶ Κύριος ἀνακαλύψει τὸ σχῆμα αὐτῶν
\VS{18}ἐν τῇ ἡμέρᾳ ἐκείνῃ, καὶ ἀφελεῖ Κύριος τὴν δόξαν τοῦ ἱματισμοῦ αὐτῶν, τὰ ἐμπλόκια, καὶ τοὺς κοσύμβους, καὶ τοὺς μηνίσκους,
\VS{19}καὶ τὸ κάθεμα, καὶ τὸν κόσμον τοῦ προσώπου αὐτῶν,
\VS{20}καὶ τὴν σύνθεσιν τοῦ κόσμου τῆς δόξης, καὶ τοὺς χλιδῶνας, καὶ τὰ ψέλλια, καὶ τὸ ἐμπλόκιον, καὶ τοὺς δακτυλίους, καὶ τὰ περιδέξια, καὶ τὰ ἐνώτια,
\VS{21}καὶ τὰ περιπόρφυρα, καὶ τὰ μεσοπόρφυρα,
\VS{22}καὶ τὰ ἐπιβλήματα τὰ κατὰ τὴν οἰκίαν, καὶ τὰ διαφανῆ Λακωνικά,
\VS{23}καὶ τὰ βύσσινα, καὶ τὰ ὑακίνθινα, καὶ κόκκινα, καὶ τὴν βύσσον, σὺν χρυσῷ καὶ ὑακίνθῳ συγκαθυφασμένα, καὶ θέριστρα κατάκλιτα.
\VS{24}Καὶ ἔσται ἀντὶ ὀσμῆς ἡδείας, κονιορτός· καὶ ἀντὶ ζώνης, σχοινίῳ ζώσῃ· καὶ ἀντὶ τοῦ κόσμου τῆς κεφαλῆς τοῦ χρυσίου, φαλάκρωμα ἕξεις διὰ τὰ ἔργα σοῦ· καὶ ἀντὶ τοῦ χιτῶνος τοῦ μεσοπορφύρου, περιζώσῃ σάκκον.
\VS{25}Καὶ ὁ υἱός σου ὁ κάλλιστος ὃν ἀγαπᾷς, μαχαίρᾳ πεσεῖται· καὶ οἱ ἰσχύοντες ὑμῶν, μαχαίρᾳ πεσοῦνται, καὶ ταπεινωθήσονται·
\VS{26}Καὶ πενθήσουσιν αἱ θῆκαι τοῦ κόσμου ὑμῶν· καὶ καταλειφθήσῃ μόνη, καὶ εἰς τὴν γῆν ἐδαφισθήσῃ.

\par }\Chap{4}{\PP \VerseOne{1}Καὶ ἐπιλήψονται ἑπτὰ γυναῖκες ἀνθρώπου ἑνὸς, λέγουσαι, τὸν ἄρτον ἡμῶν φαγόμεθα, καὶ τὰ ἱμάτια ἡμῶν περιβαλούμεθα, πλὴν τὸ ὄνομα τὸ σὸν κεκλήσθω ἐφʼ ἡμᾶς, ἄφελε τὸν ὀνειδισμὸν ἡμῶν.
\par }{\PP \VS{2}Τῇ δὲ ἡμέρᾳ ἐκείνῃ ἐπιλάμψει ὁ Θεὸς ἐν βουλῇ μετὰ δόξης ἐπὶ τῆς γῆς, τοῦ ὑψῶσαι καὶ δοξάσαι τὸ καταλειφθὲν τοῦ Ἰσραήλ.
\VS{3}Καὶ ἔσται τὸ ὑπολειφθὲν ἐν Σιών, καὶ τὸ καταλειφθὲν ἐν Ἱερουσαλήμ, ἅγιοι κληθήσονται πάντες οἱ γραφέντες εἰς ζωὴν ἐν Ἱερουσαλὴμ.
\VS{4}Ὅτι ἐκπλυνεῖ Κύριος τὸν ῥύπον τῶν υἱῶν καὶ τῶν θυγατέρων Σειών, καὶ τὸ αἷμα ἐκκαθαριεῖ ἐκ μέσου αὐτῶν, ἐν πνεύματι κρίσεως καὶ πνεύματι καύσεως.
\VS{5}Καὶ ἥξει, καὶ ἔσται πᾶς τόπος τοῦ ὄρους Σιὼν καὶ πάντα τὰ περικύκλῳ αὐτῆς σκιάσει νεφέλη ἡμέρας, καὶ ὡς καπνοῦ καὶ φωτὸς πυρὸς καιομένου νυκτὸς, καὶ πάσῃ τῇ δόξῃ σκεπασθήσεται.
\VS{6}Καὶ ἔσται εἰς σκιὰν ἀπὸ καύματος, καὶ ἐν σκέπῃ καὶ ἐν ἀποκρύφῳ ἀπὸ σκληρότητος καὶ ὑετοῦ.

\par }\Chap{5}{\PP \VerseOne{1}Ἄσω δὴ τῷ ἠγαπημένῳ ᾆσμα τοῦ ἀγαπητοῦ μου τῷ ἀμπελῶνί μου.
\par }{\PP Ἀμπελὼν ἐγενήθη τῷ ἠγαπημένῳ, ἐν κέρατι, ἐν τόπῳ πίονι.
\VS{2}Καὶ φραγμὸν περιέθηκα, καὶ ἐχαράκωσα, καὶ ἐφύτευσα ἄμπελον σωρὴκ, καὶ ᾠκοδόμησα πύργον ἐν μέσῳ αὐτοῦ, καὶ προλήνιον ὤρυξα ἐν αὐτῷ, καὶ ἔμεινα τοῦ ποιῆσαι σταφυλὴν, καὶ ἐποίησεν ἀκάνθας.
\VS{3}Καὶ νῦν οἱ ἐνοικοῦντες ἐν Ἱερουσαλὴμ, καὶ ἄνθρωπος τοῦ Ἰούδα, κρίνατε ἐν ἐμοὶ καὶ ἀναμέσον τοῦ ἀμπελῶνός μου.
\VS{4}Τί ποιήσω ἔτι τῷ ἀμπελῶνί μου, καὶ οὐκ ἐποίησα αὐτῷ; διότι ἔμεινα τοῦ ποιῆσαι σταφυλὴν, ἐποίησε δὲ ἀκάνθας.
\VS{5}Νῦν δὲ ἀναγγελῶ ὑμῖν τί ἐγὼ ποιήσω τῷ ἀμπελῶνί μου· ἀφελῶ τὸν φραγμὸν αὐτοῦ, καὶ ἔσται εἰς διαρπαγήν· καὶ καθελῶ τὸν τοῖχον αὐτοῦ, καὶ ἔσται εἰς καταπάτημα·
\VS{6}Καὶ ἀνήσω τὸν ἀμπελῶνά μου, καὶ οὐ τμηθῇ, οὐδὲ μὴ σκαφῇ· καὶ ἀναβήσονται εἰς αὐτὸν, ὡς εἰς χέρσον ἄκανθαι· καὶ ταῖς νεφέλαις ἐντελοῦμαι, τοῦ μὴ βρέξαι εἰς αὐτὸν ὑετόν.
\VS{7}Ὁ γὰρ ἀμπελὼν Κυρίου σαβαὼβ, οἶκος τοῦ Ἰσραὴλ, καὶ ἄνθρωπος τοῦ Ἰούδα νεόφυτον ἠγαπημένον· ἔμεινα τοῦ ποιῆσαι κρίσιν, ἐποίησε δὲ ἀνομίαν, καὶ οὐ δικαιοσύνην, ἀλλὰ κραυγήν.
\par }{\PP \VS{8}Οὐαὶ οἱ συνάπτοντες οἰκίαν πρὸς οἰκίαν, καὶ ἀγρὸν πρὸς ἀγρὸν ἐγγίζοντες, ἵνα τοῦ πλησίον ἀφέλωνταί τι· μὴ οἰκήσετε μόνοι ἐπὶ τῆς γῆς;
\VS{9}Ἠκούσθη γὰρ εἰς τὰ ὦτα Κυρίου σαβαὼθ ταῦτα· ἐὰν γὰρ γένωνται οἰκίαι πολλαί, εἰς ἔρημον ἔσονται μεγάλαι καὶ καλαὶ, καὶ οὐκ ἔσονται οἱ ἐνοικοῦντες ἐν αὐταῖς.
\VS{10}Οὗ γὰρ ἐργῶνται δέκα ζεύγη βοῶν, ποιήσει κεράμιον ἕν· καὶ ὁ σπείρων ἀρτάβας ἓξ, ποιήσει μέτρα τρία.
\par }{\PP \VS{11}Οὐαὶ οἱ ἐγειρόμενοι τοπρωῒ, καὶ τὸ σίκερα διώκοντες, οἱ μένοντες τὸ ὀψέ· ὁ γὰρ οἶνος αὐτοὺς συνκαύσει.
\VS{12}Μετὰ γὰρ κιθάρας καὶ ψαλτηρίου καὶ τυμπάνων καὶ αὐλῶν τον οἶνον πίνουσι, τὰ δὲ ἔργα Κυρίου οὐκ ἐμβλέπουσι, καὶ τὰ ἔργα τῶν χειρῶν αὐτοῦ οὐ κατανοοῦσι.
\par }{\PP \VS{13}Τοίνυν αἰχμάλωτος ὁ λαός μου ἐγενήθη, διὰ τὸ μὴ εἰδέναι αὐτοὺς τὸν Κύριον· καὶ πλῆθος ἐγενήθη νεκρῶν, διὰ λιμὸν καὶ δίψος ὕδατος.
\VS{14}Καὶ ἐπλάτυνεν ὁ ᾅδης τὴν ψυχὴν αὐτοῦ, καὶ διήνοιξε τὸ στόμα αὐτοῦ, τοῦ μὴ διαλιπεῖν· καὶ καταβήσονται οἱ ἔνδοξοι καὶ οἱ μεγάλοι καὶ οἱ πλούσιοι καὶ οἱ λοιμοὶ αὐτῆς.
\VS{15}Καὶ ταπεινωθήσεται ἄνθρωπος, καὶ ἀτιμασθήσεται ἀνήρ· καὶ οἱ ὀφθαλμοὶ οἱ μετέωροι ταπεινωθήσονται.
\VS{16}Καὶ ὑψωθήσεται Κύριος σαβαὼθ ἐν κρίματι, καὶ ὁ Θεὸς ὁ ἅγιος δοξασθήσεται ἐν δικαιοσύνῃ·
\VS{17}Καὶ βοσκηθήσονται οἱ διηρπασμένοι ὡς ταῦροι, καὶ τὰς ἐρήμους τῶν ἀπειλημμένων ἄρνες φάγονται.
\par }{\PP \VS{18}Οὐαὶ οἱ ἐπισπώμενοι τὰς ἁμαρτίας ὡς σχοινίῳ μακρῷ, καὶ ὡς ζυγοῦ ἱμάντι δαμάλεως τὰς ἀνομίας·
\VS{19}Οἱ λέγοντες, τὸ τάχος ἐγγισάτω ἃ ποιήσει, ἵνα ἴδωμεν· καὶ ἐλθάτω ἡ βουλὴ τοῦ ἁγίου Ἰσραὴλ, ἵνα γνῶμεν.
\par }{\PP \VS{20}Οὐαὶ οἱ λέγοντες τὸ πονηρὸν καλὸν, καὶ τὸ καλὸν πονηρὸν· οἱ τιθέντες τὸ σκότος φῶς, καὶ τὸ φῶς σκότος· οἱ τιθέντες τὸ πικρὸν γλυκὺ, καὶ τὸ γλυκὺ πικρόν.
\VS{21}Οὐαὶ οἱ συνετοὶ ἐν ἑαυτοῖς, καὶ ἐνώπιον αὐτῶν ἐπιστήμονες.
\VS{22}Οὐαὶ οἱ ἰσχύοντες ὑμῶν, οἱ πίνοντες τὸν οἶνον, καὶ οἱ δυνάσται οἱ κεραννύντες τὸ σίκερα,
\VS{23}οἱ δικαιοῦντες τὸν ἀσεβῆ ἕνεκεν δώρων, καὶ τὸ δίκαιον τοῦ δικαίου αἴροντες.
\par }{\PP \VS{24}Διατοῦτο ὃν τρόπον καυθήσεται καλάμη ὑπὸ ἄνθρακος πυρός, καὶ συγκαυθήσεται ὑπὸ φλογὸς ἀνειμένης, ἡ ῥίζα αὐτῶν ὡς χνοῦς ἔσται, καὶ τὸ ἄνθος αὐτῶν ὡς κονιορτὸς ἀναβήσεται· οὐ γὰρ ἠθέλησαν τὸν νόμον Κυρίου σαβαὼθ, ἀλλὰ τὸ λόγιον τοῦ ἁγίου Ἰσραὴλ παρώξυναν.
\VS{25}Καὶ ἐθυμώθη ὀργῇ Κύριος σαβαὼθ ἐπὶ τὸν λαὸν αὐτοῦ, καὶ ἐπέβαλε τὴν χεῖρα ἐπʼ αὐτοὺς, καὶ ἐπάταξεν αὐτούς· καὶ παρωξύνθη τὰ ὄρη, καὶ ἐγενήθη τὰ θνησιμαῖα αὐτῶν ὡς κοπρία ἐν μέσῳ ὁδοῦ· καὶ ἐν πᾶσι τούτοις οὐκ ἀπεστράφη ὁ θυμὸς αὐτοῦ, ἀλλὰ ἔτι ἡ χεὶρ ὑψηλή.
\par }{\PP \VS{26}Τοιγαροῦν ἀρεῖ σύσσημον ἐν τοῖς ἔθνεσι τοῖς μακρὰν, καὶ συριεῖ αὐτοὺς ἀπʼ ἄκρου τῆς γῆς· καὶ ἰδοὺ ταχὺ κούφως ἔρχονται.
\VS{27}Οὐ πεινάσουσιν, οὐδὲ κοπιάσουσιν, οὐδὲ νυστάξουσιν, οὐδὲ κοιμηθήσονται, οὐδὲ λύσουσι τὰς ζώνας αὐτῶν ἀπο τῆς ὀσφύος αὐτῶν, οὐδὲ μὴ ῥαγῶσιν οἱ ἱμάντες τῶν ὑποδημάτων αὐτῶν·
\VS{28}Ὧν τὰ βέλη ὀξέα ἐστί, καὶ τὰ τόξα αὐτῶν ἐντεταμένα· οἱ πόδες τῶν ἵππων αὐτῶν ὡς στερεὰ πέτρα ἐλογίσθησαν· οἱ τροχοὶ τῶν ἁρμάτων αὐτῶν ὡς καταιγίς.
\VS{29}Ὀργιῶσιν ὡς λέοντες, καὶ παρέστηκαν ὡς σκύμνοι λέοντος· καὶ ἐπιλήψεται, καὶ βοήσει ὡς θηρίον, καὶ ἐκβαλεῖ, καὶ οὐκ ἔσται ὁ ῥυόμενος αὐτούς.
\VS{30}Καὶ βοήσει διʼ αὐτοὺς τῇ ἡμέρᾳ ἐκείνῃ, ὡς φωνὴ θαλάσσης κυμαινούσης· καὶ ἐμβλέψονται εἰς τὴν γῆν, καὶ ἰδοὺ σκότος σκληρὸν ἐν τῇ ἀπορίᾳ αὐτῶν.

\par }\Chap{6}{\PP \VerseOne{1}Καὶ ἐγένετο τοῦ ἐνιαυτοῦ οὗ ἀπέθανεν Ὀζίας ὁ βασιλεὺς, εἶδον τὸν Κύριον καθήμενον ἐπὶ θρόνου ὑψηλοῦ καὶ ἐπῃρμένου· καὶ πλήρης ὁ οἶκος τῆς δόξης αὐτοῦ.
\VS{2}Καὶ σεραφὶμ εἱστήκεισαν κύκλῳ αὐτοῦ, ἓξ πτέρυγες τῷ ἑνὶ, καὶ ἓξ πτέρυγες τῷ ἑνί· καὶ ταῖς μὲν δυσὶ, κατεκάλυπτον τὸ πρόσωπον· ταῖς δὲ δυσὶ κατεκάλυπτον τοὺς πόδας· καὶ ταῖς δυσὶν ἐπέταντο.
\VS{3}Καὶ ἐκέκραγεν ἕτερος πρὸς τὸν ἕτερον, καὶ ἔλεγον, ἅγιος ἅγιος ἅγιος Κύριος σαβαὼθ, πλήρης πᾶσα ἡ γῆ τῆς δόξης αὐτοῦ.
\par }{\PP \VS{4}Καὶ ἐπῄρθη τὸ ὑπέρθυρον ἀπὸ τῆς φωνῆς ἧς ἐκέκραγον, καὶ ὁ οἶκος ἐνεπλήσθη καπνοῦ.
\VS{5}Καὶ εἶπον, ὢ τάλας ἐγώ, ὅτι κατανένυγμαι, ὅτι ἄνθρωπος ὢν, καὶ ἀκάθαρτα χείλη ἔχων, ἐν μέσῳ λαοῦ ἀκάθαρτα χείλη ἔχοντος ἐγὼ οἰκῶ, καὶ τὸν βασιλέα Κύριον σαβαὼθ εἶδον τοῖς ὀφθαλμοῖς μου.
\VS{6}Καὶ ἀπεστάλη πρὸς μὲ ἓν τῶν σεραφὶμ, καὶ ἐν τῇ χειρὶ εἶχεν ἄνθρακα, ὃν τῇ λαβίδι ἔλαβεν ἀπὸ τοῦ θυσιαστηρίου,
\VS{7}καὶ ἥψατο τοῦ στόματός μου, καὶ εἶπεν, ἰδοὺ ἥψατο τοῦτο τῶν χειλέων σου, καὶ ἀφελεῖ τὰς ἀνομίας σου, καὶ τὰς ἁμαρτίας σου περικαθαριεῖ.
\par }{\PP \VS{8}Καὶ ἤκουσα τῆς φωνῆς Κυρίου λέγοντος, τίνα ἀποστείλω, καὶ τίς πορεύσεται πρὸς τὸν λαὸν τοῦτον; καὶ εἶπα, ἰδοὺ ἐγώ εἰμι· ἀπόστειλόν με.
\VS{9}Καὶ εἶπε, πορεύθητι, καὶ εἰπὸν τῷ λαῷ τούτῳ, ἀκοῇ ἀκούσετε, καὶ οὐ μὴ συνῆτε, καὶ βλέποντες βλέψετε, καὶ οὐ μὴ ἴδητε.
\VS{10}Ἐπαχύνθη γὰρ ἡ καρδία τοῦ λαοῦ τούτου, καὶ τοῖς ὠσὶν αὐτῶν βαρέως ἤκουσαν, καὶ τοὺς ὀφθαλμοὺς ἐκάμμυσαν· μήποτε ἴδωσι τοῖς ὀφθαλμοῖς, καὶ τοῖς ὠσὶν ἀκούσωσι, καὶ τῇ καρδίᾳ συνῶσι καὶ ἐπιστρέψωσι, καὶ ἰάσομαι αὐτούς.
\VS{11}Καὶ εἶπα, ἕως πότε Κύριε; καὶ εἶπεν, ἕως ἂν ἐρημωθῶσι πόλεις παρὰ τὸ μὴ κατοικεῖσθαι, καὶ οἶκοι παρὰ τὸ μὴ εἶναι ἀνθρώπους, καὶ ἡ γῆ καταλειφθήσεται ἔρημος.
\VS{12}Καὶ μετὰ ταῦτα μακρυνεῖ ὁ Θεὸς τοὺς ἀνθρώπους, καὶ πληθυνθήσονται οἱ ἐγκαταλειφθέντες ἐπὶ τῆς γῆς·
\VS{13}καὶ ἔτι ἐπʼ αὐτῆς ἐστι τὸ ἐπιδέκατον, καὶ πάλιν ἔσται εἰς προνομὴν ὡς τερέβινθος, καὶ ὡς βάλανος ὅταν ἐκπέσῃ ἐκ τῆς θήκης αὐτῆς.

\par }\Chap{7}{\PP \VerseOne{1}Καὶ ἐγένετο ἐν ταῖς ἡμέραις Ἄχαζ τοῦ Ἰωάθαμ τοῦ υἱοῦ Ὀζίου βασιλέως Ἰούδα, ἀνέβη Ῥασὶν βασιλεὺς Ἀραμ, καὶ Φακεὲ υἱὸς Ῥομελίου βασιλεὺς Ἰσραὴλ ἐπὶ Ἱερουσαλὴμ πολεμῆσαι αὐτὴν, καὶ οὐκ ἠδυνήθησαν πολιορκῆσαι αὐτήν.
\VS{2}Καὶ ἀνηγγέλη εἰς τὸν οἶκον Δαυὶδ, λέγων, συνεφώνησεν Ἀρὰμ πρὸς τὸν Ἐφραίμ· καὶ ἐξέστη ἡ ψυχὴ αὐτοῦ, καὶ ἡ ψυχὴ τοῦ λαοῦ αὐτοῦ, ὃν τρόπον ἐν δρυμῷ ξύλον ὑπὸ πνεύματος σαλευθῇ·
\VS{3}Καὶ εἶπε Κύριος πρὸς Ἡσαΐαν, ἔξελθε εἰς συνάντησιν Ἄχαζ σὺ, καὶ ὁ καταλειφθεὶς Ἰασοὺβ ὁ υἱός σου, πρὸς τὴν κολυμβήθραν τῆς ἄνω ὁδοῦ ἀγροῦ τοῦ κναφέως.
\VS{4}Καὶ ἐρεῖς αὐτῷ, φύλαξαι τοῦ ἡσυχάσαι, καὶ μὴ φοβοῦ, μηδὲ ἡ ψυχή σου ἀσθενείτω ἀπὸ τῶν δύο ξύλων τῶν δαλῶν τῶν καπνιζομένων τούτων· ὅταν γὰρ ὀργὴ τοῦ θυμοῦ μου γένηται, πάλιν ἰάσομαι.
\VS{5}Καὶ ὁ υἱὸς τοῦ Ἀρὰμ, καὶ ὁ υἱὸς τοῦ Ῥομελίου, ὅτι ἐβουλεύσαντο βουλὴν πονηράν·
\VS{6}Ἀναβησόμεθα εἰς τὴν Ἰουδαίαν, καὶ συλλαλήσαντες αὐτοῖς, ἀποστρέψομεν αὐτοὺς πρὸς ἡμᾶς, καὶ βασιλεύσομεν αὐτῆς τὸν υἱὸν Ταβεήλ·
\VS{7}Τάδε λέγει Κύριος σαβαὼθ, οὐ μὴ μείνῃ ἡ βουλὴ αὕτη, οὐδὲ ἔσται,
\VS{8}ἀλλʼ ἡ κεφαλὴ Ἀρὰμ, Δαμασκὸς, καὶ ἡ κεφαλὴ Δαμασκοῦ, Ῥασίμ· ἀλλʼ ἔτι ἐξήκοντα καὶ πέντε ἐτῶν ἐκλείψει ἡ βασιλεία Ἐφραίμ ἀπὸ λαοῦ,
\VS{9}καὶ ἡ κεφαλὴ Ἐφραὶμ Σομόρων, καὶ ἡ κεφαλὴ Σομορων, υἱὸς τοῦ Ῥομελίου, καὶ ἐὰν μὴ πιστεύσητε, οὐδὲ μὴ συνῆτε.
\par }{\PP \VS{10}Καὶ προσέθετο Κύριος λαλῆσαι τῷ Ἄχαζ, λέγων,
\VS{11}αἴτησαι σεαυτῷ σημεῖον παρὰ Κυρίου Θεοῦ σου εἰς βάθος, ἢ εἰς ὕψος.
\VS{12}Καὶ εἶπεν Ἄχαζ, οὐ μὴ αἰτήσω, οὐδὲ μὴ πειράσω Κύριον.
\VS{13}Καὶ εἶπεν, ἀκούσατε δὴ οἶκος Δαυίδ· μὴ μικρὸν ὑμῖν ἀγῶνα παρέχειν ἀνθρώποις, καὶ πῶς Κυρίῳ παρέχετε ἀγῶνα;
\VS{14}Διατοῦτο δώσει Κύριος αὐτὸς ὑμῖν σημεῖον· ἰδοὺ ἡ παρθένος ἐν γαστρὶ λήψεται, καὶ τέξεται υἱὸν, καὶ καλέσεις τὸ ὄνομα αὐτοῦ Ἐμμανουήλ.
\VS{15}Βούτυρον καὶ μέλι φάγεται πρινὴ γνῶναι αὐτὸν ἢ προελέσθαι πονηρὰ, ἐκλέξασθαι τὸ ἀγαθόν·
\VS{16}Διότι πρινὴ γνῶναι τὸ παιδίον ἀγαθὸν ἢ κακὸν, ἀπειθεῖ πονηρίᾳ, ἐκλέξασθαι τὸ ἀγαθόν· καὶ καταλειφθήσεται ἡ γῆ ἣν σὺ φοβῇ, ἀπὸ προσώπου τῶν δύο βασιλέων.
\par }{\PP \VS{17}Ἀλλὰ ἐπάξει ὁ Θεὸς ἐπὶ σὲ καὶ ἐπὶ τὸν λαόν σου καὶ ἐπὶ τὸν οἶκον τοῦ πατρός σου ἡμέρας, αἳ οὔπω ἥκασιν ἀφʼ ἧς ἡμέρας ἀφεῖλεν Ἐφραὶμ ἀπὸ Ἰούδα τὸν βασιλέα τῶν Ἀσσυρίων.
\VS{18}Καὶ ἔσται ἐν τῇ ἡμέρᾳ ἐκείνῃ συριεῖ Κύριος μυίαις, ὃ κυριεύσει μέρος ποταμοῦ Αἰγύπτου, καὶ τῇ μελίσσῃ, ἥ ἐστιν ἐν χώρᾳ Ἀσσυρίων·
\VS{19}Καὶ ἐλεύσονται πάντες ἐν ταῖς φάραγξι τῆς χώρας, καὶ ἐν ταῖς τρώγλαις τῶν πετρῶν, καὶ εἰς τὰ σπήλαια, καὶ εἰς πᾶσαν ῥαγάδα.
\VS{20}Ἐν τῇ ἡμέρᾳ ἐκείνῃ ξυρήσει Κύριος ἐν τῷ ξυρῷ τῷ μεμισθωμένῳ πέραν τοῦ ποταμοῦ βασιλέως Ἀσσυρίων τὴν κεφαλὴν, καὶ τὰς τρίχας τῶν ποδῶν, καὶ τὸν πώγωνα ἀφελεῖ.
\VS{21}Καὶ ἔσται ἐν τῇ ἡμέρᾳ ἐκείνῃ θρέψει ἄνθρωπος δάμαλιν βοῶν, καὶ δύο πρόβατα·
\VS{22}Καὶ ἔσται ἀπὸ τοῦ πλεῖστον πιεῖν γάλα, βούτυρον καὶ μέλι φάγεται πᾶς ὁ καταλειφθεὶς ἐπὶ τῆς γῆς.
\par }{\PP \VS{23}Καὶ ἔσται ἐν τῇ ἡμέρᾳ ἐκείνῃ, πᾶς τόπος οὗ ἐὰν ὦσι χίλιαι ἄμπελοι χιλίων σίκλων, εἰς χέρσον ἔσονται, καὶ εἰς ἄκανθαν.
\VS{24}Μετὰ βέλους καὶ τοξεύματος εἰσελεύσονται ἐκεῖ· ὅτι χέρσος καὶ ἄκανθα ἔσται πᾶσα ἡ γῆ,
\VS{25}καὶ πᾶν ὄρος ἠροτριωμένον ἀροτριωθήσεται· οὐ μὴ ἐπέλθῃ ἐκεῖ φόβος· ἔσται γὰρ ἀπὸ τῆς χέρσου καὶ ἀκάνθης εἰς βόσκημα προβάτου, καὶ καταπάτημα βοός.

\par }\Chap{8}{\PP \VerseOne{1}Καὶ εἶπε Κύριος πρὸς μὲ, λάβε σεαυτῷ τόμον καινοῦ μεγαλοῦ, καὶ γράψον εἰς αὐτὸν γραφίδι ἀνθρώπου, τοῦ ὀξέως προνομὴν ποιῆσαι σκύλων·
\VS{2}Πάρεστι γάρ· καὶ μάρτυράς μοι ποίησον πιστοὺς ἀνθρώπους, τὸν Οὐρίαν καὶ Ζαχαρίαν υἱὸν Βαραχίου.
\VS{3}Καὶ προσῆλθον πρὸς τὴν προφῆτιν, καὶ ἐν γαστρὶ ἔλαβε, καὶ ἔτεκεν υἱόν· καὶ εἶπε Κύριός μοι, κάλεσον τὸ ὄνομα αὐτοῦ, Ταχέως σκύλευσον, ὀξέως προνόμευσον·
\VS{4}Διότι πρινὴ γνῶναι τὸ παιδίον καλεῖν πατέρα ἢ μητέρα, λήψεται δύναμιν Δαμασκοῦ, καὶ τὰ σκῦλα Σαμαρείας ἔναντι βασιλέως Ἀσσυρίων.
\par }{\PP \VS{5}Καὶ προσέθετο Κύριος λαλῆσαί μοι ἔτι·
\VS{6}Διὰ τὸ μὴ βούλεσθαι τὸν λαὸν τοῦτον τὸ ὕδωρ τοῦ Σιλωὰμ τὸ πορευόμενον ἡσυχῆ, ἀλλὰ βούλεσθαι ἔχειν τὸν Ῥασσὶν καὶ τὸν υἱὸν Ῥομελίου βασιλέα ἐφʼ ὑμῶν,
\VS{7}διατοῦτο ἰδοὺ Κύριος ἀνάγει ἐφʼ ὑμᾶς τὸ ὕδωρ τοῦ ποταμοῦ, τὸ ἰσχυρὸν καὶ τὸ πολὺ, τὸν βασιλέα τῶν Ἀσσυρίων, καὶ τὴν δόξαν αὐτοῦ· καὶ ἀναβήσεται ἐπὶ πᾶσαν φάραγγα ὑμῶν, καὶ περιπατήσει ἐπὶ πᾶν τεῖχος ὑμῶν,
\VS{8}καὶ ἀφελεῖ ἀπὸ τῆς Ιουδαίας ἄνθρωπον, ὃς δυνήσεται κεφαλὴν ἆραι, ἢ δυνατὸν συντελέσασθαί τι· καὶ ἔσται ἡ παρεμβολὴ αὐτοῦ ὥστε πληρῶσαι τὸ πλάτος τῆς χώρας σου, μεθʼ ἡμῶν ὁ Θεός.
\par }{\PP \VS{9}Γνῶτε ἔθνη καὶ ἡττᾶσθε, ἐπακούσατε ἕως ἐσχάτου τῆς γῆς· ἰσχυκότες ἡττᾶσθε· ἐὰν γὰρ πάλιν ἰσχύσητε, πάλιν ἡττηθήσεσθε.
\VS{10}Καὶ ἣν ἂν βουλεύσησθε βουλὴν, διασκεδάσει Κύριος· καὶ λόγον ὃν ἐὰν λαλήσητε, οὐ μὴ ἐμμείνῃ ἐν ὑμῖν, ὅτι μεθʼ ἡμῶν ὁ Θεός.
\VS{11}Οὕτω λέγει Κύριος, Τῇ ἰσχυρᾷ χειρὶ ἀπειθοῦσι τῇ πορεὶᾳ τῆς ὁδοῦ τοῦ λαοῦ τούτου, λέγοντες,
\VS{12}μήποτε εἴπωσι, σκληρόν· πᾶν γὰρ ὃ ἐὰν εἴπῃ ὁ λαὸς οὕτος, σκληρόν ἐστι· τὸν δὲ φόβον αὐτοῦ οὐ μὴ φοβηθῆτε οὐδὲ μὴ ταραχθῆτε.
\VS{13}Κύριον αὐτὸν ἁγιάσατε, καὶ αὐτὸς ἔσται σου φόβος.
\VS{14}κᾂν ἐπʼ αὐτῷ πεποιθὼς ᾖς, ἔσται σοι εἰς ἁγίασμα, καὶ οὐχ ὡς λίθου προσκόμματι συναντήσεσθε, οὐδὲ ὡς πέτρας πτώματι· οἱ δὲ οἶκοι Ἰακὼβ ἐν παγίδι, καὶ ἐν κοιλάσματι ἐγκαθήμενοι ἐν Ἱερουσαλήμ.
\VS{15}Διατοῦτο ἀδυνατήσουσιν ἐν αὐτοῖς πολλοὶ, καὶ πεσοῦνται καὶ συντριβήσονται, καὶ ἐγγιοῦσι, καὶ ἁλώσονται ἄνθρωποι ἐν ἀσφαλείᾳ.
\VS{16}Τότε φανεροὶ ἔσονται οἱ σφραγιζόμενοι τὸν νόμον τοῦ μὴ μαθεῖν.
\par }{\PP \VS{17}Καὶ ἐρεῖ, μενῶ τὸν Θεὸν τὸν ἀποστρέψαντα τὸ πρόσωπον αὐτοῦ ἀπὸ τοῦ οἴκου Ἰακὼβ, καὶ πεποιθὼς ἔσομαι ἐπʼ αὐτῷ.
\VS{18}Ἰδοὺ ἐγὼ καὶ τὰ παιδία ἅ μοι ἔδωκεν ὁ Θεός· καὶ ἔσται σημεῖα καὶ τέρατα ἐν τῷ οἴκῳ Ἰσραὴλ παρὰ Κυρίου σαβαὼθ, ὃς κατοικεῖ ἐν τῷ ὄρει Σιών.
\par }{\PP \VS{19}Καὶ ἐὰν εἴπωσι πρὸς ὑμᾶς, ζητήσατε τοὺς ἐγγαστριμύθους, καὶ τοὺς ἀπὸ τῆς γῆς φωνοῦντας, τοὺς κενολογοῦντας, οἳ ἐκ τῆς κοιλίας φωνοῦσιν· οὐκ ἔθνος πρὸς Θεὸν αὐτοῦ ἐκζητήσουσι; τί ἐκζητοῦσι περὶ τῶν ζώντων τοὺς νεκρούς;
\VS{20}Νόμον γὰρ εἰς βοήθειαν ἔδωκεν, ἵνα εἴπωσιν οὐχ ὡς τὸ ῥῆμα τοῦτο, περὶ οὗ οὐκ ἔστι δῶρα δοῦναι περὶ αὐτοῦ.
\par }{\PP \VS{21}Καὶ ἥξει ἐφʼ ὑμᾶς σκληρὰ λιμὸς, καὶ ἔσται ὡς ἂν πεινάσητε, λυπηθήσεσθε. καὶ κακῶς ἐρεῖτε τὸν ἄρχοντα καὶ τὰ πάτρια· καὶ ἀναβλέψονται εἰς τὸν οὐρανὸν ἄνω,
\VS{22}καὶ εἰς τὴν γῆν κάτω ἐμβλέψονται· καὶ ἰδοὺ ἀπορία στενὴ, καὶ σκότος, θλίψις, καὶ στενοχωρία, καὶ σκότος ὥστε μὴ βλέπειν· καὶ οὐκ ἀπορηθήσεται ὁ ἐν στενοχωρίᾳ ὢν ἕως καιροῦ.
\par }{\PP \VS{23}Τοῦτο πρῶτον πίε· ταχὺ ποίει χώρα Ζαβυλὼν, ἡ γῆ Νεφθαλεὶμ, καὶ οἱ λοιποὶ οἱ τὴν παραλίαν, καὶ πέραν τοῦ Ἰορδάνου Γαλιλαία τῶν ἐθνῶν.

\par }\Chap{9}{\PP \VerseOne{1}Ὁ λαὸς ὁ πορευόμενος ἐν σκότει, ἴδετε φῶς μέγα· οἱ κατοικοῦντες ἐν χώρᾳ σκιᾷ θανάτου, φῶς λάμψει ἐφʼ ὑμᾶς.
\VS{2}Τὸ πλεῖστον τοῦ λαοῦ, ὃ κατήγαγες ἐν εὐφροσύνῃ σου· καὶ εὐφρανθήσονται ἐνώπιόν σου, ὡς οἱ εὐφραινόμενοι ἐν ἀμήτῳ, καὶ ὃν τρόπον οἱ διαιρούμενοι σκῦλα.
\VS{3}Διότι ἀφῄρηται ὁ ζυγὸς ὁ ἐπʼ αὐτῶν κείμενος, καὶ ἡ ῥάβδος ἡ ἐπὶ τοῦ τραχήλου αὐτῶν· τὴν γὰρ ῥάβδον τῶν ἀπαιτούντων διεσκέδασεν, ὡς τῇ ἡμέρᾳ τῇ ἐπὶ Μαδιάμ.
\VS{4}Ὅτι πᾶσαν στολὴν ἐπισυνήγμένην δόλῳ, καὶ ἱμάτιον μετὰ καταλλαγῆς ἀποτίσουσι· καὶ θελήσουσιν, εἰ ἐγένοντο πυρίκαυστοι.
\par }{\PP \VS{5}Ὅτι παιδίον ἐγεννήθη ἡμῖν, υἱὸς καὶ ἐδόθη ἡμῖν, οὗ ἡ ἀρχὴ ἐγενήθη ἐπὶ τοῦ ὤμου αὐτοῦ, καὶ καλεῖται τὸ ὄνομα αὐτοῦ, Μεγάλης βουλῆς ἄγγελος· ἄξω γὰρ εἰρήνην ἐπὶ τοὺς ἄρχοντας, καὶ ὑγίειαν αὐτῷ.
\VS{6}Μεγάλη ἡ ἀρχὴ αὐτοῦ, καὶ τῆς εἰρήνης αὐτοῦ οὐκ ἔστιν ὅριον· ἐπὶ τὸν θρόνον Δαυεὶδ, καὶ τὴν βασιλείαν αὐτοῦ, κατορθῶσαι αὐτὴν, καὶ ἀντιλαβέσθαι ἐν κρίματι καὶ ἐν δικαιοσύνῃ, ἀπὸ τοῦ νῦν καὶ εἰς τὸν αἰῶνα· ὁ ζῆλος Κυρίου σαβαὼθ ποιήσει ταῦτα.
\par }{\PP \VS{7}Θάνατον ἀπέστειλε Κύριος ἐπὶ Ἰακώβ, καὶ ἦλθεν ἐπὶ Ἰσραήλ.
\VS{8}Καὶ γνώσονται πᾶς ὁ λαὸς τοῦ Ἐφραὶμ, καὶ οἱ καθήμενοι ἐν Σαμαρείᾳ, ἐφʼ ὕβρει καὶ ὑψηλῇ καρδίᾳ λέγοντες,
\VS{9}πλίνθοι πεπτώκασιν, ἀλλὰ δεῦτε λαξεύσωμεν λίθους, καὶ κόψωμεν συκαμίνους καὶ κέδρους, καὶ οἰκοδομήσωμεν ἑαυτοῖς πύργον.
\VS{10}Καὶ ῥάξει ὁ Θεὸς τοὺς ἐπανισταμένους ἐπὶ ὄρος Σιὼν ἐπὶ αὐτὸν, καὶ τοὺς ἐχθροὺς διασκεδάσει·
\VS{11}Συρίαν ἀφʼ ἡλίου ἀνατολῶν, καὶ τοὺς Ἕλληνας ἀφʼ ἡλίου δυσμῶν, τοὺς κατεσθίοντας τὸν Ἰσραὴλ ὅλῳ τῷ στόματι· ἐπὶ πᾶσι τούτοις οὐκ ἀπεστράφη ὁ θυμὸς, ἀλλʼ ἔτι ἡ χεὶρ ὑψηλή.
\par }{\PP \VS{12}Καὶ ὁ λαὸς οὐκ ἀπεστράφη, ἕως ἐπλήγη, καὶ τὸν κύριον οὐκ ἐζήτησαν.
\VS{13}Καὶ ἀφεῖλε Κύριος ἀπὸ Ἰσραὴλ κεφαλὴν καὶ οὐρὰν, μέγαν καὶ μικρὸν, ἐν μιᾷ ἡμέρᾳ, πρεσβύτην, καὶ τοὺς τὰ πρόσωπα θαυμάζοντας, αὕτη ἡ ἀρχή·
\VS{14}καὶ προφήτην διδάσκοντα ἄνομα, οὗτος ἡ οὐρά.
\VS{15}Καὶ ἔσονται οἱ μακαρίζοντες τὸν λαὸν τοῦτον πλανῶντες, καὶ πλανῶσιν, ὅπως καταπίνωσιν αὐτούς.
\VS{16}Διατοῦτο ἐπὶ τοὺς νεανίσκους αὐτῶν οὐκ εὐφρανθήσεται ὁ Κύριος, καὶ τοὺς ὀρφανοὺς αὐτῶν καὶ τὰς χήρας αὐτῶν οὐκ ἐλεήσει· ὅτι πάντε ἄνομοι καὶ πονηροί, καὶ πᾶν στόμα λαλεῖ ἄδικα· ἐπὶ πᾶσι τούτοις οὐκ ἀπεστράφη ὁ θυμός, ἀλλʼ ἔτι ἡ χεὶρ ὑψηλή.
\par }{\PP \VS{17}Καὶ καυθήσεται ὡς πῦρ ἡ ἀνομία, καὶ ὡς ἄγρωστις ξηρὰ βρωθήσεται ὑπὸ πυρός· καὶ καυθήσεται ἐν τοῖς δάσεσι τοῦ δρυμοῦ, καὶ συγκαταφάγεται τὰ κύκλῳ τῶν βουνῶν πάντα.
\VS{18}Διὰ θυμὸν ὀργῆς Κυρίου συγκέκαυται ἡ γῆ ὅλη· καὶ ἔσται ὁ λαὸς ὡς κατακεκαυμένος ὑπὸ πυρὸς· ἄνθρωπος τὸν ἀδελφὸν αὐτοῦ οὐκ ἐλεήσει.
\VS{19}Ἀλλὰ ἐκκλινεῖ εἰς τὰ δεξιά, ὅτι πεινάσει, καὶ φάγεται ἐκ τῶν ἀριστερῶν, καὶ οὐ μὴ ἐμπλησθῇ ἄνθρωπος ἔσθων τὰς σάρκας τοῦ βραχίονος αὐτοῦ.
\VS{20}Φάγεται γὰρ Μανασσῆς τοῦ Ἐφραὶμ, καὶ Ἐφραὶμ τοῦ Μανασσῆ, ὅτι ἅμα πολιορκήσουσι τὸν Ἰούδαν. ἐπὶ τούτοις πᾶσιν οὐκ ἀπεστράφη ὁ θυμός, ἀλλʼ ἔτι ἡ χεὶρ ὑψηλή.

\par }\Chap{10}{\PP \VerseOne{1}Οὐαὶ τοῖς γράφουσι πονηρίαν, γράφοντες γὰρ, πονηρίαν γράφουσιν·
\VS{2}Ἐκκλίνοντες κρίσιν πτωχῶν, ἁρπάζοντες κρίμα πενήτων τοῦ λαοῦ μου, ὥστε εἶναι αὐτοῖς χήραν εἰς διαρπαγὴν, καὶ ὀρφανὸν εἰς προνομήν.
\VS{3}Καὶ τί ποιήσουσιν ἐν τῇ ἡμέρᾳ τῆς ἐπισκοπῆς; ἡ γὰρ θλίψις ὑμῖν πόῤῥωθεν ἥξει· καὶ πρὸς τίνα καταφεύξεσθε τοῦ βοηθηθῆναι; καὶ ποῦ καταλείψετε τὴν δόξαν ὑμῶν, τοῦ μὴ ἐμπεσεῖν εἰς ἀπαγωγήν;
\par }{\PP \VS{4}Ἐπὶ πᾶσι τούτοις οὐκ ἀπεστράφη ἡ ὀργή, ἀλλʼ ἔτι ἡ χεὶρ ὑψηλή.
\par }{\PP \VS{5}Οὐαὶ Ἀσσυρίοις, ἡ ῥάβδος τοῦ θυμοῦ μου, καὶ ὀργή ἐστιν ἐν ταῖς χερσὶν αὐτῶν.
\VS{6}Τὴν ὀργήν μου εἰς ἔθνος ἄνομον ἀποστελῶ, καὶ τῷ ἐμῷ λαῷ συντάξω ποιῆσαι σκῦλα καὶ προνομήν, καὶ καταπατεῖν τὰς πόλεις, καὶ θεῖναι αὐτὰς εἰς κονιορτόν.
\VS{7}Αὐτὸς δὲ οὐχ οὕτως ἐνεθυμήθη, καὶ τῇ ψυχῇ οὐχ οὕτως λελόγισται, ἀλλὰ ἀπαλλάξει ὁ νοῦς αὐτοῦ, καὶ τοῦ ἔθνη ἐξολοθρεῦσαι οὐκ ὀλίγα.
\VS{8}Καὶ ἐὰν εἴπωσιν αὐτῷ, σὺ μόνος εἶ ἄρχων·
\VS{9}Καὶ ἐρεῖ, οὐκ ἔλαβον τὴν χώραν τὴν ἐπάνω Βαβυλῶνος καὶ Χαλάνης, οὗ ὁ πύργος ᾠκοδομήθη, καὶ ἔλαβον Ἀραβίαν καὶ Δαμασκὸν καὶ Σαμάρειαν;
\VS{10}Ὃν τρόπον ταύτας ἔλαβον, καὶ πάσας τὰς ἀρχὰς λήψομαι· ὀλολύξατε τὰ γλυπτὰ ἐν Ἱερουσαλὴμ, καὶ ἐν Σαμαρείᾳ.
\VS{11}Ὃν τρόπον γὰρ ἐποίησα Σαμαρείᾳ, καὶ τοῖς χειροποιήτοις αὐτῆς, οὕτω ποιήσω καὶ Ἱερουσαλὴμ, καὶ τοῖς εἰδώλοις αὐτῆς.
\VS{12}Καὶ ἔσται ὅταν συντελέσῃ Κύριος πάντα ποιῶν ἐν τῷ ὄρει Σιὼν καὶ Ἱερουσαλήμ, ἐπισκέψομαι ἐπὶ τὸν νοῦν τὸν μέγαν ἐπὶ τὸν ἄρχοντα τῶν Ἀσσυρίων, καὶ ἐπὶ τὸ ὕψος τῆς δόξης τῶν ὀφθαλμῶν αὐτοῦ.
\VS{13}Εἶπε γὰρ, ἐν τῇ ἰσχύϊ ποιήσω, καὶ ἐν τῇ σοφίᾳ τῆς συνέσεως ἀφελῶ ὅρια ἐθνῶν, καὶ τὴν ἰσχὺν αὐτῶν προνομεύσω·
\VS{14}Καὶ σείσω πόλεις κατοικουμένας, καὶ τὴν οἰκουμένην ὅλην καταλήψομαι τῇ χειρὶ ὡς νοσσιὰν, καὶ ὡς καταλελειμμένα ὠὰ ἀρῶ· καὶ οὐκ ἔστιν ὃς διαφεύξεταί με, ἢ ἀντείπῃ μοι.
\VS{15}Μὴ δοξασθήσεται ἀξίνη ἄνευ τοῦ κόπτοντος ἐν αὐτῇ; ἢ ὑψωθήσεται πριὼν ἄνευ τοῦ ἔλκοντος αὐτόν; ὡς ἄν τις ἄρῃ ῥάβδον ἢ ξύλον· καὶ οὐχ οὕτως,
\VS{16}ἀλλὰ ἀποστελεῖ Κύριος σαβαὼθ εἰς τὴν σὴν τιμὴν ἀτιμίαν, καὶ εἰς τὴν σὴν δόξαν πῦρ καιόμενον καυθήσεται.
\VS{17}Καὶ ἔσται τὸ φῶς τοῦ Ἰσρὴλ εἰς πῦρ, καὶ ἁγιάσει αὐτὸν ἐν πυρὶ καιομένῳ, καὶ φάγεται ὡσεὶ χόρτον τὴν ὕλην·
\VS{18}τῇ ἡμέρᾳ ἐκείνῃ ἀποσβεσθήσεται τὰ ὄρη, καὶ οἱ βουνοὶ, καὶ οἱ δρυμοὶ, καὶ καταφάγεται ἀπὸ ψυχὴς ἕως σαρκῶν· καὶ ἔσται ὁ φεύγων, ὡς ὁ φεύγων ἀπὸ φλογὸς καιομένης.
\VS{19}Καὶ οἱ καταλειφθέντες ἀπʼ αὐτῶν ἀριθμὸς ἔσονται, καὶ παιδίον γράψει αὐτούς.
\par }{\PP \VS{20}Καὶ ἔσται ἐν τῇ ἡμέρᾳ ἐκείνῃ, οὐκέτι προστεθήσεται τὸ καταλειφθὲν Ἰσραὴλ, καὶ οἱ σωθέντες τοῦ Ἰακὼβ οὐκέτι μὴ πεποιθότες ὦσιν ἐπὶ τοὺς ἀδικήσαντας αὐτοὺς, ἀλλὰ ἔσονται πεποιθότες ἐπὶ τὸν Θεὸν τὸν ἅγιον τοῦ Ἰσραὴλ τῇ ἀληθείᾳ.
\VS{21}Καὶ ἔσται τὸ καταλειφθὲν τοῦ Ἰακὼβ ἐπὶ Θεὸν ἰσχύοντα.
\VS{22}Καὶ ἐὰν γένηται ὁ λαὸς Ἰσραὴλ ὡς ἡ ἄμμος τῆς θαλάσσης, τὸ κατάλειμμα αὐτῶν σωθήσεται.
\VS{23}Λόγον συντελῶν καὶ συντέμνων ἐν δικαιοσύνῃ, ὅτι λόγον συντετμημένον Κύριος ποιήσει ἐν τῇ οἰκουμένῃ ὅλῃ.
\par }{\PP \VS{24}Διατοῦτο τάδε λέγει Κύριος σαβαὼθ, μὴ φοβοῦ ὁ λαός μου, οἱ κατοικοῦντες ἐν Σιὼν, ἀπὸ Ἀσσυρίων, ὅτι ἐν ῥάβδῳ πατάξει σε· πληγὴν γὰρ ἐπάγω ἐπὶ σὲ, τοῦ ἰδεῖν ὁδὸν Αἰγύπτου.
\VS{25}Ἔτι γὰρ μικρὸν, καὶ παύσεται ἡ ὀργὴ, ὁ δὲ θυμός μου ἐπὶ τὴν βουλὴν αὐτῶν.
\VS{26}Καὶ ἐγερεῖ ὁ Θεὸς ἐπʼ αὐτοὺς, κατὰ τὴν πληγὴν Μαδιὰμ ἐν τόπῳ θλίψεως· καὶ ὁ θυμὸς αὐτοῦ τῇ ὁδῷ τῇ κατὰ θάλασσαν, εἰς τὴν ὁδὸν τὴν κατʼ Αἴγυπτον.
\VS{27}Καὶ ἔσται ἐν τῇ ἡμέρᾳ ἐκείνῃ, ἀφαιρεθήσεται ὁ ζυγὸς αὐτοῦ ἀπὸ τοῦ ὤμου σου, καὶ ὁ φόβος αὐτοῦ ἀπό σου, καὶ καταφθαρήσεται ὁ ζυγὸς ἀπὸ τῶν ὤμων ὑμῶν.
\par }{\PP \VS{28}Ἥξει γὰρ εἰς τὴν πόλιν Ἀγγαί, καὶ παρελεύσεται εἰς Μαγεδὼ, καὶ ἐν Μαχμὰς θήσει τὰ σκεύη αὐτοῦ.
\VS{29}Καὶ παρελεύσεται φάραγγα, καὶ ἥξει εἰς Ἀγγαί, φόβος λήψεται Ῥαμᾶ, πόλιν Σαοὺλ, φεύξεται
\VS{30}ἡ θυγάτηρ Γαλλεὶμ, ἐπακούσεται Λαϊσα, ἐπακούσεται ἐν Ἀναθώθ.
\VS{31}Καὶ ἐξέστη Μαδεβηνὰ, καὶ οἱ κατοικοῦντες Γιββείρ.
\par }{\PP \VS{32}Παρακαλεῖτε σήμερον ἐν ὁδῷ τοῦ μεῖναι, τῇ χειρὶ παρακαλεῖτε τὸ ὄρος τὴν θυγατέρα Σιὼν, καὶ οἱ βουνοὶ οἱ ἐν Ἱερουσαλήμ.
\par }{\PP \VS{33}Ἰδοὺ ὁ δεσπότης Κύριος σαβαὼθ συνταράσσει τοὺς ἐνδόξους μετὰ ἰσχύος, καὶ οἱ ὑψηλοὶ τῇ ὕβρει συντριβήσονται, καὶ οἱ ὑψηλοὶ ταπεινωθήσονται,
\VS{34}καὶ πεσοῦνται ὑψηλοὶ μαχαίρᾳ, ὁ δὲ Λίβανος σὺν τοῖς ὑψηλοῖς πεσεῖται.

\par }\Chap{11}{\PP \VerseOne{1}Καὶ ἐξελεύσεται ῥάβδος ἐκ τῆς ῥίζης Ἰεσσαί, καὶ ἄνθος ἐκ τῆς ῥίζης ἀναβήσεται,
\VS{2}καὶ ἀναπαύσεται ἐπʼ αὐτὸν πνεῦμα τοῦ Θεοῦ, πνεῦμα σοφίας καὶ συνέσεως, πνεῦμα βουλῆς καὶ ἰσχύος, πνεύμα γνώσεως καὶ εὐσεβείας
\VS{3}ἐμπλήσει αὐτὸν, πνεῦμα φόβου Θεοῦ· οὐ κατὰ τὴν δόξαν κρινεῖ, οὐδὲ κατὰ τὴν λαλιὰν ἐλέγξει,
\VS{4}ἀλλὰ κρινεῖ ταπεινῷ κρίσιν, καὶ ἐλέγξει τοὺς ταπεινοὺς τῆς γῆς, καὶ πατάξει γῆν τῷ λόγῳ τοῦ στόματος αὐτοῦ, καὶ ἐν πνεύματι διὰ χειλέων ἀνελεῖ ἀσεβῆ.
\VS{5}Καὶ ἔσται δικαιοσύνῃ ἐζωσμένος τὴν ὀσφὺν αὐτοῦ, καὶ ἀληθείᾳ εἱλημένος τὰς πλευράς.
\par }{\PP \VS{6}Καὶ συμβοσκηθήσεται λύκος μετὰ ἀρνός, καὶ πάρδαλις συναναπαύσεται ἐρίφῳ, καὶ μοσχάριον καὶ ταῦρος καὶ λέων ἅμα βοσκηθήσονται, καὶ παιδίον μικρὸν ἄξει αὐτούς.
\VS{7}Καὶ βοῦς καὶ ἄρκος ἅμα βοσκηθήσονται, καὶ ἅμα τὰ παιδία αὐτῶν ἔσονται· καὶ λέων ὡς βοῦς φάγεται ἄχυρα.
\VS{8}Καὶ παιδίον νήπιον ἐπὶ τρωγλῶν ἀσπίδων, καὶ ἐπὶ κοίτην ἐκγόνων ἀσπίδων τὴν χεῖρα ἐπιβαλεῖ.
\VS{9}καὶ οὐ μὴ κακοποιήσουσιν, οὐδὲ μὴ δύνωνται ἀπολέσαι οὐδένα ἐπὶ τὸ ὄρος τὸ ἅγιόν μου· ὅτι ἐνεπλήσθη ἡ σύμπασα τοῦ γνῶναι τὸν Κύριον, ὡς ὕδωρ πολὺ κατακαλύψαι θαλάσσας.
\VS{10}Καὶ ἔσται ἐν τῇ ἡμέρᾳ ἐκείνῃ ἡ ῥίζα τοῦ Ἰεσσαὶ, καὶ ὁ ἀνιστάμενος ἄρχειν ἐθνῶν· ἐπʼ αὐτῷ ἔθνη ἐλπιοῦσι, καὶ ἔσται ἡ ἀνάπαυσις αὐτοῦ τιμή.
\VS{11}Καὶ ἔσται τῇ ἡμέρᾳ ἐκείνῃ, προσθήσει ὁ Κύριος τοῦ δεῖξαι τὴν χεῖρα αὐτοῦ, τοῦ ζηλῶσαι τὸ καταλειφθὲν ὑπόλοιπον τοῦ λαοῦ, ὃ ἂν καταλειφθῇ ὑπὸ τῶν Ἀσσυρίων, καὶ ἀπὸ Αἰγύπτου, καὶ ἀπὸ Βαβυλωνίας, καὶ ἀπὸ Αἰθιοπίας, καὶ ἀπὸ Ἐλαμιτῶν, καὶ ἀπὸ ἡλίου ἀνατολῶν, καὶ ἐξ Ἀραβίας.
\VS{12}Καὶ ἀρεῖ σημεῖον εἰς τὰ ἔθνη, καὶ συνάξει τοὺς ἀπολομένους Ἰσραήλ, καὶ τοὺς διεσπαρμένους Ἰούδα συνάξει ἐκ τῶν τεσσάρων πτερύγων τῆς γῆς.
\VS{13}Καὶ ἀφαιρεθήσεται ὁ ζῆλος Ἐφράιμ, καὶ οἱ ἐχθροὶ Ἰούδα ἀπολοῦνται. Ἐφραὶμ οὐ ζηλώσει Ἰούδαν, καὶ Ἰούδας οὐ θλίψει Ἐφραὶμ.
\VS{14}Καὶ πετασθήσονται ἐν πλοίοις ἀλλοφύλων· θάλασσαν ἅμα προνομεύσουσι, καὶ τοὺς ἀφʼ ἡλίου ἀνατολῶν, καὶ Ἰδουμαίαν, καὶ ἐπὶ Μωὰβ πρῶτον τὰς χεῖρας ἐπιβαλοῦσιν· οἱ δὲ υἱοὶ Ἀμμὼν πρῶτοι ὑπακούσονται.
\par }{\PP \VS{15}Καὶ ἐρημώσει Κύριος τὴν θάλασσαν Αἰγύπτου, καὶ ἐπιβαλεῖ τὴν χεῖρα αὐτοῦ ἐπὶ τὸν ποταμὸν πνεύματι βιαίῳ· καὶ πατάξει ἑπτὰ φάραγγας, ὥστε διαπορεύεσθαι αὐτὸν ἐν ὑποδήμασι.
\VS{16}Καὶ ἔσται δίοδος τῷ καταλειφθέντι μου λαῷ ἐν Αἰγύπτῳ· καὶ ἔσται τῷ Ἰσραὴλ, ὡς ἡ ἡμέρα ὅτε ἐξῆλθεν ἐκ γῆς Αἰγύπτου.

\par }\Chap{12}{\PP \VerseOne{1}Καὶ ἐρεῖς ἐν τῇ ἡμέρᾳ ἐκείνῃ, Εὐλογῶ σε, Κύριε, διότι ὠργίσθης μοι, καὶ ἀπέστρεψας τὸν θυμόν σου, καὶ ἠλέησάς με.
\VS{2}Ἰδοὺ ὁ Θεός μου σωτήρ μου, πεποιθὼς ἔσομαι ἐπʼ αὐτῷ, καὶ οὐ φοβηθήσομαι· διότι ἡ δόξα μου καὶ ἡ αἴνεσίς μου Κύριος, καὶ ἐγένετό μοι εἰς σωτηρίαν.
\VS{3}Καὶ ἀντλήσατε ὕδωρ μετʼ εὐφροσύνης ἐκ τῶν πηγῶν τοῦ σωτηρίου.
\VS{4}Καὶ ἐρεῖς ἐν τῇ ἡμέρᾳ ἐκείνῃ, ὑμνεῖτε Κύριον, βοᾶτε τὸ ὄνομα αὐτοῦ, ἀναγγείλατε ἐν τοῖς ἔθνεσι τὰ ἔνδοξα αὐτοῦ· μιμνήσκεσθε, ὅτι ὑψώθη τὸ ὄνομα αὐτοῦ.
\VS{5}Ὑμνήσατε τὸ ὄνομα Κυρίου, ὅτι ὑψηλὰ ἐποίησεν· ἀναγγείλατε ταῦτα ἐν πάσῃ τῇ γῇ.
\VS{6}Ἀγαλλιᾶσθε, καὶ εὐφραίνεσθε οἱ κατοικοῦντες Σιὼν, ὅτι ὑψώθη ὁ ἅγιος τοῦ Ἰσραὴλ ἐν μέσῳ αὐτῆς.

\par }\Chap{13}{\PP \VerseOne{1}ὍΡΑΣΙΣ ἫΝ ΕΙΔΕΝ ἩΣΑΙΑΣ ΥΙΟΣ ἈΜΩΣ ΚΑΤΑ ΒΑΒΥΛΩΝΟΣ.
\par }{\PP \VS{2}Ἐπʼ ὄρους πεδινοῦ ἄρατε σημεῖον, ὑψώσατε τὴν φωνὴν αὐτοῖς, παρακαλεῖτε τῇ χειρί, ἀνοίξατε οἱ ἄρχοντες.
\VS{3}Ἐγὼ συντάσσω, καὶ ἐγώ αὐτούς· γίγαντες ἔρχονται πληρῶσαι τὸν θυμόν μου χαίροντες ἅμα καὶ ὑβρίζοντες.
\VS{4}Φωνὴ ἐθνῶν πολλῶν ἐπὶ τῶν ὀρέων, ὁμοία ἐθνῶν πολλῶν, φωνὴ βασιλέων καὶ ἐθνῶν συνηγμένων· Κύριος σαβαὼθ ἐντέταλται ἔθνει ὁπλομάχῳ,
\VS{5}ἔρχεσθαι ἐκ γῆς πόῤῥωθεν ἀπʼ ἄκρου θεμελίου τοῦ οὐρανοῦ, Κύριος καὶ οἱ ὁπλομάχοι αὐτοῦ, καταφθεῖραι πᾶσαν τὴν οἰκουμένην.
\par }{\PP \VS{6}Ὀλολύζετε, ἐγγὺς γὰρ ἡμέρα Κυρίου, καὶ συντριβὴ παρὰ τοῦ Θεοῦ ἥξει.
\VS{7}Διάτοῦτο πᾶσα χεὶρ ἐκλυθήσεται, καὶ πᾶσα ψυχὴ ἀνθρώπου δειλιάσει.
\VS{8}Ταραχθήσονται οἱ πρέσβεις, καὶ ὠδῖνες αὐτοὺς ἕξουσιν, ὡς γυναικὸς τικτούσης· καὶ συμφοράσουσιν ἕτερος πρὸς τὸν ἕτερον, καὶ ἐκστήσονται, καὶ τὸ πρόσωπον αὐτῶν ὡς φλὸξ μεταβαλοῦσιν.
\VS{9}Ἰδοὺ γὰρ ἡμέρα Κυρίου ἔρχεται ἀνίατος, θυμοῦ καὶ ὀργῆς, θεῖναι τὴν οἰκουμένην ἔρημον, καὶ τοὺς ἁμαρτωλοὺς ἀπολέσαι ἐξ αὐτῆς.
\VS{10}Οἱ γὰρ ἀστέρες τοῦ οὐρανοῦ καὶ ὁ Ὠρίων καὶ πᾶς ὁ κόσμος τοῦ οὐρανοῦ, τὸ φῶς οὐ δώσουσι· καὶ σκοτισθήσεται τοῦ ἡλίου ἀνατέλλοντος, καὶ ἡ σελήνη οὐ δώσει τὸ φῶς αὐτῆς.
\VS{11}Καὶ ἐντελοῦμαι τῇ οἰκουμένῃ ὅλῃ κακὰ, καὶ τοῖς ἀσεβέσι τὰς ἁμαρτίας αὐτῶν· καὶ ἀπολῶ ὕβριν ἀνόμων, καὶ ὕβριν ὑπερηφάνων ταπεινώσω.
\VS{12}Καὶ ἔσονται οἱ καταλελειμμένοι ἔντιμοι μᾶλλον ἢ τὸ χρυσίον τὸ ἄπυρον· καὶ ἄνθρωπος μᾶλλον ἔντιμος ἔσται ἢ ὁ λίθος ὁ ἐν Σουφείρ.
\VS{13}Ὁ γὰρ οὐρανὸς θυμωθήσεται, καὶ ἡ γῆ σεισθήσεται ἐκ τῶν θεμελίων αὐτῆς, διὰ θυμὸν ὀργῆς Κυρίου σαβαὼθ, ἐν τῇ ἡμέρᾳ ᾗ ἂν ἐπέλθῃ ὁ θυμὸς αὐτοῦ.
\VS{14}Καὶ ἔσονται οἱ καταλελειμμένοι ὡς δορκάδιον φεῦγον, καὶ ὡς πρόβατον πλανώμενον, καὶ οὐκ ἔσται ὁ συνάγων, ὥστε ἄνθρωπον εἰς τὸν λαὸν αὐτοῦ ἀποστραφῆναι, καὶ ἄνθρωπος εἰς τὴν χώραν ἑαυτοῦ διώξεται.
\VS{15}Ὃς γὰρ ἂν ἁλῷ, ἡττηθήσεται, καὶ οἵτινες συνηγμένοι εἰσὶ, πεσοῦνται μαχαίρᾳ.
\VS{16}Καὶ τὰ τέκνα αὐτῶν ῥάξουσιν ἐνώπιον αὐτῶν, καὶ τὰς οἰκίας αὐτῶν προνομεύσουσιν, καὶ τὰς γυναῖκας αὐτῶν ἕξουσιν.
\par }{\PP \VS{17}Ἰδοὺ ἐπεγείρω ὑμῖν τοὺς Μήδους, οἳ ἀργύριον οὐ λογίζονται, οὐδὲ χρυσίου χρείαν ἔχουσι.
\VS{18}Τοξεύματα νεανίσκων συντρίψουσι, καὶ τὰ τέκνα ὑμῶν οὐ μὴ ἐλεήσωσιν, οὐδὲ ἐπὶ τοῖς τέκνοις σου φείσονται οἱ ὀφθαλμοὶ αὐτῶν.
\VS{19}Καὶ ἔσται Βαβυλὼν ἣ καλεῖται ἔνδοξος ἀπὸ βασιλέως Χαλδαίων, ὃν τρόπον κατέστρεψεν ὁ Θεὸς Σόδομα καὶ Γόμοῤῥα·
\VS{20}Οὐ κατοικηθήσεται εἰς τὸν αἰῶνα χρόνον, οὐδὲ μὴ εἰσέλθωσιν εἰς αὐτὴν διὰ πολλῶν γενεῶν, οὐδὲ μὴ διέλθωσιν αὐτὴν Ἄραβες, οὐδὲ ποιμένες οὐ μὴ ἀναπαύσονται ἐν αὐτῇ.
\VS{21}Καὶ ἀναπαύσονται ἐκεῖ θηρία, καὶ ἐμπλησθήσονται αἱ οἰκίαι ἤχου· καὶ ἀναπαύσονται ἐκεῖ σειρῆνες, καὶ δαιμόνια ἐκεῖ ὀρχήσονται,
\VS{22}καὶ ὀνοκένταυροι ἐκεῖ κατοικήσουσι, καὶ νοσσοποιήσουσιν ἐχῖνοι ἐν τοῖς οἴκοις αὐτῶν. Ταχὺ ἔρχεται καὶ οὐ χρονιεῖ.

\par }\Chap{14}{\PP \VerseOne{1}Καὶ ἐλεήσει Κύριος τὸν Ἰακὼβ, καὶ ἐκλέξεται ἔτι τὸν Ἰσραὴλ, καὶ ἀναπαύσονται ἐπὶ τῆς γῆς αὐτῶν, καὶ ὁ γειώρας προστεθήσεται πρὸς αὐτοὺς, καὶ προστεθήσεται πρὸς τὸν οἶκον Ἰακώβ·
\VS{2}Καὶ λήψονται αὐτοὺς ἔθνη, καὶ εἰσάξουσιν εἰς τὸν τόπον αὐτῶν, καὶ κατακληρονομήσουσι, καὶ πληθυνθήσονται ἐπὶ τῆς γῆς εἰς δούλους καὶ δούλας· καὶ ἔσονται αἰχμάλωτοι οἱ αἰχμαλωτεύσαντες αὐτοὺς, καὶ κυριευθήσονται οἱ κυριεύσαντες αὐτῶν.
\par }{\PP \VS{3}Καὶ ἔσται ἐν τῇ ἡμέρᾳ ἐκείνῃ, ἀναπαύσει σε Κύριος ἀπὸ τῆς ὀδύνης καὶ τοῦ θυμοῦ σου, τῆς δουλείας σου τῆς σκληρᾶς, ἧς ἐδούλευσας αὐτοῖς.
\VS{4}Καὶ λήψῃ τὸν θρῆνον τοῦτον ἐπὶ τὸν βασιλέα Βαβυλῶνος,
\par }{\PP Πῶς ἀναπέπαυται ὁ ἀπαιτῶν, καὶ ἀναπέπαυται ὁ ἐπισπουδαστής;
\VS{5}Συνέτριψε Κύριος τὸν ζυγὸν τῶν ἁμαρτωλῶν, τὸν ζυγὸν τῶν ἀρχόντων.
\VS{6}Πατάξας ἔθνος θυμῷ, πληγῇ ἀνιάτῳ, παίων ἔθνος πληγὴν θυμοῦ, ἣ οὐκ ἐφείσατο, ἀνεπαύσατο πεποιθώς.
\VS{7}Πᾶσα ἡ γῆ βοᾷ μετʼ εὐφροσύνης,
\VS{8}καὶ τὰ ξύλα τοῦ Λιβάνου εὐφράνθησαν ἐπὶ σοὶ, καὶ ἡ κέδρος τοῦ Λιβάνου, ἀφʼ οὗ σὺ κεκοίμησαι, οὐκ ἀνέβη ὁ κόπτων ἡμᾶς.
\VS{9}Ὁ ᾅδης κάτωθεν ἐπικράνθη συναντήσας σοι· συνηγέρθησάν σοι πάντες οἱ γίγαντες οἱ ἄρξαντες τῆς γῆς, οἱ ἐγείραντες ἐκ τῶν θρόνων αὐτῶν πάντας βασιλεῖς ἐθνῶ.
\VS{10}Πάντες ἀποκριθήσονται, καὶ ἐροῦσί σοι, Καὶ σὺ ἑάλως, ὥσπερ καὶ ἡμεῖς· ἐν ἡμῖν δὲ κατελογίσθης.
\VS{11}Κατέβη εἰς ᾅδου ἡ δόξα σου, ἡ πολλὴ εὐφροσύνη σου· ὑποκάτω σου στρώσουσι σῆψιν, καὶ τὸ κατακάλυμμά σου σκώληξ.
\VS{12}Πῶς ἐξέπεσεν ἐκ τοῦ οὐρανοῦ ὁ Ἑωσφόρος ὁ πρωὶ ἀνατέλλων; συνετρίβη εἰς τὴν γῆν ὁ ἀποστέλλων πρὸς πάντα τὰ ἔθνη.
\VS{13}Σὺ δὲ εἶπας ἐν τῇ διανοίᾳ σου, εἰς τὸν οὐρανὸν ἀναβήσομαι, ἐπάνω τῶν ἀστέρων τοῦ οὐρανοῦ θήσω τὸν θρόνον μου, καθιῶ ἐν ὄρει ὑψηλῷ, ἐπὶ τὰ ὄρη τὰ ὑψηλὰ τὰ πρὸς Βοῤῥᾶν,
\VS{14}ἀναβήσομαι ἐπάνω τῶν νεφῶν, ἔσομαι ὅμοιος τῷ ὑψίστῳ.
\VS{15}Νῦν δὲ εἰς ᾅδην καταβήσῃ, καὶ εἰς τὰ θεμέλια τῆς γῆς.
\VS{16}Οἱ ἰδόντες σε θαυμάσονται ἐπὶ σοὶ, καὶ ἐροῦσιν, οὗτος ὁ ἄνθρωπος ὁ παροξύνων τὴν γῆν, ὁ σείων βασιλεῖς,
\VS{17}ὁ θεὶς τὴν οἰκουμένην ὅλην ἔρημον, καὶ τὰς πόλεις αὐτοῦ καθεῖλε, τοὺς ἐν ἐπαγωγῇ οὐκ ἔλυσε.
\VS{18}Πάντες οἱ βασιλεῖς τῶν ἐθνῶν ἐκοιμήθησαν ἐν τιμῇ, ἄνθρωπος ἐν τῷ οἴκῳ αὐτοῦ.
\VS{19}Σὺ δὲ ῥιφήσῃ ἐν τοῖς ὄρεσιν, ὡς νεκρὸς ἐβδελυγμένος, μετὰ πολλῶν τεθνηκότων ἐκκεκεντημένων μαχαίραις, καταβαινόντων εἰς ᾅδου.
\VS{20}Ὃν τρόπον ἱμάτιον ἐν αἵματι πεφυρμένον οὐκ ἔσται καθαρὸν, οὕτως οὐδὲ σὺ ἔσῃ καθαρός· διότι τὴν γῆν μου ἀπώλεσας, καὶ τὸν λαόν μου ἐπέκτεινας· οὐ μὴ μείνῃς εἰς τὸν αἰῶνα χρόνον, σπέρμα πονηρόν.
\VS{21}Ἑτοίμασον τὰ τέκνα σου σφαγῆναι ταῖς ἁμαρτίαις τοῦ πατρὸς αὐτῶν, ἵνα μὴ ἀναστῶσι καὶ κληρονομήσωσι τὴν γῆν, καὶ ἐμπλήσωσι τὴν γῆν πολέμων.
\VS{22}Καὶ ἐπαναστήσομαι αὐτοῖς, λέγει Κύριος σαβαὼθ, καὶ ἀπολῶ αὐτῶν ὄνομα, καὶ κατάλειμμα, καὶ σπέρμα· τάδε λέγει Κύριος.
\VS{23}Καὶ θήσω τὴν Βαβυλωνίαν ἔρημον, ὥστε κατοικεῖν ἐχίνους, καὶ ἔσται εἰς οὐδέν· καὶ θήσω αὐτὴν πηλοῦ βάραθρον εἰς ἀπώλειαν.
\par }{\PP \VS{24}Τάδε λέγει Κύριος σαβαὼθ, ὃν τρόπον εἴρηκα, οὕτως ἔσται, καὶ ὃν τρόπον βεβούλευμαι, οὕτως μενεῖ,
\VS{25}τοῦ ἀπολέσαι τοὺς Ἀσσυρίους ἐπὶ τῆς γῆς τῆς ἐμῆς, καὶ ἐπὶ τῶν ὀρέων μου· καὶ ἔσονται εἰς καταπάτημα, καὶ ἀφαιρεθήσεται ἀπʼ αὐτῶν ὁ ζυγὸς αὐτῶν, καὶ τὸ κῦδος αὐτῶν ἀπὸ τῶν ὤμων ἀφαιρεθήσεται.
\VS{26}Αὕτη ἡ βουλὴ ἣν βεβούλευται Κύριος ἐπὶ τὴν ὅλην οἰκουμένην, καὶ αὕτη ἡ χεὶρ ἡ ὑψηλὴ ἐπὶ πάντα τὰ ἔθνη.
\VS{27}Ἃ γὰρ ὁ Θεὸς ὁ ἅγιος βεβούλευται, τίς διασκεδάσει; καὶ τὴν χεῖρα αὐτοῦ τὴν ὑψηλὴν τίς ἀποστρέψει;
\par }{\PP \VS{28}Τοῦ ἔτους οὗ ἀπέθανεν ὁ βασιλεὺς Ἄχαζ, ἐγενήθη τὸ ῥῆμα τοῦτο.
\par }{\PP \VS{29}Μὴ εὐφρανθείητε οἱ ἀλλόφυλοι πάντες, συνετρίβη γὰρ ὁ ζυγὸς τοῦ παίοντος ὑμᾶς· ἐκ γὰρ σπέρματος ὄφεως ἐξελεύσεται ἔκγονα ἀσπίδων, καὶ τὰ ἔκγονα αὐτῶν ἐξελεύσονται ὄφεις πετάμενοι.
\VS{30}Καὶ βοσκηθήσονται πτωχοὶ διʼ αὐτοῦ· πτωχοὶ δὲ ἄνθρωποι ἐπὶ εἰρήνης ἀναπαύσονται· ἀνελεῖ δὲ ἐν λιμῷ τὸ σπέρμα σου, καὶ τὸ κατάλειμμά σου ἀνελεῖ.
\VS{31}Ὀλολύξατε πύλαι πόλεων, κεκραγέτωσαν πόλεις τεταραγμέναι, οἱ ἀλλόφυλοι πάντες, ὅτι ἀπὸ βοῤῥᾶ καπνὸς ἔρχεται, καὶ οὐκ ἔστι τοῦ εἶναι.
\VS{32}Καὶ τί ἀποκριθήσονται βασιλεῖς ἐθνῶν; ὅτι Κύριος ἐθεμελίωσε Σιων, καὶ διʼ αὐτοῦ σωθήσονται οἱ ταπεινοὶ τοῦ λαοῦ.
\par }{\PP ΤΟ ῬΗΜΑ ΤΟ ΚΑΤΑ ΤΗΣ ΜΩΑΒΙΤΙΔΟΣ.

\par }\Chap{15}{\PP \VerseOne{1}Νυκτὸς ἀπολεῖται ἡ Μωαβῖτις, νυκτὸς γὰρ ἀπολεῖται τὸ τεῖχος τῆς Μωαβίτιδος.
\VS{2}Λυπεῖσθε ἐφʼ ἑαυτοὺς, ἀπολεῖται γὰρ καὶ Δηβὼν, οὗ ὁ βωμὸς ὑμῶν· ἐκεῖ ἀναβήσεσθε κλαίειν, ἐπὶ Ναβαῦ τῆς Μωαβίτιδος· ὀλολύξατε, ἐπὶ πάσης κεφαλῆς φαλάκρωμα, πάντες βραχίονες κατατετμημένοι.
\VS{3}Ἐν ταῖς πλατείαις αὐτῆς περιζώσασθε σάκκους, καὶ κόπτεσθε ἐπὶ τῶν δωμάτων αὐτῆς, καὶ ἐν ταῖς ῥύμαις αὐτῆς, πάντες ὀλολύζετε μετὰ κλαυθμοῦ.
\VS{4}Ὅτι κέκραγεν Ἐσεβὼν καὶ Ἐλεαλὴ, ἕως Ἰασσὰ ἠκούσθη ἡ φωνὴ αὐτῶν· διατοῦτο ἡ ὀσφὺς τῆς Μωαβίτιδος βοᾷ, ἡ ψυχὴ αὐτῆς γνώσεται.
\VS{5}Ἡ καρδία τῆς Μωαβείτιδος βοᾷ ἐν αὐτῇ ἕως Σηγώρ· δάμαλις γάρ ἐστι τριετής· ἐπὶ δὲ τῆς ἀναβάσεως Λουεὶθ, πρὸς σὲ κλαίοντες ἀναβήσονται τῇ ὁδῷ Ἀρωνιείμ· βοᾷ, σύντριμμα καὶ σεισμός.
\VS{6}Τὸ ὕδωρ τῆς Νεμηρεὶμ ἔρημον ἔσται, καὶ ὁ χόρτος αὐτῆς ἐκλείψει· χόρτος γὰρ χλωρὸς οὐκ ἔσται.
\VS{7}Μὴ καὶ οὕτως μέλλει σωθῇναι; ἐπάξω γὰρ ἐπὶ τὴν φάραγγα Ἄραβας, καὶ λήψονται αὐτήν.
\VS{8}Συνῆψε γὰρ ἡ βοὴ τὸ ὅριον τῆς Μωαβίτιδος τῆς Ἀγαλεὶμ, καὶ ὀλολυγμὸς αὐτῆς ἕως τοῦ φρέατος τοῦ Αἰλείμ.
\VS{9}Τὸ δὲ ὕδωρ τὸ Δειμὼν πλησθήσεται αἵματος, ἐπάξω γὰρ ἐπὶ Δειμὼν Ἄραβας, καὶ ἀρῶ τὸ σπέρμα Μωὰβ, καὶ Ἀριὴλ, καὶ τὸ κατάλοιπον Ἄδαμὰ.

\par }\Chap{16}{\PP \VerseOne{1}Ἀποστέλω ὡς ἑρπετὰ ἐπὶ τὴν γῆν· μὴ πέτρα ἔρημός ἐστι τὸ ὄρος θυγατρὸς Σιών;
\VS{2}Ἔσῃ γὰρ ὡς πετεινοῦ ἀνιπταμένου νοσσὸς ἀφῃρημένος, ἔσῃ θυγάτηρ Μωὰβ, ἔπειτα δέ Ἀρνῶν πλείονα
\VS{3}βουλεύου, ποίει τε σκέπην πένθους αὕτη διαπαντὸς, ἐν μεσημβρινῇ σκοτίᾳ φεύγουσιν, ἐξέστησαν· μὴ ἀχθῇς,
\VS{4}παροικήσουσί σοι οἱ φυγάδες Μωάβ· ἔσονται σκέπη ὑμῖν ἀπὸ προσώπου διώκοντος, ὅτι ἤρθη ἡ συμμαχία σου, καὶ ὁ ἄρχων ἀπώλετο ὁ καταπατῶν ἀπὸ τῆς γῆς.
\VS{5}Καὶ διορθωθήσεται μετʼ ἐλέους θρόνος, καὶ καθιεῖται ἐπʼ αὐτοῦ μετὰ ἀληθείας ἐν σκηνῇ Δαυὶδ, κρίνων καὶ ἐκζητῶν κρίμα καὶ σπεύδων δικαιοσύνην.
\par }{\PP \VS{6}Ἠκούσαμεν τὴν ὕβριν Μωὰβ, ὑβριστὴς σφόδρα τὴν ὑπερηφανίαν ἐξῇρα· οὐχ οὕτως ἡ μαντεία σου, οὐχ οὕτως.
\par }{\PP \VS{7}Ὀλολύξει Μωὰβ, ἐν γὰρ τῇ Μωαβίτιδι πάντες ὀλολύξουσι τοῖς κατοικοῦσι δέ Σεθ μελετήσεις, καὶ οὐκ ἐντραπήσῃ.
\VS{8}Τὰ πεδία Ἑσεβὼν πενθήσει, ἄμπελος Σεβαμά· καταπίνοντες τὰ ἔθνη, καταπατήστε τὰς ἀμπέλους αὐτῆς, ἕως Ἰαζήρ· οὐ μὴ συνάψητε, πλανήθητε τὴν ἔρημον, οἱ ἀπεσταλμένοι ἐνκατελείφθησαν, διέβησαν γὰρ πρὸς τὴν θάλασσαν.
\VS{9}Διατοῦτο κλαύσομαι ὡς τὸν κλαυθμὸν Ἰαζὴρ ἄμπελον Σεβαμά· τὰ δένδρα σου κατέβαλεν Ἐσεβὼν καὶ Ἐλεαλὴ, ὅτι ἐπὶ τῷ θερισμῷ καὶ ἐπὶ τῷ τρυγητῷ σου καταπατήσω, καὶ πάντα πεσοῦνται.
\VS{10}Καὶ ἀρθήσεται εὐφροσύνη καὶ ἀγαλλίαμα ἐκ τῶν ἀμπελώνων, καὶ ἐν τοῖς ἀμπελῶσί σου οὐ μὴ εὐφρανθήσονται, καὶ οὐ μὴ πατήσουσιν οἶνον εἰς τὰ ὑπολήνια, πέπαυται γάρ.
\VS{11}Διατοῦτο ἡ κοιλία μου ἐπὶ Μωὰβ ὡς κιθάρα ἠχήσει, καὶ τὰ ἐντός μου ὡς τεῖχος ἐνεκαίνισας.
\VS{12}Καὶ ἔσται εἰς τὸ ἐντραπῆναί σε, ὅτι ἐκοπίασε Μωὰβ ἐπὶ τοῖς βωμοῖς, καὶ εἰσελεύσεται εἰς τὰ χειροποίητα αὐτῆς, ὥστε προσεύξασθαι, καὶ οὐ μὴ δύνηται ἐξελέσθαι αὐτόν.
\par }{\PP \VS{13}Τοῦτο τὸ ῥῆμα ὃ ἐλάλησε Κύριος ἐπὶ Μωὰβ, ὁπότε ἐλάλησε.
\VS{14}Καὶ νῦν λέγῳ, ἐν τρισὶν ἔτεσιν ἐτῶν μισθωτοῦ ἀτιμασθήσεται ἡ δόξα Μωὰβ παντὶ τῷ πλούτῳ τῷ πολλῷ, καὶ καταλειφθήσεται ὀλιγοστὸς, καὶ οὐκ ἔντιμος.
\par }{\PP ΤΟ ῬΗΜΑ ΤΟ ΚΑΤΑ ΔΑΜΑΣΚΟΥ.

\par }\Chap{17}{\PP \VerseOne{1}Ἰδοὺ Δαμασκὸς ἀρθήσεται ἀπὸ πόλεων, καὶ ἔσται εἰς πτῶσιν,
\VS{2}καταλελειμμένη εἰς τὸν αἰῶνα, εἰς κοίτην ποιμνίων καὶ ἀνάπαυσιν, καὶ οὐκ ἔσται ὁ διώκων.
\VS{3}Καὶ οὐκέτι ἔσται ὀχυρὰ τοῦ καταφυγεῖν Ἐφραίμ· καὶ οὐκέτι βασιλεία ἐν Δαμασκῷ, καὶ τὸ λοιπὸν τῶν Σύρων οὐ γὰρ σὺ βελτίων εἶ τῶν υἱῶν Ἰσραὴλ, τῆς δόξης αὐτῶν· τάδε λέγει Κύριος σαβαώθ.
\VS{4}Ἔσται ἐν τῇ ἡμέρᾳ ἐκείνῃ ἔκλειψις δόξης Ἰακὼβ, καὶ τὰ πίονα τῆς δόξης αὐτοῦ σεισθήσεται.
\VS{5}Καὶ ἔσται ὃν τρόπον ἐάν τις συναγάγῃ ἀμητὸν ἑστηκότα, καὶ σπέρμα σταχύων ἀμήσῃ· καὶ ἔσται ὃν τρόπον ἐάν τις συναγάγῃ στάχυν ἐν φάραγγι στερεᾷ.
\VS{6}Καὶ καταλειφθῇ ἐν αὐτῇ καλάμη, ἢ ὡς ῥῶγες ἐλαίας δύο ἢ τρεῖς ἐπʼ ἄκρου μετεώρου, ἢ τέσσαρες ἢ πέντε ἀπὶ τῶν κλάδων αὐτῶν καταλειφθῇ· τὰδε λέγει Κύριος ὁ Θεὸς Ἰσραήλ.
\par }{\PP \VS{7}Τῇ ἡμέρᾳ ἐκείνῃ πεποιθὼς ἔσται ὁ ἄνθρωπος ἐπὶ τῷ ποιήσαντι αὐτὸν, οἱ δὲ ὀφθαλμοὶ αὐτοῦ εἰς τὸν ἅγιον τοῦ Ἰσραὴλ ἐμβλέψονται,
\VS{8}καὶ οὐ μὴ πεποιθότες ὦσιν ἐπὶ τοῖς βωμοῖς, οὐδὲ ἐπὶ τοῖς ἔργοις τῶν χειρῶν αὐτῶν, ἃ ἐποίησαν οἱ δάκτυλοι αὐτῶν, καὶ οὐκ ὄψονται τὰ δένδρα, οὐδὲ τὰ βδελύγματα αὐτῶν.
\par }{\PP \VS{9}Τῇ ἡμέρᾳ ἐκείνῃ ἔσονται αἱ πόλεις σου ἐγκαταλελειμμέναι, ὃν τρόπον κατέλιπον οἱ Ἀμοῤῥαῖοι καὶ οἱ Εὐαῖοι ἀπὸ προσώπου τῶν νἱῶν Ἰσραήλ· καὶ ἔσονται ἔρημοι,
\VS{10}διότι κατέλιπες τὸν Θεὸν τὸν σωτῆρά σου, καὶ Κυρίου τοῦ βοηθοῦ σου οὐκ ἐμνήσθης· διὰτοῦτο φυτεύσεις φύτευμα ἄπιστον, καὶ σπέρμα ἄπιστον.
\VS{11}Τῇ ἡμέρᾳ, ᾗ ἂν φυτεύσῃς, πλανηθήσῃ· τὸ δὲ πρωῒ ἐὰν σπείρῃς, ἀνθήσει εἰς ἀμητὸν ᾗ ἂν ἡμέρᾳ κληρώσῃ, καὶ ὡς πατὴρ ἀνθρώπου κληρώσῃ τοῖς υἱοῖς σου.
\par }{\PP \VS{12}Οὐαὶ πλῆθος ἐθνῶν πολλῶν· ὡς θάλασσα κυμαίνουσα, οὕτω ταραχθήσεσθε· καὶ νῶτος ἐθνῶν πολλῶν, ὡς ὕδωρ ἠχήσει.
\VS{13}Ὡς ὕδωρ πολὺ ἔθνη πολλὰ, ὡς ὕδατος πολλοῦ βίᾳ φερομένου· καὶ ἀποσκορακιεῖ αὐτὸν, καὶ πόῤῥω αὐτὸν διώξεται, ὡς χνοῦν ἀχύρου λικμώντων ἀπέναντι ἀνέμου, καὶ ὡς κονιορτὸν τροχοῦ καταιγὶς φέρουσα,
\par }{\PP \VS{14}Πρὸς ἑσπέραν καὶ ἔσται πένθος, πριν ἢ πρωῒ, καὶ οὐκ ἔσται· αὕτη ἡ μερὶς τῶν προνομευσάντων ὑμᾶς, καὶ κληρονομία τοῖς ὑμᾶς κληρονομήσασιν.

\par }\Chap{18}{\PP \VerseOne{1}Οὐαὶ γῆς πλοίων πτέρυγες, ἐπέκεινα ποταμῶν Αἰθιοπίας·
\VS{2}Ὁ ἀποστέλλων ἐν θαλάσσῃ ὅμηρα, καὶ ἐπιστολὰς βιβλίνας ἐπάνω τοῦ ὕδατος· πορεύσονται γὰρ ἄγγελοι κοῦφοι πρὸς ἔθνος μετέωρον, καὶ ξένον λαὸν καὶ χαλεπόν· τίς αὐτοῦ ἐπέκεινα; ἔθνος ἀνέλπιστον καὶ καταπεπατημένον· νῦν οἱ ποταμοὶ τῆς γῆς
\VS{3}πάντες, ὡς χώρα κατοικουμένη κατοικηθήσεται· ἡ χώρα αὐτῶν ὡσεὶ σημεῖον ἀπὸ ὄρους ἀρθῇ, ὡς σάλπιγγος φωνὴ ἀκουστὸν ἔσται.
\VS{4}Διότι οὕτως εἶπε Κύριός μοι, ἀσφάλεια ἔσται ἐν τῇ ἐμῇ πόλει, ὡς φῶς καύματος μεσημβρίας, καὶ ὡς νεφέλη δρόσου ἡμέρας ἀμητο ἔσται
\VS{5}πρὸ τοῦ θερισμοῦ, ὅταν συντελεσθῇ ἄνθος, καὶ ὄμφαξ ἐξανθήσῃ ἄνθος ὀμφακίζουσα· καὶ ἀφελεῖ τὰ βοτρύδια τὰ μικρὰ τοῖς δρεπάνοις, καὶ τὰς κληματίδας ἀφελεῖ, καὶ ἀποκόψει,
\VS{6}καὶ καταλείψει ἅμα τοῖς πετεινοῖς τοῦ οὐρανοῦ, καὶ τοῖς θηρίοις τῆς γῆς· καὶ συναχθήσεται ἐπʼ αὐτοὺς τὰ πετεινὰ τοῦ οὐρανοῦ, καὶ πάντα τὰ θηρία τῆς γῆς ἐπʼ αὐτὸν ἥξει.
\VS{7}Ἐν τῷ καιρῷ ἐκείνῳ ἀνενεχθήσεται δῶρα Κυρίῳ σαβαὼθ ἐκ λαοῦ τεθλιμμένου καὶ τετιλμένου, καὶ ἀπὸ λαοῦ μεγάλου ἀπὸ τοῦ νῦν καὶ εἰς τὸν αἰῶνα χρόνον· ἔθνος ἐλπίζον καὶ καταπεπατημένον, ὅ ἐστιν ἐν μέρει ποταμοῦ τῆς χώρας αὐτοῦ, εἰς τὸν τόπον οὗ τὸ ὄνομα Κυρίου σαβαὼθ, ὄρος Σιών.
\par }{\PP ὍΡΑΣΙΣ ΑΙΓΥΠΤΟΥ.

\par }\Chap{19}{\PP \VerseOne{1}Ἰδοὺ Κύριος κάθηται ἐπὶ νεφέλης κούφης, καὶ ἥξει εἰς Αἴγυπτον, καὶ σεισθήσεται τὰ χειροποίητα Αἰγύπτου ἀπὸ προσώπου αὐτοῦ· καὶ ἡ καρδία αὐτῶν ἡττηθήσεται ἐν αὐτοῖς.
\VS{2}Καὶ ἐπεγερθήσονται Αἰγύπτιοι ἐπʼ Αἰγυπτίους, καὶ πολεμήσει ἄνθρωπος τὸν ἀδελφὸν αὐτοῦ, καὶ ἄνθρωπος τὸν πλησίον αὐτοῦ, πόλις ἐπὶ πόλιν, καὶ νὸμος ἐπὶ νὸμον.
\VS{3}Καὶ ταραχθήσεται τὸ πνεῦμα τῶν Αἰγυπτίων ἐν αὐτοῖς, καὶ τὴν βουλὴν αὐτῶν διασκεδάσω, καὶ ἐπερωτήσουσι τοὺς θεοὺς αὐτῶν, καὶ τὰ ἀγάλματα αὐτῶν, καὶ τοὺς ἐκ τῆς γῆς φωνοῦντας, καὶ τοὺς ἐγγαστριμύθους.
\VS{4}Καὶ παραδώσω τὴν Αἴγυπτον εἰς χεῖρας ἀνθρώπων, κυρίων σκληρῶν, καὶ βασιλεῖς σκληροὶ κυριεύσουσιν αὐτῶν. τάδε λέγει Κύριος σαβαώθ.
\VS{5}Καὶ πίονται οἱ Αἰγύπτιοι ὕδωρ τὸ παρὰ θάλασσαν, ὁ δὲ ποταμὸς ἐκλείψει, καὶ ξηρανθήσεται.
\VS{6}Καὶ ἐκλείψουσιν οἱ ποταμοὶ, καὶ αἱ διώρυχες τοῦ ποταμοῦ· καὶ ξηρανθήσεται πᾶσα συναγωγὴ ὕδατος, καὶ ἐν παντὶ ἕλει καλάμου καὶ παπύρου,
\VS{7}καὶ τὸ ἄχι τὸ χλωρὸν πᾶν τὸ κύκλῳ τοῦ ποταμοῦ, καὶ πᾶν τὸ σπειρόμενον διὰ τοῦ ποταμοῦ ξηρανθήσεται ἀνεμόφθορον.
\VS{8}Καὶ στενάξουσιν οἱ ἁλιεῖς, καὶ στενάξουσι πάντες οἱ βάλλοντες ἄγκιστρον εἰς τὸν ποταμὸν, καὶ οἱ βάλλοντες σαγήνας, καὶ οἱ ἀμφιβολεῖς πενθήσουσι·
\VS{9}Καὶ αἰσχύνη λήψεται τοὺς ἐργαζομένους τὸ λίνον τὸ σχιστὸν, καὶ τοὺς ἐργαζομένους τὴν βύσσον.
\VS{10}Καὶ ἔσονται οἱ ἐργαζόμενοι αὐτὰ ἐν ὀδύνῃ, καὶ πάντες οἱ ποιοῦντες τὸν ζῦθον λυπηθήσονται, καὶ τὰς ψυχὰς πονέσουσι.
\VS{11}Καὶ μωροὶ ἔσονται οἱ ἄρχοντες Τάνεως, οἱ σοφοὶ σύμβουλοι τοῦ βασιλέως, ἡ βουλὴ σὐτῶν μωρανθήσεται· πῶς ἐρεῖτε τῷ βασιλεῖ, υἱοὶ συνετῶν ἡμεῖς, υἱοὶ βασιλέων τῶν ἐξ ἀρχῆς;
\VS{12}Ποῦ εἰσι νῦν οἱ σοφοί σου; καὶ ἀναγγειλάτωσάν σοι, καὶ εἰπάτωσαν, τί βεβούλευται Κύριος σαβαὼθ ἐπʼ Αἴγυπτον;
\VS{13}Ἐξέλιπον οἱ ἄρχοντες Τάνεως, καὶ ὑψώθησαν οἱ ἄρχοντες Μέμφεως, καὶ πλανήσουσιν Αἴγυπτον κατὰ φυλάς.
\VS{14}Κύριος γὰρ ἐκέρασεν αὐτοῖς πνεῦμα πλανήσεως, καὶ ἐπλάνησαν Αἴγυπτον ἐν πᾶσι τοῖς ἔργοις αὐτῶν, ὡς πλανᾶται ὁ μεθύων, καὶ ὁ ἐμῶν ἅμα,
\VS{15}καὶ οὐκ ἔσται τοῖς Αἰγυπτίοις ἔργον ὃ ποιήσει κεφαλὴν καὶ οὐρὰν, καὶ ἀρχὴν καὶ τέλος.
\par }{\PP \VS{16}Τῇ δὲ ἡμέρᾳ ἐκείνῃ ἔσονται οἱ Αἰγύπτιοι, ὡς γυναῖκες, ἐν φόβῳ καὶ ἐν τρόμῳ ἀπὸ προσώπου τῆς χειρὸς Κυρίου σαβαὼθ, ἣν αὐτὸς ἐπιβαλεῖ αὐτοῖς.
\VS{17}Καὶ ἔσται ἡ χώρα τῶν Ἰουδαίων τοῖς Αἰγυπτίοις εἰς φόβητρον· πᾶς ὃς ἐὰν ὀνομάσῃ αὐτὴν αὐτοῖς, φοβηθήσονται διὰ τὴν βουλὴν ἣν βεβούλευται Κύριος σαβαὼθ ἐπʼ αὐτήν.
\VS{18}Τῇ ἡμέρᾳ ἐκείνῃ ἔσονται πέντε πόλεις ἐν Αἰγύπτῳ λαλοῦσαι τῇ γλώσσῃ τῇ Χαναανίτιδι, καὶ ὀμνῦντες τῷ ὀνόματι Κυρίου σαβαώθ· πόλις ἀσεδὲκ κληθήσεται ἡ μία πόλις.
\VS{19}Τῇ ἡμερᾳ ἐκείνῃ ἔσται θυσιαστήριον τῷ Κυρίῳ ἐν χώρᾳ Αἰγυπτίων, καὶ στήλη πρὸς τὸ ὅριον αὐτῆς τῷ Κυρίῳ.
\VS{20}καὶ ἔσται εἰς σημεῖον εἰς τὸν αἰῶνα Κυρίῳ ἐν χώρᾳ Αἰγύπτου· ὅτι κεκράξονται πρὸς Κύριον διὰ τοὺς θλίβοντας αὐτούς, καὶ ἀποστελεῖ αὐτοῖς ἄνθρωπον ὃς σώσει αὐτοὺς, κρίνων σώσει αὐτούς.
\VS{21}Καὶ γνωστὸς ἔσται Κύριος τοῖς Αἰγυπτίοις· καὶ γνώσονται οἱ Αἰγύπτιοι τὸν Κύριον ἐν τῇ ἡμέρᾳ ἐκείνῃ, καὶ ποιήσουσι θυσίας, καὶ εὔξονται εὐχὰς τῷ Κυρίῳ, καὶ ἀποδώσουσι.
\VS{22}Καὶ πατάξει Κύριος τοὺς Αἰγυπτίους πληγῇ, καὶ ἰάσεται αὐτοὺς ἰάσει, καὶ ἐπιστραφήσονται πρὸς Κύριον, καὶ εἰσακούσεται αὐτῶν, καὶ ἰάσεται αὐτοὺς ἰάσει.
\VS{23}Τῇ ἡμέρᾳ ἐκείνῃ ἔσται ἡ ὁδὸς ἀπὸ Αἰγύπτου πρὸς Ἀσσυρίους, καὶ εἰσελεύσονται Ἀσσύριοι εἰς Αἴγυπτον, καὶ Αἰγύπτιοι πορεύσονται πρὸς Ἀσσυρίους, καὶ δουλεύσουσιν Αἰγύπτιοι τοῖς Ἀσσυρίοις.
\VS{24}Τῇ ἡμέρᾳ ἐκείνῃ ἔσται Ἰσραὴλ τρίτος ἐν τοῖς Αἰγυπτίοις, καὶ ἐν τοῖς Ἀσσυρίοις εὐλογημένος ἐν τῇ γῇ
\VS{25}ἣν εὐλόγησε Κύριος σαβαὼθ, λέγων, εὐλογημένος ὁ λαός μου ὁ ἐν Αἰγύπτῳ, καὶ ὁ ἐν Ἀσσυρίοις, καὶ ἡ κληρονομία μου Ἰσραήλ.

\par }\Chap{20}{\PP \VerseOne{1}Τοῦ ἔτους ὅτε εἰσῆλθε Τανὰθαν εἰς Ἄζωτον, ἡνίκα ἀπεστάλη ὑπὸ Ἀρνᾶ βασιλέως Ἀσσυρίων, καὶ ἐπολέμησε τὴν Ἄζωτον, καὶ ἔλαβεν αὐτὴν,
\VS{2}τότε ἐλάλησε Κύριος πρὸς Ἡσαΐαν υἱὸν Ἀμὼς, λέγων, πορεύου καὶ ἄφελε τὸν σάκκον ἀπὸ τῆς ὀσφύος σου, καὶ τὰ σανδάλιά σου ὑπόλυσαι ἀπὸ τῶν ποδῶν σου, καὶ ποίησον οὕτως, πορευόμενος γυμνὸς καὶ ἀνυπόδετος.
\VS{3}Καὶ εἶπε Κύριος, ὃν τρόπον πεπόρευται ὁ παῖς μου Ἡσαΐας γυμνὸς καὶ ἀνυπόδετος τρία ἔτη, τρία ἔτη ἔσται εἰς σημεῖα καὶ τέρατα τοῖς Αἰγυπτίοις καὶ Αἰθίοψιν·
\VS{4}Ὅτι οὕτως ἄξει βασιλεὺς Ἀσσυρίων τὴν αἰχμαλωσίαν Αἰγύπτου καὶ Αἰθιόπων, νεανίσκους καὶ πρεσβύτας, γυμνοὺς καὶ ἀνυποδέτους, ἀνακεκαλυμμένους τὴν αἰσχύνην Αἰγύπτου.
\VS{5}Καὶ αἰσχυνθήσονται ἡττηθέντες ἐπὶ τοῖς Αἰθίοψιν, ἐφʼ οἷς ἦσαν πεποιθότες οἱ Αἰγύπτιοι, ἦσαν γὰρ αὐτοῖς δόξα.
\VS{6}Καὶ ἐροῦσιν οἱ κατοικοῦντες ἐν τῇ νήσῳ ταύτῃ ἐν τῇ ἡμέρᾳ ἐκείνῃ, ἰδοὺ ἡμεῖς ἦμεν πεποιθότες τοῦ φυγεῖν εἰς αὐτοὺς εἰς βοήθειαν, οἳ οὐκ ἠδύναντο σωθῆναι ἀπὸ βασιλέως Ἀσσυρίων, καὶ πῶς ἡμεῖς σωθησόμεθα;
\par }{\PP ΤΟ ὍΡΑΜΑ ΤΗΣ ἘΡΗΜΟΥ.

\par }\Chap{21}{\PP \VerseOne{1}Ὡς καταιγὶς διʼ ἐρήμου διέλθοι, ἐξ ἐρήμου ἐρχομένη ἐκ γῆς φοβερὸν
\VS{2}τὸ ὅραμα, καὶ σκληρὸν ἀνηγγέλη μοι· ὁ ἀθετῶν ἀθετεῖ, ὁ ἀνομῶν ἀνομεῖ· ἐπʼ ἐμοὶ οἱ Ἐλαμῖται, καὶ οἱ πρέσβεις τῶν Περσῶν ἐπʼ ἐμὲ ἔρχονται· νῦν στενάξω καὶ παρακαλέσω ἐμαυτόν.
\VS{3}Διατοῦτο ἐνεπλήσθη ἡ ὀσφύς μου ἐκλύσεως, καὶ ὠδῖνες ἔλαβόν με ὡς τὴν τίκτουσαν· ἠδίκησα τοῦ μὴ ἀκοῦσαι, ἐσπούδασα τοῦ μὴ βλέπειν.
\VS{4}Ἡ καρδία μου πλανᾶται, καὶ ἡ ἀνομία με βαπτίζει, ἡ ψυχή μου ἐφέστηκεν εἰς φόβον.
\VS{5}Ἑτοίμασον τὴν τράπεζαν, φάγετε, πίετε· ἀναστάντες οἱ ἄρχοντες, ἑτοιμάσατε θυρεοὺς,
\VS{6}ὅτι οὕτως εἶπε πρὸς μὲ Κύριος, Βαδίσας σεαυτῷ στῆσον σκοπὸν, καὶ ὃ ἂν ἴδῃς ἀνάγγειλον.
\VS{7}Καὶ εἶδον ἀναβάτας ἱππεῖς δύο, καὶ ἀναβάτη ὄνου, καὶ ἀναβάτην καμήλου· ἀκρόασαι ἀκρόασιν πολλήν,
\VS{8}καὶ κάλεσον Οὐρίαν εἰς τὴν σκοπιάν· Κύριος εἶπεν, ἔστην διαπαντὸς ἡμέρας, καὶ ἐπὶ τῆς παρεμβολῆς ἐγὼ ἔστην ὅλην τὴν νύκτα,
\VS{9}καὶ ἰδοὺ αὐτὸς ἔρχεται ἀναβάτης ξυνωρίδος· καὶ ἀποκριθεὶς εἶπε, πέπτωκε πέπτωκε Βαβυλὼν, καὶ πάντα τὰ ἀγάλματα αὐτῆς, καὶ τὰ χειροποίητα αὐτῆς συνετρίβησαν εἰς τὴν γῆν.
\VS{10}Ἀκούσατε οἱ καταλελειμμένοι, καὶ οἱ ὀδυνώμενοι ἀκούσατε ἃ ἤκουσα παρὰ Κυρίου σαβαὼθ, ὁ Θεὸς τοῦ Ἰσραὴλ ἀνήγγειλεν ἡμῖν.
\par }{\PP ΤΟ ὍΡΑΜΑ ΤΗΣ ἸΔΟΥΜΑΙΑΣ.
\par }{\PP \VS{11}Πρὸς ἐμὲ κάλει παρὰ τοῦ Σηεὶρ, φυλάσσετε ἐπάλξεις.
\VS{12}Φυλάσσω τοπρωῒ καὶ τὴν νύκτα· ἐὰν ζητῇς ζήτει, καὶ παρʼ ἐμοὶ οἴκει,
\VS{13}ἐν τῷ δρυμῷ ἑσπέρας κοιμηθῇς, ἢ ἐν τῇ ὁδῷ Δαιδάν.
\par }{\PP \VS{14}Εἰς συνάντησιν διψῶντι ὕδωρ φέρετε οἱ ἐνοικοῦντες ἐν χώρᾳ Θαιμὰν, ἄρτοις συναντᾶτε τοῖς φεύγουσι
\VS{15}διὰ τὸ πλῆθος τῶν πεφονευμένων, καὶ διὰ τὸ πλῆθος τῶν πλανωμένων, καὶ διὰ τὸ πλῆθος τῆς μαχαίρας, καὶ διὰ τὸ πλῆθος τῶν τοξευμάτων τῶν διατεταμένων, καὶ διὰ τὸ πλῆθος τῶν πεπτωκότων ἐν τῷ πολέμῳ.
\VS{16}Διότι οὕτως εἶπέ μοι Κύριος, ἔτι ἐνιαυτὸς ὡς ἐνιαυτὸς μισθωτοῦ, ἐκλείψει ἡ δόξα τῶν υἱῶν Κηδὰρ,
\VS{17}καὶ τὸ κατάλοιπον τῶν τοξευμάτων τῶν ἰσχυρῶν υἱῶν Κηδὰρ ἔσται ὀλίγον, ὅτι Κύριος ὁ Θεὸς Ἰσραὴλ ἐλάλησε.
\par }{\PP ΤΟ ῬΗΜΑ ΤΗΣ ΦΑΡΑΓΓΟΣ ΣΙΩΝ.

\par }\Chap{22}{\PP \VerseOne{1}Τί ἐγένετό σοι, ὅτι νῦν ἀνέβητε πάντες εἰς δώματα μάταια;
\VS{2}Ἐνεπλήσθη ἡ πόλις βοώντων, οἱ τραυματίαι σου οὐ τραυματίαι ἐν μαχαίραις, οὐδὲ οἱ νεκροί σου νεκροὶ πολέμων.
\VS{3}Πάντες οἱ ἄρχοντές σου πεφεύγασι, καὶ οἱ ἁλόντες σκληρῶς δεδεμένοι εἰσί, καὶ οἱ ἰσχύοντες ἐν σοὶ πόῤῥω πεφεύγασι.
\VS{4}Διὰ τοῦτο εἶπα, ἄφετέ με, πικρῶς κλαύσομαι· μὴ κατισχύσητε παρακαλεῖν με ἐπὶ τὸ σύντριμμα τῆς θυγατρὸς τοῦ γένους μου,
\VS{5}ὅτι ἡμέρα ταραχῆς καὶ ἀπωλείας καὶ καταπατήματος, καὶ πλάνησις παρὰ Κυρίου σαβαώθ· ἐν φάραγγι Σιὼν πλανῶνται, ἀπὸ μικροῦ ἕως μεγάλου πλανῶνται ἐπὶ τὰ ὄρη.
\VS{6}Οἱ δὲ Ἐλαμεῖται ἔλαβον φαρέτρας, καὶ ἀναβάται ἄνθρωποι ἐφʼ ἵππους, καὶ συναγωγὴ παρατάξεως.
\VS{7}Καὶ ἔσονται αἱ ἐκλεκταὶ φάραγγές σου, πλησθήσονται ἁρμάτων, οἱ δὲ ἱππεῖς ἐμφράξουσι τὰς πύλας σου,
\VS{8}καὶ ἀνακαλύψουσι τὰς πύλας Ἰούδα· καὶ ἐμβλέψονται τῇ ἡμέρᾳ ἐκείνῃ εἰς τοὺς ἐκλεκτοὺς οἴκους τῆς πόλεως·
\VS{9}Καὶ ἀνακαλύψουσι τὰ κρυπτὰ τῶν οἴκων τῆς ἄκρας Δαυείδ· καὶ εἴδοσαν, ὅτι πλείους εἰσί, καὶ ὅτι ἀπέστρεψε τὸ ὕδωρ τῆς ἀρχαίας κολυμβήθρας εἰς τὴν πόλιν,
\VS{10}καὶ ὅτι καθείλοσαν τοὺς οἴκους Ἱερουσαλὴμ εἰς ὀχυρώματα τείχους τῇ πόλει.
\VS{11}Καὶ ἐποιήσατε ἑαυτοῖς ὕδωρ ἀναμέσον τῶν δύο τειχῶν ἐσώτερον τῆς κολυμβήθρας τῆς ἀρχαίας, καὶ οὐκ ἐνεβλέψατε εἰς τὸν ἀπʼ ἀρχῆς ποιήσαντα αὐτὴν, καὶ τὸν κτίσαντα αὐτὴν οὐκ εἴδετε.
\VS{12}Καὶ ἐκάλεσε Κύριος Κύριος σαβαὼθ ἐν τῇ ἡμέρᾳ ἐκείνῃ κλαυθμὸν καὶ κοπετὸν, καὶ ξύρησιν καὶ ζῶσιν σάκκων,
\VS{13}αὐτοὶ δὲ ἐποιήσαντο εὐφροσύνην καὶ ἀγαλλίαμα, σφάζοντες μόσχους, καὶ θύοντες πρόβατα, ὥστε φαγεῖν κρέατα, καὶ πιεῖν οἶνον, λέγονες, φάγωμεν καὶ πίωμεν, αὔριον γὰρ ἀποθνήσκομεν.
\VS{14}Καὶ ἀνακεκαλυμμένα ταῦτά ἐστιν ἐν τοῖς ὠσὶ Κυρίου σαβαὼθ, ὅτι οὐκ ἀφεθήσεται ὑμῖν αὕτη ἡ ἁμαρτία, ἕως ἂν ἀποθάνητε.
\par }{\PP \VS{15}Τάδε λέγει Κύριος σαβαὼθ, πορεύου εἰς τὸ παστοφόριον, πρὸς Σομνᾶν τὸν ταμίαν, καὶ εἰπὸν αὐτῷ,
\VS{16}τί σὺ ὧδε καὶ τί σοί ἐστιν ὧδε; ὅτι ἐλατόμησας σεαυτῷ ὧδε μνημεῖον, καὶ ἐποίησας σεαυτῷ ἐν ὑψηλῷ μνημεῖον, καὶ ἔγραψας σεαυτῷ ἐν πέτρᾳ σκηνήν;
\VS{17}Ἰδοὺ δὴ Κύριος σαβαὼθ ἐκβάλλει καὶ ἐκτρίψει ἄνδρα, καὶ ἀφελεῖ τὴν στολήν σου καὶ τὸν στέφανόν σου τὸν ἔνδοξον,
\VS{18}καὶ ῥίψει σε εἰς χώραν μεγάλην καὶ ἀμέτρητον, καὶ ἐκεῖ ἀποθανῇ· καὶ θήσει τὸ ἅρμα σου τὸ καλὸν εἰς ἀτιμίαν, καὶ τὸν οἶκον τοῦ ἄρχοντός σου εἰς καταπάτημα.
\VS{19}Καὶ ἀφαιρεθήσῃ ἐκ τῆς οἰκονομίας σου καὶ ἐκ τῆς στάσεώς σου.
\VS{20}Καὶ ἔσται ἐν τῇ ἡμέρᾳ ἐκείνῃ καὶ καλέσω τὸν παῖδά μου Ἐλιακεὶμ τὸν τοῦ Χελκίου,
\VS{21}καὶ ἐνδύσω αὐτὸν τὴν στολήν σου, καὶ τὸν στέφανόν σου δώσω αὐτῷ κατὰ κράτος, καὶ τὴν οἰκονομίαν σου δώσω εἰς τὰς χεῖρας αὐτοῦ· καὶ ἔσται ὡς πατὴρ τοῖς ἐνοικοῦσιν ἐν Ἱερουσαλὴμ, καὶ τοῖς ἐνοικοῦσιν ἐν Ἰούδᾳ.
\VS{22}Καὶ δώσω τὴν δόξαν Δαυεὶδ αὐτῷ, καὶ ἄρξει, καὶ οὐκ ἔσται ὁ ἀντιλέγων· καὶ δώσω αὐψῷ τὴν κλεῖδα οἴκου Δαυὶδ ἐπὶ τῷ ὢμῳ αὐτοῦ· καὶ ἀνοίξει, καὶ οὐκ ἔσται ὁ ἀποκλείων· καὶ κλείσει, καὶ οὐκ ἔσται ὁ ἀνοίγων.
\VS{23}Καὶ στήλω αὐτὸν ἄρχοντα ἐν τόπῳ πιστῷ, καὶ ἔσται εἰς θρόνον δόξης τοῦ οἴκου τοῦ πατρὸς αὐτοῦ.
\VS{24}Καὶ ἔσται πεποιθὼς ἐπʼ αὐτὸν πᾶς ἔνδοξος ἐν τῷ οἴκῳ τοῦ πατρὸς αὐτοῦ, ἀπὸ μικροῦ ἕως μεγάλου, καὶ ἔσονται ἐπικρεμάμενοι αὐτῷ
\VS{25}τῇ ἡμέρᾳ ἐκείνῃ· τάδε λέγει Κύριος σαβαὼθ, κινηθήσεται ὁ ἄνθρωπος ὁ ἐστηριγμένος ἐν τόπῳ πιστῷ, καὶ ἀφαιρεθήσεται, καὶ πεσεῖται, καὶ ἐξολεθρευθήσεται ἡ δόξα ἡ ἐπʼ αὐτόν, ὅτι Κύριος ἐλάλησεν.
\par }{\PP ΤΟ ῬΗΜΑ ΤΥΡΟΥ.

\par }\Chap{23}{\PP \VerseOne{1}Ὀλολύξατε, πλοῖα Καρχηδόνος, ὅτι ἀπώλετο, καὶ οὐκέτι ἔρχονται ἐκ γῆς Κιτιαίων, ἦκται αἰχμάλωτος.
\VS{2}Τίνι ὅμοιοι γεγόνασιν οἱ ἐνοικοῦντες ἐν τῇ νήσῳ, μετάβολοι Φοινίκης, διαπερῶντες τὴν θάλασσαν
\VS{3}ἐν ὕδατι πολλῷ, σπέρμα μετάβολων; ὡς ἀμητοῦ εἰσφερομένου, οἱ μεταβόλοι τῶν ἐθνῶν.
\VS{4}αἰσχύνθητι Σιδὼν, εἶπεν ἡ θάλασσα· ἡ δὲ ἰσχὺς τῆς θαλάσσης εἶπεν, οὐκ ὤδινον, οὐδὲ ἔτεκον, οὐδὲ ἐξέθρεψα νεανίσκους, οὐδὲ ὕψωσα παρθένους.
\VS{5}Ὅταν δὲ ἀκουστὸν γένηται Αἰγύπτῳ, λήμψεται αὐτοὺς ὀδύνη περὶ Τύρου.
\VS{6}Ἀπέλθατε εἰς Καρχηδόνα, ὀλολύξατε οἱ κατοικοῦντες ἐν τῇ νήσῳ ταύτῃ.
\VS{7}Οὐχ αὕτη ἦν ὑμῶν ἡ ὕβρις ἀπʼ ἀρχῆς, πρινὴ παραδοθῆναι αὐτήν;
\VS{8}Τίς ταῦτα ἐβούλευσεν ἐπὶ Τύρον; μὴ ἥσσων ἐστίν, ἢ οὐκ ἰσχύει; οἱ ἔμποροι αὐτῆς ἔνδοξοι ἄρχοντες τῆς γῆς.
\par }{\PP \VS{9}Κύριος σαβαὼθ ἐβουλεύσατο παραλῦσαι πᾶσαν τὴν ὕβριν τῶν ἐνδόξων, καὶ ἀτιμάσαι πᾶν ἔνδοξον ἐπὶ τῆς γῆς.
\VS{10}Ἐργάζου τὴν γῆν σου, καὶ γὰρ πλοῖα οὐκέτι ἔρχεται ἐκ Καρχηδόνος.
\VS{11}Ἡ δὲ χείρ σου οὐκέτι ἰσχύει κατὰ θάλασσαν, ἡ παροξύνουσα βασιλεῖς· Κύριος σαβαὼθ ἐνετείλατο περὶ Χαναὰν ἀπολέσαι αὐτῆς τὴν ἰσχύν.
\VS{12}Καὶ ἐροῦσιν, οὐκέτι οὐ μὴ προστεθῆτε τοῦ ὑβρίζειν καὶ ἀδικεῖν τὴν θυγατέρα Σιδῶνος· καὶ ἐὰν ἀπέλθῃς εἰς Κιτιεῖς, οὐδὲ ἐκεῖ ἀνάπαυσις ἔσται σοι·
\VS{13}Καὶ εἰς γῆν Χαλδαίων, καὶ αὕτη ἠρήμωται ἀπὸ τῶν Ἀσσυρίων, ὅτι ὁ τοῖχος αὐτῆς πέπτωκεν.
\VS{14}Ὀλολύξατε πλοῖα Καρχηδόνος, ὅτι ἀπόλωλε τὸ ὀχύρωμα ὑμῶν.
\par }{\PP \VS{15}Καὶ ἔσται ἐν τῇ ἡμέρᾳ ἐκείνῃ, καταλειφθήσεται Τύρος ἔτη ἑβδομήκοντα, ὡς χρόνος βασιλέως, ὡς χρόνος ἀνθρώπου· καὶ ἔσται μετὰ ἑβδομήκοντα ἔτη, ἔσται Τύρος ὡς ᾆσμα πόρνης.
\VS{16}Λάβε κιθάραν, ῥέμβευσον πόλις πόρνη ἐπιλελησμένη, καλῶς κιθάρισον, πολλὰ ᾆσον, ἵνα σου μνεία γένηται.
\VS{17}Καὶ ἔσται μετὰ τὰ ἑβδομήκοντα ἔτη, ἐπισκοπὴν ποιήσει ὁ Θεὸς Τύρου, καὶ πάλιν ἀποκαταστήσεται εἰς τὸ ἀρχαῖον, καὶ ἔσται ἐμπόριον πάσαις ταῖς βασιλείαις τῆς οἰκουμένης ἐπὶ πρόσωπον τῆς γῆς.
\VS{18}Καὶ ἔσται αὐτῆς ἡ ἐμπορία καὶ ὁ μισθὸς ἅγιον Κυρίῳ· οὐκ αὐτοῖς συναχθήσεται, ἀλλὰ τοῖς κατοικοῦσιν ἔναντι Κυρίου, πᾶσα ἡ ἐμπορία αὐτῆς, φαγεῖν καὶ πιεῖν καὶ ἐμπλησθῆναι, καὶ εἰς συμβολὴν μνημόσυνον ἔναντι Κυρίου.

\par }\Chap{24}{\PP \VerseOne{1}Ἰδοὺ Κύριος καταφθείρει τὴν οἰκουμένην, καὶ ἐρημώσει αὐτὴν, καὶ ἀνακαλύψει τὸ πρόσωπον αὐτῆς, καὶ διασπερεῖ τοὺς ἐνοικοῦντας ἐν αὐτῇ.
\VS{2}Καὶ ἔσται ὁ λαὸς ὡς ἱερεύς, καὶ ὁ παῖς ὡς ὁ κύριος, καὶ ἡ θεράπαινα ὡς ἡ κυρία· ἔσται ὁ ἀγοράζων ὡς ὁ πωλῶν, ὁ δανείζων ὡς ὁ δανειζόμενος, καὶ ὁ ὀφείλων ὡς ᾧ ὀφείλει.
\par }{\PP \VS{3}Φθορᾷ φθαρήσεται ἡ γῆ, καὶ προνομῇ προνομευθήσεται ἡ γῆ· τὸ γὰρ στόμα Κυρίου ἐλάλησε ταῦτα.
\VS{4}Ἐπένθησεν ἡ γῆ, καὶ ἐφθάρη ἡ οἰκουμένη, ἐπένθησαν οἱ ὑψηλοὶ τῆς γῆς.
\VS{5}Ἡ δὲ ἠνόμησε διὰ τοὺς κατοικοῦντας αὐτήν, διότι παρήλθοσαν τὸν νόμον, καὶ ἤλλαξαν τὰ προστάγματα διαθήκην αἰώνιον.
\VS{6}Διατοῦτο ἀρὰ ἔδεται τὴν γῆν, ὅτι ἡμάρτοσαν οἱ κατοικοῦντες αὐτήν· διατοῦτο πτωχοὶ ἔσονται οἱ ἐνοικοῦντες ἐν τῇ γῇ, καὶ καταλειφθήσονται ἄνθρωποι ὀλίγοι.
\VS{7}Πενθήσει οἶνος, πενθήσει ἄμπελος, στενάξουσιν πάντες οἱ εὐφραινόμενοι τὴν ψυχήν.
\VS{8}Πέπαυται εὐφροσύνη τυμπάνων, πέπαυται φωνὴ κιθάρας.
\VS{9}Ἠσχύνθησαν, οὐκ ἔπιον οἶνον, πικρὸν ἐγένετο τὸ σίκερα τοῖς πίνουσιν.
\VS{10}Ἠρημώθη πᾶσα πόλις, κλείσει οἰκίαν τοῦ μὴ εἰσελθεῖν.
\VS{11}Ὀλολύζεται περὶ τοῦ οἴνου πανταχῇ, πέπαυται πᾶσα εὐφροσύνη τῆς γῆς, ἀπῆλθε πᾶσα εὐφροσύνη τῆς γῆς.
\VS{12}Καὶ καταλειφθήσονται πόλεις ἔρημοι, καὶ οἶκοι ἐγκαταλελειμμένοι ἀπολοῦνται.
\par }{\PP \VS{13}Ταῦτα πάντα ἔσονται ἐν τῇ γῇ ἐν μέσῳ τῶν ἐθνῶν· ὃν τρόπον ἐάν τις καλαμήσηται ἐλαίαν, οὕτως καλαμήσονται αὐτούς· καὶ ἐὰν παύσηται ὁ τρυγητὸς,
\VS{14}οὗτοι βοῇ φωνήσουσιν. οἱ δὲ καταλειφθέντες ἐπὶ τῆς γῆς εὐφρανθήσονται ἅμα τῇ δόξῃ Κυρίου, ταραχθήσεται τὸ ὕδωρ τῆς θαλάσσης.
\VS{15}Διατοῦτο ἡ δόξα Κυρίου ἐν ταῖς νήσοις ἔσται τῆς θαλάσσης, τὸ ὄνομα Κυρίου ἔνδοξον ἔσται.
\par }{\PP \VS{16}Κύριε ὁ Θεὸς Ἰσραὴλ, ἀπὸ τῶν πτερύγων τῆς γῆς τέρατα ἠκούσαμεν, ἐλπὶς τῷ εὐσεβεῖ· καὶ ἐροῦσιν, οὐαὶ τοῖς ἀθετοῦσιν, οἱ ἀθετοῦντες τὸν νόμον.
\VS{17}Φόβος καὶ βόθυνος καὶ παγὶς ἐφʼ ὑμᾶς τοὺς ἐνοικοῦντας ἐπὶ τῆς γῆς.
\VS{18}Καὶ ἔσται ὁ φεύγων τὸν φόβον, ἐμπεσεῖται εἰς τὸν βόθυνον· καὶ ὁ ἐκβαίνων ἐκ τοῦ βοθύνου, ἁλώσεται ὑπὸ τῆς παγίδος· ὅτι θυρίδες ἐκ τοῦ οὐρανοῦ ἀνεῴχθησαν, καὶ σεισθήσεται τὰ θεμέλια τῆς γῆς.
\VS{19}Ταραχῇ ταραχθήσεται ἡ γῆ, καὶ ἀπορίᾳ ἀπορηθήσεται ἡ γῆ.
\VS{20}Ἔκλινεν ὡς ὁ μεθύων καὶ κραιπαλῶν, καὶ σεισθήσεται ὡς ὀπωροφυλάκιον ἡ γῆ· κατίσχυσεν γὰρ ἐπʼ αὐτῆς ἡ ἀνομία, καὶ πεσεῖται, καὶ οὐ μὴ δύνηται ἀναστῆναι.
\par }{\PP \VS{21}Καὶ ἐπάξει ὁ Θεὸς ἐπὶ τὸν κόσμον τοῦ οὐρανοῦ τὴν χεῖρα, καὶ ἐπὶ τοὺς βασιλεῖς τῆς γῆς.
\VS{22}Καὶ συνάξουσι συναγωγὴν αὐτῆς εἰς δεσμωτήριον, καὶ ἀποκλείσουσιν εἰς ὀχύρωμα· διὰ πολλῶν γενεῶν ἐπισκοπὴ ἔσται αὐτῶν.
\VS{23}Καὶ τακήσεται ἡ πλίνθος, καὶ πεσεῖται τὸ τεῖχος· ὅτι βασιλεύσει Κύριος ἐκ Σιὼν, καὶ ἐξ Ἱερουσαλὴμ, καὶ ἐνώπιον τῶν πρεσβυτέρων δοξασθήσεται.

\par }\Chap{25}{\PP \VerseOne{1}Κύριε ὁ Θεὸς δοξάσω σε, ὑμνήσω τὸ ὄνομά σου, ὅτι ἐποίησας θαυμαστὰ πράγματα, βουλὴν ἀρχαίαν ἀληθινήν· γένοιτο.
\VS{2}Ὅτι ἔθηκας πόλεις εἰς χῶμα, πόλεις ὀχυρὰς τοῦ μὴ πεσεῖν αὐτῶν τὰ θεμέλια· τῶν ἀσεβῶν πόλις τὸν αἰῶνα οὐ μὴ οἰκοδομηθῇ.
\VS{3}Διατοῦτο εὐλογήσει σε ὁ λαὸς ὁ πτωχὸς, καὶ πόλεις ἀνθρώπων ἀδικουμένων εὐλογήσουσί σε.
\VS{4}Ἐγένου γὰρ πάσῃ πόλει ταπεινῇ βοηθὸς, καὶ τοῖς ἀθυμήσασιν διʼ ἔνδειαν σκέπη, ἀπὸ ἀνθρώπων πονηρῶν ῥύσῃ αὐτούς· σκέπη διψώντων, καὶ πνεῦμα ἀνθρώπων ἀδικουμένων.
\par }{\PP \VS{5}Ὡς ἄνθρωποι ὀλιγόψυχοι διψῶντες ἐν Σιὼν, ἀπὸ ἀνθρώπων ἀσεβῶν, οἷς ἡμᾶς παρέδωκας.
\VS{6}Καὶ ποιήσει Κύριος σαβαὼθ πᾶσι τοῖς ἔθνεσιν· ἐπὶ τὸ ὄρος τοῦτο πίονται εὐφροσύνην, πίονται οἶνον·
\VS{7}Χρίσονται μύρον ἐν τῷ ὄρει τούτῳ· παράδος ταῦτα πάντα τοῖς ἔθνεσιν· ἡ γὰρ βουλὴ αὕτη ἐπὶ πάντα τὰ ἔθνη.
\VS{8}Κατέπιεν ὁ θάνατος ἰσχύσας, καὶ πάλιν ἀφεῖλε Κύριος ὁ Θεὸς πᾶν δάκρυον ἀπὸ παντὸς προσώπου· τὸ ὄνειδος τοῦ λαοῦ ἀφεῖλεν ἀπὸ πάσης τῆς γῆς, τὸ γὰρ στόμα Κυρίου ἐλάλησε.
\VS{9}Καὶ ἐροῦσι τῇ ἡμέρᾳ ἐκείνῃ, ἰδοὺ ὁ Θεὸς ἡμῶν ἐφʼ ᾧ ἠλπίζομεν, καὶ σώσει ἡμᾶς· οὗτος Κύριος, ὑπεμείναμεν αὐτῷ, καὶ ἠγαλλιώμεθα καὶ εὐφρανθησόμεθα ἐπὶ τῇ σωτηρίᾳ ἡμῶν.
\par }{\PP \VS{10}Ἀνάπαυσιν δώσει ὁ Θεὸς ἐπὶ τὸ ὄρος τοῦτο, καὶ καταπατηθήσεται ἡ Μωαβίτις, ὃν τρόπον πατοῦσιν ἅλωνα ἐν ἁμάξαις.
\VS{11}Καὶ ἀνήσει τὰς χεῖρας αὐτοῦ, ὃν τρόπον καὶ αὐτὸς ἐταπείνωσε τοῦ ἀπολέσαι· καὶ ταπεινώσει τὴν ὕβριν αὐτοῦ, ἐφʼ ἃ τὰς χεῖρας ἐπέβαλε.
\VS{12}Καὶ τὸ ὕψος τῆς καταφυγῆς τοῦ τοίχου ταπεινώσει, καὶ καταβήσεται ἕως τοῦ ἐδάφους.

\par }\Chap{26}{\PP \VerseOne{1}Τῇ ἡμέρᾳ ἐκείνῃ ᾂσονται τὸ ᾆσμα τοῦτο ἐπὶ γῆς τῆς Ἰουδαίας· ἰδοὺ πόλις ἰσχυρά, καὶ σωτήριον θήσει τὸ τεῖχος, καὶ περίτειχος.
\VS{2}Ἀνοίξατε πύλας, εἰσελθέτω λαὸς φυλάσσων δικαιοσύνην, καὶ φυλάσσων ἀλήθειαν,
\VS{3}ἀντιλαμβανόμενος ἀληθείας, καὶ φυλάσσων εἰρήνην· ὅτι ἐπὶ σοὶ ἐλπίδι
\VS{4}ἤλπισαν Κύριε ἕως τοῦ αἰῶνος, ὁ Θεὸς ὁ μέγας, ὁ αἰώνιος,
\VS{5}ὃς ταπεινώσας κατήγαγες τοὺς ἐνοικοῦντας ἐν ὑψηλοῖς· πόλεις ὀχυρὰς καταβαλεῖς, καὶ κατάξεις ἕως ἐδάφους.
\VS{6}Καὶ πατήσουσιν αὐτοὺς πόδες πρᾳέων καὶ ταπεινῶν.
\par }{\PP \VS{7}Ὁδὸς εὐσεβῶν εὐθεῖα ἐγένετο, ἡ ὁδὸς τῶν εὐσεβῶν καὶ παρεσκευασμένη.
\VS{8}Ἡ γὰρ ὁδὸς Κυρίου κρίσις· ἠλπίσαμεν ἐπὶ τῷ ὀνόματί σου, καὶ ἐπὶ τῇ μνείᾳ
\VS{9}ᾗ ἐπιθυμεῖ ἡ ψυχὴ ἡμῶν· ἐκ νυκτὸς ὀρθρίζει τὸ πνεῦμά μου πρὸς σὲ ὁ Θεὸς, διότι φῶς τὰ προστάγματά σου ἐπὶ τῆς γῆς· δικαιοσύνην μάθετε οἱ ἐνοικοῦντες ἐπὶ τῆς γῆς.
\VS{10}Πέπαυται γὰρ ὁ ἀσεβής· πᾶς ὃς οὐ μὴ μάθῃ δικαιοσύνην ἐπὶ τῆς γῆς, ἀλήθειαν οὐ μὴ ποιήσει· ἀρθήτω ὁ ἀσεβὴς, ἵνα μὴ ἴδῃ τὴν δόξαν Κυρίου.
\VS{11}Κύριε ὑψηλός σου ὁ βραχίων, καὶ οὐκ ᾔδεισαν, γνόντες δὲ αἰσχυνθήσονται· ζῆλος λήψεται λαὸν ἀπαίδευτον, καὶ νῦν πῦρ τοὺς ὑπεναντίους ἔδεται.
\VS{12}Κύριε ὁ Θεὸς ἡμῶν, εἰρήνην δὸς ἡμῖν, πάντα γὰρ ἀπέδωκας ἡμῖν.
\VS{13}Κύριε ὁ Θεὸς ἡμῶν, κτῆσαι ἡμᾶς· Κύριε ἐκτός σου ἄλλον οὐκ οἴδαμεν· τὸ ὄνομά σου ὀνομάζομεν.
\par }{\PP \VS{14}Οἱ δὲ νεκροὶ ζωὴν οὐ μὴ ἴδωσιν, οὐδὲ ἰατροὶ οὐ μὴ ἀναστήσουσι· διατοῦτο ἐπήγαγες, καὶ ἀπώλεσας, καὶ ᾖρας πᾶν ἄρσεν αὐτῶν.
\VS{15}Πρόσθες αὐτοῖς κακὰ Κύριε, πρόσθες κακὰ τοῖς ἐνδόξοις τῆς γῆς.
\par }{\PP \VS{16}Κύριε, ἐν θλίψει ἐμνήσθην σου, ἐν θλίψει μικρᾷ ἡ παιδεία σου ἡμῖν.
\VS{17}Καὶ ὡς ἡ ὠδίνουσα ἐγγίζει τεκεῖν, ἐπὶ τῇ ὠδῖνι αὐτῆς ἐκέκραξεν, οὕτως ἐγενήθημεν τῷ ἀγαπητῷ σου.
\VS{18}Διὰ τὸν φόβον σου Κύριε ἐν γαστρὶ ἐλάβομεν, καὶ ὠδινήσαμεν, καὶ ἐτέκομεν πνεῦμα σωτηρίας σου, ὃ ἐποιήσαμεν ἐπὶ τῆς γῆς· οὐ πεσούμεθα, ἀλλὰ πεσοῦνται πάντες οἱ ἐνοικοῦντες ἐπὶ τῆς γῆς.
\VS{19}Ἀναστήσονται οἱ νεκροὶ, καὶ ἐγερθήσονται οἱ ἐν τοῖς μνημείοις, καὶ εὐφρανθήσονται οἱ ἐν τῇ γῇ· ἡ γὰρ δρόσος ἡ παρὰ σοῦ ἴαμα αὐτοῖς ἐστιν, ἡ δὲ γῆ τῶν ἀσεβῶν πεσεῖται.
\VS{20}Βάδιζε λαός μου, εἴσελθε εἰς τὰ ταμεῖά σου, ἀπόκλεισον τὴν θύραν σου, ἀποκρύβηθι μικρὸν ὅσον ὅσον, ἕως ἂν παρέλθῃ ἡ ὀργὴ Κυρίου.
\VS{21}Ἰδοὺ γὰρ Κύριος ἀπὸ τοῦ ἁγίου ἐπάγει τὴν ὀργὴν ἐπὶ τοὺς ἐνοικοῦντασἐ πὶ τῆς γῆς· καὶ ἀνακαλύψει ἡ γῆ τὸ αἷμα αὐτῆς, καὶ οὐ κατακαλύψει τοὺς ἀνῃρημένους.

\par }\Chap{27}{\PP \VerseOne{1}Ἐν τῇ ἡμέρᾳ ἐκείνῃ ἐπάξει ὁ Θεὸς τὴν μάχαιραν τὴν ἁγίαν, καὶ τὴν μεγάλην, καὶ τὴν ἰσχυρὰν ἐπὶ τὸν δράκοντα ὄφιν φεύγοντα, ἐπὶ τὸν δράκοντα ὄφιν σκολιόν· ἀνελεῖ τὸν δράκοντα.
\VS{2}Τῇ ἡμέρᾳ ἐκείνῃ ἀμπελὼν καλός, ἐπιθύμημα ἐξάρχειν κατʼ αὐτῆς.
\VS{3}Ἐγὼ πόλις ὀχυρά, πόλις πολιορκουμένη, μάτην ποτιῶ αὐτήν· ἁλώσεται γὰρ νυκτὸς, ἡμέρας δὲ πεσεῖται τεῖχος.
\VS{4}Οὐκ ἔστιν, ἣ οὐκ ἐπελάβετο αὐτῆς· τίς με θήσει φυλάσσειν καλάμην ἐν ἀγρῷ; διὰ τὴν πολεμίαν ταύτην ἠθέτηκα αὐτήν· τοίνυν διατοῦτο ἐποίησε Κύριος πάντα, ὅσα συνέταξε· κατακέκαυμαι
\VS{5}Βοήσονται οἱ ἐνοικοῦντες ἐν αὐτῇ, ποιήσωμεν εἰρήνην αὐτῷ, ποιήσωμεν εἰρήνην,
\VS{6}οἱ ἐρχόμενοι τέκνα Ἰακώβ· βλαστήσει καὶ ἐξανθήσει Ἰσραὴλ, καὶ ἐμπλησθήσεται ἡ οἰκουμένη τοῦ καρποῦ αὐτοῦ.
\par }{\PP \VS{7}Μὴ ὡς αὐτὸς ἐπάταξε, καὶ αὐτὸς οὕτως πληγήσεται; καὶ ὡς αὐτὸς ἀνεῖλεν, οὕτως ἀναιρεθήσεται;
\VS{8}Μαχόμενος καὶ ὀνειδίζων ἐξαποστελεῖ αὐτούς· οὐ σὺ ἦσθα μελετῶν τῷ πνεύματι τῷ σκληρῷ, ἀνελεῖν αὐτοὺς πνεύματι θυμοῦ;
\VS{9}Διατοῦτο ἀφαιρεθήσεται ἀνομία Ἰακὼβ, καὶ τοῦτό ἐστιν ἡ εὐλογία αὐτοῦ, ὅταν ἀφέλωμαι τὴν ἁμαρτίαν αὐτοῦ, ὅταν θῶσι πάντας τοὺς λίθους τῶν βωμῶν κατακεκομμένους, ὡς κονίαν λεπτήν· καὶ οὐ μὴ μείνῃ τὰ δένδρα αὐτῶν, καὶ τὰ εἴδωλα αὐτῶν ἐκκεκομμένα, ὥσπερ δρυμὸς μακράν.
\VS{10}Τὸ κατοικούμενον ποίμνιον ἀνειμένον ἔσται, ὡς ποίμνιον καταλελειμμένον· καὶ ἔσται πολὺν χρόνον εἰς βόσκημα, καὶ ἐκεῖ ἀναπαύσονται ποίμνια.
\VS{11}Καὶ μετὰ χρόνον οὐκ ἔσται ἐν αὐτῇ πᾶν χλωρὸν διὰ τὸ ξηρανθῆναι· γυναῖκες ἐρχόμεναι ἀπὸ θέας δεῦτε· οὐ γὰρ λαός ἐστιν ἔχων σύνεσιν, διατοῦτο οὐ μὴ οἰκτειρήσῃ ὁ ποιήσας αὐτοὺς, οὐδὲ ὁ πλάσας αὐτοὺς οὐ μὴ ἐλεήσῃ.
\par }{\PP \VS{12}Καὶ ἔσται ἐν τῇ ἡμέρᾳ ἐκείνῃ, συμφράξει ὁ Θεὸς ἀπὸ τῆς διώρυχος τοῦ ποταμοῦ ἕως Ῥινοκορούρων· ὑμεῖς δὲ συναγάγετε κατὰ ἕνα τοὺς υἱοὺς Ἰσραήλ.
\VS{13}Καὶ ἔσται ἐν τῇ ἡμέρᾳ ἐκείνῃ, σαλπιοῦσι τῇ σάλπιγγι τῇ μεγάλῃ, καὶ ἥξουσιν οἱ ἀπολόμενοι ἐν τῇ χώρᾳ τῶν Ἀσσυρίων, καὶ οἱ ἀπολόμενοι ἐν Αἰγύπτῳ, καὶ προσκυνήσουσι τῷ κυρίῳ ἐπὶ τὸ ὄρος τὸ ἅγιον ἐν Ἱερουσαλήμ.

\par }\Chap{28}{\PP \VerseOne{1}Οὐαὶ τῷ στεφάνῳ τῆς ὕβρεως, οἱ μισθωτοὶ Ἐφραῒμ, τὸ ἄνθος τὸ ἐκπεσὸν ἐκ τῆς δόξης ἐπὶ τῆς κορυφῆς τοῦ ὄρους τοῦ παχέος, οἱ μεθύοντες ἄνευ οἴνου.
\VS{2}Ἰδοὺ ἰσχυρὸν καὶ σκληρὸν ὁ θυμὸς Κυρίου, ὡς χάλαζα καταφερομένη οὐκ ἔχουσα σκέπην, βίᾳ καταφερομένη· ὡς ὕδατος πολὺ πλῆθος σῦρον χώραν, τῇ γῇ ποιήσει ἀνάπαυμα· ταῖς χερσὶ,
\VS{3}καὶ τοῖς ποσὶ καταπατηθήσεται ὁ στέφανος τῆς ὕβρεως, οἱ μισθωτοὶ τοῦ Ἐφραΐμ.
\VS{4}Καὶ ἔσται τὸ ἄνθος τὸ ἐκπεσὸν τῆς ἐλπίδος τῆς δόζῆς, ἐπʼ ἄκρου τοῦ ὄρους τοῦ ὑψηλοῦ· ὡς πρόδρομος σύκου, ὁ ἰδὼν αὐτό, πρὶν εἰς τὴν χεῖρα αὐτοῦ λαβεῖν αὐτό, θελήσει αὐτὸ καταπιεῖν.
\par }{\PP \VS{5}Τῇ ἡμέρᾳ ἐκείνῃ ἔσται Κύριος σαβαὼθ ὁ στέφανος τῆς ἐλπίδος, ὁ πλεκεὶς τῆς δόξης, τῷ καταλειφθέντι τοῦ λαοῦ.
\VS{6}Καταλειφθήσονται ἐπὶ πνεύματι κρίσεως ἐπὶ κρίσιν, καὶ ἰσχὺν κωλύων ἀνελεῖν.
\VS{7}Οὗτοι γὰρ οἴνῳ πεπλημμελημένοι εἰσίν· ἐπλανήθησαν διὰ τὸ σίκερα, ἱερεὺς καὶ προφήτης ἐξέστησαν διὰ τὸ σίκερα, κατεπόθησαν διὰ τὸν οἶνον, ἐσείσθησαν ἀπὸ τῆς μέθης, ἐπλανήθησαν· τοῦτέστι φάσμα.
\VS{8}Ἀρὰ ἔδεται ταύτην τὴν βουλήν, αὕτη γὰρ ἡ βουλὴ ἕνεκα πλεονεξίας.
\VS{9}Τίνι ἀνηγγείλαμεν κακά, καὶ τίνι ἀνηγγείλαμεν ἀγγελίαν; οἱ ἀπογεγαλακτισμένοι ἀπὸ γάλακτος, οἱ ἀπεσπασμένοι ἀπὸ μαστοῦ.
\VS{10}Θλίψιν ἐπὶ θλίψιν προσδέχου, ἐλπίδα ἐπʼ ἐλπίδι, ἐτι μικρὸν ἔτι μικρόν,
\VS{11}διὰ φαυλισμὸν χειλέων, διὰ γλώσσης ἑτέρας, ὅτι λαλήσουσι τῷ λαῷ τούτῳ,
\VS{12}λέγοντες αὐτοῖς, τοῦτο τὸ ἀνάπαυμα τῷ πεινῶντι, καὶ τοῦτο τὸ σύντριμμα· καὶ οὐκ ἠθέλησαν ἀκούειν.
\par }{\PP \VS{13}Καὶ ἔσται αὐτοῖς τὸ λόγιον τοῦ Θεοῦ, θλίψις ἐπὶ θλίψιν, ἐλπὶς ἐπʼ ἐλπίδι, ἔτι μικρὸν ἔτι μικρὸν, ἵνα πορεύσωσι καὶ πέσωσιν ὀπίσω· καὶ συντριβήσονται, καὶ κινδυνεύσουσι, καὶ ἁλώσονται.
\par }{\PP \VS{14}Διατοῦτο ἀκούσατε λόγον Κυρίου ἄνδρες τεθλιμμένοι, καὶ οἱ ἄρχοντες τοῦ λαοῦ τούτου τοῦ ἐν Ἰερουσαλήμ·
\VS{15}Ὅτι εἴπατε, ἐποιήσαμεν διαθήκην μετὰ τοῦ ᾅδου, καὶ μετὰ τοῦ θανάτου συνθήκας· καταιγὶς φερομένη ἐὰν παρέλθῃ, οὐ μὴ ἔλθῃ ἐφʼ ἡμᾶς· ἐθήκαμεν ψεῦδος τὴν ἐλπίδα ἡμῶν, καὶ τῷ ψεύδει σκεπασθησόμεθα.
\VS{16}Διατοῦτο οὕτω λέγει κύριος Κύριος,
\par }{\PP Ἰδοὺ ἐγὼ ἐμβάλλω εἰς τὰ θεμέλια Σιὼν λίθον πολυτελῆ, ἐκλεκτὸν, ἀκρογωνιαῖον, ἔντιμον, εἰς τὰ θεμέλια αὐτῆς, καὶ ὁ πιστεύων οὐ μὴ καταισχυνθῇ.
\VS{17}Καὶ θήσω κρίσιν εἰς ἐλπίδα, ἡ δὲ ἐλεημοσύνη μου εἰς σταθμούς, καὶ οἱ πεποιθότες μάτην ψεύδει· ὅτι οὐ μὴ παρέλθῃ ὑμᾶς καταιγίς,
\VS{18}μὴ καὶ ἀφέλῃ ὑμῶν τὴν διαθήκην τοῦ θανάτου, καὶ ἡ ἐλπὶς ὑμῶν ἡ πρὸς τὸν ᾅδην οὐ μὴ ἐμμείνῃ· καταιγὶς φερομένη ἐὰν ἐπέλθῃ, ἔσεσθε αὐτῇ εἰς καταπάτημα.
\VS{19}Ὅταν παρέλθῃ, λήμψεται ὑμᾶς, πρωῒ πρωῒ παρελεύσεται ἡμέρας, καὶ ἐν νυκτὶ ἔσται ἐλπὶς πονηρά.
\par }{\PP \VS{20}Μάθετε ἀκούειν στενοχωρούμενοι· Οὐ δυνάμεθα μάχεσθαι, αὐτοὶ δὲ ἀσθενοῦμεν τοῦ ὑμᾶς συναχθῆναι.
\VS{21}Ὥσπερ ὄρος ἀσεβῶν ἀναστήσεται Κύριος, καὶ ἔσται ἐν τῇ φάραγγι Γαβαὼν, μετὰ θυμοῦ ποιήσει τὰ ἔργα αὐτοῦ, πικρίας ἔργον· ὁ δὲ θυμὸς αὐτοῦ ἀλλοτρίως χρήσεται, καὶ ἡ σαπρία αὐτοῦ ἀλλοτρία.
\VS{22}Καὶ ὑμεῖς μὴ εὐφρανθείητε, μηδὲ ἰσχυσάτωσαν ὑμῶν οἱ δεσμοί· διότι συντετελεσμένα καὶ συντετμημένα πράγματα ἤκουσα παρὰ Κυρίου σαβαὼθ, ἃ ποιήσει ἐπὶ πᾶσαν τὴν γῆν.
\par }{\PP \VS{23}Ἐνωτίζεσθε καὶ ἀκούετε τῆς φωνῆς μου, προσέχετε καὶ ἀκούετε τοὺς λόγους μου.
\VS{24}Μὴ ὅλην τὴν ἡμέραν ἀροτριάσει ὁ ἀροτριῶν; ἢ σπόρον προετοιμάσει, πρὶν ἐργάσασθαι τὴν γῆν;
\VS{25}Οὐχ ὅταν ὁμαλίσῃ τὸ πρόσωπον αὐτῆς, τότε σπείρει μικρὸν μελάνθιον ἢ κύμινον, καὶ πάλιν σπείρει πυρὸν, καὶ κριθὴν, καὶ κέγχρον καὶ ζέαν ἐν τοῖς ὁρίοις σου;
\VS{26}Καὶ παιδευθήσῃ κρίματι Θεοῦ σου, καὶ εὐφρανθήσῃ.
\VS{27}Οὐ γὰρ μετὰ σκληρότητος καθαίρεται τὸ μελάνθιον, οὐδὲ τροχὸς ἁμάξης περιάξει ἐπὶ τὸ κύμινον· ἀλλὰ ῥάβδῳ τινάσσεται τὸ μελάνθιον, τὸ δὲ κύμινον
\VS{28}μετὰ ἄρτου βρωθήσεται· οὐ γὰρ εἰς τὸν αἰῶνα ἐγώ εἰμι ὑμῖν ὀργισθήσομαι, οὐδὲ φωνὴ τῆς πικρίας μου καταπατήσει ὑμᾶς.
\VS{29}Καὶ ταῦτα παρὰ Κυρίου σαβαὼθ ἐξῆλθε τὰ τέρατα· βουλεύσασθε, ὑψώσατε ματαίαν παράκλησιν.

\par }\Chap{29}{\PP \VerseOne{1}Οὐαὶ Ἀριὴλ πόλις, ἣν ἐπολέμησε Δαυείδ· συναγάγετε γενήματα ἐνιαυτὸν ἐπὶ ἐνιαυτὸν, φάγεσθε, φάγεσθε γὰρ σὺν Μωάβ,
\VS{2}ἐκθλίψω γὰρ Ἀριήλ· καὶ ἔσται αὐτῆς ἡ ἰσχὺς καὶ ὁ πλοῦτος ἐμοί.
\VS{3}Καὶ κυκλώσω ὡς Δαυὶδ ἐπὶ σὲ, καὶ βαλῶ περὶ σὲ χάρακα, καὶ θήσω περὶ σὲ πύργους,
\VS{4}καὶ ταπεινωθήσονται εἰς τὴν γῆν οἱ λόγοι σου, καὶ εἰς τὴν γῆν οἱ λόγοι σου δύσονται· καὶ ἔσονται ὡς οἱ φωνοῦντες ἐκ τῆς γῆς ἡ φωνή σου, καὶ πρὸς τὸ ἔδαφος ἡ φωνή σου ἀσθενήσει.
\VS{5}Καὶ ἔσται ὡς κονιορτὸς ἀπὸ τροχοῦ ὁ πλοῦτος τῶν ἀσεβῶν, καὶ ὡς χνοῦς φερόμενος τὸ πλῆθος τῶν καταδυναστευόντων σε, καὶ ἔσται ὡς στιγμὴ παραχρῆμα
\VS{6}παρὰ Κυρίου σαβαώθ· ἐπισκοπὴ γὰρ ἔσται μετὰ βροντῆς καὶ σεισμοῦ καὶ φωνὴς μεγάλης, καταιγὶς φερομένη, καὶ φλὸξ πυρὸς κατεσθίουσα.
\VS{7}Καὶ ἔσται ὡς ἐνυπνιαζόμενος καθʼ ὕπνους νυκτὸς, ὁ πλοῦτος ἁπάντων τῶν ἐθνῶν, ὅσοι ἐπεστράτευσαν ἐπὶ Ἀριὴλ, καὶ πάντες οἱ στρατευόμενοι ἐπὶ Ἰερουσαλὴμ, καὶ πάντες οἱ συνηγμένοι ἐπʼ αὐτὴν, καὶ οἱ θλίβοντες αὐτήν.
\VS{8}Καὶ ὡς οἱ ἐν τῷ ὕπνῳ πίνοντες καὶ ἔσθοντες, καὶ ἐξαναστάντων, μάταιον τὸ ἐνύπνιον· καὶ ὃν τρόπον ἐνυπνιάζεται ὁ διψῶν ὡς ὁ πίνων, καὶ ἐξαναστὰς ἔτι διψᾷ, ἡ δὲ ψυχὴ αὐτοῦ εἰς κενὸν ἤλπισεν· οὕτως ἔσται ὁ πλοῦτος τῶν ἐθνῶν πάντων, ὅσοι ἐπεστράτευσαν ἐπὶ τὸ ὄρος Σιών.
\par }{\PP \VS{9}Ἐκλύθητε καὶ ἔκστητε, καὶ κραιπαλήσατε οὐκ ἀπὸ σίκερα οὐδὲ ἀπὸ οἴνου.
\VS{10}ὅτι πεπότικεν ὑμᾶς Κύριος πνεύματι κατανύξεως, καὶ καμμύσει τοὺς ὀφθαλμοὺς αὐτῶν, καὶ τῶν προφητῶν αὐτῶν, καὶ τῶν ἀρχόντων αὐτῶν, οἱ ὁρῶντες τὰ κρυπτά.
\VS{11}Καὶ ἔσται ὑμῖν τὰ ῥήματα πάντα ταῦτα, ὡς οἱ λόγοι τοῦ βιβλίου τοῦ ἐσφραγισμένου τσύτου, ὃ ἐὰν δῶσιν αὐτὸ ἀνθρώπῳ ἐπισταμένῳ γράμματα, λέγοντες, ἀνάγνωθι ταῦτα· καὶ ἐρεῖ, οὐ δύναμαι ἀναγνῶναι, ἐσφράγισται γάρ.
\VS{12}καὶ δοθήσεται τὸ βιβλίον τοῦτο εἰς χεῖρας ἀνθρώπου μὴ ἐπισταμένου γράμματα, καὶ ἐρεῖ αὐτῷ, ἀνάγνωθι τοῦτο· καὶ ἐρεῖ, οὐκ ἐπίσταμαι γράμματα.
\par }{\PP \VS{13}Καὶ εἶπε Κύριος, ἐγγίζει μοι ὁ λαὸς οὗτος ἐν τῷ στόματι αὐτοῦ, καὶ ἐν τοῖς χείλεσιν αὐτῶν τιμῶσί με, ἡ δὲ καρδία αὐτῶν πόῤῥω ἀπέχει ἀπʼ ἐμοῦ· μάτην δὲ σέβονταί με, διδάσκοντες ἐντάλματα ἀνθρώπων καὶ διδασκαλίας.
\VS{14}Διατοῦτο ἰδοὺ προσθήσω τοῦ μεταθεῖναι τὸν λαὸν τοῦτον, καὶ μεταθήσω αὐτούς, καὶ ἀπολῶ τὴν σοφίαν τῶν σοφῶν, καὶ τὴν σύνεσιν τῶν συνετῶν κρύψω.
\VS{15}Οὐαὶ οἱ βαθέως βουλὴν ποιοῦντες, καὶ οὐ διὰ Κυπίυο· οὐαὶ οἱ ἐν κρυφῇ βουλὴν ποιοῦντες, καὶ ἔσται ἐν σκότει τὰ ἔργα αὐτῶν· καὶ ἐροῦσι, τίς ἑόρακεν ἡμᾶς; καὶ τίς ἡμᾶς γνώσεται, ἢ ἃ ἡμεῖς ποιοῦμεν;
\VS{16}Οὐχ ὡς πηλὸς τοῦ κεραμέως λογισθήσεσθε; μὴ ἐρεῖ τὸ πλάσμα τῷ πλάσαντι αὐτό, οὐ σύ με ἔπλασας; ἢ τὸ ποίημα τῷ ποιήσαντι, οὐ συνετῶς με ἐποίησας;
\VS{17}Οὐκέτι μικρὸν καὶ μετατεθήσεται ὁ Λίβανος, ὡς τὸ ὄρος τὸ Χέρμελ, καὶ τὸ Χέρμελ εἰς δρυμὸν λογισθήσεται;
\VS{18}Καὶ ἀκούσονται ἐν τῇ ἡμέρᾳ ἐκείνῃ κωφοὶ λόγους βιβλίου, καὶ οἱ ἐν τῷ σκότει, καὶ οἱ ἐν τῇ ὁμίχλῃ· ὀφθαλμοὶ τυφλῶν ὄψονται,
\VS{19}καὶ ἀγαλλιάσονται πτωχοὶ διὰ Κύριον ἐν εὐφροσύνῃ, καὶ οἱ ἀπηλπισμένοι τῶν ἀνθρώπων ἐμπλησθήσονται εὐφροσύνης.
\VS{20}Ἐξέλιπεν ἄνομος, καὶ ἀπώλετο ὑπερήφανος, καὶ ἐξωλοθρεύθησαν οἱ ἀνομοῦντες ἐπὶ κακίᾳ,
\VS{21}καὶ οἱ ποιοῦντες ἁμαρτεῖν ἀνθρώπους ἐν λόγῳ· πάντας δὲ τοὺς ἐλέγχοντας ἐν πύλαις πρόσκομμα θήσουσιν, ὅτι ἐπλαγίασαν ἐπʼ ἀδίκοις δίκαιον.
\par }{\PP \VS{22}Διατοῦτο τάδε λέγει Κύριος ἐπὶ τὸν οἶκον Ἰακώβ, ὃν ἀφώρισεν ἐξ Ἀβραὰμ, οὐ νῦν αἰσχυνθήσεται Ἰακώβ, οὐδὲ νῦν τὸ πρόσωπον μεταβαλεῖ.
\VS{23}Ἀλλʼ ὅταν ἴδωσιν τὰ τέκνα αὐτῶν τὰ ἔργα μου, διʼ ἐμὲ ἁγιάσουσι τὸ ὄνομά μου, καὶ ἁγιάσουσι τὸν ἅγιον Ἰακὼβ, καὶ τὸν Θεὸν τοῦ Ἰσραὴλ φοβηθήσονται.
\VS{24}Καὶ γνώσονται οἱ πλανώμενοι τῷ πνεύματι σύνεσιν, οἱ δὲ γογγύζοντες μαθήσονται ὑπακούειν, καὶ αἱ γλῶσσαι αἱ ψελλίζουσαι μαθήσονται λαλεῖν εἰρήνην.

\par }\Chap{30}{\PP \VerseOne{1}Οὐαὶ τέκνα ἀποστάται, λέγει Κύριος· ἐποιήσατε βουλὴν οὐ διʼ ἐμοῦ, καὶ συνθήκας οὐ διὰ τοῦ πνεύματός μου, προσθεῖναι ἁμαρτίας ἐφʼ ἁμαρτίας,
\VS{2}οἱ πορευόμενοι καταβῆναι εἰς Αἴγυπτον, ἐμὲ δὲ οὐκ ἐπερώτησαν τοῦ βοηθηθῆναι ὑπὸ Φαραὼ, καὶ σκεπασθῆναι ὑπὸ Αἰγυπτίων.
\VS{3}Ἔσται γὰρ ὑμῖν σκέπη Φαραὼ εἰς αἰσχύνην, καὶ τοῖς πεποιθόσιν ἐπʼ Αἴγυπτον ὄνειδος.
\VS{4}Ὅτι εἰσὶν ἐν Τάνει ἀρχηγοὶ ἄγγελοι πονηροί.
\VS{5}Μάτην κοπιάσουσι πρὸς λαὸν, ὃς οὐκ ὠφελήσει αὐτοὺς εἰς βοήθειαν, ἀλλὰ εἰς αἰσχύνην καὶ ὄνειδος.
\par }{\PP Ἡ ὍΡΑΣΙΣ ΤΩΝ ΤΕΤΡΑΠΟΔΩΝ ΤΩΝ ἘΝ ΤΗ ἘΡΗΜΩ.
\par }{\PP \VS{6}Ἐν τῇ θλίψει καὶ τῇ στενοχωρίᾳ, λέων καὶ σκύμνος λέοντος, ἐκεῖθεν καὶ ἀσπίδες, καὶ ἔκγονα ἀσπίδων πετομένων, οἳ ἔφερον ἐπὶ ὄνων καὶ καμήλων τὸν πλοῦτον αὐτῶν πρὸς ἔθνος, ὃ οὐκ ὠφελήσει αὐτούς.
\VS{7}Αἰγύπτιοι μάταια καὶ κενὰ ὠφελήσουσιν ὑμᾶς· ἀπάγγειλον αὐτοῖς, ὅτι ματαία ἡ παράκλησις ὑμῶν αὕτη.
\par }{\PP \VS{8}Νῦν οὖν καθίσας γράψον ἐπὶ πυξίου ταῦτα καὶ εἰς βιβλίον, ὅτι ἔσται εἰς ἡμέρας ταῦτα καιρῷ, καὶ ἕως εἰς τὸν αἰῶνα.
\VS{9}Ὅτι ὁ λαὸς ἀπειθής ἐστιν, υἱοὶ ψευδεῖς, οἳ οὐκ ἠβούλοντο ἀκούειν τὸν νόμον τοῦ Θεοῦ·
\VS{10}Οἱ λέγοντες τοῖς προφήταις, μὴ ἀναγγέλλετε ἡμῖν, καὶ τοῖς τὰ ὁράματα ὁρῶσι, μὴ λαλεῖτε ἡμῖν, ἀλλὰ ἡμῖν λαλεῖτε καὶ ἀναγγέλλετε ἡμῖν ἑτέραν πλάνησιν,
\VS{11}καὶ ἀποστρέψατε ἡμᾶς ἀπὸ τῆς ὁδοῦ ταύτης· ἀφέλετε ἀφʼ ἡμῶν τὸν τρίβον τοῦτον, καὶ ἀφέλετε ἀφʼ ἡμῶν τὸ λόγιον τοῦ Ἰσραήλ.
\par }{\PP \VS{12}Διατοῦτο τάδε λέγει ὁ ἅγιος τοῦ Ἰσραήλ, ὅτι ἠπειθήσατε τοῖς λόγοις τούτοις, καὶ ἠλπίσατε ἐπὶ ψεύδει, καὶ ὅτι ἐγόγγυσας, καὶ πεποιθὼς ἐγένου ἐπὶ τῷ λόγῳ τούτῳ,
\VS{13}διατοῦτο ἔσται ὑμῖν ἡ ἁμαρτία αὕτη, ὡς τεῖχος πίπτον παραχρῆμα πόλεως ὀχυρᾶς ἑαλωκυίας, ἧς παραχρῆμα πάρεστι τὸ πτῶμα·
\VS{14}Καὶ τὸ πτῶμα αὐτῆς ἔσται ὡς σύντριμμα ἀγγείου ὀστρακίνου, ἐκ κεραμίου λεπτὰ, ὥστε μὴ εὑρεῖν ἐν αὐτοῖς ὄστρακον, ἐν ᾧ πῦρ ἀρεῖς, καὶ ἐν ᾧ ἀποσυριεῖς ὕδωρ μικρόν.
\par }{\PP \VS{15}Οὕτω λέγει Κύριος, Κύριος ὁ ἅγιος τοῦ Ἰσραήλ, ὅταν ἀποστραφεὶς στενάξῃς, τότε σωθήσῃ, καὶ γνώσῃ ποῦ ἦσθα, ὅτε ἐπεποίθεις ἐπὶ τοῖς ματαίοις, ματαία ἡ ἰσχὺς ὑμῶν ἐγενήθη· καὶ οὐκ ἠβούλεσθε ἀκούειν,
\VS{16}ἀλλʼ εἴπατε, ἐφʼ ἵππων φευξόμεθα· διατοῦτο φεύξεσθε· καὶ ἐπὶ κούφοις ἀναβάταις ἐσόμεθα· διατοῦτο κοῦφοι ἔσονται οἱ διώκοντες ὑμᾶς.
\VS{17}Χίλιοι διὰ φωνὴν ἑνὸς φεύξονται, καὶ διὰ φωνὴν πέντε φεύξονται πολλοί, ἕως ἂν καταλειφθῆτε ὡς ἱστὸς ἐπʼ ὄρους, καὶ ὡς σημαῖαν φέρων ἐπὶ βουνοῦ.
\par }{\PP \VS{18}Καὶ πάλιν μενεῖ ὁ Θεὸς τοῦ οἰκτειρῆσαι ὑμᾶς, καὶ διατοῦτο ὑψωθήσεται τοῦ ἐλεῆσαι ὑμᾶς· διότι κριτὴς Κύριος ὁ Θεὸς ὑμῶν· μακάριοι οἱ ἐμμένοντες ἐπʼ αὐτῷ.
\par }{\PP \VS{19}Διότι λαὸς ἅγιος ἐν Σιὼν οἰκήσει· καὶ Ἰερουσαλὴμ κλαυθμῷ ἔκλαυσεν, ἐλέησόν με· ἐλεήσει σε, τὴν φωνὴν τῆς κραυγῆς σου ἡνίκα εἶδεν, ἐπήκουσέ σου.
\VS{20}Καὶ δώσει Κύριος ὑμῖν ἄρτον θλίψεως, καὶ ὕδωρ στενὸν, καὶ οὐκ ἔτι μὴ ἐγγίσωσί σοι οἱ πλανῶντές σε· ὅτι οἱ ὀφθαλμοί σου ὄψονται τοὺς πλανῶντάς σε,
\VS{21}καὶ τὰ ὦτά σου ἀκούσονται τοὺς λόγους τῶν ὀπίσω σε πλανησάντων, οἱ λέγοντες, αὕτη ἡ ὁδὸς, πορευθῶμεν ἐν αὐτῇ, εἴτε δεξιὰ εἴτε ἀριστερά.
\VS{22}Καὶ μιανεῖς τὰ εἴδωλα τὰ περιηργυρωμένα, καὶ περικεχρυσωμένα λεπτὰ ποιήσῃς, καὶ λικμήσῃς ὡς ὕδωρ ἀποκαθημένης, καὶ ὡς κόπρον ὤσεις αὐτά.
\VS{23}Τότε ἔσται ὁ ὑετὸς τῷ σπέρματι τῆς γῆς σου, καὶ ὁ ἄρτος τοῦ γενήματος τῆς γῆς σου ἔσται πλησμονὴ καὶ λιπαρός· καὶ βοσκηθήσεταί σου τὰ κτήνη τῇ ἡμέρᾳ ἐκείνῃ τόπον πίονα καὶ εὐρύχωρον.
\VS{24}Οἱ ταῦροι ὑμῶν καὶ οἱ βόες οἱ ἐργαζόμενοι τὴν γῆν, φάγονται ἄχυρα ἀναπεποιημένα ἐν κριθῇ λελικμημένῃ.
\VS{25}Καὶ ἔσται ἐπὶ παντὸς ὄρους ὑψηλοῦ, καὶ ἐπὶ παντὸς βουνοῦ μετεώρου ὕδωρ διαπορευόμενον ἐν τῇ ἡμέρᾳ ἐκείνῃ, ὅταν ἀπόλωνται πολλοὶ, καὶ ὅταν πέσωσι πύργοι.
\VS{26}Καὶ ἔσται τὸ φῶς τῆς σελήνης ὡς τὸ φῶς τοῦ ἡλίου, καὶ τὸ φῶς τοῦ ἡλίου ἔσται ἑπταπλάσιον ἐν τῇ ἡμέρα, ὅταν ἰάσηται Κύριος τὸ σύντριμμα τοῦ λαοῦ αὐτοῦ, καὶ τὴν ὀδύνην τῆς πληγῆς σου ἰάσεται.
\par }{\PP \VS{27}Ἰδοὺ τὸ ὄνομα Κυρίου ἔρχεται διὰ χρόνου, καιόμενος θυμός· μετὰ δόξης τὸ λόγιον τῶν χειλέων αὐτοῦ, λόγιον ὀργῆς πλῆρες, καὶ ἡ ὀργὴ τοῦ θυμοῦ ὡς πῦρ ἔδεται.
\VS{28}Καὶ τὸ πνεῦμα αὐτοῦ ὡς ὕδωρ ἐν φάραγγι σὺρον, ἥξει ἕως τοῦ τραχήλου, καὶ διαιρεθήσεται, τοῦ ταράξαι ἔθνη ἐπὶ πλανήσει ματαίᾳ, καὶ διώξεται αὐτοὺς πλάνησις, καὶ λήψεται αὐτοὺς κατὰ πρόσωπον αὐτῶν.
\VS{29}Μὴ διαπαντὸς δεῖ ὑμᾶς εὐφραίνεσθαι, καὶ εἰσπορεύεσθαι εἰς τὰ ἅγιά μου διαπαντὸς, ὡσεὶ ἑορτάζοντας, καὶ ὡσεὶ εὐφραινομένους εἰσελθεῖν μετὰ αὐλοῦ εἰς τὸ ὄρος Κυρίου πρὸς τὸν Θεὸν τοῦ Ἰσραήλ;
\VS{30}Καὶ ἀκουστὴν ποιήσει Κύριος τὴν δόξαν τῆς φωνῆς αὐτοῦ, καὶ τὸν θυμὸν τοῦ βραχίονος αὐτοῦ, δεῖξαι μετὰ θυμοῦ καὶ ὀργῆς, καὶ φλογὸς κατεσθιούσης, κεραυνώσει βιαίως, καὶ ὡς ὕδωρ καὶ χάλαζα συγκαταφερομένη βίᾳ.
\VS{31}Διὰ γὰρ τῆς φωνῆς Κυρίου ἡττηθήσονται Ἀσσύριοι, τῇ πληγῇ ᾗ ἂν πατάξῃ αὐτούς.
\VS{32}Καὶ ἔσται αὐτῷ κυκλόθεν, ὅθεν ἦν αὐτῶν ἡ ἐλπὶς τῆς βοηθείας, ἐφʼ ᾗ αὐτὸς ἐπεποίθει, αὐτοὶ μετὰ τυμπάνων καὶ κιθάρας πολεμήσουσιν αὐτὸν ἐκ μεταβολῆς.
\VS{33}Σὺ γὰρ πρὸ ἡμερῶν ἀπαιτηθήσῃ· μὴ καί σοὶ ἡτοιμάσθη βασιλεύειν; φάραγγα βαθείαν, ξύλα κείμενα, πῦρ καὶ ξύλα πολλὰ, ὁ θυμὸς Κυρίου ὡς φάραγξ ὑπὸ θείου καιομένη.

\par }\Chap{31}{\PP \VerseOne{1}Οὐαὶ οἱ καταβαίνοντες εἰς Αἴγυπτον ἐπὶ βοήθειαν, οἱ ἐφʼ ἵπποις πεποιθότες καὶ ἐφʼ ἅρμασιν, ἔστι γὰρ πολλὰ, καὶ ἐφʼ ἵπποις πλῆθος σφόδρα· καὶ οὐκ ἦσαν πεποιθότες ἐπὶ τὸν ἅγιον τοῦ Ἰσραὴλ, καὶ τὸν κύριον οὐκ ἐζήτησαν·
\VS{2}Καὶ αὐτὸς σοφῶς ἦγεν ἐπʼ αὐτοὺς κακά, καὶ ὁ λόγος αὐτοῦ οὐ μὴ ἀθετηθῇ, καὶ ἐπαναστήσεται ἐπʼ οἴκους ἀνθρώπων πονηρῶν, καὶ ἐπὶ τὴν ἐλπίδα αὐτῶν τὴν ματαίαν,
\VS{3}Αἰγύπτιον, ἄνθρωπον καὶ οὐ Θεὸν, ἵππων σάρκας, καὶ οὐκ ἔστι βοήθεια· ὁ δὲ Κύριος ἐπάξει τὴν χεῖρα αὐτοῦ ἐπʼ αὐτούς· καὶ κοπιάσουσιν οἱ βοηθοῦντες, καὶ ἅμα πάντες ἀπολοῦνται.
\VS{4}Ὅτι οὕτως εἶπέ μοι Κύριος, ὃν τρόπον βοήσῃ ὁ λέων, ἢ ὁ σκύμνος ἐπὶ τῇ θήρᾳ ᾗ ἔλαβε, καὶ κεκράξῃ ἐπʼ αὐτῇ, ἕως ἂν ἐμπλησθῇ τὰ ὄρη τῆς φωνῆς αὐτοῦ, καὶ ἡττήθησαν, καὶ τὸ πλῆθος τοῦ θυμοῦ ἐπτοήθησαν, οὕτως καταβήσεται Κύριος σαβαὼθ ἐπιστρατεῦσαι ἐπὶ τὸ ὄρος τὸ Σιὼν, ἐπὶ τὰ ὄρη αὐτῆς.
\VS{5}Ὡς ὄρνεα πετόμενα, οὕτως ὑπερασπιεῖ Κύριος σαβαὼθ, ὑπὲρ Ἱερουσαλὴμ ὑπερασπιεῖ, καὶ ἐξελεῖται, καὶ περιποιήσεται, καὶ σώσει.
\VS{6}Ἐπιστράφητε οἱ τὴν βαθείαν βουλὴν βουλευόμενοι καὶ ἄνομον, υἱοὶ Ἰσραήλ.
\VS{7}Ὅτι τῇ ἡμέρᾳ ἐκείνῃ ἀπαρνήσονται οἱ ἄνθρωποι τὰ χειροποίητα αὐτῶν τὰ ἀργυρᾶ, καὶ τὰ χειροποίητα τὰ χρυσᾶ, ἃ ἐποίησαν αἱ χεῖρες αὐτῶν.
\VS{8}Καὶ πεσεῖται Ασσούρ· οὐ μάχαιρα ἀνδρὸς, οὐδὲ μάχαιρα ἀνθρώπου καταφάγεται αὐτὸν, καὶ φεύξεται οὐκ ἀπὸ προσώπου μαχαίρας· οἱ δὲ νεανίσκοι ἔσονται εἰς ἥττημα,
\VS{9}πέτρᾳ γὰρ περιληφθήσονται ὡς χάρακι, καὶ ἡττηθήσονται, ὁ δὲ φεύγων ἁλώσεται· τάδε λέγει Κύριος, μακάριος ὃς ἔχει ἐν Σιὼν σπέρμα, καὶ οἰκείους ἐν Ἱερουσαλήμ.

\par }\Chap{32}{\PP \VerseOne{1}Ἰδοὺ γὰρ βασιλεὺς δίκαιος βασιλεύσει, καὶ ἄρχοντες μετὰ κρίσεως ἄρξουσι.
\VS{2}Καὶ ἔσται ὁ ἄνθρωπος κρύπτων τοὺς λόγους αὐτοῦ, καὶ κρυβήσεται, ὡς ἀφʼ ὕδατος φερομένου· καὶ φανήσεται ἐν Σειὼν ὡς ποταμὸς φερόμενος ἔνδοξος ἐν γῇ διψώσῃ.
\VS{3}Καὶ οὐκέτι ἔσονται πεποιθότες ἐπʼ ἀνθρώποις, ἀλλὰ τὰ ὦτα ἀκούειν δώσουσι.
\VS{4}Καὶ ἡ καρδία τῶν ἀσθενούντων προσέξει τῷ ἀκούειν, καὶ αἱ γλῶσσαι αἱ ψελλίζουσαι ταχὺ μαθήσονται λαλεῖν εἰρήνην.
\VS{5}Καὶ οὐκέτι μὴ εἴπωσι τῷ μωρῷ ἄρχειν, καὶ οὐκέτι μὴ εἴπωσιν οἱ ὑπηρέται σου, σίγα.
\VS{6}Ὁ γὰρ μωρὸς μωρὰ λαλήσει, καὶ ἡ καρδία αὐτοῦ μάταια νοήσει, τοῦ συντελεῖν ἄνομα, καὶ λαλεῖν πρὸς Κύριον πλάνησιν, τοῦ διασπεῖραι ψυχὰς πεινώσας, καὶ τὰς ψυχὰς τὰς διψώσας κενὰς ποιήσει.
\VS{7}Ἡ γὰρ βουλὴ τῶν πονηρῶν ἄνομα βουλεύσεται, καταφθεῖραι ταπεινοὺς ἐν λόγοις ἀδίκοις, καὶ διασκεδάσαι λόγους ταπεινῶν ἐν κρίσει.
\VS{8}Οἱ δὲ εὐσεβεῖς συνετὰ ἐβουλεύσαντο, καὶ αὕτη ἡ βουλὴ μενεῖ.
\par }{\PP \VS{9}Γυναῖκες πλούσιαι ἀνάστητε, καὶ ἀκούσατε τῆς φωνῆς μου· θυγατέρες ἐν ἐλπίδι εἰσακούσατε λόγους μου.
\VS{10}Ἡμέρας ἐνιαυτοῦ μνείαν ποιήσασθε ἐν ὀδύνῃ μετʼ ἐλπίδος· ἀνήλωται ὁ τρυγητὸς, πέπαυται, οὐκέτι μὴ ἔλθῃ.
\VS{11}Ἔκστητε, λυπήθητε αἱ πεποιθυῖαι, ἐκδύσασθε, γυμναὶ γένεσθε, περιζώσασθε τὰς ὀσφῦας,
\VS{12}καὶ ἐπὶ τῶν μαστῶν κόπτεσθε, ἀπὸ ἀγροῦ ἐπιθυμήματος, καὶ ἀμπέλου γεννήματος.
\VS{13}Ἡ γῆ τοῦ λαοῦ μου, ἄκανθα καὶ χόρτος ἀναβήσεται, καὶ ἐκ πάσης οἰκίας εὐφροσύνη ἀρθήσεται.
\VS{14}Πόλις πλουσία, οἶκοι ἐγκαταλελειμμένοι πλοῦτον πόλεως ἀφήσουσιν, οἴκους ἐπιθυμήματος· καὶ ἔσονται αἱ κῶμαι σπήλαια ἕως τοῦ αἰῶνος, εὐφροσύνη ὄνων ἀγρίων, βοσκήματα ποιμένων,
\VS{15}ἕως ἂν ἔλθῃ ἐφʼ ὑμᾶς πνεῦμα ἀφʼ ὑψηλοῦ· καὶ ἔσται ἔρημος ὁ Χέρμελ, καὶ ὁ Χέρμελ εἰς δρυμὸν λογισθήσεται.
\VS{16}Καὶ ἀναπαύσεται ἐν τῇ ἐρήμῳ κρίμα, καὶ δικαιοσύνη ἐν τῷ Καρμήλῳ κατοικήσει.
\VS{17}Καὶ ἔσται τὰ ἔργα τῆς δικαιοσύνης, εἰρήνη· καὶ κρατήσει ἡ δικαιοσύνη ἀνάπαυσιν, καὶ πεποιθότες ἕως τοῦ αἰῶνος.
\VS{18}Καὶ κατοικήσει ὁ λαὸς αὐτοῦ ἐν πόλει εἰρήνης, καὶ ἐνοικήσει πεποιθὼς, καὶ ἀναπαύσονται μετὰ πλούτου.
\VS{19}Ἡ δὲ χάλαζα ἐὰν καταβῇ, οὐκ ἐφʼ ὑμᾶς ἥξει· καὶ ἔσονται οἱ ἐνοικοῦντες ἐν τοῖς δρυμοῖς πεποιθότες, ὡς οἱ ἐν τῷ πεδινῇ.
\VS{20}Μακάριοι οἱ σπείροντες ἐπὶ πᾶν ὕδωρ, οὗ βοῦς καὶ ὄνος πατεῖ.

\par }\Chap{33}{\PP \VerseOne{1}Οὐαὶ τοῖς ταλαιπωροῦσιν ὑμᾶς, ὑμᾶς δὲ οὐδεὶς ποιεῖ ταλαιπώρους, καὶ ὁ ἀθετῶν ὑμᾶς οὐκ ἀθετεῖ· ἁλώσονται οἱ ἀθετοῦντες, καὶ παραδοθήσονται, καὶ ὡς σὴς ἐφʼ ἱματίου, οὕτως ἡττηθήσονται.
\par }{\PP \VS{2}Κύριε ἐλέησον ἡμᾶς, ἐπὶ σοὶ γὰρ πεποίθαμεν· ἐγενήθη τὸ σπέρμα τῶν ἀπειθούντων εἰς ἀπώλειαν, ἡ δὲ σωτηρία ἡμῶν ἐν καιρῷ θλίψεως.
\VS{3}Διὰ φωνὴν τοῦ φόβου ἐξέστησαν λαοὶ ἀπὸ τοῦ φόβου σου, καὶ διεσπάρησαν τὰ ἔθνη.
\par }{\PP \VS{4}Νῦν δὲ συναχθήσεται τὰ σκῦλα ὑμῶν μικροῦ καὶ μεγάλου· ὃν τρόπον ἐάν τις συναγάγῃ ἀκρίδας, οὕτως ἐμπαίξουσιν ὑμῖν.
\VS{5}Ἅγιος ὁ Θεὸς ὁ κατοικῶν ἐν ὑψηλῷ, ἔνεπλήσθη Σιὼν κρίσεως καὶ δικαιοσύνης.
\VS{6}Ἐν νόμῳ παραδοθήσονται, ἐν θησαυροῖς ἡ σωτηρία ἡμῶν, ἐκεῖ σοφία καὶ ἐπιστήμη καὶ εὐσέβεια πρὸς τὸν κύριον· οὗτοί εἰσι θησαυροὶ δικαιοσύνης.
\par }{\PP \VS{7}Ἰδοὺ δὴ ἐν τῷ φόβῳ ὑμῶν οὗτοι φοβηθήσονται· οὓς ἐφοβεῖσθε, βοήσονται ἀφʼ ὑμῶν· ἄγγελοι ἀποσταλήσονται, πικρῶς κλαίοντες, παρακαλοῦντες εἰρήνην.
\VS{8}Ἐρημωθήσονται γὰρ αἱ τούτων ὁδοί· πέπαυται ὁ φόβος τῶν ἐθνῶν, καὶ ἡ πρὸς τούτους διαθήκη αἴρεται, καὶ οὐ μὴ λογίσησθε αὐτοὺς ἀνθρώπους.
\VS{9}Ἐπένθησεν ἡ γῆ, ᾐσχύνθη ὁ Λίβανος, ἕλη ἐγένετο ὁ Σάρων· φανερὰ ἔσται ἡ Γαλιλαία, καὶ ὁ Χέρμελ.
\par }{\PP \VS{10}Νῦν ἀναστήσομαι, λέγει Κύριος, νῦν δοξασθήσομαι, νῦν ὑψωθήσομαι.
\VS{11}Νῦν ὄψεσθε, νῦν αἰσθηθήσεσθε, ματαία ἔσται ἡ ἰσχὺς τοῦ πνεύματος ὑμῶν· πῦρ κατέδεται ὑμᾶς.
\VS{12}Καὶ ἔσονται ἔθνη κατακεκαυμένα, ὡς ἄκανθα ἐν ἀγρῷ ἐῤῥιμμένη καὶ κατακεκαυμένη.
\par }{\PP \VS{13}Ἀκούσονται οἱ πόῤῥωθεν ἃ ἐποίησα, γνώσονται οἱ ἐγγίζοντες τὴν ἰσχύν μου.
\VS{14}Ἀπέστησαν οἱ ἐν Σιὼν ἄνομοι, λήψεται τρόμος τοὺς ἀσεβεῖς· τίς ἀναγγελεῖ ὑμῖν, ὅτι πῦρ καίεται; τίς ἀναγγελεῖ ὑμῖν τὸν τόπον τὸν αἰώνιον;
\par }{\PP \VS{15}Πορευόμενος ἐν δικαιοσύνῃ, λαλῶν εὐθεῖαν ὁδόν, μισῶν ἀνομίαν καὶ ἀδικίαν, καὶ τὰς χεῖρας ἀποσειόμενος ἀπὸ δώρων· βαρύνων τὰ ὦτα, ἵνα μὴ ἀκούσῃ κρίσιν αἵματος· καμμύων τοὺς ὀφθαλμοὺς, ἵνα μὴ ἴδῃ ἀδικίαν,
\VS{16}οὗτος οἰκήσει ἐν ὑψηλῷ σπηλαίῳ πέτρας ἰσχυρᾶς· ἄρτος αὐτῷ δοθήσεται, καὶ τὸ ὕδωρ αὐτοῦ πιστόν.
\VS{17}Βασιλέα μετὰ δόξης ὄψεσθε, οἱ ὀφθαλμοὶ ὑμῶν ὄψονται γῆν πόῤῥωθεν,
\VS{18}ἡ ψυχὴ ἡμῶν μελετήσει φόβον· ποῦ εἰσιν οἱ γραμματικοί; ποῦ εἰσιν οἱ συμβουλεύοντες; ποῦ ἐστιν ὁ ἀριθμῶν τοὺς τρεφομένους
\VS{19}μικρὸν καὶ μέγαν λαόν; ᾧ οὐ συνεβουλεύσατο, οὐσὲ ᾔδει βαθύφωνον, ὥστε μὴ ἀκοῦσαι λαὸς πεφαυλισμένος, καὶ οὐκ ἔστι τῷ ἀκούοντι σύνεσις.
\par }{\PP \VS{20}Ἰδοὺ Σιὼν ἡ πόλις, τὸ σωτήριον ἡμῶν, οἱ ὀφθαλμοί σου ὄψονται Ἱερουσαλὴμ, πόλις πλουσία, σκηναὶ αἳ οὐ μὴ σεισθῶσιν, οὐδὲ μὴ κινηθῶσιν οἱ πάσσαλοι τῆς σκηνῆς αὐτῆς εἰς τὸν αἰῶνα χρόνον, οὐδὲ τὰ σχοινία αὐτῆς οὐ μὴ διαῤῥαγῶσιν·
\VS{21}ὅτι τὸ ὄνομα Κυρίου μέγα ὑμῖν· τόπος ὑμῖν ἔσται, ποταμοὶ καὶ διώρυχες πλατεῖς καὶ εὐρύχωροι· οὐ πορεύσῃ ταύτην τὴν ὁδὸν, οὐδὲ πορεύσεται πλοῖον ἐλαύνον.
\VS{22}Ὁ γὰρ Θεός μου μέγας ἐστίν· οὐ παρελεύσεταί με Κύριος κριτὴς ἡμῶν, Κύριος ἄρχων ἡμῶν, Κύριος βασιλεὺς ἡμῶν, Κύριος οὗτος ἡμᾶς σώσει.
\par }{\PP \VS{23}Ἐῤῥάγησαν τὰ σχοινία σου, ὅτι οὐκ ἐνίσχυσαν· ὁ ἱστός σου ἔκλινεν, οὐ χαλάσει τὰ ἱστία, οὐκ ἀρεῖ σημεῖον, ἕως οὗ παραδοθῇ εἰς προνομήν· τοίνυν πολλοὶ χωλοὶ προνομὴν ποιήσουσι,
\VS{24}καὶ οὐ μὴ εἴπωσι, Κοπιῶ ὁ λαὸς ἐνοικῶν ἐν αὐτοῖς· ἀφεθῆ γὰρ αὐτοῖς ἡ ἁμαρτία.

\par }\Chap{34}{\PP \VerseOne{1}Προσαγάγετε ἔθνη, καὶ ἀκούσατε ἄρχοντες· ἀκουσάτω ἡ γῆ, καὶ οἱ ἐν αὐτῇ, ἡ οἰκουμένη, καὶ ὁ λαὸς ὁ ἐν αὐτῇ.
\VS{2}Διότι θυμὸς Κυρίου ἐπὶ πάντα τὰ ἔθνη καὶ ὀργὴ ἐπὶ τὸν ἀριθμὸν αὐτῶν, τοῦ ἀπολέσαι αὐτοὺς, καὶ παραδοῦναι αὐτοὺς εἰς σφαγήν.
\VS{3}Οἱ δὲ τραυματίαι αὐτῶν ῥιφήσονται, καὶ οἱ νεκροί, καὶ ἀναβήσεται αὐτῶν ἡ ὀσμή, καὶ βραχήσεται τὰ ὄρη ἀπὸ τοῦ αἵματος αὐτῶν.
\VS{4}Καὶ τακήσονται πᾶσαι αἱ δυνάμεις τῶν οὐρανῶν, καὶ ἑλιγήσεται ὁ οὐρανός ὡς βιβλίον, καὶ πάντα τὰ ἄστρα πεσεῖται ὡς φύλλα ἐξ ἀμπέλου, καὶ ὡς πίπτει φύλλα ἀπὸ συκῆς.
\par }{\PP \VS{5}Ἐμεθύσθη ἡ μάχαιρὰ μου ἐν τῷ οὐρανῷ· ἰδοὺ ἐπὶ τὴν Ἰδουμαίαν καταβήσεται, καὶ ἐπὶ τὸν λαὸν τῆς ἀπωλείας μετὰ κρίσεως.
\VS{6}Ἡ μάχαιρα τοῦ Κυρίου ἐνεπλήσθη αἵματος, ἐπαχύνθη ἀπὸ στέατος, ἀπὸ αἵματος τράγων καὶ ἀμνῶν, καὶ ἀπὸ στέατος τράγων καὶ κριῶν· ὅτι θυσία τῷ κυρίῳ ἐν Βοσὸρ, καὶ σφαγὴ μεγάλη ἐν τῇ Ἰδουμαίᾳ.
\VS{7}Καὶ συνπεσοῦνται οἱ ἁδροὶ μετʼ αὐτῶν, καὶ οἱ κριοὶ καὶ οἱ ταῦροι, καὶ μεθυσθήσεται ἡ γῆ ἀπὸ τοῦ αἵματος, καὶ ἀπὸ τοῦ στέατος αὐτῶν ἐμπλησθήσεται.
\VS{8}Ἡμέρα γὰρ κρίσεως Κυρίου, καὶ ἐνιαυτὸς ἀνταποδόσεως κρίσεως Σιών.
\VS{9}Καὶ στραφήσονται αἱ φάραγγες αὐτῆς εἰς πίσσαν, καὶ ἡ γῆ αὐτῆς εἰς θεῖον· καὶ ἔσται ἡ γῆ αὐτῆς ὡς πίσσα καιομένη
\VS{10}νυκτὸς καὶ ἡμέρας, καὶ οὐ σβεσθήσεται εἰς τὸν αἰῶνα χρόνον, καὶ ἀναβήσεται ὁ καπνὸς αὐτῆς ἄνω, εἰς γενεὰς αὐτῆς ἐρημωθήσεται, καὶ εἰς χρόνον πολὺν
\VS{11}ὄρνεα καὶ ἐχῖνοι, καὶ ἴβεις καὶ κόρακες κατοικήσουσιν ἐν αὐτῇ· καὶ ἐπιβληθήσεται ἐπʼ αὐτὴν σπαρτίον γεωμετρίας ἐρήμου, καὶ ὀνοκένταυροι οἰκήσουσιν ἐν αὐτῇ.
\VS{12}Οἱ ἄρχοντες αὐτῆς οὐκ ἔσονται· οἱ γὰρ βασιλεῖς καὶ οἱ μεγιστᾶνες αὐτῆς ἔσονται εἰς ἀπώλειαν.
\VS{13}Καὶ ἀναφυὴσει εἰς τὰς πόλεις αὐτῶν ἀκάνθινα ξύλα, καὶ εἰς τὰ ὀχυρώματα αὐτῆς· καὶ ἔσται ἐπαύλεις σειρήνων, καὶ αὐλὴ στρουθῶν.
\VS{14}Καὶ συναντήσουσι δαιμόνια ὀνοκενταύροις, καὶ βοήσονται ἕτερος πρὸς τὸν ἕτερον, ἐκεῖ ἀναπαύσονται ὀνοκένταυροι, εὑρόντες αὑτοῖς ἀνάπαυσιν.
\VS{15}Ἐκεῖ ἐνόσσευσεν ἐχῖνος, καὶ ἔσωσεν ἡ γῆ τὰ παιδία αὐτῆς μετὰ ἀσφαλείας· ἐκεῖ συνήντησαν ἔλαφοι καὶ εἶδον τὰ πρόσωπα ἀλλήλων.
\VS{16}Ἀριθμῷ παρῆλθον, καὶ μία αὐτῶν οὐκ ἀπώλετο· ἑτέρα τὴν ἑτέραν οὐκ ἐζήτησαν, ὅτι ὁ Κύριος αὐτοῖς ἐνετείλατο, καὶ τὸ πνεῦμα αὐτοῦ συνήγαγεν αὐτά.
\VS{17}Καὶ αὐτὸς ἐπιβαλεῖ αὐτοῖς κλήρους, καὶ ἡ χεὶρ αὐτοῦ διεμέρισε βόσκεσθαι· εἰς τὸν αἰῶνα χρόνον κληρονομήσετε, γενεὰς γενεῶν ἀναπαύσονται ἐπʼ αὐτῆς.

\par }\Chap{35}{\PP \VerseOne{1}Εὐφράνθητι ἔρημος διψῶσα, ἀγαλλιάσθω ἔρημος, καὶ ἀνθείτω ὡς κρίνον.
\VS{2}Καὶ ἐξανθήσει καὶ ἀγαλλιάσεται τὰ ἔρημα τοῦ Ἰορδανου, ἡ δόξα τοῦ Λιβάνου ἐδόθη αὐτῇ, καὶ ἡ τιμὴ τοῦ Καρμήλου, καὶ ὁ λαός μου ὄψεται τὴν δόξαν Κυρίου, καὶ τὸ ὕψος τοῦ Θεοῦ.
\par }{\PP \VS{3}Ἰσχύσατε χεῖρες ἀνειμέναι, καὶ γόνατα παραλελυμένα.
\VS{4}Παρακαλέσατε οἱ ὀλιγόψυχοι τῇ διανοίᾳ· ἰσχύσατε, μὴ φοβεῖσθε· ἰδοὺ ὁ Θεὸς ἡμῶν κρίσιν ἀνταποδίδωσι, καὶ ἀνταποδώσει, αὐτὸς ἥξει καὶ σώσει ἡμᾶς.
\VS{5}Τότε ἀνοιχθήσονται ὀφθαλμοὶ τυφλῶν, καὶ ὦτα κωφῶν ἀκούσονται.
\VS{6}Τότε ἁλεῖται ὡς ἔλαφος ὁ χωλὸς, τρανὴ δὲ ἔσται γλῶσσα μογιλάλων, ὅτι ἐῤῥάγη ἐν τῇ ἐρήμῳ ὕδωρ, καὶ φάραγξ ἐν γῇ διψώσῃ.
\VS{7}Καὶ ἔσται ἡ ἄνυδρος εἰς ἕλη, καὶ εἰς τὴν διψῶσαν γῆν πηγὴ ὕδατος ἔσται· ἐκεῖ εὐφροσύνη ὀρνέων, ἐπαύλεις καλάμου καὶ ἕλη.
\VS{8}Ἔσται ἐκεῖ ὁδὸς καθαρά, καὶ ὁδὸς ἁγία κληθήσεται, καὶ οὐ μὴ παρέλθῃ ἐκεῖ ἀκάθαρτος, οὐδὲ ἔσται ἐκεῖ ὁδὸς ἀκάθαρτος· οἱ δὲ διεσπαρμένοι πορεύσονται ἐπʼ αὐτῆς, καὶ οὐ μὴ πλανηθῶσι.
\VS{9}Καὶ οὐκ ἔσται ἐκεῖ λέων, οὐδὲ τῶν πονηρῶν θηρίων οὐ μὴ ἀναβῇ εἰς αὐτὴν, οὐδὲ μὴ εὑρεθῇ ἐκεῖ, ἀλλὰ πορεύσονται ἐν αὐτῇ λελυτρωμένοι,
\VS{10}Καὶ συνηγμένοι διὰ Κύριον, καὶ ἀποστραφήσονται, καὶ ἥξουσιν εἰς Σιὼν μετʼ εὐφροσύνης, καὶ εὐφροσύνη αἰώνιος ὑπὲρ κεφαλῆς αὐτῶν· ἐπὶ γὰρ τῆς κεφαλῆς αὐτῶν αἴνεσις καὶ ἀγαλλίαμα, καὶ εὐφροσύνη καταλήμψεται αὐτοὺς, ἀπέδρα ὀδύμη καὶ λύπη καὶ στεναγμός.

\par }\Chap{36}{\PP \VerseOne{1}Καὶ ἐγένετο τοῦ τεσσαρεσκαιδεκάτου ἔτους βασιλεύοντος Ἑζεκίου, ἀνέβη Σενναχηρεὶμ βασιλεὺς Ἀσσυρίων ἐπὶ τὰς πόλεις τῆς Ἰουδαίας τὰς ὀχυρὰς, καὶ ἔλαβεν αὐτάς.
\VS{2}Καὶ ἀπέστειλε βασιλεὺς Ἀσσυρίων τὸν Ῥαβσάκην ἐκ Λάχης εἰς Ἱερουσαλὴμ πρὸς τὸν βασιλέα Ἐζεκίαν μετὰ δυνάμεως πολλῆς· καὶ ἔστη ἐν τῷ ὑδραγωγῷ τῆς κολυμβήθρας τῆς ἄνω ἐν τῇ ὁδῷ τοῦ ἀγροῦ τοῦ κναφέως.
\VS{3}Καὶ ἐξῆλθε πρὸς αὐτὸν Ἑλιακεὶμ ὁ τοῦ Χελκίου ὁ οἰκονόμος, καὶ Σόμνᾶς ὁ γραμματεὺς, καὶ Ἰωὰχ ὁ τοῦ Ἀσὰφ ὁ ὑπομνηματογράφος.
\par }{\PP \VS{4}Καὶ εἶπεν αὐτοῖς Ῥαβσάκης, εἴπατε Ἐζεκίᾳ, Τάδε λέγει ὁ βασιλεὺς ὁ μέγας, βασιλεὺς Ἀσσυρίων, τί πεποιθὼς εἶ;
\VS{5}Μὴ ἐν βουλῇ καὶ λόγοις χειλέων παράταξις γίνεται; καὶ νῦν ἐπὶ τίνα πέποιθας, ὅτι ἀπειθεῖς μοι;
\VS{6}Ἰδοὺ πεποιθὼς εἶ ἐπὶ τὴν ῥάβδον τὴν καλαμίνην τὴν τεθλασμένην ταύτην, ἐπʼ Αἴγυπτον· ὡς ἂν ἐπιστηρισθῇ ἀνὴρ ἐπʼ αὐτὴν, εἰσελεύσεται εἰς τὴν χεῖρα αὐτοῦ, καὶ τρήσει αὐτὴν. οὕτως ἐστὶ Φαραὼ βασιλεὺς Αἰγύπτου, καὶ πάντες οἱ πεποιθότες ἐπʼ αὐτῷ.
\VS{7}Εἰ δὲ λέγετε, ἐπὶ Κύριον τὸν Θεὸν ἡμῶν πεποίθαμεν,
\VS{8}νῦν μίχθητε τῷ κυρίῳ μου τῷ βασιλεῖ Ἀσσυρίων, καὶ δώσω ὑμῖν δισχιλίαν ἵππον, εἰ δυνήσεσθε δοῦναι ἀναβάτας ἐπʼ αὐτούς.
\VS{9}Καὶ πῶς δύνασθε ἀποστρέψαι εἰς πρόσωπον τῶν τοπαρχῶν· οἰκέται εἰσὶν, οἱ πεποιθότες ἐπʼ Αἴγυπτίοις, εἰς ἵππον καὶ ἀναβάτην.
\VS{10}Καὶ νῦν μὴ ἄνευ Κυρίου ἀνέβημεν ἐπὶ τὴν χώραν ταύτην πολεμῆσαι αὐτήν; Κύριος εἶπε πρὸς μὲ, ἀνάβηθι ἐπὶ τὴν γῆν ταύτην, καὶ διάφθειρον αὐτήν.
\par }{\PP \VS{11}Καὶ εἶπε πρὸς αὐτὸν Ἑλιακεὶμ, καὶ Σομνᾶς, καὶ Ἰωὰχ, λάλησον πρὸς τοὺς παῖδάς σου Συριστί· ἀκούομεν γὰρ ἡμεῖς· καὶ μὴ λάλει πρὸς ἡμᾶς Ἰουδαϊστί· καὶ ἱνατί λαλεῖς εἰς τὰ ὦτα τῶν ἀνθρώπων ἐπὶ τῷ τείχει;
\VS{12}Καὶ εἶπε πρὸς αὐτοὺς Ῥαβσάκης, μὴ πρὸς τὸν κύριον ὑμῶν ἢ πρὸς ὑμᾶς ἀπέσταλκέ με ὁ κύριός μου, λαλῆσαι τοὺς λόγους τούτους; οὐχὶ πρὸς τοὺς ἀνθρώπους τοὺς καθημένους ἐπὶ τῷ τείχει, ἵνα φάγωσι κόπρον, καὶ πίωσιν οὖρον μεθʼ ὑμῶν ἅμα;
\par }{\PP \VS{13}Καὶ ἔστη Ῥαβσάκης, καὶ ἀνεβόησε φωνῇ μεγάλῃ Ἰουδαϊστὶ, καὶ εἶπεν, ἀκούσατε τοὺς λόγους τοῦ βασιλέως τοῦ μεγάλου, βασιλέως Ἀσσυρίων.
\VS{14}Τάδε λέγει ὁ βασιλεύς, μὴ ἀπατάτω ὑμᾶς Ἐζεκίας λόγοις, οὐ δύνηται ῥύσασθαι ὑμᾶς.
\VS{15}Καὶ μὴ λεγέτω ὑμῖν Ἐζεκίας, ὅτι ῥύσεται ὑμᾶς ὁ Θεὸς, καὶ οὐ μὴ παραδοθῇ ἡ πόλις αὕτη ἐν χειρὶ βασιλέως Ἀσσυρίων.
\VS{16}Μὴ ἀκούετε Ἐζεκίου· τάδε λέγει ὁ βασιλεὺς Ἀσσυρίων, εἰ βούλεσθε εὐλογηθῆναι, ἐκπορεύεσθε πρὸς μέ, καὶ φάγεσθε ἕκαστος τὴν ἄμπελον αὐτοῦ καὶ τὰς συκὰς, καὶ πίεσθε ὕδωρ ἐκ τοῦ λάκκου ὑμῶν,
\VS{17}ἕως ἂν ἔλθω, καὶ λάβω ὑμᾶς εἰς γῆν, ὡς ἡ γῆ ὑμῶν, γῆ σίτου καὶ οἴνου καὶ ἄρτων καὶ ἀμπελώνων.
\VS{18}Μὴ ἀπατάτω ὑμᾶς Ἐζεκίας, λέγων, ὁ Θεὸς ῥύσεται ὑμᾶς· μὴ ἐῤῥύσαντο οἱ θεοὶ τῶν ἐθνῶν, ἕκαστος τὴν ἑαῦτοῦ χώραν ἐκ χειρὸς βασιλέως Ἀσσυρίων;
\VS{19}Ποῦ ἐστιν ὁ θεὸς Ἐμὰθ καὶ Ἀρφάθ; καὶ ποῦ ὁ θεὸς τῆς πόλεως Ἐπφαρουαίμ; μὴ ἐδύναντο ῥύσασθαι Σαμάρειαν ἐκ χειρός μου;
\VS{20}Τίς τῶν θεῶν πάντων τῶν ἐθνῶν τούτων, ὅστις ἐῤῥύσατο τὴν γῆν αὐτοῦ ἐκ χειρός μου, ὅτι ῥύσεται ὁ Θεὸς τὴν Ἱερουσαλὴμ ἐκ χειρός μου;
\VS{21}Καὶ ἐσιώπησαν, καὶ οὐδεὶς ἀπεκρίθη αὐτῷ λόγον, διὰ τὸ προστάξαι τὸν βασιλέα μηδένα ἀποκριθῆναι.
\par }{\PP \VS{22}Καὶ εἰσῆλθεν Ἑλιακεὶμ ὁ τοῦ Χελκίου, οἰκονόμος, καὶ Σομνᾶς ὁ γραμματεὺς τῆς δυνάμεως, καὶ Ἰωὰχ ὁ τοῦ Ἀσὰφ ὁ ὑπομνηματογράφος, πρὸς Ἐζεκίαν, ἐσχισμένοι τοὺς χιτῶνας, καὶ ἀνήγγειλαν αὐτῷ τοὺς λόγους Ῥαβσάκου.

\par }\Chap{37}{\PP \VerseOne{1}Καὶ ἐγένετο ἐν τῷ ἀκοῦσαι τὸν βασιλέα Ἐζεκίαν, ἔσχισε τὰ ἱμάτια, καὶ περιεβάλετο σάκκον, καὶ ἀνέβη εἰς τὸν οἶκον Κυρίου.
\par }{\PP \VS{2}Καὶ ἀπέστειλεν Ἑλιακεὶμ τὸν οἰκονόμον, καὶ Σόμνᾶν τὸν γραμματέα, καὶ τοὺς πρεσβυτέρους τῶν ἱερέων περιβεβλημένους σάκκους, πρὸς Ἠσαΐαν υἱὸν Ἀμὼς τὸν προφήτην.
\VS{3}Καὶ εἶπαν αὐτῷ, τάδε λέγει Ἐζεκίας, ἡμέρα, θλίψεως καὶ ὀνειδισμοῦ καὶ ἐλεγμοῦ καὶ ὀργῆς ἡ σήμερον ἡμέρα, ὅτι ἥκει ἡ ὠδὶν τῇ τικτούσῃ, ἰσχὺν δὲ οὐκ ἔχει τοῦ τεκεῖν.
\VS{4}Εἰσακούσαι Κύριος ὁ Θεός σου τοὺς λόγους Ῥαβσάκου, οὓς ἀπέστειλε βασιλεὺς Ἀσσυρίων, ὀνειδίζειν Θεὸν ζῶντα, καὶ ὀνειδίζειν λόγους οὓς ἤκουσε Κύριος ὁ Θεός σου, καὶ δεηθήσῃ πρὸς Κύριόν σου περὶ τῶν καταλελειμμένων τούτων.
\par }{\PP \VS{5}Καὶ ἦλθον οἱ παῖδες τοῦ βασιλέως Ἐζεκίου πρὸς Ἡσαΐαν.
\VS{6}Καὶ εἶπεν αὐτοῖς Ἡσαΐας, οὕτως ἐρεῖτε πρὸς τὸν κύριον ὑμῶν, τάδε λέγεἱ Κύριος, μὴ φοβηθῇς ἀπὸ τῶν λόγων ὧν ἤκουσας, οὓς ὠνείδισάν με οἱ πρέσβεις βασιλέως Ἀσσυρίων.
\VS{7}Ἰδοὺ ἐγὼ ἐμβάλλω εἰς αὐτὸν πνεῦμα, καὶ ἀκούσας ἀγγελίαν, ἀποστραφήσεται εἰς τὴν χώραν αὐτοῦ, καὶ πεσεῖται μαχαίρᾳ ἐν τῇ γῇ αὐτοῦ.
\par }{\PP \VS{8}Καὶ ἀπέστρεψε Ῥαβσάκης, καὶ κατέλαβε τὸν βασιλέα Ἀσσυρίων πολιορκοῦντα Λοβνάν· καὶ ἤκουσεν ὅτι ἀπῇρεν ἀπὸ Λαχίς.
\VS{9}Καὶ ἐξῆλθε Θαρακὰ βασιλεὺς Αἰθιόπων πολιορκῆσαι αὐτόν· καὶ ἀκούσας ἀπέστρεψε, καὶ ἀπέστειλεν ἀγγέλους πρὸς Ἐζεκίαν, λέγων,
\VS{10}οὕτως ἐρεῖτε Ἐζεκίᾳ βασιλεῖ τῆς Ἰουδαίας, μή σε ἀπατάτω ὁ Θεός σου, ἐφʼ ᾧ πέποιθας ἐπʼ αὐτῷ, λέγων, οὐ μὴ παραδοθῇ Ἱερουσαλὴμ ἐν χειρὶ βασιλέως Ἀσσυρίων.
\par }{\PP \VS{11}Σὺ οὐκ ἤκουσας ἃ ἐποίησαν βασιλεῖς Ἀσσυρίων, πᾶσαν τὴν γῆν ὡς ἀπώλεσαν; καὶ σὺ ῥυσθήσῃ;
\VS{12}Μὴ ἐῤῥύσαντο αὐτοὺς οἱ θεοὶ τῶν ἐθνῶν, οὓς ἀπώλεσαν οἱ πατέρες μου, τήν τε Γωζᾶν, καὶ Χαῤῥὰν, καὶ Ῥαφὲθ, αἵ εἰσιν ἐν χώρᾳ Θεεμάθ;
\VS{13}Ποῦ εἰσι βασιλεῖς Ἐμάθ; καὶ ποῦ Ἀρφάθ; καὶ ποῦ πόλεως Ἐπφαρουαὶμ, Ἀναγονγάυα;
\par }{\PP \VS{14}Καὶ ἔλαβεν Ἐζεκίας τὸ βιβλίον παρὰ τῶν ἀγγέλων, καὶ ἀνέγνω αὐτὸ, καὶ ἀνέβη εἰς οἶκον Κυρίου, καὶ ἤνοιξεν αὐτὸ ἐναντίον Κυρίου.
\VS{15}Καὶ προσηύξατο Ἐζεκίας πρὸς Κύριον, λέγων,
\par }{\PP \VS{16}Κύριος σαβαὼθ ὁ Θεὸς Ἰσραὴλ, ὁ καθήμενος ἐπὶ τῶν χερουβὶμ, σὺ εἶ ὁ Θεὸς μόνος πάσης βασιλείας τῆς οἰκουμένης, σὺ ἐποίησας τὸν οὐρανὸν καὶ τὴν γῆν·
\VS{17}Κλῖνον Κύριε τὸ οὖς σου, εἰσάκουσον Κύριε, ἄνοιξον Κύριε τοὺς ὀφθαλμούς σου, εἴσβλεψον Κύριε, καὶ ἴδε τοὺς λόγους Σενναχηρεὶμ, οὓς ἀπέστειλεν ὀνειδίζειν Θεὸν ζῶντα.
\VS{18}Ἐπʼ ἀληθείας γὰρ Κύριε ἠρήμωσαν βασιλεῖς Ἀσσυρίων τὴν οἰκουμένην ὅλην, καὶ τὴν χώραν αὐτῶν,
\VS{19}καὶ ἀνέβαλον τὰ εἴδωλα αὐτῶν εἰς τὸ πῦρ· οὐ γὰρ θεοὶ ἦσαν, ἀλλὰ ἔργα χειρῶν ἀνθρώπων, ξύλα καὶ λίθοι· καὶ ἀπώσαντο αὐτούς.
\VS{20}Νῦν δὲ Κύριε ὁ Θεὸς ἡμῶν σῶσον ἡμᾶς ἐκ χειρὸς αὐτοῦ, ἵνα γνῷ πᾶσα βασιλεία τῆς γῆς, ὅτι σῦ εἶ ὁ Θεὸς μόνος.
\par }{\PP \VS{21}Καὶ ἀπεστάλη Ἡσαΐας υἱὸς Ἀμὼς πρὸς Ἐζεκίαν, καὶ εἶπεν αὐτῷ, τάδε λέγει Κύριος ὁ Θεὸς Ἰσραὴλ, ἤκουσα ἃ προσηύξω πρὸς μὲ περὶ Σενναχηρεὶμ βασιλέως Ἀσσυρίων.
\VS{22}Οὗτος ὁ λόγος ὃν ἐλάλησε περὶ αὐτοῦ ὁ Θεὸς, ἐφαύλισέ σε, καὶ ἐμυκτήρισέ σε παρθένος θυγάτηρ Σιὼν, ἐπὶ σοὶ κεφαλὴν ἐκίνησε θυγάτηρ Ἱερουσαλήμ.
\VS{23}Τίνα ὠνείδισας καὶ παρώξυνας; ἢ πρὸς τίνα ὕψωσας τὴν φωνήν σου; καὶ οὐκ ᾖρας εἰς ὕψος τοὺς ὀφθαλμούς σου πρὸς τὸν ἅγιον τοῦ Ἰσραήλ;
\VS{24}Ὅτι διʼ ἀγγέλων ὠνείδισας Κύριον· σὺ γὰρ εἶπας, τῷ πλήθει τῶν ἁρμάτων ἐγὼ ἀνέβην εἰς ὕψος ὀρέων, καὶ εἰς τὰ ἔσχατα τοῦ Λιβάνου, καὶ ἔκοψα τὸ ὕψος τῆς κέδρου αὐτοῦ, καὶ τὸ κάλλος τῆς κυπαρίσσου, καὶ εἰσῆλθον εἰς ὕψος μέρους τοῦ δρυμοῦ,
\VS{25}καὶ ἔθηκα γέφυραν, καὶ ἠρήμωσα ὕδατα καὶ πᾶσαν συναγωγὴν ὕδατος.
\par }{\PP \VS{26}Οὐ ταῦτα ἤκουσας πάλαὶ ἃ ἐγὼ ἐποίησα; ἐξ ἡμερῶν ἀρχαίων συνέταξα, νῦν δὲ ἐπέδειξα ἐξερημῶσαι ἔθνη ἐν ὀχυροῖς, καὶ οἰκοῦντας ἐν πόλεσιν ὀχυραῖς.
\VS{27}Ἀνῆκα τὰς χεῖρας, καὶ ἐξηράνθησαν, καὶ ἐγένοντο ὡς χόρτος ξηρὸς ἐπὶ δωμάτων, καὶ ὡς ἄγρωστις.
\VS{28}Νῦν δὲ τὴν ἀνάπαυσίν σου, καὶ τὴν ἔξοδόν σου, καὶ τὴν εἴσοδόν σου ἐγὼ ἐπίσταμαι.
\VS{29}Ὁ δὲ θυμός σου ὃν ἐθυμώθης, καὶ ἡ πικρία σου ἀνέβη πρὸς μὲ, καὶ ἐμβαλῶ φιμὸν εἰς τὴν ῥῖνά σου, καὶ χαλινὸν εἰς τὰ χείλη σου, καὶ ἀποστρέψω σε τῇ ὁδῷ ᾗ ἦλθες ἐν αὐτῇ.
\par }{\PP \VS{30}Τοῦτο δέ σοι τὸ σημεῖον· φάγε τοῦτον τὸν ἐνιαυτὸν ἃ ἔσπαρκας, τῷ δὲ ἐνιαὐτῷ τῷ δευτέρῳ τὸ κατάλειμμα, τῷ δὲ τρίτῳ σπείραντες ἀμήσατε, καὶ φυτεύσατε ἀμπελῶνας, καὶ φάγεσθε τὸν καρπὸν αὐτῶν.
\VS{31}Καὶ ἔσονται οἱ καταλελιμμένοι ἐν τῇ Ἰουδαίᾳ, φυήσουσι ῥίζαν κάτω, καὶ ποιήσουσι σπέρμα ἄνω·
\VS{32}Ὅτι ἐξ Ἱερουσαλὴμ ἔσονται οἱ καταλελειμμένοι, καὶ οἱ σωζόμενοι ἐξ ὄρους Σιών· ὁ ζῆλος Κυρίου σαβαὼθ ποιήσει ταῦτα.
\VS{33}Διατοῦτο οὕτως λέγει Κύριος ἐπὶ βασιλέα Ἀσσυρίων, οὐ μὴ εἰσέλθῃ εἰς τὴν πόλιν ταύτην, οὐδὲ μὴ βάλῃ ἐπʼ αὐτὴν βέλος, οὐδὲ μὴ ἐπιβάλῃ ἐπʼ αὐτὴν θυρεὸν, οὐδὲ μὴ κυκλώσῃ ἐπʼ αὐτὴν χάρακα·
\VS{34}Ἀλλὰ τῇ ὁδῷ ᾗ ἦλθεν, ἐν αὐτῇ ἀποστραφήσεται, καὶ εἰς τὴν πόλιν ταύτην οὐ μὴ εἰσέλθῃ· τάδε λέγει Κύριος.
\VS{35}Ὑπερασπιῶ ὑπὲρ τῆς πόλεως ταύτης τοῦ σῶσαι αὐτὴν διʼ ἐμὲ, καὶ διὰ Δαυὶδ τὸν παῖδά μου.
\par }{\PP \VS{36}Καὶ ἐξῆλθεν ἄγγελος Κυρίου, καὶ ἀνεῖλεν ἐκ τῆς παρεμβολῆς τῶν Ἀσσυρίων ἑκατὸν ὀγδοηκονταπέντε χιλιάδας· καὶ ἀναστάντες τοπρωῒ, εὗρον πάντα τὰ σώματα νεκρά.
\VS{37}Καὶ ἀπῆλθεν ἀποστραφεὶς Σενναχηρεὶμ βασιλεὺς Ἀσσυρίων, καὶ ᾤκησεν ἐν Νινευῇ.
\VS{38}Καὶ ἐν τῷ αὐτὸν προσκυνεῖν ἐν τῷ οἴκῳ Νασαρὰχ τὸν πάτραρχον αὐτοῦ, Ἀδραμέλεχ καὶ Σαρασὰρ οἱ υἱοὶ αὐτοῦ ἐπάταξαν αὐτὸν μαχαίραις, αὐτοὶ δὲ διεσώθησαν εἰς Ἀρμενίαν, καὶ ἐβασίλευσεν Ἀσορδὰν ὁ υἱὸς αὐτοῦ ἀντʼ αὐτοῦ.

\par }\Chap{38}{\PP \VerseOne{1}Ἐγένετο δὲ ἐν τῷ καιρῷ ἐκείνῳ, ἐμαλακίσθη Ἐζεκίας ἕως θανάτου· καὶ ἦλθε πρὸς αὐτὸν Ἡσαΐας υἱὸς Ἀμὼς ὁ προφήτης, καὶ εἶπε πρὸς αὐτὸν, τάδε λέγει Κύριος, τάξαι περὶ τοῦ οἴκου σου, ἀποθνήσκεις γὰρ σὺ, καὶ οὐ ζήσῃ.
\VS{2}Καὶ ἀπέστρεψεν Ἐζεκίας τὸ πρόσωπον αὐτοῦ πρὸς τὸν τεῖχον, καὶ προσηύξατο πρὸς Κύριον,
\VS{3}λέγων, μνήσθητι Κύριε, ὡς ἐπορεύθην ἐνώπιόν σου μετὰ ἀληθείας, ἐν καρδίᾳ ἀληθινῇ, καὶ τὰ ἀρεστὰ ἐνώπιόν σου ἐποίησα· καὶ ἔκλαυσεν Ἐζεκίας κλαυθμῷ μεγάλῳ.
\VS{4}Καὶ ἐγένετο λόγος Κυρίου πρὸς Ἡσαΐαν, λέγων,
\VS{5}πορεύθητι, καὶ εἰπὸν Ἐζεκίᾳ, τάδε λέγει Κύριος ὁ Θεὸς Δαυὶδ τοῦ πατρός σου, ἤκουσα τῆς προσευχῆς σου, καὶ εἶδον τὰ δάκρυά σου· ἰδοὺ προστίθημι πρὸς τὸν χρόνον σου δεκαπέντε ἔτη,
\VS{6}καὶ ἐκ χειρὸς βασιλέως Ἀσσυρίων ῥύσομαί σε καὶ τὴν πόλιν ταύτην, καὶ ὑπερασπιῶ ὑπὲρ τῆς πόλεως ταύτης.
\VS{7}Τοῦτο δέ σοι τὸ σημεῖον παρὰ Κυρίου, ὅτι ποιήσει ὁ Θεὸς τὸ ῥῆμα τοῦτο·
\VS{8}Ἰδοὺ ἐγὼ στρέφω τὴν σκιὰν τῶν ἀναβαθμῶν οὓς κατέβη τοὺς δέκα ἀναβαθμοὺς τοῦ οἴκου τοῦ πατρός σου ὁ ἥλιος, ἀποστρέψω τὸν ἥλιον τοὺς δέκα ἀναβαθμούς· καὶ ἀνέβη ὁ ἥλιος τοὺς δέκα ἀναβαθμοὺς, οὓς κατέβη ἡ σκιά.
\par }{\PP \VS{9}ΠΡΟΣΕΥΧΗ ἘΖΕΚΙΟΥ ΒΑΣΙΛΕΩΣ ΤΗΣ ἸΟΥΔΑΙΑΣ, ἩΝΙΚΑ ἘΜΑΛΑΚΙΣΘΗ, ΚΑΙ ἈΝΕΣΤΗ ἘΚ ΤΗΣ ΜΑΛΑΚΙΑΣ ΑΥΤΟΥ.
\par }{\PP \VS{10}Ἐγὼ εἶπα ἐν τῷ ὕψει τῶν ἡμερῶν μου, πορεύσομαι ἐν πύλαις ἅδου, καταλείψω τὰ ἔτη τὰ ἐπίλοιπα.
\VS{11}Εἶπα, οὐκέτι οὐ μὴ ἴδω τὸ σωτήριον τοῦ Θεοῦ ἐπὶ γῆς ζώντων, οὐκέτι μὴ ἴδω τὸ σωτήριον τοῦ Ἰσραὴλ ἐπὶ γῆς, οὐκέτι μὴ ἴδω ἄνθρωπον.
\VS{12}Ἐξέλιπεν ἐκ τῆς συγγενείας μου, κατέλιπον τὸ ἐπίλοιπον τῆς ζωῆς μου, ἐξῆλθε καὶ ἀπῆλθεν ἀπʼ ἐμοῦ ὥσπερ ὁ σκηνὴν καταλύων πήξας· ὡς ἱστὸς τὸ πνεῦμά μου παρʼ ἐμοὶ ἐγένετο, ἐρίθου ἐγγιζούσης ἐκτεμεῖν.
\VS{13}Ἐν τῇ ἡμέρᾳ ἐκείνῃ παρεδόθην ἕως πρωῒ ὡς λέοντι, οὕτως συνέτριψε πάντα τὰ ὀστᾶ μου· ἀπὸ γὰρ τῆς ἡμέρας ἕως νυκτὸς παρεδόθην.
\VS{14}Ὡς χελιδὼν, οὕτω φωνήσω, καὶ ὡς περιστερὰ, οὕτω μελετῶ· ἐξέλιπον γάρ μου οἱ ὀφθαλμοὶ τοῦ βλέπειν εἰς τὸ ὕψος τοῦ οὐρανοῦ πρὸς τὸν Κύριον, ὃς ἐξείλατό με,
\VS{15}καὶ ἀφείλατό μου τὴν ὀδύνην τῆς ψυχῆς.
\VS{16}Κύριε, περὶ αὐτῆς γὰρ ἀνηγγέλη σοι, καὶ ἐξήγειράς μου τὴν πνοὴν, καὶ παρακληθεὶς ἔζησα.
\VS{17}Εἵλου γάρ μου τὴν ψυχὴν, ἴνα μὴ ἀπόληται, καὶ ἀπέῤῥιψας ὀπίσω μου πάσας τὰς ἁμαρτίας.
\VS{18}Οὐ γὰρ οἱ ἐν ᾅδου αἰνέσουσί σε, οὐδὲ οἱ ἀποθανόντες εὐλογήσουσί σε, οὐδὲ ἐλπιοῦσιν οἱ ἐν ᾄδου τὴν ἐλεημοσύνην σου.
\VS{19}Οἱ ζῶντες εὐλογήσουσί σε ὃν τρόπον κᾀγώ· ἀπὸ γὰρ τῆς σήμερον παιδία ποιήσω, ἅ ἀναγγελοῦσι τὴν δικαιοσύνην σου
\VS{20}Θεὲ τῆς σωτηρίας μου, καὶ οὐ παύσομαι εὐλογῶν σε μετὰ ψαλτηρίου πάσας τὰς ἡμέρας τῆς ζωῆς μου, κατέναντι τοῦ οἴκου τοῦ Θεοῦ.
\par }{\PP \VS{21}Καὶ εἶπεν Ἡσαΐας πρὸς Ἐζεκίαν, λάβε παλάθην ἐκ σύκων, καὶ τρίψον, καὶ κατάπλασαι, καὶ ὑγιὴς ἔσῃ.
\VS{22}Καὶ εἶπεν Ἐζεκίας, τοῦτο σημεῖον πρὸς Ἐζεκίαν, ὅτι ἀναβήσομαι εἰς τὸν οἶκον τοῦ Θεοῦ.

\par }\Chap{39}{\PP \VerseOne{1}Ἐν τῷ καιρῷ ἐκείνῳ ἀπέστειλε Μαρωδὰχ Βαλαδὰν ὁ υἱὸς τοῦ Βαλαδὰν, ὁ βασιλεὺς τῆς Βαβυλωνίας, ἐπιστολὰς καὶ πρέσβεις καὶ δῶρα Ἐζεκίᾳ· ἤκουσε γὰρ, ὅτι ἐμαλακίσθῃ ἕως θανάτου, καὶ ἀνέστη.
\VS{2}Καὶ ἐχάρη ἐπʼ αὐτοῖς Ἐξεκίας, καὶ ἔδειξεν αὐτοῖς τὸν οἶκον τοῦ νεχωθᾶ, καὶ τοῦ ἀργυρίου, καὶ τοῦ χρυσίου, καὶ τῆς στακτῆς, καὶ τῶν θυμιαμάτων, καὶ τοῦ μύρου, καὶ πάντας τοὺς οἴκους τῶν σκευῶν τῆς γάζης, καὶ πάντα ὅσα ἦν ἐν τοῖς θησαυροῖς αὐτοῦ· καὶ οὐκ ἦν οὐθὲν ὃ οὐκ ἔδειξεν Ἐζεκίας ἐν τῷ οἴκῳ αὐτοῦ, καὶ ἐν πάσῃ τῇ ἐξουσίᾳ αὐτοῦ.
\par }{\PP \VS{3}Καὶ ἦλθεν Ἡσαΐας ὁ προφήτης πρὸς τὸν βασιλέα Ἐζεκίαν, καὶ εἶπε πρὸς αὐτὸν, τί λέγουσιν οἱ ἄνθρωποι οὗτοι; καὶ πόθεν ἥκασι πρὸς σέ; καὶ εἶπεν Ἐζεκίας, ἐκ γῆς πόῤῥωθεν ἥκασι πρὸς μέ, ἐκ Βαβυλῶνος.
\VS{4}Καὶ εἶπεν Ἠσαΐας, τί εἴδοσαν ἐν τῷ οἴκῳ σου; καὶ εἶπεν Ἐζεκίας, πάντα τὰ ἐν τῷ οἴκῳ μου εἴδοσαν, καὶ οὐκ ἔστιν ἐν τῷ οἴκῳ μου ὃ οὐκ εἴδοσαν, ἀλλὰ καὶ τὰ ἐν τοῖς θησαυροῖς μου.
\VS{5}Καὶ εἶπεν Ἡσαΐας αὐτῷ, ἄκουσον τὸν λόγον Κυρίου σαβαώθ.
\VS{6}Ἰδοὺ ἡμέραι ἔρχονται, καὶ λήψονται πάντα τὰ ἐν τῷ οἴκῳ σου, καὶ ὅσα συνήγαγον οἱ πατέρες σου ἕως τῆς ἡμέρας ταύτης, εἰς Βαβυλῶνα ἥξει, καὶ οὐδὲν οὐ μὴ καταλείπωσιν· εἶπε δὲ ὁ Θεὸς,
\VS{7}ὅτι καὶ ἀπὸ τῶν τέκνων σου ὧν γεννήσεις, λήψονται, καὶ ποιήσουσι σπάδοντας ἐν τῷ οἴκῳ τοῦ βασιλέως τῶν Βαβυλωνίων.
\VS{8}Καὶ εἶπεν Ἐζεκίας Ἡσαΐᾳ, ἀγαθὸς ὁ λόγος Κυρίου, ὃν ἐλάλησε· γενέσθω δὴ εἰρήνη καὶ δικαιοσύνη ἐν ταῖς ἡμέραις μου.

\par }\Chap{40}{\PP \VerseOne{1}Παρακαλεῖτε παρακαλεῖτε τὸν λαόν μου, λέγει ὁ Θεός.
\VS{2}Ἱερεῖς λαλήσατε εἰς τὴν καρδίαν Ἱερουσαλὴμ, παρακαλέσατε αὐτὴν, ὅτι ἐπλήσθη ἡ ταπείνωσις αὐτῆς, λέλυται αὐτῆς ἡ ἁμαρτία, ὅτι ἐδέξατο ἐκ χειρὸς Κυρίου διπλᾶ τὰ ἁμαρτήματα αὐτῆς.
\par }{\PP \VS{3}Φωνὴ βοῶντος ἐν τῇ ἐρήμῳ, ἑτοιμάσατε τὴν ὁδὸν Κυρίου, εὐθείας ποιεῖτε τὰς τρίβους τοῦ Θεοῦ ἡμῶν.
\VS{4}Πᾶσα φάραγξ πληρωθήσεται, καὶ πᾶν ὄρος καὶ βουνὸς ταπεινωθήσεται· καὶ ἔσται πάντα τὰ σκολιὰ εἰς εὐθεῖαν, καὶ ἡ τραχεῖα εἰς πεδία.
\VS{5}Καὶ ὀφθήσεται ἡ δόξα Κυρίου, καὶ ὄψεται πᾶσα σὰρξ τὸ σωτήριον τοῦ Θεοῦ, ὅτι Κύριος ἐλάλησε.
\par }{\PP \VS{6}Φωνὴ λέγοντος, βόησον· καὶ εἶπα, τί βοήσω; πᾶσα σὰρξ χορτὸς, καὶ πᾶσα δόξα ἀνθρώπου ὡς ἄνθος χόρτου·
\VS{7}Ἐξηράνθη ὁ χόρτος, καὶ τὸ ἄνθος ἐξέπεσε·
\VS{8}τὸ δὲ ῥῆμα τοῦ Θεοῦ ἡμῶν μένει εἰς τὸν αἰῶνα.
\par }{\PP \VS{9}Ἐπʼ ὄρος ὑψηλὸν ἀνάβηθι ὁ εὐαγγελιζόμενος Σιὼν, ὕψωσον τῇ ἰσχύϊ τὴν φωνήν σου ὁ εὐαγγελιζόμενος Ἱερουσαλήμ· ὑψώσατε, μὴ φοβεῖσθε· εἰπὸν ταῖς πόλεσιν Ἰούδα, ἰδοὺ ὁ Θεὸς ὑμῶν, ἰδοὺ Κύριος·
\VS{10}Κύριος μετὰ ἰσχύος ἔρχεται, καὶ ὁ βραχίων μετὰ κυρίας· ἰδοὺ ὁ μισθὸς αὐτοῦ μετʼ αὐτοῦ, καὶ τὸ ἔργον ἐναντίον αὐτοῦ.
\VS{11}Ὡς ποιμὴν ποιμανεῖ τὸ ποίμνιον αὐτοῦ, καὶ τῷ βραχίονι αὐτοῦ συνάξει ἄρνας, καὶ ἐν γαστρὶ ἐχούσας παρακαλέσει.
\VS{12}Τίς ἐμέτρησε τῇ χειρὶ τὸ ὕδωρ, καὶ τὸν οὐρανὸν σπιθαμῇ, καὶ πᾶσαν τὴν γῆν δρακί; τίς ἔστησε τὰ ὄρη σταθμῷ, καὶ τὰς νάπας ζυγῷ;
\VS{13}Τίς ἔγνω νοῦν Κυρίου; καὶ τίς αὐτοῦ σύμβουλος ἐγένετο, ὃς συμβιβᾷ αὐτόν;
\VS{14}Ἢ πρὸς τίνα συνεβουλεύσατο, καὶ συνεβίβασεν αὐτόν; ἢ τίς ἔδειξεν αὐτῷ κρίσιν; ἢ ὁδὸν συνέσεως τίς ἔδειξεν αὐτῷ;
\VS{15}Εἰ πάντα τὰ ἔθνη ὡς σταγὼν ἀπὸ κάδου, καὶ ὡς ῥοπὴ ζυγοῦ ἐλογίσθησαν, ὡς σίελος λογισθήσονται;
\VS{16}Ὁ δὲ Λίβανος οὐχ ἱκανὸς εἰς καῦσιν, καὶ πάντα τὰ τετράποδα οὐχ ἱκανὰ εἰς ὁλοκάρπωσιν,
\VS{17}καὶ πάντα τὰ ἔθνη ὡς οὐδέν εἰσι, καὶ εἰς οὐθὲν ἐλογίσθησαν.
\par }{\PP \VS{18}Τίνι ὡμοιώσατε Κύριον; καὶ τίνι ὁμοιώματι ὡμοιώσατε αὐτόν;
\VS{19}Μὴ εἰκόνα ἐποίησε τέκτων, ἢ χρυσοχόος χωνεύσας χρυσίον περιεχρύσωσεν αὐτὸν, ὁμοίωμα κατεσκεύασεν αὐτόν;
\VS{20}Ξύλον γὰρ ἄσηπτον ἐκλέγεται τέκτων, καὶ σοφῶς ζητήσει πῶς στήσει εἰκόνα αὐτοῦ, καὶ ἵνα μὴ σαλεύηται.
\VS{21}Οὐ γνώσεσθε; οὐκ ἀκούσεσθε; οὐκ ἀνηγγέλη ἐξ ἀρχῆς ὑμῖν; οὐκ ἔγνωτε τὰ θεμέλια τῆς γῆς;
\VS{22}Ὁ κατέχων τὸν γῦρον τῆς γῆς, καὶ οἱ ἐνοικοῦντες ἐν αὐτῇ ὡς ἀκρίδες· ὁ στήσας ὡς καμάραν τὸν οὐρανὸν, καὶ διατείνας ὡς σκηνὴν κατοικεῖν·
\VS{23}Ὁ διδοὺς ἄρχοντας ὡς οὐδὲν ἄρχειν, τὴν δὲ γῆν ὡς οὐδὲν ἐποίησεν.
\VS{24}Οὐ γὰρ μὴ φυτεύσωσιν, οὐδὲ μὴ σπείρωσιν, οὐδὲ μὴ ῥιζωθῇ εἰς τὴν γῆν ἡ ῥίζα αὐτῶν· ἔπνευσεν ἐπʼ αὐτοὺς, καὶ ἐξηράνθησαν, καὶ καταιγὶς ὡς φρύγανα λήψεται αὐτούς.
\par }{\PP \VS{25}Νῦν οὖν τίνι με ὡμοιώσατε, καὶ ὑψωθήσομαι; εἶπεν ὁ ἅγιος.
\VS{26}Ἀναβλέψατε εἰς ὕψος τοὺς ὀφθαλμοὺς ὑμῶν, καὶ ἴδετε, τίς κατέδειξε ταῦτα πάντα; ὁ ἐκφέρων κατʼ ἀριθμὸν τὸν κόσμον αὐτοῦ, πάντας ἐπʼ ὀνόματι καλέσει ἀπὸ πολλῆς δόξης, καὶ ἐν κράτει ἰσχύος αὐτοῦ· οὐδέν σε ἔλαθε.
\par }{\PP \VS{27}Μὴ γὰρ εἴπῃς Ἰακὼβ, καὶ τί ἐλάλησας Ἰσραήλ; ἀπεκρύβη ἡ ὁδός μου ἀπὸ τοῦ Θεοῦ, καὶ ὁ Θεός μου τὴν κρίσιν ἀφεῖλε, καὶ ἀπέστη.
\VS{28}Καὶ νῦν οὐκ ἔγνως; εἰ μὴ ἤκουσας; Θεὸς αἰώνιος, ὁ Θεὸς ὁ κατασκευάσας τὰ ἄκρα τῆς γῆς· οὐ πεινάσει, οὐδὲ κοπιάσει, οὐδὲ ἔστιν ἐξεύρεσις τῆς φρονήσεως αὐτοῦ,
\VS{29}διδοὺς τοῖς πεινῶσιν ἰσχὺν, καὶ τοῖς μὴ ὀδυνωμένοις λύπην.
\VS{30}Πεινάσουσι γὰρ νεώτεροι, καὶ κοπιάσουσι νεανίσκοι, καὶ ἐκλεκτοὶ ἀνίσχυες ἔσονται.
\VS{31}Οἱ δὲ ὑπομένοντες τὸν Θεὸν, ἀλλάξουσιν ἰσχὺν, πτεροφυήσουσιν ὡς ἀετοὶ, δραμοῦνται καὶ οὐ κοπιάσουσι, βαδιοῦνται καὶ οὐ πεινάσουσιν.

\par }\Chap{41}{\PP \VerseOne{1}Ἐγκαινίζεσθε πρὸς μὲ νῆσοι, οἱ γὰρ ἄρχοντες ἀλλάξουσιν ἰσχύν· ἐγγισάτωσαν καὶ λαλησάτωσαν ἅμα, τότε κρίσιν ἀναγγειλάτωσαν.
\par }{\PP \VS{2}Τίς ἐξήγειρεν ἀπὸ ἀνατολῶν δικαιοσύνην, ἐκάλεσεν αὐτὴν κατὰ πόδας αὐτοῦ, καὶ πορεύσεται; δώσει ἐναντίον ἐθνῶν, καὶ βασιλεῖς ἐκστήσει· καὶ δώσει εἰς γῆν τὰς μαχαίρας αὐτῶν, καὶ ὡς φρύγανα ἐξωσμένα τὰ τόξα αὐτῶν.
\VS{3}Καὶ διώξεται αὐτοὺς, διελεύσεται ἐν εἰρήνῃ ἡ ὁδὸς τῶν ποδῶν αὐτοῦ.
\VS{4}Τίς ἐνήργησε, καὶ ἐποίησε ταῦτα; ἐκάλεσεν αὐτὴν ὁ καλῶν αὐτὴν ἀπὸ γενεῶν ἀρχῆς· ἐγὼ Θεὸς πρῶτος, καὶ εἰς τὰ ἐπερχόμενα ἐγώ εἰμι.
\par }{\PP \VS{5}Εἴδοσαν ἔθνη καὶ ἐφοβηθήσαν, τὰ ἄκρα τῆς γῆς ἤγγισαν, καὶ ἦλθον ἅμα,
\VS{6}κρίνων ἕκαστος τῷ πλησίον, καὶ τῷ ἀδελφῷ βοηθῆσαι·
\VS{7}καὶ ἐρεῖ, ἴσχυσεν ἀνὴρ τέκτων, καὶ χαλκεὺς τύπτων σφύρῃ, ἅμα ἐλαύνων· πότε μὲν ἐρεῖ, σύμβλημα καλόν ἐστιν, ἰσχύρωσαν αὐτὰ ἐν ἥλοις, θήσουσιν αὐτὰ, καὶ οὐ κινηθήσονται.
\par }{\PP \VS{8}Σὺ δὲ, Ἰσραὴλ παῖς μου Ἰακὼβ, καὶ ὃν ἐξελεξάμην, σπέρμα Ἁβραὰμ, ὃν ἠγάπησα·
\VS{9}Οὗ ἀντελαβόμην ἀπʼ ἄκρων τῆς γῆς, καὶ ἐκ τῶν σκοπιῶν αὐτῆς ἐκάλεσά σε, καὶ εἶπά σοι, παῖς μου εἶ, ἐξελεξάμην σε, καὶ οὐκ ἐγκατέλιπόν σε.
\VS{10}Μὴ φοβοῦ, μετὰ σοῦ γάρ εἰμι, μὴ πλανῶ· ἐγὼ γάρ εἰμι ὁ Θεός σου, ὁ ἐνισχύσας σε, καὶ ἐβοήθησά σοι, καὶ ἠσφαλισάμην σε τῇ δεξιᾷ τῇ δικαίᾳ μου.
\par }{\PP \VS{11}Ἰδοὺ αἰσχυνθήσονται καὶ ἐντραπήσονται πάντες οἱ ἀντικείμενοί σοι· ἔσονται γὰρ ὡς οὐκ ὄντες, καὶ ἀπολοῦνται πάντες οἱ ἀντίδικοί σου.
\VS{12}Ζητήσεις αὐτοὺς, καὶ οὐ μὴ εὕρῃς τοὺς ἀνθρώπους οἳ παροινήσουσιν εἰς σέ· ἔσονται γὰρ ὡς οὐκ ὄντες, καὶ οὐκ ἔσονται οἱ ἀντιπολεμοῦντές σε·
\VS{13}Ὅτι ἐγὼ ὁ Θεός σου, ὁ κρατῶν τῆς δεξιᾶς σου, ὁ λέγων σοι,
\VS{14}μὴ φοβοῦ Ἰακὼβ ὀλιγοστὸς Ἰσραὴλ· ἐγὼ ἐβοήθησά σοι, λέγει ὁ Θεὸς σου, ὁ λυτρούμενὸσσε Ἰσραήλ.
\VS{15}Ἰδοὺ ἐποίησά σε ὡς τροχοὺς ἁμάξης ἀλοῶντας καινοὺς πριστηροειδεῖς, καὶ ἀλοήσεις ὄρη, καὶ λεπτυνεῖς βουνοὺς, καὶ ὡς χνοῦν θήσεις,
\VS{16}καὶ λικήσεις, καὶ ἄνεμος λήμψεται αὐτοὺς, καὶ καταιγὶς διασπερεῖ αὐτούς· σὺ δὲ εὐφρανθήσῃ ἐν τοῖς ἁγίοις Ἰσραήλ.
\par }{\PP \VS{17}Καὶ ἀγαλλιάσονται οἱ πτωχοὶ καὶ οἱ ἐνδεεῖς· ζητήσουσι γὰρ ὕδωρ, καὶ οὐκ ἔσται, ἡ γλῶσσα αὐτῶν ἀπὸ τῆς δίψης ἐξηράνθη· ἐγὼ Κύριος ὁ Θεὸς, ἐγὼ ἐπακούσομαι ὁ Θεὸς Ἰσραὴλ, καὶ οὐκ ἐγκαταλείψω αὐτοὺς,
\VS{18}ἀλλὰ ἀνοίξω ἐπὶ τῶν ὀρέων ποταμοὺς, καὶ ἐν μέσῳ πεδίων πηγάς· ποιήσω τὴν ἔρημον εἰς ἕλη ὑδάτων, καὶ τὴν διψῶσαν γῆν ἐν ὑδραγωγοῖς.
\VS{19}Θήσω εἰς τὴν ἄνυδρον γῆν, κέδρον καὶ πύξον, μυρσίνην καὶ κυπάρισσον, καὶ λεύκην·
\VS{20}Ἵνα ἴδωσι καὶ γνῶσι, καὶ ἐννοηθῶσι καὶ ἐπιστῶνται ἅμα, ὅτι χεὶρ Κυρίου ἐποίησε ταῦτα, καὶ ὁ ἅγιος τοῦ Ἰσραὴλ κατέδειξεν.
\par }{\PP \VS{21}Ἐγγίζει ἡ κρίσις ὑμῶν, λέγει Κύριος ὁ Θεός· ἤγγισαν αἱ βουλαὶ ὑμῶν, λέγει ὁ βασιλεὺς Ἰακώβ.
\VS{22}Ἐγγισάτωσαν, καὶ ἀναγγειλάτωσαν ὑμῖν ἃ συμβήσεται, ἢ τὰ πρότερον τίνα ἦν, εἴπατε, καὶ ἐπιστήσομεν τὸν νοῦν, καὶ γνωσόμεθα τί τὰ ἔσχατα καὶ τὰ ἐπερχόμενα·
\VS{23}εἴπατε ἡμῖν, ἀναγγείλατε ἡμῖν τὰ ἐπερχόμενα ἐπʼ ἐσχάτου, καὶ γνωσόμεθα ὅτι θεοί ἐστε· εὐποιήσατε καὶ κακώσατε, καὶ θαυμασόμεθα, καὶ ὀψόμεθα ἅμα
\VS{24}ὅτι πόθεν ἐστὲ ὑμεῖς, καὶ πόθεν ἡ ἐργασία ὑμῶν· ἐκ γῆς βδέλυγμα ἐξελέξαντο ὑμᾶς.
\par }{\PP \VS{25}Ἐγὼ δὲ ἤγειρα τὸν ἀπὸ Βοῤῥᾶ, καὶ τὸν ἀφʼ ἡλίου ἀνατολῶν· κληθήσονται τῷ ὀνόματί μου· ἐρχέσθωσαν ἄρχοντες, καὶ ὡς πηλὸς κεραμέως, καὶ ὡς κεραμεὺς καταπατῶν τὸν πηλὸν, οὕτω καταπατηθήσεσθε.
\VS{26}Τίς γὰρ ἀναγγελεῖ τὰ ἐξ ἀρχῆς, ἵνα γνῶμεν καὶ τὰ ἔμπροσθεν, καἰ ἐροῦμεν ὅτι ἀληθῆ ἐστιν; οὐκ ἔστιν ὁ προλέγων, οὐδὲ ὁ ἀκούων ὑμῶν τοὺς λόγους.
\VS{27}Ἀρχὴν Σιὼν δώσω, καὶ Ἱερουσαλὴμ παρακαλέσω εἰς ὁδόν.
\VS{28}Ἀπὸ γὰρ τῶν ἐθνῶν, ἰδοὺ οὐδείς· καὶ ἀπὸ τῶν εἰδώλων αὐτῶν οὐκ ἦν ὁ ἀναγγέλλων· καὶ ἐὰν ἐρωτήσω αὐτοὺς, πόθεν ἐστέ; οὐ μὴ ἀποκριθῶσί μοι.
\VS{29}Εἰσὶ γὰρ οἱ ποιοῦντες ὑμᾶς, καὶ μάτην οἱ πλανῶντες ὑμᾶς.

\par }\Chap{42}{\PP \VerseOne{1}Ἰακὼβ ὁ παῖς μου, ἀντιλήψομαι αὐτοῦ· Ἰσραὴλ ὁ ἐκλεκτός μου, προσεδέξατο αὐτὸν ἡ ψυχή μου· ἔδωκα τὸ πνεῦμά μου ἐπʼ αὐτὸν, κρίσιν τοῖς ἔθνεσιν ἐξοίσει.
\VS{2}Οὐ κεκράξεται, οὐδὲ ἀνήσει, οὐδὲ ἀκουσθήσεται ἔξω ἡ φωνὴ αὐτοῦ.
\VS{3}Κάλαμον τεθλασμένον οὐ συντρίψει, καὶ λίνον καπνιζόμενον οὐ σβέσει, ἀλλὰ εἰς ἀλήθειαν ἐξοίσει κρίσιν.
\VS{4}Ἀναλάμψει, καὶ οὐ θραυσθήσεται, ἕως ἂν θῇ ἐπὶ τῆς γῆς κρίσιν, καὶ ἐπὶ τῷ ὀνόματι αὐτοῦ ἔθνη ἐλπιοῦσιν.
\par }{\PP \VS{5}Οὕτω λέγει Κύριος ὁ Θεὸς, ὁ ποιήσας τὸν οὐρανὸν, καὶ πήξας αὐτὸν, ὁ στερεώσας τὴν γῆν, καὶ τὰ ἐν αὐτῇ, καὶ διδοὺς πνοὴν τῷ λαῷ τῷ ἐπʼ αὐτῆς, καὶ πνεῦμα τοῖς πατοῦσιν αὐτήν·
\VS{6}Ἐγὼ Κύριος ὁ Θεὸς ἐκάλεσά σε ἐν δικαιοσύνῃ, καὶ κρατήσω τῆς χειρός σου, καὶ ἐνισχύσω σε, καὶ ἔδωκά σε εἰς διαθήκην γένους, εἰς φῶς ἐθνῶν,
\VS{7}ἀνοῖξαι ὀφθαλμοὺς τυφλῶν, ἐξαγαγεῖν ἐκ δεσμῶν δεδεμένους καὶ ἐξ οἴκου φυλακῆς, καὶ καθημένους ἐν σκότει.
\par }{\PP \VS{8}Ἐγὼ Κύριος ὁ Θεός, τοῦτό μου ἐστὶ τὸ ὄνομα, τὴν δόξαν μου ἑτέρῳ οὐ δώσω, οὐδὲ τὰς ἀρετάς μου τοῖς γλυπτοῖς.
\VS{9}Τὰ ἀπʼ ἀρχῆς ἰδοὺ ἥκασι, καὶ καινὰ ἃ ἐγὼ ἀναγγέλλω, καὶ πρὸ τοῦ ἀναγγεῖλαι ἐδηλώθη ὑμῖν.
\par }{\PP \VS{10}Ὑμνήσατε τῷ Κυρίῳ ὕμνον καινόν· ἡ ἀρχὴ αὐτοῦ, δοξάζετε τὸ ὄνομα αὐτοῦ ἀπʼ ἄκρου τῆς γῆς, οἱ καταβαίνοντες εἰς τὴν θάλασσαν, καὶ πλέοντες αὐτὴν, αἱ νῆσοι καὶ οἱ κατοικοῦντες αὐτάς.
\VS{11}Εὐφράνθητι ἔρημος, καὶ αἱ κῶμαι αὐτῆς, ἐπαύλεις, καὶ οἱ κατοικοῦντες Κηδάρ· εὐφρανθήσονται οἱ κατοικοῦντες πέτραν, ἀπʼ ἄκρου τῶν ὀρέων βοήσουσι,
\VS{12}δώσουσι τῷ Θεῷ δόξαν, τὰς ἀρετὰς αὐτοῦ ἐν ταῖς νήσοις ἀναγγελοῦσι.
\par }{\PP \VS{13}Κύριος ὁ Θεὸς τῶν δυνάμεων ἐξελεύσεται, καὶ συντρίψει πόλεμον, ἐπεγερεῖ ζῆλον, καὶ βοήσεται ἐπὶ τοὺς ἐχθροὺς αὐτοῦ μετὰ ἰσχύος.
\VS{14}Ἐσιώπησα, μὴ καὶ ἀεὶ σιωπήσομαι καὶ ἀνέξομαι; ὡς ἡ τίκτουσα ἐκαρτέρησα, ἐκστήσω καὶ ξηρανῶ ἅμα·
\VS{15}Ἐρημώσω ὄρη καὶ βουνοὺς, καὶ πάντα χόρτον αὐτῶν ξηρανῶ· καὶ θήσω ποταμοὺς εἰς νήσους, καὶ ἕλη ξηρανῶ.
\VS{16}Καὶ ἄξω τυφλοὺς ἐν ὁδῷ ᾗ οὐκ ἔγνωσαν, καὶ τρίβους ἃς οὐκ ᾔδεισαν, πατῆσαι ποιήσω αὐτούς· ποιήσω αὐτοῖς τὸ σκότος εἰς φῶς, καὶ τὰ σκολιὰ εἰς εὐθεῖαν· ταῦτα τὰ ῥήματα ποιήσω, καὶ οὐκ ἐγκαταλείψω αὐτούς·
\VS{17}Αὐτοὶ δὲ ἀπεστράφησαν εἰς τὰ ὀπίσω· αἰσχύνθητε αἰσχύνην οἱ πεποιθότες ἐπὶ τοῖς γλυπτοῖς, οἱ λέγοντες τοῖς χωνευτοῖς, ὑμεῖς ἐστε θεοὶ ἡμῶν.
\par }{\PP \VS{18}Οἱ κωφοὶ ἀκούσατε, καὶ οἱ τυφλοὶ ἀναβλέψατε ἰδεῖν.
\VS{19}Καὶ τίς τυφλὸς ἀλλʼ ἢ οἱ παῖδές μου, καὶ κωφοὶ ἀλλʼ ἢ οἱ κυριεύοντες αὐτῶν; καὶ ἐτυφλώθησαν οἱ δοῦλοι τοῦ Θεοῦ.
\VS{20}Εἴδετε πλεονάκις, καὶ οὐκ ἐφυλάξασθε· ἠνοιγμένα τὰ ὦτα. καὶ οὐκ ἠκούσατε.
\VS{21}Κύριος ὁ Θεὸς ἐβουλεύσατο ἵνα δικαιωθῇ, καὶ μεγαλύνῃ αἴνεσιν.
\VS{22}Καὶ εἶδον, καὶ ἐγένετο ὁ λαὸς πεπρονομευμένος, καὶ διηρπασμένος· ἡ γὰρ παγὶς ἐν τοῖς ταμείοις πανταχοῦ, καὶ ἐν οἴκοις ἅμα, ὅπου ἔκρυψαν αὐτούς· ἐγένοντο εἰς προνομὴν, καὶ οὐκ ἦν ἐξαιρούμενος ἅρπαγμα, καὶ οὐκ ἦν ὁ λέγων, ἀπόδος.
\par }{\PP \VS{23}Τίς ἐν ὑμῖν ὃς ἐνωτιεῖται ταῦτα; εἰσακούσατε εἰς τὰ ἐπερχόμενα.
\VS{24}Οἷς ἔδωκεν εἰς διαρπαγὴν Ἰακὼβ καὶ Ἰσραὴλ τοῖς προνομεύουσιν αὐτόν; οὐχὶ ὁ Θεὸς ᾧ ἡμάρτοσαν αὐτῷ, καὶ οὐκ ἠβούλοντο ἐν ταῖς ὁδοῖς αὐτοῦ πορεύεσθαι, οὐδὲ ἀκούειν τοῦ νόμου αὐτοῦ;
\VS{25}Καὶ ἐπήγαγεν ἐπʼ αὐτοὺς ὀργὴν θυμοῦ αὐτοῦ, καὶ κατίσχυσεν αὐτοὺς πόλεμος, καὶ οἱ συμφλέγοντες αὐτοὺς κύκλῳ, καὶ οὐκ ἔγνωσαν ἕκαστος αὐτῶν, οὐδὲ ἔθεντο ἐπὶ ψυχήν.

\par }\Chap{43}{\PP \VerseOne{1}Καὶ νῦν οὕτως λέγει Κύριος ὁ Θεὸς ὁ ποιήσας σε Ἰακὼβ, καὶ ὁ πλάσας σε Ἰσραὴλ, μὴ φοβοῦ, ὅτι ἐλυτρωσάμην σε, ἐκάλεσά σε τὸ ὄνομά σου· ἐμὸς εἶ σύ.
\VS{2}Καὶ ἐὰν διαβαίνῃς διʼ ὕδατος, μετὰ σοῦ εἰμι, καὶ ποταμοὶ οὐ συνκλύσουσί σε· καὶ ἐὰν διέλθῃς διὰ πυρὸς, οὐ μὴ κατακαυθῇς, φλὸξ οὐ κατακαύσει σε.
\VS{3}Ὅτι ἐγὼ Κύριος ὁ Θεός σου ὁ ἅγιος Ἰσραὴλ, ὁ σώζων σε· ἐποίησα ἄλλαγμά σου Αἴγυπτον καὶ Αἰθιοπίαν, καὶ Σοήνην ὑπὲρ σοῦ.
\VS{4}Αφʼ οὗ ἔντιμος ἐγένου ἐναντίον ἐμοῦ, ἐδοξάσθης, καὶ ἐγώ σε ἠγάπησα, καὶ δώσω ἀνθρώπους ὑπὲρ σοῦ, καὶ ἄρχοντας ὑπὲρ τῆς κεφαλῆς σου.
\VS{5}Μὴ φοβοῦ, ὅτι μετὰ σοῦ εἰμι· ἀπὸ ἀνατολῶν ἄξω τὸ σπέρμα σου, καὶ ἀπὸ δυσμῶν συνάξω σε.
\VS{6}Ἐρῶ τῷ Βοῤῥᾷ, ἄγε, καὶ τῷ Λιβὶ, μὴ κώλυε· ἄγε τοὺς υἱούς μου ἀπὸ τῆς πόῤῥωθεν, καὶ τὰς θυγατέρας μου ἀπʼ ἄκρων τῆς γῆς,
\VS{7}πάντας ὅσοι ἐπικέκληνται τῷ ὀνόματί μου· ἐν γὰρ τῇ δόξῃ μου κατεσκεύασα αὐτὸν, καὶ ἔπλασα αὐτὸν, καὶ ἐποίησα αὐτὸν,
\VS{8}καὶ ἐξήγαγον λαὸν τυφλὸν, καὶ ὀφθαλμοί εἰσιν ὡσαύτως τυφλοὶ, καὶ κωφοὶ τὰ ὦτα ἔχοντες.
\par }{\PP \VS{9}Πάντα τὰ ἔθνη συνήχθησαν ἅμα, καὶ συναχθήσονται ἄρχοντες ἐξ αὐτῶν· τίς ἀναγγελεῖ ταῦτα; ἢ τὰ ἐξ ἀρχῆς τίς ἀναγγελεῖ ὑμῖν; ἀγαγέτωσαν τοῦς μάρτυρας αὐτῶν καὶ δικαιωθήτωσαν, καὶ ἀκουσάτωσαν, καὶ εἰπάτωσαν ἀληθῆ.
\par }{\PP \VS{10}Γένεσθέ μοι μάρτυρες, καὶ ἐγὼ μάρτυς, λέγει Κύριος ὁ Θεὸς, καὶ ὁ παῖς μου ὃν ἐξελεξάμην, ἵνα γνῶτε καὶ πιστεύσητε, καὶ συνῆτε ὅτι ἐγώ εἰμι· ἔμπροσθέν μου οὐκ ἐγένετο ἄλλος Θεὸς, καὶ μετʼ ἐμὲ οὐκ ἔσται.
\VS{11}Ἐγὼ ὁ Θεὸς, καὶ οὐκ ἔστι πάρεξ ἐμοῦ σώζων.
\VS{12}Ἐγὼ ἀνήγγειλα καὶ ἔσωσα, ὠνείδισα καὶ οὐκ ἦν ἐν ὑμῖν ἀλλότριος· ὑμεῖς ἐμοὶ μάρτυρες, καὶ ἐγὼ Κύριος ὁ Θεὸς
\VS{13}ἔτι ἀπʼ ἀρχῆς, καὶ οὐκ ἔστιν ὁ ἐκ τῶν χειρῶν μου ὁ ἐξαιρούμενος· ποιήσω, καὶ τίς ἀποστρέψει αὐτό;
\par }{\PP \VS{14}Οὕτως λέγει Κύριος ὁ Θεὸς ὁ λυτρούμενος ὑμᾶς, ὁ ἅγιος τοῦ Ἰσραὴλ, ἕνεκεν ὑμῶν ἀποστελῶ εἰς Βαβυλῶνα, καὶ ἐπεγερῶ φεύγοντας πάντας, καὶ Χαλδαῖοι ἐν πλοίοις δεθήσονται.
\VS{15}Ἐγὼ Κύριος ὁ Θεὸς ὁ ἅγιος ὑμῶν, ὁ καταδείξας Ἰσραὴλ βασιλέα ὑμῶν.
\par }{\PP \VS{16}Οὕτως λέγει Κύριος, ὁ διδοὺς ἐν θαλάσσῃ ὁδὸν, καὶ ἐν ὕδατι ἰσχυρῷ τρίβον,
\VS{17}ὁ ἐξαγαγὼν ἅρματα καὶ ἵππον καὶ ὄχλον ἰσχυρον· ἀλλʼ ἐκοιμήθησαν, καὶ οὐκ ἀναστήσονται, ἐσβέσθησαν ὡς λῖνον ἐσβεσμένον.
\par }{\PP \VS{18}Μὴ μνημονεύετε τὰ πρῶτα, καὶ τὰ ἀρχαῖα μὴ συλλογίζεσθε.
\VS{19}Ἰδοὺ ἐγὼ ποιῶ καινὰ, ἃ νῦν ἀνατελεῖ, καὶ γνώσεσθε αὐτά· καὶ ποιήσω ἐν τῇ ἐρήμῳ ὁδὸν, καὶ ἐν τῇ ἀνύδρῳ ποταμούς.
\VS{20}Εὐλογήσουσί με τὰ θηρία τοῦ ἀγροῦ, σειρῆνες, καὶ θυγατέρες στρουθῶν, ὅτι ἔδωκα ἐν τῇ ἐρήμῳ ὕδωρ, καὶ ποταμοὺς ἐν τῇ ἀνύδρῳ, ποτίσαι τὸ γένος μου τὸ ἐκλεκτὸν,
\VS{21}λαόν μου ὃν περιεποιησάμην τὰς ἀρετάς μου διηγεῖσθαι.
\par }{\PP \VS{22}Οὐ νῦν ἐκάλεσά σε Ἰακὼβ, οὐδὲ κοπιάσαι σε ἐποίησα Ἰσραήλ.
\VS{23}Οὐκ ἤνεγκάς μοι πρόβατά σου τῆς ὁλοκαρπώσεώς σου, οὐδὲ ἐν ταῖς θυσίαις σου ἐδόξασάς με· οὐκ ἐδούλωσά σε ἐν θυσίαις, οὐδὲ ἔγκοπον ἐποίησά σε ἐν λιβάνῳ,
\VS{24}οὐδὲ ἐκτήσω μοι ἀργυρίου θυσίασμα, οὐδὲ τὸ στέαρ τῶν θυσιῶν σου ἐπεθύμησα· ἀλλὰ ἐν ταῖς ἁμαρτίαις σου προέστης μου, καὶ ἐν ταῖς ἀδικίαις σου.
\VS{25}Ἐγώ εἰμι ἐγώ εἰμι ὁ ἐξαλείφων τὰς ἀνομίας σου ἕνεκεν ἐμοῦ, καὶ τὰς ἁμαρτίας σου, καὶ οὐ μὴ μνησθήσομαι.
\VS{26}Σὺ δὲ μνήσθητι, καὶ κριθῶμεν· λέγε σὺ τὰς ἀνομίας σου πρῶτος, ἵνα δικαιωθῇς.
\VS{27}Οἱ πατέρες ὑμῶν πρῶτοι, καὶ οἱ ἄρχοντες ὑμῶν ἠνόμησαν εἰς ἐμέ.
\VS{28}Καὶ ἐμίαναν οἱ ἄρχοντες τὰ ἅγιά μου· καὶ ἔδωκα ἀπωλέσαι Ἰακὼβ, καὶ Ἰσραὴλ εἰς ὀνειδισμόν.

\par }\Chap{44}{\PP \VerseOne{1}Νῦν δὲ ἄκουσον Ἰακὼβ ὁ παῖς μου, καὶ Ἰσραὴλ ὃν ἐξελεξάμην.
\VS{2}Οὕτω λέγει Κύριος ὁ Θεὸς ὁ ποιήσας σε, καὶ ὁ πλάσας σε ἐκ κοιλίας, ἔτι βοηθηθήσῃ· μὴ φοβοῦ παῖς μου Ἰακὼβ, καὶ ἠγαπημένος Ἰσραὴλ ὃν ἐξελεξάμην.
\VS{3}Ὅτι ἐγὼ δώσω ὕδωρ ἐν δίψει τοῖς πορευομένοις ἐν ἀνύδρῳ, ἐπιθήσω τὸ πνεῦμά μου ἐπὶ τὸ σπέρμα σου, καὶ τὰς εὐλογίας μου ἐπὶ τὰ τέκνα σου,
\VS{4}καὶ ἀνατελοῦσιν ὡς ἀναμέσον ὕδατος χόρτος, καὶ ὡς ἰτέα ἐπὶ παραῤῥέον ὕδωρ.
\VS{5}Οὗτος ἐρεῖ, τοῦ Θεοῦ εἰμι, καὶ οὗτος βοήσεται ἐπὶ τῷ ὀνόματι Ἰακώβ· καὶ ἕτερος ἐπιγράψει χειρὶ αὐτοῦ, τοῦ Θεοῦ εἰμι, καὶ ἐπὶ τῷ ὀνόματι Ἰσραὴλ βοήσεται.
\par }{\PP \VS{6}Οὕτως λέγει ὁ Θεὸς ὁ βασιλεὺς Ἰσραὴλ, καὶ ῥυσάμενος αὐτὸν Θεὸς σαβαὼθ, ἐγὼ πρῶτος, καὶ ἐγὼ μετὰ ταῦτα· πλὴν ἐμοῦ οὐκ ἔστι Θεός.
\VS{7}Τίς ὥσπερ ἐγὼ; στήτω, καὶ καλεσάτω, καὶ ἀναγγειλάτω, καὶ ἑτοιμασάτω μοι ἀφʼ οὗ ἐποίησα ἄνθρωπον εἰς τὸν αἰῶνα, καὶ τὰ ἐπερχόμενα πρὸ τοῦ ἐλθεῖν ἀναγγειλάτωσαν ὑμῖν.
\VS{8}Μὴ παρακαλύπτεσθε, μηδὲ πλανᾶσθε· οὐκ ἀπʼ ἀρχῆς ἠνωτίσασθε, καὶ ἀπήγγειλα ὑμῖν; μάρτυρες ὑμεῖς ἐστε, εἰ ἕστι Θεὸς πλὴν ἐμοῦ.
\par }{\PP \VS{9}Καὶ οὐκ ἤκουσαν τότε οἱ πλάσσοντες· καὶ οἱ γλύφοντες, πάντες μάταιοι, ποιοῦντες τὰ καταθύμια αὐτῶν, ἃ οὐκ ὠφελήσει αὐτούς· ἀλλὰ αἰσχυνθήσονται
\VS{10}οἱ πλάσσοντες θεὸν, καὶ γλύφοντες πάντες ἀνωφελῆ,
\VS{11}καὶ πάντες ὅθεν ἐγένοντο ἐξηράνθησαν· καὶ κωφοὶ ἀπὸ ἀνθρώπων συναχθήτωσαν πάντες, καὶ στησάτωσαν ἅμα· καὶ ἐντραπήτωσαν, καὶ αἰσχυνθήτωσαν ἅμα·
\par }{\PP \VS{12}Ὅτι ὤξυνε τέκτων σίδηρον· σκεπάρνῳ εἰργάσατο αὐτὸ, καὶ ἐν τερέτρῳ ἔστησεν αὐτὸ, καὶ εἰργάσατο αὐτὸ ἐν τῷ βραχίονι τῆς ἰσχύος αὐτοῦ, καὶ πεινάσει, καὶ ἀσθενήσει, καὶ οὐ μὴ πίῃ ὕδωρ.
\VS{13}Ἐκλεξάμενος τέκτων ξύλον, ἔστησεν αὐτὸ ἐν μέτρῳ, καὶ ἐν κόλλῃ ἐῤῥύθμισεν αὐτὸ, καὶ ἐποίησεν αὐτὸ ὡς μορφὴν ἀνδρὸς, καὶ ὡς ὡραιότητα ἀνθρώπου, στῆσαι αὐτὸ ἐν οἴκῳ.
\VS{14}Ἔκοψε ξύλον ἐκ τοῦ δρυμοῦ, ὃ ἐφύτευσε Κύριος, πίτυν, καὶ ὑετὸς ἐμήκυνεν,
\VS{15}ἵνα ᾖ ἀνθρώποις εἰς καῦσιν· καὶ λαβῶν ἀπʼ αὐτοῦ, ἐθερμάνθη, καὶ καύσαντες ἔπεψαν ἄρτους ἐπʼ αὐτῶν· τὸ δὲ λοιπὸν εἰργάσαντο θεοὺς, καὶ προσκυνοῦσιν αὐτοῖς·
\VS{16}Οὗ τὸ ἥμισυ αὐτοῦ κατέκαυσεν ἐν πυρὶ, καὶ ἐπὶ τοῦ ἡμίσους αὐτοῦ ἔπεψεν ἐν τοῖς ἄνθραξιν ἄρτους, καὶ ἐπʼ αὐτοῦ κρέας ὀπτήσας ἔφαγε, καὶ ἐνεπλήσθη, καὶ θερμανθεὶς εἶπεν, ἡδύ μοι, ὅτι ἐθερμάνθην, καὶ εἶδον πῦρ.
\VS{17}Τὸ δὲ λοιπὸν ἐποίησεν εἰς θεὸν γλυπτὸν, καὶ προσκυνεῖ, καὶ προσεύχεται λέγων, ἐξελοῦ με, ὅτι θεός μου εἶ σύ.
\par }{\PP \VS{18}Οὐκ ἔγνωσαν φρονῆσαι, ὅτι ἀπημαυρώθησαν τοῦ βλέπειν τοῖς ὀφθαλμοῖς αὐτῶν, καὶ τοῦ νοῆσαι τῇ καρδίᾳ αὐτῶν.
\VS{19}Καὶ οὐκ ἐλογίσατο τῇ ψυχῇ αὐτοῦ, οὐδὲ ἔγνω τῇ φρονήσει, ὅτι τὸ ἥμισυ αὐτοῦ κατέκαυσεν ἐν πυρὶ, καὶ ἔπεψεν ἐπὶ τῶν ἀνθράκων αὐτοῦ ἄρτους, καὶ ὀπτήσας κρέα ἔφαγε, καὶ τὸ λοιπὸν αὐτοῦ εἰς βδέλυγμα ἐποίησε, καὶ προσκυνοῦσιν αὐτῷ.
\VS{20}Γνῶθι ὅτι σποδὸς ἡ καρδία αὐτῶν, καὶ πλανῶνται, καὶ οὐδεὶς δύναται ἐξελέσθαι τὴν ψυχὴν αὐτοῦ· ἴδετε, οὐκ ἐρεῖτε, ὅτι ψεῦδος ἐν τῇ δεξιᾷ μου.
\par }{\PP \VS{21}Μνήσθητι ταῦτα Ἰακὼβ καὶ Ἰσραὴλ, ὅτι παῖς μου εἶ σὺ, ἔπλασά σε παῖδά μου, καὶ σὺ Ἰσραὴλ μὴ ἐπιλανθάνου μοῦ.
\VS{22}Ἰδοὺ γὰρ ἀπήλειψα ὡς νεφέλην τὰς ἀνομίας σου, καὶ ὡς γνόφον τὴν ἁμαρτίαν σου· ἐπιστράφηθι πρὸς μὲ, καὶ λυτρώσομαί σε.
\par }{\PP \VS{23}Εὐφράνθητε οὐρανοὶ, ὅτι ἠλέησεν ὁ Θεὸς τὸν Ἰσραήλ· σαλπίσατε τὰ θεμέλια τῆς γῆς, βοήσατε ὄρη εὐφροσύνην, οἱ βουνοὶ καὶ πάντα τὰ ξύλα τὰ ἐν αὐτοῖς, ὅτι ἐλυτρώσατο ὁ Θεὸς τὸν Ἰακὼβ, καὶ Ἰσραὴλ δοξασθήσεται.
\par }{\PP \VS{24}Οὕτω λέγει Κύριος ὁ λυτρούμενός σε, καὶ πλάσσων σε ἐκ κοιλίας, ἐγὼ Κύριος ὁ συντελῶν πάντα, ἐξέτεινα τὸν οὐρανὸν μόνος, καὶ ἐστερέωσα τὴν γῆν.
\VS{25}Τίς ἕτερος διασκεδάσει σημεῖα ἐγγαστριμύθων, καὶ μαντείας ἀπὸ καρδίας; ἀποστρέφων φρονίμους εἰς τὰ ὀπίσω, καὶ τὴν βουλὴν αὐτῶν μωραίνων,
\VS{26}καὶ ἱστῶν ῥὴμα παιδὸς αὐτοῦ, καὶ τὴν βουλὴν τῶν ἀγγέλων αὐτοῦ ἀληθεύων· ὁ λέγων τῇ Ἱερουσαλὴμ, κατοικηθήσῃ, καὶ ταῖς πόλεσι τῆς Ἰδουμαίας, οἰκοδομηθήσεσθε, καὶ τὰ ἔρημα αὐτῆς ἀνατελεῖ·
\VS{27}Ὁ λέγων τῇ ἀβύσσῳ, ἐρημωθήσῃ, καὶ τοὺς ποταμούς σου ξηρανῶ·
\VS{28}Ὁ λέγων Κύρῳ φρονεῖν, καὶ πάντα τὰ θελήματά μου ποιήσει· ὁ λέγων Ἱερουσαλὴμ, οἰκοδομηθήσῃ, καὶ τὸν οἶκον τὸν ἅγιόν μου θεμελιώσω.

\par }\Chap{45}{\PP \VerseOne{1}Οὕτῳ λέγει Κύριος ὁ Θεὸς τῷ χριστῷ μου Κύρῳ, οὗ ἐκράτησα τῆς δεξιᾶς, ἐπακοῦσαι ἔμπροσθεν αὐτοῦ ἔθνη, καὶ ἰσχὺν βασιλέων διαῤῥήξω, ἀνοίξω ἔμπροσθεν αὐτοῦ θύρας, καὶ πόλεις οὐ συγκλεισθήσονται.
\VS{2}Ἐγὼ ἔμπροσθέν σου πορεύσομαι, καὶ ὄρη ὁμαλιῶ, θύρας χαλκᾶς συντρίψω, καὶ μοχλοὺς σιδηροῦς συνκλάσω.
\VS{3}Καὶ δώσω σοι θησαυροὺς σκοτεινοὺς, ἀποκρύφους, ἀοράτους ἀνοίξω σοι, ἵνα γνῷς ὅτι ἐγὼ Κύριος ὁ Θεός σου ὁ καλῶν τὸ ὄνομά σουὁ, Θεὸς Ἰσραήλ.
\VS{4}Ἕνεκεν τοῦ παιδός μου Ἰακὼβ, καὶ Ἰσραὴλ τοῦ ἐκλεκτοῦ μου, ἐγὼ καλέσω σε τῷ ὀνόματί σου, καὶ προσδέξομαί σε· σὺ δὲ οὐκ ἔγνως με,
\VS{5}ὅτι ἐγὼ Κύριος ὁ Θεὸς, καὶ οὐκ ἔστιν ἔτι πλὴν ἐμοῦ Θεός· ἐνίσχυσά σε, καὶ οὐκ ᾔδεις με,
\VS{6}ἵνα γνῶσιν οἱ ἀπʼ ἀνατολῶν ἡλίου καὶ οἱ ἀπὸ δυσμῶν, ὅτι οὐκ ἔστι Θεὸς πλὴν ἐμοῦ· ἐγὼ Κύριος ὁ Θεὸς, καὶ οὐκ ἔστιν ἔτι.
\VS{7}Ἐγὼ ἡ κατασκευάσας φῶς, καὶ ποιήσας σκότος, ὁ ποιῶν εἰρήνην, καὶ κτίζων κακά· ἐγὼ Κύριος ὁ Θεὸς, ὁ ποιῶν πάντα ταῦτα.
\par }{\PP \VS{8}Εὐφρανθήτω ὁ οὐρανὸς ἄνωθεν, καὶ αἱ νεφέλαι ῥανάτωσαν δικαιοσύνην· ἀνατειλάτω ἡ γῆ, καὶ βλαστησάτω ἔλεος, καὶ δικαιοσύνην ἀνατειλάτω ἅμα· ἐγώ εἰμι Κύριος ὁ κτίσας σε.
\par }{\PP \VS{9}Ποῖον βέλτιον κατεσκεύασα ὡς πηλὸν κεραμέως; μὴ ὁ ἀροτριῶν ἀροτριάσει τὴν γῆν ὅλην τὴν ἡμέραν; μὴ ἐρεῖ ὁ πηλὸς τῷ κεραμεῖ, τί ποιεῖς ὅτι οὐκ ἐργάζῃ, οὐδὲ ἔχεις χεῖρας; μὴ ἀποκριθήσεται τὸ πλάσμα πρὸς τὸν πλάσαντα αὐτό;
\VS{10}Ὁ λέγων τῷ πατρὶ, τί γεννήσεις; καὶ τῇ μητρί, τί ὠδίνεις;
\par }{\PP \VS{11}Ὅτι οὕτω λέγει Κύριος ὁ Θεὸς ὁ ἅγιος Ἰσραὴλ, ὁ ποιήσας τὰ ἐπερχόμενα, ἐρωτήσατέ με περὶ τῶν υἱῶν μου, καὶ περὶ τῶν ἔργων τῶν χειρῶν μου ἐντείλασθέ μοι.
\VS{12}Ἐγὼ ἐποίησα γῆν, καὶ ἄνθρωπον ἐπʼ αὐτῆς, ἐγὼ τῇ χειρί μου ἐστερέωσα τὸν οὐρανὸν, ἐγὼ πᾶσι τοῖς ἄστροις ἐνετειλάμην.
\VS{13}Ἐγὼ ἤγειρα αὐτὸν μετὰ δικαιοσύνης βασιλέα, καὶ πᾶσαι αἱ ὁδοὶ αὐτοῦ εὐθεῖαι· οὗτος οἰκοδομήσει τὴν πόλιν μου, καὶ τὴν αἰχμαλωσίαν τοῦ λαοῦ μου ἐπιστρέψει, οὐ μετὰ λύτρων, οὐδὲ μετὰ δώρων, εἶπε Κύριος σαβαώθ.
\par }{\PP \VS{14}Οὕτω λέγει Κύριος σαβαὼθ, ἐκοπίασεν Αἴγυπτος, καὶ ἐμπορία Αἰθιόπων, καὶ οἱ Σαβαεὶμ ἄνδρες ὑψηλοὶ ἐπὶ σὲ διαβήσονται, καὶ σοὶ ἔσονται δοῦλοι, καὶ ὀπίσω σου ἀκολουθήσουσι δεδεμένοι χειροπέδαις, καὶ διαβήσονται πρὸς σὲ, καὶ προσκυνήσουσί σοι, καὶ ἐν σοὶ προσεύξονται· ὅτι ἐν σοὶ ὁ Θεός ἐστι, καὶ οὐκ ἔστι Θεὸς πλήν σου.
\VS{15}Σὺ γὰρ εἶ Θεὸς, καὶ οὐκ ᾔδειμεν, ὁ Θεὸς τοῦ Ἰσραὴλ σωρήρ.
\VS{16}Αἰσχυνθήσονται καὶ ἐντραπήσονται πάντες οἱ ἀντικείμενοι αὐτῷ, καὶ πορεύσονται ἐν αἰσχύνῃ· ἐγκαινίζεσθε πρὸς μὲ νῆσοι.
\VS{17}Ἰσραὴλ σώζεται ὑπὸ Κυρίου σωτηρίαν αἰώνιον· οὐκ αἰσχυνθήσονται, οὐδὲ μὴ ἐντραπῶσιν ἕως τοῦ αἰῶνος ἔτι.
\par }{\PP \VS{18}Οὕτως λέγει Κύριος ὁ ποιήσας τὸν οὐρανὸν, οὗτος ὁ Θεὸς ὁ καταδείξας τὴν γῆν, καὶ ποιήσας αὐτὴν, αὐτὸς διώρισεν αὐτὴν, οὐκ εἰς κενὸν ἐποίησεν αὐτὴν, ἀλλὰ κατοικεῖσθαι ἔπλασεν αὐτὴν, ἐγώ εἰμι Κύριος, καὶ οὐκ ἔστιν ἔτι.
\VS{19}Οὐκ ἐν κρυφῇ λελάληκα, οὐδὲ ἐν τόπῳ γῆς σκοτεινῷ· οὐκ εἶπα τῷ σπέρματι Ἰακὼβ, μάταιον ζητήσατε· ἐγώ εἰμι ἐγώ εἰμι Κύριος ὁ λαλῶν δικαιοσύνην, καὶ ἀναγγέλλων ἀλήθειαν.
\par }{\PP \VS{20}Συνάχθητε, καὶ ἥκετε, βουλεύσασθε ἅμα οἱ σωζόμενοι ἀπὸ τῶν ἐθνῶν· οὐκ ἔγνωσαν οἱ αἴροντες τὸ ξύλον γλύμμα αὐτῶν, καὶ οἱ προσευχόμενοι πρὸς θεοὺς, οἳ οὐ σώζουσιν.
\VS{21}Εἰ ἀναγγελοῦσιν, ἐγγισάτωσαν, ἵνα γνῶσιν ἅμα, τίς ἀκουστὰ ἐποίησε ταῦτα ἀπʼ ἀρχῆς· τότε ἀνηγγέλη ὑμῖν· ἐγὼ ὁ Θεὸς, καὶ οὐκ ἔστιν ἄλλος πλὴν ἐμοῦ· δίκαιος καὶ σωτὴρ, οὐκ ἔστι πάρεξ ἐμοῦ.
\VS{22}Ἐπιστράφητε ἐπʼ ἐμὲ, καὶ σωθήσεσθε, οἱ ἀπʼ ἐσχάτου τῆς γῆς· ἐγώ εἰμι ὁ Θεὸς, καὶ οὐκ ἔστιν ἄλλος.
\VS{23}Κατʼ ἐμαυτοῦ ὀμνύω, εἰ μὴ ἐξελεύσεται ἐκ τοῦ στόματός μου δικαιοσύνη, οἱ λόγοι μου οὐκ ἀποστραφήσονται· Ὅτι ἐμοὶ κάμψει πᾶν γόνυ, καὶ ὀμεῖται πᾶσα γλῶσσα τὸν Θεὸν,
\VS{24}λέγων, δικαιοσύνη καὶ δόξα πρὸς αὐτὸν ἥξει, καὶ αἰσχυνθήσονται πάντες οἱ διορίζοντες αὐτούς.
\VS{25}Ἀπὸ Κυρίου δικαιωθήσονται, καὶ ἐν τῷ Θεῷ ἐνδοξασθήσεται πᾶν τὸ σπέρμα τῶν υἱῶν Ἰσραήλ.

\par }\Chap{46}{\PP \VerseOne{1}Ἔπεσε Βὴλ, συνετρίβη Ναβὼ, ἐγένετο τὰ γλυπτὰ αὐτῶν εἰς θηρία, καὶ τὰ κτήνη· αἴρετε αὐτὰ καταδεδεμένα ὡς φορτίον κοπιῶντι ἐκλευμένῳ,
\VS{2}καὶ πεινῶντι, οὐκ ἰσχύοντι, ἅμα, οἳ οὐ δυνήσονται σωθῆναι ἀπὸ πολέμου, αὐτοὶ δὲ αἰχμάλωτοι ἤχθησαν.
\par }{\PP \VS{3}Ἀκούετέ μου οἶκος τοῦ Ἰακὼβ, καὶ πᾶν τὸ κατάλοιπον τοῦ Ἰσραὴλ, οἱ αἰρόμενοι ἐκ κοιλίας, καὶ παιδευόμενοι ἐκ παιδίου ἕως γήρως·
\VS{4}ἐγώ εἰμι, καὶ ἕως ἂν καταγηράσητε, ἐγώ εἰμι, ἐγὼ ἀνέχομαι ὑμῶν, ἐγὼ ἐποίησα, καὶ ἐγὼ ἀνήσω, ἐγὼ ἀναλήμψομαι, καὶ σώσω ὑμᾶς.
\par }{\PP \VS{5}Τίνι με ὡμοιώσατε; ἴδετε, τεχνάσασθε, οἱ πλανώμενοι.
\VS{6}Οἱ συμβαλλόμενοι χρυσίον ἐκ μαρσυππίου, καὶ ἀργύριον ἐν ζυγῷ, στήσουσιν ἐν σταθμῷ, καὶ μισθωσάμενοι χρυσοχόον ἐποίησαν χειροποίητα, καὶ κύψαντες προσκυνοῦσιν αὐτοῖς.
\VS{7}Αἴρουσιν αὐτὸ ἐπὶ τοῦ ὤμου, καὶ πορεύονται· ἐὰν δὲ θῶσιν αὐτό ἐπὶ τοῦ τόπου αὐτοῦ, μένει, οὐ μὴ κινηθῇ· καὶ ὃς ἐὰν βοήσῃ πρὸς αὐτὸν, οὐ μὴ εἰσακούσῃ, ἀπὸ κακῶν οὐ μὴ σώσῃ αὐτόν.
\par }{\PP \VS{8}Μνήσθητε ταῦτα, καὶ στενάξατε, μετανοήσατε οἱ πεπλανημένοι, ἐπιστρέψατε τῇ καρδίᾳ,
\VS{9}καὶ μνήσθητε τὰ πρότερα ἀπὸ τοῦ αἰῶνος, ὅτι ἐγώ εἰμι ὁ Θεὸς, καὶ οὐκ ἔστιν ἔτι πλὴν ἐμοῦ,
\VS{10}ἀναγγέλλων πρότερον τὰ ἔσχατα πρὶν γενέσθαι, καὶ ἅμα συνετελέσθη· καὶ εἶπα, πᾶσα ἡ βουλή μου στήσεται, καὶ πάντα ὅσα βεβούλευμαι, ποιήσω·
\VS{11}Καλῶν ἀπὸ ἀνατολῶν πετεινὸν, καὶ ἀπὸ γῆς πόῤῥωθεν περὶ ὧν βεβούλευμαι, ἐλάλησα, καὶ ἤγαγον, ἔκτισα καὶ ἐποίησα, ἤγαγον αὐτὸν, καὶ εὐώδωσα τὴν ὁδὸν αὐτοῦ.
\par }{\PP \VS{12}Ἀκούσατέ μου οἱ ἀπολωλεκότες τὴν καρδίαν, οἱ μακρὰν ἀπὸ τῆς δικαιοσύνης.
\VS{13}Ἤγγισα τὴν δικαιοσύνην μου, καὶ τὴν σωτηρίαν τὴν παρʼ ἐμοῦ οὐ βραδυνῶ· δέδωκα ἐν Σιὼν σωτηρίαν τῷ Ἰσραὴλ εἰς δόξασμα.

\par }\Chap{47}{\PP \VerseOne{1}Κατάβηθι, κάθισον ἐπὶ τὴν γῆν παρθένος θυγάτηρ Βαβυλῶνος, κάθισον εἰς τὴν γῆν θυγάτηρ Χαλδαίων, ὅτι οὐκέτι προστεθήσῃ κληθῆναι ἁπαλὴ καὶ τρυφερά.
\VS{2}Λάβε μύλον, ἄλεσον ἄλευρον, ἀποκὸλυψαι τὸ κατακάλυμμά σου, ἀνακάλυψαι τὰς πολιὰς, ἀνάσυρε τὰς κνήμας, διάβηθι ποταμούς.
\VS{3}Ἀνακαλυφθήσεται ἡ αἰσχύνη σου, φανήσονται οἱ ὀνειδισμοί σου· τὸ δίκαιον ἐκ σοῦ λήμψομαι, οὐκέτι μὴ παραδῶ ἀνθρώποις.
\par }{\PP \VS{4}Ὁ ῥυσάμενός σε Κύριος σαβαὼθ, ὄνομα αὐτῷ Ἅγιος Ἰσραήλ.
\par }{\PP \VS{5}Κάθισον κατανενυγμένη, εἴσελθε εἰς τὸ σκότος θύγατεηρ Χαλδαίων, οὐκέτι μὴ κληθήσῃ ἰσχὺς βασιλείας.
\VS{6}Παρωξύνθην ἐπὶ τῷ λαῷ μου, ἐμίανας τὴν κληρονομίαν μου· ἐγὼ ἔδωκα αὐτοὺς εἰς τὴν χεῖρά σου, σὺ δὲ οὐκ ἔδωκας αὐτοῖς ἔλεος, τοῦ πρεσβυτέρου ἐβάρυνας τὸν ζυγὸν σφόδρα,
\VS{7}καὶ εἶπας, εἰς τὸν αἰῶνα ἔσομαι ἄρχουσα· οὐκ ἐνόησας ταῦτα ἐν τῇ καρδίᾳ σου, οὐδὲ ἐμνήσθης τὰ ἔσχατα.
\par }{\PP \VS{8}Νῦν δὲ ἄκουε ταῦτα τρυφερὰ, ἡ καθημένη, ἡ πεποιθυῖα, ἡ λέγουσα ἐν καρδίᾳ αὐτῆς, ἐγώ εἰμι, καὶ οὐκ ἔστιν ἑτέρα, οὐ καθιῶ χήρα, οὐδὲ γνώσομαι ὀρφανίαν.
\VS{9}Νῦν δὲ ἥξει ἐπὶ σὲ τὰ δύο ταῦτα ἐξαίφνης ἐν ἡμέρᾳ μιᾷ, ἀτεκνία καὶ χηρεία ἥξει ἐξαίφνης ἐπὶ σὲ, ἐν τῇ φαρμακείᾳ σου, ἐν τῇ ἰσχύϊ τῶν ἐπαοιδῶν σου σφόδρα,
\VS{10}τῇ ἐλπίδι τῆς πονηρίας σου· σὺ γὰρ εἶπας, ἐγώ εἰμι, καὶ οὐκ ἔστιν ἑτέρα· γνῶθι, ἡ σύνεσις τούτων ἔσται, καὶ ἡ πορνεία σου σοὶ αἰσχύνη· καὶ εἶπας τῇ καρδίᾳ σου, ἐγώ εἰμι, καὶ οὐκ ἔστιν ἑτέρα.
\par }{\PP \VS{11}Καὶ ἥξει ἐπὶ σὲ ἀπώλεια, καὶ οὐ μὴ γνῷς· βόθυνος, καὶ ἐμπεσῇ εἰς αὐτόν· καὶ ἥξει ἐπὶ σὲ ταλαιπωρία, καὶ οὐ μὴ δυνήσῃ καθαρὰ γενέσθαι· καὶ ἥξει ἐπὶ σὲ ἐξαπίνης ἀπώλεια, καὶ οὐ μὴ γνώσῃ.
\VS{12}Στῆθι νῦν ἐν ταῖς ἐπαοιδαῖς σου, καὶ ἐν τῇ πολλῇ φαρμακείᾳ σου, ἃ ἐμάνθανες ἐκ νεότητός σου, εἰ δυνήσῃ ὠφεληθῆναι.
\VS{13}Κεκοπίακας ἐν ταῖς βουλαῖς σου· στήτωσαν δὴ καὶ σωσάτωσάν σε οἱ ἀστρολόγοι τοῦ οὐρανοῦ, οἱ ὁρῶντες τοὺς ἀστέρας ἀναγγειλάτωσάν σοι, τί μέλλει ἐπὶ σὲ ἔρχεσθαι.
\VS{14}Ἰδοὺ πάντες ὡς φρύγανα ἐπὶ πυρὶ κατακαυθήσονται, καὶ οὐ μὴ ἐξέλωνται τὴν ψυχὴν αὐτῶν ἐκ φλογός· ὅτι ἔχεις ἄνθρακας πυρός· κάθισαι ἐπʼ αὐτοὺς,
\VS{15}οὗτοι ἔσονταί σοι βοήθεια· ἐκοπίασας ἐν τῇ μεταβολῇ ἐκ νεότητος, ἄνθρωπος καθʼ ἑαυτὸν ἐπλανήθη, σοὶ δὲ οὐκ ἔσται σωτηρία.

\par }\Chap{48}{\PP \VerseOne{1}Ἀκούσατε ταῦτα οἶκος Ἰακὼβ, οἱ κεκλημένοι ἐπὶ τῷ ὀνόματι Ἰσραὴλ, καὶ ἐκ Ἰούδα ἐξελθόντες, οἱ ὀμνύοντες τῷ ὀνόματι Κυρίου Θεοῦ Ἰσραὴλ, μιμνησκόμενοι οὐ μετὰ ἀληθείας, οὐδὲ μετὰ δικαιοσύνης,
\VS{2}καὶ ἀντεχόμενοι τῷ ὀνόματι τῆς πόλεως τῆς ἁγίας, καὶ ἐπὶ τῷ Θεῷ Ἰσραὴλ ἀντιστηριζόμενοι· Κύριος σαβαὼθ ὄνομα αὐτῷ.
\VS{3}Τὰ πρότερα ἔτι ἀνήγγειλα, καὶ ἐκ τοῦ στόματός μου ἐξῆλθε, καὶ ἀκουστὸν ἐγένετο· ἐξάπινα ἐποίησα, καὶ ἐπῆλθε.
\par }{\PP \VS{4}Γινώσκω ὅτι σκληρὸς εἶ, καὶ νεῦρον σιδηροῦν ὁ τράχηλός σου, καὶ τὸ μέτωπόν σου χαλκοῦν.
\VS{5}Καὶ ἀνήγγειλά σοι πάλαι ἃ πρὶν ἐλθεῖν ἐπὶ σέ· ἀκουστόν σοι ἐποίησα, μή ποτε εἴπῃς, ὅτι τὰ εἴδωλά μοι ἐποίησε, καὶ εἴπῃς, ὅτι τὰ γλυπτὰ καὶ τὰ χωνευτὰ ἐνετείλατό μοι.
\VS{6}Ἠκούσατε πάντα, καὶ ὑμεῖς οὐκ ἔγνωτε· ἀλλὰ ἀκουστά σοι ἐποίησα τὰ καινὰ ἀπὸ τοῦ νῦν, ἃ μέλλει γίνεσθαι· καὶ οὐκ εἶπας,
\VS{7}νῦν γίνεται, καὶ οὐ πάλαι, καὶ οὐ προτέραις ἡμέραις ἤκουσας αὐτά· μὴ εἴπῃς, ναὶ, γινώσκω αὐτά.
\VS{8}Οὔτε ἔγνως, οὔτε ἠπίστω, οὔτε ἀπʼ ἀρχῆς ἤνοιξά σου τὰ ὦτα· ἔγνων γὰρ, ὅτι ἀθετῶν ἀθετήσεις, καὶ ἄνομος ἔτι ἐκ κοιλίας κληθήσῃ.
\par }{\PP \VS{9}Ἕνεκεν τοῦ ἐμοῦ ὀνόματος δείξω σοι τὸν θυμόν μου, καὶ τὰ ἔνδοξά μου ἐπάξω ἐπὶ σὲ, ἵνα μὴ ἐξολοθρεύσω σε.
\VS{10}Ἰδοὺ πέπρακά σε, οὐχ ἕνεκεν ἀργυρίου· ἐξειλάμην δέ σε ἐκ καμίνου πτωχείας·
\VS{11}Ἕνεκεν ἐμοῦ ποιήσω σοι, ὅτι τὸ ἐμὸν ὄνομα βεβηλοῦται, καὶ τὴν δόξαν μου ἑτέρῳ οὐ δώσω.
\par }{\PP \VS{12}Ἄκουέ μου Ἰακὼβ, καὶ Ἰσραὴλ ὃν ἐγὼ καλῶ· ἐγώ εἰμι πρῶτος, καὶ ἐγώ εἰμι εἰς τὸν αἰῶνα.
\VS{13}Καὶ ἡ χείρ μου ἐθεμελίωσε τὴν γῆν, καὶ ἡ δεξιά μου ἐστερέωσε τὸν οὐρανόν· καλέσω αὐτοὺς, καὶ στήσονται ἅμα,
\VS{14}καὶ συναχθήσονται πάντες, καὶ ἀκούσονται· τίς αὐτοῖς ἀνήγγειλε ταῦτα; ἀγαπῶν σε ἐποίησα τὸ θέλημά σου ἐπὶ Βαβυλῶνα, τοῦ ἆραι σπέρμα Χαλδαίων.
\VS{15}Ἐγὼ ἐλάλησα, ἐγὼ ἐκάλεσα, ἤγαγον αὐτὸν, καὶ εὐώδωσα τὴν ὁδὸν αὐτοῦ.
\par }{\PP \VS{16}Προσαγάγετε πρὸς μὲ, καὶ ἀκούσατε ταῦτα· οὐκ ἀπʼ ἀρχῆς ἐν κρυφῇ λελάληκα· ἡνίκα ἐγένετο, ἐκεῖ ἤμην, καὶ νῦν Κύριος Κύριος ἀπέστειλέ με, καὶ τὸ πνεῦμα αὐτοῦ.
\VS{17}Οὕτω λέγει Κύριος; ὁ ῥυσάμενος σε, ἅγιος Ἰσραὴλ, ἐγώ εἰμι ὁ Θεός σου, δέδειχά σοι τοῦ εὑρεῖν σε τὴν ὁδὸν ἐν ᾗ πορεύσῃ ἐν αὐτῇ.
\VS{18}Καὶ εἰ ἤκουσας τῶν ἐντολῶν μου, ἐγένετο ἂν ὡσεὶ ποταμὸς ἡ εἰρήνη σου, καὶ ἡ δικαιοσύνη σου ὡς κῦμα θαλάσσης·
\VS{19}Καὶ ἐγένετο ἂν ὡς ἡ ἄμμος τὸ σπέρμα σου, καὶ τὰ ἔκγονα τῆς κοιλίας σου ὡς ὁ χοῦς τῆς γῆς· οὐδὲ νῦν οὐ μὴ ἐξολοθρευθῇς, οὐδὲ ἀπολεῖται τὸ ὄνομά σου ἐνώπιον ἐμοῦ.
\par }{\PP \VS{20}Ἔξελθε ἐκ Βαβυλῶνος φεύγων ἀπὸ τῶν Χαλδαίων, φωνὴν εὐφροσύνης ἀναγγειλατε, καὶ ἀκουστὸν γενέσθω τοῦτο, ἀναγγείλατε ἕως ἐσχάτου τῆς γῆς· λέγετε, Ἐῤῥύσατο Κύριος τὸν δοῦλον αὐτοῦ Ἰακώβ·
\VS{21}Καὶ ἐὰν διψήσωσι, διʼ ἐρήμου ἄξει αὐτοὺς, ὕδωρ ἐκ πέτρας ἐξάξει αὐτοῖς· σχισθήσεται πέτρα, καὶ ῥυήσεται ὕδωρ, καὶ πίεται ὁ λαός μου.
\VS{22}Οὐκ ἔστι χαίρειν, λέγει Κύριος, τοῖς ἀσεβέσιν.

\par }\Chap{49}{\PP \VerseOne{1}Ἀκούσατε μου νῆσοι, καὶ προσέχετε ἔθνη, διὰ χρόνου πολλοῦ στήσεται, λέγει Κύριος, ἐκ κοιλίας μητρός μου ἐκάλεσε τὸ ὄνομά μου.
\VS{2}Καὶ ἔθηκε τὸ στόμα μου ὡς μάχαιραν ὀξεῖαν, καὶ ὑπὸ τὴν σκέπην τῆς χειρὸς αὐτοῦ ἔκρυψέ με· ἔθηκέ με ὡς βέλος ἐκλεκτὸν, καὶ ἐν τῇ φαρέτρᾳ αὐτοῦ ἔκρυψέ με,
\VS{3}καὶ εἶπέ μοι, δοῦλός μου εἶ σὺ Ἰσραὴλ, καὶ ἐν σοὶ ἐνδοξασθήσομαι.
\VS{4}Καὶ ἐγὼ εἶπα, κενῶς ἐκοπίασα, εἰς μάταιον καὶ εἰς οὐδὲν ἔδωκα τὴν ἰσχύν μου· διατοῦτο ἡ κρίσις μου παρὰ· Κυρίῳ, καὶ ὁ πόνος μου ἐναντίον τοῦ Θεοῦ μου.
\VS{5}Καὶ νῦν οὕτω λέγει Κύριος, ὁ πλάσας με ἐκ κοιλίας δοῦλον ἑαυτῷ, τοῦ συναγαγεῖν τὸν Ἰακὼβ πρὸς αὐτὸν καὶ Ἰσραήλ· συναχθήσομαι καὶ δοξασθήσομαι ἐναντίον Κυρίου, καὶ ὁ Θεὸς μου ἔσται μοι ἰσχύς.
\VS{6}Καὶ εἶπέ μοι, μέγα σοι ἐστὶ τοῦ κληθῆναί σε παῖδά μου, τοῦ στῆσαι τὰς φυλὰς Ἰακὼβ, καὶ τὴν διασπορὰν τοῦ Ἰσραὴλ ἐπιστρέψαι· ἰδοὺ δέδωκά σε εἰς διαθήκην γένους, εἰς φῶς ἐθνῶν, τοῦ εἶναί σε εἰς σωτηρίαν ἕως ἐσχάτου τῆς γῆς.
\par }{\PP \VS{7}Οὕτως λέγει Κύριος, ὁ ῥυσάμενός σε, ὁ Θεὸς Ἰσραὴλ, ἁγιάσατε τὸν φαυλίζοντα τὴν ψυχὴν αὐτοῦ, τὸν βδελυσσόμενον ὑπὸ τῶν ἐθνῶν τῶν δούλων τῶν ἀρχόντων· βασιλεῖς ὄψονται αὐτὸν, καὶ ἀναστήσονται ἄρχοντες, καὶ προσκυνήσουσιν αὐτῷ, ἕνεκεν Κυρίου· ὅτι πιστός ἐστιν ὁ ἅγιος Ἰσραὴλ, καὶ ἐξελεξάμην σε.
\par }{\PP \VS{8}Οὕτως λέγει Κύριος, καιρῷ δεκτῷ ἐπήκουσά σου, καὶ ἐν ἡμέρᾳ σωτηρίας ἐβοήθησά σοι, καὶ ἔπλασά σε, καὶ ἔδωκά σε εἰς διαθήκην ἐθνῶν τοῦ καταστῆσαι τὴν γῆν, καὶ κληρονομῆσαι κληρονομίας ἐρήμους,
\VS{9}λέγοντα τοῖς ἐν δεσμοῖς, ἐξέλθατε, καὶ τοῖς ἐν τῷ σκότει, ἀνακαλυφθῆναι· ἐν πάσαις ταῖς ὁδοῖς βοσκηθήσονται, καὶ ἐν πάσαις ταῖς τρίβοις ἡ νομὴ αὐτῶν·
\VS{10}Οὐ πεινάσουσιν, οὐδὲ διψήσουσιν, οὐδὲ πατάξει αὐτοὺς ὁ καύσων, οὐδὲ ὁ ἥλιος, ἀλλʼ ὁ ἐλεῶν αὐτοὺς παρακαλέσει, καὶ διὰ πηγῶν ὑδάτων ἄξει αὐτούς·
\VS{11}Καὶ θήσω πᾶν ὄρος εἰς ὁδὸν, καὶ πᾶσαν τρίβον εἰς βόσκημα αὐτοῖς.
\VS{12}Ἰδοὺ οὗτοι πόῤῥωθεν ἥξουσιν, οὗτοι ἀπὸ Βοῤῥᾶ καὶ θαλάσσης, ἄλλοι δὲ ἐκ γῆς Περσῶν.
\par }{\PP \VS{13}Εὐφραίνεσθε οὐρανοί, καὶ ἀγαλλιάσθω ἡ γῆ, ῥηξάτωσαν τὰ ὄρη εὐφροσύνην, ὅτι ἠλέησεν ὁ Θεὸς τὸν λαὸν αὐτοῦ, καὶ τοὺς ταπεινοὺς τοῦ λαοῦ αὐτοῦ παρεκάλεσεν.
\par }{\PP \VS{14}Εἶπεν δὲ Σιὼν, ἐγκατέλιπέ με Κύριος, καὶ ὅτι Κύριος ἐπελάθετό μου.
\VS{15}Μὴ ἐπιλήσεται γυνὴ τοῦ παιδίου αὐτῆς, ἢ τοῦ μὴ ἐλεῆσαι τὰ ἔκγονα τῆς κοιλίας αὐτῆς; εἰ δὲ καὶ ταῦτα ἐπιλάθοιτο γυνὴ, ἀλλʼ ἐγὼ οὐκ ἐπιλήσομαί σου, εἶπεν Κύριος.
\par }{\PP \VS{16}Ἰδοὺ ἐπὶ τῶν χειρῶν μου ἐζωγράφηκά σου τὰ τείχη, καὶ ἐνώπιόν μου εἶ διαπαντὸς,
\VS{17}καὶ ταχὺ οἰκοδομηθήσῃ ὑφʼ ὧν καθῃρέθης, καὶ οἱ ἐρημώσαντές σε ἐξελεύσονται ἐκ σοῦ.
\par }{\PP \VS{18}Ἆρον κύκλῳ τοὺς ὀφθαλμούς σου, καὶ ἴδε πάντας, ἰδοὺ συνήχθησαν καὶ ἤλθοσαν πρὸς σέ. Ζῶ ἐγὼ, λέγει Κύριος, ὅτι πάντας αὐτοὺς ὡς κόσμον ἐνδύσῃ, καὶ περιθήσεις αὐτοὺς ὡς κόσμον νύμφη.
\VS{19}Ὅτι τὰ ἔρημά σου καὶ τὰ κατεφθαρμένα καὶ πεπτωκότα, ὅτι νῦν στενοχωρήσει ἀπὸ τῶν κατοικούντων, καὶ μακρυνθήσονται ἀπὸ σοῦ οἱ καταπίνοντές σε.
\VS{20}Ἐροῦσι γὰρ εἰς τὰ ὦτά σου οἱ υἱοί σου, οὓς ἀπολώλεκας, στενός μοι ὁ τόπος, ποίησόν μοι τόπον ἵνα κατοικήσω.
\VS{21}Καὶ ἐρεῖς ἐν τῇ καρδίᾳ σου, τίς ἐγέννησέ μοι τούτους; ἐγὼ δὲ ἄτεκνος καὶ χήρα, τούτους δὲ τίς ἐξέθρεψέ μοι; ἐγὼ δὲ κατελείφθην μόνη, οὗτοι δέ μοι ποῦ ἦσαν;
\par }{\PP \VS{22}Οὕτως λέγει Κύριος Κύριος, ἰδοὺ αἴρω εἰς τὰ ἔθνη τὴν χεῖρά μου, καὶ εἰς τὰς νήσους ἀρῶ σύσσημόν μου, καὶ ἄξουσι τοὺς υἱούς σου ἐν κόλπῳ, τὰς δὲ θυγατέρας σου ἐπʼ ὤμων ἀροῦσι.
\VS{23}Καὶ ἔσονται βασιλεῖς τιθηνοί σου, αἱ δὲ ἄρχουσαι αὐτῶν τροφοί σου· ἐπὶ πρόσωπον τῆς γῆς προσκυνήσουσί σε, καὶ τὸν χοῦν τῶν ποδῶν σου λείξουσι, καὶ γνώσῃ, ὅτι ἐγὼ Κύριος, καὶ οὐκ αἰσχυνθήσονται οἱ ὑπομένοντές με.
\par }{\PP \VS{24}Μὴ λήμψεταί τις παρὰ γίγαντος σκῦλα; καὶ ἐὰν αἰχμαλωτεύσῃ τις ἀδίκως, σωθήσεται;
\VS{25}Ὅτι οὕτω λέγει Κύριος, ἐάν τις αἰχμαλωτεύσῃ γίγαντα, λήψεται σκῦλα· λαμβάνων δὲ παρὰ ἰσχύοντος σωθήσεται· ἐγὼ δὲ τὴν κρίσιν σου κρινῶ, καὶ ἐγὼ τοὺς υἱούς σου ῥύσομαι·
\VS{26}Καὶ φάγονται οἱ θλίψαντές σε τὰς σάρκας αὐτῶν, καὶ πίονται ὡς οἶνον νέον τὸ αἷμα αὐτῶν, καὶ μεθυσθήσονται· καὶ αἰσθανθήσεται πᾶσα σὰρξ, ὅτι ἐγὼ Κύριος ὁ ῥυσάμενός σε, καὶ ἀντιλαμβανόμενος ἰσχύος Ἰακώβ.

\par }\Chap{50}{\PP \VerseOne{1}Οὕτως λέγει Κύριος, ποῖον τὸ βιβλίον τοῦ ἀποστασίου τῆς μητρὸς ὑμῶν, ᾧ ἐξαπέστειλα αὐτήν; ἢ τίνι ὑπόχρεῳ πέπρακα ὑμᾶς; ἰδοὺ ταῖς ἁμαρτίαις ὑμῶν ἐπράθητε, καὶ ταῖς ἀνομίαις ὑμῶν ἐξαπέστειλα τὴν μητέρα ὑμῶν.
\VS{2}Τί ὅτι ἦλθον, καὶ οὐκ ἦν ἄνθρωπος; ἐκάλεσα, καὶ οὐκ ἦν ὁ ὑπακούων; μὴ οὐκ ἰσχύει ἡ χείρ μου τοῦ ῥύσασθαι; ἢ οὐκ ἰσχύω τοῦ ἐξελέσθαι; ἰδοὺ τῷ ἐλεγμῷ μου ἐξερημώσω τὴν θάλασσαν, καὶ θήσω ποταμοὺς ἐρήμους, καὶ ξηρανθήσονται οἱ ἰχθύες αὐτῶν ἀπὸ τοῦ μὴ εἶναι ὕδωρ, καὶ ἀποθανοῦνται ἐν δίψει.
\VS{3}Ἐνδύσω τὸν οὐρανὸν σκότος, καὶ ὡς σάκκον θήσω τὸ περιβόλαιον αὐτοῦ.
\par }{\PP \VS{4}Κύριος Κύριος δίδωσί μοι γλῶσσαν παιδείας, τοῦ γνῶναι ἡνίκα δεῖ εἰπεῖν λόγον· ἔθηκέ μοι πρωῒ, προσέθηκέ μοι ὠτίον ἀκούειν,
\VS{5}καὶ ἡ παιδεία Κυρίου Κυρίου ἀνοίγει μου τὰ ὦτα· ἐγὼ δὲ οὐκ ἀπειθῶ, οὐδὲ ἀντιλέγω.
\VS{6}Τὸν νῶτόν μου ἔδωκα εἰς μάστιγας, τὰς δὲ σιαγόνας μου εἰς ῥαπίσματα, τὸ δὲ πρόσωπόν μου οὐκ ἀπέστρεψα ἀπὸ αἰσχύνης ἐμπτυσμάτων,
\VS{7}καὶ Κύριος Κύριος βοηθός μοι ἐγενήθη· διατοῦτο οὐκ ἐνετράπην, ἀλλὰ ἔθηκα τὸ πρόσωπόν μου ὡς στερεὰν πέτραν, καὶ ἔγνων ὅτι οὐ μὴ αἰσχυνθῶ,
\VS{8}ὅτι ἐγγίζει ὁ δικαιώσας με· τίς ὁ κρινόμενός μοι; ἀντιστήτω μοι ἅμα· καὶ τις ὁ κρινόμενός μοι; ἐγγισάτω μοι.
\VS{9}Ἰδοὺ Κύριος Κύριος βοηθήσει μοι· τίς κακώσει με; ἰδοὺ πάντες ὑμεῖς ὡς ἱμάτιον παλαιωθήσεσθε, καὶ σὴς καταφάγεται ὑμᾶς.
\par }{\PP \VS{10}Τίς ἐν ὑμῖν ὁ φοβούμενος τὸν Κύριον; ὑπακουσάτω τῆς φωνῆς τοῦ παιδὸς αὐτοῦ· οἱ πορευόμενοι ἐν σκότει, καὶ οὐκ ἔστιν αὐτοῖς φῶς, πεποίθατε ἐπὶ τῷ ὀνόματι Κυρίου, καὶ ἀντιστηρίσασθε ἐπὶ τῷ Θεῷ.
\VS{11}Ἰδοὺ πάντες ὑμεῖς πῦρ καίετε, καὶ κατισχύετε φλόγα· πορεύεσθε τῷ φωτὶ τοῦ πυρὸς ὑμῶν, καὶ τῇ φλογὶ ᾗ ἐξεκαύσατε· διʼ ἐμὲ ἐγένετο ταῦτα ὑμῖν, ἐν λύπῃ κοιμηθήσεσθε.

\par }\Chap{51}{\PP \VerseOne{1}Ἀκούσατε μου οἱ διώκοντες τὸ δίκαιον, καὶ ζητοῦντες τὸν Κύριον, ἐμβλέψατε εἰς τὴν στερεὰν πέτραν, ἣν ἐλατομήσατε, καὶ εἰς τὸν βόθυνον τοῦ λάκκου, ὃν ὠρύξατε.
\VS{2}Ἐμβλέψατε εἰς Ἁβραὰμ τὸν πατέρα ὑμῶν, καὶ εἰς Σάῤῥαν τὴν ὠδίνουσαν ὑμᾶς· ὅτι εἷς ἦν, καὶ ἐκάλεσα αὐτὸν, καὶ εὐλόγησα αὐτὸν, καὶ ἠγάπησα αὐτὸν, καὶ ἐπλήθυνα αὐτόν.
\VS{3}Καὶ σὲ νῦν παρακαλέσω, Σιὼν, καὶ παρεκάλεσα πάντα τὰ ἔρημα αὐτῆς, καὶ θήσω τὰ ἔρημα αὐτῆς ὡς παράδεισον, καὶ τὰ πρὸς δυσμὰς αὐτῆς ὡς παράδεισον Κυρίου· εὐφροσύνην καὶ ἀγαλλίαμα εὑρήσουσιν ἐν αὐτῇ, ἐξομολόγησιν καὶ φωνὴν αἰνέσεως.
\par }{\PP \VS{4}Ἀκούσατέ μου, ἀκούσατέ μου λαός μου, καὶ οἱ βασιλεῖς πρὸς μὲ ἐνωτίσασθε, ὅτι νόμος παρʼ ἐμοῦ ἐξελεύσεται, καὶ ἡ κρίσις μου εἰς φῶς ἐθνῶν.
\VS{5}Ἐγγίζει ταχὺ ἡ δικαιοσύνη μου, καὶ ἐξελεύσεται ὡς φῶς τὸ σωτήριόν μου, καὶ εἰς τὸν βραχίονά μου ἔθνη ἐλπιοῦσιν· ἐμὲ νῆσοι ὑπομενοῦσι, καὶ εἰς τὸν βραχίονά μου ἐλπιοῦσιν.
\VS{6}Ἄρατε εἰς τὸν οὐρανὸν τοὺς ὀφθαλμοὺς ὑμῶν, καὶ ἐμβλέψατε εἰς τὴν γῆν κάτω· ὅτι ὁ οὐρανὸς ὡς καπνὸς ἐστερεώθη, ἡ δὲ γῆ ὡς ἱμάτιον παλαιωθήσεται, οἱ δὲ κατοικοῦντες ὥσπερ ταῦτα ἀποθανοῦνται, τὸ δὲ σωτήριόν μου εἰς τὸν αἰῶνα ἔσται, ἡ δὲ δικαιοσύνη μου οὐ μὴ ἐκλίπῃ.
\par }{\PP \VS{7}Ἀκούσατέ μου οἱ εἰδότες κρίσιν, λαὸς οὗ ὁ νόμος μου ἐν τῇ καρδίᾳ ὑμῶν· μὴ φοβεῖσθε ὀνειδισμὸν ἀνθρώπων, καὶ τῷ φαυλισμῷ αὐτῶν μὴ ἡττᾶσθε·
\VS{8}Ὡς γὰρ ἱμάτιον βρωθήσεται ὑπὸ χρόνου, καὶ ὡς ἔρια βρωθήσεται ὑπὸ σητός· ἡ δὲ δικαιοσύνη μου εἰς τὸν αἰῶνα ἔσται, τὸ δὲ σωτήριόν μου εἰς γενεὰς γενεῶν.
\par }{\PP \VS{9}Ἐξεγείρου ἐξεγείρου Ἱερουσαλὴμ, καὶ ἔνδυσαι τὴν ἰσχὺν τοῦ βραχίονός σου, ἐξεγείρου ὡς ἐν ἀρχῇ ἡμέρας, ὡς γενεὰ αἰῶνος.
\VS{10}Οὐ σὺ εἶ ἡ ἐρημοῦσα θάλασσαν, ὕδωρ ἀβύσσου πλῆθος; ἡ θεῖσα τὰ βάθη τῆς θαλάσσης ὁδὸν διαβάσεως ῥυομένοις καὶ λελυτρωμένοις;
\VS{11}ὑπὸ γὰρ Κυρίου ἀποστραφήσονται, καὶ ἥξουσιν εἰς Σιὼν μετʼ εὐφροσύνης καὶ ἀγαλλιάματος αἰωνίου· ἐπὶ κεφαλῆς γὰρ αὐτῶν αἴνεσις καὶ εὐφροσύνη καταλήμψεται αὐτούς· ἀπέδρα ὀδύνη καὶ λύπη καὶ στεναγμός.
\par }{\PP \VS{12}Ἐγώ εἰμι, ἐγώ εἰμι ὁ παρακαλῶν σε· γνῶθι τίς οὖσα ἐφοβήθης ἀπὸ ἀνθρώπου θνητοῦ, καὶ ἀπὸ υἱοῦ ἀνθρώπου, οἳ ὡσεὶ χόρτος ἐξηράνθησαν.
\VS{13}Καὶ ἐπελάθου Θεὸν τὸν ποιήσαντά σε, τὸν ποιήσαντα τὸν οὐρανὸν καὶ θεμελιώσαντα τὴν γῆν· καὶ ἐφόβου ἀεὶ πάσας τὰς ἡμέρας τὸ πρόσωπον τοῦ θυμοῦ τοῦ θλίβοντός σε· ὃν τρόπον γὰρ ἐβουλεύσατο τοῦ ἆραί σε, καὶ νῦν ποῦ ὁ θυμὸς τοῦ θλίβοντός σε;
\VS{14}Ἐν γὰρ τῷ σώζεσθαί σε οὐ στήσεται, οὐδὲ χρονιεῖ,
\VS{15}ὅτι ἐγὼ ὁ Θεός σου, ὁ ταράσσων τὴν θάλασσαν, καὶ ἠχῶν τὰ κύματα αὐτῆς· Κύριος σαβαὼθ ὄνομά μοι.
\VS{16}Θήσω τοὺς λόγους μου εἰς τὸ στόμα σου, καὶ ὑπὸ τὴν σκιὰν τῆς χειρός μου σκεπάσω σε, ἐν ᾗ ἔστησα τὸν οὐρανὸν, καὶ ἐθεμελίωσα τὴν γῆν· καὶ ἐρεῖ Σιὼν, λαός μου εἶ σύ.
\par }{\PP \VS{17}Ἐξεγείρου ἐξεγείρου, ἀνάστηθι Ἱερουσαλὴμ, ἡ πιοῦσα ἐκ χειρὸς Κυρίου τὸ ποτήριον τοῦ θυμοῦ αὐτοῦ· τὸ ποτήριον γὰρ τῆς πτώσεως, τὸ κόνδυ τοῦ θυμοῦ ἐξέπιες καὶ ἐξεκένωσας,
\VS{18}καὶ οὐκ ἦν ὁ παρακαλῶν σε ἀπὸ πάντων τῶν τέκνων σου ὧν ἔτεκες, καὶ οὐκ ἦν ὁ ἀντιλαμβανόμενος τῆς χειρός σου, οὐδὲ ἀπὸ πάντων τῶν υἱῶν σου ὧν ὕψωσας.
\VS{19}Διὸ ταῦτα ἀντικείμενά σοι, τίς συλλυπηθήσεταί σοι; πτῶμα καὶ σύντριμμα, λιμὸς καὶ μάχαιρα, τίς παρακαλέσει σε;
\VS{20}Οἱ υἱοί σου οἱ ἀπορούμενοι, οἱ καθεύδοντες ἐπʼ ἄκρου πάσης ἐξόδου ὡς σευτλίον ἡμίεφθον, οἱ πλήρεις θυμοῦ Κυρίου, ἐκλελυμένοι διὰ Κυρίου τοῦ Θεοῦ.
\par }{\PP \VS{21}Διατοῦτο ἄκουε τεταπεινωμένη, καὶ μεθύουσα οὐκ ἀπὸ οἴνου.
\VS{22}Οὕτως λέγει Κύριος ὁ Θεὸς ὁ κρίνων τὸν λαὸν αὐτοῦ, ἰδοὺ εἴληφα ἐκ τῆς χειρός σου τὸ ποτήριον τῆς πτώσεως, τὸ κόνδυ τοῦ θυμοῦ μου, καὶ οὐ προσθήσῃ ἔτι πιεῖν αὐτό·
\VS{23}καὶ δώσω αὐτὸ εἰς τὰς χεῖρας τῶν ἀδικησάντων σε καὶ τῶν ταπεινωσάντων σε, οἳ εἶπαν τῇ ψυχῇ σου, κύψον, ἵνα παρέλθωμεν· καὶ ἔθηκας ἴσα τῇ γῇ τὰ μέσα σου ἔξω τοῖς παραπορευομένοις.

\par }\Chap{52}{\PP \VerseOne{1}Ἐξεγείρου ἐξεγείρου Σιὼν, ἔνδυσαι τὴν ἰσχύν σου Σιὼν, καὶ σὺ ἔνδυσαι τὴν δόξαν σου Ἱερουσαλὴμ πόλις ἡ ἁγία· οὐκέτι προστεθήσεται διελθεῖν διὰ σοῦ ἀπερίτμητος καὶ ἀκάθαρτος.
\VS{2}Ἐκτίναξαι τὸν χοῦν καὶ ἀνάστηθι, κάθισον Ἱερουσαλὴμ, ἔκδυσαι τὸν δεσμὸν τοῦ τραχήλου σου ἡ αἰχμάλωτος θυγάτηρ Σιών.
\par }{\PP \VS{3}Ὅτι τάδε λέγει Κύριος, δωρεὰν ἐπράθητε, καὶ οὐ μετὰ ἀργυρίου λυτρωθήσεσθε.
\VS{4}Οὕτως λέγει Κύριος, εἰς Αἴγυπτον κατέβη ὁ λαός μου τὸ πρότερον παροικῆσαι ἐκεῖ, καὶ εἰς Ἀσσυρίους βίᾳ ἤχθησαν.
\VS{5}Καὶ νῦν τί ἐστὲ ὧδε; τάδε λέγει Κύριος, ὅτι ἐλήφθη ὁ λαός μου δωρεὰν, θαυμάζετε καὶ ὀλολύζετε· τάδε λέγει Κύριος, διʼ ὑμᾶς διαπαντὸς τὸ ὄνομά μου βλασφημεῖται ἐν τοῖς ἔθνεσι.
\VS{6}Διατοῦτο γνώσεται ὁ λαός μου τὸ ὄνομά μου ἐν τῇ ἡμέρᾳ ἐκείνῃ, ὅτι ἐγώ εἰμι αὐτὸς ὁ λαλῶν, πάρειμι ὡς ὥρα ἐπὶ τῶν ὀρέων,
\VS{7}ὡς πόδες εὐαγγελιζομένου ἀκοὴν εἰρήνης, ὡς εὐαγγελιζόμενος ἀγαθὰ, ὅτι ἀκουστὴν ποιήσω τὴν σωτηρίαν σου, λέγων, Σιὼν βασιλεύσει σου ὁ Θεός.
\VS{8}Ὅτι φωνὴ τῶν φυλασσόντων σε ὑψώθη, καὶ τῇ φωνῇ ἅμα εὐφρανθήσονται· ὅτι ὀφθαλμοὶ πρὸς ὀφθαλμοὺς ὄψονται, ἡνίκα ἂν ἐλεήσῃ Κύριος τὴν Σιών.
\VS{9}Ῥηξάτω εὐφροσύνην ἅμα τὰ ἔρημα Ἱερουσαλὴμ, ὅτι ἠλέησε Κύριος αὐτὴν, καὶ ἐῤῥύσατο Ἱερουσαλήμ.
\VS{10}Καὶ ἀποκαλύψει Κύριος τὸν βραχίονα τὸν ἅγιον αὐτοῦ ἐνώπιον πάντων τῶν ἐθνῶν, καὶ ὄψονται πάντα ἄκρα τῆς γῆς τὴν σωτηρίαν τὴν παρὰ τοῦ Θεοῦ ἡμῶν.
\par }{\PP \VS{11}Ἀπόστητε, ἀπόστητε, ἐξέλθατε ἐκεῖθεν, καὶ ἀκαθάρτου μὴ ἅψησθε, ἐξέλθατε ἐκ μέσου αὐτῆς, ἀφορίσθητε οἱ φέροντες τὰ σκεύη Κυρίου·
\VS{12}Ὅτι οὐ μετὰ ταραχῆς ἐξελεύσεσθε, οὐδὲ φυγῇ πορεύσεσθε· προπορεύσεται γὰρ πρότερος ὑμῶν Κύριος, καὶ ὁ ἐπισυνάγων ὑμᾶς Θεὸς Ἰσραήλ.
\par }{\PP \VS{13}Ἰδοὺ, συνήσει ὁ παῖς μου, καὶ ὑψωθήσεται, καὶ δοξασθήσεται σφόδρα.
\VS{14}Ὃν τρόπον ἐκστήσονται ἐπὶ σὲ πολλοὶ, οὕτως ἀδοξήσει ἀπὸ τῶς ἀνθρώπων τὸ εἶδός σου, καὶ ἡ δόξα σου ἀπὸ υἱῶν ἀνθρώπων.
\VS{15}Οὕτω θαυμάσονται ἔθνη πολλὰ ἐπʼ αὐτῷ, καὶ συνέξουσι βασιλεῖς τὸ στόμα αὐτῶν· ὅτι οἷς οὐκ ἀνηγγέλη περὶ αὐτοῦ, ὄψονται, καὶ οἳ οὐκ ἀκηκόασι, συνήσουσι.

\par }\Chap{53}{\PP \VerseOne{1}Κύριε τίς ἐπίστευσε τῇ ἀκοῇ ἡμῶν; καὶ ὁ βραχίων Κυρίου τίνι ἀπεκαλύφθη;
\VS{2}Ἀνηγγείλαμεν ὡς παιδίον ἐναντίον αὐτοῦ, ὡς ῥίζα ἐν γῇ διψώσῃ· οὐκ ἔστιν εἶδος αὐτῷ, οὐδὲ δόξα· καὶ εἴδομεν αὐτὸν, καὶ οὐκ εἶχεν εἶδος οὐδὲ κάλλος·
\VS{3}Ἀλλὰ τὸ εἶδος αὐτοῦ ἄτιμον, καὶ ἐκλεῖπον παρὰ τοὺς υἱοὺς τῶν ἀνθρώπων· ἄνθρωπος ἐν πληγῇ ὢν, καὶ εἰδὼς φέρειν μαλακίαν, ὅτι ἀπέστραπται τὸ πρόσωπον αὐτοῦ, ἠτιμάσθη, καὶ οὐκ ἐλογίσθη.
\VS{4}Οὗτος τὰς ἁμαρτίας ἡμῶν φέρει, καὶ περὶ ἡμῶν ὀδυνᾶται, καὶ ἡμεῖς ἐλογισάμεθα αὐτὸν εἶναι ἐν πόνῳ, καὶ ἐν πληγῇ, καὶ ἐν κακώσει.
\VS{5}Αὐτὸς δὲ ἐτραυματίσθη διὰ τὰς ἁμαρτίας ἡμῶν, καὶ μεμαλάκισται διὰ τὰς ἀνομίας ἡμῶν· παιδεία εἰρήνης ἡμῶν ἐπʼ αὐτὸν, τῷ μώλωπι αὐτοῦ ἡμεῖς ἰάθημεν.
\VS{6}Πάντες ὡς πρόβατα ἐπλανήθημεν· ἄνθρωπος τῇ ὁδῷ αὐτοῦ ἐπλανήθη· καὶ Κύριος παρέδωκεν αὐτὸν ταῖς ἁμαρτίαις ἡμῶν.
\par }{\PP \VS{7}Καὶ αὐτὸς διὰ τὸ κεκακῶσθαι οὐκ ἀνοίγει τὸ στόμα αὐτοῦ· ὡς πρόβατον ἐπὶ σφαγὴν ἤχθη, καὶ ὡς ἀμνὸς ἐναντίον τοῦ κείροντος ἄφωνος, οὕτως οὐκ ἀνοίγει τὸ στόμα.
\VS{8}Ἐν τῇ ταπεινώσει ἡ κρίσις αὐτοῦ ᾔρθη, τὴν γενεὰν αὐτοῦ τίς διηγήσεται; ὅτι αἴρεται ἀπὸ τῆς γῆς ἡ ζωὴ αὐτοῦ, ἀπὸ τῶν ἀνομιῶν τοῦ λαοῦ μου ἤχθη εἰς θάνατον.
\VS{9}Καὶ δώσω τοὺς πονηροὺς ἀντὶ τῆς ταφῆς αὐτοῦ, καὶ τοὺς πλουσίους ἀντὶ τοῦ θανάτου αὐτοῦ· ὅτι ἀνομίαν οὐκ ἐποίησεν, οὐδὲ δόλον ἐν τῷ στόματι αὐτοῦ.
\VS{10}Καὶ Κύριος βούλεται καθαρίσαι αὐτὸν τῆς πληγῆς· ἐὰν δῶτε περὶ ἁμαρτίας, ἡ ψυχὴ ὑμῶν ὄψεται σπέρμα μακρόβιον·
\VS{11}καὶ βούλεται Κύριος ἀφελεῖν ἀπὸ τοῦ πόνου τῆς ψυχῆς αὐτοῦ, δεῖξαι αὐτῷ φῶς, καὶ πλάσαι τῇ συνέσει, δικαιῶσαι δίκαιον εὖ δουλεύοντα πολλοῖς, καὶ τὰς ἁμαρτίας αὐτῶν αὐτὸς ἀνοίσει.
\VS{12}Διατοῦτο αὐτὸς κληρονομήσει πολλοὺς, καὶ τῶν ἰσχυρῶν μεριεῖ σκῦλα· ἀνθʼ ὧν παρεδόθη εἰς θάνατον ἡ ψυχὴ αὐτοῦ, καὶ ἐν τοῖς ἀνόμοις ἐλογίσθη, καὶ αὐτὸς ἁμαρτίας πολλῶν ἀνήνεγκε, καὶ διὰ τὰς ἀνομίας αὐτῶν παρεδόθη.

\par }\Chap{54}{\PP \VerseOne{1}Εὐφράνθητι στεῖρα ἡ οὐ τίκτουσα, ῥῆξον καὶ βόησον ἡ οὐκ ὠδίνουσα, ὅτι πολλὰ τὰ τέκνα τῆς ἐρήμου, μᾶλλον ἢ τῆς ἐχούσης τὸν ἄνδρα· εἶπε γὰρ Κύριος,
\VS{2}πλάτυνον τὸν τόπον τῆς σκηνῆς σου, καὶ τῶν αὐλαιῶν σου, πῆξον, μὴ φείσῃ, μάκρυνον τὰ σχοινίσματά σου, καὶ τοὺς πασσάλους σου κατίσχυσον,
\VS{3}ἔτι εἰς τὰ δεξιὰ καὶ τὰ ἀριστερὰ ἐκπέτασον· καὶ τὸ σπέρμα σου ἔθνη κληρονομήσει, καὶ πόλεις ἠρημωμένας κατοικιεῖς.
\VS{4}Μὴ φοβοῦ, ὅτι κατῃσχύνθης, μηδὲ ἐντραπῇς, ὅτι ὠνειδίσθης, ὅτι αἰσχύνην αἰώνιον ἐπιλήσῃ, καὶ ὄνειδος τῆς χηρείας σου οὐ μὴ μνησθήσῃ ἔτι.
\VS{5}Ὅτι Κύριος ὁ ποιῶν σε, Κύριος σαβαὼθ ὄνομα αὐτῷ· καὶ ὁ ῥυσάμενός σε, αὐτὸς Θεὸς Ἰσραὴλ, πάσῃ τῇ γῇ κληθήσεται.
\VS{6}Οὐχ ὡς γυναῖκα καταλελειμμένην καὶ ὀλιγόψυχον κέκληκέ σε ὁ Κύριος, οὐδʼ ὡς γυναῖκα ἐκ νεότητος μεμισημένην, εἶπεν ὁ Θεός σου.
\par }{\PP \VS{7}Χρόνον μικρὸν κατέλιπόν σε, καὶ μετʼ ἐλέους μεγάλου ἐλεήσω σε.
\VS{8}Ἐν θυμῷ μικρῷ ἀπέστρεψα τὸ πρόσωπόν μου ἀπὸ σοῦ, καὶ ἐν ἐλέει αἰωνίῳ ἐλεήσω σε, εἶπεν ὁ ῥυσάμενός σε Κύριος·
\par }{\PP \VS{9}Ἀπὸ τοῦ ὕδατος τοῦ ἐπὶ Νῶε τοῦτό μοι ἐστί· καθότι ὤμοσα αὐτῷ ἐν τῷ χρόνῳ ἐκείνῳ, τῇ γῇ μὴ θυμωθήσεσθαι ἐπὶ σοὶ ἔτι, μηδὲ ἐν ἀπειλῇ σου τὰ ὄρη μεταστήσεσθαι,
\VS{10}οὐδʼ οἱ βουνοί σου μετακινηθήσονται· οὕτως οὐδὲ τὸ παρʼ ἐμοῦ σοι ἔλεος ἐκλείψει, οὐδὲ ἡ διαθήκη τῆς εἰρήνης σου οὐ μὴ μεταστῇ· εἶπε γὰρ ἵλεώς σοι Κύριε.
\par }{\PP \VS{11}Ταπεινὴ καὶ ἀκατάστατος οὐ παρεκλήθης· ἰδοὺ, ἐγὼ ἑτοιμάζω σοι ἄνθρακα τὸν λίθον σου, καὶ τὰ θεμέλιά σου σάπφειρον,
\VS{12}καὶ θήσω τὰς ἐπάλξεις σου ἴασπιν, καὶ τὰς πύλας σου λίθους κρυστάλλου, καὶ τὸν περίβολόν σου λίθους ἐκλεκτούς·
\VS{13}καὶ πάντας τοὺς υἱοὺς σου διδακτοὺς Θεοῦ, καὶ ἐν πολλῇ εἰρήνῃ τὰ τέκνα σου.
\VS{14}Καὶ ἐν δικαιοσύνῃ οἰκοδομηθήσῃ· ἀπέχου ἀπὸ ἀδίκου, καὶ οὐ φοβηθήσῃ, καὶ τρόμος οὐκ ἐγγιεῖ σοι.
\VS{15}Ἰδοὺ προσήλυτοι προσελεύσονταί σοι διʼ ἐμοῦ, καὶ παροικήσουσί σοι, καὶ ἐπὶ σὲ καταφεύξονται.
\par }{\PP \VS{16}Ἰδοὺ ἐγὼ ἔκτισά σε, οὐχ ὡς χαλκεὺς φυσῶν ἄνθρακας, καὶ ἐκφέρων σκεῦος εἰς ἔργον· ἐγὼ δὲ ἔκτισά σε, οὐκ εἰς ἀπώλειαν φθεῖραι.
\VS{17}Πᾶν σκεῦος σκευαστὸν ἐπὶ σὲ, οὐκ εὐοδώσω· καὶ πᾶσα φωνὴ ἀναστήσεται ἐπὶ σὲ εἰς κρίσιν, πάντας αὐτοὺς ἡττήσεις, οἱ δὲ ἔνοχοί σου ἔσονται ἐν αὐτῇ. Ἔστι κληρονομία τοῖς θεραπεύουσι Κύριον· καὶ ὑμεῖς ἔσεσθέ μοι δίκαιοι, λέγει Κύριος.

\par }\Chap{55}{\PP \VerseOne{1}Οἱ διψῶντες πορεύεσθε ἐφʼ ὕδωρ, καὶ ὅσοι μὴ ἔχετε ἀργύριον, βαδίσαντες ἀγοράσατε, καὶ φάγετε ἄνευ ἀργυρίου καὶ τιμῆς οἶνον καὶ στέαρ.
\VS{2}Ἱνατί τιμᾶσθε ἀργυρίου, καὶ τὸν μόχθον ὑμῶν οὐκ εἰς πλησμονήν; ἀκούσατέ μου, καὶ φάγεσθε ἀγαθὰ, καὶ ἐντρυφήσει ἐν ἀγαθοῖς ἡ ψυχὴ ὑμῶν.
\par }{\PP \VS{3}Προσέχετε τοῖς ὠσὶν ὑμῶν, καὶ ἐπακολουθήσατε ταῖς ὁδοῖς μου· εἰσακούσατέ μου, καὶ ζήσεται ἐν ἀγαθοῖς ἡ ψυχὴ ὑμῶν, καὶ διαθήσομαι ὑμῖν διαθήκην αἰώνιον, τὰ ὅσια Δαυὶδ τὰ πιστά.
\VS{4}Ἰδοὺ, μαρτύριον ἐν ἔθνεσιν ἔδωκα αὐτὸν, ἄρχοντα καὶ προστάσσοντα ἔθνεσιν.
\VS{5}Ἔθνη ἃ οὐκ οἴδασί σε, ἐπικαλέσονταί σε, καὶ λαοὶ οἳ οὐκ ἐπίστανταί σε, ἐπὶ σὲ καταφεύξονται, ἕνεκεν Κυρίου τοῦ Θεοῦ σου τοῦ ἁγίου Ἰσραὴλ, ὅτι ἐδόξασέν σε.
\par }{\PP \VS{6}Ζητήσατε τὸν Κύριον, καὶ ἐν τῷ εὑρίσκειν αὐτὸν, ἐπικαλέσασθε· ἡνίκα δʼ ἂν ἐγγίζῃ ὑμῖν,
\VS{7}ἀπολιπέτω ὁ ἀσεβὴς τὰς ὁδοὺς αὐτοῦ, καὶ ἀνὴρ ἄνομος τὰς βουλὰς αὐτοῦ, καὶ ἐπιστραφήτω ἐπὶ Κύριον, καὶ ἐλεηθήσεται, ὅτι ἐπὶ πολὺ ἀφήσει τὰς ἁμαρτίας ὑμῶν.
\VS{8}Οὐ γάρ εἰσιν αἱ βουλαί μου ὥσπερ αἱ βουλαὶ ὑμῶν, οὐδʼ ὥσπερ αἱ ὁδοὶ ὑμῶν αἱ ὁδοί μου, λέγει Κύριος.
\VS{9}Ἀλλʼ ὡς ἀπέχει ὁ οὐρανὸς ἀπὸ τῆς γῆς, οὕτως ἀπέχει ἡ ὁδός μου ἀπὸ τῶν ὁδῶν ὑμῶν, καὶ τὰ διανοήματα ὑμῶν ἀπὸ τῆς διανοίας μου.
\VS{10}Ὡς γὰρ ἂν καταβῇ ὁ ὑετὸς ἢ χιὼν ἐκ τοῦ οὐρανοῦ, καὶ οὐ μὴ ἀποστραφῇ ἕως ἂν μεθύσῃ τὴν γῆν, καὶ ἐκτέκῃ, καὶ ἐκβλαστήσῃ, καὶ δῷ σπέρμα τῷ σπείροντι, καὶ ἄρτον εἰς βρῶσιν·
\VS{11}οὕτως ἔσται τὸ ῥῆμά μου, ὃ ἐὰν ἐξέλθῃ ἐκ τοῦ στόματός μου, οὐ μὴ ἀποστραφῇ, ἕως ἂν τελεσθῇ ὅσα ἂν ἠθέλησα, καὶ εὐοδώσω τὰς ὁδούς σου, καὶ τὰ ἐντάλματά μου.
\VS{12}Ἐν γὰρ εὐφροσύνῃ ἐξελεύσεσθε, καὶ ἐν χαρᾷ διδαχθήσεσθε· τὰ γὰρ ὄρη καὶ οἱ βουνοὶ ἐξαλοῦνται προσδεχόμενοι ὑμᾶς ἐν χαρᾷ, καὶ πάντα τὰ ξύλα τοῦ ἀγροῦ ἐπικροτήσει τοῖς κλάδοις.
\VS{13}Καὶ ἀντὶ τῆς στοιβῆς ἀναβήσεται κυπάρισσος, ἀντὶ δὲ τῆς κονύζης ἀναβήσεται μυρσίνη· καὶ ἔσται Κύριος εἰς ὄνομα, καὶ εἰς σημεῖον αἰώνιον, καὶ οὐκ ἐκλείψει.

\par }\Chap{56}{\PP \VerseOne{1}Τάδε λέγει Κύριος, φυλάσσεσθε κρίσιν, καὶ ποιήσατε δικαιοσύνην· ἤγγικε γὰρ τὸ σωτήριόν μου παραγίνεσθαι, καὶ τὸ ἔλεός μου ἀποκαλυφθῆναι.
\VS{2}Μακάριος ἀνὴρ ὁ ποιῶν ταῦτα, καὶ ἄνθρωπος ὁ ἀντεχόμενος αὐτῶν, καὶ φυλάσσων τὰ σάββατα μὴ βεβηλοῦν, καὶ διατηρῶν τὰς χεῖρας αὐτοῦ μὴ ποιεῖν ἄδικα.
\par }{\PP \VS{3}Μὴ λεγέτω ὁ ἀλλογενὴς ὁ προσκείμενος πρὸς Κύριον, ἀφοριεῖ με ἄρα Κύριος ἀπὸ τοῦ λαοῦ αὐτοῦ· καὶ μὴ λεγέτω ὁ εὐνοῦχος, ὅτι ξύλον ἐγώ εἰμι ξηρόν.
\VS{4}Τάδε λέγει Κύριος τοῖς εὐνούχοις, ὅσοι ἄν φυλάξωνται τὰ σάββατά μου, καὶ ἐκλέξωνται ἃ ἐγὼ θέλω, καὶ ἀντέχωνται τῆς διαθήκης μου,
\VS{5}δώσω αὐτοῖς ἐν τῷ οἴκῳ μου καὶ ἐν τῷ τείχει μου τόπον ὀνομαστὸν, κρείττω υἱῶν καὶ θυγατέρων· ὄνομα αἰώνιον δώσω αὐτοῖς, καὶ οὐκ ἐκλείψει·
\VS{6}καὶ τοῖς ἀλλογενέσι τοῖς προσκειμένοις Κυρίῳ δουλεύειν αὐτῷ, καὶ ἀγαπᾷν τὸ ὄνομα Κυρίου, τοῦ εἶναι αὐτῷ εἰς δούλους καὶ δούλας· καὶ πάντας τοὺς φυλασσομένους τὰ σάββατά μου μὴ βεβηλοῦν, καὶ ἀντεχομένους τῆς διαθήκης μου,
\VS{7}εἰσάξω αὐτοὺς εἰς τὸ ὄρος τὸ ἅγιόν μου, καὶ εὐφρανῶ αὐτοὺς ἐν τῷ οἴκῳ τῆς προσευχῆς μου· τὰ ὁλοκαυτώματα αὐτῶν, καὶ αἱ θυσίαι αὐτῶν ἔσονται δεκταὶ ἐπὶ τὸ θυσιαστήριόν μου· ὁ γάρ οἶκός μου, οἶκος προσευχῆς κληθήσεται πᾶσι τοῖς ἔθνεσιν,
\VS{8}εἶπε Κύριος ὁ συνάγων τοὺς διεσπαρμένους Ἰσραὴλ, ὅτι συνάξω ἐπʼ αὐτὸν συναγωγήν.
\par }{\PP \VS{9}Πάντα τὰ θηρία τὰ ἄγρια, δεῦτε, φάγετε, πάντα τὰ θηρία τοῦ δρυμοῦ.
\VS{10}Ἴδετε, ὅτι ἐκτετύφλωνται πάντες, οὐκ ἔγνωσαν, κύνες ἐνεοί οὐ δυνήσονται ὑλακτεῖν, ἐνυπνιαζόμενοι κοίτην, φιλοῦντες νυστάξαι.
\VS{11}Καὶ οἱ κύνες ἀναιδεῖς τῇ ψυχῇ, οὐκ εἰδότες πλησμονήν· καί εἰσι πονηροὶ οὐκ εἰδότες σύνεσιν, πάντες ταῖς ὁδοῖς αὐτων ἐξηκολούθησαν, ἕκαστος κατὰ τὸ ἑαυτοῦ.

\par }\Chap{57}{\PP \VerseOne{1}Ἴδετε ὡς ὁ δίκαιος ἀπώλετο, καὶ οὐδεὶς ἐκδέχεται τῇ καρδίᾳ, καὶ ἄνδρες δίκαιοι αἴρονται, καὶ οὐδεὶς κατανοεῖ· ἀπὸ γὰρ προσώπου ἀδικίας ᾖρται ὁ δίκαιος.
\VS{2}Ἔσται ἐν εἰρήνῃ ἡ ταφὴ αὐτοῦ, ᾖρται ἐκ τοῦ μέσου.
\par }{\PP \VS{3}Ὑμεῖς δὲ προσαγάγετε ὧδε υἱοὶ ἄνομοι, σπέρμα μοιχῶν καὶ πόρνης.
\VS{4}Ἐν τίνι ἐνετρυφήσατε; καὶ ἐπὶ τίνα ἠνοίξατε τὸ στόμα ὑμῶν; καὶ ἐπὶ τίνα ἐχαλάσατε τὴν γλῶσσαν ὑμῶν; οὐχ ὑμεῖς ἐστέ τέκνα ἀπωλείας; σπέρμα ἄνομον;
\VS{5}Οἱ παρακαλοῦντες εἴδωλα ὑπὸ δένδρα δασέα, σφάζοντες τὰ τέκνα αὐτῶν ἐν ταῖς φάραγξιν ἀναμέσον τῶν πετρῶν;
\VS{6}Ἐκείνη σου ἡ μερίς, οὗτός σου ὁ κλῆρος, κᾀκείνοις ἐξέχεας σπονδὰς, καὶ τούτοις ἀνήνεγκας θυσίας· ἐπὶ τούτοις οὖν οὐκ ὀργισθήσομαι;
\par }{\PP \VS{7}Ἐπʼ ὄρος ὑψηλὸν καὶ μετέωρον, ἐκεῖ σου ἡ κοίτη, καὶ ἐκεῖ ἀνεβίβασας θυσίας σου,
\VS{8}καὶ ὀπίσω τῶν σταθμῶν τῆς θύρας σου ἔθηκας μνημόσυνά σου· ᾤου, ὅτι ἐὰν ἀπʼ ἐμοῦ ἀποστῇς, πλεῖόν τι ἕξεις; ἠγάπησας τοὺς κοιμωμένους μετὰ σοῦ,
\VS{9}καὶ ἐπλήθυνας τὴν πορνείαν σου μετʼ αὐτῶν, καὶ πολλοὺς ἐποίησας τοὺς μακρὰν ἀπὸ σοῦ, καὶ ἀπέστειλας πρέσβεις ὑπὲρ τὰ ὅριά σου, καὶ ἐταπεινώθης ἕως ᾅδου.
\VS{10}Ταῖς πολυοδίαις σου ἐκοπίασας, καὶ οὐκ εἶπας παύσομαι ἐνισχύουσα· ὅτι ἔπραξας ταῦτα, διατοῦτο οὐ κατεδεήθης μου σύ.
\par }{\PP \VS{11}Τίνα εὐλαβηθεῖσα ἐφοβήθης, καὶ ἐψεύσω με, καὶ οὐκ ἐμνήσθης, οὐδὲ ἔλαβές με εἰς τὴν διάνοιαν, οὐδὲ εἰς τὴν καρδίαν σου; καὶ ἐγώ σε ἰδὼν παρορῶ, καὶ ἐμὲ οὐκ ἐφοβήθης.
\par }{\PP \VS{12}Καὶ ἐγὼ ἀπαγγελῶ τὴν δικαιοσύνην σου, καὶ τὰ κακά σου, ἃ οὐκ ὠφελήσει σε,
\VS{13}ὅταν ἀναβοήσῃς ἐξελέσθωσάν σε ἐν τή θλίψει σου· τούτους γὰρ πάντας ἄνεμος λήψεται, καὶ ἀποίσει καταιγίς· οἱ δὲ ἀντεχόμενοί μου κτήσονται γῆν, καὶ κληρονομήσουσι τὸ ὄρος τὸ ἅγιόν μου·
\VS{14}Καὶ ἐροῦσι, καθαρίσατε ἀπὸ προσώπου αὐτοῦ ὁδοὺς, καὶ ἄρατε σκῶλα ἀπὸ τῆς ὁδοῦ τοῦ λαοῦ μου.
\par }{\PP \VS{15}Τάδε λέγει ὁ ὕψιστος, ἐν ὑψηλοῖς κατοικῶν τὸν αἰῶνα, ἅγιος ἐν ἁγίοις, ὄνομα αὐτῷ, ὕψιστος ἐν ἁγίοις ἀναπαυόμενος, καὶ ὀλιγοψύχοις διδοὺς μακροθυμίαν, καὶ διδοὺς ζωὴν τοῖς συντετριμμένοις τὴν καρδίαν.
\VS{16}Οὐκ εἰς τὸν αἰῶνα ἐκδικήσω ὑμᾶς, οὐδὲ διαπαντὸς ὀργισθήσομαι ὑμῖν· πνεῦμα γὰρ παρʼ ἐμοῦ ἐξελεύσεται, καὶ πνοὴν πᾶσαν ἐγὼ ἐποίησα.
\VS{17}Διʼ ἁμαρτίαν βραχύ τι ἐλύπησα αὐτὸν, καὶ ἐπάταξα αὐτὸν, καὶ ἀπέστρεψα τὸ πρόσωπόν μου ἀπʼ αὐτοῦ, καὶ ἐλυπήθη, καὶ ἐπορεύθη στυγνὸς ἐν ταῖς ὁδοῖς αὐτοῦ.
\VS{18}Τὰς ὁδοὺς αὐτοῦ ἑώρακα, καὶ ἰασάμην αὐτὸν, καὶ παρεκάλεσα αὐτὸν, καὶ ἔδωκα αὐτῷ παράκλησιν ἀληθινὴν,
\VS{19}εἰρήνην ἐπʼ εἰρήνῃ τοῖς μακρὰν καὶ τοῖς ἐγγὺς οὖσι· καὶ εἶπε Κύριος, ἰάσομαι αὐτοὺς.
\par }{\PP \VS{20}Οἱ δὲ ἄδικοι κλυδωνισθήσονται, καὶ ἀναπαύσασθαι οὐ δυνήσονται.
\VS{21}Οὐκ ἔστι χαίρειν τοῖς ἀσεβέσιν, εἶπεν ὁ Θεός.

\par }\Chap{58}{\PP \VerseOne{1}Ἀναβοήσον ἐν ἰσχύϊ, καὶ μὴ φείσῃ, ὡς σάλπιγγι ὕψωσον τὴν φωνήν σου καὶ ἀνάγγειλον τῷ λαῷ μου τὰ ἁμαρτήματα αὐτῶν, καὶ τῷ οἴκῳ Ἰακὼβ τὰς ἀνομίας αὐτῶν.
\VS{2}Ἐμὲ ἡμέραν ἐξ ἡμέρας ζητοῦσι, καὶ γνῶναί μου τὰς ὁδοὺς ἐπιθυμοῦσιν, ὡς λαὸς δικαιοσύνην πεποιηκὼς καὶ κρίσιν Θεοῦ αὐτοῦ μὴ ἐγκαταλελοιπώς· αἰτοῦσί με νῦν κρίσιν δικαίαν, καὶ ἐγγίζειν Θεῷ ἐπιθυμοῦσι,
\VS{3}λέγοντες, τί ὅτι ἐνηστεύσαμεν, καὶ οὐκ εἶδες; ἐταπεινώσαμεν τὰς ψυχὰς ἡμῶν, καὶ οὐκ ἔγνως;
\par }{\PP Ἐν γὰρ ταῖς ἡμέραις τῶν νηστειῶν ὑμῶν εὑρίσκετε τὰ θελήματα ὑμῶν, καὶ πάντας τοὺς ὑποχειρίους ὑμῶν ὑπονύσσετε.
\VS{4}Εἰ εἰς κρίσεις καὶ μάχας νηστεύετε, καὶ τύπτετε πυγμαῖς ταπεινὸν, ἱνατί μοι νηστεύετε ὡς σήμερον, ἀκουσθῆναι ἐν κραυγῇ τὴν φωνὴν ὑμῶν;
\VS{5}Οὐ ταύτην τὴν νηστείαν ἐξελεξάμην, καὶ ἡμέραν ταπεινοῦν ἄνθρωπον τὴν ψυχὴν αὐτοῦ· οὐδʼ ἂν κάμψῃς ὡς κρίκον τὸν τράχηλόν σου, καὶ σάκκον καὶ σποδὸν ὑποστρώσῃ, οὐδʼ οὕτω καλέσετε νηστείαν δεκτήν.
\VS{6}Οὐχὶ τοιαύτην νηστείαν ἐξελεξάμην, λέγει Κύριος· ἀλλὰ λύε πάντα σύνδεσμον ἀδικίας, διάλυε στραγγαλιὰς βιαίων συναλλαγμάτων, ἀπόστελλε τεθραυσμένους ἐν ἀφέσει, καὶ πᾶσαν συγγραφὴν ἄδικον διάσπα.
\VS{7}Διάθρυπτε πεινῶντι τὸν ἄρτον σου, καὶ πτωχοὺς ἀστέγους εἴσαγε εἰς τὸν οἶκόν σου· ἐὰν ἴδῃς γυμνὸν, περίβαλε, καὶ ἀπὸ τῶν οἰκείων τοῦ σπέρματός σου οὐχ ὑπερόψει.
\par }{\PP \VS{8}Τότε ῥαγήσεται πρώϊμον τὸ φῶς σου, καὶ τὰ ἰάματά σου ταχὺ ἀνατελεῖ· καὶ προπορεύσεται ἔμπροσθέν σου ἡ δικαιοσύνη σου, καὶ ἡ δόξα τοῦ Θεοῦ περιστελεῖ σε.
\VS{9}Τότε βοήσῃ, καὶ ὁ Θεὸς εἰσακούσεταί σου, ἔτι λαλοῦντός σου ἐρεῖ, ἰδοὺ, πάρειμι· ἐὰν ἀφέλῃς ἀπὸ σοῦ σύνδεσμον, καὶ χειροτονίαν, καὶ ῥῆμα γογγυσμοῦ,
\VS{10}καὶ δῷς πεινῶντι τὸν ἄρτον ἐκ ψυχῆς σου, καὶ ψυχὴν τεταπεινωμένην ἐμπλήσῃς, τότε ἀνατελεῖ ἐν τῷ σκότει τὸ φῶς σου, καὶ τὸ σκότος σου ὡς μεσημβρία,
\VS{11}καὶ ἔσται ὁ Θεός σου μετὰ σοῦ διαπαντός· καὶ ἐμπλησθήσῃ καθάπερ ἐπιθυμεῖ ἡ ψυχή σου, καὶ τὰ ὀστᾶ σου πιανθήσεται· καὶ ἔσται ὡς κῆπος μεθύων, καὶ ὡς πηγὴ ἣν μὴ ἐξέλιπεν ὕδωρ·
\VS{12}Καὶ οἰκοδομηθήσονταί σου αἱ ἔρημοι αἰώνιοι, καὶ ἔσται τὰ θεμέλιά σου αἰώνια γενεῶν γενεαῖς, καὶ κληθήσῃ οἰκοδόμος φραγμῶν, καὶ τὰς τρίβους σου ἀναμέσον παύσεις.
\par }{\PP \VS{13}Ἐὰν ἀποστρέψῃς τὸν πόδα σου ἀπὸ τῶν σαββάτων τοῦ μὴ ποιεῖν τὰ θελήματά σου ἐν τῇ ἡμέρᾳ τῇ ἁγίᾳ, καὶ καλέσεις τὰ σάββατα τρυφερὰ, ἅγια τῷ Θεῷ· οὐκ ἀρεῖς τὸν πόδα σου ἐπʼ ἔργῳ, οὐδὲ λαλήσεις λόγον ἐν ὀργῇ ἐκ τοῦ στόματός σου,
\VS{14}καὶ ἔσῃ πεποιθὼς ἐπὶ Κύριον, καὶ ἀναβιβάσει σε ἐπὶ τὰ ἀγαθὰ τῆς γῆς, καὶ ψωμιεῖ σε τὴν κληρονομίαν Ἰακὼβ τοῦ πατρός σου· τὸ γὰρ στόμα Κυρίου ἐλάλησε ταῦτα.

\par }\Chap{59}{\PP \VerseOne{1}Μὴ οὐκ ἰσχύει ἡ χεὶρ Κυρίου τοῦ σῶσαι; ἢ ἐβάρυνε τὸ οὖς αὐτοῦ τοῦ μὴ εἰσακοῦσαι;
\VS{2}Ἀλλὰ τὰ ἁμαρτήματα ὑμῶν διϊστῶσιν ἀναμέσον ὑμῶν καὶ ἀναμέσον τοῦ Θεοῦ, καὶ διὰ τὰς ἁμαρτίας ὑμῶν ἀπέστρεψεν τὸ πρόσωπον ἀφʼ ὑμῶν τοῦ μὴ ἐλεῆσαι.
\VS{3}Αἱ γὰρ χεῖρες ὑμῶν μεμολυσμέναι αἵματι, καὶ οἱ δάκτυλοι ὑμῶν ἐν ἁμαρτίαις, τὰ δὲ χείλη ὑμῶν ἐλάλησεν ἀνομίαν, καὶ ἡ γλῶσσα ὑμῶν ἀδικίαν μελετᾷ.
\par }{\PP \VS{4}Οὐθεὶς λαλεῖ δίκαια, οὐδὲ ἐστι κρίσις ἀληθινή· πεποίθασιν ἐπὶ ματαίοις, καὶ λαλοῦσι κενὰ, ὅτι κύουσι πόνον, καὶ τίκτουσιν ἀνομίαν.
\VS{5}Ὠὰ ἀσπίδων ἔῤῥηξαν, καὶ ἱστὸν ἀράχνης ὑφαίνουσι, καὶ ὁ μέλλων τῶν ὠῶν αὐτῶν φαγεῖν, συντρίψας οὔριον, εὗρε καὶ ἐν αὐτῷ βασιλίσκου.
\VS{6}Ὁ ἱστὸς αὐτῶν οὐκ ἔσται εἰς ἱμάτιον, οὐδὲ μὴ περιβάλωνται ἀπὸ τῶν ἔργων αὐτῶν· τὰ γὰρ ἔργα αὐτῶν, ἔργα ἀνομίας.
\VS{7}Οἱ δὲ πόδες αὐτῶν ἐπὶ πονηρίαν τρέχουσι, ταχινοὶ ἐκχέαι αἷμα, καὶ οἱ διαλογισμοὶ αὐτὼν, διαλογισμοὶ ἀπὸ φόνων· σύντριμμα καὶ ταλαιπωρία ἐν ταῖς ὁδοῖς αὐτῶν,
\VS{8}καὶ ὁδὸν εἰρήνης οὐκ οἴδασι, καὶ οὐκ ἔστι κρίσις ἐν ταῖς ὁδοῖς αὐτῶν· αἱ γὰρ τρίβοι αὐτῶν διεστραμμέναι ἃς διοδεύουσι, καὶ οὐκ οἴδασιν εἰρήνην.
\par }{\PP \VS{9}Διατοῦτο ἀπέστη ἡ κρίσις ἀπʼ αὐτῶν, καὶ οὐ μὴ καταλάβῃ αὐτοὺς δικαιοσύνη· ὑπομεινάντων αὐτῶν φῶς ἐγένετο αὐτοῖς σκότος, μείναντες αὐγὴν ἐν ἀωρίᾳ περιεπάτησαν.
\VS{10}Ψηλαφήσουσιν ὡς τυφλοὶ τοῖχον, καὶ ὡς οὐχ ὑπαρχόντων ὀφθαλμῶν ψηλαφήσουσι· καὶ πεσοῦνται ἐν μεσημβρίᾳ ὡς ἐν μεσονυκτίῳ, ὡς ἀποθνήσκοντες στενάξουσιν·
\VS{11}Ὡς ἄρκος καὶ ὡς περιστερὰ ἅμα πορεύσονται· ἀνεμείναμεν κρίσιν, καὶ οὐκ ἔστι σωτηρία, μακρὰν ἀφέστηκεν ἀφʼ ἡμῶν.
\par }{\PP \VS{12}Πολλὴ γὰρ ἡμῶν ἡ ἀνομία ἐναντίον σου, καὶ αἱ ἁμαρτίαι ἡμῶν ἀντέστησαν ἡμῖν· αἱ γὰρ ἀνομίαι ἡμῶν ἐν ἡμῖν, καὶ τὰ ἀδικήματα ἡμῶν ἔγνωμεν.
\VS{13}Ἠσεβήσαμεν καὶ ἐψευσάμεθα, καὶ ἀπέστημεν ὄπισθεν τοῦ Θεοῦ ἡμῶν· ἐλαλήσαμεν ἄδικα, καὶ ἠπειθήσαμεν· ἐκύομεν, καὶ ἐμελετήσαμεν ἀπὸ καρδίας ἡμῶν λόγους ἀδίκους·
\VS{14}Καὶ ἀπεστήσαμεν ὀπίσω τὴν κρίσιν, καὶ ἡ δικαιοσύνη μακρὰν ἀφέστηκεν· ὅτι κατηναλώθη ἐν ταῖς ὁδοῖς αὐτῶν ἡ ἀλήθεια, καὶ διʼ εὐθείας οὐκ ἐδύναντο διελθεῖν.
\VS{15}Καὶ ἡ ἀλήθεια ᾖρται, καὶ μετέστησαν τὴν διάνοιαν τοῦ συνιέναι.
\par }{\PP Καὶ εἶδε Κύριος, καὶ οὐκ ἤρεσεν αὐτῷ, ὅτι οὐκ ἦν κρίσις.
\VS{16}Καὶ εἶδε, καὶ οὐκ ἦν ἀνήρ, καὶ κατενοήσε, καὶ οὐκ ἦν ὁ ἀντιληψόμενος· καὶ ἠμύνατο αὐτοὺς τῷ βραχίονι αὐτοῦ, καὶ τῇ ἐλεημοσύνῃ ἐστηρίσατο.
\VS{17}Καὶ ἐνεδύσατο δικαιοσύνην ὡς θώρακα, καὶ περιέθετο περικεφαλαίαν σωτηρίου ἐπὶ τῆς κεφαλῆς, καὶ περιεβάλετο ἱμάτιον ἐκδικήσεως, καὶ τὸ περιβόλαιον αὐτοῦ,
\VS{18}ὡς ἀνταποδώσων ἀνταπόδοσιν ὄνειδος τοῖς ὑπεναντίοις.
\VS{19}Καὶ φοβηθήσονται οἱ ἀπὸ δυσμῶν τὸ ὄνομα Κυρίου, καὶ οἱ ἀπʼ ἀνατολῶν ἡλίου τὸ ὄνομα τὸ ἔνδοξον· ἥξει γὰρ ὡς ποταμὸς βίαιος ἡ ὀργὴ παρὰ Κυρίου, ἥξει μετὰ θυμοῦ.
\par }{\PP \VS{20}Καὶ ἥξει ἕνεκεν Σιὼν ὁ ῥυόμενος, καὶ ἀποστρέψει ἀσεβείας ἀπὸ Ἰακώβ.
\VS{21}Καὶ αὕτη αὐτοῖς ἡ παρʼ ἐμοῦ διαθήκη, εἶπε Κύριος· τὸ πνεῦμα τὸ ἐμὸν, ὅ ἐστιν ἐπὶ σοί, καὶ τὰ ῥήματα, ἃ ἔδωκα εἰς τὸ στόμα σου, οὐ μὴ ἐκλίπῃ ἐκ τοῦ στόματός σου, καὶ ἐκ τοῦ στόματος τοῦ σπέρματός σου· εἶπε γὰρ Κύριος ἀπὸ τοῦ νῦν καὶ εἰς τὸν αἰῶνα.

\par }\Chap{60}{\PP \VerseOne{1}Φωτίζου φωτίζου Ἱερουσαλὴμ, ἥκει γάρ σου τὸ φῶς, καὶ ἡ δόξα Κυρίου ἐπὶ σὲ ἀνατέταλκεν.
\VS{2}Ἰδοὺ, σκότος καλύψει γῆν, καὶ γνόφος ἐπʼ ἔθνη, ἐπὶ δὲ σὲ φανήσεται Κύριος, καὶ ἡ δόξα αὐτοῦ ἐπὶ σὲ ὀφθήσεται.
\VS{3}Καὶ πορεύσονται βασιλεῖς τῷ φωτί σου, καὶ ἔθνη τῇ λαμπρότητί σου.
\par }{\PP \VS{4}Ἆρον κύκλῳ τοὺς ὀφθαλμούς σου, καὶ ἴδε συνηγμένα τὰ τέκνα σου· ἥκασι πάντες οἱ υἱοί σου μακρόθεν, καὶ αἱ θυγατέρες σου ἐπʼ ὤμων ἀρθήσονται.
\VS{5}Τότε ὄψῃ, καὶ φοβηθήσῃ, καὶ ἐκστήσῃ τῇ καρδίᾳ, ὅτι μεταβαλεῖ εἰς σὲ πλοῦτος θαλάσσης, καὶ ἐθνῶν καὶ λαῶν, καὶ ἥξουσί σοι ἀγέλαι καμήλων,
\VS{6}καὶ καλύψουσί σε κάμηλοι Μαδιὰμ καὶ Γαιφά· πάντες ἐκ Σαβὰ ἥξουσι φέροντες χρυσίον καὶ λίβανον οἴσουσιν, καὶ τὸ σωτήριον Κυρίου εὐαγγελιοῦνται.
\VS{7}Καὶ πάντα τὰ πρόβατα Κηδὰρ συναχθήσονται, καὶ κριοὶ Ναβαιὼθ ἥξουσι, καὶ ἀνενεχθήσεται δεκτὰ ἐπὶ τὸ θυσιαστήριόν μου, καὶ ὁ οἶκος τῆς προσευχῆς μου δοξασθήσεται.
\par }{\PP \VS{8}Τίνες οἵδε, ὡς νεφέλαι πέτονται, καὶ ὡσεὶ περιστεραὶ σὺν νοσσοῖς ἐπʼ ἐμέ;
\VS{9}Ἐμὲ αἱ νῆσοι ὑπέμειναν, καὶ πλοῖα Θαρσεὶς ἐν πρώτοις, ἀγαγεῖν τὰ τέκνα σου μακρόθεν, καὶ τὸν ἄργυρον καὶ τὸν χρυσὸν αὐτῶν μετʼ αὐτῶν, καὶ διὰ τὸ ὄνομα Κυρίου τὸ ἅγιον, καὶ διὰ τὸ τὸν ἅγιον τοῦ Ἰσραὴλ ἔνδοξον εἶναι.
\VS{10}Καὶ οἰκοδομήσουσιν ἀλλογενεῖς τὰ τείχη σου, καὶ οἱ βασιλεῖς αὐτῶν παραστήσονταί σοι· διὰ γὰρ ὀργήν μου ἐπάταξά σε, καὶ διὰ ἔλεον ἠγάπησά σε.
\VS{11}Καὶ ἀνοιχθήσονται αἱ πύλαι σου διαπαντὸς, ἡμέρας καὶ νυκτὸς οὐ κλεισθήσονται, εἰσαγαγεῖν πρὸς σὲ δύναμιν ἐθνῶν, καὶ βασιλεῖς αὐτῶν ἀγομένους.
\VS{12}Τὰ γὰρ ἔθνη καὶ οἱ βασιλεῖς, οἵτινες οὐ δουλεύσουσί σοι, ἀπολοῦνται, καὶ τὰ ἔθνη ἐρημίᾳ ἐρημωθήσεται.
\par }{\PP \VS{13}Καὶ ἡ δόξα τοῦ Λιβάνου πρὸς σὲ ἥξει, ἐν κυπαρίσσῳ καὶ πεύκῃ καὶ κέδρῳ ἅμα, δοξάσαι τὸν τόπον τὸν ἅγιόν μου.
\par }{\PP \VS{14}Καὶ πορεύσονται πρὸς σὲ δεδοικότες υἱοὶ ταπεινωσάντων σε, καὶ παροξυνάντων σε, καὶ κληθήσῃ Πόλις Σιὼν ἀγίου Ἰσραήλ.
\VS{15}Διὰ τὸ γεγενῆσθαί σε ἐνκαταλελειμμένην καὶ μεμισημένην, καὶ οὐκ ἦν ὁ βοηθῶν· καὶ θήσω σε ἀγαλλίαμα αἰώνιον, εὐφροσύνην γενεῶν γενεαῖς.
\par }{\PP \VS{16}Καὶ θηλάσεις γάλα ἐθνῶν, καὶ πλοῦτον βασιλέων φάγεσαι, καὶ γνώσῃ ὅτι ἐγὼ Κύριος ὁ σώζων σε, καὶ ἐξαιρούμενός σε Θεὸς Ἰσραήλ.
\VS{17}Καὶ ἀντὶ χαλκοῦ οἴσω σοι χρυσίον, ἀντὶ δὲ σιδήρου οἴσω σοι ἀργύριον, ἀντὶ δὲ ξύλων οἴσω σοι χαλκόν, ἀντὶ δὲ λίθων, σιδήρον· καὶ δώσω τοὺς ἄρχοντάς σου ἐν εἰρήνῃ, καὶ τοὺς ἐπισκόπους σου ἐν δικαιοσύνῃ·
\VS{18}Καὶ οὐκ ἀκουσθήσεται ἔτι ἀδικία ἐν τῇ γῇ σου, οὐδὲ σύντριμμα, οὐδὲ ταλαιπωρία ἐν τοῖς ὁρίοις σου, ἀλλὰ κληθήσεται Σωτήριον τὰ τείχη σου, καὶ αἱ πύλαι σου Γλύμμα.
\VS{19}Καὶ οὐκ ἔσται σοι ἔτι ὁ ἥλιος εἰς φῶς ἡμέρας, οὐδὲ ἀνατολὴ σελήνης φωτιεῖ σοι τὴν νύκτα, ἀλλʼ ἔσται σοι Κύριος φῶς αἰώνιον, καὶ ὁ Θεὸς δόξα σου.
\VS{20}Οὐ γὰρ δύσεται ὁ ἥλιος σοι, καὶ ἡ σελήνη σοι οὐκ ἐκλείψει· ἔσται γὰρ σοι Κύριός φῶς αἰώνιον, καὶ ἀναπληρωθήσονται αἱ ἡμέραι τοῦ πένθους σου.
\VS{21}Καὶ ὁ λαός σου πᾶς δίκαιος, διʼ αἰῶνος κληρονομήσουσι τὴν γῆν, φυλάσσων τὸ φύτευμα, ἔργα χειρῶν αὐτοῦ, εἰς δόξαν.
\VS{22}Ὁ ὀλιγοστὸς ἔσται εἰς χιλιάδας, καὶ ὁ ἐλάχιστος εἰς ἔθνος μέγα· ἐγὼ Κύριος κατὰ καιρὸν συνάξω αὐτούς.

\par }\Chap{61}{\PP \VerseOne{1}Πνεῦμα Κυρίου ἐπʼ ἐμὲ, οὗ εἵνεκε ἔχρισέν με, εὐαγγελίσασθαι πτωχοῖς ἀπέσταλκέ με, ἰάσασθαι τοὺς συντετριμμένους τὴν καρδίαν, κηρῦξαι αἰχμαλώτοις ἄφεσιν, καὶ τυφλοῖς ἀνάβλεψιν,
\VS{2}καλέσαι ἐνιαυτὸν Κυρίου δεκτὸν, καὶ ἡμέραν ἀνταποδόσεως, παρακαλέσαι πάντας τοὺς πενθοῦντας,
\VS{3}δοθῆναι τοῖς πενθοῦσι Σιὼν αὐτοῖς δόξαν ἀντὶ σποδοῦ, ἄλειμμα εὐφροσύνης τοῖς πενθοῦσι, κατὰ στολὴν δόξης ἀντὶ πνεύματος ἀκηδίας· καὶ κληθήσονται γενεαὶ δικαιοσύνης, φύτευμα Κυρίου εἰς δόξαν.
\par }{\PP \VS{4}Καὶ οἰκοδομήσουσιν ἐρήμους αἰωνίας, ἐξηρημωμένας πρότερον ἐξαναστήσουσι, καὶ καινιοῦσι πόλεις ἐρήμους, ἐξηρημωμένας εἰς γενεάς.
\VS{5}Καὶ ἥξουσιν ἀλλογενεῖς ποιμαίνοντες τὰ πρόβατά σου, καὶ ἀλλόφυλοι ἀροτῆρες, καὶ ἀμπελουργοί.
\VS{6}Ὑμεῖς δὲ ἱερεῖς Κυρίου κληθήσεσθε, λειτουργοὶ Θεοῦ, ἰσχὺν ἐθνῶν κατέδεσθε, καὶ ἐν τῷ πλούτῳ αὐτῶν θαυμασθήσεσθε.
\VS{7}Οὕτως ἐκ δευτέρας κληρονομήσουσι τὴν γῆν, καὶ εὐφροσύνη αἰώνιος ὑπὲρ κεφαλῆς αὐτῶν.
\VS{8}Ἐγὼ γάρ εἰμι Κύριος ὁ ἀγαπῶν δικαιοσύνην, καὶ μισῶν ἁρπάγματα ἐξ ἀδικίας· καὶ δώσω τὸν μόχθον αὐτῶν δικαίοις, καὶ διαθήκην αἰώνιον διαθήσομαι αὐτοῖς.
\VS{9}Καὶ γνωσθήσεται ἐν τοῖς ἔθνεσι τὸ σπέρμα αὐτῶν, καὶ τὰ ἔκγονα αὐτῶν ἐν μέσῳ τῶν λαῶν· πᾶς ὁ ὁρῶν αὐτοὺς ἐπιγνώσεται αὐτοὺς, ὅτι οὗτοί εἰσι σπέρμα ηὐλογημένον ὑπὸ Θεοῦ,
\VS{10}καὶ εὐφροσύνῃ εὐφρανθήσονται ἐπὶ Κύριον.
\par }{\PP Ἀγαλλιάσθω ἡ ψυχή μου ἐπὶ τῷ Κυρίῳ, ἐνέδυσε γάρ με ἱμάτιον σωτηρίου, καὶ χιτῶνα εὐφροσύνης, ὡς νυμφίῳ περιέθηκέ μοι μίτραν, καὶ ὡς νύμφην κατεκόσμησέ με κόσμῳ.
\par }{\PP \VS{11}Καὶ ὡς γῆν αὔξουσαν τὸ ἄνθος αὐτῆς, καὶ ὡς κῆπος τὰ σπέρματα αὐτοῦ, οὕτως ἀνατελεῖ Κύριος Κύριος δικαιοσύνην, καὶ ἀγαλλίαμα ἐναντίον πάντων τῶν ἐθνῶν.

\par }\Chap{62}{\PP \VerseOne{1}Διὰ Σιὼν οὐ σιωπήσομαι, καὶ διὰ Ἱερουσαλὴμ οὐκ ἀνήσω, ἕως ἂν ἐξέλθῃ ὡς φῶς ἡ δικαιοσύνη αὐτῆς, τὸ δὲ σωτήριόν μου ὡς λαμπὰς καυθήσεται.
\VS{2}Καὶ ὄψονται ἔθνη τὴν δικαιοσύνην σου, καὶ βασιλεῖς τὴν δόξαν σου, καὶ καλέσει σε τὸ ὄνομα τὸ καινὸν, ὃ ὁ Κύριος ὀνομάσει αὐτό.
\VS{3}Καὶ ἔσῃ στέφανος κάλλους ἐν χειρὶ Κυρίου, καὶ διάδημα βασιλείας ἐν χειρὶ Θεοῦ σου.
\VS{4}Καὶ οὐκέτι κληθήσῃ Καταλελειμμένη, καὶ ἡ γῆ σου οὐ κληθήσεται ἔτι Ἔρημος· σοὶ γὰρ κληθήσεται, Θέλημα ἐμὸν, καὶ τῇ γῇ σου, Οἰκουμένη, ὅτι εὐδόκησε Κύριος ἐν σοὶ, καὶ ἡ γῆ σου συνοικισθήσεται.
\par }{\PP \VS{5}Καὶ ὡς συνοικῶν νεανίσκος παρθένῳ, οὕτω κατοικήσουσιν οἱ υἱοί σου· καὶ ἔσται ὃν τρόπον εὐφρανθήσεται νυμφίος ἐπὶ νύμφῃ, οὕτως εὐφρανθήσεται Κύριος ἐπὶ σοί.
\par }{\PP \VS{6}Καὶ ἐπὶ τῶν τειχῶν σου Ἱερουσαλὴμ κατέστησα φύλακας ὅλην τὴν ἡμέραν καὶ ὅλην τὴν νύκτα, οἳ διὰ τέλους οὐ σιωπήσονται μιμνησκόμενοι Κυρίου.
\VS{7}Οὐκ ἔστι γὰρ ὑμῖν ὅμοιος· ἐὰν διορθώσῃ, καὶ ποιήσῃ Ἱερουσαλὴμ γαυρίαμα ἐπὶ τῆς γῆς.
\VS{8}Ὤμοσε Κύριος κατὰ τῆς δόξης αὐτοῦ, καὶ κατὰ τῆς ἰσχύος τοῦ βραχίονος αὐτοῦ, εἰ ἔτι δώσω τὸν σῖτόν σου, καὶ τὰ βρώματα σου τοῖς ἐχθροῖς σου, καὶ εἰ ἔτι πίονται υἱοὶ ἀλλότριοι τὸν οἶνόν σου, ἐφʼ ᾧ ἐμόχθησας.
\VS{9}Ἀλλʼ οἱ συναγαγόντες φάγονται αὐτὰ, καὶ αἰνέσουσι Κύριον, καὶ οἱ συναγαγόντες πίονται αὐτὰ ἐν ταῖς ἐπαύλεσι ταῖς ἁγίαις μου.
\par }{\PP \VS{10}Πορεύεσθε διὰ τῶν πυλῶν μοῦ, καὶ ὁδοποιήσατε τῷ λαῷ μου, καὶ τοὺς λίθους ἐκ τῆς ὁδοῦ διαῤῥίψατε, ἐξάρατε σύσσημον εἰς τὰ ἔθνη.
\VS{11}Ἰδοὺ γὰρ Κύριος ἐποίησεν ἀκουστὸν ἕως ἐσχάτου τῆς γῆς· εἴπατε τῇ θυγατρὶ Σιὼν, ἰδοὺ ὁ σωτήρ σοι παραγέγονεν ἔχων τὸν ἑαυτοῦ μισθὸν, καὶ τὸ ἔργον αὐτοῦ πρὸ προσώπου αὐτοῦ.
\VS{12}Καὶ καλέσει αὐτὸν Λαὸν ἅγιον, λελυτρωμένον ὑπὸ Κυρίου· σὺ δὲ κληθήσῃ Ἐπιζητουμένη πόλις, καὶ οὐκ ἐγκαταλελιμμένη.

\par }\Chap{63}{\PP \VerseOne{1}Τίς οὗτος ὁ παραγνόμενος ἐξ Ἐδὼμ, ἐρύθημα ἱματίων ἐκ Βοσόρ; οὕτως ὡραῖος ἐν στολῇ, βίᾳ μετὰ ἰσχύος; ἐγὼ διαλέγομαι δικαιοσύνην καὶ κρίσιν σωτηρίου.
\par }{\PP \VS{2}Διατί σου ἐρυθρὰ τὰ ἱμάτια, καὶ τὰ ἐνδύματά σου ὡς ἀπὸ πατητοῦ ληνοῦ;
\VS{3}Πλήρης καταπεπατημένης, καὶ τῶν ἐθνῶν οὐκ ἔστιν ἀνὴρ μετʼ ἐμοῦ, καὶ κατεπάτησα αὐτοὺς ἐν θυμῷ μου, καὶ κατέθλασα αὐτοὺς ὡς γῆν, καὶ κατήγαγον τὸ αἷμα αὐτῶν εἰς γῆν.
\VS{4}Ἡμέρα γὰρ ἀνταποδόσεως ἐπῆλθεν αὐτοῖς, καὶ ἐνιαυτὸς λυτρώσεως πάρεστι.
\VS{5}Καὶ ἐπέβλεψα, καὶ οὐκ ἦν βοηθός· καὶ προσενόησα, καὶ οὐθεὶς ἀντελαμβάνετο· καὶ ἐῤῥύσατο αὐτοὺς ὁ βραχίων μου, καὶ ὁ θυμός μου ἐπέστη.
\VS{6}Καὶ κατεπάτησα αὐτοὺς τῇ ὀργῇ μου, καὶ κατήγαγον τὸ αἷμα αὐτῶν εἰς γῆν.
\par }{\PP \VS{7}Τὸν ἔλεον Κυρίου ἐμνήσθην, τὰς ἀρετὰς Κυρίου ἐν πᾶσιν οἷς ἡμῖν ἀνταποδίδωσι· Κύριος κριτὴν ἀγαθὸς τῷ οἴκῳ Ἰσραὴλ, ἐπάγει ἡμῖν κατὰ τὸ ἔλεος αὐτοῦ, καὶ κατὰ τὸ πλῆθος τῆς δικαιοσύνης αὐτοῦ.
\par }{\PP \VS{8}Καὶ εἶπεν, οὐχ ὁ λαός μου; τέκνα οὐ μὴ ἀθετήσωσι· καὶ ἐγένετο αὐτοῖς εἰς σωτηρίαν
\VS{9}ἐκ πάσης θλίψεως αὐτῶν· οὐ πρέσβυς, οὐδὲ ἄγγελος, ἀλλʼ αὐτὸς ἔσωσεν αὐτοὺς, διὰ τὸ ἀγαπᾷν αὐτοὺς καὶ φείδεσθαι αὐτῶν· αὐτὸς ἐλυτρώσατο αὐτοὺς, καὶ ἀνέλαβεν αὐτοὺς, καὶ ὕψωσεν αὐτοὺς πάσας τὰς ἡμέρας τοῦ αἰῶνος·
\par }{\PP \VS{10}Αὐτοὶ δὲ ἠπείθησαν, καὶ παρώξυναν τὸ πνεῦμα τὸ ἅγιον αὐτοῦ· καὶ ἐστράφη αὐτοῖς εἰς ἔχθραν, αὐτὸς ἐπολέμησεν αὐτούς.
\VS{11}Καὶ ἐμνήσθη ἡμερῶν αἰωνίων· ποῦ ὁ ἀναβιβάσας ἐκ τῆς θαλάσσης τὸν ποιμένα τῶν προβάτων; ποῦ ἐστιν ὁ θεὶς ἐν αὐτοῖς τὸ πνεῦμα τὸ ἅγιον;
\VS{12}ὁ ἀγαγὼν τῇ δεξιᾷ Μωσῆν, ὁ βραχίων τῆς δόξης αὐτοῦ; κατίσχυσεν ὕδωρ ἀπὸ προσώπου αὐτοῦ, ποιῆσαι ἑαυτῷ ὄνομα αἰώνιον.
\VS{13}Ἤγαγεν αὐτοὺς διʼ ἀβύσσου, ὡς ἵππον διʼ ἐρήμου, καὶ οὐκ ἐκοπίασαν,
\VS{14}καὶ ὡς κτήνη διὰ πεδίου· κατέβη πνεῦμα παρὰ Κυρίου, καὶ ὡδήγησεν αὐτούς· οὕτως ἤγαγες τὸν λαόν σου ποιῆσαι σεαυτῷ ὄνομα δόξης.
\par }{\PP \VS{15}Ἐπίστρεψον ἐκ τοῦ οὐρανοῦ, καὶ ἴδε ἐκ τοῦ οἴκου τοῦ ἁγίου σου, καὶ δόξης σου· ποῦ ἐστιν ὁ ζῆλός σου καὶ ἡ ἰσχύς σου; ποῦ ἐστι τὸ πλῆθος τοῦ ἐλέους σου, καὶ οἰκτιρμῶν σου, ὅτι ἀνέσχου ἡμῶν;
\VS{16}Σὺ γὰρ εἶ πατὴρ ἡμῶν, ὅτι Ἁβραὰμ οὐκ ἔγνω ἡμᾶς, καὶ Ἰσραὴλ οὐκ ἐπέγνω ἡμᾶς, ἀλλὰ σὺ Κύριε πατὴρ ἡμῶν ῥῦσαι ἡμᾶς, ἀπʼ ἀρχῆς τὸ ὄνομά σου ἐφʼ ἡμᾶς ἐστι.
\par }{\PP \VS{17}Τί ἐπλάνησας ἡμᾶς, Κύριε, ἀπὸ τῆς ὁδοῦ σου; ἐσκλήρυνας τὰς καρδίας ἡμῶν, τοῦ μὴ φοβεῖσθαί σε; ἐπίστρεψον διὰ τοὺς δούλους σου, διὰ τὰς φυλὰς τῆς κληρονομίας σου,
\VS{18}ἵνα μικρὸν κληρονομήσωμεν τοῦ ὄρους τοῦ ἁγίου σου.
\VS{19}Ἐγενόμεθα ὡς τὸ ἀπʼ ἀρχῆς ὅτε οὐκ ἦρξας ἡμῶν, οὐδὲ ἐκλήθη τὸ ὄνομά σου ἐφʼ ἡμᾶς.
\par }{\PP Ἐὰν ἀνοίξῃς τὸν οὐρανὸν, τρόμος λήψεται ἀπὸ σοῦ ὄρη, καὶ τακήσονται,

\Chap{64}\VerseOne{1}ὡς κηρὸς ἀπὸ προσώπου πυρὸς τήκεται, καὶ κατακαύσει πῦρ τοὺς ὑπεναντίους, καὶ φανερὸν ἔσται τὸ ὄνομά σου ἐν τοῖς ὑπεναντίοις· ἀπὸ προσώπου σου ἔθνη ταραχθήσονται,
\VS{2}ὅταν ποιῇς τὰ ἔνδοξα· τρόμος λήψεται ἀπὸ σοῦ ὄρη.
\par }{\PP \VS{3}Ἀπὸ τοῦ αἰῶνος οὐκ ἠκούσαμεν, οὐδὲ οἱ ὀφθαλμοὶ ἡμῶν εἶδον Θεὸν πλήν σου, καὶ τὰ ἔργα σου, ἃ ποιήσεις τοῖς ὑπομένουσιν ἔλεον.
\VS{4}Συναντήσεται γὰρ τοῖς ποιοῦσι τὸ δίκαιον, καὶ τῶν ὁδῶν σου μνησθήσονται· ἰδοὺ σὺ ὠργίσθης, καὶ ἡμεῖς ἡμάρτομεν· διατοῦτο ἐπλανήθημεν,
\VS{5}καὶ ἐγενήθημεν ὡς ἀκάθαρτοι πάντες ἡμεῖς, ὡς ῥάκος ἀποκαθημένης πᾶσα ἡ δικαιοσύνη ἡμῶν· καὶ ἐξεῤῥύημεν ὡς φύλλα διὰ τὰς ἀνομίας ἡμῶν· οὕτως ἄνεμος οἴσει ἡμᾶς.
\VS{6}Καὶ οὐκ ἔστιν ὁ ἐπικαλούμενος τὸ ὄνομά σου, καὶ ὁ μνησθεὶς ἀντιλαβέσθαι σου· ὅτι ἀπέστρεψας τὸ πρόσωπόν σου ἀφʼ ἡμῶν, καὶ παρέδωκας ἡμᾶς διὰ τὰς ἁμαρτίας ἡμῶν.
\par }{\PP \VS{7}Καὶ νῦν Κύριε, πατὴρ ἡμῶν σὺ, ἡμεῖς δὲ πηλός, ἔργα τῶν χειρῶν σου πάντες.
\VS{8}Μὴ ὀργίζου ἡμῖν σφόδρα, καὶ μὴ ἐν καιρῷ μνησθῇς ἁμαρτιῶν ἡμῶν· καὶ νῦν ἐπίβλεψον, ὅτι λαός σου πάντες ἡμεῖς.
\VS{9}Πόλις τοῦ ἁγίου σου ἐγενήθη ἔρημος, Σιὼν ὡς ἔρημος ἐγενήθη, Ἱερουσαλὴμ εἰς κατάραν.
\VS{10}Ὁ οἶκος τὸ ἅγιον ἡμῶν, καὶ ἡ δόξα ἣν εὐλόγησαν οἱ πατέρες ἡμῶν, ἑγενήθη πυρίκαυστος, καὶ πάντα ἔνδοξα ἡμῶν συνέπεσε.
\VS{11}Καὶ ἐπὶ πᾶσι τούτοις ἀνέσχου Κύριε, καὶ ἐσιώπησας, καὶ ἐταπείνωσας ἡμᾶς σφόδρα.

\par }\Chap{65}{\PP \VerseOne{1}Ἐμφανῆς ἐγενήθην τοῖς ἐμὲ μὴ ἐπερωτῶσιν, εὑρέθην τοῖς ἐμὲ μὴ ζητοῦσιν· εἶπα, ἰδοὺ εἰμὶ τῷ ἔθνει, οἳ οὐκ ἐκάλεσάν μου τὸ ὄνομα.
\VS{2}Ἐξεπέτασα τὰς χεῖράς μου ὅλην τὴν ἡμέραν πρὸς λαὸν ἀπειθοῦντα καὶ ἀντιλέγοντα, τοῖς πορευομένοις ὁδῷ οὐ καλῇ, ἀλλʼ ὀπίσω τῶν ἁμαρτιῶν αὐτῶν.
\VS{3}Ὁ λαὸς οὗτος ὁ παροξύνων με ἐναντίον ἐμοῦ διαπαντός· αὐτοὶ θυσιάζουσιν ἐν τοῖς κήποις, καὶ θυμιῶσιν ἐπὶ ταῖς πλίνθοις τοῖς δαιμονίοις, ἃ οὐκ ἔστιν·
\VS{4}Ἐν τοῖς μνήμασι, καὶ ἐν τοῖς σπηλαίοις κοιμῶνται διὰ ἐνὺπνια, οἱ ἔσθοντες κρέας ὕειον, καὶ ζωμὸν θυσιῶν, μεμολυμμένα πάντα τὰ σκεύη αὐτῶν,
\VS{5}οἱ λέγοντες, πόῤῥω ἀπʼ ἐμοῦ, μὴ ἐγγίσῃς μοι, ὅτι καθαρός εἰμι·
\par }{\PP Οὗτος καπνὸς τοῦ θυμοῦ μου, πῦρ καίεται ἐν αὐτῷ πάσας τὰς ἡμέρας.
\VS{6}Ἰδοὺ, γέγραπται ἐνώπιόν μου, οὐ σιωπήσω ἕως ἂν ἀποδώσω εἰς τὸν κόλπον αὐτῶν
\VS{7}τὰς ἁμαρτίας αὐτῶν, καὶ τῶν πατέρων αὐτῶν, λέγει Κύριος· οἳ ἐθυμίασαν ἐπὶ τῶν ὀρέων, καὶ ἐπὶ τῶν βουνῶν ὠνείδισάν με, ἀποδώσω τὰ ἔργα αὐτῶν εἰς τὸν κόλπον αὐτῶν.
\par }{\PP \VS{8}Οὕτως λέγει Κύριος, ὃν τρόπον εὑρεθήσεται ὁ ῥὼξ ἐν τῷ βότρυϊ, καὶ ἐροῦσι, μὴ λυμήνῃ αὐτὸν, ὅτι εὐλογία ἐστὶν ἐν αὐτῷ, οὕτως ποιήσω ἕνεκεν τοῦ δουλεύοντός μοι, τούτου ἕνεκεν οὐ μὴ ἀπολέσω πάντας.
\VS{9}Καὶ ἐξάξω τὸ ἐξ Ἰακὼβ σπέρμα καὶ ἐξ Ἰούδα, καὶ κληρονομήσει τὸ ὄρος τὸ ἅγιόν μου, καὶ κληρονομήσουσιν οἱ ἐκλεκτοί μου, καὶ οἱ δοῦλοί μου, καὶ κατοικήσουσιν ἐκεῖ.
\VS{10}Καὶ ἔσονται ἐν τῷ δρυμῷ ἐπαύλεις ποιμνίων, καὶ φάραγξ Ἀχὼρ εἰς ἀνάπαυσιν βουκολίων τῷ λαῷ μου, οἳ ἐζήτησάν με.
\par }{\PP \VS{11}Ὑμεῖς δὲ οἱ ἐγκαταλιπόντες με, καὶ ἐπιλανθανόμενοι τὸ ὄρος τὸ ὄρος τὸ ἅγιόν μου, καὶ ἐτοιμάζοντες τῷ δαιμονίῳ τράπεζαν, καὶ πληροῦντες τῇ τύχῃ κέρασμα,
\VS{12}ἐγὼ παραδώσω ὑμᾶς εἰς μάχαιραν, πάντες ἐν σφαγῇ πεσεῖσθε· ὅτι ἐκάλεσα ὑμᾶς, καὶ οὐχ ὑπηκούσατε· ἐλάλησα, καὶ παρηκούσατε, καὶ ἐποιήσατε τὸ πονηρὸν ἐναντίον ἐμοῦ, καὶ ἃ οὐκ ἐβουλόμην, ἐξελέξασθε.
\VS{13}Διατοῦτο τάδε λέγει Κύριος, ἰδοὺ, οἱ δουλεύοντές μοι φάγονται, ὑμεῖς δὲ πεινάσετε· ἰδοὺ, οἱ δουλεύοντές μοι πίονται, ὑμεῖς δὲ διψήσετε· ἰδοὺ, οἱ δουλεύοντές μοι εὐφρανθήσονται, ὑμεῖς δὲ αἰσχυνθήσεσθε·
\VS{14}ἰδοὺ, οἱ δουλεύοντές μοι ἀγαλλιάσονται ἐν εὐφροσύνῃ, ὑμεῖς δὲ κεκράξεσθε διὰ τὸν πόνον τῆς καρδίας ὑμῶν, καὶ ἀπὸ συντριβῆς πνεύματος ὑμῶν ὀλολύξετε.
\VS{15}Καταλείψετε γὰρ τὸ ὄνομα ὑμῶν εἰς πλησμονὴν τοῖς ἐκλεκτοῖς μου, ὑμᾶς δὲ ἀνελεῖ Κύριος, τοῖς δὲ δουλεύουσί μοι κληθήσεται ὄνομα καινὸν,
\VS{16}ὃ εὐλογηθήσεται ἐπὶ τῆς γῆς, εὐλογήσουσι γὰρ τὸν Θεὸν τὸν ἀληθινόν· καὶ οἱ ὀμνύοντες ἐπὶ τῆς γῆς, ὀμοῦνται τὸν Θεὸν τὸν ἀληθινόν· ἐπιλήσονται γὰρ τὴν θλίψιν τὴν πρώτην, καὶ οὐκ ἀναβήσεται αὐτῶν ἐπὶ τὴν καρδίαν.
\par }{\PP \VS{17}Ἔσται γὰρ ὁ οὐρανὸς καινὸς, καὶ ἡ γῆ καινὴ, καὶ οὐ μὴ μνησθῶσι τῶν προτέρων, οὐδʼ οὐ μὴ ἐπέλθῃ αὐτῶν ἐπὶ τὴν καρδίαν,
\VS{18}ἀλλʼ εὐφροσύνην καὶ ἀγαλλίαμα εὑρήσουσιν ἐν αὐτῇ· ὅτι ἰδοὺ ἐγὼ ποιῶ ἀγαλλίαμα Ἱερουσαλὴμ, καὶ τὸν λαόν μου εὐφροσύνην.
\VS{19}Καὶ ἀγαλλιάσομαι ἐπὶ Ἱερουσαλὴμ, καὶ εὐφρανθήσομαι ἐπὶ τῷ λαῷ μου· καὶ οὐκέτι μὴ ἀκουσθῇ ἐν αὐτῇ φωνὴ κλαῦθμου, καὶ φωνὴ κραυγῆς,
\VS{20}οὐδʼ οὐ μὴ γένηται ἔτι ἐκεῖ ἄωρος καὶ πρεσβύτης, ὃς οὐκ ἐμπλήσει τὸν χρόνον αὐτοῦ· ἔσται γὰρ ὁ νέος ἐκατὸν ἐτῶν, ὁ δὲ ἀποθνήσκων ἁμαρτωλὸς ἑκατὸν ἐτῶν, καὶ ἐπικατάρατος ἔσται.
\VS{21}Καὶ οἰκοδομήσουσιν οἰκίας, καὶ αὐτοὶ ἐνοικήσουσι· καὶ καταφυτεύσουσιν ἀμπελῶνας, καὶ αὐτοὶ φάγονται τὰ γεννήματα αὐτῶν.
\VS{22}Οὐ μὴ οἰκοδομήσουσι, καὶ ἄλλοι ἐνοικήσουσι, καὶ οὐ μὴ φυτεύσουσι, καὶ ἄλλοι φάγονται· κατὰ γὰρ τὰς ἡμέρας τοῦ ξύλου τῆς ζωῆς ἔσονται αἱ ἡμέραι τοῦ λαοῦ μου· τὰ γὰρ ἔργα τῶν πόνων αὐτῶν παλαιώσουσιν.
\VS{23}Οἱ ἐκλεκτοί μου οὐ κοπιάσουσιν εἰς κενὸν, οὐδὲ τεκνοποιήσουσιν εἰς κατάραν, ὅτι σπέρμα εὐλογημένον ὑπὸ Θεοῦ ἐστι, καὶ τὰ ἔκγονα αὐτῶν μετʼ αὐτῶν.
\par }{\PP \VS{24}Καὶ ἔσται πρὶν ἢ κεκράξαι αὐτοὺς, ἐγὼ ὑπακούσομαι αὐτῶν· ἔτι λαλούντων αὐτῶν, ἐρῶ, τί ἐστι;
\VS{25}Τότε λύκοι καὶ ἄρνες βοσκηθήσονται ἅμα, καὶ λέων ὡς βοῦς φάγεται ἄχυρα, ὄφις δὲ γῆν ὡς ἄρτον· οὐκ ἀδικήσουσιν, οὐδὲ λυμανοῦνται ἐπὶ τῷ ὄρει τῷ ἁγίῳ μου, λέγει Κύριος.

\par }\Chap{66}{\PP \VerseOne{1}Οὕτως λέγει Κύριος, ὁ οὐρανός μου θρόνος, καὶ ἡ γῆ ὑποπόδιον τῶν ποδῶν μου· ποῖον οἶκον οἰκοδομήσετέ μοι; καὶ ποῖος τόπος τῆς καταπαύσεώς μου;
\VS{2}Πάντα γὰρ ταῦτα ἐποίησεν ἡ χείρ μου, καὶ ἔστιν ἐμὰ πάντα ταῦτα, λέγει Κύριος· καὶ ἐπὶ τίνα ἐπιβλέψω, ἀλλʼ ἢ ἐπὶ τὸν ταπεινὸν καὶ ἡσύχιον, καὶ τρέμοντα τοὺς λόγους μου;
\par }{\PP \VS{3}Ὁ δὲ ἄνομος ὁ θύων μοι μόσχον, ὡς ὁ ἀποκτέννων κύνα· ὁ δὲ ἀναφέρων σεμίδαλιν, ὡς αἷμα ὕειον· ὁ διδοὺς λίβανον εἰς μνημόσυνον, ὡς βλάσφημος.
\par }{\PP Καὶ αὐτοὶ ἐξελέξαντο τὰς ὁδοὺς αὐτῶν, καὶ τὰ βδελύγματα αὐτῶν ἡ ψυχὴ αὐτῶν ἠθέλησε.
\VS{4}Καὶ ἐγὼ ἐκλέξομαι τὰ ἐμπαίγματα αὐτῶν, καὶ τὰς ἁμαρτίας ἀνταποδώσω αὐτοῖς· ὅτι ἐκάλεσα αὐτοὺς, καὶ οὐχ ὑπήκουσάν μου· ἐλάλησα καὶ οὐκ ἤκουσαν, καὶ ἐποίησαν τὸ πονηρὸν ἐναντίον ἐμοῦ, καὶ ἃ οὐκ ἐβουλόμην, ἐξελέξαντο.
\par }{\PP \VS{5}Ἀκούσατε ῥήματα Κυρίου οἱ τρέμοντες τὸν λόγον αὐτοῦ· εἴπατε ἀδελφοὶ ὑμῶν τοῖς μισοῦσιν ὑμᾶς καὶ βδελυσσομένοις, ἵνα τὸ ὄνομα Κυρίου δοξασθῇ, καὶ ὀφθῇ ἐν τῇ εὐφροσύνῃ αὐτῶν, καὶ ἐκεῖνοι αἰσχυνθήσονται.
\par }{\PP \VS{6}Φωνὴ κραυγῆς ἐκ πόλεως, φωνὴ ἐκ ναοῦ, φωνὴ Κυρίου ἀνταποδιδόντος ἀνταπόδοσιν τοῖς ἀντικειμένοις.
\VS{7}Πρὶν τὴν ὠδίνουσαν τεκεῖν, πρὶν ἐλθεῖν τὸν πόνον τῶν ὠδίνων, ἐξέφυγε καὶ ἔτεκεν ἄρσεν.
\VS{8}Τίς ἤκουσε τοιοῦτο, καὶ τίς ἑὼρακεν οὕτως; εἰ ὤδινε γῆ ἐν ἡμέρᾳ μιᾷ, ἢ καὶ ἐτέχθη ἔθνος εἰς ἅπαξ, ὅτι ὤδινε καὶ ἔτεκε Σιὼν τὰ παιδία αὐτῆς;
\VS{9}Ἐγὼ δὲ ἔδωκα τὴν προσδοκίαν ταύτην, καὶ οὐκ ἐμνήσθης μου, εἶπε Κύριος· οὐκ ἰδοὺ ἐγὼ γεννῶσαν καὶ στεῖραν ἐποίησα; εἶπεν ὁ Θεός σου.
\par }{\PP \VS{10}Εὐφράνθητι Ἱερουσαλὴμ, καὶ πανηγυρίσατε ἐν αὐτῇ πάντες οἱ ἀγαπῶντες αὐτὴν, χάρητε ἅμα αὐτῇ χαρᾷ πάντες ὅσοι πενθεῖτε ἐπʼ αὐτῇ,
\VS{11}ἵνα θηλάσητε, καὶ ἐμπλησθῆτε ἀπὸ μαστοῦ παρακλήσεως αὐτῆς, ἵνα ἐκθηλάσαντες τρυφήσητε ἀπὸ εἰσόδου δόξης αὐτῆς.
\par }{\PP \VS{12}Ὅτι τάδε λέγει Κύριος, ἰδοὺ ἐγὼ ἐκκλίνω εἰς αὐτοὺς ὡς ποταμὸς εἰρήνης, καὶ ὡς χειμάῤῥους ἐπικλύζων δόξαν ἐθνῶν· τὰ παιδία αὐτῶν ἐπʼ ὤμων ἀρθήσονται, καὶ ἐπὶ γονάτων παρακληθήσονται.
\VS{13}Ὡς εἴ τινα μήτηρ παρακαλέσει, οὕτω κᾀγὼ παρακαλέσω ὑμᾶς, καὶ ἐν Ἱερουσαλὴμ παρακληθήσεσθε.
\VS{14}Καὶ ὄψεσθε, καὶ χαρήσεται ἡ καρδία ὑμῶν, καὶ τὰ ὀστᾶ ὑμῶν ὡς βοτάνη ἀνατελεῖ· καὶ γνωσθήσεται ἡ χεὶρ Κυρίου τοῖς φοβουμένοις αὐτὸν, καὶ ἀπειλήσει τοῖς ἀπειθοῦσιν.
\par }{\PP \VS{15}Ἰδοὺ γὰρ Κύριος ὡς πῦρ ἥξει, καὶ ὡς καταιγὶς τὰ ἅρματα αὐτοῦ, ἀποδοῦναι ἐν θυμῷ ἐκδίκησιν αὐτοῦ, καὶ ἀποσκορακισμὸν αὐτοῦ ἐν φλογὶ πυρός.
\VS{16}Ἐν γὰρ τῷ πυρὶ Κυρίου κριθήσεται πᾶσα ἡ γῆ, καὶ ἐν τῇ ῥομφαίᾳ αὐτοῦ πᾶσα σάρξ· πολλοὶ τραυματίαι ἔσονται ὑπὸ Κυρίου.
\par }{\PP \VS{17}Οἱ ἁγνιζόμενοι καὶ καθαριζόμενοι εἰς τοὺς κήπους, καὶ ἐν τοῖς προθύροις ἔσθοντες κρέας ὕειον, καὶ τὰ βδελύγματα, καὶ τὸν μῦν, ἐπιτοαυτὸ ἀναλωθήσονται, εἶπε Κύριος.
\VS{18}Κᾀγὼ τὰ ἔργα αὐτῶν καὶ τὸν λογισμὸν αὐτῶν· ἔρχομαι συναγαγεῖν πάντα τὰ ἔθνη καὶ τὰς γλώσσας, καὶ ἥξουσι καὶ ὄψονται τὴν δόξαν μου.
\VS{19}Καὶ καταλείψω ἐπʼ αὐτῶν σημεῖον, καὶ ἐξαποστελῶ ἐξ αὐτῶν σεσωσμένους εἰς τὰ ἔθνη, εἰς Θαρσὶς, καὶ Φοὺδ, καὶ Λοὺδ, καὶ Μοσόχ, καὶ εἰς Θοβὲλ, καὶ εἰς τὴν Ἑλλάδα, καὶ εἰς τὰς νήσους τὰς πόῤῥω, οἳ οὐκ ἀκηκόασί μου τὸ ὄνομα, οὔτε ἑωράκασί μου τὴν δόξαν· καὶ ἀναγγελοῦσι τὴν δόξαν μου ἐν τοῖς ἔθνεσι,
\VS{20}καὶ ἄξουσι τοὺς ἀδελφοὺς ὑμῶν ἐκ πάντων τῶν ἐθνῶν δῶρον Κυρίῳ, μεθʼ ἵππων καὶ ἁρματων ἐν λαμπήναις ἡμιόνων μετὰ σκιαδίων εἰς τὴν ἁγίαν πόλιν Ἱερουσαλήμ, εἶπε Κύριος, ὡς ἀνενέγκαισαν οἱ υἱοὶ Ἰσραὴλ τὰς θυσίας αὐτῶν ἐμοὶ μετὰ ψαλμῶν εἰς τὸν οἶκον Κυρίου.
\VS{21}Καὶ ἀπʼ αὐτῶν λήμψομαι ἱερεῖς καὶ Λευίτας, εἶπε Κύριος.
\par }{\PP \VS{22}Ὃν τρόπον γὰρ ὁ οὐρανὸς καινὸς καὶ ἡ γῆ καινὴ, ἃ ἐγὼ ποιῶ, μένει ἐνώπιον ἐμοῦ, λέγει Κύριος, οὕτω στήσεται τὸ σπέρμα ὑμῶν, καὶ τὸ ὄνομα ὑμῶν.
\VS{23}Καὶ ἔσται μὴν ἐκ μηνὸς, καὶ σάββατον ἐκ σαββάτου, ἥξει πᾶσα σὰρξ τοῦ προσκυνῆσαι ἐνώπιον ἐμοῦ ἐν Ἱερουσαλὴμ, εἶπεν Κύριος.
\VS{24}Καὶ ἐξελεύσονται καὶ ὄψονται τὰ κῶλα τῶν ἀνθρώπων τῶν παραβεβηκότων ἐν ἐμοί· ὁ γὰρ σκώληξ αὐτῶν οὐ τελευτήσει, καὶ τὸ πῦρ αὐτῶν οὐ σβεσθήσεται, καὶ ἔσονται εἰς ὅρασιν πάσῃ σαρκί.
\par }