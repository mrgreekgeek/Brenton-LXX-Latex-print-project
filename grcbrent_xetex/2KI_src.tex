\NormalFont\ShortTitle{ΒΑΣΙΛΕΙΩΝ Δ}
{\MT ΒΑΣΙΛΕΙΩΝ Δ

\par }\ChapOne{1}{\PP \VerseOne{1}ΚΑΙ ἠθέτησε Μωὰβ ἐν Ἰσραὴλ μετὰ τὸ ἀποθανεῖν Ἀχαάβ.
\par }{\PP \VS{2}Καὶ ἔπεσεν Ὀχοζίας διὰ τοῦ δικτυωτοῦ τοῦ ἐν τῷ ὑπερῴῳ αὐτοῦ τῷ ἐν Σαμαρείᾳ, καὶ ἠῤῥώστησε· καὶ ἀπέστειλεν ἀγγέλους, καὶ εἶπε πρὸς αὐτοὺς, δεῦτε καὶ ἐπιζητήσατε ἐν τῷ Βάαλ μυΐαν θεὸν Ἀκκαρὼν, εἰ ζήσομαι ἐκ τῆς ἀῤῥωστίας μου ταύτης· καὶ ἐπορεύθησαν ἐπερωτῆσαι διʼ αὐτοῦ.
\VS{3}Καὶ ἄγγελος Κυρίου ἐκάλεσεν Ἠλιοὺ τὸν Θεσβίτην, λέγων, ἀναστὰς δεῦρο εἰς συνάντησιν τῶν ἀγγέλων Ὀχοζίου βασιλέως Σαμαρείας, καὶ λαλήσεις πρὸς αὐτοὺς, εἰ παρὰ τὸ μὴ εἶναι Θεὸν ἐν Ἰσραὴλ, ὑμεῖς πορεύεσθε ἐπιζητῆσαι ἐν τῷ Βάαλ μυΐαν θεὸν Ἀκκαρών; καὶ οὐχ οὕτως·
\VS{4}Ὅτι τάδε λέγει Κύριος, ἡ κλίνη ἐφʼ ἧς ἀνέβης ἐκεῖ, οὐ καταβήσῃ ἀπʼ αὐτῆς, ὅτι θανάτῳ ἀποθανῇ· καὶ ἐπορεύθη Ἠλιοὺ, καὶ εἶπε πρὸς αὐτούς.
\par }{\PP \VS{5}Καὶ ἐπεστράφησαν οἱ ἄγγελοι πρὸς αὐτόν· καὶ εἶπε πρὸς αὐτοὺς, τί ὅτι ἐπεστρέψατε;
\VS{6}Καὶ εἶπαν πρὸς αὐτὸν, ἀνὴρ ἀνέβη εἰς συνάντησιν ἡμῶν, καὶ εἶπε πρὸς ἡμᾶς, δεῦτε, ἐπιστράφητε πρὸς τὸν βασιλέα τὸν ἀποστείλαντα ὑμᾶς, καὶ λαλήσατε πρὸς αὐτὸν, τάδε λέγει Κύριος, εἰ παρὰ τὸ μὴ εἶναιι Θεὸν ἐν Ἰσραὴλ, σὺ πορεύῃ ἐπιζητῆσαι ἐν τῷ Βάαλ μυΐαν θεὸν Ἀκκαρών; οὐχ οὕτως· ἡ κλίνη ἐφʼ ἧς ἀνέβης ἐκεῖ, οὐ καταβήσῃ ἀπʼ αὐτῆς, ὅτι θανάτῳ ἀποθανῇ. Καὶ ἐπιστρέψαντες ἀπήγγειλαν τῷ βασιλεῖ καθὰ ἐλάλησεν Ἠλιού·
\VS{7}καὶ ἐλάλησε πρὸς αὐτοὺς, τίς ἡ κρίσις τοῦ ἀνδρὸς τοῦ ἀναβάντος εἰς συνάντησιν ὑμῖν καὶ λαλήσαντος πρὸς ὑμᾶς τοὺς λόγους τούτους;
\VS{8}Καὶ εἶπον πρὸς αὐτὸν, ἀνὴρ δασὺς, καὶ ζώνην δερματίνην περιεζωσμένος τὴν ὀσφῦν αὐτοῦ· καὶ εἶπεν, Ἠλιοὺ ὁ Θεσβίτης οὗτός ἐστι.
\par }{\PP \VS{9}Καὶ ἀπέστειλε πρὸς αὐτὸν πεντηκόνταρχον καὶ τοὺς πεντήκοντα αὐτοῦ, καὶ ἀνέβη πρὸς αὐτόν· καὶ ἰδοὺ Ἠλιοὺ ἐκάθητο ἐπὶ τῆς κορυφῆς τοῦ ὄρους· καὶ ἐλάλησεν ὁ πεντηκόνταρχος πρὸς αὐτὸν, καὶ εἶπεν, ἄνθρωπε τοῦ Θεοῦ, ὁ βασιλεὺς ἐκάλεσέ σε, κατάβηθι.
\VS{10}Καὶ ἀπεκρίθη Ἠλιοὺ, καὶ εἶπε πρὸς τὸν πεντηκόνταρχον, καὶ εἰ ἄνθρωπος Θεοῦ ἐγὼ, καταβήσεται πῦρ ἐκ τοῦ οὐρανοῦ, καὶ καταφάγεταί σε καὶ τοὺς πεντήκοντά σου· καὶ κατέβη πῦρ ἐκ τοῦ οὐρανοῦ, καὶ κατέφαγεν αὐτὸν καὶ τοὺς πεντήκοντα αὐτοῦ.
\VS{11}Καὶ προσέθετο ὁ βασιλεὺς, καὶ ἀπέστειλε πρὸς αὐτὸν ἄλλον πεντηκόνταρχον καὶ τοὺς πεντήκοντα αὐτοῦ· καὶ ἐλάλησεν ὁ πεντηκόνταρχος πρὸς αὐτὸν, καὶ εἶπεν, ἄνθρωπε τοῦ Θεοῦ, τάδε λέγει ὁ βασιλεὺς, ταχέως κατάβηθι.
\VS{12}Καὶ ἀπεκρίθη Ἠλιοὺ καὶ ἐλάλησε πρὸς αὐτὸν, καὶ εἶπεν, εἰ ἄνθρωπος Θεοῦ ἐγὼ, καταβήσεται πῦρ ἐκ τοῦ οὐρανοῦ, καὶ καταφάγεταί σε καὶ τοὺς πεντήκοντά σου· καὶ κατέβη πῦρ ἐκ τοῦ οὐρανοῦ, καὶ κατέφαγεν αὐτὸν καὶ τοὺς πεντήκοντα αὐτοῦ.
\VS{13}Καὶ προσέθετο ὁ βασιλεὺς ἔτι ἀποστεῖλαι ἡγούμενον καὶ τοὺς πεντήκοντα αὐτοῦ· καὶ ἦλθεν ὁ πεντηκόνταρχος ὁ τρίτος, καὶ ἔκαμψεν ἐπὶ τὰ γόνατα αὐτοῦ κατέναντι Ἠλιοὺ, καὶ ἐδεήθη αὐτοῦ, καὶ ἐλάλησε πρὸς αὐτὸν, καὶ εἶπεν, ἄνθρωπε τοῦ Θεοῦ, ἐντιμωθήτω ἡ ψυχή μου, καὶ ἡ ψυχὴ τῶν δούλων σου τούτων τῶν πεντήκοντα ἐν ὀφθαλμοῖς σου.
\VS{14}Ἰδοὺ κατέβη πῦρ ἐκ τοῦ οὐρανοῦ, καὶ κατέφαγε τοὺς δύο πεντηκοντάρχους τοὺς πρώτους· καὶ νῦν ἐντιμωθήτω δὴ ἡ ψυχή μου ἐν ὀφθαλμοῖς σου.
\VS{15}Καὶ ἐλάλησεν ἄγγελος Κυρίου πρὸς Ἠλιοὺ, καὶ εἶπε, κατάβηθι μετʼ αὐτοῦ, μὴ φοβηθῇς ἀπὸ προσώπου αὐτῶν· καὶ ἀνέστη Ἠλιοὺ καὶ κατέβη μετʼ αὐτοῦ πρὸς τὸν βασιλέα.
\VS{16}Καὶ ἐλάλησε πρὸς αὐτὸν, καὶ εἶπεν Ἠλιού, τάδε λέγει Κύριος, τί ὅτι ἀπέστειλας ἀγγέλους ἐκζητῆσαι ἐν τῷ Βάαλ μυΐαν θεὸν Ἀκκαρών; οὐχ οὕτως· ἡ κλίνη ἐφʼ ἧς ἀνέβης ἐκεῖ, οὐ καταβήσῃ ἀπʼ αὐτῆς, ὅτι θανάτῳ ἀποθανῇ.
\par }{\PP \VS{17}Καὶ ἀπέθανε κατὰ τὸ ῥῆμα Κυρίου ὃ ἐλάλησεν Ἠλιού.
\VS{18}Καὶ τὰ λοιπὰ τῶν λόγων Ὀχοζίου ἃ ἐποίησεν, οὐκ ἰδοὺ ταῦτα γεγραμμένα ἐν βιβλίῳ λόγων τῶν ἡμερῶν τοῖς βασιλεῦσιν Ἰσραήλ;
\VS{18a}καὶ Ἰωρὰμ υἱὸς Ἀχαὰβ βασιλεύει ἐπὶ Ἰσραὴλ ἐν Σαμαρείᾳ ἔτη δεκαδύο, ἐν ἔτει ὀκτωκαιδεκάτῳ Ἰωσαφὰτ βασιλέως Ἰούδα·
\VS{18b}καὶ ἐποίησε τὸ πονηρὸν ἐνώπιον Κυρίου· πλὴν οὐχ ὡς οἱ ἀδελφοὶ αὐτοῦ, οὐδὲ ὡς ἡ μήτηρ αὐτοῦ·
\VS{18c}καὶ ἀπέστησε τὰς στήλας τοῦ Βάαλ ἃς ἐποίησεν ὁ πατὴρ αὐτοῦ, καὶ συνέτριψεν αὐτάς· πλὴν ἐν ταῖς ἁμαρτίαις οἴκου Ἰεροβοὰμ, ὃς ἐξήμαρτε τὸν Ἰσραὴλ, ἐκολλήθη, οὐκ ἀπέστη ἀπʼ αὐτῶν·
\VS{18d}καὶ ἐθυμώθη ὀργῇ Κύριος εἰς τὸν οἶκον Ἀχαάβ.

\par }\Chap{2}{\PP \VerseOne{1}Καὶ ἐγένετο ἐν τῷ ἀνάγειν Κύριον ἐν συσσεισμῷ τὸν Ἠλιοὺ ὡς εἰς τὸν οὐρανὸν, καὶ ἐπορεύθη Ἠλιοὺ καὶ Ἑλισαιὲ ἐκ Γαλγάλων.
\VS{2}Καὶ εἶπεν Ἠλιοὺ πρὸς Ἑλισαιὲ, κάθου δὴ ἐνταῦθα, ὅτι ὁ Θεὸς ἀπέσταλκέ με ἕως Βαιθήλ· καὶ εἶπεν Ἑλισαιὲ, ζῇ Κύριος καὶ ζῇ ἡ ψυχή σου, εἰ ἐγκαταλείψω σε· καὶ ἦλθον εἰς Βαιθήλ.
\VS{3}Καὶ ἦλθον οἱ υἱοὶ τῶν προφητῶν οἱ ἐν Βαιθὴλ πρὸς Ἑλισαιὲ, καὶ εἶπον πρὸς αὐτὸν, εἰ ἔγνως, ὅτι Κύριος σήμερον λαμβάνει τὸν κύριόν σου ἐπάνωθεν τῆς κεφαλῆς σου; καὶ εἶπε, κἀγὼ ἔγνωκα, σιωπᾶτε.
\VS{4}Καὶ εἶπεν Ἠλιοὺ πρὸς Ἑλισαιὲ, κάθου δὴ ἐνταῦθα, ὅτι Κύριος ἀπέσταλκέ με εἰς Ἱεριχώ· καὶ εἶπε, ζῇ Κύριος καὶ ζῇ ἡ ψυχή σου, εἰ ἐγκαταλείψω σε· καὶ ἦλθον εἰς Ἱεριχώ.
\par }{\PP \VS{5}Καὶ ἤγγισαν οἱ υἱοῖ τῶν προφητῶν οἱ ἐν Ἱεριχὼ πρὸς Ἑλισαιὲ, καὶ εἶπον πρὸς αὐτὸν, εἰ ἔγνως, ὅτι σήμερον λαμβάνει Κύριος τὸν κύρίον σου ἐπάνωθεν τῆς κεφαλῆς σου; καὶ εἶπε, καί γε ἐγὼ ἔγνων, σιωπᾶτε.
\VS{6}Καὶ εἶπεν αὐτῷ Ἠλιοὺ, κάθου δὴ ὧδε, ὅτι Κύριος ἀπέσταλκέ με ἕως εἰς τὸν Ἰορδάνην· καὶ εἶπεν Ἑλισαιὲ, ζῇ Κύριος καὶ ζῇ ἡ ψυχή σου, εἰ ἐγκαταλείψω σε· καὶ ἐπορεύθησαν ἀμφότεροι,
\VS{7}καὶ πεντήκοντα ἄνδρες υἱοὶ τῶν προφητῶν, καὶ ἔστησαν ἐξεναντίας μακρόθεν· καὶ ἀμφότεροι ἔστησαν ἐπὶ τοῦ Ἰορδάνου.
\VS{8}Καὶ ἔλαβεν Ἠλιοὺ τὴν μηλωτὴν αὐτοῦ καὶ εἴλησε καὶ ἐπάταξε τὸ ὕδωρ, καὶ διῃρέθη τὸ ὕδωρ ἔνθα καὶ ἔνθα· καὶ διέβησαν ἀμφότεροι ἐν ἐρήμῳ.
\par }{\PP \VS{9}Καὶ ἐγένετο ἐν τῷ διαβῆναι αὐτοὺς, καὶ Ἠλιοὺ εἶπε πρὸς Ἑλισαιὲ, αἴτησαι τί ποιήσω σοι πρὶν ἢ ἀναληφθῆναί με ἀπὸ σοῦ· καὶ εἶπεν Ἑλισαιὲ, γενηθήτω δὴ διπλᾶ ἐν πνεύματί σου ἐπʼ ἐμέ.
\VS{10}Καὶ εἶπεν Ἠλιοὺ, ἐσκλήρυνας τοῦ αἰτήσασθαι· ἐὰν ἴδῃς με ἀναλαμβανόμενον ἀπὸ σοῦ, καὶ ἔσται σοι οὕτως· καὶ ἐὰν μὴ, οὐ μὴ γένηται.
\par }{\PP \VS{11}Καὶ ἐγένετο αὐτῶν πορευομένων, ἐπορεύοντο καὶ ἐλάλουν· καὶ ἰδοὺ ἅρμα πυρὸς καὶ ἵπποι πυρὸς, καὶ διέστειλεν ἀναμέσον ἀμφοτέρων· καὶ ἀνελήμφθη Ἠλιοὺ ἐν συσσεισμῷ ὡς εἰς τὸν οὐρανόν.
\VS{12}Καὶ Ἑλισαιὲ ἑώρα, καὶ ἐβόα, πάτερ, πάτερ, ἅρμα Ἰσραὴλ καὶ ἱππεὺς αὐτοῦ· καὶ οὐκ εἶδεν αὐτὸν ἔτι· καὶ ἐπελάβετο τῶν ἱματίων αὐτοῦ, καὶ διέῤῥηξεν αὐτὰ εἰς δύο ῥήγματα.
\VS{13}Καὶ ὕψωσε τὴν μηλωτὴν Ἠλιοὺ, ἣ ἔπεσεν ἐπάνωθεν Ἑλισαιέ· καὶ ἐπέστρεψεν Ἑλισαιὲ, καὶ ἔστη ἐπὶ τοῦ χείλους τοῦ Ἰορδάνου,
\VS{14}καὶ ἔλαβε τὴν μηλωτὴν Ἠλιοὺ, ἣ ἔπεσεν ἐπάνωθεν αὐτοῦ, καὶ ἐπάταξε τὸ ὕδωρ, καὶ εἶπε, ποῦ ὁ Θεὸς Ἠλιοὺ ἀφφώ; καὶ ἐπάταξε τὰ ὕδατα, καὶ διεῤῥάγησαν ἔνθα καὶ ἔνθα· καὶ διέβη Ἑλισαιέ.
\par }{\PP \VS{15}Καὶ εἶδον αὐτὸν οἱ υἱοὶ τῶν προφητῶν οἱ ἐν Ἱεριχὼ ἐξεναντίας, καὶ εἶπον, ἐπαναπέπαυται τὸ πνεῦμα Ἠλιοὺ ἐπὶ Ἑλισαιέ· καὶ ἦλθον εἰς συναντὴν αὐτοῦ, καὶ προσεκύνησαν αὐτῷ ἐπὶ τὴν γῆν,
\VS{16}καὶ εἶπον πρὸς αὐτὸν, ἰδοὺ δὴ μετὰ τῶν παίδων σου πεντήκοντα ἄνδρες υἱοὶ δυνάμεως· πορευθέντες δὴ ζητησάτωσαν τὸν κύριόν σου, μή ποτε ᾖρεν αὐτὸν πνεῦμα Κυρίου, καὶ ἔῤῥιψεν αὐτὸν ἐν τῷ Ἰορδάνῃ ἢ ἐφʼ ἓν τῶν ὀρέων ἢ ἐφʼ ἕνα τῶν βουνῶν· καὶ εἶπεν Ἑλισαιὲ, οὐκ ἀποστελεῖτε.
\VS{17}Καὶ παρεβιάσαντο αὐτὸν, ἕως οὗ ᾐσχύνετο· καὶ εἶπεν, ἀποστείλατε· καὶ ἀπέστειλαν πεντήκοντα ἄνδρας, καὶ ἐζήτησαν τρεῖς ἡμέρας, καὶ οὐχ εὗρον αὐτόν.
\VS{18}Καὶ ἀνέστρεψαν πρὸς αὐτόν, καὶ αὐτὸς ἐκάθητο ἐν Ἱεριχώ· καὶ εἶπεν Ἑλισαῖε, οὐκ εἶπον πρὸς ὑμᾶς, μὴ πορευθῆτε;
\par }{\PP \VS{19}Καὶ εἶπον οἱ ἄνδρες τῆς πόλεως πρὸς Ἑλισαιὲ, ἰδοὺ ἡ κατοίκησις τῆς πόλεως ἀγαθὴ, καθὼς ὁ κύριος βλέπει, καὶ τὰ ὕδατα πονηρὰ, καὶ ἡ γῆ ἀτεκνουμένη.
\VS{20}Καὶ εἶπεν Ἑλισαιὲ λάβετέ μοι ὑδρίσκην καινὴν, καὶ θέτε ἐκεῖ ἅλα· καὶ ἔλαβον, καὶ ἤνεγκαν πρὸς αὐτόν.
\VS{21}Καὶ ἐξῆλθεν Ἐλισαιὲ εἰς τὴν διέξοδον τῶν ὑδάτων, καὶ ἔῤῥιψεν ἐκεῖ ἅλα, καὶ εἶπε, τάδε λέγει Κύριος, ἴαμαι τὰ ὕδατα ταῦτα, οὐκ ἔσται ἔτι ἐκεῖθεν θάνατος καὶ ἀτεκνουμένη.
\VS{22}Καὶ ἰάθησαν τὰ ὕδατα ἕως τῆς ἡμέρας ταύτης, κατὰ τὸ ῥῆμα Ἑλισαιὲ ὃ ἐλάλησε.
\par }{\PP \VS{23}Καὶ ἀνέβη ἐκεῖθεν εἰς Βαιθήλ· καὶ ἀναβαίνοντος αὐτοῦ ἐν τῇ ὁδῷ καὶ παιδάρια μικρὰ ἐξῆλθον ἐκ τῆς πόλεως καὶ κατέπαιζον αὐτοῦ, καὶ εἶπον αὐτῷ, ἀνάβαινε φαλακρὲ, ἀνάβαινε.
\VS{24}Καὶ ἐξένευσεν ὀπίσω αὐτῶν, καὶ εἶδεν αὐτὰ, καὶ κατηράσατο αὐτοῖς ἐν ὀνόματι Κυρίου· καὶ ἰδοὺ ἐξῆλθον δύο ἄρκοι ἐκ τοῦ δρυμοῦ, καὶ ἀνέῤῥηξαν ἀπʼ αὐτῶν τεσσαράκοντα καὶ δύο παῖδας.
\VS{25}Καὶ ἐπορεύθη ἐκεῖθεν εἰς τὸ ὄρος τὸ Καρμήλιον, κᾀκεῖθεν ἐπέστρεψεν εἰς Σαμάρειαν.

\par }\Chap{3}{\PP \VerseOne{1}Καὶ Ἰωρὰμ υἱὸς Ἀχαὰμ ἐβασίλευσεν ἐν Ἰσραὴλ ἐν ἔτει ὀκτωκαιδεκάτῳ Ἰωσαφὰτ βασιλέως Ἰούδα, καὶ ἐβασίλευσε δώδεκα ἔτη.
\VS{2}Καὶ ἐποίησε τὸ πονηρὸν ἐν ὀφθαλμοῖς Κυρίου· πλὴν οὐχ ὡς ὁ πατὴρ αὐτοῦ, καὶ οὐχ ὡς ἡ μήτηρ αὐτοῦ· καὶ μετέστησε τὰς στήλας τοῦ Βάαλ, ἃς ἐποίησεν ὁ πατὴρ αὐτοῦ.
\VS{3}Πλὴν ἐν τῇ ἁμαρτίᾳ Ἱεροβοὰμ υἱοῦ Ναβὰτ ὃς ἐξήμαρτε τὸν Ἰσραὴλ, ἐκολλήθη, οὐκ ἀπέστη ἀπʼ αὐτῆς.
\par }{\PP \VS{4}Καὶ Μωσὰ βασιλεὺς Μωὰβ ἦν νωκὴδ, καὶ ἐπέστρεφε τῷ βασιλεῖ Ἰσραὴλ ἐν τῇ ἐπαναστάσει ἑκατὸν χιλιάδας ἀρνῶν, καὶ ἑκατὸν χιλιάδας κριῶν ἐπὶ πόκων.
\VS{5}Καὶ ἐγένετο μετὰ τὸ ἀποθανεῖν Ἀχαὰβ, καὶ ἠθέτησε βασιλεὺς Μωὰβ ἐν βασιλεῖ Ἰσραήλ.
\par }{\PP \VS{6}Καὶ ἐξῆλθεν ὁ βασιλεὺς Ἰωρὰμ ἐν τῇ ἡμέρᾳ ἐκείνῃ ἐκ Σαμαρείας, καὶ ἐπεσκέψατο τὸν Ἰσραήλ.
\VS{7}Καὶ ἐπορεύθη καὶ ἐξαπέστειλε πρὸς Ἰωσαφὰτ βασιλέα Ἰούδα, λέγων, βασιλεὺς Μωὰβ ἠθέτησεν ἐν ἐμοί· εἰ πορεύσῃ μετʼ ἐμοῦ εἰς Μωὰβ εἰς πόλεμον; καὶ εἶπεν, ἀναβήσομαι· ὅμοιός μοι, ὅμοιός σοι· ὡς ὁ λαός μου, ὁ λαός σου· ὡς οἱ ἵπποι μου, οἱ ἵπποι σου.
\VS{8}καὶ εἶπε, ποίᾳ ὁδῷ ἀναβῶ; καὶ εἶπεν, ὁδὸν ἔρημον Ἐδώμ.
\VS{9}Καὶ ἐπορεύθη ὁ βασιλεὺς Ἰσραὴλ καὶ ὁ βασιλεὺς Ἰούδα καὶ ὁ βασιλεὺς Ἐδὼμ, καὶ ἐκύκλωσαν ὁδὸν ἑπτὰ ἡμερῶν· καὶ οὐκ ἦν ὕδωρ τῇ παρεμβολῇ καὶ τοῖς κτήνεσι τοῖς ἐν τοῖς ποσὶν αὐτῶν.
\par }{\PP \VS{10}Καὶ εἶπεν ὁ βασιλεὺς Ἰσραὴλ, ὢ, ὅτι κέκληκε Κύριος τοὺς τρεῖς βασιλεῖς παρερχομένους δοῦναι αὐτοὺς ἐν χειρὶ Μωάβ.
\VS{11}Καὶ εἶπεν Ἰωσαφὰτ, οὐκ ἔστιν ὧδε προφήτης τοῦ Κυρίου, καὶ ἐπιζητήσωμεν τὸν Κύριον παρʼ αὐτοῦ; καὶ ἀπεκρίθη εἷς τῶν παίδων τοῦ βασιλέως Ἰσραὴλ, καὶ εἶπεν, ὧδε Ἑλισαιὲ υἱὸς Σαφὰτ, ὃς ἐπέχεεν ὕδωρ ἐπὶ χεῖρας Ἠλιού.
\VS{12}Καὶ εἶπεν Ἰωσαφὰτ, ἔστιν αὐτῷ ῥῆμα Κυρίου· καὶ κατέβη πρὸς αὐτὸν βασιλεὺς Ἰσραὴλ, καὶ Ἰωσαφὰτ βασιλεὺς Ἰούδα, καὶ βασιλεὺς Ἐδώμ.
\par }{\PP \VS{13}Καὶ εἶπεν Ἑλισαιὲ πρὸς βασιλέα Ἰσραὴλ, τί ἐμοὶ καὶ σοί; δεῦρο πρὸς τοὺς προφήτας τοῦ πατρός σου καὶ τοὺς προφήτας τῆς μητπός σου· καὶ εἶπεν αὐτῷ ὁ βασιλεὺς Ἰσραήλ, μή ὅτι κέκληκε Κύριος τοὺς τρεῖς βασιλεῖς τοῦ παραδοῦναι αὐτοὺς εἰς χεῖρας Μωάβ;
\VS{14}Καὶ εἶπεν Ἑλισαιὲ, ζῇ Κύριος τῶν δυνάμεων ᾧ παρέστην ἐνώπιον αὐτοῦ, ὅτι εἰ μὴ πρόσωπον Ἰωσαφὰτ βασιλέως Ἰούδα ἐγὼ λαμβάνω, εἰ ἐπέβλεψα πρὸς σὲ, καὶ εἶδόν σε.
\VS{15}Καὶ νῦν λάβε μοι ψάλλοντα· καὶ ἐγένετο ὡς ἔψαλλεν ὁ ψάλλων, καὶ ἐγένετο ἐπʼ αὐτὸν χεὶρ Κυρίου,
\VS{16}καὶ εἶπε, τάδε λέγει Κύριος, ποιήσατε τὸν χειμάῤῥουν τοῦτον βοθύνους βοθύνους,
\VS{17}ὅτι τάδε λέγει Κύριος, οὐκ ὄψεσθε πνεῦμα, καὶ οὐκ ὄψεσθε ὑετὸν, καὶ ὁ χειμάῤῥους οὗτος πλησθήσεται ὕδατος, καὶ πίεσθε ὑμεῖς καὶ αἱ κτήσεις ὑμῶν καὶ τὰ κτήνη ὑμῶν.
\VS{18}Καὶ κούφη αὑτὴ ἐν ὀφθαλμοῖς Κυρίου· καὶ παραδώσω τὴν Μωὰβ ἐν χειρὶ ὑμῶν.
\VS{19}Καὶ πατάξετε πᾶσαν πόλιν ὀχυρὰν, καὶ πᾶν ξύλον ἀγαθὸν καταβαλεῖτε, καὶ πάσας πηγὰς ὕδατος ἐμφράξεσθε, καὶ πᾶσαν μερίδα ἀγαθὴν ἀχρειώσετε ἐν λίθοις.
\par }{\PP \VS{20}Καὶ ἐγένετο πρωῒ ἀναβαινούσης τῆς θυσίας, καὶ ἰδοὺ ὕδατα ἤρχοντο ἐξ ὁδοῦ Ἐδώμ, καὶ ἐπλήσθη ἡ γῆ ὕδατος.
\par }{\PP \VS{21}Καὶ πᾶσα Μωὰβ ἤκουσαν ὅτι ἀνέβησαν οἱ τρεῖς βασιλεῖς πολεμεῖν αὐτούς· καὶ ἀνεβόησαν ἐκ παντὸς περιεζωσμένοι ζώνην· καὶ εἶπον, ὤ· καὶ ἔστησαν ἐπὶ τοῦ ὁρίου.
\VS{22}Καὶ ὤρθρισαν τοπρωῒ, καὶ ὁ ἥλιος ἀνέτειλεν ἐπὶ τὰ ὕδατα· καὶ εἶδε Μωὰβ ἐξεναντίας τὰ ὕδατα πυῤῥὰ ὡς αἷμα,
\VS{23}καὶ εἶπαν, αἷμα τοῦτο τῆς ῥομφαίας· καὶ ἐμαχέσαντο οἱ βασιλεῖς, καὶ ἐπάταξεν ἀνὴρ τὸν πλησίον αὐτοῦ· καὶ νῦν ἐπὶ τὰ σκῦλα Μωάβ.
\VS{24}Καὶ εἰσῆλθον εἰς τὴν παρεμβολὴν Ἰσραήλ· καὶ Ἰσραὴλ ἀνέστησαν καὶ ἐπάταξαν τὴν Μωὰβ, καὶ ἔφυγον ἀπὸ προσώπου αὐτῶν· καὶ εἰσῆλθον εἰσπορευόμενοι καὶ τύπτοντες τὴν Μωάβ,
\VS{25}καὶ τὰς πόλεις καθεῖλον, καὶ πᾶσαν μερίδα ἀγαθὴν ἔῤῥιψαν ἀνὴρ τὸν λίθον καὶ ἐνέπλησαν αὐτὴν, καὶ πᾶσαν πηγὴν ἐνέφραξαν, καὶ πᾶν ξύλον ἀγαθὸν κατέβαλον ἕως τοῦ καταλιπεῖν τοὺς λίθους τοῦ τοίχου καθῃρημένους· καὶ ἐκύκλευσαν οἱ σφενδονῆται, καὶ ἐπάταξαν αὐτήν.
\VS{26}Καὶ εἶδεν ὁ βασιλεὺς Μωὰβ ὅτι ἐκραταίωσεν ὑπὲρ αὐτὸν ὁ πόλεμος· καὶ ἔλαβε μεθʼ ἑαυτοῦ ἑπτακοσίους ἄνδρας ἐσπασμένους ῥομφαίαν διακόψαι πρὸς βασιλέα Ἐδὼμ, καὶ οὐκ ἠδυνήθησαν.
\VS{27}Καὶ ἔλαβε τὸν υἱὸν αὐτοῦ τὸν πρωτότοκον ὃν ἐβασίλευσεν ἀντʼ αὐτοῦ, καὶ ἀνήνεγκεν αὐτὸν ὁλοκαύτωμα ἐπὶ τοῦ τείχους, καὶ ἐγένετο μετάμελος μέγας ἐπὶ Ἰσραήλ· καὶ ἀπῆραν ἀπʼ αὐτοῦ, καὶ ἐπέστρεψαν εἰς τὴν γῆν.

\par }\Chap{4}{\PP \VerseOne{1}Καὶ γυνὴ μία ἀπὸ τῶν υἱῶν τῶν προφητῶν ἐβόα πρὸς τὸν Ἑλισαῖε, λέγουσα, ὁ δοῦλός σου ἀνήρ μου ἀπέθανε, καὶ σὺ ἔγνως, ὅτι δοῦλὸς σου ἦν φοβούμενος τὸν Κύριον· καὶ ὁ δανειστὴς ἦλθε λαβεῖν τοὺς δύο υἱούς μου ἑαυτῷ εἰς δούλους.
\VS{2}Καὶ εἶπεν Ἑλισαιὲ, τί ποιήσω σοι; ἀνάγγειλόν μοι τί ἐστι σοι ἐν τῷ οἴκῳ; ἡ δὲ εἶπεν, οὐκ ἔστι τῇ δούλῃ σου οὐδὲν ἐν τῷ οἴκῳ, ὅτι ἀλλʼ ἢ ὃ ἀλείψομαι ἔλαιον.
\VS{3}Καὶ εἶπε πρὸς αὐτήν, δεῦρο, αἴτησαι σεαυτῇ σκεύη ἔξωθεν παρὰ πάντων τῶν γειτόνων σκεύη κενὰ, μὴ ὀλιγώσῃς.
\VS{4}Καὶ εἰσελεύσῃ καὶ ἀποκλείσεις τὴν θύραν κατὰ σοῦ καὶ κατὰ τῶν υἱῶν σου, καὶ ἀποχεεῖς εἰς τὰ σκεύη ταῦτα, καὶ τὸ πληρωθὲν ἀρεῖς.
\VS{5}Καὶ ἀπῆλθε παρʼ αὐτοῦ, καὶ ἀπέκλεισε τὴν θύραν καθʼ ἑαυτῆς καὶ κατὰ τῶν υἱῶν αὐτῆς· αὐτοὶ προσήγγιζον πρὸς αὐτὴν, καὶ αὐτὴ ἐπέχεεν ἕως ἐπλήσθησαν τὰ σκεύη.
\VS{6}Καὶ εἶπε πρὸς τοὺς υἱοὺς αὐτῆς, ἐγγίσατε ἔτι πρὸς μὲ τὸ σκεῦος· καὶ εἶπον αὐτῇ, οὐκ ἔστιν ἔτι σκεῦος· καὶ ἔστη τὸ ἔλαιον.
\VS{7}Καὶ ἦλθε, καὶ ἀπήγγειλε τῷ ἀνθρώπῳ τοῦ Θεοῦ· καὶ εἶπεν Ἑλισαιὲ, δεῦρο καὶ ἀπόδου τὸ ἔλαιον, καὶ ἀποτίσεις τοὺς τόκους σου, καὶ σὺ καὶ οἱ υἱοί σου ζήσεσθε ἐν τῷ ἐπιλοίπῳ ἐλαίῳ.
\par }{\PP \VS{8}Καὶ ἐγένετο ἡμέρα, καὶ διέβη Ἐλισαιὲ εἰς Σωμὰν, καὶ ἐκεῖ γυνὴ μεγάλη, καὶ ἐκράτησεν αὐτὸν φαγεῖν ἄρτον· καὶ ἐγένετο ἀφʼ ἱκανοῦ τοῦ εἰσπορεύεσθαι αὐτὸν, ἐξέκλινε τοῦ ἐκεῖ φαγεῖν.
\VS{9}Καὶ εἶπεν ἡ γυνὴ πρὸς τὸν ἄνδρα αὐτῆς, ἰδοὺ δὴ ἔγνων ὅτι ἄνθρωπος τοῦ Θεοῦ ἅγιος οὗτος διαπορεύεται ἐφ ἡμᾶς διαπαντός.
\VS{10}Ποιήσωμεν δὴ αὐτῷ ὑπερῷον τόπον μικρόν, καὶ θῶμεν αὐτῷ ἐκεῖ κλίνην, καὶ τράπεζαν, καὶ δίφρον, καὶ λυχνίαν· καὶ ἔσται ἐν τῷ εἰσπορεύεσθαι πρὸς ἡμᾶς, καὶ ἐκκλινεῖ ἐκεῖ.
\par }{\PP \VS{11}Καὶ ἐγένετο ἡμέρα, καὶ εἰσῆλθεν ἐκεῖ, καὶ ἐξέκλινεν εἰς τὸ ὑπερῷον, καὶ ἐκοιμήθη ἐκεῖ.
\VS{12}Καὶ εἶπε πρὸς Γιεζὶ τὸ παιδάριον αὐτοῦ, κάλεσόν μοι τὴν Σωμανίτιν ταύτην· καὶ ἐκάλεσεν αὐτήν, καὶ ἔστη ἐνώπιον αὐτοῦ.
\VS{13}Καὶ εἶπεν αὐτῷ, εἶπον δὴ πρὸς αὐτὴν, ἰδοὺ, ἐξέστησας ἡμῖν πᾶσαν τὴν ἔκστασιν ταύτην, τί δεῖ ποιῆσαί σοι; εἰ ἔστι λόγος σοι πρὸς τὸν βασιλέα, ἢ πρὸς τὸν ἄρχοντα τῆς δυνάμεως; ἡ δὲ εἶπεν, ἐν μέσῳ τοῦ λαοῦ ἐγώ εἰμι οἰκῶ.
\VS{14}Καὶ εἶπε πρὸς Γιεζὶ, τί δεῖ ποιῆσαι αὐτῇ; καὶ εἶπε Γιεζὶ τὸ παιδάριον αὐτοῦ, καὶ μάλα υἱὸς οὐκ ἔστιν αὐτῇ· καὶ ὁ ἀνὴρ αὐτῆς πρεσβύτης.
\par }{\PP \VS{15}Καὶ ἐκάλεσεν αὐτήν, καὶ ἔστη παρὰ τὴν θύραν.
\VS{16}Καὶ εἶπεν Ἐλισαιὲ πρὸς αὐτήν, εἰς τὸν καιρὸν τοῦτον, ὡς ἡ ὥρα, ζῶσα σὺ, περιειληφυῖα υἱόν· ἡ δὲ εἶπε, μή κύριε, μὴ διαψεύσῃ τὴν δούλην σου.
\VS{17}Καὶ ἐν γαστρὶ ἔλαβεν ἡ γυνή, καὶ ἔτεκεν υἱὸν εἰς τὸν καιρὸν τοῦτον, ὡς ἡ ὥρα, ζῶσα, ὡς ἐλάλησε πρὸς αὐτὴν Ἑλισαιὲ.
\par }{\PP \VS{18}Καὶ ἡδρύνθη τὸ παιδάριον· καὶ ἐγένετο ἡνίκα ἐξῆλθε πρὸς τὸν πατέρα αὐτοῦ πρὸς τοὺς θερίζοντας,
\VS{19}καὶ εἶπε πρὸς τὸν πατέρα αὐτοῦ, τὴν κεφαλήν μου, τὴν κεφαλήν μου· καὶ εἶπε τῷ παιδαρίῳ, ἆρον αὐτὸν πρὸς τὴν μητέρα αὐτοῦ.
\VS{20}Καὶ ᾖρεν αὐτὸν πρὸς τὴν μητέρα αὐτοῦ, καὶ ἐκοιμήθη ἐπὶ τῶν γονάτων αὐτῆς ἕως μεσημβρίας, καὶ ἀπέθανε.
\VS{21}Καὶ ἀνήνεγκεν αὐτὸν, καὶ ἐκοίμισεν αὐτὸν ἐπὶ τὴν κλίνην τοῦ ἀνθρώπου τοῦ Θεοῦ· καὶ ἀπέκλεισε κατʼ αὐτοῦ, καὶ ἐξῆλθε,
\VS{22}καὶ ἐκάλεσε τὸν ἄνδρα αὐτῆς, και εἶπεν, ἀπόστειλον δή μοι ἓν τῶν παιδαρίων καὶ μίαν τῶν ὄνων, καὶ δραμοῦμαι ἕως τοῦ ἀνθρώπου τοῦ Θεοῦ, καὶ ἐπιστρέψω.
\VS{23}Καὶ εἶπε, τι ὅτι σὺ πορεύῃ πρὸς αὐτὸν σήμερον; οὐ νεομηνία, οὐδὲ σάββατον· ἡ δὲ εἶπεν, εἰρήνη.
\par }{\PP \VS{24}Καὶ ἐπέσαξε τὴν ὄνον, καὶ εἶπε πρὸς τὸ παιδάριον αὐτῆς, ἄγε, πορεύου, μὴ ἐπίσχῃς μοι τοῦ ἐπιβῆναι ὅτι ἐὰν εἴπω σοι· δεῦρο καὶ πορεύσῃ, καὶ ἐλεύσῃ πρὸς τὸν ἄνθρωπον τοῦ θεοῦ εἰς ὄρος τὸ Καρμήλιον.
\VS{25}Καὶ ἐπορεύθη καὶ ἦλθεν ἕως τοῦ ἀνθρώπου τοῦ Θεοῦ εἰς τὸ ὄρος· καὶ ἐγένετο ὡς εἶδεν Ἑλισαιὲ ἐρχομένην αὐτὴν, καὶ εἶπε πρὸς Γιεζὶ τὸ παιδάριον αὐτοῦ, ἰδοὺ δὴ ἡ Σωμανίτις ἐκείνη.
\VS{26}Νῦν δράμε εἰς ἀπαντὴν αὐτῆς, καὶ ἐρεῖς, ἡ εἰρήνη σοι; ἡ εἰρήνη τῷ ἀνδρί σου; ἡ εἰρήνη τῷ παιδαρίῳ; ἡ δὲ εἶπεν, εἰρήνη.
\VS{27}Καὶ ἦλθε πρὸς Ἑλισαιὲ εἰς τὸ ὄρος, καὶ ἐπελάβετο τῶν ποδῶν αὐτοῦ· καὶ ἤγγιζε Γιεζὶ ἀπώσασθαι αὐτήν. καὶ εἶπεν Ἑλισαιὲ, ἄφες αὐτὴν, ὅτι ἡ ψυχὴ αὐτῆς κατώδυνος αὐτῇ, καὶ Κύριος ἀπέκρυψεν ἀπʼ ἐμοῦ καὶ σοῦ καὶ οὐκ ἀνήγγειλέ μοι.
\VS{28}Ἡ δὲ εἶπε, μὴ ᾐτησάμην υἱὸν παρὰ τοῦ κυρίου μου; ὅτι οὐκ εἶπα, οὐ πλανήσεις μετʼ ἐμοῦ;
\par }{\PP \VS{29}Καὶ εἶπεν Ἐλισαιὲ τῷ Γιεζί, ζῶσαι τὴν ὀσθύν σου, καὶ λάβε τὴν βακτηρίαν μου ἐν τῇ χειρί σου, καὶ δεῦρο, ὅτι ἐὰν εὕρῃς ἄνδρα οὐκ εὐλογήσεις αὐτὸν, καὶ ἐὰν εὐλογήσῃ σε ἀνὴρ οὐκ ἀποκριθήσῃ αὐτῷ· καὶ ἐπιθήσεις τὴν βακτηρίαν μου ἐπὶ πρόσωπον τοῦ παιδαρίου.
\VS{30}Καὶ εἶπεν ἡ μήτηρ τοῦ παιδαρίου, ζῇ Κύριος καὶ ζῇ ἡ ψυχή σου, εἰ ἐνκαταλείψω σε· καὶ ἀνέστη Ἑλισαιὲ, καὶ ἐπορεύθη ὀπίσω αὐτῆς.
\VS{31}Καὶ Γιεζὶ διῆλθεν ἔμπροσθεν αὐτῆς, καὶ ἀπέθηκε τὴν βακτηρίαν ἐπὶ πρόσωπον τοῦ παιδαρίου· καὶ οὐκ ἦν φωνὴ καὶ οὖκ ἦν ἀκρόασις· καὶ ἐπέστρεψεν εἰς ἀπαντὴν αὐτοῦ, καὶ ἀπήγγειλεν αὐτῷ, λέγων, οὐκ ἠγέρθη τὸ παιδάριον.
\par }{\PP \VS{32}Καὶ εἰσῆλθεν Ἑλισαιὲ εἰς τὸν οἶκον, καὶ ἰδοὺ τὸ παιδάριον τεθνηκός κεκοιμισμένον ἐπὶ τὴν κλίνην αὐτοῦ·
\VS{33}Καὶ εἰσῆλθεν Ἑλισαιὲ εἰς τὸν οἶκον, καὶ ἀπέκλεισε τὴν θύραν κατὰ τῶν δύο ἑαυτῶν, καὶ προσηύξατο πρὸς Κύριον.
\VS{34}Καὶ ἀνέβη καὶ ἐκοιμήθη ἐπὶ τὸ παιδάριον· καὶ ἔθηκε τὸ στόμα αὐτοῦ ἐπὶ τὸ στόμα αὐτοῦ, καὶ τοὺς ὀφθαλμοὺς αὐτοῦ ἐπὶ τοὺς ὀφθαλμοὺς αὐτοῦ, καὶ τὰς χεῖρας αὐτοῦ ἐπὶ τὰς χεῖρας αὐτοῦ· καὶ διέκαμψεν ἐπʼ αὐτὸν, καὶ διεθερμάνθη ἡ σὰρξ τοῦ παιδαρίου.
\VS{35}Καὶ ἐπέστρεψε, καὶ ἐπορεύθη ἐν τῇ οἰκίᾳ ἔνθεν καὶ ἔνθεν· καὶ ἀνέβη καὶ συνέκαμψεν ἐπὶ τὸ παιδάριον ἕως ἑπτάκις· καὶ ἤνοιξε τὸ παιδάριον τοὺς ὀφθαλμοὺς αὐτοῦ.
\VS{36}Καὶ ἐξεβόησεν Ἑλισαιὲ πρὸς Γιεζὶ, καὶ εἶπε, κάλεσον τὴν Σωμανίτην ταύτην· καὶ ἐκάλεσε, καὶ εἰσῆλθε πρὸς αὐτόν· καὶ εἶπεν Ἑλισαιὲ, λάβε τὸν υἱόν σου.
\VS{37}Καὶ εἰσῆλθεν ἡ γυνὴ, καὶ ἔπεσεν ἐπὶ τοὺς πόδας αὐτοῦ, καὶ προσεκύνησεν ἐπὶ τὴν γῆν· καὶ ἔλαβε τὸν υἱὸν αὐτῆς, καὶ ἐξῆλθε.
\par }{\PP \VS{38}Καὶ Ἑλισαιὲ ἐπέστρεψεν εἰς Γάλγαλα· καὶ ὁ λιμὸς ἐν τῇ γῇ, καὶ υἱοὶ τῶν προφητῶν ἐκάθηντο ἐνώπιον αὐτοῦ· καὶ εἶπεν Ἑλισαιὲ τῷ παιδαρίῳ αὐτοῦ, ἐπίστησον τὸν λέβητα τὸν μέγαν, καὶ ἕψε ἕψεμα τοῖς υἱοῖς τῶν προφητῶν.
\VS{39}Καὶ ἐξῆλθεν εἰς τὸν ἀγρὸν συλλέξαι ἀριώθ· καὶ εὗρεν ἄμπελον ἐν τῷ ἀγρῷ, καὶ συνέλεξεν ἀπʼ αὐτῆς τολύπην ἀγρίαν πλῆρες τὸ ἱμάτιον αὐτοῦ, καὶ ἐνέβαλεν εἰς τὸν λέβητα τοῦ ἑψέματος, ὅτι οὐκ ἔγνωσαν, καὶ ἐνέχει τοῖς ἀνδράσι φαγεῖν·
\VS{40}καὶ ἐγένετο ἐν τῷ ἐσθίειν αὐτοὺς ἐκ τοῦ ἑψέματος, καὶ ἰδοὺ ἀνεβόησαν, καὶ εἶπαν, θάνατος ἐν τῷ λέβητι, ἄνθρωπε τοῦ Θεοῦ· καὶ οὐκ ἠδύναντο φαγεῖν.
\VS{41}Καὶ εἶπε, λάβετε ἅλευρον, καὶ ἐμβάλετε εἰς τὸν λέβητα· καὶ εἶπεν Ἑλισαιὲ πρὸς Γιεζὶ τὸ παιδάριον, ἔγχει τῷ λαῷ καὶ ἐσθιέτωσαν· καὶ οὐκ ἐγενήθη ἐκεῖ ἔτι ῥῆμα πονηρὸν ἐν τῷ λέβητι.
\par }{\PP \VS{42}Καὶ ἀνὴρ διῆλθεν ἐκ Βαιθαρισὰ, καὶ ἤνεγκε πρὸς τὸν ἄνθρωπον τοῦ Θεοῦ πρωτογεννημάτων εἴκοσι ἄρτους κριθίνους καὶ παλάθας· καὶ εἶπε, δότε τῷ λαῷ καὶ ἐσθιέτωσαν.
\VS{43}Καὶ εἶπεν ὁ λειτουργὸς αὐτοῦ, τί δῶ τοῦτο ἐνώπιον ἑκατὸν ἀνδρῶν; καὶ εἶπε, δὸς τῷ λαῷ καὶ ἐσθιέτωσαν, ὅτι τάδε λέγει Κύριος, φάγονται καὶ καταλείψουσι.
\VS{44}Καὶ ἔφαγον καὶ κατέλιπον, κατὰ τὸ ῥῆμα Κυρίου.

\par }\Chap{5}{\PP \VerseOne{1}Καὶ Ναιμὰν ὁ ἄρχων τῆς δυνάμεως Συρίας ἦν ἀνὴρ μέγας ἐνώπιον τοῦ κυρίου αὐτοῦ, καὶ τεθαυμασμένος προσώπῳ, ὅτι ἐν ὑτῶ ἔδωκε Κύριος σωτηρίαν Συρίᾳ· καὶ ὁ ἀνὴρ ἦν δυνατὸς ἰσχύϊ, λελεπρωμένος.
\VS{2}Καὶ Συρία ἐξῆλθον μονόζωνοι, καὶ ᾐχμαλώτευσαν ἐκ γῆς Ἰσραὴλ νεάνιδα μικρὰν, καὶ ἦν ἐνώπιον τῆς γυναικὸς Ναιμάν.
\VS{3}Ἡ δὲ εἶπε τῇ κυρίᾳ αὐτῆς, ὄφελον ὁ κύριός μου ἐνώπιον τοῦ προφήτου τοῦ Θεοῦ τοῦ ἐν Σαμαρείᾳ, τότε ἀποσυνάξει αὐτὸν ἀπὸ τῆς λέπρας αὐτοῦ.
\VS{4}Καὶ εἰσῆλθε καὶ ἀπήγγειλε τῷ κυρίῳ ἑαυτῆς, καὶ εἶπεν, οὕτως καὶ οὕτως ἐλάλησεν ἡ νεᾶνις ἡ ἐκ γῆς Ἰσραήλ.
\par }{\PP \VS{5}Καὶ εἶπε βασιλεὺς Συρίας πρὸς Ναιμὰν, δεῦρο, εἴσελθε καὶ ἐξαποστελῶ βιβλίον πρὸς βασιλέα Ἰσραήλ· καὶ ἐπορεύθη, καὶ ἔλαβεν ἐν τῇ χειρὶ αὐτοῦ δέκα τάλαντα ἀργυρίου, καὶ ἑξακισχιλίους χρυσοῦς, καὶ δέκα ἀλλασσομένας στολὰς.
\VS{6}Καὶ ἤνεγκε τὸ βιβλίον πρὸς τὸν βασιλέα Ἰσραὴλ, λέγων, καὶ νῦν ὡς ἂν ἔλθῃ τὸ βιβλίον τοῦτο πρὸς σὲ, ἰδοὺ ἀπέστειλα πρὸς σὲ Ναιμὰν τὸν δοῦλόν μου, καὶ ἀποσυνάξεις αὐτὸν ἀπὸ τῆς λέπρας αὐτοῦ.
\VS{7}Καὶ ἐγένετο ὡς ἀνέγνω βασιλεὺς Ἰσραὴλ τὸ βιβλίον, διέῤῥηξε τὰ ἱμάτια αὐτοῦ, καὶ εἶπεν, ὁ Θεὸς ἐγὼ τοῦ θανατῶσαι καὶ ζωοποιῆσαι, ὅτι οὗτος ἀποστέλλει πρὸς μὲ ἀποσυνάξαι ἄνδρα ἀπὸ τῆς λέπρας αὐτοῦ; ὅτι πλὴν γνῶτε δὴ καὶ ἴδετε ὅτι προφασίζεται οὗτός μοι.
\par }{\PP \VS{8}Καὶ ἐγένετο ὡς ἤκουσεν Ἐλισαιὲ, ὅτι διέῤῥηξεν ὁ βασιλεὺς Ἰσραὴλ τὰ ἱμάτια αὐτοῦ, καὶ ἀπέστειλε πρὸς τὸν βασιλέα Ἰσραὴλ, λέγων, ἱνατί διέῤῥηξας τὰ ἱμάτιά σου; ἐλθέτω δὴ πρὸς μὲ Ναιμὰν, καὶ γνώτω ὅτι ἐστι προφήτης ἐν Ἰσραήλ.
\par }{\PP \VS{9}Καὶ ἦλθε Ναιμὰν ἐν ἵππῳ καὶ ἅρματι, καὶ ἔστη ἐπὶ θύρας οἴκου Ἐλισαιέ.
\VS{10}Καὶ ἀπέστειλεν Ἐλισαιὲ ἄγγελον πρὸς αὐτὸν, λέγων, πορευθεὶς λοῦσαι ἑπτάκις ἐν τῷ Ἰορδάνῃ, καὶ ἐπιστρέψει ἡ σάρξ σου σοὶ καὶ καθαρισθήσῃ.
\VS{11}Καὶ ἐθυμώθη Ναιμάν καὶ ἀπῆλθε, καὶ εἶπεν, ἰδοὺ εἶπον, πρὸς μὲ πάντως ἐξελεύσεται καὶ στήσεται, καὶ ἐπικαλέσεται ἐν ὀνόματι Θεοῦ αὐτοῦ, καὶ ἐπιθήσει τὴν χεῖρα αὐτοῦ ἐπὶ τὸν τόπον, καὶ ἀποσυνάξει τὸ λεπρόν.
\VS{12}Οὐχὶ ἀγαθὸς Ἀβανὰ καὶ Φαρφὰρ ποταμοὶ Δαμασκοῦ ὑπὲρ πάντα τὰ ὕδατα Ἰσραὴλ; οὐχὶ πορευθεὶς λούσομαι ἐν αὐτοῖς, καὶ καθαρισθήσομαι; καὶ ἐξέκλινε καὶ ἀπῆλθεν ἐν θυμῷ.
\VS{13}Καὶ ἤγγισαν οἱ παῖδες αὐτοῦ, καὶ ἐλάλησαν πρὸς αὐτὸν, μέγαν λόγον ἐλάλησεν ὁ προφήτης πρὸς σέ· οὐχὶ ποιήσεις; καὶ ὅτι εἶπε πρὸς σέ, λοῦσαι καὶ καθαρίσθητι.
\VS{14}Καὶ κατέβη Ναιμὰν καὶ ἐβαπτίσατο ἐν τῷ Ἰορδάνῃ ἑπτάκις κατὰ τὸ ῥῆμα Ἐλισαῖε· καὶ ἐπέστρεψεν ἡ σὰρξ αὐτοῦ ὡς σὰρξ παιδαρίου μικροῦ, καὶ ἐκαθαρίσθη.
\par }{\PP \VS{15}Καὶ ἐπέστρεψε πρὸς Ἐλισαιὲ αὐτὸς καὶ πᾶσα ἡ παρεμβολὴ αὐτοῦ, καὶ ἦλθε καὶ ἔστη ἐνώπιον αὐτοῦ, καὶ εἶπεν, ἰδοὺ ἔγνωκα ὅτι οὐκ ἔστι Θεὸς ἐν πάσῃ τῇ γῇ, ὅτι ἀλλʼ ἢ ἐν τῷ Ἰσραήλ· καὶ νῦν λάβε τὴν εὐλογίαν παρὰ τοῦ δούλου σου.
\VS{16}Καὶ εἶπεν Ἐλισαιὲ ζῇ Κύριος ᾧ παρέστην ἐνώπιον αὐτοῦ, εἰ λήψομαι· καὶ παρεβιάσατο αὐτὸν λαβεῖν, καὶ ἠπείθησε·
\VS{17}Καὶ εἶπε Ναιμὰν, καὶ εἰ μὴ, δοθήτω δὴ τῷ δούλῳ σου γόμος ζεῦγος ἡμιόνων, καὶ σύ μοι δώσεις ἐκ τῆς γῆς τῆς πυῤῥᾶς, ὅτι οὐ ποιήσει ἔτι ὁ δοῦλός σου ὁλοκαύτωμα καὶ θυσίασμα θεοῖς ἑτέροις ἀλλʼ ἢ τῷ Κυρίῳ τῷ ῥήματι τούτῳ.
\VS{18}Καὶ ἱλάσεται Κύριος τῷ δούλῳ σου ἐν τῷ εἰσπορεύεσθαι τὸν κύριόν μου εἰς οἶκον Ῥεμμὰν προσκυνῆσαι ἐκεῖ· καὶ αὐτὸς ἐπαναπαύσεται ἐπὶ τῆς χειρός μου, καὶ προσκυνήσω ἐν οἴκῳ Ῥεμμὰν ἐν τῷ προσκυνεῖν αὐτὸν ἐν οἴκῳ Ῥεμμάν· καὶ ἱλάσεται δὴ Κύριος τῷ δούλῳ σου ἐν τῷ λόγῳ τούτῳ.
\VS{19}Καὶ εἶπεν Ἐλισαιὲ πρὸς Ναιμὰν, δεῦρο εἰς εἰρήνην· καὶ ἀπῆλθεν ἀπʼ αὐτοῦ εἰς δεβραθὰ τῆς γῆς.
\par }{\PP \VS{20}Καὶ εἶπε Γιεζὶ τὸ παιδάριον Ἐλισαιὲ, ἰδοὺ ἐφείσατο ὁ κύριός μου τοῦ Ναιμὰν τοῦ Σύρου τούτου, τοῦ μὴ λαβεῖν ἐκ χειρὸς αὐτοῦ ἃ ἐνήνοχε· ζῇ Κύριος, ὅτι εἰ μὴ δραμοῦμαι ὀπίσω αὐτοῦ, καὶ λήψομαι ἀπʼ αὐτοῦ τι.
\VS{21}Καὶ ἐδίωξε Γιεζὶ ὀπίσω τοῦ Ναιμάν· καὶ εἶδεν αὐτὸν Ναιμὰν τρέχοντα ὀπίσω αὐτοῦ, καὶ ἐπέστρεψεν ἀπὸ τοῦ ἅρματος εἰς ἀπαντὴν αὐτοῦ.
\VS{22}Καὶ εἶπεν, εἰρήνη· ὁ κύριός μου ἀπεστειλέ με, λέγων, ἰδοὺ νῦν ἦλθον πρὸς μὲ δύο παιδάρια ἐξ ὄρους Ἐφραὶμ ἀπὸ τῶν υἱῶν τῶν προφητῶν· δὸς δὴ αὐτοῖς τάλαντον ἀργυρίου, καὶ δύο ἀλλασσομένας στολάς.
\VS{23}Καὶ εἶπε, λάβε διτάλαντον ἀργυρίου· καὶ ἔλαβε δύο τάλαντα ἀρλυρίον ἐν δυσὶ θυλάκοις, καὶ δύο ἀλλασσομένας στολὰς, καὶ ἔδωκεν ἐπὶ δύο παιδάρια αὐτοῦ, καὶ ᾖραν ἔμπροσθεν αὐτοῦ.
\VS{24}Καὶ ἦλθεν εἰς τὸ σκοτεινὸν, καὶ ἔλαβεν ἐκ τῶν χειρῶν αὐτῶν, καὶ παρέθετο ἐν οἴκῳ, καὶ ἐξαπέστειλε τοὺς ἄνδρας.
\par }{\PP \VS{25}Καὶ αὐτὸς εἰσῆλθε, καὶ παρειστήκει πρὸς τὸν κύριον αὐτοῦ· καὶ εἶπε πρὸς αὐτὸν Ἐλισαιὲ, πόθεν Γιεζί; καὶ εἶπε Γιεζί, Οὐ πεπόρευται ὁ δοῦλός σου ἔνθα καὶ ἔνθα.
\VS{26}Καὶ εἶπε πρὸς αὐτὸν Ἐλισαιὲ, οὐχὶ ἡ καρδία μου ἐπορεύθη μετὰ σοῦ ὅτε ἐπέστρεψεν ὁ ἀνὴρ ἀπὸ τοῦ ἅρματος εἰς συναντήν σοι; καὶ νῦν ἔλαβες τὸ ἀργύριον, καὶ νῦν ἔλαβες τὰ ἱμάτια, καὶ ἐλαιῶνας καὶ ἀμπελῶνας καὶ πρόβατα καὶ βόας καὶ παῖδας καὶ παιδίσκας.
\VS{27}Καὶ ἡ λέπρα Ναιμὰν κολληθήσεται ἐν σοὶ καὶ ἐν τῷ σπέρματί σου εἰς τὸν αἰῶνα· καὶ ἐξῆλθεν ἐκ προσώπου αὐτοῦ λελεπρωμένος ὡσεὶ χιών.

\par }\Chap{6}{\PP \VerseOne{1}Καὶ εἶπον υἱοὶ τῶν προφητῶν πρὸς Ἐλισαιὲ, ἰδοὺ δὴ ὁ τόπος ἐν ᾧ ἡμεῖς οἰκοῦμεν ἐνώπιόν σου στενὸς ἀφʼ ἡμῶν.
\VS{2}Πορευθῶμεν δὴ ἕως τοῦ Ἰορδάνου, καὶ λάβωμεν ἐκεῖθεν ἀνὴρ εἷς δοκὸν μίαν, καὶ ποιήσωμεν ἑαυτοῖς ἐκεῖ τοῦ οἰκεῖν ἐκεῖ· καὶ εἶπε, δεῦτε.
\VS{3}Καὶ εἶπεν ὁ εἷς ἐπιεικῶς, δεῦρο μετὰ τῶν δούλων σου· καὶ εἶπεν, ἐγὼ πορεύσομαι.
\VS{4}Καὶ ἐπορεύθη μετʼ αὐτῶν, καὶ ἦλθον εἰς τὸν Ἰορδάνην, καὶ ἔτεμνον τὰ ξύλα.
\VS{5}Καὶ ἰδοὺ ὁ εἷς καταβάλλων τὴν δοκὸν, καὶ τὸ σιδήριον ἐξέπεσεν εἰς τὸ ὕδωρ, καὶ ἐβόησεν, ὤ κύριε, καὶ αὐτὸ κεκρυμμένον.
\VS{6}Καὶ εἶπεν ὁ ἄνθρωπος τοῦ Θεοῦ, ποῦ ἔπεσε; καὶ ἔδειξεν αὐτῷ τὸν τόπον· καὶ ἀπέκνισε ξύλον καὶ ἔῤῥιψεν ἐκεῖ, καὶ ἐπεπόλασεν τὸ σιδήριον.
\VS{7}Καὶ εἴρηκεν, ὕψωσον σεαυτῷ· καὶ ἐξέτεινε τὴν χεῖρα, καὶ ἔλαβεν αὐτό.
\par }{\PP \VS{8}Καὶ ὁ βασιλεὺς Συρίας ἦν πολεμῶν ἐν Ἰσραήλ· καὶ ἐβουλεύσατο πρὸς τοὺς παῖδας αὐτοῦ, λέγων, εἰς τὸν τόπον τόνδε τινὰ ἐλμωνὶ παρεμβαλῶ.
\VS{9}Καὶ ἀπέστειλεν Ἐλισαιὲ πρὸς τὸν βασιλέα Ἰσραὴλ, λέγων, φύλαξαι μὴ παρελθεῖν ἐν τῷ τόπῳ τούτῳ, ὅτι ἐκεῖ Συρία κέκρυπται.
\VS{10}Καὶ ἀπέστειλεν ὁ βασιλεὺς Ἰσραὴλ εἰς τὸν τόπον ὃν εἶπεν αὐτῷ Ἐλισαιὲ, καὶ ἐφυλάξατο ἐκεῖθεν οὐ μίαν οὐδὲ δύο.
\par }{\PP \VS{11}Καὶ ἐξεκινήθη ἡ ψυχὴ βασιλέως Συρίας περὶ τοῦ λόγου τούτου· καὶ ἐκάλεσε τοὺς παῖδας αὐτοῦ, καὶ εἶπε πρὸς αὐτοὺς, οὐκ ἀναγγελεῖτέ μοι τίς προδίδωσί με βασιλεῖ Ἰσραήλ;
\VS{12}Καὶ εἶπεν εἷς τῶν παίδων αὐτοῦ, οὐχί κύριέ μου βασιλεῦ, ὅτι Ἐλισαιὲ ὁ προφήτης ὁ ἐν Ἰσραὴλ ἀναγγέλλει τῷ βασιλεῖ Ἰσραὴλ πάντας τοὺς λόγους, οὓς ἐὰν λαλήσῃς ἐν τῷ ταμείῳ τοῦ κοιτῶνός σου.
\VS{13}Καὶ εἶπε, δεῦτε ἴδετε ποῦ οὗτος, καὶ ἀποστείλας λήψομαι αὐτόν· καὶ ἀπήγγειλαν αὐτῷ, λέγοντες, ἰδοὺ ἐν Δωθαΐμ.
\par }{\PP \VS{14}Καὶ ἀπέστειλεν ἐκεῖ ἵππον καὶ ἅρμα καὶ δύναμιν βαρεῖαν, καὶ ἦλθον νυκτὸς καὶ περιεκύκλωσαν τὴν πόλιν.
\VS{15}Καὶ ὤρθρισεν ὁ λειτουργὸς Ἐλισαιὲ ἀναστῆναι, καὶ ἐξῆλθε· καὶ ἰδοὺ δύναμις κυκλοῦσα τὴν πόλιν, καὶ ἵππος καὶ ἅρμα· καὶ εἶπε τὸ παιδάριον πρὸς αὐτὸν, ὦ κύριε, πῶς ποιήσομεν;
\VS{16}Καὶ εἶπεν Ἐλισαιὲ, μὴ φοβοῦ, ὅτι πλείους οἱ μεθʼ ἡμῶν ὑπὲρ τοὺς μετʼ αὐτῶν.
\VS{17}Καὶ προσηύξατο Ἐλισαιὲ, καὶ εἶπε, Κύριε, διάνοιξον δὴ τοὺς ὀφθαλμοὺς τοῦ παιδαρίου καὶ ἰδέτω· καὶ διήνοιξε Κύριος τοὺς ὀφθαλμοὺς αὐτοῦ καὶ εἶδε· καὶ ἰδοὺ τὸ ὄρος πλῆρες ἵππων, καὶ ἅρμα πυρὸς περικύκλῳ Ἐλισαιέ.
\VS{18}Καὶ κατέβησαν πρὸς αὐτόν· καὶ προσηύξατο πρὸς Κύριον, καὶ εἶπε, πάταξον δὴ τὸ ἔθνος τοῦτο ἀορασίᾳ· καὶ ἐπάταξεν αὐτοὺς ἀορασίᾳ, κατὰ τὸ ῥῆμα Ἑλισαιὲ.
\VS{19}Καὶ εἶπε πρὸς αὐτοὺς Ἑλισαιὲ, οὐχὶ αὕτη ἡ πόλις καὶ αὕτη ἡ ὁδός· δεῦτε ὀπίσω μου, καὶ ἄξω ὑμᾶς πρὸς τὸν ἄνδρα ὃν ζητεῖτε· καὶ ἀπήγαγεν αὐτοὺς πρὸς Σαμάρειαν.
\VS{20}Καὶ ἐγένετο ὡς εἰσῆλθον εἰς Σαμάρειαν, καὶ εἶπεν Ἑλισαιὲ, ἄνοιξον δή Κύριε τοὺς ὀφθαλμοὺς αὐτῶν καὶ ἰδέτωσαν· καὶ διήνοιξε Κύριος τοὺς ὀφθαλμοὺς αὐτῶν, καὶ εἶδου· καὶ ἰδοὺ ἦσαν ἐν μέσῳ Σαμαρείας.
\par }{\PP \VS{21}Καὶ εἶπεν ὁ βασιλεὺς Ἰσραήλ πρὸς Ἑλισαιὲ, ὡς εἶδεν αὐτοὺς, εἰ πατάξας πατάξω, πάτερ;
\VS{22}Καὶ εἶπεν, Οὐ πατάξεις, εἰ μὴ οὓς ᾐχμαλώτευσας ἐν ῥομφαίᾳ σου καὶ τόξῳ σου σὺ τύπτεις· παράθες ἄρτους καὶ ὕδωρ ἐνώπιον αὐτῶν, καὶ φαγέτωσαν καὶ πιέτωσαν, καὶ ἀπελθέτωσαν πρὸς τὸν κύριον αὐτῶν.
\VS{23}Καὶ παρέθηκεν αὐτοῖς παράθεσιν μεγάλην, καὶ ἔφαγον καὶ ἔπιον· καὶ ἀπέστειλεν αὐτοὺς, καὶ ἀπῆλθον πρὸς τὸν κύριον αὐτῶν· καὶ οὐ προσέθεντο ἔτι μονόζωνοι Συρίας τοῦ ἐλθεῖν εἰς γῆν Ἰσραήλ.
\par }{\PP \VS{24}Καὶ ἐγένετο μετὰ ταῦτα, καὶ ἤθροισεν υἱὸς Ἅδερ βασιλεὺς Συρίας πᾶσαν τὴν παρεμβολὴν αὐτοῦ, καὶ ἀνέβη, καὶ περιεκάθισαν ἐπὶ Σαμάρειαν.
\VS{25}Καὶ ἐγένετο λιμὸς μέγας ἐν Σαμαρείᾳ· καὶ ἰδοὺ περιεκάθηντο ἐπʼ αὐτὴν ἕως οὗ ἐγενήθη κεφαλὴ ὄνου πεντήκοντα ἀργυρίου, καὶ τέταρτον τοῦ κάβου κόπρου περιστερῶν πέντε ἀργυρίου.
\par }{\PP \VS{26}Καὶ ἦν ὁ βασιλεὺς Ἰσραὴλ διαπορευόμενος ἐπὶ τοῦ τείχους· καὶ γυνὴ ἐβόησε πρὸς αὐτὸν, λέγουσα, Σῶσον κύριε βασιλεῦ.
\VS{27}Καὶ εἶπεν αὐτῇ, μὴ σὲ σώσαι Κύριος, πόθεν σώσω σε; μὴ ἀπὸ ἅλωνος ἢ ἀπὸ ληνοῦ;
\VS{28}Καὶ εἶπεν αὐτῇ ὁ βασιλεύς, τί ἐστι σοι; καὶ εἶπεν ἡ γυνή, αὕτη εἶπε πρὸς μέ, δὸς τὸν υἱόν σου καὶ φαγόμεθα αὐτὸν σήμερον, καὶ τὸν υἱόν μου φαγόμεθα αὐτὸν αὔριον.
\VS{29}Καὶ ἡψήσαμεν τὸν υἱόν μου καὶ ἐφάγομεν αὐτὸν, καὶ εἶπον πρὸς αὐτὴν τῇ ἡμέρᾳ τῇ δευτέρᾳ, δὸς τὸν υἱόν σου καὶ φάγωμεν αὐτόν· καὶ ἔκρυψε τὸν υἱὸν αὐτῆς.
\VS{30}Καὶ ἐγένετο ὡς ἤκουσεν ὁ βασιλεὺς Ἰσραὴλ τοὺς λόγους τῆς γυναικὸς, διέῤῥηξε τὰ ἱμάτια αὐτοῦ, καὶ αὐτὸς διεπορεύετο ἐπὶ τοῦ τείχους, καὶ εἶδεν ὁ λαὸς τὸν σάκκον ἐπὶ τῆς σαρκὸς αὑτοῦ ἔσωθεν.
\VS{31}Καὶ εἶπε, τάδε ποιήσαι μοι ὁ Θεὸς καὶ τάδε προσθείη, εἰ στήσεται ἡ κεφαλὴ Ἑλισαιὲ ἐπʼ αὐτῷ σήμερον.
\par }{\PP \VS{32}Καὶ Ἑλισαιὲ ἐκάθητο ἐν τῷ οἴκῳ αὐτοῦ, καὶ οἱ πρεσβύτεροι ἐκάθηντο μετʼ αὐτοῦ· καὶ ἀπέστειλεν ἄνδρα πρὸ προσώπου αὐτοῦ· πρὶν ἐλθεῖν τὸν ἄγγελον πρὸς αὐτόν, καὶ αὐτὸς εἶπε πρὸς τοὺς πρεσβυτέρους, εἰ εἴδετε ὅτι ἀπέστειλεν ὁ υἱὸς τοῦ φονευτοῦ οὗτος ἀφελεῖν τὴν κεφαλήν μου; ἴδετε ὡς ἂν ἔλθῃ ὁ ἄγγελος, ἀποκλείσατε τὴν θύραν, καὶ παραθλίψατε αὐτὸν ἐν τῇ θύρᾳ· οὐχὶ φωνὴ τῶν ποδῶν τοῦ κυρίου αὐτοῦ κατόπισθεν αὐτοῦ;
\VS{33}Ἔτι αὐτοῦ λαλοῦντος μετʼ αὐτῶν, καὶ ἰδοὺ ἄγγελος κατέβη πρὸς αὐτὸν, καὶ εἶπεν, ἰδοὺ αὕτη ἡ κακία παρὰ Κυρίου· τί ὑπομείνω τῷ Κυρίῳ ἔτι;

\par }\Chap{7}{\PP \VerseOne{1}Καὶ εἶπεν Ἑλισαιὲ ἄκουσον λόγον Κυρίου· τάδε λέλει Κύριος, ὡς ἡ ὥρα αὕτη, αὔριον μέτρον σεμιδάλεως σίκλου, καὶ δίμετρον κριθῶν σίκλου, ἐν ταῖς πύλαις Σαμαρείας.
\VS{2}Καὶ ἀπεκρίθη ὁ τριστάτης ἐφʼ ὃν ὁ βασιλεὺς ἐπανεπαύετο ἐπὶ τὴν χεῖρα αὐτοῦ τῷ Ἑλισαιὲ, καὶ εἶπεν, ἰδοὺ ποιήσει Κύριος καταράκτας ἐν οὐρανῷ, μὴ ἔσται τὸ ῥῆμα τοῦτο; καὶ Ἑλισαιὲ εἶπεν, ἰδοὺ σὺ ὄψει τοῖς ὀφθαλμοῖς σου, καὶ ἐκεῖθεν οὐ φάγῃ.
\par }{\PP \VS{3}Καὶ τέσσαρες ἄνδρες ἦσαν λεπροὶ παρὰ τὴν θύραν τῆς πόλεως, καὶ εἶπεν ἀνὴρ πρὸς τὸν πλησίον αὐτοῦ, τί ἡμεῖς καθήμεθα ὧδε ἕως ἀποθάνωμεν;
\VS{4}Ἐὰν εἴπωμεν, εἰσέλθωμεν εἰς τὴν πόλιν, καὶ ὁ λιμὸς ἐν τῇ πόλει, καὶ ἀποθανούμεθα ἐκεῖ· καὶ ἐὰν καθίσωμεν ὧδε, καὶ ἀποθανούμεθα· καὶ νῦν δεῦτε, καὶ ἐμπεσωμεν εἰς τὴν παρεμβολὴν Συρίας· ἐὰν ζωογονήσωσιν ἡμᾶς, καὶ ζησόμεθα· καὶ ἐὰν θανατώσωσιν ἡμᾶς, καὶ ἀποθανούμεθα.
\VS{5}Καὶ ἀνέστησαν ἐν τῷ σκότει εἰσελθεῖν εἰς τὴν παρεμβολὴν Συρίας· καὶ ἦλθον εἰς μέρος παρεμβολῆς Συρίας, καὶ ἰδοὺ οὐκ ἔστιν ἀνὴρ ἐκεῖ.
\VS{6}Καὶ Κύριος ἀκουστὴν ἐποίησε παρεμβολὴν τὴν Συρίας φωνὴν ἅρματος καὶ φωνὴν ἵππου, φωνὴν δυνάμεως μεγάλης· καὶ εἶπεν ἀνὴρ πρὸς τὸν ἀδελφὸν αὐτοῦ, νῦν ἐμισθώσατο ἐφʼ ἡμᾶς ὁ βασιλεὺς Ἰσραὴλ τοὺς βασιλέας τῶν Χετταίων καὶ τοὺς βασιλέας Αἰγύπτου τοῦ ἐλθεῖν ἐφʼ ἡμᾶς.
\VS{7}Καὶ ἀνέστησαν καὶ ἀπέδρασαν ἐν τῷ σκότει καὶ ἐγκατέλιπον τὰς σκηνὰς αὐτῶν, καὶ τοὺς ἵππους αὐτῶν, καὶ τοὺς ὄνους αὐτῶν ἐν τῇ παρεμβολῇ ὡς ἔστι, καὶ ἔφυγον πρὸς τὴν ψυχὴν ἑαυτῶν.
\par }{\PP \VS{8}Καὶ εἰσῆλθον οἱ λεπροὶ οὗτοι ἕως μέρους τῆς παρεμβολῆς, καὶ εἰσῆλθον εἰς σκηνὴν μίαν, καὶ ἔφαγον, καὶ ἔπιον, καὶ ἦραν ἐκεῖθεν ἀργύριον, καὶ χρυσίον, καὶ ἱματισμόν· καὶ ἐπορεύθησαν, καὶ ἐπέστρεψαν ἐκεῖθεν, καὶ εἰσῆλθον εἰς σκηνὴν ἄλλην, καὶ ἔλαβον ἐκεῖθεν καὶ ἐπορεύθησαν, καὶ κατέκρυψαν.
\VS{9}Καὶ εἶπεν ἀνὴρ πρὸς τὸν πλησιον αὐτοῦ, οὐχ οὕτως ἡμεῖς ποιοῦμεν· ἡ ἡμέρα αὕτη, ἡμέρα εὐαγγελίας ἐστὶ, καὶ ἡμεῖς σιωπῶμεν, καὶ μένομεν ἕως φωτὸς τοῦ πρωῒ, καὶ εὑρήσομεν ἀνομίαν· καὶ νῦν δεῦρο, καὶ εἰσέλθωμεν καὶ ἀναγγείλωμεν εἰς τὸν οἶκον τοῦ βασιλέως.
\par }{\PP \VS{10}Καὶ εἰσῆλθον καὶ ἐβόησαν πρὸς τὴν πύλην τῆς πόλεως, καὶ ἀνήγγειλαν αὐτοῖς, λέγοντες, εἰσήλθομεν εἰς τὴν παρεμβολὴν Συρίας, καὶ ἰδοὺ οὐκ ἔστιν ἐκεῖ ἀνὴρ καὶ φωνὴ ἀνθρώπου, ὅτι εἰ μὴ ἵππος δεδεμένος καὶ ὄνος, καὶ αἱ σκηναὶ αὐτῶν ὡς εἰσί.
\VS{11}Καὶ ἐβόησαν οἱ θυρωροὶ, καὶ ἀνήγγειλαν εἰς τὸν οἶκον τοῦ βασιλέως ἔσω.
\par }{\PP \VS{12}Καὶ ἀνέστη ὁ βασιλεὺς νυκτὸς, καὶ εἶπε πρὸς τοὺς παῖδας αὐτοῦ, ἀναγγελῶ δὴ ὑμῖν ἃ ἐποίησεν ἡμῖν Συρία· ἔγνωσαν ὅτι πεινῶμεν ἡμεῖς, καὶ ἐξῆλθαν ἐκ τῆς παρεμβολῆς καὶ ἐκρύβησαν ἐν τῷ ἀγρῷ, λέγοντες, ὅτι ἐξελεύσονται ἐκ τῆς πόλεως, καὶ συλληψόμεθα αὐτοὺς ζῶντας, καὶ εἰς τὴν πόλιν εἰσελευσόμεθα.
\VS{13}Καὶ ἀπεκρίθη εἷς τῶν παίδων αὐτοῦ καὶ εἶπε, λαβέτωσαν δὴ πέντε τῶν ἵππων τῶν ὑπολελειμμένων οἳ κατελείφθησεν ὧδε, ἰδού εἰσι πρὸς πᾶν τὸ πλῆθος Ἰσραὴλ τὸ ἐκλεῖπον, καὶ ἀποστελοῦμεν ἐκεῖ καὶ ὀψόμεθα.
\VS{14}Καὶ ἔλαβον δύο ἐπιβάτας ἵππων· καὶ ἀπέστειλεν ὁ βασιλεὺς Ἰσραὴλ ὀπίσω τοῦ βασιλέως Συρίας, λέγων, δεῦτε, καὶ ἴδετε.
\VS{15}Καὶ ἐπορεύθησαν ὀπίσω αὐτῶν ἕως τοῦ Ἰορδάνου. καὶ ἰδοὺ πᾶσα ἡ ὁδὸς πλήρης ἱματίων καὶ σκευῶν ὧν ἔῤῥιψε Συρία ἐν τῷ θαμβεῖσθαι αὐτούς· καὶ ἐπέστρεψαν οἱ ἄγγελοι καὶ ἀνήγγειλαν τῷ βασιλεῖ.
\par }{\PP \VS{16}Καὶ ἐξῆλθεν ὁ λαὸς καὶ διήρπασεν τὴν παρεμβολὴν Συρίας· καὶ ἐγένετο μέτρον σεμισάλεως σίκλου, κατὰ τὸ ῥῆμα Κυρίου, καὶ δίμετρον κριθῶν σίκλου.
\VS{17}Καὶ ὁ βασιλεὺς κατέστησε τὸν τριστάτην ἐφʼ ὃν ὁ βασιλεὺς ἐπανεπαύετο τῇ χειρὶ αὐτοῦ ἐπὶ τῆς πυλῆς· καὶ συνεπάτησεν αὐτὸν ὁ λαὸς ἐν τῇ πύλῃ, καὶ ἀπέθανε καθὰ ἐλάλησεν ὁ ἄνθρωπος τοῦ Θεοῦ, ὃς ἐλάλησεν ἐν τῷ καταβῆναι τὸν ἄγγελον πρὸς αὐτόν.
\VS{18}Καὶ ἐγένετο καθὰ ἐλάλησεν Ἑλισαιὲ πρὸς τὸν βασιλέα, λέγων, δίμετρον κριθῆς σίκλου καὶ μέτρον σεμιδάλεως σίκλου· καὶ ἔσται ὡς ἡ ὥρα αὔριον ἐν τῇ πύλῃ Σαμαρείας.
\VS{19}Καὶ ἀπεκρίθη ὁ τριστάτης τῷ Ἑλισαιὲ, καὶ εἶπεν, ἰδοὺ Κύριος ποιεῖ καταράκτας ἐν τῷ οὐρανῷ, μὴ ἔσται τὸ ῥῆμα τοῦτο; καὶ εἶπεν Ἑλισαιὲ, ἰδοὺ ὄψει τοῖς ὀφθαλμοῖς σου, καὶ ἐκεῖθεν οὐ μὴ φάγῃ.
\VS{20}Καὶ ἐγένετο οὕτως, καὶ συνεπάτησαν αὐτὸν ὁ λαὸς ἐν τῇ πύλῃ, καὶ ἀπέθανε.

\par }\Chap{8}{\PP \VerseOne{1}Καὶ Ἑλισαιὲ ἐλάλησε πρὸς τὴν γυναῖκα, ἧς ἐζωπύρησε τὸν υἱὸν, λέγων, ἀνάστηθι καὶ δεῦρο σὺ καὶ ὁ οἶκός σου, καὶ παροίκει οὗ ἐὰν παροικήσῃς, ὅτι κέκληκε Κύριος λιμὸν ἐπὶ τὴν γῆν, καί γε ἦλθεν ἐπὶ τὴν γῆν ἑπτὰ ἔτη.
\VS{2}Καὶ ἀνέστη ἡ γυνὴ, καὶ ἐποίησε κατὰ τὸ ῥῆμα Ἑλισαιὲ καὶ αὐτὴ καὶ ὁ οἶκος αὐτῆς, καὶ παρῴκει ἐν γῇ ἀλλοφύλων ἑπτὰ ἔτη.
\par }{\PP \VS{3}Καὶ ἐγένετο μετὰ τὸ τέλος τῶν ἑπτὰ ἐτῶν, καὶ ἐπέστρεψεν ἡ γυνὴ ἐκ γῆς ἀλλοφύλων εἰς τὴν πόλιν, καὶ ἦλθε βοῆσαι πρὸς τὸν βασιλέα περὶ τοῦ οἴκου ἑαυτῆς καὶ περὶ τῶν ἀγρῶν αὐτῆς.
\VS{4}Καὶ ὁ βασιλεὺς ἐλάλει πρὸς Γιεζὶ τὸ παιδάριον Ἑλισαιὲ τοῦ ἀνθρώπου τοῦ Θεοῦ, λέγων, διήγησαι δὴ ἐμοὶ πάντα τὰ μεγάλα ἃ ἐποίησεν Ἑλισαιέ.
\VS{5}Καὶ ἐγένετο αὐτοῦ ἐξηγουμένου τῷ βασιλεῖ, ὡς ἑζωπύρησεν υἱὸν τεθνηκότα, καὶ ἰδοὺ ἡ γυνὴ ἧς ἐζωπύρησε τὸν υἱὸν αὐτῆς Ἑλισαιὲ, βοῶσα πρὸς τὸν βασιλέα περὶ τοῦ οἴκου ἑαυτῆς καὶ περὶ τῶν ἀγρῶν ἑαυτῆς· καὶ εἶπε Γιεζὶ, Κύριε βασιλεῦ, αὕτη ἡ γυνὴ, καὶ οὗτος ὁ υἱὸς αὐτῆς, ὃν ἐζωπύρησεν Ἑλισαιέ.
\VS{6}Καὶ ἐπηρώτησεν ὁ βασιλεὺς τὴν γυναῖκα· καὶ διηγήσατο αὐτῷ· καὶ ἔδωκεν αὐτῇ ὁ βασιλεὺς εὐνοῦχον ἕνα, λέγων, ἐπίστρεψον πάντα τὰ αὐτῆς, καὶ πάντα τὰ γεννήματα τοῦ ἀγροῦ ἀπὸ τῆς ἡμέρας ἧς κατέλιπε τὴν γῆν ἕως τοῦ νῦν.
\par }{\PP \VS{7}Καὶ ἦλθεν Ἑλισαιὲ εἰς Δαμασκόν· καὶ υἱὸς Ἄδερ βασιλεὺς Συρίας ἠῤῥώστησε, καὶ ἀνήγγειλαν αὐτῷ, λέγοντες, ἥκει ὁ ἄνθρωπος τοῦ Θεοῦ ἕως ᾧδε.
\VS{8}Καὶ εἶπεν ὁ βασιλεὺς πρὸς Ἀζαὴλ, λάβε ἐν τῇ χειρί σου μαναὰ, καὶ δεῦρο εἰς ἀπαντὴν τοῦ ἀνθρώπου τοῦ Θεοῦ, καὶ ἐπιζήτησον τὸν Κύριον παρʼ αὐτοῦ, λέγων, εἰ ζήσομαι ἐκ τῆς ἀῤῥωστίας μου ταύτης;
\VS{9}Καὶ ἐπορεύθη Ἀζαὴλ εἰς ἀπαντὴν αὐτοῦ, καὶ ἔλαβε μαναὰ ἐν τῇ χειρὶ αὐτοῦ, καὶ πάντα τὰ ἀγαθὰ Δαμασκοῦ, ἄρσιν τεσσαράκοντα καμήλων, καὶ ἦλθε καὶ ἔστη ἐνώπιον αὐτοῦ, καὶ εἶπε πρὸς Ἑλισαιὲ, υἱός σου υἱὸς Ἅδερ βασιλεὺς Συρίας ἀπέστειλέ με πρὸς σὲ ἐπερωτῆσαι, λέγων, εἰ ζήσομαι ἐκ τῆς ἀῤῥωστίας μου ταύτης;
\VS{10}Καὶ εἶπεν Ἑλισαιὲ, δεῦρο, εἶπον, ζωῇ ζήσῃ, καὶ ἔδειξέ μοι Κύριος ὅτι θανάτῳ ἀποθανῇ.
\VS{11}Καὶ παρέστη τῷ προσώπῳ αὐτοῦ, καὶ ἔθηκεν ἕως αἰσχύνης· καὶ ἔκλαυσεν ὁ ἄνθρωπος τοῦ Θεοῦ.
\VS{12}Καὶ εἶπεν Ἀζαὴλ, τί ὅτι ὁ κύριός μου κλαίει; καὶ εἶπεν, ὅτι οἶδα ὅσα ποιήσεις τοῖς υἱοῖς Ἰσραὴλ κακά· τὰ ὀχυρώματα αὐτῶν ἐξαποστελεῖς ἐν πυρὶ, καὶ τοὺς ἐκλεκτοὺς αὐτῶν ἐν ῥομφαίᾳ ἀποκτενεῖς, καὶ τὰ νήπια αὐτῶν ἐνσείσεις, καὶ τὰς ἐν γαστρὶ ἐχούσας αὐτῶν ἀναῥῥήξεις.
\VS{13}Καὶ εἶπεν Ἀζαὴλ, τίς ἐστιν ὁ δοῦλός σου, ὁ κύων ὁ τεθνηκὼς, ὅτι ποιήσει τὸ ῥῆμα τοῦτο; καὶ εἶπεν Ἑλισαιὲ, ἔδειξέ μοι Κύριός σε βασιλεύοντα ἐπὶ Συρίαν.
\VS{14}Καὶ ἀπῆλθεν ἀπὸ Ἑλισαιὲ, καὶ εἰσῆλθε πρὸς τὸν κύριον αὐτοῦ, καὶ εἶπεν αὐτῷ, τί εἶπέ σοι Ἑλισαιέ; καὶ εἶπεν, εἶπέ μοι, ζωῇ ζήσῃ.
\VS{15}Καὶ ἐγένετο τῇ ἐπαύριον, καὶ ἔλαβε τὸ μαχβὰρ καὶ ἔβαψεν ἐν τῷ ὕδατι, καὶ περιέβαλεν ἐπὶ τὸ πρόσωπον αὐτοῦ, καὶ ἀπέθανε· καὶ ἐβασίλευσεν Ἀζαὴλ ἀντʼ αὐτοῦ.
\par }{\PP \VS{16}Ἐν ἔτει πέμπτῳ τῷ Ἰωρὰμ υἱῷ Ἀχαὰβ βασιλεῖ Ἰσραὴλ, καὶ Ἰωσαφὰτ βασιλεῖ Ἰούδα, ἐβασίλευσεν Ἰωρὰμ υἱὸς Ἰωσαφὰτ βασιλεὺς Ἰούδα.
\VS{17}Υἱὸς τριάκοντα καὶ δύο ἐτῶν ἦν ἐν τῷ βασιλεύειν αὐτὸν, καὶ ὀκτὼ ἔτη ἐβασίλευσεν ἐν Ἱερουσαλήμ.
\VS{18}Καὶ ἐπορεύθη ἐν ὁδῷ βασιλέων Ἰσραὴλ καθὼς ἐποίησεν οἶκος Ἀχαὰβ, ὅτι θυγάτηρ Ἀχαὰβ ἦν αὐτῷ εἰς γυναῖκα, καὶ ἐποίησε τὸ πονηρὸν ἐνώπιον Κυρίου.
\VS{19}Καὶ οὐκ ἠθέλησε Κύριος διαφθεῖραι τὸν Ἰούδαν διὰ Δαυὶδ τὸν δοῦλον αὐτοῦ, καθὼς εἶπε δοῦναι αὐτῷ λύχνον καὶ τοῖς υἱοῖς αὐτοῦ πάσας τὰς ἡμέρας.
\par }{\PP \VS{20}Ἐν ταῖς ἡμέραις αὐτοῦ ἠθέτησεν Ἐδὼμ ὑποκάτωθεν χειρὸς Ἰούδα, καὶ ἐβασίλευσαν ἐφʼ ἑαυτὸν βασιλέα.
\VS{21}Καὶ ἀνέβη Ἰωρὰμ εἰς Σιὼρ, καὶ πάντα τὰ ἅρματα τὰ μετʼ αὐτοῦ· καὶ ἐγένετο αὐτοῦ ἀναστάντος, καὶ ἐπάταξε τὸν Ἐδὼμ τὸν κυκλώσαντα ἐπʼ αὐτὸν, καὶ τοὺς ἄρχοντας τῶν ἁρμάτων, καὶ ἔφυγεν ὁ λαὸς εἰς τὰ σκηνώματα αὐτῶν.
\VS{22}Καὶ ἠθέτησεν Ἐδὼμ ὑποκάτω τῆς χειρὸς Ἰούδα ἕως τῆς ἡμέρας ταύτης· τότε ἠθέτησε Λοβνὰ ἐν τῷ καιρῷ ἐκείνῳ.
\par }{\PP \VS{23}Καὶ τὰ λοιπὰ τῶν λόγων Ἰωρὰμ καὶ πάντα ὅσα ἐποίησεν, οὐκ ἰδοὺ ταῦτα γέγραπται ἐπὶ βιβλίῳ λόγων τῶν ἡμερῶν τοῖς βασιλεῦσιν Ἰούδα;
\VS{24}Καὶ ἐκοιμήθη Ἰωρὰμ μετὰ τῶν πατέρων αὐτοῦ, καὶ ἐτάφη μετὰ τῶν πατέρων αὐτοῦ ἐν πόλει Δαυὶδ τοῦ πατρὸς αὐτοῦ· καὶ ἐβασίλευσεν Ὀχοζίας υἱὸς αὐτοῦ ἀντʼ αὐτοῦ.
\par }{\PP \VS{25}Ἐν ἔτει δωδεκάτῳ τῷ Ἰωρὰμ υἱῷ Ἀχαὰβ βασιλεῖ Ἰσραὴλ ἐβασίλευσεν Ὀχοζίας υἱὸς Ἰωράμ.
\VS{26}Υἱὸς εἴκοσι καὶ δύο ἐτῶν Ὀχοζίας ἐν τῷ βασιλεύειν αὐτόν, καὶ ἐνιαυτὸν ἕνα ἐβασίλευσεν ἐν Ἱερουσαλὴμ, καὶ ὄνομα τῆς μητρὸς αὐτοῦ Γοθολία θυγάτηρ Ἀμβρὶ βασιλέως Ἰσραήλ.
\VS{27}Καὶ ἐπορεύθη ἐν ὁδῷ οἴκου Ἀχαὰβ, καὶ ἐποίησε τὸ πονηρὸν ἐνώπιον Κυρίου καθὼς ὁ οἶκος Ἀχαάβ.
\VS{28}Καὶ ἐπορεύθη μετὰ Ἰωρὰμ υἱοῦ Ἀχαὰβ εἰς πόλεμον μετὰ Ἁζαὴλ βασιλέως ἀλλοφύλων ἐν Ῥεμμὼθ Γαλαὰδ, καὶ ἐπάταξαν οἱ Σύροι τὸν Ἰωράμ.
\VS{29}Καὶ ἐπέστρεψεν ὁ βασιλεὺς Ἰωρὰμ τοῦ ἰατρευθῆναι ἐν Ἰεζράελ ἀπὸ τῶν πληγῶν ὧν ἐπάταξαν αὐτὸν ἐν Ῥεμμὼθ, ἐν τῷ πολεμεῖν αὐτὸν μετὰ Ἁζαὴλ βασιλέως Συρίας· καὶ Ὀχοζίας υἱὸς Ἰωρὰμ κατέβη τοῦ ἰδεῖν τὸν Ἰωρὰμ υἱὸν Ἀχαὰβ ἐν Ἰεζράελ, ὅτι ἠῤῥώστει αὐτός.

\par }\Chap{9}{\PP \VerseOne{1}Καὶ Ἐλισαιὲ ὁ προφήτης ἐκάλεσεν ἕνα τῶν υἱῶν τῶν προφητῶν, καὶ εἶπεν αὐτῷ, ζῶσαι τὴν ὀσφύν σου, καὶ λάβε τὸν φακὸν τοῦ ἐλαίου τούτου ἐν τῇ χειρί σου, καὶ δεῦρο εἰς Ῥεμμὼθ Γαλαάδ.
\VS{2}Καὶ εἰσελεύσῃ ἐκεῖ, καὶ ὄψει ἐκεῖ Ἰοὺ υἱὸν Ἰωσαφὰτ υἱοῦ Ναμεσσὶ, καὶ εἰσελεύσῃ καὶ ἀναστήσεις αὐτὸν ἐκ μέσου τῶν ἀδελφῶν αὐτοῦ, καὶ εἰσάξεις αὐτὸν εἰς τὸ ταμεῖον ἐν ταμείῳ.
\VS{3}Καὶ λήψῃ τὸν φακὸν τοῦ ἐλαίου, καὶ ἐπιχεεῖς ἐπὶ τὴν κεφαλὴν αὐτοῦ, καὶ εἰπον, τάδε λέγει Κύριος, κέχρικά σε εἰς βασιλέα ἐπὶ Ἰσραήλ· καὶ ἀνοίξεις τὴν θύραν, καὶ φεύξῃ καὶ οὐ μενεῖς.
\VS{4}Καὶ ἐπορεύθη τὸ προδάριον ὁ παιφήτης εἰς Ῥεμμὼθ Γαλαάδ.
\par }{\PP \VS{5}Καὶ εἰσῆλθε· καὶ ἰδοὺ οἱ ἄρχοντες τῆς δυνάμεως ἐκάθηντο, καὶ εἶπε, λόγος μοι πρὸς σὲ ὁ ἄρχων· καὶ εἶπεν Ἰοὺ, πρὸς τίνα ἐκ πάντων ἡμῶν; καὶ εἶπε, πρὸς σέ ὁ ἄρχων.
\VS{6}Καὶ ἀνέστη καὶ εἰσῆλθεν εἰς τὸν οἶκον, καὶ ἐπέχεε τὸ ἔλαιον ἐπὶ τὴν κεφαλὴν αὐτοῦ, καὶ εἶπεν αὐτῷ, τάδε λέγει Κύριος ὁ Θεὸς Ἰσραήλ, κέχρικά σε εἰς βασιλέα ἐπὶ λαὸν Κυρίου ἐπὶ τὸν Ἰσραὴλ.
\VS{7}Καὶ ἐξολοθρεύσεις τὸν οἶκον Ἀχαὰβ τοῦ κυρίου σου ἐκ προσώπου σου, καὶ ἐκδικὴσεις τὰ αἵματα τῶν δούλων μου τῶν προφητῶν, καὶ τὰ αἵματα πάντων τῶν δούλων Κυρίου ἐκ χειρὸς Ἰεζάβελ,
\VS{8}καὶ ἐκ χειρὸς ὅλου τοῦ οἴκου Ἀχαὰβ, καὶ ἐξολθρεύσεις τῷ οἴκῳ Ἀχαὰβ οὐροῦντα πρὸς τοῖχον, καὶ συνεχόμενον καὶ ἐγκαταλελειμμένον ἐν Ἰσραήλ.
\VS{9}Καὶ δώσω τὸν οἶκον Ἀχαὰβ ὡς τὸν οἶκον Ἰεροβοὰμ υἱοῦ Ναβὰτ, καὶ ὡς τὸν οἶκον Βαασὰ υἱοῦ Ἀχιά.
\VS{10}Καὶ τὴν Ἰεζάβελ καταφάγονται οἱ κύνες ἐν τῇ μερίδι Ἰσζράελ, καὶ οὐκ ἔστιν ὁ θάπτων· καὶ ἤνοιξε τὴν θύραν καὶ ἔφυγε.
\par }{\PP \VS{11}Καὶ Ἰοὺ ἐξῆλθε πρὸς τοὺς παῖδας τοῦ κυρίου αὐτοῦ, καὶ εἶπαν αὐτῷ, εἰρήνη; τί ὅτι εἰσῆλθεν ὁ ἐπίληπτος οὗτος πρὸς σέ; καὶ εἶπεν αὐτοῖς, ὑμεῖς οἴδατε τὸν ἄνδρα καὶ τὴν ἀδολεσχίαν αὐτοῦ.
\VS{12}Καὶ εἶπον, ἄδικον, ἀπάγγειλον δὴ ἡμῖν. Καὶ εἶπεν Ἰοὺ πρὸς αὐτοὺς, οὕτω καὶ οὕτω ἐλάλησε πρὸς μὲ, λέγων, καὶ εἶπε, τάδε λέγει Κύριος, κέχρικά σε εἰς βασιλέα ἐπὶ Ἰσραήλ.
\VS{13}καὶ ἀκούσαντες ἔσπευσαν, καὶ ἔλαβεν ἕκαστος τὸ ἱμάτιον αὐτοῦ, καὶ ἔθηκαν ὑποκάτω αὐτοῦ ἐπὶ τὸ γαρὲμ τῶν ἀναβάθμων· καὶ ἐσάλπισαν ἐν κερατίνῃ, καὶ εἶπαν, ἐβασίλευσεν Ἰού.
\par }{\PP \VS{14}Καὶ συνεστράφη Ἰοὺ υἱὸς Ἰωσαφὰτ υἱοῦ Ναμεσσὶ πρὸς Ἰωράμ· καὶ Ἰωρὰμ αὐτὸς ἐφύλασσεν ἐν Ῥεμμὼθ Γαλαὰδ, καὶ πᾶς Ἰσραὴλ ἀπὸ προσώπου Ἀζαὴλ βασιλέως Συρίας.
\VS{15}Καὶ ἀπέστρεψεν Ἰωρὰμ ὁ βασιλεὺς ἰατρευθῆναι ἐν Ἰσζράελ ἀπὸ τῶν πληγῶν ὧν ἔπαισαν αὐτὸν οἱ Σύροι ἐν τῷ πολεμεῖν αὐτὸς μετὰ Ἀζαὴλ βασιλέως Συρίας.
\par }{\PP Καὶ εἶπεν Ἰοὺ, εἰ ἔστι ψυχὴ ὑμῶν μετʼ ἐμοῦ, μὴ ἐξελθέτω ἐκ τῆς πόλεως διαπεφευγὼς τοῦ πορευθῆναι καὶ ἀπαγγεῖλαι ἐν Ἰεζράελ.
\VS{16}Καὶ ἵππευσε καὶ ἐπορεύθη Ἰοὺ, καὶ κατέβη εἰς Ἰεζράελ, ὅτι Ἰωρὰμ βασιλεὺς Ἰσραὴλ ἐθεραπεύετο ἐν τῷ Ἰεζράελ ἀπὸ τῶν τοξευμάτων, ὧν κατετόξευσαν αὐτὸν οἱ Ἀραμὶν ἐν τῇ Ῥαμμὰθ ἐν τῷ πολέμῳ μετὰ Ἁζαὴλ βασιλέως Συρίας, ὅτι αὐτὸς δυνατὸς καὶ ἀνὴρ δυνάμεως· καὶ Ὀχοζίας βασιλεὺς Ἰούδα κατέβη ἰδεῖν τὸν Ἰωράμ.
\VS{17}Καὶ ὁ σκοπὸς ἀνέβη ἐπὶ τὸν πύργον Ἰεζράελ, καὶ εἶδε τὸν κονιορτὸν Ἰοὺ ἐν τῷ παραγίνεσθαι αὐτὸν, καὶ εἶπε, Κονιορτὸν ἐγὼ βλέπω· καὶ εἶπεν Ἰωρὰμ, λάβε ἐπιβάτην, καὶ ἀπόστειλον ἔμπροσθεν αὐτῶν, καὶ εἰπάτω ἡ εἰρήνη.
\VS{18}Καὶ ἐπορεύθη ἐπιβάτης ἵππου εἰς ἀπαντὴν αὐτῶν, καὶ εἶπε, τάδε λέγει ὁ βασιλεὺς, ἡ εἰρήνη· καὶ εἶπεν Ἰού, τί σοι καὶ εἰρήνῃ; ἐπίστρεφε εἰς τὰ ὀπίσω μου· καὶ ἀπήγγειλεν ὁ σκοπὸς, λέγων, ἦλθεν ὁ ἄγγελος ἕως αὐτῶν, καὶ οὐκ ἀνέστρεψε.
\VS{19}Καὶ ἀπέστειλεν ἐπιβάτην ἵππου δεύτερον, καὶ ἦλθε πρὸς αὐτὸν καὶ εἶπε, τάδε λέγει ὁ βασιλεύς, ἡ εἰρήνη· καὶ εἶπεν Ἰοὺ, τί σοι καὶ εἰρήνῃ; ἐπιστρέφου εἰς τὰ ὀπίσω μου.
\VS{20}Καὶ ἀπήγγειλεν ὁ σκοπὸς, λέγων, ἦλθεν ἕως αὐτῶν καὶ οὐκ ἀνέστρεψε, καὶ ὁ ἄγων ἦγε τὸν Ἰοὺ υἱὸν Ναμεσσὶ, ὅτι ἐν παραλλαγῇ ἐγένετο.
\VS{21}Καὶ εἶπεν Ἰωρὰμ, ζεῦξον· καὶ ἔζευξεν ἅρμα· καὶ ἐξῆλθεν Ἰωρὰμ βασιλεὺς Ἰσραὴλ, καὶ Ὀχοζίας βασιλεὺς Ἰούδα, ἀνὴρ ἐν τῷ ἅρματι αὐτοῦ, καὶ ἐξῆλθον εἰς ἀπαντὴν Ἰοὺ, καὶ εὗρον αὐτὸν ἐν τῇ μερίδι Ναβουθαὶ τοῦ Ἰεζραηλίτου.
\par }{\PP \VS{22}Καὶ ἐγένετο ὡς εἶδεν Ἰωρὰμ τὸν Ἰοὺ, καὶ εἶπεν, ἡ εἰρήνη Ἰοὺ; καὶ εἶπεν Ἰού, τί εἰρήνη; ἔτι αἱ πορνεῖαι Ἰεζάβελ τῆς μητρός σου καὶ τὰ φάρμακα αὐτῆς τὰ πολλά.
\VS{23}Καὶ ἐπέστρεψεν Ἰωρὰμ τὰς χεῖρας αὐτοῦ, καὶ ἔφυγε· καὶ εἶπε πρὸς Ὀχοζίαν, δόλος Ὀχοζία.
\VS{24}Καὶ ἔπλησεν Ἰοὺ τὴν χεῖρα αὐτοῦ ἐν τῷ τόξῳ, καὶ ἐπάταξε τὸν Ἰωρὰμ ἀνὰμέσον τῶν βραχιόνων αὐτοῦ, καὶ ἐξῆλθε τὸ βέλος αὐτοῦ διὰ τῆς καρδίας αὐτοῦ, καὶ ἔκαμψεν ἐπὶ τὰ γόνατα αὐτοῦ.
\VS{25}Καὶ εἶπε πρὸς Βαδεκὰρ τὸν τριστάτην αὐτοῦ, ῥίψον αὐτὸν ἐν τῇ μερίδι ἀγροῦ Ναβουθαὶ τοῦ Ἰεζραηλίτου, ὅτι μνημονεύω ἐγὼ καὶ σὺ ἐπιβεβηκότες ἐπὶ ζεύγη ὀπίσω Ἀχαὰβ τοῦ πατρὸς αὐτοῦ, καὶ Κύριος ἔλαβεν ἐπʼ αὐτὸν τὸ λῆμμα τοῦτο·
\VS{26}Εἰ μὴ τὰ αἵματα Ναβουθαὶ καὶ τὰ αἵματα τῶν υἱῶν αὐτοῦ εἶδον ἐχθὲς, φησὶ Κύριος, καὶ ἀνταποδώσω αὐτῷ ἐν τῇ μερίδι ταύτῃ, φησὶ Κύριος· καὶ νῦν ἄρας δὴ ῥίψον αὐτὸν ἐν τῇ μερίδι κατὰ τὸ ῥῆμα Κυρίου.
\par }{\PP \VS{27}Καὶ Ὀχοζίας βασιλεὺς Ἰούδα εἶδε καὶ ἔφυγεν ὁδὸν Βαιθγάν· καὶ ἐδίωξεν ὀπίσω αὐτοῦ Ἰοὺ, καὶ εἶπε, καί γε αὐτόν· καὶ ἐπάταξεν αὐτὸν πρὸς τῷ ἅρματι ἐν τῷ ἀναβαίνειν Γαῒ, ἥ ἐστιν Ἰεβλαάμ· καὶ ἔφυγεν εἰς Μαγεδδὼ, καὶ ἀπέθανεν ἐκεῖ.
\VS{28}Καὶ ἐπεβίβασαν αὐτὸν οἱ παῖδες αὐτοῦ ἐπὶ τὸ ἅρμα, καὶ ἤγαγον αὐτὸν εἰς Ἱερουσαλὴμ, καὶ ἔθαψαν αὐτὸν ἐν τῷ τάφῳ αὐτοῦ ἐν πόλει Δαυίδ.
\par }{\PP \VS{29}Καὶ ἐν ἔτει ἑνδεκάτῳ Ἰωρὰμ βασιλέως Ἰσραὴλ ἐβασίλευσεν Ὀχοζίας ἐπὶ Ἰούδαν.
\par }{\PP \VS{30}Καὶ ἦλθεν Ἰοὺ ἐπὶ Ἰεζράελ· καὶ Ἰεζάβελ ἤκουσε, καὶ ἐστιμμίσατο τοὺς ὀφθαλμοὺς αὐτῆς, καὶ ἠγάθυνε τὴν κεφαλὴν αὐτῆς, καὶ διέκυψε διὰ τῆς θυρίδος.
\VS{31}Καὶ Ἰοὺ εἰσεπορεύετο ἐν τῇ πόλει, καὶ εἶπε, ἡ εἰρήνη Ζαμβρὶ ὁ φονευτὴς τοῦ κυρίου αὐτοῦ;
\VS{32}Καὶ ἐπῇρε τὸ πρόσωπον αὐτοῦ εἰς τὴν θυρίδα, καὶ εἶδεν αὐτὴν, καὶ εἶπε, τίς εἶ σύ; κατάβηθι μετʼ ἐμοῦ· καὶ κατέκυψαν πρὸς αὐτὸν δύο εὐνοῦχοι.
\VS{33}Καὶ εἶπε, κυλίσατε αὐτήν· καὶ ἐκύλισαν αὐτὴν, καὶ ἐῤῥαντίσθη τοῦ αἵματος αὐτῆς πρὸς τὸν τοῖχον καὶ πρὸς τοὺς ἵππους, καὶ συνεπάτησαν αὐτήν.
\VS{34}Καὶ εἰσῆλθε καὶ ἔφαγε καὶ ἔπιεν, καὶ εἶπε, ἐπισκέψασθε δὴ τὴν κατηραμένην ταύτην, καὶ θάψατε αὐτὴν, ὅτι θυλάτηρ βασιλέως ἐστί.
\VS{35}Καὶ ἐπορεύθησαν θάψαι αὐτὴν, καὶ οὐχ εὗρον ἐν αὐτῇ ἄλλο τι ἢ τὸ κρανίον καὶ οἱ πόδες καὶ τὰ ἴχνη τῶν χειρῶν.
\VS{36}Καὶ ἐπέστρεψαν καὶ ἀνήγγειλαν αὐτῷ· καὶ εἶπε, λόγος Κυρίου ὃν ἐλάλησεν ἐν χειρὶ Ἠλιοὺ τοῦ Θεσβίτου, λέγων, ἐν τῇ μερίδι Ἰεζρὰελ καταφάγονται οἱ κύνες τὰς σάρκας Ἰεζάβελ.
\VS{37}Καὶ ἔσται τὸ θνησιμαῖον Ἰεζάβελ ὡς κοπρία ἐπὶ προσώπου τοῦ ἀγροῦ ἐν τῇ μερίδι Ἰεζραλ, ὥστε μὴ εἰπεῖν αὐτοὺς, Ἰεζάβελ.

\par }\Chap{10}{\PP \VerseOne{1}Καὶ τῷ Ἀχαὰβ ἑβδομήκοντα υἱοὶ ἐν Σαμαρείᾳ· καὶ ἔγραψεν Ἰοὺ βιβλίον, καὶ ἀπέστειλεν ἐν Σαμαρείᾳ πρὸς τοὺς ἄρχοντας Σαμαρείας, καὶ πρὸς τοὺς πρεσβυτέρους, καὶ πρὸς τοὺς τιθηνοὺς Αχαὰβ, λέγων,
\VS{2}Καὶ νῦν ὡς ἂν ἔλθῃ τὸ βιβλίον τοῦτο πρὸς ὑμᾶς, καὶ μεθʼ ὑμῶν οἱ υἱοὶ τοῦ κυρίου ὑμῶν, καὶ μεθʼ ὑμῶν τὸ ἅρμα καὶ οἱ ἵπποι καὶ πόλεις ὀχυραὶ καὶ τὰ ὅπλα,
\VS{3}καὶ ὄψεσθε τὸν ἀγαθὸν καὶ τὸν εὐθῆ ἐν τοῖς υἱοῖς τοῦ κυρίου ὑμῶν, καὶ καταστήσετε αὐτὸν ἐπὶ τὸν θρόνον τοῦ πατρὸς αὐτοῦ, καὶ πολεμεῖτε ὑπὲρ τοῦ οἴκου τοῦ κυρίου ὑμῶν.
\VS{4}Καὶ ἐφοβήθησαν σφόδρα, καὶ εἶπον, ἰδοὺ οἱ δύο βασιλεῖς οὐκ ἔστησαν κατὰ πρόσωπον αὐτοῦ, καὶ πῶς στησόμεθα ἡμεῖς;
\VS{5}Καὶ ἀπέστειλαν οἱ ἐπὶ τοῦ οἴκου καὶ οἱ ἐπὶ τῆς πόλεως καὶ οἱ πρεσβύτεροι καὶ οἱ τιθηνοὶ πρὸς Ἰοὺ, λέγοντες, παῖδές σου καὶ ἡμεῖς, καὶ ὅσα ἐὰν εἴπῃς πρὸς ἡμᾶς ποιήσομεν· οὐ βασιλεύσομεν ἄνδρα, τὸ ἀγαθὸν ἐν ὀφθαλμοῖς σου ποιήσομεν.
\par }{\PP \VS{6}Καὶ ἔγραψε πρὸς αὐτοὺς Ἰοὺ βιβλίον δεύτερον, λέγων, εἰ ἐμοὶ ὑμεῖς, καὶ τῆς φωνῆς μου ὑμεῖς εἰσακούετε, λάβετε τὴν κεφαλὴν ἀνδρῶν τῶν υἱῶν τοῦ κυρίου ὑμῶν, καὶ ἐνέγκατε πρὸς μὲ, ὡς ἡ ὥρα αὔριον ἐν Ἰεζράελ· καὶ οἱ υἱοὶ τοῦ βασιλέως ἦσαν ἑβδομήκοντα ἄνδρες, οὗτοι ἁδροὶ τῆς πόλεως ἐξέτρεφον αὐτούς.
\VS{7}Καὶ ἐγένετο ὡς ἦλθε τὸ βιβλίον πρὸς αὐτοὺς, καὶ ἔλαβον τοὺς υἱοὺς τοῦ βασιλέως, καὶ ἔσφαξαν αὐτοὺς ἑβδομήκοντα ἄνδρας· καὶ ἔθηκαν τὰς κεφαλὰς αὐτῶν ἐν καρτάλλοις, καὶ ἀπέστειλαν αὐτὰς πρὸς αὐτὸν εἰς Ἰεζράελ.
\VS{8}Καὶ ἦλθεν ὁ ἄγγελος καὶ ἀπήγγειλε, λέγων, ἤνεγκαν τὰς κεφαλὰς τῶν υἱῶν τοῦ βασιλέως· καὶ εἶπε, θέτε αὐτὰς βουνοὺς δύο παρὰ τὴν θύραν τῆς πύλης εἰς πρωΐ.
\VS{9}Καὶ ἐγένετο πρωΐ καὶ ἐξῆλθε καὶ ἔστη, καὶ εἶπε πρὸς πάντα τὸν λαόν, δίκαιοι ὑμεῖς· ἰδοὺ ἐγώ εἰμι συνεστράφην ἐπὶ τὸν κύριόν μου, καὶ ἀπέκτεινα αὐτόν· καὶ τίς ἐπάταξε πάντας τούτους;
\VS{10}Ἴδετε ἀφφὼ, ὅτι οὐ πεσεῖται ἀπὸ τοῦ ῥήματος Κυρίου εἰς τὴν γῆν οὗ ἐλάλησε Κύριος ἐπὶ τὸν οἶκον Ἀχαάβ· καὶ Κύριος ἐποίησεν ὅσα ἐλάλησεν ἐν χειρὶ δούλου αὐτοῦ Ἠλιού.
\VS{11}Καὶ ἐπάταξεν Ἰοὺ πάντας τοὺς καταλειφθέντας ἐν τῷ οἴκῳ Ἀχαὰβ ἐν Ἰεζράελ, καὶ πάντας τοὺς ἁδροὺς αὐτοῦ, καὶ τοὺς γνωστοὺς αὐτοῦ, καὶ τοὺς ἱερεῖς αὐτοῦ, ὥστε μὴ καταλιπεῖν αὐτοὺ κατάλειμμα.
\par }{\PP \VS{12}Καὶ ἀνέστη καὶ ἐπορεύθῃ εἰς Σαμάρειαν, αὐτὸς ἐν βαιθακὰθ τῶν ποιμένων ἐν τῇ ὁδῷ.
\VS{13}Καὶ Ἰοὺ εὗρε τοὺς ἀδελφοὺς Ὀχοζίου βασιλέως Ἰούδα, καὶ εἶπε, τίνες ὑμεῖς; καὶ εἶπον, ἀδελφοὶ Ὀχοζίου ἡμεῖς, καὶ κατέβημεν εἰς εἰρήνην τῶν υἱῶν τοῦ βασιλέως, καὶ τῶν υἱῶν τῆς δυναστευούσης.
\VS{14}Καὶ εἶπε, συλλάβετε αὐτοὺς ζῶντας· καὶ ἔσφαξαν αὐτοὺς εἰς βαιθακὰθ τεσσαράκοντα καὶ δύο ἄνδρας· οὐ κατέλιπεν ἄνδρα ἐξ αὐτῶν.
\par }{\PP \VS{15}Καὶ ἐπορεύθη ἐκεῖθεν καὶ εὗρε τὸν Ἰωναδὰβ υἱὸν Ῥηχὰβ εἰς ἀπαντὴν αὐτοῦ, καὶ εὐλόγησεν αὐτόν· καὶ εἶπε πρὸς αὐτὸν Ιοὺ, εἰ ἔστι καρδία σου μετὰ καρδίας μου εὐθεῖα καθὼς ἡ καρδία μου μετὰ τῆς καρδίας σου; καὶ εἶπεν Ἰωναδάβ ἔστι· καὶ εἶπεν Ἰοὺ, καὶ εἰ ἔστι, δὸς τὴν χεῖρά σου· καὶ ἔδωκε τὴν χεῖρα αὐτοῦ· καὶ ἀνεβίβασεν αὐτὸν πρὸς αὐτὸν ἐπὶ τὸ ἅρμα,
\VS{16}καὶ εἶπε πρὸς αὐτὸν, δεῦρο μετʼ ἐμοῦ, καὶ ἴδε ἐν τῷ ζηλῶσαί με τῷ Κυρίῳ· καὶ ἐπεκάθισε αὐτὸν ἐν τῷ ἅρματι αὐτοῦ.
\par }{\PP \VS{17}Καὶ εἰσῆλθεν εἰς Σαμάρειαν· καὶ ἐπάταξε πάντας τοὺς καταλειφθέντας τοῦ Ἀχαὰβ ἐν Σαμαρείᾳ ἕως τοῦ ἀφανίσαι αὐτὸν κατὰ τὸ ῥῆμα Κυρίου, ὃ ἐλάλησε πρὸς Ἠλιού.
\VS{18}Καὶ συνήθροισεν Ἰοὺ πάντα τὸν λαὸν, καὶ εἶπε πρὸς αὐτοὺς, Ἀχαὰβ ἐδούλευσε τῷ Βάαλ ὀλίγα, Ἰοὺ δουλεύσει αὐτῷ πολλά.
\VS{19}Καὶ νῦν πάντες οἱ προφῆται τοῦ Βάαλ πάντας τοὺς δούλους αὐτοῦ καὶ τοὺς ἱερεῖς αὐτοῦ καλέσατε πρὸς μὲ, ἀνὴρ μὴ ἐπισκεπήτω, ὅτι θυσία μεγάλη μοι τῷ Βάαλ· πᾶς ὃς ἐὰν ἐπισκεπῇ, οὐ ζήσεται· καὶ Ἰοὺ ἐποίησεν ἐν πτερνισμῷ, ἵνʼ ἀπολέσῃ τοὺς δούλους τοῦ Βάαλ.
\par }{\PP \VS{20}Καὶ εἶπεν Ἰοὺ, ἁγιάσατε ἱερείαν τῷ Βάαλ· καὶ ἐκήρυξν.
\VS{21}Καὶ ἀπέστειλεν Ἰοὺ ἐν παντὶ Ἰσραὴλ, λέγων, καὶ νῦν πάντες οἱ δοῦλοι, καὶ πάντες οἱ ἱερεῖς αὐτοῦ, καὶ πάντες οἱ προθῆται αὐτοῦ, μηδεὶς ἀπολιπέσθω, ὅτι θυσίαν μεγάλην ποιῶ· ὃς ἂν ἀπολειφθῇ, οὐ ζήσεται· καὶ ἦλθον πάντες οἱ δοῦλοι τοῦ Βάαλ, καὶ πάντες οἱ ἱερεῖς αὐτοῦ, καὶ πάντες οἱ προφῆται αὐτοῦ· οὐ κατελείφθη ἀνὴρ ὃς οὐ παρεγένετο· καὶ εἰσῆλθον εἰς τὸν οἶκον τοῦ Βάαλ· καὶ ἐπλήσθη ὁ οἶκος τοῦ Βάαλ στόμα εἰς στόμα.
\VS{22}Καὶ εἶπε τῷ ἐπὶ τοῦ οἴκου μεσθάαλ, ἐξάγαγε ἔνδυμα πᾶσι τοῖς δούλοις τοῦ Βάαλ· καὶ ἐξήνεγκεν αὐτοῖς ὁ στολιστής.
\VS{23}Καὶ εἰσῆλθεν Ἰοὺ καὶ Ἰωναδὰβ υἱὸς Ῥηχὰβ εἰς οἶκον τοῦ Βάαλ, καὶ εἶπε τοῖς δούλοις τοῦ Βάαλ, ἐρευνήσατε καὶ ἴδετε, εἰ ἔστι μεθʼ ὑμῶν τῶν δούλων Κυρίου, ὅτι ἀλλʼ ἢ οἱ δοῦλοι τοῦ Βάαλ μονώτατοι.
\VS{24}Καὶ εἰσῆλθε τοῦ ποιῆσαι τὰ θύματα καὶ τὰ ὁλοκαυτώματα· καὶ Ἰοὺ ἔταξεν ἑαυτῷ ἔξω ὀγδοήκοντα ἄνδρας, καὶ εἶπεν, ἀνὴρ ὃς ἐὰν διασωθῇ ἀπὸ τῶν ἀνδρῶν ὧν ἐγὼ ἀνάγω ἐπὶ χεῖρα ὑμῶν, ἡ ψυχὴ αὐτοῦ ἀντὶ τῆς ψυχῆς αὐτοῦ.
\par }{\PP \VS{25}Καὶ ἐγένετο ὡς συνετέλεσε ποιῶν τὴν ὁλοκαύτωσιν, καὶ εἶπεν Ἰοὺ τοῖς παρατρέχουσιν καὶ τοῖς τριστάταις, εἰσελθόντες πατάξατε αὐτοὺς, μὴ ἐξελθάτω ἐξ αὐτῶν ἀνήρ· καὶ ἐπάταξαν αὐτοὺς ἐν στόματι ῥομφαίας, καὶ ἔῤῥιψαν οἱ παρατρέχοντες καὶ οἱ τριστάται, καὶ ἐπορεύθησαν ἕως πόλεως οἴκου τοῦ Βάαλ.
\VS{26}Καὶ ἐξήνεγκαν τὴν στήλην τοῦ Βάαλ, καὶ ἐνέπρησαν αὐτήν.
\VS{27}Καὶ κατέσπασαν τὰς στήλας τοῦ Βάαλ, καὶ ἔπαξαν αὐτὸν εἰς λυτρῶνας ἕως τῆς ἡμέρας ταύτης.
\VS{28}Καὶ ἠφάνισεν Ἰοὺ τὸν Βάαλ ἐξ Ἰσραήλ.
\par }{\PP \VS{29}Πλὴν ἁμαρτιῶν Ἰεροβοὰμ υἱοῦ Ναβὰτ ὃς ἐξήμαρτε τὸν Ἰσραὴλ, οὐκ ἀπέστη Ἰοὺ ἀπὸ ὄπισθεν αὐτῶν· αἱ δαμάλεις αἱ χρυσαῖ ἐν Βαιθὴλ, καὶ ἐν Δάν.
\par }{\PP \VS{30}Καὶ εἶπε Κύριος πρὸς Ἰοὺ, ἀνθʼ ὧν ὅσα ἠγάθυνας ποιῆσαι τὸ εὐθὲς ἐν ὀφθαλμοῖς μου κατὰ πάντα ὅσα ἐν τῇ καρδίᾳ μου ἐποίησας τῷ οἴκῳ Ἀχαὰβ, υἱοὶ τέταρτοι καθήσονταί σοι ἐπὶ θρόνου Ἰσραήλ.
\VS{31}Καὶ Ἰοὺ οὐκ ἐφύλαξε πορεύεσθαι ἐν νόμῳ Κυρίου θεοῦ Ἰσραὴλ ἐν ὅλῃ καρδίᾳ αὐτοῦ· οὐκ ἀπέστη ἑπάνωθεν ἁμαρτιῶν Ἰεροβοὰμ ὃς ἐξήμαρτε τὸν Ἰσραήλ.
\VS{32}Ἐν ταῖς ἡμέραις ἐκείναις ἤρξατο Κύριος συγκόπτειν ἐν τῷ Ἰσραήλ· καὶ ἐπάταξεν αὐτοὺς Ἀζαὴλ ἐν παντὶ ὁρίῳ Ἰσραὴλ,
\VS{33}ἀπὸ τοῦ Ἰορδάνου κατʼ ἀνατολὰς ἡλίου πᾶσαν τὴν Γαλαάδ τοῦ Γαδδὶ, καὶ τοῦ Ῥουβὴν, καὶ τοῦ Μανασσῆ, ἀπὸ Ἀροὴρ, ἥ ἐστιν ἐπὶ τοῦ χείλους χειμάῤῥου Ἀρνὼν, καὶ τὴν Γαλαὰδ καὶ τὴν Βασάν.
\par }{\PP \VS{34}Καὶ τὰ λοιπὰ τῶν λόγων Ἰοὺ καὶ πάντα ὅσα ἐποίησε, καὶ πᾶσα ἡ δυναστεία αὐτοῦ, καὶ τὰς συνάψεις ἃς συνῆψεν, οὐχὶ ταῦτα γεγραμμένα ἐπὶ βιβλίου λόγων τῶν ἡμερῶν τοῖς βασιλεῦσιν Ἰσραήλ;
\VS{35}Καὶ ἐκοιμήθη Ἰοὺ μετὰ τῶν πατέρων αὐτοῦ, καὶ ἔθαψαν αὐτὸν ἐν Σαμαρείᾳ· καὶ ἐβασίλευσεν Ἰωάχαζ υἱὸς αὐτοῦ ἀντʼ αὐτοῦ.
\VS{36}Καὶ αἱ ἡμέραι ἃς ἐβασίλευσεν Ἰοὺ ἐπὶ Ἰσραὴλ εἰκοσιοκτὼ ἔτη ἐν Σαμαρείᾳ.

\par }\Chap{11}{\PP \VerseOne{1}Καὶ Γοθολία ἡ μήτηρ Ὀχοζίου εἶδεν ὅτι ἀπέθανεν ὁ υἱὸς αὐτῆς, καὶ ἀπώλεσε πᾶν τὸ σπέρμα τῆς βασιλείας.
\VS{2}Καὶ ἔλαβεν Ἰωσαβεὲ θυγάτηρ τοῦ βασιλέως Ἰωρὰμ ἀδελφὴ Ὀχοζίου τὸν Ἰωὰς υἱὸν ἀδελφοῦ αὐτῆς, καὶ ἔκλεψεν αὐτὸν ἐκ μέσου τῶν υἱῶν τοῦ βασιλέως τῶν θανατουμένων, αὐτὸν καὶ τὴν τροφὸν αὐτοῦ ἐν τῷ ταμείῳ τῶν κλινῶν, καὶ ἔκρυψεν αὐτὸν ἀπὸ προσώπου Γοθολίας, καὶ οὐκ ἐθανατώθη.
\VS{3}Καὶ ἦν μετʼ αὐτῆς κρυβόμενος ἐν οἴκῳ Κυρίου ἓξ ἔτη· καὶ Γοθολία βασιλεύουσα ἐπὶ τῆς γῆς.
\par }{\PP \VS{4}Καὶ ἐν τῷ ἔτει τῷ ἑβδόμῳ ἀπέστειλεν Ἰωδαὲ, καὶ ἔλαβε τοὺς ἑκατοντάρχους τῶν Χοῤῥὶ καὶ τῶν Ῥασίμ, καὶ ἀπήγαγεν αὐτοὺς πρὸς αὐτὸν εἰς οἶκον Κυρίου, καὶ διέθετο αὐτοῖς διαθήκην Κυρίου, καὶ ὥρκωσε· καὶ ἔδειξεν αὐτοῖς Ἰωδαὲ τὸν υἱὸν τοῦ βασιλέως,
\VS{5}καὶ ἐνετείλατο αὐτοῖς, λέγων, οὗτος ὁ λόγος ὃν ποιήσετε· Τὸ τρίτον ἐξ ὑμῶν εἰσελθέτω τὸ σάββατον,
\VS{6}καὶ φυλάξατε φυλακὴν οἴκου τοῦ βασιλέως ἐν τῷ πυλῶνι, καὶ τὸ τρίτον ἐν τῇ πύλῃ τῶν ὁδῶν, καὶ τὸ τρίτον τῆς πύλης ὀπίσω τῶν παρατρεχόντων, καὶ φυλάξτε τὴν φυλακὴν τοῦ οἴκου.
\VS{7}Καὶ δύο χεῖρες ἐν ὑμῖν, πᾶς ὁ ἐκπορευόμενος τὸ σάββατον, καὶ φυλάξουσι τὴν φυλακὴν οἴκου Κυρίου πρὸς τὸν βασιλέα.
\VS{8}Καὶ κυκλώσατε ἐπὶ τὸν βασιλέα κύκλῳ, ἀνὴρ καὶ τὸ σκεῦος αὐτοῦ ἐν χειρὶ αὐτοῦ, καὶ ὁ εἰσπορευόμενος εἰς τὰς σαδηρὼθ, ἀποθανεῖται· καὶ ἔσονται μετὰ τοῦ βασιλέως ἐν τῷ ἐκπορεύεσθαι αὐτὸν καὶ ἐν τῷ εἰσπορεύεσθαι αὐτόν.
\par }{\PP \VS{9}Καὶ ἐποίησαν οἱ ἑκατόνταρχοι πάντα ὅσα ἐνετείλατο Ἰωδαὲ ὁ συνετός· καὶ ἔλαβεν ἀνὴρ τοὺς ἄνδρας αὐτοῦ καὶ τοὺς εἰσπορευομένους τὸ σάββατον μετὰ τῶν ἐκπορευομένων τὸ σάββατον, καὶ εἰσῆλθον πρὸς Ἰωδαὲ τὸν ἱερέα.
\VS{10}Καὶ ἔδωκεν ὁ ἱερεὺς τοῖς ἑκατοντάρχοις τοὺς σειρομάστας καὶ τοὺς τρισσοὺς τοῦ βασιλέως Δαυὶδ τοὺς ἐν οἴκῳ Κυρίου.
\VS{11}Καὶ ἔστησαν οἱ παρατρέχοντες ἀνὴρ καὶ τὸ σκεῦος αὐτοῦ, ἐν τῇ χειρὶ αὐτοῦ ἀπὸ τῆς ὠμίας τοῦ οἴκου τῆς δεξιᾶς ἕως τῆς ὠμίας τοῦ οἴκου τῆς εὐωνύμου τοῦ θυσιαστηρίου καὶ τοῦ οἴκου, ἐπὶ τὸν βασιλέα κύκλῳ.
\VS{12}Καὶ ἐξαπέστειλε τὸν υἱὸν τοῦ βασιλέως, καὶ ἔδωκεν ἐπʼ αὐτὸν νεζὲρ καὶ τὸ μαρτύριον, καὶ ἐβασίλευσεν αὐτὸν καὶ ἔχρισεν αὐτόν· καὶ ἐκρότησαν τῇ χειρὶ, καὶ εἶπαν, ζήτω ὁ βασιλεύς.
\par }{\PP \VS{13}Καὶ ἤκουσε Γοθολία τὴν φωνὴν τῶν τρεχόντων τοῦ λαοῦ, καὶ εἰσῆλθε πρὸς τὸν λαὸν εἰς οἶκον Κυρίου,
\VS{14}καὶ εἶδε, καὶ ἰδοὺ ὁ βασιλεὺς εἱστήκει ἐπὶ τοῦ στύλου κατὰ τὸ κρίμα· καὶ οἱ ᾠδοὶ καὶ αἱ σάλπιγγες πρὸς τὸν βασιλέα, καὶ πᾶς ὁ λαὸς τῆς γῆς χαίρων καὶ σαλπίζων ἐν σάλπιγξι· καὶ διέῤῥηξε Γοθολία τὰ ἱμάτια ἑαυτῆς, καὶ ἐβόησε σύνδεσμος, σύνδεσμος.
\VS{15}Καὶ ἐνετείλατο Ἰωδαὲ ὁ ἱερεὺς τοῖς ἑκατοντάρχοις τοῖς ἐπισκόποις τῆς δυνάμεως, καὶ εἶπε πρὸς αὐτούς, ἐξαγάγετε αὐτὴν ἔσωθεν τῶν σαδηρὼθ, ὁ εἰσπορευόμενος ὀπίσω αὐτῆς θανάτῳ θανατωθήσεται ἐν ῥομφαίᾳ· ὅτι εἶπεν ὁ ἱερεύς, καὶ μὴ ἀποθάνῃ ἐν οἴκῳ Κυρίου.
\VS{16}Καὶ ἐπέθηκαν αὑτῇ χεῖρας, καὶ εἰσῆλθον ὁδὸν εἰσόδου τῶν ἵππων οἴκου τοῦ βασιλέως, καὶ ἀπέθανεν ἐκεῖ.
\par }{\PP \VS{17}Καὶ διέθετο Ἰωδαὲ διαθήκην ἀναμέσον Κυρίου καὶ ἀναμέσον τοῦ βασιλέως καὶ ἀναμέσον τοῦ λαοῦ, τοῦ εἶναι εἰς λαὸν τῷ Κυρίῳ· καὶ ἀναμέσον τοῦ βασιλέως καὶ ἀναμέσον τοῦ λαοῦ.
\VS{18}Καὶ εἰσῆλθε πᾶς ὁ λαὸς τῆς γῆς εἰς οἶκον τοῦ Βάαλ, καὶ κατέσπασαν αὐτὸν, καὶ τὰ θυσιαστήρια αὐτοῦ καὶ τὰς εἰκόνας αὐτοῦ συνέτριψαν ἀγαθῶς· καὶ τὸν Μαθὰν τὸν ἱερέα τοῦ Βάαλ ἀπέκτειναν κατὰ πρόσωπον τῶν θυσιαστηρίων· καὶ ἔθηκεν ὁ ἱερεὺς ἐπισκόπους εἰς τὸν οἶκον Κυρίου.
\VS{19}Καὶ ἔλαβε τοὺς ἑκατοντάρχους, καὶ τὸν Χοῤῥεὶ, καὶ τὸν Ῥασὶμ, καὶ πάντα τὸν λαὸν τῆς γῆς, καὶ κατήγαγον τὸν βασιλέα ἐξ οἴκου Κυρίου· καὶ εἰσῆλθον ὁδὸν πύλης τῶν παρατρεχόντων οἴκου τοῦ βασιλέως, καὶ ἐκάθισαν αὐτὸν ἐπὶ θρόνου τῶν βασιλέων.
\VS{20}Καὶ ἐχάρη πᾶς ὁ λαὸς τῆς γῆς, καὶ ἡ πόλις ἡσύχασε· καὶ τὴν Γοθολίαν ἐθανάτωσαν ἐν ῥομφαίᾳ ἐν οἴκῳ τοῦ βασιλέως.

\par }\Chap{12}{\PP \VerseOne{1}Υἱὸς ἐπτὰ ἐτῶν Ἰωὰς ἐν τῷ βασιλεύειν αὐτόν.
\par }{\PP \VS{2}Ἐν ἔτει ἑβδόμῳ τῷ Ἰοὺ ἐβασίλευσεν Ἰωὰς, καὶ τεσσαράκοντα ἔτη ἐβασίλευσεν ἐν Ἱερουσαλὴμ, καὶ ὄνομα τῆς μητρὸς αὐτοῦ Σαβιὰ ἐκ τῆς Βηρσαβεέ.
\VS{3}Καὶ ἐποίησεν Ἰωὰς τὸ εὐθὲς ἐνώπιον Κυρίου πάσας τὰς ἡμέρας ἃς ἐφώτισεν αὐτὸν Ἰωδαὲ ὁ ἱερεύς.
\VS{4}Πλὴν τῶν ὑψηλῶν οὐ μετεστάθησαν, καὶ ἐκεῖ ἔτι ὁ λαὸς ἐθυσίαζε, καὶ ἐθυμίων ἐν τοῖς ὑψηλοῖς.
\par }{\PP \VS{5}Καὶ εἶπεν Ἰωὰς πρὸς τοὺς ἱερεῖς, πᾶν τὸ ἀργύριον τῶν ἁγίων τὸ εἰσοδιαζόμενον ἐν τῷ οἴκῳ Κυρίου, ἀργύριον συντιμήσεως, ἀνὴρ ἀργύριον λαβὼν συντιμήσεως, πᾶν ἀργύριον ὃ ἐὰν ἀναβῇ ἐπὶ καρδίαν ἀνδρὸς ἐνεγκεῖν ἐν οἴκῳ Κυρίου,
\VS{6}λαβέτωσαν ἑαυτοῖς οἱ ἱερεῖς, ἀνὴρ ἀπὸ τῆς πράσεως αὐτοῦ, καὶ αὐτοὶ κρατήσουσι τὸ βεδὲκ τοῦ οἴκου εἰς πάντα οὗ ἐὰν εὑρεθῇ ἐκεῖ βεδέκ.
\par }{\PP \VS{7}Καὶ ἐγενήθη ἐν τῷ εἰκοστῷ καὶ τρίτῳ ἔτει τῷ βασιλεῖ Ἰωὰς οὐκ ἐκραταίωσαν οἱ ἱερεῖς τὸ βεδέκ τοῦ οἴκου.
\VS{8}Καὶ ἐκάλεσεν Ἰωὰς ὁ βασιλεὺς Ἰωδαὲ τὸν ἱερέα καὶ τοὺς ἱερεῖς, καὶ εἶπε πρὸς αὐτούς, τί ὅτι οὐκ ἐκραταιοῦτε τὸ βεδὲκ τοῦ οἴκου; καὶ νῦν μὴ λάβητε ἀργύριον ἀπὸ τῶν πράσεων ὑμῶν, ὅτι εἰς τὸ βεδὲκ τοῦ οἴκου δώσετε αὐτό.
\VS{9}Καὶ συνεφώνησαν οἱ ἱερεῖς τοῦ μὴ λαβεῖν ἀργύριον παρὰ τοῦ λαοῦ, καὶ τοῦ μὴ ἐνισχῦσαι τὸ βεδὲκ τοῦ οἴκου.
\VS{10}Καὶ ἔλαβεν Ἰωδαὲ ὁ ἱερεὺς κιβωτὸν μίαν, καὶ ἔτρησε τρώγλην ἐπὶ τῇς σανίδος αὐτῆς, καὶ ἔδωκεν αὐτὴν παρὰ ἁμμαζειβὶ ἐν τῷ οἴκῳ ἀδρος οἴκου Κυρίου· και ἔδωκας οἱ ἱερεῖς οἱ ἱεπεῖς φυλάσσοντες τὸν σταθμὸν πᾶν τὸ ἀργύριον τὸ εὑρεθὲν ἐν οἴκῳ Κυρίου·
\par }{\PP \VS{11}Καὶ ἐγένετο ὡς εἶδον ὅτι πολὺ τὸ ἀργύριον ἐν τῇ κιβωτῷ, καὶ ἀνέβη ὁ γραμματεὺς τοῦ βασιλέως καὶ ὁ ἱερεὺς ὁ μέγας, καὶ ἔσθιγξαν καὶ ἠρίθμησαν τὸ ἀργύριον τὸ εὑρεθὲν ἐν οἴκῳ Κυρίου.
\VS{12}Καὶ ἔδωκαν τὸ ἀργύριον τὸ ἑτοιμασθὲν ἐπὶ χεῖρας ποιούντων τὰ ἔργα τῶν ἐπισκοπῶν οἴκου· Κυρίου, καὶ ἐξέδοσαν τοῖς τέκτοσι τῶν ξύλων, καὶ τοῖς οἱκοδόμοις τοῖς ποιοῦσιν ἐν οἴκῳ Κυρίου,
\VS{13}καὶ τοῖς τειχισταῖς, καὶ τοῖς λατόμοις τῶν λίθων τοῦ κτήσασθαι ξύλα καὶ λίθους λατομητοὺς τοῦ κατασχεῖν τὸ βεδὲκ οἴκου Κυρίου, εἰς πάντα ὅσα ἐξωδιάσθη ἐπὶ τὸν οἶκον τοῦ κραταιῶσαι.
\VS{14}Πλὴν οὐ ποιηθήσονται οἴκῳ Κυρίου θύραι ἀργυραῖ, ἧλοι, φιάλαι, καὶ σάλπιγγες, πᾶν σκεῦος χρυσοῦν, καὶ σκεῦος ἀργυροῦν, ἐκ τοῦ ἀργυρίου τοῦ εἰσενεχθέντος ἐν οἴκῳ Κυρίου,
\VS{15}ὅτι τοῖς ποιοῦσι τὰ ἔργα δώσουσιν αὐτό· καὶ ἐκραταίωσαν ἐν αὐτῷ τὸν οἶκον Κυρίου.
\VS{16}Καὶ οὐκ ἐξελογίζοντο τοὺς ἄνδρας οἷς ἐδίδουν τὸ ἀργύριον ἐπὶ χεῖρας αὐτῶν δοῦναι τοῖς ποιοῦσι τὰ ἔργα, ὅτι ἐν πίστει αὐτῶν ποιοῦσιν.
\VS{17}Ἀργύριον περὶ ἁμαρτίας, καὶ ἀργύριον περὶ πλημμελείας, ὅ, τι εἰσηνέχθη ἐν οἴκῳ Κυρίου, τοῖς ἱερεῦσιν ἐγένετο.
\par }{\PP \VS{18}Τότε ἀνέβη Ἀζαὴλ βασιλεὺς Συρίας, καὶ ἐπολέμησεν ἐπὶ Γὲθ, καὶ προκατελάβετο αὐτήν· καὶ ἔταξεν Ἀζαὴλ τὸ πρόσωπον αὐτοῦ ἀναβῆναι ἐπὶ Ἱερουσαλήμ.
\VS{19}Καὶ ἔλαβεν Ἰωὰς βασιλεὺς Ἰούδα πάντα τὰ ἅγια ὅσα ἡγίασεν Ἰωσαφὰτ καὶ Ἰωρὰμ καὶ Ὀχοζίας οἱ πατέρες αὐτοῦ καὶ βασιλεῖς Ἰούδα, καὶ τὰ ἅγια αὐτοῦ, καὶ πᾶν τὸ χρυσίον τὸ εὑρεθὲν ἐν θησαυροῖς οἴκου Κυρίου καὶ οἴκου τοῦ βασιλέως, καὶ ἀπέστειλεν τῷ Ἀζαὴλ βασιλεῖ Συρίας, καὶ ἀνέβη ἀπὸ Ἱερουσαλήμ.
\par }{\PP \VS{20}Καὶ τὰ λοιπὰ τῶν λόγων Ἰωὰς καὶ πάντα ὅσα ἐποίησεν, οὐκ ἰδοὺ ταῦτα γεγραμμένα ἐπὶ βιβλίῳ λόγων τῶν ἡμερῶν τοῖς βασιλεῦσιν Ἰούδα;
\VS{21}Καὶ ἀνέστησαν οἱ δοῦλοι αὐτοῦ καὶ ἔδησαν πάντα σύνδεσμον, καὶ ἐπάταξαν τὸν Ἰωὰς ἐν οἴκῳ Μαλλὼ τῷ ἐν Σελά.
\VS{22}Καὶ Ἰεζιρχὰρ υἱὸς Ἱεμουὰθ, καὶ Ἰεζεβοὺθ ὁ υἱὸς αὐτοῦ Σωμὴρ, οἱ δοῦλοι αὐτοῦ, ἐπάταξαν αὐτὸν καὶ ἀπέθανε, καὶ ἔθαψαν αὐτὸν μετὰ τῶν πατέρων αὐτοῦ ἑν πόλει Δαυίδ· καὶ ἐβασίλευσεν Ἀμεσσείας υἱὸς αὐτοῦ ἀντʼ αὐτοῦ.

\par }\Chap{13}{\PP \VerseOne{1}Ἐν ἔτει εἰκοστῷ καὶ τρίτῳ ἔτει τῷ Ἰωὰς υἱῷ Ὀχοζίου βασιλεῖ Ἰούδα ἐβασίλευσεν Ἰωάχαζ υἱὸς Ἰοὺ ἐν Σαμαρείᾳ ἑπτακαίδεκα ἔτη.
\VS{2}Καὶ ἐποίησε τὸ πονηρὸν ἐν ὀφθαλμοῖς Κυρίου, καὶ ἐπορεύθη ὀπίσω ἁμαρτιῶν Ἱεροβοὰμ υἱοῦ Ναβὰτ, ὃς ἐξήμαρτε τὸν Ἰσραὴλ, οὐκ ἀπέστη ἀπʼ αὐτῆς.
\par }{\PP \VS{3}Καὶ ὠργίσθη θυμῷ Κύριος ἐν τῷ Ἰσραὴλ, καὶ ἔδωκεν αὐτοὺς ἐν χειρὶ Ἀζαὴλ βασιλέως Συρίας, καὶ ἐν χειρὶ υἱοῦ Ἄδερ υἱοῦ Ἀζαὴλ πάσας τὰς ἡμέρας.
\VS{4}Καὶ ἐδεήθη Ἰωάχαζ τοῦ προσώπου Κυρίου, καὶ ἐπήκουσεν αὐτοῦ Κύριος, ὅτι εἶδε τὴν θλίψιν Ἰσραὴλ, ὅτι ἔθλιψεν αὐτοὺς βασιλεὺς Συρίας.
\VS{5}Καὶ ἔδωκε Κύριος σωτηρίαν τῷ Ἰσραὴλ, καὶ ἐξῆλθεν ὑποκάτωθεν χειρὸς Συρίας· καὶ ἐκάθισαν οἱ υἱοὶ Ἰσραὴλ ἐν τοῖς σκηνώμασιν αὐτῶν καθὼς ἐχθὲς καὶ τρίτης.
\VS{6}Πλὴν οὐκ ἀπέστησαν ἀπὸ ἁμαρτιῶν οἴκου Ἱεροβοὰμ ὃς ἐξήμαρτε τὸν Ἰσραὴλ, ἐν αὐτῇ ἐπορεύθη· καὶ γε τὸ ἄλσος ἐστάθη ἐν Σαμαρείᾳ.
\VS{7}Ὅτι οὐχ ὑπελείφθη τῷ Ἰωάχαζ λαὸς, ἀλλʼ ἢ πεντήκοντα ἱππεῖς καὶ δέκα ἅρματα καὶ δέκα χιλιάδες πεζῶν, ὅτι ἀπώλεσεν αὐτοὺς βασιλεὺ Συρίας, καὶ ἔθεντο αὐτοὺς ὡς χοῦν εἰς καταπάτησιν.
\par }{\PP \VS{8}Καὶ τὰ λοιπὰ τῶν λόγων Ἰωάχαζ καὶ πάντα ὅσα ἐποίησε καὶ αἱ δυναστεῖαι αὐτοῦ, οὐχὶ ταῦτα γεγραμμένα ἐπὶ βιβλίῳ λόγων τῶν ἡμερῶν τοῖς βασιλεῦσιν Ἰσραήλ;
\VS{9}Καὶ ἐκοιμήθη Ἰωάχαζ μετὰ τῶν πατέρων αὐτοῦ, καὶ ἔθαψαν αὐτὸν ἐν Σαμαρείᾳ· καὶ ἐβασίλευσεν Ἰωὰς υἱὸς αὐτοῦ ἀντʼ αὐτοῦ.
\par }{\PP \VS{10}Ἐν ἔτει τριακοστῷ καὶ ἑβδόμῳ ἔτει τῷ Ἰωὰς βασιλεῖ Ἰούδα ἐβασίλευσεν Ἰωὰς υἱὸς Ἰωάχαζ ἐπὶ Ἰσραὴλ ἐν Σαμαρείᾳ ἑκκαίδεκα ἔτη.
\VS{11}Καὶ ἐποίησε τὸ πονηρὸν ἐν ὀφθαλμοῖς Κυρίου· οὐκ ἀπέστη ἀπὸ πάσης Ἰεροβοὰμ υἱοῦ Ναβὰτ ἁμαρτίας, ὃς ἐξήμαρτε τὸν Ἰσραὴλ· ἐν αὐτῇ ἐπορεύθη.
\VS{12}Καὶ τὰ λοιπὰ τῶν λόγων Ἰωὰς καὶ πάντα ὅσα ἐποίησε, καὶ αἱ δυναστεῖαι αὐτοῦ ἃς ἐποίησε μετὰ Ἀμεσσίου βασιλέως Ἰούδα, οὐχὶ ταῦτα γεγραμμένα ἐπὶ βιβλίῳ λόγων τῶν ἡμερῶν τοῖς βασιλεῦσιν Ἰσραήλ;
\VS{13}Καὶ ἐκοιμήθη Ἰωὰς μετὰ τῶν πατέρων αὐτοῦ, καὶ Ἰεροβοὰμ ἐκάθισεν ἐτὶ τοῦ θρόνου αὐτοῦ, καὶ ἐτάφη ἐν Σαμαρείᾳ μετὰ τῶν βασιλέων Ἰσραήλ.
\par }{\PP \VS{14}Καὶ Ἑλισαιὲ ἠῤῥώστησε τὴν ἀῤῥωστίαν αὐτοῦ, διʼ ἣν ἀπέθανε· καὶ κατέβη πρὸς αὐτὸν Ἰωὰς βασιλεὺς Ἰσραὴλ, καὶ ἔκλαυσεν ἐπὶ πρόσωπον αὐτοῦ, καὶ εἶπε, πάτερ πάτερ, ἅρμα Ἰσραὴλ καὶ ἱππεὺς αὐτοῦ.
\VS{15}Καὶ εἶπεν αὐτῷ Ἐλισαιὲ, λάβε τόξον καὶ βέλη· καὶ ἔλαβε πρὸς ἑαυτὸν τόξον καὶ βέλη.
\VS{16}Καὶ εἶπε τῷ βασιλεῖ, ἐπιβίβασον τὴν χεῖρά σου ἐπὶ τὸ τόξον· καὶ ἐπεβίβασεν Ἰωὰς τὴν χεῖρα αὐτοῦ· καὶ ἐπέθηκεν Ἑλισαιὲ τὰς χεῖρας αὐτοῦ ἐπὶ τὰς χεῖρας τοῦ βασιλέως,
\VS{17}καὶ εἶπεν, ἄνοιξον τὴν θυρίδα κατʼ ἀνατολάς· καὶ ἤνοιξε· καὶ εἶπεν Ἑλισαιὲ, τόζευσον· καὶ ἐτόζευσε· καὶ εἶπε, βέλος σωτηρίας τῷ Κυρίῳ, καὶ βέλος σωτηρίας ἐν Συρίᾳ, καὶ πατάξεις τὴν Συρίαν ἐν Ἀφὲκ ἕως συντελείας.
\VS{18}Καὶ εἶπεν αὐτῷ Ἑλισαιὲ, λάβε τόξα· καὶ ἔλαβε· καὶ εἶπε τῷ βασιλεῖ Ἰσραὴλ, πάταξον εἰς τὴν γῆν· καὶ ἐπάταξεν ὁ βασιλεὺς τρὶς, καὶ ἔστη·
\VS{19}Καὶ ἐλυπήθη ἐπʼ αὐτῷ ὁ ἄνθρωπος τοῦ Θεοῦ, καὶ εἶπεν, εἰ ἐπάταξας πεντάκις ἢ ἑξάκις, τότε ἂν ἐπάταξας τὴν Συρίαν ἕως συντελείας, καὶ νῦν τρὶς πατάξεις τὴν Συρίαν.
\par }{\PP \VS{20}Καὶ ἀπέθανεν Ἑλισαιὲ, καὶ ἔθαψαν αὐτόν· καὶ μονόζωνοι Μωὰβ ἦλθον ἐν τῇ γῇ ἐλθόντος τοῦ ἐνιαυτοῦ.
\VS{21}Καὶ ἐγένετο αὐτῶν θαπτόντων τὸν ἄνδρα, καὶ ἰδοὺ εἶδον τὸν μονόζωνον, καὶ ἔῤῥιψαν τὸν ἄνδρα ἐν τῷ τάφῳ Ἑλισαιέ· καὶ ἐπορεύθη καὶ ἥψατο τῶν ὀστέων Ἑλισαιὲ, καὶ ἔζησε καὶ ἀνέστη ἐπὶ τοὺς πόδας αὐτοῦ.
\par }{\PP \VS{22}Καὶ Ἀζαὴλ ἐξέθλιψε τὸν Ἰσραὴλ πάσας τὰς ἡμέρας Ἰωάχαζ.
\VS{23}Καὶ ἠλέησε Κύριος αὐτοὺς καὶ ᾠκτείρησεν αὐτούς, καὶ ἐπέβλεψεν ἐπʼ αὐτοὺς διὰ τὴν διαθήκην αὐτοῦ τὴν μετὰ Ἁβραὰμ καὶ Ἰσαὰκ καὶ Ἰακὼβ, καὶ οὐκ ἠθέλησε Κύριος διαφθεῖραι αὐτοὺς, καὶ οὐκ ἀπέῤῥιψεν αὐτοὺς ἀπὸ τοῦ προσώπου αὐτοῦ.
\VS{24}Καὶ ἀπέθανεν Ἀζαὴλ βασιλεὺς Συρίας, καὶ ἐβασίλευσεν υἱὸς Ἄδερ υἱὸς αὐτοῦ ἀντʼ αὐτοῦ.
\VS{25}Καὶ ἐπέστρεψεν Ἰωὰς υἱὸς Ἰωάχαζ, καὶ ἔλαβε τὰς πόλεις ἐκ χειρὸς υἱοῦ Ἄδερ υἱοῦ Ἄζαὴλ, ἃς ἔλαβεν ἐκ χειρὸς Ἰωάχας τοῦ πατρὸς αὐτοῦ ἐν τῷ πολέμῳ· τρὶς ἐπάταξεν αὐτὸν Ἰωὰς, καὶ ἐπέστρεψε τὰς πόλεις Ἰσραήλ.

\par }\Chap{14}{\PP \VerseOne{1}Ἐν ἔτει δευτέρῳ τῷ Ἰωὰς υἱῷ Ἰωάχαζ βασιλεῖ Ἰσραὴλ, καὶ ἐβασιλευσεν Ἀμεσσίας υἱὸς Ἰωὰς βασιλεὺς Ἰούδα.
\VS{2}Υἱὸς εἴκοσι καὶ πέντε ἐτῶν ἦν ἐν τῷ βασιλεύειν αὐτὸν, καὶ εἴκοσι καὶ ἐννέα ἔτη ἐβασίλευσεν ἐν Ἰερουσαλὴμ, καὶ ὄνομα τῆς μητρὸς αὐτοῦ Ἰωαδὶμ ἐξ Ἱερουσαλήμ.
\VS{3}Καὶ ἐποίησε τὸ εὐθὲς ἐν ὀφθαλμοῖς Κυρίου, πλὴν οὐχ ὡς Δαυὶδ ὁ πατὴρ αὐτοῦ· κατὰ πάντα ὅσα ἐποίησεν Ἰωὰς ὁ πατὴρ αὐτοῦ ἐποίησε.
\VS{4}Πλὴν τὰ ὑψηλὰ οὐκ ἐξῇρεν· ἔτι ὁ λαὸς ἐθυσίαζε καὶ ἐθυμίων ἐν τοῖς ὑψηλοῖς.
\VS{5}Καὶ ἐγένετο ὅτε κατίσχυσεν ἡ βασιλεία ἐν χειρὶ αὐτοῦ, καὶ ἐπάταξε τοὺς δούλους αὐτοῦ τοὺς πατάξαντας τὸν πατέρα αὐτοῦ.
\VS{6}Καὶ τοὺς υἱοὺς τῶν παταξάντων οὐκ ἐθανάτωσε, καθὼς γέγραπται ἐν βιβλίῳ νόμων Μωυσῆ, ὡς ἐνετείλατο Κύριος, λέγων, οὐκ ἀποθανοῦνται πατέρες ὑπὲρ υἱῶν, καὶ υἱοὶ οὐκ ἀποθανοῦνται ὑπὲρ πατέρων, ὅτι ἀλλʼ ἢ ἕκαστος ἐν ταῖς ἁμαρτίαις αὐτοῦ ἀποθανεῖται.
\VS{7}Αὐτὸς ἐπάταξε τὴν Ἐδὼμ ἐν γεμελὲδ δέκα χιλιάδας, καὶ συνέλαβε τὴν πέτραν ἐν τῷ πολέμῳ, καὶ ἐκάλεσε τὸ ὄνομα αὐτῆς Ἰεθοὴλ ἕως τῆς ἡμέρας ταύτης.
\par }{\PP \VS{8}Τότε ἀπέστειλεν Ἀμεσσίας ἀγγέλους πρὸς Ἰωὰς υἱὸν Ἰωάχαζ υἱοῦ Ἰοὺ βασιλέως Ἰσραὴλ, λέγων, δεῦρο ὀφθῶμεν προσώποις.
\VS{9}Καὶ ἀπέστειλεν Ἰωὰς βασιλεὺς Ἰσραὴλ πρὸς Ἀμεσσίαν βασιλέα Ἰούδα, λέγων, ὁ ἄκαν ὁ ἐν τῷ Λιβάνῳ ἀπέστειλε πρὸς τὴν κέδρον τὴν ἐν τῷ Λιβάνῳ, λέγων δὸς τὴν θυγατέρα σου τῷ υἱῷ μου εἰς γυναῖκα· καὶ διῆλθον τὰ θηρία τοῦ ἀγροῦ τὰ ἐν τῷ Λιβάνῳ, καὶ συνεπάτησαν τὴν ἄκανα.
\VS{10}Τύπτων ἐπάταξας τὴν Ἰδουμαίαν, καὶ ἐπῇρέ σε καρδία σου· ἐνδοξάσθητι καθήμενος ἐν τῷ οἴκῳ σου, καὶ ἱνατί ἐρίζεις ἐν κακίᾳ σου; καὶ πεσῇ σὺ καὶ Ἰούδας μετὰ σοῦ.
\par }{\PP \VS{11}Καὶ οὐκ ἤκουσεν Ἀμεσσίας· καὶ ἀνέβη Ἰωὰς βασιλεὺς Ἰσραὴλ, καὶ ὤφθησαν προσώποις αὐτὸς καὶ Ἀμεσσίας βασιλεὺς Ἰούδα ἐν Βαιθσαμὺς τῇ τοῦ Ἰούδα·
\VS{12}Καὶ ἔπταισεν Ἰούδας ἀπὸ προσώπου Ἰσραὴλ, καὶ ἔφυγεν ἀνὴρ εἰς τὸ σκήνωμα αὐτοῦ.
\VS{13}Καὶ τὸν Ἀμεσσίαν υἱὸν Ἰωὰς υἱοῦ Ὀχοζίου συνέλαβεν Ἰωὰς βασιλεὺς Ἰσραὴλ ἐν Βαιθσαμύς· καὶ ἦλθεν εἰς Ἱερουσαλήμ, καὶ καθεῖλεν ἐν τῷ τείχει Ἱερουσαλὴμ ἐν τῇ πύλῃ Ἐφραὶμ ἕως πύλης τῆς γωνίας τετρακοσίους πήχεις.
\VS{14}Καὶ ἔλαβε τὸ χρυσίον, καὶ τὸ ἀργύριον, καὶ πάντα τὰ σκεύη τὰ εὑρεθέντα ἐν οἴκῳ Κυρίου, καὶ ἐν θησαυροῖς οἴκου τοῦ βασιλέως, καὶ τοὺς υἱοὺς τῶν συμμίξεων, καὶ ἀπέστρεψεν εἰς Σαμάρειαν.
\par }{\PP \VS{15}Καὶ τὰ λοιπὰ τῶν λόγων Ἰωὰς ὅσα ἐποίησεν ἐν δυναστείᾳ αὐτοῦ, ἃ ἐπολέμησε μετὰ Ἀμεσσείου βασιλέως Ἰούδα, οὐχὶ ταῦτα γεγραμμένα ἐπὶ βιβλίῳ λόγων τῶν ἡμερῶν τοῖς βασιλεῦσιν Ἰσραήλ;
\VS{16}Καὶ ἐκοιμήθη Ἰωὰς μετὰ τῶν πατέρων αὐτοῦ, καὶ ἐτάφη ἐν Σαμαρείᾳ μετὰ τῶν βασιλέων Ἰσραήλ· καὶ ἐβασίλευσεν Ἰεροβοὰμ υἱὸς αὐτοῦ ἀντʼ αὐτοῦ.
\par }{\PP \VS{17}Καὶ ἔζησεν Ἀμεσσίας υἱὸς Ἰωὰς βασιλεὺς Ἰούδα μετὰ τὸ ἀποθανεῖν Ἰωὰς υἱὸν Ἰωάχαζ βασιλέα Ἰσραὴλ, πεντεκαίδεκα ἔτη.
\VS{18}Καὶ τὰ λοιπὰ τῶν λόγων Ἀμεσσίου καὶ πάντα ἅσα ἐποίησεν, οὐχὶ ταῦτα γεγραμμένα ἐπὶ βιβλίῳ λόγων τῶν ἡμερῶν τοῖς βασιλεῦσιν Ἰούδα;
\VS{19}Καὶ συνεστράφησαν ἐπʼ αὐτὸν σύστρεμμα ἐν Ἱερουσαλὴμ, καὶ ἔφυγεν εἰς Λαχίς· καὶ ἀπέστειλαν ὀπίσω αὐτοῦ εἰς Λαχὶς, καὶ ἐθανάτωσαν αὐτὸν ἐκεῖ.
\VS{20}Καὶ ῃραν αὐτὸν ἐφʼ ἵππων, καὶ ἐτάφη ἐν Ἱερουσαλὴμ μετὰ τῶν πατέρων αὐτοῦ ἐν πόλει Δαυίδ.
\par }{\PP \VS{21}Καὶ ἔλαβε πᾶς ὁ λαὸς Ἰούδα τὸν Ἀζαρίαν, καὶ αὐτὸς υἱὸς ἑκκαίδεκα ἐτῶν, καὶ ἐβασίλευσαν αὐτὸς ἀντὶ τοῦ πατρὸς αὐτοῦ Ἀμεσσίου.
\VS{22}Αὐτὸς ᾠκοδόμησε τὴν Αἰλὼθ, καὶ ἐπέστρεψεν αὐτὴν τῷ Ἰούδα μετὰ τὸ κοιμηθῆναι τὸν βασιλέα μετὰ τῶν πατέρων αὐτοῦ.
\par }{\PP \VS{23}Ἐν ἔτει πεντεκαιδεκάτῳ τοῦ Ἀμεσσίου υἱῷ Ἰωὰς βασιλεῖ Ἰούδα ἐβασίλευσεν Ἱεροβοὰμ υἱὸς Ἰωὰς ἐπὶ Ἰσραὴλ ἐν Σαμαρείᾳ τεσσαράκοντα καὶ ἓν ἔτος.
\VS{24}Καὶ ἐποίησε τὸ πονηρὸν ἐνώπιον Κυρίου· οὐκ ἀπέστη ἀπὸ πασῶν ἁμαρτιῶν Ἰεροβοὰμ υἱοῦ Ναβὰτ, ὃς ἐξήμαρτε τὸν Ἰσραήλ.
\VS{25}Αὐτὸς ἀπέστησε τὸ ὅριον Ἰσραὴλ ἀπὸ εἰσόδου Αἰμὰθ ἕως τῆς θαλάσσης τῆς Ἄραβα, κατὰ τὸ ῥῆμα Κυρίου Θεοῦ Ἰσραὴλ ὃ ἐλάλησεν ἐν χειρὶ δούλου αὐτοῦ Ἰῶνα υἱοῦ Ἀμαθὶ τοῦ προφήτου τοῦ ἐκ Γεθχοφέρ.
\VS{26}Ὅτι εἶδε Κύριος τὴν ταπείνωσιν Ἰσραὴλ πικρὰν σφόδρα, καὶ ὀλιγοστοὺς συνεχομένους, καὶ ἐσπανισμένους, καὶ ἐγκαταλελειμμένους, καὶ οὐκ ἦν ὁ βοηθῶν τῷ Ἰσραήλ.
\VS{27}Καὶ οὐκ ἐλάλησε Κύριος ἐξαλεῖψαι τὸ σπέρμα Ἰσραὴλ ὑποκάτωθεν τοῦ οὐρανοῦ· καὶ ἔσωσεν αὐτοὺς διὰ χειρὸς Ἱεροβοὰμ υἱοῦ Ἰωάς.
\par }{\PP \VS{28}Καὶ τὰ λοιπὰ τῶν λόγων Ἱεροβοὰμ καὶ πάντα ὅσα ἐποίησε, καὶ αἱ δυναστεῖαι αὐτοῦ, ὅσα ἐπολέμησε, καὶ ὅσα ἐπέστρεψε τὴν Δαμασκὸν, καὶ τὴν Αἰμὰθ τῷ Ἰούδα ἐν Ἰσραὴλ, οὐχὶ ταῦτα γεγραμμένα ἐπὶ βιβλίῳ λόγων τῶν ἡμερῶν τοῖς βασιλεῦσιν Ἰσραήλ;
\VS{29}Καὶ ἐκοιμήθη Ἱεροβοὰμ μετὰ τῶν πατέρων αὐτοῦ μετὰ βασιλέων Ἰσραὴλ, καὶ ἐβασίλευσε Ζαχαρίας υἱὸς αὐτοῦ ἀντʼ αὐτοῦ.

\par }\Chap{15}{\PP \VerseOne{1}Ἐν ἔτει εἰκοστῷ καὶ ἑβδόμῳ τῷ Ἱεροβοὰμ βασιλεῖ Ἰσραὴλ ἐβασίλευσεν Ἀζαρίας υἱὸς Ἀμεσσίου βασιλέως Ἰούδα.
\VS{2}Υἱὸς ἑκκαίδεκα ἐτῶν ἦν ἐν τῷ βασιλεύειν αὐτὸν, καὶ πεντηκονταδύο ἔτη ἐβασίλευσεν ἐν Ἱερουσαλὴμ, καὶ ὄνομα τῇ μητρὶ αὐτοῦ Ἱεχελία ἐξ Ἱερουσαλήμ.
\VS{3}Καὶ ἐποίησε τὸ εὐθὲς ἐν ὀφθαλμοῖς Κυρίου, κατὰ πάντα ὅσα ἐποίησεν Ἀμεσσίας ὁ πατὴρ αὐτοῦ.
\VS{4}Πλὴν τῶν ὑψηλῶν οὐκ ἐξῇρεν, ἔτι ὁ λαὸς ἐθυσίαζε καὶ ἐθυμίων ἐν τοῖς ὑψηλοῖς.
\par }{\PP \VS{5}Καὶ ἥψατο Κύριος τὸν βασιλέα, καὶ ἦν λελεπρωμένος ἕως ἡμέρας θανάτου αὐτοῦ· καὶ ἐβασίλευσεν ἐν οἴκῳ ἀφφουσώθ· καὶ Ἰωάθαμ υἱὸς τοῦ βασιλέως ἐπὶ τῷ οἴκῳ κρίνων τὸν λαὸν τῆς γῆς.
\par }{\PP \VS{6}Καὶ τὰ λοιπὰ τῶν λόγων Ἀζαρίου καὶ πάντα ὅσα ἐποίησεν, οὐχὶ ταῦτα γεγραμμένα ἐπὶ βιβλίῳ λόγων τῶν ἡμερῶν τοῖς βασιλεῦσιν Ἰούδα;
\VS{7}Καὶ ἐκοιμήθη Ἀζαρίας μετὰ τῶν πατέρων αὐτοῦ, καὶ ἔθαψαν αὐτὸν μετὰ τῶν πατέρων αὐτοῦ ἐν πόλει Δαυὶδ, καὶ ἐβασίλευσεν Ἰωάθαν υἱὸς αὐτοῦ ἀντʼ αὐτοῦ.
\par }{\PP \VS{8}Ἐν ἔτει τριακοστῷ καὶ ὀγδόῳ τῷ Ἀζαρίου βασιλεῖ Ἰούδα ἐβασίλευσε Ζαχαρίας υἱὸς Ἰεροβοὰμ ἐπὶ Ἰσραὴλ ἐν Σαμαρείᾳ ἑξάμηνον.
\VS{9}Καὶ ἐποίησε τὸ πονηρὸν ἐν ὀφθαλμοῖς Κυρίου καθὰ ἐποίησαν οἱ πατέρες αὐτοῦ· οὐκ ἀπέστη ἀπὸ πασῶν τῶν ἁμαρτιῶν Ἰεροβοὰμ υἱοῦ Ναβὰτ, ὃς ἐξήμαρτε τὸν Ἰσραήλ.
\VS{10}Καὶ συνεστράφησαν ἐπʼ αὐτὸν Σελλοὺμ υἱὸς Ἰαβίς· καὶ ἐπάταξαν αὐτὸν Κεβλαὰμ καὶ ἐθανάτωσαν αὐτὸν, καὶ ἐβασίλευσεν ἀντʼ αὐτοῦ.
\VS{11}Καὶ τὰ λοιπὰ τῶν λόγων Ζαχαρίου, ἰδού εἰσι γεγραμμένα ἐπὶ βιβλίῳ λόγων τῶν ἡμερῶν τοῖς βασιλεῦσιν Ἰσραήλ.
\VS{12}Ὁ λόγος Κυρίου ὃν ἐλάλησε πρὸς Ἰοὺ, λέγων, υἱοὶ τέταρτοι καθήσονταί σοι ἐπὶ θρόνου Ἰσραήλ· καὶ ἐγένετο οὕτως.
\par }{\PP \VS{13}Καὶ Σελλοὺμ υἱὸς Ἰαβὶς ἐβασίλευσε· καὶ ἐν ἔτει τριακοστῷ καὶ ἐννάτῳ Ἀζαρίᾳ βασιλεῖ Ἰούδα ἐβασίλευσε Σελλοὺμ μῆνα ἡμερῶν ἐν Σαμαρείᾳ.
\VS{14}Καὶ ἀνέβη Μαναὴμ υἱὸς Γαδδὶ ἐκ Θαρσιλὰ, καὶ ἦλθεν εἰς Σαμάρειαν, καὶ ἐπάταξε τὸν Σελλοὺμ υἱὸν Ἰαβὶς ἐν Σαμαρείᾳ, καὶ ἐθανάτωσεν αὐτόν.
\VS{15}Καὶ τὰ λοιπὰ τῶν λόγων Σελλοὺμ, καὶ ἡ συστροφὴ αὐτοῦ ᾗ συνεστράφη, ἰδού εἰσι γεγραμμένα ἐπὶ βιβλίῳ λόγων τῶν ἡμερῶν τοῖς βασιλεῦσιν Ἰσραήλ.
\par }{\PP \VS{16}Τότε ἐπάταξε Μαναὴμ καὶ τὴν Θερσὰ καὶ πάντα τὰ ἐν αὐτῇ, καὶ τὰ ὅρια αὐτῆς ἀπὸ Θερσὰ, ὅτι οὐκ ἤνοιξαν αὐτῷ, καὶ ἐπάταξεν αὐτὴν, καὶ τὰς ἐν γαστρὶ ἐχούσας ἀνέῤῥηξεν.
\par }{\PP \VS{17}Ἐν ἔτει τριακοστῷ καὶ ἐννάτῳ τῷ Ἀζαρίᾳ βασιλεῖ Ἰούδα ἐβασίλευσε Μαναὴμ υἱὸς Γαδδὶ ἐπὶ Ἰσραὴλ ἐν Σαμαρείᾳ δέκα ἔτη.
\VS{18}Καὶ ἐποίησε τὸ πονηρὸν ἐν ὀφθαλμοῖς Κυρίου, οὐκ ἀπέστη ἀπὸ πασῶν ἁμαρτιῶν Ἱεροβοὰμ υἱοῦ Ναβὰτ ὃς ἐξήμαρτε τὸν Ἰσραήλ.
\VS{19}Ἐν ταῖς ἡμέραις αὐτοῦ ἀνέβη Φονὰ βασιλεὺς Ἀσσυρίων ἐπὶ τὴν γῆν, καὶ Μαναὴμ ἔδωκε τῷ Φουὰ χίλια τάλαντα ἀργυρίου εἶναι τὴν χεῖρα αὐτοῦ μετʼ αὐτοῦ.
\VS{20}Καὶ ἐξήνεγκε Μαναὴμ τὸ ἀργύριον ἐπὶ τὸν Ἰσραὴλ ἐπὶ πᾶν δυνατὸν ἰσχύϊ, δοῦναι τῷ βασιλεῖ τῶν Ἀσσυρίων, πεντήκοντα σίκλους τῷ ἀνδρὶ τῷ ἑνί· καὶ ἀπέστρεψε βασιλεὺς Ἀσσυρίων, καὶ οὐκ ἔστη ἐκεῖ ἐν τῇ γῇ.
\VS{21}Καὶ τὰ λοιπὰ τῶν λόγων Μαναὴμ καὶ πάντα ὅσα ἐποίησεν, οὐκ ἰδοὺ ταῦτα γεγραμμένα ἐπὶ βιβλίῳ λόγων τῶν ἡμερῶν τοῖς βασιλεῦσιν Ἰσραήλ;
\VS{22}Καὶ ἐκοιμήθη Μαναὴμ μετὰ τῶν πατέρων αὐτοῦ, καὶ ἐβασίλευσε Φακεσίας υἱὸς αὐτοῦ ἀντʼ αὐτοῦ.
\par }{\PP \VS{23}Ἐν ἔτει πεντηκοστῷ τοῦ Ἀζαρίου βασιλεῖ Ἰούδα ἐβασίλευσε Φακεσίας υἱὸς Μαναὴμ ἐπὶ Ἰσραὴλ ἐν Σαμαρείᾳ δύο ἔτη.
\VS{24}Καὶ ἐποίησε τὸ πονηρὸν ἐν ὀφθαλμοῖς Κυρίου, οὐκ ἀπέστη ἀπὸ ἁμαρτιῶν Ἰεροβοὰμ υἱοῦ Ναβὰτ ὃς ἐξήμαρτε τὸν Ἰσραήλ.
\VS{25}Καὶ συνεστράφη ἐπʼ αὐτὸν Φακεὲ υἱὸς Ῥομελίου ὁ τριστάτης αὐτοῦ, καὶ ἐπάταξεν αὐτὸν ἐν Σαμαρείᾳ ἐναντίον οἴκου τοῦ βασιλέως μετὰ τοῦ Ἀργὸβ καὶ μετὰ τοῦ Ἀρία, καὶ μετʼ αὐτοῦ πεντήκοντα ἄνδρες ἀπὸ τῶν τετρακοσίων· καὶ ἐθανάτωσε αὐτὸν, καὶ ἐβασίλευσεν ἀντʼ αὐτοῦ.
\VS{26}Καὶ τὰ λοιπὰ τῶν λόγων Φακεσίου καὶ πάντα ὅσα ἐποίησεν, ἰδού εἰσι γεγραμμένα ἐπὶ βιβλίῳ λόγων τῶν ἡμερῶν τοῖς βασιλεῦσιν Ἰσραήλ.
\par }{\PP \VS{27}Ἐν ἔτει πεντηκοστῷ καὶ δευτέρῳ τοῦ Ἀζαρίου βασιλεῖ Ἰούδα ἐβασίλευσε Φακεὲ υἱὸς Ῥομελίου ἐπὶ Ἰσραὴλ ἐν Σαμαρείᾳ εἴκοσι ἔτη.
\VS{28}Καὶ ἐποίησε τὸ πονηρὸν ἐν ὀφθαλμοῖς Κυρίου, οὐκ ἀπέστη ἀπὸ πασῶν ἁμαρτιῶν Ἰεροβοὰμ υἱοῦ Ναβὰτ ὃς ἐξήμαρτε τὸν Ἰσραήλ.
\VS{29}Ἐν ταῖς ἡμέραις Φακεὲ βασιλέως Ἰσραὴλ ἦλθε Θαλγαθφελλασὰρ βασιλεὺς Ἀσσυρίων, καὶ ἔλαβε τὴν Αἲν καὶ τὴν Ἀβὲλ καὶ τὴν Θαμααχὰ καὶ τὴν Ἀνιὼχ καὶ τὴν Κενὲζ καὶ τὴν Ἀσὼρ καὶ τὴν Γαλαὰν καὶ τὴν Γαλιλαίαν, πᾶσαν γῆν Νεφθαλὶ, καὶ ἀπῳκισεν αὐτοὺς εἰς Ἀσσυρίους.
\VS{30}Καὶ συνέστρεψε σύστρεμμα Ὠσηὲ υἱὸς Ἠλὰ ἐπὶ Φακεὲ υἱὸν Ῥομελίου, καὶ ἐπάταξεν αὐτὸν, καὶ ἐθανάτωσε, καὶ ἐβασίλευσεν ἀντʼ αὐτοῦ, ἐν ἔτει εἰκοστῷ Ἰωάθαμ υἱοῦ Ἀζαρίου.
\VS{31}Καὶ τὰ λοιπὰ τῶν λόγων Φακεὲ καὶ πάντα ὅσα ἐποίησεν, ἰδοὺ ταῦτα γεγραμμένα ἐπὶ βιβλίῳ λόγων τῶν ἡμερῶν τοῖς βασιλεῦσιν Ἰσραήλ.
\par }{\PP \VS{32}Ἐν ἔτει δευτέρῳ Φακεὲ υἱοῦ Ῥομελίου βασιλεῖ Ἰσραὴλ ἐβασίλευσεν Ἰωάθαμ υἱὸς Ἀζαρίου βασιλέως Ἰούδα.
\VS{33}Υἱὸς εἴκοσι καὶ πέντε ἐτῶν ἦν ἐν τῷ βασιλεύειν αὐτὸν, καὶ ἐκκαίδεκα ἔτη ἐβασίλευσεν ἐν Ἱερουσαλήμ, καὶ ὄνομα τῆς μητρὸς αὐτοῦ Ἰερουσὰ θυγάτηρ Σαδώκ.
\VS{34}Καὶ ἐποίησε τὸ εὐθὲς ἐν ὀφθαλμοῖς Κυρίου, κατὰ πάντα ὅσα ἐποίησεν Ἀζαρίας ὁ πατὴρ αὐτοῦ.
\VS{35}Πλὴν τὰ ὑψηλὰ οὐκ ἐξῇρεν, ἔτι ὁ λαὸς ἐθυσίαζε καὶ ἐθυμία ἐν τοῖς ὑψηλοῖς· αὐτὸς ᾠκοδόμησε τὴν πύλην οἴκου Κυρίου τὴν ἐπάνω.
\VS{36}Καὶ τὰ λοιπὰ τῶν λόγων Ἰωάθαμ καὶ πάντα ὅσα ἐποίησεν, οὐχὶ ταῦτα γεγραμμένα ἐπὶ βιβλίῳ λόγων τῶν ἡμερῶν τοῖς βασιλεῦσιν Ἰούδα;
\par }{\PP \VS{37}Ἐν ταῖς ἡμέραις ἐκείναις ἤρξατο Κύριος ἐξαποστέλλειν ἐν Ἰούδᾳ τὸν Ῥαασσὼν βασιλέα Συρίας, καὶ τὸν Φακεὲ υἱὸν Ῥομελίου.
\VS{38}Καὶ ἐκοιμήθη Ἰωάθαμ μετὰ τῶν πατέρων αὐτοῦ, καὶ ἐτάφη μετὰ τῶν πατέρων αὐτοῦ ἐν πόλει Δαυὶδ τοῦ πατρὸς αὐτοῦ· καὶ ἐβασίλευσεν Ἄχαζ υἱὸς αὐτοῦ ἀντʼ αὐτοῦ.

\par }\Chap{16}{\PP \VerseOne{1}Ἐν ἔτει ἑπτακαιδεκάτῳ Φακεὲ υἱοῦ Ῥομελίου ἐβασίλευσεν Ἄχαζ υἱὸς Ἰωάθαμ βασιλέως Ἰούδα.
\VS{2}Υἱὸς εἴκοσι ἐτῶν ἦν Ἄχαζ ἐν τῷ βασιλεύειν αὐτὸν, καὶ ἑκκαίδεκα ἔτη ἐβασίλευσεν ἐν Ἱερουσαλήμ· καὶ οὐκ ἐποίησε τὸ εὐθὲς ἐν ὀφθαλμοῖς Κυρίου Θεοῦ αὐτοῦ πιστῶς, ὡς Δαυὶδ ὁ πατὴρ αὐτοῦ.
\VS{3}Καὶ ἐπορεύθη ἐν ὁδῷ βασιλέων Ἰσραὴλ, καί γε τὸν υἱὸν αὐτοῦ διῆγεν ἐν πυρὶ, κατὰ τὰ βδελύγματα τῶν ἐθνῶν ὧν ἐξῇρε Κύριος ἀπὸ προσώπου τῶν υἱῶν Ἰσραήλ.
\VS{4}Καὶ ἐθυσίαζε καὶ ἐθυμία ἐν τοῖς ὑψηλοῖς, καὶ ἐπὶ τῶν βουνῶν, καὶ ὑποκάτω παντὸς ξύλου ἀλσώδους.
\par }{\PP \VS{5}Τότε ἀνέβη Ῥαασσὼν βασιλεὺς Συρίας καὶ Φακεὲ υἱὸς Ῥομελίου βασιλεὺς Ἰσραὴλ εἰς Ἱερουσαλὴμ εἰς πόλεμον, καὶ ἐπολιόρκουν ἐπὶ Ἄχαζ, καὶ οὐκ ἠδύναντο πολεμεῖν.
\VS{6}Ἐν τῷ καιρῷ ἐκείνῳ ἐπέστρεψε Ῥαασσὼν βασιλεὺς Συρίας τὴν Αἰλὰθ τῇ Συρίᾳ, καὶ ἐξέβαλε τοὺς Ἰουδαίους ἐξ Αἰλὰθ, καὶ Ἰδουμαῖοι ἦλθον εἰς Αἰλὰθ, καὶ κατῴκησαν ἐκεῖ ἕως τῆς ἡμέρας ταύτης.
\VS{7}Καὶ ἀπέστειλεν Ἄχαζ ἀγγέλους πρὸς Θαλγαθφελλασὰρ βασιλέα Ἀσσυρίων, λέγων, δοῦλός σου καὶ υἱός σου ἐγὼ, ἀνάβηθι, σῶσόν με ἐκ χειρὸς βασιλέως Συρίας, καὶ ἐκ χειρὸς βασιλέως Ἰσραὴλ, τῶν ἐπανισταμένων ἐπʼ ἐμέ.
\VS{8}Καὶ ἔλαβεν Ἄχαζ ἀργύριον καὶ χρυσίον τὸ εὑρεθὲν ἐν θησαυροῖς οἴκου Κυρίου καὶ οἴκου τοῦ βασιλέως, καὶ ἀπέστειλε τῷ βασιλεῖ δῶρα.
\VS{9}Καὶ ἤκουσεν αὐτοῦ βασιλεὺς Ἀσσυρίων· καὶ ἀνέβη βασιλεὺς Ἀσσυρίων εἰς Δαμασκὸν, καὶ συνέλαβεν αὐτὴν, καὶ ἀπῴκισεν αὐτὴν, καὶ τὸν Ῥαασσὼν βασιλέα ἐθανάτωσε.
\par }{\PP \VS{10}Καὶ ἐπορεύθη βασιλεὺς Ἄχαζ εἰς Δαμασκὸν εἰς ἀπαντὴν Θαλγαθφελλασὰρ βασιλεῖ Ἀσσυρίων εἰς Δαμασκόν· καὶ εἶδε τὸ θυσιαστήριον ἐν Δαμασκῷ· καὶ ἀπέστειλεν ὁ βασιλεὺς Ἄχαζ πρὸς Οὐρίαν τὸν ἱερέα τὸ ὁμοίωμα τοῦ θυσιαστηρίου καὶ τὸν ῥυθμὸν αὐτοῦ καὶ πᾶσαν ποίησιν αὐτοῦ.
\VS{11}Καὶ ᾠκοδόμησεν Οὐρίας ὁ ἱερεὺς τὸ θυσιαστήριον, κατὰ πάντα ὅσα ἀπέστειλεν ὁ βασιλεὺς Ἄχαζ ἐκ Δαμασκοῦ.
\par }{\PP \VS{12}Καὶ εἶδεν ὁ βασιλεὺς τὸ θυσιαστήριον, καὶ ἀνέβη ἐπʼ αὐτὸ,
\VS{13}καὶ ἐθυμίασε τὴν ὁλοκαύτωσιν αὐτοῦ, καὶ τὴν θυσίαν αὐτοῦ, καὶ τὴν σπονδὴν αὐτοῦ, καὶ προσέχεε τὸ αἷμα τῶν εἰρηνικῶν τῶν αὐτοῦ ἐπὶ τὸ θυσιαστήριον τὸ χαλκοῦν
\VS{14}τὸ ἀπέναντι Κυρίου· καὶ προσήγαγε τὸ πρόσωπον τοῦ οἴκου Κυρίου, ἀπὸ τοῦ ἀναμέσον τοῦ θυσιαστηρίου καὶ ἀπὸ τοῦ ἀναμέσον τοῦ οἴκου Κυρίου· καὶ ἔδειξεν αὐτὸ ἐπὶ μηρὸν τοῦ θυσιαστηρίου κατὰ βοῤῥᾶν.
\VS{15}Καὶ ἐνετείλατο ὁ βασιλεὺς Ἄχαζ τῷ Οὐρίᾳ τῷ ἱερεῖ, λέγων, ἐπὶ τὸ θυσιαστήριον τὸ μέγα πρόσφερε τὴν ὁλοκαύτωσιν τὴν πρωϊνὴν καὶ τὴν θυσίαν τὴν ἑσπερινήν, καὶ τὴν ὁλοκαύτωσιν τοῦ βασιλέως καὶ τὴν θυσίαν αὐτοῦ, καὶ τὴν ὁλοκαύτωσιν παντὸς τοῦ λαοῦ, καὶ τὴν θυσίαν αὐτῶν καὶ τὴν σπονδὴν αὐτῶν, καὶ πᾶν αἷμα ὁλοκαυτώσεως, καὶ πᾶν αἷμα θυσίας ἐπʼ αὐτῶ ἐκχεεῖς· καὶ τὸ θυσιαστήριον τὸ χαλκοῦν ἔσται μοι εἰς τοπρωΐ.
\VS{16}Καὶ ἐποίησεν Οὐρίας ὁ ἱερεὺς κατὰ πάντα ὅσα ἐνετείλατο αὐτῷ ὁ βασιλεὺς Ἄχαζ.
\VS{17}Καὶ συνέκοψεν ὁ βασιλεὺς Ἄχαζ τὰ συνκλείσματα τῶν μεχωνὼθ, καὶ μετῇρεν ἀπʼ αὐτῶν τὸν λουτῆρα, καὶ τὴν θάλασσαν καθεῖλεν ἀπὸ τῶν βοῶν τῶν χαλκῶν τῶν ὑποκάτω αὐτῆς, καὶ ἔδωκεν αὐτὴν ἐπὶ βάσιν λιθίνην.
\VS{18}Καὶ τὸν θεμέλιον τῆς καθέδρας ᾠκοδόμησεν ἐν οἴκῳ Κυρίου, καὶ τὴν εἴσοδον τοῦ βασιλέως τὴν ἔξω ἐπέστρεψεν ἐν οἴκῳ Κυρίου ἀπὸ προσώπου βασιλέως Ἀσσυρίων.
\par }{\PP \VS{19}Καὶ τὰ λοιπὰ τῶν λόγων Ἄχαζ ὅσα ἐποίησεν, οὐχὶ ταῦτα γεγραμμένα ἐπὶ βιβλίῳ λόγων τῶν ἡμερῶν τοῖς βασιλεῦσιν Ἰούδα;
\VS{20}Καὶ ἐκοιμήθη Ἄχαζ μετὰ τῶν πατέρων αὐτοῦ, καὶ ἐτάφη ἐν πόλει Δαυὶδ, καὶ ἐβασίλευσεν Ἐζεκίας υἱὸς αὐτοῦ ἀντʼ αὐτοῦ.

\par }\Chap{17}{\PP \VerseOne{1}Ἐν ἔτει δωδεκάτῳ τοῦ Ἄχαζ βασιλέως Ἰούδα ἐβασίλευσεν Ὠσηὲ υἱὸς Ἠλὰ ἐν Σαμαρείᾳ ἐπὶ Ἰσραὴλ ἐννέα ἔτη.
\VS{2}Καὶ ἐποίησε τὸ πονηρὸν ἐν ὀφθαλμοῖς Κυρίου, πλὴν οὐχ ὡς οἱ βασιλεῖς Ἰσραὴλ οἳ ἦσαν ἔμπροσθεν αὐτοῦ.
\par }{\PP \VS{3}Ἐπʼ αὐτὸν ἀνέβη Σαλαμανασσὰρ βασιλεὺς Ἀσσυρίων· καὶ ἐγενήθη αὐτῷ Ὠσηὲ δοῦλος, καὶ ἐπέστρεψεν αὐτῷ μαναά.
\VS{4}Καὶ εὗρε βασιλεὺς Ἀσσυρίων ἐν τῷ Ὠσηὲ ἀδικίαν, ὅτι ἀπέστειλεν ἀγγέλους πρὸς Σηγὼρ βασιλέα Αἰγύπτου, καὶ οὐκ ἤνεγκε μαναὰ τῷ βασιλεῖ Ἀσσυρίων ἐν τῷ ἐνιαυτῷ ἐκείνῳ· καὶ ἐπολιόρκησεν αὐτὸν ὁ βασιλεὺς Ἀσσυρίων, καὶ ἔδησεν αὐτὸν ἐν οἴκῳ φυλακῆς.
\VS{5}Καὶ ἀνέβη ὁ βασιλεὺς Ἀσσυρίων ἐν πάσῃ τῇ γῇ, καὶ ἀνέβη εἰς Σαμάρειαν, καὶ ἐπολιόρκησεν ἐπʼ αὐτὴν τρία ἔτη.
\par }{\PP \VS{6}Ἐν ἔτει ἐννάτῳ Ὠσῆε συνέλαβε βασιλεὺς Ἀσσυρίων τὴν Σαμάρειαν, καὶ ἀπῴκισεν Ἰσραὴλ εἰς Ἀσσυρίους, καὶ κατῴκισεν αὐτοὺς ἐν Ἀλαὲ καὶ ἐν Ἀβὼρ ποταμοῖς Γωζὰν, καὶ ὄρη Μήδων.
\VS{7}Καὶ ἐγένετο ὅτι ἥμαρτον οἱ υἱοὶ Ἰσραὴλ τῷ Κυρίῳ Θεῷ αὐτῶν τῷ ἀναγαγόντι αὐτοὺς ἐκ γῆς Αἰγύπτου ὑποκάτωθεν χειρὸς Φαραὼ βασιλέως Αἰγύπτου, καὶ ἐφοβήθησαν θεοὺς ἐτέρους,
\VS{8}καὶ ἐπορεύθησαν τοῖς δικαιώμασι τῶν ἐθνῶν ὧν ἐξῇρε Κύριος ἐκ προσώπου υἱῶν Ἰσραὴλ, καὶ οἱ βασιλεῖς Ἰσραὴλ ὅσοι ἐποίησαν,
\VS{9}καὶ ὅσοι ἠμφιέσαντο οἱ υἱοὶ Ἰσραὴλ λόγους, οὐχ οὕτως κατὰ Κυρίου Θεου αὐτῶν· καὶ ᾠκοδόμησαν ἑαυτοῖς ὑψηλὰ ἐν πάσαις ταῖς πόλεσιν αὐτῶν ἀπὸ πύργου φυλασσόντων ἕως πόλεως ὀχυρᾶς,
\VS{10}καὶ ἐστήλωσαν ἑαυτοῖς στήλας καὶ ἄλση ἐπὶ παντὶ βουνῷ ὑψηλῷ καὶ ὑποκάτω παντὸς ξύλου ἀλσώδους,
\VS{11}καὶ ἐθυμίασαν ἐκεῖ ἐν πᾶσιν ὑψηλοῖς, καθὼς τὰ ἔθνη ἃ ἀπῴκησε Κύριος ἐκ προσώπου αὐτῶν, καὶ ἐποίησαν κοινωνοὺς, καὶ ἐχάραξαν τοῦ παροργίσαι τὸν Κύριον,
\VS{12}καὶ ἐλάτρευσαν τοῖς εἰδώλοις οἷς εἶπε Κύριος αὐτοῖς, οὐ ποιήσετε τὸ ῥῆμα τοῦτο τῷ Κυρίῳ.
\par }{\PP \VS{13}Καὶ διεμαρτύρατο Κύριος ἐν τῷ Ἰσραὴλ καὶ ἐν τῷ Ἰούδα καὶ ἐν χειρὶ πάντων τῶν προφητῶν αὐτοῦ παντὸς ὁρῶντος, λέγων, ἀποστράφητε ἀπὸ τῶν ὁδῶν ὑμῶν τῶν πονηρῶν, καὶ φυλάξατε τὰς ἐντολάς μου, καὶ τὰ δικαιώματά μου, καὶ πάντα τὸν νόμον ὃν ἐνετειλάμην τοῖς πατράσιν ὑμῶν, ὅσα ἀπέστειλα αὐτοῖς ἐν χειρὶ τῶν δούλων μου τῶν προφητῶν.
\VS{14}Καὶ οὐκ ἤκουσαν, καὶ ἐσκλήρυναν τὸν νῶτον αὐτῶν ὑπὲρ τὸν νώτον τῶν πατέρων αὐτῶν.
\VS{15}Καὶ τὰ μαρτύρια αὐτοῦ ὅσα διεμαρτύρατο αὐτοῖς οὐκ ἐφύλαξαν, καὶ ἐπορεύθησαν ὀπίσω τῶν ματαίων, καὶ ἐματαιώθησαν, καὶ ὀπίσω τῶν ἐθνῶν τῶν περικύκλῳ αὐτῶν, ὧν ἐνετείλατο Κύριος αὐτοῖς μὴ ποιῆσαι κατὰ ταῦτα.
\VS{16}Ἐγκατέλιπον τὰς ἐντολὰς Κυρίου Θεοῦ αὐτῶν, καὶ ἐποίησαν ἑαυτοῖς χώνευμα δύο δαμάλεις, καὶ ἐποίησαν ἄλση, καὶ προσεκύνησαν πάσῃ τῇ δυνάμει τοῦ οὐρανοῦ, καὶ ἐλάτρευσαν τῷ Βάαλ.
\VS{17}Καὶ διῆγον τοὺς υἱοὺς αὐτῶν καὶ τὰς θυγατέρας αὐτῶν ἐν πυρὶ, καὶ ἐμαντεύοντο μαντείας, καὶ οἰωνίζοντο· καὶ ἐπράθησαν τοῦ ποιῆσαι τὸ πονηρὸν ἐν ὀφθαλμοῖς Κυρίου παροργίσαι αὐτόν.
\par }{\PP \VS{18}Καὶ ἐθυμώθη Κύριος σφόδρα ἐν τῷ Ἰσραὴλ, καὶ ἀπέστησεν αὐτοὺς ἀπὸ τοῦ προσώπου αὐτοῦ, καὶ οὐχ ὑμελείφθη πλὴν φυλὴ Ἰούδα μονωτάτη.
\VS{19}Καί γε Ἰούδας οὐκ ἐφύλαξε τὰς ἐντολὰς Κυρίου τοῦ Θεοῦ αὐτῶν· καὶ ἐπορεύθησαν ἐν τοῖς δικαιώμασιν Ἰσραὴλ οἷς ἐποίησαν, καὶ ἀπεώσαντο τὸν Κύριος.
\par }{\PP \VS{20}Καὶ ἐθυμώθη Κύριος παντὶ σπέρματι Ἰσραὴλ, καὶ ἐσάλευσεν αὐτοὺς, καὶ ἔδωκεν αὐτοὺς ἐν χειρὶ διαρπαζόντων αὐτοὺς, ἕως οὗ ἀπέῤῥιψεν αὐτοὺς ἀπὸ προσώπου αὐτοῦ.
\VS{21}Ὅτι πλὴν Ἰσραὴλ ἐπάνωθεν οἴκου Δαυὶδ, καὶ ἐβασίλευσαν τὸν Ἱεροβοὰμ υἱὸν Ναβάτ· καὶ ἐξέωσεν Ἱεροβοὰμ τὸν Ἰσραὴλ ἐξόπισθε Κυρίου, καὶ ἐξήμαρτεν αὐτοὺς ἁμαρτίαν μεγάλην.
\VS{22}Καὶ ἐπορεύθησαν οἱ υἱοὶ Ἰσραὴλ ἐν πάσῃ ἁμαρτία Ἱεροβοὰμ ἧς ἐποιήσεν, οὐκ ἀπέστησαν ἀπʼ αὐτῆς
\VS{23}ἕως οὗ μετέστησε Κύριος τὸν Ἰσραὴλ ἀπὸ προσώπου αὐτοῦ, καθὼς ἐλάλησε Κύριος ἐν χειρὶ πάντων τῶν δούλων αὐτοῦ τῶν προφητῶν· καὶ ἀπῳκίσθη Ἰσραὴλ ἐπάνωθεν τῆς γῆς αὐτοῦ εἰς Ἀσσυρίους ἕως τῆς ἡμέρας ταύτης.
\par }{\PP \VS{24}Καὶ ἤγαγε βασιλεὺς Ἀσσυρίων ἐκ Βαβυλῶνος τὸν ἐκ Χουνθὰ, ἀπὸ Ἀϊὰ, καὶ ἀπὸ Αἱμὰθ, καὶ Σεπφαρουαῒν, καὶ κατῳκίσθησαν ἐν πόλεσι Σαμαρείας ἀντὶ τῶν υἱῶν Ἰσραὴλ, καὶ ἐκληρονόμησαν τὴν Σαμάρειαν, καὶ κατῴκίσθησαν ἐν ταῖς πόλεσιν αὐτῆς.
\VS{25}Καὶ ἐγένετο ἐν ἀρχῇ τῆς καθέδρας αὐτῶν οὐκ ἐφοβήθησαν τὸν Κύριον, καὶ ἀπέστειλε Κύριος ἐν αὐτοῖς τοὺς λέοντας, καὶ ἦσαν ἀποκτενίοντες ἐν αὐτοῖς.
\VS{26}Καὶ εἶπαν τῷ βασιλεῖ Ἀσσυρίων λέγοντες, τὰ ἔθνη ἃ ἀπῴκισας καὶ ἀντεκάθισας ἐν πόλεσι Σαμαρείας οὐκ ἔγνωσαν τὸ κρίμα τοῦ Θεοῦ τῆς γῆς, καὶ ἀπέστειλεν εἰς αὐτοὺς τοὺς λέοντας, καὶ ἰδού εἰσι θανατοῦντες αὐτοὺς, καθότι οὐκ οἴδασι τὸ κρίμα τοῦ Θεοῦ τῆς γῆς.
\VS{27}Καὶ ἐνετείλατο ὁ βασιλεὺς Ἀσσυρίων, λέγων, ἀπαγάγετε ἐκεῖθεν, καὶ πορευέσθωσαν, καὶ κατοικήτωσαν ἐκεῖ, καὶ φωτιοῦσιν αὐτοὺς τὸ κρίμα τοῦ Θεοῦ τῆς γῆς.
\VS{28}Καὶ ἤγαγον ἕνα τῶν ἱερέων ὧν ἀπῴκισαν ἀπὸ Σαμαρείας, καὶ ἐκάθισεν ἐν Βαιθὴλ, καὶ ἦν φωτίζων αὐτοὺς πῶς φοβηθῶσι τὸν Κύριον.
\par }{\PP \VS{29}Καὶ ἦσαν ποιοῦντες ἔθνη ἔθνη θεοὺς αὐτῶν· καὶ ἔθηκαν ἐν οἴκῳ τῶν ὑψηλῶν ὧν ἐποίησαν οἱ Σαμαρεῖται, ἔθνη ἔθνη ἐν ταῖς πόλεσιν αὐτῶν ἐν αἷς κατῴκουν.
\VS{30}Καὶ οἱ ἄνδρες Βαβυλῶνος ἐποίησαν τὴν Σωκχὼθ Βενὶθ, καὶ οἱ ἄνδρες Χοὺθ ἐποίησαν τὴν Ἐργέλ, καὶ οἱ ἄνδρες Αἱμὰθ ἐποίησαν τὴν Ἀσιμὰθ,
\VS{31}καὶ οἱ Εὐαῖοι ἐποίησαν τὴν Ἐβλαζὲρ καὶ τὴν Θαρθὰκ, καὶ ὁ Σεπφαρουαῒμ ἡνίκα κατέκαιον τοὺς υἱοὺς αὐτῶν ἐν πυρὶ τῷ Ἀδραμέλεχ καὶ Ἀνημελὲχ θεοῖς Σεπφαρουαΐμ.
\VS{32}Καὶ ἦσαν φοβούμενοι τὸν Κύριον· καὶ κατῴκισαν τὰ βδελύγματα αὐτῶν ἐν τοῖς οἴκοις τῶν ὑψηλῶν ἃ ἐποίησαν ἐν Σαμαρείᾳ, ἔθνος ἔθνος ἐν πόλει ἐν ᾗ κατῴκουν ἐν αὐτῇ· καὶ ἦσαν φοβούμενοι τὸν Κύριον· καὶ ἐποίησαν ἑαυτοῖς ἱερεῖς τῶν ὑψηλῶν, καὶ ἐποίησαν ἑαυτοῖς ἐν οἴκῳ τῶν ὑψηλῶν.
\VS{33}Καὶ τὸν Κύριον ἐφοβοῦντο, καὶ τοῖς θεοῖς αὐτῶν ἐλάτρευον κατὰ τὸ κρίμα τῶν ἐθνῶν, ὅθεν ἀπῴκισαν αὐτοὺς ἐκεῖθεν.
\par }{\PP \VS{34}Ἕως τῆς ἡμέρας ταύτης αὐτοὶ ἐποίουν κατὰ τὸ κρίμα αὐτῶν· αὐτοὶ φοβοῦνται, καὶ αὐτοὶ ποιοῦσι κατὰ τὰ δικαιώματα αὐτῶν, καὶ κατὰ τὴν κρίσιν αὐτῶν, καὶ κατὰ τὸν νόμον, καὶ κατὰ τὴν ἐντολὴν ἣν ἐνετείλατο Κύριος τοῖς υἱοῖς Ἰακὼβ, οὗ ἔθηκε τὸ ὄνομα αὐτοῦ Ἰσραήλ.
\VS{35}Καὶ διέθετο Κύριος μετʼ αὐτῶν διαθήκην, καὶ ἐνετείλατο αὐτοῖς, λέγων, οὐ φοβηθήσεσθε θεοὺς ἑτέρους, καὶ οὐ προσκυνήσετε αὐτοῖς, καὶ οὐ λατρεύσετε αὐτοῖς, καὶ οὐ θυσιάσετε αὐτοῖς·
\VS{36}Ὅτι ἀλλʼ ἢ τῷ Κυρίῳ ὃς ἀνήγαγεν ὑμᾶς ἐκ γῆς Αἰγύπτου ἐν ἰσχύϊ μεγάλῃ καὶ ἐν βραχίονι ὑψηλῷ, αὐτὸν φοβηθήσεσθε, καὶ αὐτῷ προσκυνήσετε, αὐτῷ θύσετε.
\VS{37}Τὰ δικαιώματα καὶ τὰ κρίματα καὶ τὸν νόμον καὶ τὰς ἐντολὰς, ἃς ἔγραψεν ὑμῖν ποιεῖν, φυλάσσεσθε πάσας τὰς ἡμέρας, καὶ οὐ φοβηθήσεσθε θεοὺς ἑτέρους.
\VS{38}Καὶ τὴν διαθήκην ἣν διέθετο μεθʼ ὑμῶν οὐκ ἐπιλήσεσθε· καὶ οὐ φοβηθήσεσθε θεοὺς ἑτέρους·
\VS{39}Ἀλλʼ ἢ τὸν Κύριον Θεὸν ὑμῶν φοβηθήσεσθε, καὶ αὐτὸς ἐξελεῖται ὑμᾶς ἐκ πάντων τῶν ἐχθρῶν ὑμῶν.
\par }{\PP \VS{40}Καὶ οὐκ ἀκούσεσθε ἐτὶ τῷ κρίματι αὐτῶν, ὃ αὐτοὶ ποιοῦσι.
\VS{41}Καὶ ἦσαν τὰ ἔθνη ταῦτα φοβούμενοι τὸν Κύριον, καὶ τοῖς γλυπτοῖς αὐτῶν ἦσαν δουλεύοντες· καί γε οἱ υἱοὶ καὶ υἱοὶ τῶν υἱῶν αὐτῶν, καθὰ ἐποίησαν οἱ πατέρες αὐτῶν, ποιοῦσιν ἕως τῆς ἡμέρας ταύτης.

\par }\Chap{18}{\PP \VerseOne{1}Καὶ ἐγένετο ἐν ἔτει τρίτῳ τῷ Ὠσηὲ υἱῷ Ἠλὰ βασιλεῖ Ἰσραὴλ ἐβασίλευσεν Ἐζεκίας υἱὸς Ἄχαζ βασιλέως Ἰούδα.
\VS{2}Υἱὸς εἴκοσι καὶ πέντε ἐτῶν ἐν τῷ βασιλεύειν αὐτὸν, καὶ εἴκοσι καὶ ἐννέα ἔτη ἐβασίλευσεν ἐν Ἱερουσαλὴμ, καὶ ὄνομα τῇ μητρὶ αὐτοῦ Ἄβού, θυγάτηρ Ζαχαρίου.
\VS{3}Καὶ ἐποίησε τὸ εὐθὲς ἐν ὀφθαλμοῖς Κυρίου κατὰ πάντα ὅσα ἐποίησε Δαυὶδ ὁ πατὴρ αὐτοῦ.
\VS{4}Αὐτὸς ἐξῆρε τὰ ὑψηλὰ, καὶ συνέτριψε τὰς στήλας, καὶ ἐξωλέθρευσε τὰ ἄλση, καὶ τὸν ὄφιν τὸν χαλκοῦν ὃν ἐποίησε Μωυσῆς, ὅτι ἕως τῶν ἡμερῶν ἐκείνων ἦσαν οἱ υἱοὶ Ἰσραὴλ θυμιῶντες αὐτῷ· καὶ ἐκάλεσεν αὐτὸν Νεεσθάν.
\VS{5}Ἐν Κυρίῳ Θεῷ Ἰσραὴλ ἤλπισεν, καὶ μετʼ αὐτὸν οὐκ ἐγενήθη ὅμοιος αὐτῷ ἐν βασιλεῦσιν Ἰούδα, καὶ ἐν τοῖς γενομένοις ἔμπροσθεν αὐτοῦ.
\VS{6}Καὶ ἐκολλήθη τῷ Κυρίῳ, οὐκ ἀπέστη ὄπισθεν αὐτοῦ, καὶ ἐφύλαξε τὰς ἐντολὰς αὐτοῦ ὅσας ἐνετείλατο Μωυσῇ.
\par }{\PP \VS{7}Καὶ ἦν Κύριος μετʼ αὐτοῦ, καὶ ἐν πᾶσιν οἷς ἐποίει, συνῆκε· καὶ ἠθέτησεν ἐν τῷ βασιλεῖ Ἀσσυρίων, καὶ οὐκ ἐδούλευσεν αὐτῷ.
\VS{8}Αὐτὸς ἐπάταξεν τοὺς ἀλλοφύλους ἕως Γάζης, καὶ ἕως ὁρίου αὐτῆς ἀπὸ πύργου φυλασσόντων καὶ ἕως πόλεως ὀχυρᾶς.
\par }{\PP \VS{9}Καὶ ἐγένετο ἐν τῷ ἔτει τῷ τετάρτῳ βασιλεῖ Ἐζεκίᾳ, αὐτὸς ἐνιαυτὸς ὁ ἕβδομος τῷ Ὠσηὲ υἱῷ Ἠλὰ βασιλεῖ Ἰσραήλ, ἀνέβη Σαλαμανασσὰρ βασιλεὺς Ἀσσυρίων ἐπὶ Σαμάρειαν, καὶ ἐπολιόρκει ἐπʼ αὐτὴν,
\VS{10}καὶ κατελάβετο αὐτὴν ἀπὸ τέλους τριῶν ἐτῶν ἐν ἔτει ἕκτῳ τῷ Ἑζεκίᾳ, αὐτὸς ἐνιαυτὸς ἔννατος τῷ Ὠσηὲ βασιλεῖ Ἰσραὴλ, καὶ συνελήμφθη Σαμάρεια.
\VS{11}Καὶ ἀπῴκισε βασιλεὺς Ἀσσυρίων τὴν Σαμάρειαν εἰς Ἀσσυρίους, καὶ ἔθηκεν αὐτοὺς ἐν Ἁλαὲ καὶ ἐν Ἀβὼρ ποταμῷ Γωζὰν καὶ ὄρὴ Μήδων·
\VS{12}ἀνθʼ ὧν ὅτι οὐκ ἤκουσαν τῆς φωνῆς Κυρίου Θεοῦ αὐτῶν, καὶ παρέβησαν τὴν διαθήκην αὐτοῦ πάντα ὅσα ἐνετείλατο Μωυσῆς ὁ δοῦλος Κυρίου, καὶ οὐκ ἤκουσαν καὶ οὐκ ἐποίησαν.
\par }{\PP \VS{13}Καὶ τῷ τεσσαρεσκαιδεκάτῳ ἔτει ἔτει τοῦ βασιλέως βασιλέως Ἑζεκίου ἀνέβη Σενναχηρὶμ βασιλεὺς Ἀσσυρίων ἐπὶ τὰς πόλεις Ἰούδα τὰς ὀχυρὰς, καὶ συνέλαβεν αὐτάς.
\VS{14}Καὶ ἀπέστειλεν Ἐζεκίας βασιλεὺς Ἰούδα ἀγγέλους πρὸς βασιλέα Ἀσσυρίων εἰς Λαχὶς, λέγων, ἡμάρτηκα, ἀποστράφηθι ἀπʼ ἐμοῦ· ὃ ἐὰν ἐπιθῇς ἐπʼ ἐμὲ, βαστάσω· καὶ ἐπέθηκεν ὁ βασιλεὺς Ἀσσυρίων ἐπὶ Ἑζεκίαν βασιλέα Ἰούδα τριακόσια τάλαντα ἀργυρίου, καὶ τριάκοντα τάλαντα χρυσίου.
\VS{15}Καὶ ἔδωκεν Ἑζεκίας πᾶν τὸ ἀργύριον τὸ εὑρεθὲν ἐν οἴκῳ Κυρίου, καὶ ἐν θησαυροῖς οἴκου τοῦ βασιλέως.
\VS{16}Ἐν τῷ καιρῷ ἐκείνῳ συνέκοψεν Ἑζεκίας τὰς θύρας ναοῦ, καὶ τὰ ἐστηριγμένα ἃ ἐχρύσωσεν Ἑζεκίας ὁ βασιλεὺς Ἰούδα, καὶ ἔδωκεν αὐτὰ βασιλεῖ Ἀσσυρίων.
\par }{\PP \VS{17}Καὶ ἀπέστειλε βασιλεὺς Ἀσσυρίων τὸν Θανθὰν καὶ τὸν Ῥαφὶς καὶ τὸν Ῥαψάκην ἐκ Λαχὶς πρὸς τὸν βασιλέα Ἑζεκίαν ἐν δυνάμει βαρείᾳ ἐπὶ Ἱερουσαλήμ· καὶ ἀνέβησαν καὶ ἦλθον εἰς Ἱερουσαλὴμ, καὶ ἔστησαν ἐν τῷ ὑδραγωγῷ τῆς κολυμβήθρας τῆς ἄνω, ἥ ἐστιν ἐν τῇ ὁδῷ τοῦ ἀγροῦ τοῦ γναφέως.
\VS{18}Καὶ ἐβόησαν πρὸς Ἐζεκίαν· καὶ ἦλθον πρὸς αὐτὸν Ἑλιακὶμ υἱὸς Χελκίου ὁ οἰκονόμος, καὶ Σωμνὰς ὁ γραμματεὺς, καὶ Ἰωὰς ὁ νἱὸς Σαφὰτ ὁ ἀναμιμνήσκων.
\par }{\PP \VS{19}Καὶ εἶπε πρὸς αὐτοὺς Ῥαψάκης, εἴπατε δὴ πρὸς Ἐζεκίαν, τάδε λέγει ὁ βασιλεὺς, ὁ μέγας βασιλεὺς Ἀσσυρίων, τί ἡ πεποίθησις αὕτη ἣν πέποιθας;
\VS{20}Εἶπας, πλὴν λόγοι χειλέων, Βουλὴ καὶ δύναμις εἰς πόλεμον· νῦν οὖν τίνι πεποιθὼς ἠθέτησας ἐν ἐμοί;
\VS{21}Νῦν ἰδοὺ πέποιθας σαυτῷ ἐπὶ τὴν ῥἁβδον τὴν καλαμίνην τὴς τεθλασμένην ταύτην, ἐπʼ Αἴγυπτον; ὃς ἂν στηριχθῇ ἀνὴρ ἐπʼ αὐτὴν, καὶ εἰσελεύσεται εἰς τὴν χεῖρα αὐτοῦ, καὶ τρήσει αὐτήν· οὕτως Φαραὼ βασιλεὺς Αἰγύπτου πᾶσι τοῖς πεποιθόσιν ἐπʼ αὐτόν.
\VS{22}Καὶ ὅτι εἶπας πρὸς μὲ, ἐπὶ Κύριον Θεὸν πεποίθαμεν· οὐχὶ αὐτὸς οὗτος ἀπέστησεν Ἐζεκίας τὰ ὑψηλὰ αὐτοῦ καὶ τὰ θυσιαστήρια αὐτοῦ, καὶ εἶπε τῷ Ἰούδᾳ καὶ τῇ Ἱερουσαλὴμ, ἐνώπιον τοῦ θυσιαστηρίου τούτου προσκυνήσετε ἐν Ἰερουσαλήμ;
\VS{23}Καὶ νῦν μίχθητε δὴ τῷ κυρίῳ μου βασιλεῖ Ἀσσυρίων, καὶ δώσω σοι δισχιλίους ἵππους, εἰ δυνήσῃ δοῦναι σεαυτῷ ἐπιβάτας ἐπʼ αὐτούς.
\VS{24}Καὶ πῶς ἀποστρέψεις τὸ πρόσωπον τοπάρχου ἑνὸς τῶν δούλων τοῦ κυρίου μου τῶν ἐλαχίστων; καὶ ἤλπισας σαυτῷ ἐπʼ Αἴγυπτον εἰς ἅρματα καὶ ἱππεῖς.
\VS{25}Καὶ νῦν μὴ ἄνευ Κυρίου ἀνέβημεν ἐπὶ τὸν τόπον τοῦτον τοῦ διαφθεῖραι αὐτόν; Κύριος εἶπε πρὸς μὲ, ἀνάβηθι ἐπὶ τὴν γῆν ταύτην καὶ διάφθειρον αὐτήν.
\par }{\PP \VS{26}Καὶ εἶπεν Ἑλιακὶμ υἱὸς Χελκίου καὶ Σωμνὰς καὶ Ἰωὰς πρὸς Ῥαψάκην, λάλησον δὴ πρὸς τοὺς παῖδάς σου Συριστὶ, ὅτι ἀκούομεν ἡμεῖς· καὶ οὐ λαλήσεις μεθʼ ἡμῶν Ἰουδαϊστὶ· καὶ ἱνατί λαλεῖς ἐν τοῖς ὠσὶ τοῦ λαοῦ τοῦ ἐπὶ τοῦ τείχους;
\VS{27}Καὶ εἶπε πρὸς αὐτοὺς Ῥαψάκης, μὴ ἐπὶ τὸν κύριόν σου καὶ πρὸς σὲ ἀπέστειλέ με ὁ κύριός μου λαλῆσαι τοὺς λόγους τούτους; οὐχὶ ἐπὶ τοὺς ἄνδρας τοὺς καθημένους ἐπὶ τοῦ τείχους, τοῦ φαγεῖν τὴν κόπρον αὐτῶν, καὶ πιεῖν τὸ οὖρον αὐτῶν μεθʼ ὑμῶν ἅμα;
\par }{\PP \VS{28}Καὶ ἔστη Ῥαψάκης καὶ ἐβόησε φωνῇ μεγάλῃ Ἰουδαϊστὶ, καὶ ἐλάλησε καὶ εἶπεν, ἀκούσατε τοὺς λόγους τοῦ μεγάλου βασιλέως Ἀσσυρίων.
\VS{29}Τάδε λέγει ὁ βασιλεὺς, μὴ ἐπαιρέτω ὑμᾶς Ἐζεκίας λόγοις, ὅτι οὐ μὴ δύνηται ὑμᾶς ἐξελέσθαι ἐκ χειρὸς αὐτοῦ.
\VS{30}Καὶ μὴ ἐπελπιζέτω ὑμᾶς Ἐζεκίας πρὸς Κὺριον, λέγων, ἐξαιρούμενος ἐξελεῖται Κύριος, οὐ μὴ παραδοθῇ ἡ πόλις αὕτη ἐν χειρὶ βασιλέως Ἀσσυρίων.
\VS{31}Μὴ ἀκούετε Ἐζεκίου· ὅτι τάδε λέγει ὁ βασιλεὺς Ἀσσυρίων, ποιήσατε μετʼ ἐμοῦ εὐλογίαν, καὶ ἐξέλθατε πρὸς μὲ, καὶ πίεται ἀνὴρ τὴν ἄμπελον αὐτοῦ, καὶ ἀνὴρ τὴν συκῆν αὐτοῦ φάγεται, καὶ πίεται ὕδωρ τοῦ λάκκου αὐτοῦ,
\VS{32}ἕως ἔλθω καὶ λάβω ὑμᾶς εἰς γῆν ὡς γῆ ὑμῶν, γῆ σίτου καὶ οἴνου καὶ ἄρτου καὶ ἀμπελώνων, γῆ ἐλαίας ἐλαίου καὶ μέλιτος, καὶ ζήσετε καὶ οὐ μὴ ἀποθάνητε· καὶ μὴ ἀκούετε Ἐζεκίου, ὅτι ἀπατᾷ ὑμᾶς, λέγων, Κύριος ῥύσεται ὑμᾶς.
\VS{33}Μὴ ῥυόμενοι ἐῤῥύσαντο οἱ θεοὶ τῶν ἐθνῶν ἕκαστος τὴν ἑαυτοῦ χώραν ἐκ χειρὸς βασιλέως Ἀσσυρίων;
\VS{34}Ποῦ ἐστιν ὁ θεὸς Αἱμὰθ, καὶ Ἀρφάδ; ποῦ ἐστιν ὁ θεὸς Σεπφαρουαῒμ, Ἀνὰ, καὶ Ἀβὰ, ὅτι ἐξείλαντο Σαμάρειαν ἐκ χειρός μου;
\VS{35}Τίς ἐν πᾶσι τοῖς θεοῖς τῶν γαιῶν, οἳ ἐξείλαντο τὰς γᾶς αὐτῶν ἐκ χειρός μου, ὅτι ἐξελεῖται Κύριος τὴν Ἱερουσαλὴμ ἐκ χειρός μου;
\par }{\PP \VS{36}Καὶ ἐκώφευσαν καὶ οὐκ ἀπεκρίθησαν αὐτῷ λόγον, ὅτι ἐντολὴ τοῦ βασιλέως, λέγων, οὐκ ἀποκριθήσεσθε αὐτῷ.
\VS{37}Καὶ εἰσῆλθεν Ἐλιακὶμ υἱὸς Χελκίου ὁ οἰκονόμος, καὶ Σωμνὰς ὁ γραμματεὺς, καὶ Ἰωὰς υἱὸς Σαφὰτ ὁ ἀναμιμνήσκων πρὸς Ἐζεκίαν, διεῤῥηχότες τὰ ἱμάτια, καὶ ἀνήγγειλαν αὐτῷ τοὺς λόγους Ῥαψάκου.

\par }\Chap{19}{\PP \VerseOne{1}Καὶ ἐγένετο ὡς ἤκουσεν ὁ βασιλεὺς Ἐζεκίας, καὶ διέῤῥηξε τὰ ἱμάτια αὐτοῦ, καὶ περιεβάλετο σάκκον, καὶ εἰσῆλθεν εἰς οἶκον Κυρίου.
\VS{2}Καὶ ἀπέστειλεν Ἑλιακὶμ τὸν οἰκονόμον, καὶ Σωμνὰν τὸν γραμματέα, καὶ τοὺς πρεσβυτέρους τῶν ἱερέων περιβεβλημένους σάκκους, πρὸς Ἡσαΐαν τὸν προφήτην υἱὸν Ἀμώς.
\VS{3}Καὶ εἶπον πρὸς αὐτὸν, τάδε λέγει Ἐζεκίας, ἡμέρα θλίψεως καὶ ἐλεγμοῦ καὶ παροργισμοῦ ἡ ἡμέρα αὕτη· ὅτι ἦλθον υἱοὶ ἕως ὠδίνων, καὶ ἰσχὺς οὐκ ἔστι τῇ τικτούσῃ.
\VS{4}Εἴ πως εἰσακούσεται Κύριος ὁ Θεός σου πάντας τοὺς λόγους Ῥαψάκου, ὃν ἀπέστειλεν αὐτὸν βασιλεὺς Ἀσσυρίων ὁ κύριος αὐτοῦ ὀνειδίζειν Θεὸν ζῶντα, καὶ βλασφημεῖν ἐν λόγοις οἷς ἤκουσε Κύριος ὁ Θεός σου, καὶ λήψῃ προσευχὴν περὶ τοῦ λείμματος τοῦ εὑρισκομένου.
\par }{\PP \VS{5}Καὶ ἦλθον οἱ παῖδες τοῦ βασιλέως Ἐζεκίου πρὸς Ἡσαΐαν.
\VS{6}Καὶ εἶπεν αὐτοῖς Ἡσαΐας, Τάδε ἐρεῖτε πρὸς τὸν κύριον ὑμῶν, τάδε λέγει Κύριος, μὴ φοβηθῇς ἀπὸ τῶν λόγων ὧν ἤκουσας, ὧν ἐβλασφήμησαν τὰ παιδάρια βασιλέως Ἀσσυρίων.
\VS{7}Ἰδοὺ ἐγὼ δίδωμι ἐν αὐτῷ πνεῦμα, καὶ ἀκούσεται ἀγγελίαν, καὶ ἀποστραφήσεται εἰς τὴν γῆν αὐτοῦ· καὶ καταβαλῶ αὐτὸν ἐν ῥομφαίᾳ ἐν τῇ γῇ αὐτοῦ.
\par }{\PP \VS{8}Καὶ ἐπέστρεψε Ῥαψάκης, καὶ εὗρε τὸν βασιλέα Ἀσσυρίων πολεμοῦντα ἐπὶ Λοβνὰ, ὅτι ἤκουσεν ὅτι ἀπῇρεν ἐκ Λαχίς.
\VS{9}Καὶ ἤκουσε περὶ Θαρακὰ βασιλέως Αἰθιόπων, λέγων, ἰδοὺ ἐξῆλθε πολεμεῖν μετὰ σοῦ· καὶ ἐπέστρεψε, καὶ ἀπέστειλεν ἀγγέλους πρὸς Ἐζεκίαν, λέγων,
\VS{10}μὴ ἐπαιρέτω σε ὁ Θεός σου ἐφʼ ᾧ σὺ πέποιθας ἐν αὐτῷ, λέγων, οὐ μὴ παραδοθῇ Ἱερουσαλὴμ εἰς χεῖρας βασιλέως Ἀσσυρίων.
\VS{11}Ἰδοὺ σὺ ἤκουσας πάντα ὅσα ἐποίησαν βασιλεῖς Ἀσσυρίων πάσαις ταῖς γαίαις τοῦ ἀναθεματίσαι αὐτάς· καὶ σὺ ῥυσθήσῃ;
\VS{12}Μὴ ἐξαιρούμενοι ἐξείλαντο αὐτοὺς οἱ θεοὶ τῶν ἐθνῶν, οὓς διέφθειραν οἱ πατρές μου, τήν τε Γωζὰν, καὶ τὴν Χαῤῥὰν, καὶ τὴν Ῥαφὶς, καὶ υἱοὺς Ἐδὲμ τοὺς ἐν Θαεσθέν;
\VS{13}Ποῦ ἐστιν ὁ βασιλεὺς Αἱμὰθ, καὶ ὁ βασιλεὺς Ἀρφάδ; καὶ ποῦ ἐστιν ὁ βασιλεὺς τῆς πόλεως Σεπφαρουαῒν, Ἀνὰ, καὶ Ἀβά;
\par }{\PP \VS{14}Καὶ ἔλαβεν Ἐζεκίας τὰ βιβλία ἐκ χειρὸς τῶν ἀγγέλων, καὶ ἀνέγνω αὐτά· καὶ ἀνέβη εἰς οἶκον Κυρίου, καὶ ἀνέπτυξεν αὐτὰ Ἐζεκίας ἐναντίον Κυρίου,
\VS{15}καὶ εἶπε, Κύριε ὁ Θεὸς Ἰσραὴλ ὁ καθήμενος ἐπὶ τῶν χερουβὶμ, σὺ εἶ ὁ Θεὸς μόνος ἐν πάσαις ταῖς βασιλείαις τῆς γῆς, σὺ ἐποίησας τὸν οὐρανὸν καὶ τὴν γῆν.
\VS{16}Κλῖνον Κύριε τὸ οὖς σου καὶ ἄκουσον, ἄνοιξον Κύριε τοὺς ὀφθαλμούς σου καὶ ἴδε, καὶ ἄκουσον τοὺς λόγους Σενναχηρὶμ οὓς ἀπέστειλεν ὀνειδίζειν Θεὸν ζῶντα.
\VS{17}Ὅτι ἀληθείᾳ Κύριε ἠρήμωσαν βασιλεῖς Ἀσσυρίων τὰ ἔθνη,
\VS{18}καὶ ἔδωκαν τοὺς θεοὺς αὐτῶν εἰς τὸ πῦρ, ὅτι οὐ θεοί εἰσιν, ἀλλʼ ἢ ἔργα χειρῶν ἀνθρώπων ξύλα καὶ λίθος· καὶ ἀπώλεσαν αὐτούς.
\VS{19}Καὶ νῦν, Κύριε ὁ Θεὸς ἡμῶν, σῶσον ἡμᾶς ἐκ χειρὸς αὐτοῦ, καὶ γνώσονται πᾶσαι αἱ βασιλεῖαι τῆς γῆς, ὅτι σὺ Κύριος ὁ Θεὸς μόνος.
\par }{\PP \VS{20}Καὶ ἀπέστειλεν Ἡσαΐας υἱὸς Ἀμὼς πρὸς Ἐζεκίαν, λέγων, τάδε λέγει Κύριος ὁ Θεὸς τῶν δυνάμεων Θεὸς Ἰσραὴλ, ἃ προσηύξω πρὸς μὲ περὶ Σενναχηρὶμ βασιλέως Ἀσσυρίων, ἤκουσα.
\VS{21}Οὗτος ὁ λόγος ὃν ἐλάλησε Κύριος ἐπʼ αὐτὸν, ἐξουδένησέ σε καὶ ἐμυκτήρισέ σε παρθένος θυγάτηρ Σιὼν, ἐπὶ σοὶ κεφαλὴν αὐτῆς ἐκίνησε θυγάτηρ Ἱερουσαλήμ.
\VS{22}Τίνα ὠνείδισας, καὶ τίνα ἐβλασφήμησας; καὶ ἐπὶ τίνα ὕψωσας φωνὴν, καὶ ἦρας εἰς ὕψος τοὺς ὀφθαλμούς σου; εἰς τὸν ἅγιον τοῦ Ἰσραήλ;
\par }{\PP \VS{23}Ἐν χειρὶ ἀγγέλων σου ὠνείδισας Κύριον, καὶ εἶπας, ἐν τῷ πλήθει τῶν ἁρμάτων μου ἐγὼ ἀναβήσομαι εἰς ὕψος ὀρέων μηροὺς τοῦ Λιβάνου, καὶ ἔκοψα τὸ μέγεθος τῆς κέδρου αὐτοῦ, τὰ ἐκλεκτὰ κυπαρίσσων αὐτοῦ, καὶ ἦλθον εἰς μέσον δρυμοῦ καὶ Καρμήλου.
\VS{24}Ἐγὼ ἔψυξα καὶ ἔπιον ὕδατα ἀλλότρια, καὶ ἐξερήμωσα τῷ ἴχνει τοῦ ποδός μου πάντας ποταμοὺς περιοχῆς.
\VS{25}Ἔπλασα αὐτὴν, συνήγαγον αὐτήν· καὶ ἐγενήθη εἰς ἐπάρσεις ἀποικεσιῶν μαχίμων πόλεις ὀχυράς.
\VS{26}Καὶ οἱ ἐνοικοῦντες ἐν αὐταῖς ἠσθένησαν τῇ χειρὶ, ἔπτηξαν καὶ κατῃσχύνθησαν· ἐγένοντο χόρτος ἀγροῦ, ἢ χλωρὰ βοτάνη, χλόη δωμάτων, καὶ πάτημα ἀπέναντι ἑστηκότος.
\VS{27}Καὶ τὴν καθέδραν σου καὶ τὴν ἔξοδόν σου ἔγνων, καὶ τὸν θυμόν σου ἐπʼ ἐμὲ,
\VS{28}διὰ τὸ ὀργισθῆναί σε ἐπʼ ἐμὲ, καὶ τὸ στρῆνός σου ἀνέβη ἐν τοῖς ὠσί μου· καὶ θήσω τὰ ἄγκιστρά μου ἐν τοῖς μυκτῆρσί σου, καὶ χαλινὸν ἐν τοῖς χείλεσί σου, καὶ ἀποστρέψω σε ἐν τῇ ὁδῷ ᾗ ἦλθες ἐν αὐτῇ.
\par }{\PP \VS{29}Καὶ τοῦτό σοι τὸ σημεῖον· φάγε τοῦτον τὸν ἐνιαυτὸν αὐτόματα, καὶ τῷ ἔτει τῷ δευτέρῳ τὰ ἀνατέλλοντα, καὶ ἔτει τρίτῳ σπορὰ καὶ ἀμητὸς καὶ φυτεία ἀμπελώνων, καὶ φάγεσθε τὸν καρπὸν αὐτῶν.
\VS{30}Καὶ προσθήσει τὸν διασεσωσμένον οἴκου Ἰούδα τὸ ὑπολειφθὲν ῥίζαν κάτω, καὶ ποιήσει καρπὸν ἄνω.
\VS{31}Ὅτι ἐξ Ἱερουσαλὴμ ἐξελεύσεται κατάλειμμα, καὶ ἀνασωζόμενος ἐξ ὄρους Σιών· ὁ ζῆλος Κυρίου τῶν δυνάμεων ποιήσει τοῦτο.
\VS{32}Οὐχ οὕτως;
\par }{\PP Τάδε λέγει Κύριος πρὸς βασιλέα Ἀσσυρίων, οὐκ εἰσελεύσεται εἰς τὴν πόλιν ταύτην, καὶ οὐ τοξεύσει ἐκεῖ βέλος, καὶ οὐ προφθάσει ἐπʼ αὐτὴν θυρεὸς, καὶ οὐ μὴ ἐκχέῃ πρὸς αὐτὴν πρόσχωμα.
\par }{\PP \VS{33}Τῇ ὁδῷ ᾗ ἦλθεν, ἐν αὐτῇ ἀποστραφήσεται, καὶ εἰς τὴν πόλιν ταύτην οὐκ εἰσελεύσεται, λέγει Κύριος.
\VS{34}Καὶ ὑπερασπιῶ ὑπὲρ τῆς πόλεως ταύτης διʼ ἐμὲ καὶ διὰ Δαυὶδ τὸν δοῦλόν μου.
\par }{\PP \VS{35}Καὶ ἐγένετο νυκτὸς, καὶ ἐξῆλθεν ἄγγελος Κυρίου καὶ ἐπάταξεν ἐν τῇ παρεμβολῇ τῶν Ἀσσυρίων ἑκατὸν ὀγδοηκονταπέντε χιλιάδας· καὶ ὤρθρισαν τοπρωῒ, καὶ ἰδοὺ πάντες σώματα νεκρά.
\VS{36}Καὶ ἀπῇρε καὶ ἐπορεύθη καὶ ἀπέστρεψε Σενναχηρὶμ βασιλεὺς Ἀσσυρίων, καὶ ᾤκησεν ἐν Νινευῆ.
\VS{37}Καὶ ἐγένετο αὐτοῦ προσκυνοῦντος ἐν οἴκῳ Μεσερὰχ τοῦ θεοῦ αὐτοῦ, καὶ Ἀδραμέλεχ καὶ Σαρασὰρ οἱ υἱοὶ αὐτοῦ ἐπάταξαν αὐτὸν ἐν μαχαίρᾳ· καὶ αὐτοὶ ἐσώθησαν εἰς γῆν Ἀραράθ· καὶ ἐβασίλευσεν Ἀσορδὰν ὁ υἱὸς αὐτοῦ ἀντʼ αὐτοῦ.

\par }\Chap{20}{\PP \VerseOne{1}Ἐν ταῖς ἡμέραις ἐκείναις ἠῤῥώστησεν Ἐζεκίας εἰς θάνατον· καὶ εἰσῆλθε πρὸς αὐτὸν Ἡσαΐας υἱὸς Ἀμὼς ὁ προφήτης, καὶ εἶπε πρὸς αὐτὸν, τάδε λέγει Κύριος, ἔντειλαι τῷ οἴκῳ σου, ἀποθνήσκεις σὺ καὶ οὐ ζήσῃ.
\VS{2}Καὶ ἐπέστρεψεν Ἐζεκίας πρὸς τὸν τοῖχον, καὶ ηὔξατο πρὸς Κύριον λέγων
\VS{3}Κύριε, μνήσθητι δὴ ὅσα περιεπάτησα ἐνώπιόν σου ἐν ἀληθείᾳ καὶ καρδίᾳ πλήρει, καὶ τὸ ἀγαθὸν ἐν ὀφθαλμοῖς σου ἐποίησα· καὶ ἔκλαυσεν Ἐζεκίας κλαυθμῷ μεγάλῳ.
\par }{\PP \VS{4}Καὶ ἦν Ἡσαΐας ἐν τῇ αὐλῇ τῇ μέσῃ, καὶ ῥῆμα Κυρίου ἐγένετο πρὸς αὐτὸν, λέγων,
\VS{5}ἐπίστρεψον, καὶ ἐρεῖς πρὸς Ἐζεκίαν τὸν ἡγούμενον τοῦ λαοῦ μου, τάδε λέγει Κύριος ὁ Θεὸς Δαυὶδ τοῦ πατρός σου, ἤκουσα τῆς προσευχῆς σου, εἶδον τὰ δάκρυά σου· ἰδοὺ ἐγὼ ἰάσομαί σε· τῇ ἡμέρᾳ τῇ τρίτῃ ἀναβήσῃ εἰς οἶκον Κυρίου.
\VS{6}Καὶ προσθήσω ἐπὶ τὰς ἡμέρας σου πεντεκαίδεκα ἔτη· καὶ ἐκ χειρὸς· βασιλέως Ἀσσυρίων σώσω σε καὶ τὴν πόλιν ταύτην, καὶ ὑπερασπιῶ ὑπὲρ τῆς πόλεως ταύτης διʼ ἐμὲ καὶ διὰ Δαυὶδ τὸν δοῦλόν μου.
\VS{7}Καὶ εἶπε, λαβέτωσαν παλάθην σύκων, καὶ ἐπιθέτωσαν ἐπὶ τὸ ἕλκος, καὶ ὑγιάσει.
\VS{8}Καὶ εἶπεν Ἐζεκίας πρὸς Ἡσαΐαν, τί τὸ σημεῖον ὅτι ἰάσεταί με Κύριος, καὶ ἀναβήσομαι εἰς οἶκον Κυρίου τῇ ἡμέρᾳ τῇ τρίτῃ;
\VS{9}Καὶ εἶπεν Ἡσαΐας, τοῦτο τὸ σημεῖον παρὰ Κυρίου, ὅτι ποιήσει Κύριος τὸν λόγον ὃν ἐλάλησε· πορεύσεται ἡ σκιὰ δέκα βαθμοὺς, ἐὰν ἐπιστρέψῃ δέκα βαθμούς.
\VS{10}Καὶ εἶπεν Ἐζεκίας, κοῦφον τὴν σκιὰν κλῖναι δέκα βαθμούς· οὐχὶ, ἀλλʼ ἐπιστραφήτω ἡ σκιὰ ἐν τοῖς ἀναβαθμοῖς δέκα βαθμοὺς εἰς τὰ ὀπίσω.
\VS{11}Καὶ ἐβόησεν Ἡσαΐας ὁ προφήτης πρὸς Κύριον, καὶ ἐπέστρεψεν ἡ σκιὰ ἐν τοῖς ἀναβαθμοῖς εἰς τὰ ὀπίσω δέκα βαθμούς.
\par }{\PP \VS{12}Ἐν τῷ καιρῷ ἐκείνῳ ἀπέστειλε Μαρωδὰχ Βαλαδὰν υἱὸς Βαλαδὰν βασιλεὺς Βαβυλῶνος βιβλία καὶ μαναὰ πρὸς Ἐζεκίαν, ὅτι ἤκουσεν ὅτι ἠῤῥώστησεν Ἐζεκίας.
\VS{13}Καὶ ἐχάρη ἐπʼ αὐτοῖς Ἐζεκίας, καὶ ἔδειξεν αὐτοῖς ὅλον τὸν οἶκον τοῦ νεχωθὰ, τὸ ἀργύριον καὶ τὸ χρυσίον, τὰ ἀρώματα καὶ τὸ ἔλαιον τὸ ἀγαθὸν, καὶ τὸν οἶκον τῶν σκευῶν, καὶ ὅσα εὑρέθη ἐν τοῖς θησαυροῖς αὐτοῦ· οὐκ ἦν λόγος ὃν οὐκ ἔδειξεν αὐτοῖς Ἐζεκίας ἐν τῷ οἴκῳ αὐτοῦ καὶ ἐν πάσῃ τῇ ἐξουσίᾳ αὐτοῦ.
\par }{\PP \VS{14}Καὶ εἰσῆλθεν Ἡσαΐας ὁ προφήτης πρὸς τὸν βασιλέα Ἑζεκίαν, καὶ εἶπε πρὸς αὐτόν, τί ἐλάλησαν οἱ ἄνδρες οὖτοι, καὶ πόθεν ἥκασι πρὸς σέ; καὶ εἶπεν Ἐζεκίας, ἐκ γῆς πόῤῥωθεν ἥκασιν πρὸς μὲ, ἐκ Βαβυλῶνος.
\VS{15}Καὶ εἶπε, τί εἶδον ἐν τῷ οἴκῳ σου; καὶ εἶπε πάντα ὅσα ἐν τῷ οἴκῳ μου εἶδον· οὐκ ἦν ἐν τῷ οἴκῳ μου ὃ οὐκ ἔδειξα αὐτοῖς, ἀλλὰ καὶ τὰ ἐν τοῖς θησαυροῖς μου.
\VS{16}Καὶ εἶπεν Ἡσαΐας πρὸς Ἐζεκίαν, ἄκουσον λόγον Κυρίου,
\VS{17}ἰδοὺ ἡμέραι ἔρχονται, καὶ ληφθήσεται πάντα τὰ ἐν τῷ οἴκῳ σου, καὶ ὅσα ἐθησαύρισαν οἱ πατέρες σου ἕως τῆς ἡμέρας ταύτης, εἰς Βαβυλῶνα· καὶ οὐχ ὑπολειφθήσεται ῥῆμα ὃ εἶπε Κύριος.
\VS{18}Καὶ οἱ υἱοί σου οἳ ἐξελεύσονται ἐκ σοῦ οὓς γεννήσεις, λήψεται, καὶ ἔσονται εὐνοῦχοι ἐν τῷ οἴκῳ τοῦ βασιλέως Βαβυλῶνος.
\VS{19}Καὶ εἶπεν Ἐζεκίας πρὸς Ἡσαΐαν, ἀγαθὸς ὁ λόγος Κυρίου ὃν ἐλάλησεν· ἔστω εἰρήνη ἐν ταῖς ἡμέραις μου.
\par }{\PP \VS{20}Καὶ τὰ λοιπὰ τῶν λόγων Ἐζεκίου καὶ πᾶσα ἡ δυναστεία αὐτοῦ καὶ ὅσα ἐποίησε, τὴν κρήνην καὶ τὸν ὑδραγωγόν, καὶ εἰσήγαγε τὸ ὕδωρ εἰς τὴν πόλιν, οὐχὶ ταῦτα γεγραμμένα ἐπὶ βιβλίῳ λόγων τῶν ἡμερῶν τοῖς βασιλεῦσιν Ἰούδα;
\VS{21}Καὶ ἐκοιμήθη Ἐζεκίας μετὰ τῶν πατέρων αὐτοῦ, καὶ ἐβασίλευσε Μανασσῆς υἱὸς αὐτοῦ ἀντʼ αὐτοῦ.

\par }\Chap{21}{\PP \VerseOne{1}Υἱὸς δώδεκα ἐτῶν Μανασσὴς ἐν τῷ βασιλεύειν αὐτόν, καὶ πεντήκοντα καὶ πέντε ἔτη ἐβασίλευσεν ἐν Ἰερουσαλήμ, καὶ ὄνομα τῇ μητρὶ αὐτοῦ Ἁψιβά.
\VS{2}Καὶ ἐποίησε τὸ πονηρὸν ἐν ὀφθαλμοῖς Κυρίου, κατὰ τὰ βδελύγματα τῶν ἐθνῶν ὧν ἐξῆρε Κύριος ἀπὸ προσώπον τῶν υἱῶν Ἰσραήλ.
\VS{3}Καὶ ἐπέστρεψε καὶ ᾠκοδόμησε τὰ ὑψηλὰ ἃ κατέσπασεν Ἑζεκίας ὁ πατὴρ αὐτοῦ, καὶ ἀνέστησε θυσιαστήριον τῇ Βάαλ, καὶ ἐποίησε τὰ ἄλση καθὼς ἐποίησεν Ἀχαὰβ βασιλεὺς Ἰσραήλ, καὶ προσεκύνησε πάσῃ τῇ δυνάμει τοῦ οὐρανοῦ, καὶ ἐδούλευσεν αὐτοῖς.
\VS{4}Καὶ ᾠκοδόμησε θυσιαστήριον ἐν ὄκῳ Κυρίου, ὡς εἶπεν, ἐν Ἰερουσάλὴμ θήσω τὸ ὄνομά μου.
\VS{5}Καὶ ᾠκοδόμησε θυσιαστήριον πάσῃ τῇ δυνάμει τοῦ οὐρανοῦ ἐν ταῖς δυσὶν αὐλαῖς οἴκου Κυρίου.
\VS{6}Καὶ διῆγε τοὺς υἱοὺς αὐτοῦ ἐν πυρί, καὶ ἐκληδονίζετο καὶ οἰωνίζετο, καὶ ἐποίησε τεμένη, καὶ γνώστας ἐπλήθυνε τοῦ ποιεῖν τὸ πονηρὸν ἐν ὀφθαλμοῖς Κυρίου παροργίσαι αὐτόν.
\VS{7}Καὶ ἔθηκε τὸ γλυπτὸν τοῦ ἄλσους ἐν τῷ οἴκῳ ᾧ εἶπε Κύριος πρὸς Δαυὶδ καὶ πρὸς Σαλωμὼν τὸν υἱὸν αὐτοῦ, ἐν τῷ οἴκῳ τούτῳ καὶ ἐν Ἰερουσαλὴμ ᾗ ἐξελεξάμην ἐκ πασῶν φυλῶν τοῦ Ἰσραὴλ καὶ θήσω τὸ ὄνομά μου εἰς τὸν αἰῶνα,
\VS{8}καὶ οὐ προσθήσω τοῦ σαλεῦσαι τὸν πόδα Ἰσραὴλ ἀπὸ τῆς γῆς ἧς ἔδωκα τοῖς πατράσιν αὐτῶν, οἵ τινες φυλάξουσι πάντα ὅσα ἐνετειλάμην κατὰ πᾶσαν τὴν ἐντολὴν ἣν ἐνετείλατο αὐτοῖς ὁ δοῦλός μου Μωυσῆς.
\VS{9}Καὶ οὐκ ἤκουσαν, καὶ ἐπλάνησεν αὐτοὺς Μανασσῆς τοῦ ποιῆσαι τὸ πονηρὸν ἐν ὀφθαλμοῖς Κυρίου ὑπὲρ τὰ ἔθνη, ἃ ἠθάνισε Κύριος ἐκ προσώπου υἱῶν Ἰσραήλ.
\par }{\PP \VS{10}Καὶ ἐλάλησε Κύριος ἐν χειρὶ δούλων αὐτοῦ τῶν προφητῶν, λέγων,
\VS{11}ἀνθʼ ὧν ὅσα ἐποίησε Μανασσῆς ὁ βασιλεὺς Ἰούδα τὰ βδελύγματα ταῦτα τὰ πονηρὰ ἀπὸ πάντων ὧν ἐποίησεν ὁ Ἀμοῤῥαῖος ὁ ἔμπροσθεν, καὶ ἐξήμαρτε καί γε Ἰούδαν ἐν τοῖς εἰδώλοις αὐτῶν,
\VS{12}οὐχ οὕτως· τάδε λέγει Κύριος ὁ Θεὸς Ἰσραήλ, ἰδοὺ ἐγὼ φέρω κακὰ ἐπὶ Ἰερουσαλὴμ καὶ Ἰούδαν, ὥστε παντὸς ἀκούοντος ἠχήσει ἀμφότερα τὰ ὦτα αὐτοῦ·
\VS{13}Καὶ ἐκτενῶ ἐπὶ Ἱερουσαλὴμ τὸ μέτρον Σαμαρείας καὶ τὸ στάθμιον οἴκου Ἀχαάβ· καὶ ἀπαλείψω τὴν Ἱερουσαλὴμ, καθὼς ἀπαλείφεται ὁ ἀλάβαστρος ἀπαλειφόμενος καὶ καταστρέφεται ἐπὶ πρόσωπον αὐτοῦ.
\VS{14}Καὶ ἀπεώσομαι τὸ ὑπόλειμμα τῆς κληρονομίας μου, καὶ παραδώσω αὐτοὺς εἰς χεῖρας ἐχθρῶν αὐτῶν, καὶ ἔσονται εἰς διαρπαγὴν καὶ εἰς προνομὴν πᾶσι τοῖς ἐχθροῖς αὐτῶν,
\VS{15}ἀνθʼ ὧν ὅσα ἐποίησαν τὸ πονηρὸν ἐν ὀφθαλμοῖς μου, καὶ ἦσαν παροργίζοντές με ἀπὸ τῆς ἡμέρας ἧς ἐξήγαγον τοὺς πατέρας αὐτῶν ἐξ Αἰγύπτου καὶ ἕως τῆς ἡμέρας ταύτης.
\VS{16}Καί γε αἷμα ἀθῶον ἐξέχεε Μανασσῆς πολὺ σφόδρα ἕως οὗ ἔπλησε τὴν Ἱερουσαλὴμ στόμα εἰς στόμα, πλὴν ἀπὸ τῶν ἁμαρτιῶν αὐτοῦ ὧν ἐξήμαρτε τὸν Ἰούδαν τοῦ ποιῆσαι τὸ πονηρὸν ἐν ὀφθαλμοῖς Κυρίου.
\par }{\PP \VS{17}Καὶ τὰ λοιπὰ τῶν λόγων Μανασσῆ καὶ πάντα ὅσα ἐποίησε, καὶ ἡ ἁμαρτία αὐτοῦ ἣν ἥμαρτεν, οὐχὶ ταῦτα γεγραμμένα ἐπὶ βιβλίῳ λόγων τῶν ἡμερῶν τοῖς βασιλεῦσιν Ἰούδα;
\VS{18}Καὶ ἐκοιμήθη Μανασσὴς μετὰ τῶν πατέρων αὐτοῦ, καὶ ἐτάφη ἐν τῷ κήπῳ τοῦ οἴκου αὐτοῦ ἐν κήπῳ Ὀζά· καὶ ἐβασίλευσεν Ἀμὼς υἱὸς αὐτοῦ ἀντʼ αὐτοῦ.
\par }{\PP \VS{19}Υἱὸς εἴκοσι καὶ δύο ἐτῶν Ἀμὼς ἐν τῷ βασιλεύειν αὐτόν, καὶ δύο ἔτη ἐβασίλευσεν ἐν Ἱερουσαλήμ, καὶ ὄνομα τῇ μητρὶ αὐτοῦ Μεσολλάμ, θυγάτηρ Ἀροὺς ἐξ Ἰετέβα.
\VS{20}Καὶ ἐποίησε τὸ πονηρὸν ἐν ὀφθαλμοῖς Κυρίου καθὼς ἐποίησε Μανασσὴς, ὁ πατὴρ αὐτοῦ.
\VS{21}Καὶ ἐπορεύθη ἐν πάσῃ ὁδῷ ᾗ ἐπορεύθη ὁ πατὴρ αὐτοῦ, καὶ ἐλάτρευσε τοῖς εἰδώλοις οἷς ἐλάτρευσεν ὁ πατὴρ αὐτοῦ, καὶ προσεκύνησεν αὐτοῖς.
\VS{22}Καὶ ἐγκατέλιπε τὸν Κύριον Θεὸν τῶν πατέρων αὐτοῦ, καὶ οὐκ ἐπορεύθη ἐν ὁδῷ Κυρίου.
\VS{23}Καὶ συνεστράφησαν οἱ παῖδες Ἀμὼς πρὸς αὐτὸν, καὶ ἐθανάτωσαν τὸν βασιλέα ἐν τῷ οἴκῳ αὐτοῦ.
\VS{24}Καὶ ἐπάταξεν ὁ λαὸς τῆς γῆς πάντας τοὺς συστραφέντας ἐπὶ τὸν βασιλέα Ἀμώς, καὶ ἐβασίλευσεν ὁ λαὸς τῆς γῆς τὸν Ἰωσίαν υἱὸν αὐτοῦ ἀντʼ αὐτοῦ.
\par }{\PP \VS{25}Καὶ τὰ λοιπὰ τῶν λόγων Ἀμὼς ὅσα ἐποίησεν, οὐκ ἰδοὺ ταῦτα γεγραμμένα ἐπὶ βιβλίῳ λόγων τῶν ἡμερῶν τοῖς βασιλεῦσιν Ἰούδα;
\VS{26}Καὶ ἔθαψαν αὐτὸν ἑν τῷ τάφῳ αὐτοῦ ἑν τῷ κήτῳ Ὀζά, καὶ ἐβασίλευσεν Ἰωσίας υἱὸς αὐτοῦ ἀντʼ αὐτοῦ.

\par }\Chap{22}{\PP \VerseOne{1}Υἱὸς ὀκτὼ ἐτῶν Ἰωσείας ἐν τῷ βασιλεύειν αὐτὸν, καὶ τριάκοντα καὶ ἓν ἔτος ἐβασίλευσεν ἐν Ἰερουσαλὴμ, καὶ ὄνομα τῇ μητρὶ αὐτοῦ Ἰεδία, θυλάτηρ Ἐδεϊὰ ἐκ Βασουρώθ.
\VS{2}Καὶ ἐποίησε τὸ εὐθὲς ἐν ὀφθαλμοῖς Κυρίου, καὶ ἐπορεύθη ἐν πάσῃ ὁδῷ Δαυὶδ τοῦ πατρὸς αὐτοῦ, οὐκ ἀπέστη δεξιὰ καὶ ἀριστερά.
\par }{\PP \VS{3}Καὶ ἐγενήθη ἐν τῷ ὀκτωκαιδεκάτῳ ἔτει τῷ βασιλεῖ Ἰωσείᾳ, ἐν τῷ μηνὶ τῷ ὀγδόῳ ἀπέστειλεν ὁ βασιλεὺς τὸν Σαπφὰν υἱὸν Ἐζελίου υἱοῦ Μεσολλὰμ τὸν γραμματέα οἴκου Κυρίου, λέγων,
\VS{4}ἀνάβηθι πρὸς Χελκίαν τὸν ἱερέα τὸν μέγαν, καὶ σφράγισον τὸ ἀργύριον τὸ εἰσενεχθὲν ἐν οἴκῳ Κυρίου, ὃ συνήγαγον οἱ φυλάσσοντες τὸν σταθμὸν παρὰ τοῦ λαοῦ.
\VS{5}Καὶ δότωσαν αὐτὸ ἐπὶ χεῖρα ποιούντων τὰ ἔργα τῶν καθεσταμένων ἐν οἴκῳ Κυρίου· καὶ ἔδωκεν αὐτὸ τοῖς ποιοῦσι τὰ ἔργα τοῖς ἐν οἴκῳ Κυρίου τοῦ κατισχῦσαι τὸ βεδὲκ τοῦ οἴκου,
\VS{6}τοῖς τέκτοσι καὶ τοῖς οἰκοδόμοις καὶ τοῖς τειχισταῖς, καὶ τοῦ κτήσασθαι ξύλα καὶ λίθους λατομητοὺς, τοῦ κραταιῶσαι τὸ βεδὲκ τοῦ οἴκου.
\VS{7}Πλὴν οὐκ ἐξελογίζοντο αὐτοὺς τὸ ἀργύριον τὸ διδόμενον αὐτοῖς, ὅτι ἐν πίστει αὐτοὶ ποιοῦσι.
\par }{\PP \VS{8}Καὶ εἶπε Χελκίας ὁ ἱερεὺς ὁ μέγας πρὸς Σαπφὰν τὸν γραμματέα, βιβλίον τοῦ νόμου εὗρον ἐν οἴκῳ Κυρίου· καὶ ἔδωκε Χελκίας τὸ βιβλίον πρὸς Σαπφὰν, καὶ ἀνέγνω αὐτό.
\VS{9}Καὶ εἰσῆλθεν ἐν οἴκῳ Κυρίου πρὸς τὸν βασιλέα, καὶ ἀπέστρεψε τῷ βασιλεῖ ῥῆμα, καὶ εἶπεν, ἐχώνευσαν οἱ δοῦλοί σου τὸ ἀργύριον τὸ εὑρεθὲν ἐν οἴκῳ Κυρίου, καὶ ἔδωκαν αὐτὸ ἐπὶ χεῖρα ποιούντων τὰ ἔργα καθεσταμένων ἐν οἴκῳ Κυρίου.
\VS{10}Καὶ εἶπε Σαπφὰν ὁ γραμματεὺς πρὸς τὸν βασιλέα, λέγων, βιβλίον ἔδωκέ μοι Χελκίας ὁ ἱερεύς· καὶ ἀνέγνω αὐτὸ Σαπφὰν ἐνώπιον τοῦ βασιλέως.
\VS{11}Καὶ ἐγένετο ὡς ἤκουσεν ὁ βασιλεὺς τοὺς λόγους βιβλίου τοῦ νόμου, καὶ διέῤῥηξε τὰ ἱμάτια ἑαυτοῦ.
\VS{12}Καὶ ἐνετείλατο ὁ βασιλεὺς τῷ Χελκίᾳ τῷ ἱερεῖ, καὶ τῷ Ἀχικὰμ υἱῷ Σαπφὰν, καὶ τῷ Ἀχοβὼρ υἱῷ Μιχαίου, καὶ τῷ Σαπφὰν τῷ γραμματεῖ, καὶ τῷ Ἀσαΐᾳ δούλῳ τοῦ βασιλέως, λέγων,
\VS{13}δεῦτε, ἐκζητήσατε τὸν Κύριον περὶ ἐμοῦ, καὶ περὶ παντὸς τοῦ λαοῦ, καὶ περὶ παντὸς τοῦ Ἰούδα, καὶ περὶ τῶν λόγων τοῦ βιβλίου τοῦ εὑρεθέντος τούτου, ὅτι μεγάλη ἡ ὀργὴ Κυρίου ἐκκεκαυμένη ἐν ἡμῖν, ὑπὲρ οὗ οὐκ ἤκουσαν οἱ πατέρες ἡμῶν τῶν λόγων τοῦ βιβλίου τούτου τοῦ ποιεῖν κατὰ πάντα τὰ γεγραμμένα καθʼ ἡμῶν.
\par }{\PP \VS{14}Καὶ ἐπορεύθη Χελκίας ὁ ἱερεὺς, καὶ Ἀχικὰμ, καὶ Ἀχοβὼρ, καὶ Σαπφὰν, καὶ Ἀσαΐας πρὸς Ὄλδαν τὴν προφῆτιν μητέρα Σελλὴμ υἱοῦ Θεκουαὺ υἱοῦ Ἀρὰς τοῦ ἱματιοφύλακος· καὶ αὕτη κατῴκει ἐν Ἰερουσαλὴμ ἐν τῇ Μασενᾷ· καὶ ἐλάλησαν πρὸς αὐτήν.
\par }{\PP \VS{15}Καὶ εἶπεν αὐτοῖς, τάδε λέγει Κύριος ὁ θεὸς Ἰσραὴλ, εἴπατε τῷ ἀνδρὶ τῷ ἀποστείλαντι ὑμᾶς πρὸς μέ,
\VS{16}τάδε λέγει Κύριος, ἰδοὺ ἐγὼ ἐπάγω κακὰ ἐπὶ τὸν τόπον τοῦτον, καὶ ἐπὶ τοὺς ἐνοικοῦντας αὐτὸν πάντας τοὺς λόγους τοῦ βιβλίου οὓς ἀνέγνω βασιλεὺς Ἰούδα,
\VS{17}ἀνθʼ ὧν ἐγκατέλιπόν με, καὶ ἐθυμίων θεοῖς ἑτέροις, ὅπως παροργίσωσί με ἐν τοῖς ἔργοις τῶν χειρῶν αὐτῶν, καὶ ἐκκαυθήσεται θυμός μου ἐν τῷ τόπῳ τούτῳ καὶ οὐ σβεσθήσεται.
\VS{18}Καὶ πρὸς βασιλέα Ἰούδα τὸν ἀποστείλαντα ὑμᾶς ἐπιζητῆσαι τὸν Κύριον, τάδε ἐρεῖτε πρὸς αὐτὸν, τάδε λέγει Κύριος ὁ Θεὸς Ἰσραὴλ, οἱ λόγοι οὓς ἤκουσας,
\VS{19}ἀνθʼ ὧν ὅτι ἡπαλύνθη ἡ καρδία σου, καὶ ἐνετράπης ἀπὸ προσώπου, ὡς ἤκουσας ὅσα ἐλάλησα ἐπὶ τὸν τόπον τοῦτον καὶ ἐπὶ τοὺς ἐνοικοῦντας αὐτὸν, τοῦ εἶναι εἰς ἀφανισμὸν καὶ εἰς κατάραν, καὶ διέῤῥηξας τὰ ἱμάτιά σου καὶ ἔκλαυσας ἐνώπιόν μου, καί γε ἐγὼ ἤκουσα, λέγει Κύριος.
\VS{20}Οὐχ οὕτως· ἰδοὺ προστίθημί σε πρὸς τοὺς πατέρας σου, καὶ συναχθήσῃ εἰς τὸν τάφον σου ἐν εἰρήνῃ, καὶ οὐκ ὀφθήσεται ἐν τοῖς ὀφθαλμοῖς σου ἐν πᾶσι τοῖς κακοῖς οἷς ἐγώ εἰμι ἐπάγω ἐπὶ τὸν τόπον τοῦτον.

\par }\Chap{23}{\PP \VerseOne{1}Καὶ ἐπέστρεψαν τῷ βασιλεῖ τὸ ῥῆμα· καὶ ἀπέστειλεν ὁ βασιλεὺς, καὶ συνήγαγε πρὸς ἑαυτὸν πάντας τοὺς πρεσβυτέρους Ἰούδα καὶ Ἰερουσαλήμ.
\VS{2}Καὶ ἀνέβη ὁ βασιλεὺς εἰς οἶκον Κυρίου, καὶ πᾶς ἀνὴρ Ἰούδα καὶ πάντες οἱ κατοικοῦντες ἐν Ἰερουσαλὴμ μετʼ αὐτοῦ, καὶ οἱ ἱερεῖς, καὶ οἱ προφῆται, καὶ πᾶς ὁ λαὸς ἀπὸ μικροῦ καὶ ἕως μεγάλου, καὶ ἀνέγνω ἐν ὠσὶν αὐτῶν πάντας τοὺς λόγους τοῦ βιβλίου τῆς διαθήκης τοῦ εὑρεθέντος ἐν οἴκῳ Κυρίου.
\VS{3}Καὶ ἔστη ὁ βασιλεὺς πρὸς τὸν στύλον, καὶ διέθετο διαθήκην ἐνώπιον Κυρίου, τοῦ πορεύεσθαι ὀπίσω Κυρίου, τοῦ φυλάσσειν τὰς ἐντολὰς αὐτοῦ, καὶ τὰ μαρτύρια αὐτοῦ, καὶ τὰ δικαιώματα αὐτοῦ ἐν πάσῃ καρδίᾳ καὶ ἐν πάσῃ ψυχῇ, τοῦ ἀναστῆσαι τοὺς λόγους τῆς διαθήκης ταύτης, τὰ γεγραμμένα ἐπὶ τὸ βιβλίον τοῦτο· καὶ ἔστη πᾶς ὁ λαὸς ἐν τῇ διαθήκῃ.
\par }{\PP \VS{4}Καὶ ἐνετείλατο ὁ βασιλεὺς τῷ Χελκίᾳ τῷ ἱερεῖ τῷ μεγάλῳ καὶ τοῖς ἱερεῦσι τῆς δευτερώσεως καὶ τοῖς φυλάσσουσι τὸν σταθμὸν, ἐξαγαγεῖν ἐκ τοῦ ναοῦ Κυρίου πάντα τὰ σκεύη τὰ πεποιημένα τῷ Βάαλ καὶ τῷ ἄλσει καὶ πάσῃ τῇ δυνάμει τοῦ οὐρανοῦ· καὶ κατέκαυσεν αὐτὰ ἔξω Ἰερουσαλὴμ ἐν σαλημὼθ Κέδρών, καὶ ἔλαβεν τὸν χοῦν αὐτῶν εἰς Βαιθήλ.
\VS{5}Καὶ κατέκαυσε τοὺς χωμαρὶμ οὓς ἔδωκαν βασιλεῖς Ἰούδα, καὶ ἐθυμίων ἐν τοῖς ὑψηλοῖς καὶ ἐν ταῖς πόλεσιν Ἰούδα καὶ τοῖς περικύκλῳ Ἰερουσαλήμ, καὶ τοὺς θυμιῶντας τῷ Βάαλ καὶ τῷ ἡλίῳ, καὶ τῇ σελήνῃ, καὶ τοῖς μαζουρὼθ, καὶ πάσῃ τῇ δυνάμει τοῦ οὐρανοῦ.
\par }{\PP \VS{6}Καὶ ἐξήνεγκε τὸ ἄλσος ἐξ οἴκου Κυρίου ἔξωθεν Ἰερουσαλὴμ εἰς τὸν χειμάῤῥουν Κέδρών, καὶ κατέκαυσεν αὐτὸν ἐν τῷ χειμάῤῥῳ Κεδρων, καὶ ἐλέπτυνεν εἰς χοῦν· καὶ ἔῤῥιψε τὸν χοῦν αὐτοῦ εἰς τὸν τάφον τῶν υἱῶν τοῦ λαοῦ.
\VS{7}καὶ καθεῖλε τὸν οἶκον τῶν καδησὶμ τῶν ἐν τῷ οἴκῳ Κυρίου, οὗ αἱ γυναῖκες ὕφαινον ἐκεῖ χεττιῒμ τῷ ἄλσει.
\VS{8}Καὶ ἀνήγαγεν πάντας τοὺς ἱερεῖς ἐκ πόλεων Ἰούδα, καὶ ἐμίανε τὰ ὑψηλὰ οὗ ἐθυμίασαν ἐκεῖ οἱ ἱερεῖς ἀπὸ Γαιβὰλ καὶ ἕως Βηρσάβεέ· καὶ καθεῖλε τὸν οἶκον τῶν πυλῶν τὸν παρὰ τὴν θύραν τῆς πύλης Ἰησοῦ ἄρχοντος τῆς πόλεως, τῶν ἐξ ἀριστερῶν ἀνδρὸς ἐν τῇ πύλῃ τῆς πόλεως.
\VS{9}Πλὴν οὐκ ἀνέβησαν οἱ ἱερεῖς τῶν ὑψηλῶν πρὸς τὸ θυσιαστήριον Κυρίου ἐν Ἰερουσαλήμ, ὅτι εἰ μὴ ἔφαγον ἄζυμα ἐν μέσῳ τῶν ἀδελφῶν αὐτῶν.
\VS{10}Καὶ ἐμίανε τὸν Ταφὲθ τὸν ἐν φάραγγι υἱοῦ Ἑννὸμ, τοῦ διαγαγεῖν ἄνδρα τὸν υἱὸν αὐτοῦ καὶ ἄνδρα τὴν θυγατέρα αὐτοῦ τῷ Μόλοχ ἐν πυρί.
\par }{\PP \VS{11}Καὶ κατέκαυσε τοὺς ἵππους οὓς ἔδωκαν βασιλεῖς Ἰούδα τῷ ἡλίῳ ἐν τῇ εἰσόδῳ οἴκου Κυρίου εἰς τὸ γαζοφυλάκιον Ναθὰν βασιλέως τοῦ εὐνούχου ἐν φαρουρείμ· καὶ τὸ ἅρμα τοῦ ἡλίου κατέκαυσεν πυρί,
\VS{12}καὶ τὰ θυσιαστήρια τὰ ἐπὶ τοῦ δώματος τοῦ ὑπερῴου Ἄχαζ, ἃ ἐποίησαν βασιλεὶς Ἰούδα· καὶ τὰ θυσιαστήρια ἃ ἐποίησε Μανασσῆς ἐν ταῖς δυσὶν αὐλαῖς οἴκου Κυρίου καθεῖλεν ὁ βασιλεὺς καὶ κατέσπασεν ἐκεῖθεν, καὶ ἔῤῥιψε τὸν χοῦν αὐτῶν εἰς τὸν χειμάῤῥουν Κέδρων.
\VS{13}Καὶ τὸν οἶκον τὸν ἐπὶ πρόσωπον Ἱερουσαλὴμ τὸν ἐκ δεξιῶν τοῦ ὄρους τοῦ Μοσθὰθ, ὃν ᾠκοδόμησε Σαλωμὼν βασιλεὺς Ἰσραὴλ τῇ Ἀστάρτῃ προσοχθίσματι Σιδωνίων, καὶ τῷ Χαμὼς προσοχθίσματι Μωὰβ, καὶ τῷ Μολὸχ βδελύγματι υἱῶν Ἀμμὼν, ἐμίανεν ὁ βασιλεύς.
\VS{14}Καὶ συνέτριψε τὰς στήλας, καὶ ἐξωλόθρευσε τὰ ἄλση, καὶ ἔπλησε τοὺς τόπους αὐτῶν ὀστέων ἀνθρώπων.
\par }{\PP \VS{15}Καί γε τὸ θυσιαστήριον τὸ ἐν Βαιθὴλ τὸ ὑψηλὸν ὃ ἐποίησεν Ἱεροβοὰμ υἱὸς Ναβὰτ, ὃς ἐξήμαρτε τὸν Ἰσραὴλ, καί γε τὸ θυσιαστήριον ἐκεῖνο τὸ ὑψηλὸν κατέσπασε, καὶ συνέτριψε τοὺς λίθους αὐτοῦ καὶ ἐλέπτυνεν εἰς χοῦν, καὶ κατέκαυσε τὸ ἄλσος.
\VS{16}Καὶ ἐξένευσεν Ἰωσίας καὶ εἶδε τοὺς τάφους τοὺς ἐκεῖ ἐν τῇ πόλει, καὶ ἀπέστειλε, καὶ ἔλαβε τὰ ὀστᾶ ἐκ τῶν τάφων, καὶ κατέκαυσεν ἐπὶ τὸ θυσιαστήριον, καὶ ἐμίανεν αὐτὸ κατὰ τὸ ῥῆμα Κυρίου ὃ ἐλάλησεν ὁ ἄνθρωπος τοῦ Θεοῦ ἐν τῷ ἑστᾶναι Ἱεροβοὰμ ἐν τῇ ἑορτῇ ἐπὶ τὸ θυσιαστήριον· καὶ ἐπιστρέψας ᾖρε τοὺς ὀφθαλμοὺς αὐτοῦ ἐπὶ τὸν τάφον τοῦ ἀνθρώπου τοῦ Θεοῦ τοῦ λαλήσαντος τοὺς λόγους τούτους.
\VS{17}Καὶ εἶπεν, τί τὸ σκόπελον ἐκεῖνο ὃ ἐγὼ ὁρῶ; καὶ εἶπον αὐτῷ οἱ ἄνδρες τῆς πόλεως, ὁ ἄνθρωπος τοῦ Θεοῦ ὁ ἐξεληλυθὼς ἐξ Ἰούδα, καὶ ἐπικαλεσάμενος τοὺς λόγους τούτους οὓς ἐπεκαλέσατο ἐπὶ τὸ θυσιαστήριον Βαιθήλ.
\VS{18}Καὶ εἶπεν, ἄφετε αὐτὸν, ἀνὴρ μὴ κινησάτωσαν τὰ ὀστᾶ αὐτοῦ· καὶ ἐῤῥύσθησαν τὰ ὀστᾶ αὐτοῦ μετὰ τῶν ὀστῶν τοῦ προφήτου τοῦ ἥκοντος ἐκ Σαμαρείας.
\par }{\PP \VS{19}Καί γε πάντας τοὺς οἴκους τῶν ὑψηλῶν τοὺς ἐν ταῖς πόλεσιν Σαμαρείας, οὓς ἐποίησαν βασιλεῖς Ἰσραὴλ παροργίζειν Κύριον, ἀπέστησεν Ἰωσίας, καὶ ἐποίησεν ἐν αὐτοῖς πάντα τὰ ἔργα ἃ ἐποίησεν ἐν Βαιθήλ.
\VS{20}Καὶ ἐθυσίασε πάντας τοὺς ἱερεῖς τῶν ὑψηλῶν τοὺς ὄντας ἐκεῖ ἐπὶ τῶν θυσιαστηρίων, καὶ κατέκαυσε τὰ ὀστᾶ τῶν ἀνθρώπων ἐπʼ αὐτὰ, καὶ ἐπεστράφη εἰς Ἱερουσαλήμ.
\par }{\PP \VS{21}Καὶ ἐνετείλατο ὁ βασιλεὺς παντὶ τῷ λαῷ, λέγων, ποιήσατε πάσχα τῷ Κυρίῳ Θεῷ ἡμῶν, καθὼς γέγραπται ἐπὶ βιβλίου τῆς διαθήκης ταύτης.
\VS{22}Ὅτι οὐκ ἐγενήθη τὸ πάσχα τοῦτο ἀφʼ ἡμερῶν τῶν κριτῶν οἳ ἔκρινον τὸν Ἰσραὴλ, καὶ πάσας τὰς ἡμέρας βασιλέων Ἰσραὴλ καὶ βασιλέων Ἰούδα.
\VS{23}ὅτι ἀλλʼ ἢ τῷ ὀκτωκαιδεκάτῳ ἔτει τοῦ βασιλέως Ἰωσίον ἐγενήθη τὸ πάσχα τῷ Κυρίῳ ἐν Ἰερουσαλήμ.
\par }{\PP \VS{24}Καί γε τοὺς θελητὰς, καὶ τοὺς γνωριστὰς καὶ τὰ θεραφὶν, καὶ τὰ εἴδωλα, καὶ πάντα τὰ προσοχθίσματα τὰ γεγονότα ἐν τῇ γῇ Ἰούδα καὶ ἐν Ἱερουσαλὴμ ἐξῇρεν Ἰωσίας, ἵνα στήσῃ τοὺς λόγους τοῦ νόμου τοὺς γεγραμμένους ἐπὶ τοῦ βιβλίον, οὗ εὗρε Χελκίας ὁ ἱερεὺς ἐν οἴκῳ Κυρίου.
\VS{25}Ὅμοιος αὐτῷ οὐκ ἐγενήθη ἔμπροσθεν αὐτοῦ βασιλεὺς, ὃς ἐπέστρεψε πρὸς Κύριον ἐν ὅλῃ καρδίᾳ αὐτοῦ, καὶ ἐν ὅλῃ ψυχῇ αὐτοῦ, καὶ ἐν ὅλῃ ἰσχύϊ αὐτοῦ κατὰ πάντα τὸν νόμον Μωυσῆ, καὶ μετʼ αὐτὸν οὐκ ἀνέστη ὅμοιος αὐτῷ.
\VS{26}Πλὴν οὐκ ἀπεστράφη Κύριος ἀπὸ θυμοῦ τῆς ὀργῆς αὐτοῦ τῆς μεγάλης οὗ ἐθυμώθη ὀργῇ αὐτοῦ ἐν τῷ Ἰούδᾳ ἐπὶ τοὺς παροργισμοὺς, οὓς παρώργισεν αὐτὸν Μανασσῆς.
\VS{27}Καὶ εἶπε Κύριος, καί γε τὸν Ἰούδα ἀποστήσω ἀπὸ τοῦ προσώπου μου, καθὼς ἀπέστησα τὸν Ἰσραὴλ, καὶ ἀπεώσομαι τὴν πόλιν ταύτην ἣν ἐξελεξάμην, τὴν Ἱερουσαλὴμ, καὶ τὸν οἶκον οὗ εἶπον, ἔσται τὸ ὄνομά μου ἐκεῖ.
\VS{28}Καὶ τὰ λοιπὰ τῶν λόγων Ἰωσίου καὶ πάντα ὅσα ἐποίησεν, οὐχὶ ταῦτα γεγραμμένα ἐπὶ βιβλίῳ λόγων ἡμερῶν τοῖς βασιλεῦσιν Ἰούδα;
\par }{\PP \VS{29}Ἐν δὲ ταῖς ἡμέραις αὐτοῦ ἀνέβη Φαραὼ Νεχαὼ βασιλεὺς Αἰγύπτου ἐπὶ βασιλέα Ἀσσυρίων ἐπὶ ποταμὸν Εὐφράτην· καὶ ἐπορεύθη Ἰωσίας εἰς ἀπαντὴν αὐτοῦ, καὶ ἐθανάτωσεν αὐτὸν Νεχαὼ ἐν Μαγεδδὼ ἐν τῷ ἰδεῖν αὐτόν.
\VS{30}Καὶ ἐπεβίβασαν αὐτὸν οἱ παῖδες αὐτοῦ νεκρὸν ἐκ Μαγεδδὼ, καὶ ἤγαγον αὐτὸν εἰς Ἱερουσαλὴμ, καὶ ἔθαψαν αὐτὸν ἐν τῷ τάφῳ αὐτοῦ· καὶ ἔλαβεν ὁ λαὸς τῆς γῆς τὸν Ἰωάχαζ υἱὸν Ἰωσίου, καὶ ἔχρισαν αὐτὸν, καὶ ἐβασίλευσαν αὐτὸν ἀντὶ τοῦ πατρὸς αὐτοῦ.
\par }{\PP \VS{31}Υἱὸς εἴκοσι καὶ τριῶν ἐτῶν ἦν Ἰωάχαζ ἐν τῷ βασιλεύειν αὐτὸν, καὶ τρίμηνον ἐβασίλευσεν ἐν Ἱερουσαλὴμ, καὶ ὄνομα τῇ μητρὶ αὐτοῦ Ἀμιτὰλ, θυγάτηρ Ἱερεμίου ἐκ Λοβνά.
\VS{32}Καὶ ἐποίησε τὸ πονηρὸν ἐν ὀφθαλμοῖς Κυρίου, κατὰ πάντα ὅσα ἐποίησαν οἱ πατέρες αὐτοῦ.
\VS{33}Καὶ μετέστησεν αὐτὸν Φαραὼ Νεχαὼ ἐν Ῥαβλαὰμ ἐν γῇ Ἐμὰθ τοῦ μὴ βασιλεύειν ἐν Ἱερουσαλὴμ, καὶ ἔδωκε ζημίαν ἐπὶ τὴν γῆν ἑκατὸν τάλαντα ἀργυρίου καὶ ἑκατὸν τάλαντα χρυσίου.
\VS{34}Καὶ ἐβασίλευσε Φαραὼ Νεχαὼ ἐπʼ αὐτοὺς τὸν Ἐλιακὶμ υἱὸν Ἰωσίου βασιλέως Ἰούδα ἀντὶ Ἰωσίου τοῦ πατρὸς αὐτοῦ· καὶ ἐπέστρεψε τὸ ὄνομα αὐτοῦ Ἰωακίμ· καὶ τὸν Ἰωάχαζ ἔλαβε καὶ εἰσήνεγκεν εἰς Αἴγυπτον, καὶ ἀπέθανεν ἐκεῖ.
\VS{35}Καὶ τὸ ἀργύριον καὶ τὸ χρυσίον ἔδωκεν Ἰωακὶμ τῷ Φαραῷ, πλὴν ἐτιμογράφησε τὴν γῆν τοῦ δοῦναι τὸ ἀργύριον ἐπὶ στόματος Φαραώ ἀνὴρ κατὰ τὴν συντίμησιν αὐτοῦ ἔδωκαν τὸ ἀργύριον καὶ τὸ χρυσίον μετὰ τοῦ λαοῦ τῆς γῆς τοῦ δοῦναι τῷ Φαραῷ Νεχαώ.
\par }{\PP \VS{36}Υἱὸς εἴκοσι καὶ πέντε ἐτῶν Ἰωακὶμ ἐν τῷ βασιλεύειν αὐτὸν, καὶ ἕνδεκα ἔτη ἐβασίλευσεν ἐν Ἱερουσαλὴμ, καὶ ὄνομα τῇ μητρὶ αὐτοῦ Ἰελδὰφ θυγάτηρ Φαδαῒλ ἐκ Ῥουμά.
\VS{37}Καὶ ἐποίησε τὸ πονηρὸν ἐν ὀφθαλμοῖς Κυρίου, κατὰ πάντα ὅσα ἐποίησαν οἱ πατέρες αὐτοῦ.

\par }\Chap{24}{\PP \VerseOne{1}Ἐν ταῖς ἡμέραις αὐτοῦ ἀνέβη Ναβουχοδονόσορ βασιλεὺς Βαβυλῶνος, καὶ ἐγενήθη αὐτῷ Ἰωακὶμ δοῦλος τρία ἔτη· καὶ ἐπέστρεψε καὶ ἠθέτησεν ἐν αὐτῷ.
\VS{2}Καὶ ἀπέστειλε Κύριος αὐτῷ τοὺς μονοζώνους τῶν Χαλδαίων, καὶ τοὺς μονοζώνους Συρίας, καὶ τοὺς μονοζώνους Μωὰβ, καὶ τοὺς μονοζώνους υἱῶν Ἀμμὼν, καὶ ἐξαπέστειλεν αὐτοὺς ἐν τῇ γῇ Ἰούδα τοῦ κατισχῦσαι κατὰ τὸν λόγον Κυρίου, ὃν ἐλάλησεν ἐν χειρὶ τῶν δούλων αὐτοῦ τῶν προφητῶν.
\VS{3}Πλὴν ἐπὶ τὸν θυμὸν Κυρίου ἦν ἐν τῷ Ἰούδα, ἀποστῆσαι αὐτὸν ἀπὸ τοῦ προσώπου αὐτὸν ἐν ἁμαρτίαις Μανασσὴ κατὰ πάντα ὅσα ἐποίησε.
\VS{4}Καί γε τὸ αἷμα ἀθῶον, ἐξέχεε, καὶ ἔπλησε τὴν Ἰερουσαλὴμ αἵματος ἀθώου, καὶ οὐκ ἠθέλησε Κύριος ἱλασθῆναι.
\VS{5}Καὶ τὰ λοιπὰ τῶν λόγων Ἰωακὶμ καὶ πάντα ὅσα ἐποίησεν, οὐκ ἰδοὺ ταῦτα γεγραμμένα ἐπὶ βιβλίῳ λόγων τῶν ἡμερῶν τοῖς βασιλεῦσιν Ἰούδα;
\par }{\PP \VS{6}καὶ ἐκοιμήθη Ἰωακεὶμ μετὰ τῶν πατέρων αὐτοῦ, καὶ ἐβασίλευσεν Ἰωακεὶμ υἱὸς αὐτοῦ ἀντʼ αὐτοῦ.
\VS{7}Καὶ οὐ προσέθετο ἔτι βασιλεὺς Αἰγύπτου ἐξελθεῖν ἐκ τῆς γῆς αὐτοῦ, ὅτι ἔλαβε βασιλεὺς Βαβυλῶνος ἀπὸ τοῦ χειμάῤῥου Αἰγύπτου ἕως τοῦ ποταμοῦ Εὐφράτου πάντα ὅσα ἦν τοῦ βασιλέως Αἰγύπτου.
\par }{\PP \VS{8}Υἱὸς ὀκτωκαίδεκα ἐτῶν Ἰωαχὶμ ἐν τῷ βασιλεύειν αὐτὸν, καὶ τρίμηνον ἐβασίλευσεν ἐν Ἱερουσαλὴμ, καὶ ὄνομα τῇ μητρὶ αὐτοῦ Νέσθα, θυγάτηρ Ἐλλαναθὰμ, ἐξ Ἱερουσαλήμ.
\VS{9}Καὶ ἐποίησε τὸ πονηρὸν ἐν ὀφθαλμοῖς Κυρίου, κατὰ πάντα ὅσα ἐποίησεν ὁ πατὴρ αὐτοῦ.
\par }{\PP \VS{10}Ἐν τῷ καιρῷ ἐκείνῳ ἀνέβη Ναβουχοδονόσορ βασιλεὺς Βαβυλῶνος εἰς Ἱερουσαλὴμ, καὶ ἦλθεν ἡ πόλις ἐν περιοχῇ.
\VS{11}Καὶ εἰσῆλθε Ναβουχοδονόσορ βασιλεὺς Βαβυλῶνος εἰς πόλιν, καὶ οἱ παῖδες αὐτοῦ ἐπολιόρκουν ἐπʼ αὐτήν.
\VS{12}Καὶ ἐξῆλθεν Ἰωαχὶμ βασιλεὺς Ἰούδα ἐπὶ βασιλέα Βαβυλῶνος, αὐτὸς καὶ οἱ παῖδες αὐτοῦ, καὶ ἡ μήτηρ αὐτοῦ, καὶ οἱ ἄρχοντες αὐτοῦ, καὶ οἱ εὐνοῦχοι αὐτοῦ· καὶ ἔλαβεν αὐτὸν βασιλεὺς Βαβυλῶνος ἐν τῷ ὀγδόῳ ἔτει τῆς βασιλείας αὐτοῦ.
\VS{13}Καὶ ἐξήνεγκεν ἐκεῖθεν πάντας τοὺς θησαυροὺς οἴκου Κυρίου, καὶ τοὺς θησαυροὺς οἴκου τοῦ βασιλέως, καὶ συνέκοψε πάντα τὰ σκεύη τὰ χρυσᾶ ἃ ἐποίησε Σαλωμὼν ὁ βασιλεὺς Ἰσραὴλ ἐν τῷ ναῷ Κυρίου κατὰ τὸ ῥῆμα Κυρίου.
\VS{14}Καὶ ἀπῴκισε τὴν Ἱερουσαλὴμ καὶ πάντας τοὺς ἄρχοντας καὶ τοὺς δυνατοὺς ἰσχύϊ αἰχμαλωσίας δέκα χιλιάδας αἰχμαλωτίσας, καὶ πᾶν τέκτονα καὶ τὸν συγκλείοντα, καὶ οὐχ ὑπελείφθη πλὴν οἱ πτωχοὶ τῆς γῆς.
\VS{15}Καὶ ἀπῴκισε τὸν Ἰωαχὶμ εἰς Βαβυλῶνα, καὶ τὴν μητέρα τοῦ βασιλέως, καὶ τὰς γυναῖκας τοῦ βασιλέως, καὶ τοὺς εὐνούχους αὐτοῦ· καὶ τοὺς ἰσχυροὺς τῆς γῆς ἀπήγαγεν εἰς ἀποικεσίαν ἐξ Ἱερουσαλὴμ εἰς Βαβυλῶνα·
\VS{16}Καὶ πάντας τοὺς ἄνδρας τῆς δυνάμεως ἑπτακισχιλίους, καὶ τὸν τέκτονα καὶ τὸν συγκλείοντα χιλίους· πάντες δυνατοὶ ποιοῦντες πόλεμον· καὶ ἤγαγεν αὐτοὺς βασιλεὺς Βαβυλῶνος μετοικεσίαν εἰς Βαβυλῶνα.
\VS{17}Καὶ ἐβασίλευσε βασιλεὺς Βαβυλῶνος τὸν Βατθανίαν υἱὸν αὐτοῦ ἀντʼ αὐτοῦ, καὶ ἐπέθηκε τὸ ὄνομα αὐτοῦ, Σεδεκία.
\par }{\PP \VS{18}Υἱὸς εἴκοσι καὶ ἑνὸς ἑνιαυτῶν Σεδεκίας ἐν τῷ βασιλεύειν αὐτὸν, καὶ ἔνδεκα ἔτη ἐβασίλευσεν ἐν Ἱερουσαλὴμ, καὶ ὄνομα τῇ μητρὶ αὐτοῦ Ἀμιτὰλ, θυγὰτηρ Ἱερεμίου.
\VS{19}Καὶ ἐποίησε τὸ πονηρὸν ἐνώπιον Κυρίου, κατὰ πάντα ὅσα ἐποίησεν Ἰωακίμ.
\VS{20}Ὅτι ἐπὶ τὸν θυμὸν Κυρίου ἦν ἐπὶ Ἱερουσαλὴμ καὶ ἐν τῷ Ἰούδα, ἕως ἀπέῤῥιψεν αὐτοὺς ἀπὸ προσώπου αὐτοῦ· καὶ ἠθέτησε Σεδεκίας ἐν τῷ βασιλεῖ Βαβυλῶνος.

\par }\Chap{25}{\PP \VerseOne{1}Καὶ ἐγενήθη ἐν τῷ ἔτει τῷ ἐννάτῳ τῆς βασιλείας αὐτοῦ ἐν τῷ μηνὶ τῷ δεκάτῳ, ἦλθε Ναβουχοδονόσορ ὁ βασιλεὺς Βαβυλῶνος, καὶ πᾶσα ἡ δύναμις αὐτοῦ ἐπὶ Ἱερουσαλήμ· καὶ παρενέβαλεν ἐπʼ αὐτὴν, καὶ ᾠκοδόμησεν ἐπʼ αὐτὴν περίτειχος κύκλῳ.
\VS{2}Καὶ ἦλθεν ἡ πόλις ἐν περιοχῇ ἕως τοῦ ἑνδεκάτου ἔτους τοῦ βασιλέως Σεδεκίου ἐννάτῃ τοῦ μηνός.
\VS{3}Καὶ ἐνίσχυσεν ὁ λιμὸς ἐν τῇ πόλει, καὶ οὐκ ἦσαν ἄρτοι τῷ λαῷ τῆς γῆς.
\VS{4}Καὶ ἐῤῥάγη ἡ πόλις, καὶ πάντες οἱ ἄνδρες τοῦ πολέμου ἐξῆλθον νυκτὸς ὁδὸν πύλης τῆς ἀναμέσον τῶν τειχῶν, αὕτη ἐστὶ τοῦ κήπου τοῦ βασιλέως, καὶ οἱ Χαλδαῖοι ἐπὶ τὴν πόλιν κύκλῳ· καὶ ἐπορεύθη ὁδὸν τὴν Ἄραβα·
\VS{5}Καὶ ἐδίωξεν ἡ δύναμις τῶν Χαλδαίων ὀπίσω τοῦ βασιλέως, καὶ κατέλαβον αὐτὸν ἐν Ἀραβὼθ Ἰεριχὼ, καὶ πᾶσα ἡ δύναμις αὐτοῦ διεσπάρη ἐπάνωθεν αὐτοῦ.
\VS{6}Καὶ συνέλαβον τὸν βασιλέα, καὶ ἤγαγον αὐτὸν πρὸς βασιλέα Βαβυλῶνος εἰς Ῥεβλαθά· καὶ ἐλάλησε μετʼ αὐτοῦ κρίσιν.
\VS{7}Καὶ τοὺς υἱοὺς Σεδεκίου ἔσφαξε κατʼ ὀφθαλμοὺς αὐτοῦ, καὶ τοὺς ὀφθαλμοὺς Σεδεκίου ἐξετύφλωσε, καὶ ἔδησεν αὐτὸν ἐν πέδαις, καὶ ἤγαγεν εἰς Βαβυλῶνα.
\par }{\PP \VS{8}Καὶ ἐν τῷ μηνὶ τῷ πέμπτῳ ἑβδόμῃ τοῦ μηνὸς, αὐτὸς ἐνιαυτὸς ἐννεακαιδέκατος τῷ Ναβουχοδονόσορ βασιλεῖ Βαβυλῶνος, ἦλθε Ναβουζαρδὰν ὁ ἀρχιμάγειρος ἑστὼς ἐνώπιον βασιλέως Βαβυλῶνος εἰς Ἱερουσαλήμ·
\VS{9}Καὶ ἐνέπρησε τὸν οἶκον Κυρίου, καὶ τὸν οἶκον τοῦ βασιλέως, καὶ πάντας τοὺς οἴκους Ἱερουσαλὴμ, καὶ πᾶν οἶκον ἐνέπρησεν ὁ ἀρχιμάγειρος.
\VS{10}Καὶ τὸ τεῖχος Ἱερουσαλὴμ κυκλόθεν κατέσπασεν ἡ δύναμις τῶν χαλδαίων.
\VS{11}Καὶ τὸ περισσὸν τοῦ λαοῦ τὸ καταλειφθὲν ἐν τῇ πόλει, καὶ τοὺς ἐμπεπτωκότας οἳ ἐνέπεσον πρὸς τὸν βασιλέα Βαβυλῶνος, καὶ τὸ λοιπὸν τοῦ στηρίγματος μετῇρε Ναβουζαρδὰν ὁ ἀρχιμάγειρος.
\VS{12}Καὶ ἀπὸ τῶν πτωχῶν τῆς γῆς ὑπέλιπεν ὁ ἀρχιμάγειρος εἰς ἀμπελουργοὺς καὶ εἰς γαβίν.
\par }{\PP \VS{13}Καὶ τοὺς στύλους τοὺς χαλκοῦς τοὺς ἐν οἴκῳ Κυρίου, καὶ τὰς μεχωνὼθ, καὶ τὴν θάλασσαν τὴν χαλκῆν τὴν ἐν οἴκῳ Κυρίου συνέτριψαν οἱ Χαλδαῖοι, καὶ ᾖραν τὸν χαλκὸν αὐτῶν εἰς Βαβυλῶνα.
\VS{14}Καὶ τοὺς λέβητας, καὶ τὰ ἰαμὶν, καὶ τὰς φιάλας, καὶ τὰς θυΐσκας, καὶ πάντα τὰ σκεύη τὰ χαλκᾶ ἐν οἷς λειτουργοῦσιν ἐν αὐτοῖς, ἔλαβε.
\VS{15}Καὶ τὰ πυρεῖα, καὶ τὰς φιάλας τὰς χρυσᾶς καὶ τὰς ἀργυρᾶς ἔλαβεν ὁ ἀρχιμάγειρος,
\VS{16}στύλους δύο, καὶ τὴν θάλασσαν μίαν, καὶ τὰς μεχωνὼθ ἃς ἐποίησε Σαλωμὼν τῷ οἴκῳ Κυρίου· οὐκ ἦν σταθμὸς τοῦ χαλκοῦ πάντων τῶν σκευῶν.
\VS{17}Ὀκτωκαίδεκα πήχεων ὕψος τοῦ στύλου τοῦ ἑνὸς, καὶ τὸ χωθὰρ ἐπʼ αὐτοῦ τὸ χαλκοῦν· καὶ τὸ ὕψος τοῦ χωθὰρ τριῶν πήχεων· σαβαχὰ, καὶ ῥοαὶ ἐπὶ τῷ χωθὰρ κύκλῳ τὰ πάντα χαλκᾶ, καὶ κατὰ ταῦτα τῷ στύλῳ τῷ δευτέρῳ ἐπὶ τῷ σαβαχά.
\par }{\PP \VS{18}Καὶ ἔλαβεν ὁ ἀρχιμάγειρος τὸν Σαραίαν ἱερέα τὸν πρῶτον, καὶ τὸν Σοφονίαν υἱὸν τῆς δευτερώσεως, καὶ τοὺς τρεῖς τοὺς φυλάσσοντας τὸν σταθμόν.
\VS{19}Καὶ ἐκ τῆς πόλεως ἔλαβον εὐνοῦχον ἕνα, ὅς ἦν ἐπιστάτης τῶν ἀνδρῶν τῶν πολεμιστῶν, καὶ πέντε ἄνδρας τῶν ὁρώντων τὸ πρόσωπον τοῦ βασιλέως τοὺς εὑρεθέντας ἐν τῇ πόλει, καὶ τὸν γραμματέα τοῦ ἄρχοντος τῆς δυνάμεως τὸν ἐκτάσσοντα τὸν λαὸν τῆς γῆς, καὶ ἑξήκοντα ἄνδρας τοῦ λαοῦ τῆς γῆς τοὺς εὑρεθέντας ἐν τῇ πόλει.
\VS{20}Καὶ ἔλαβεν αὐτοὺς Ναβουζαρδὰν ὁ ἀρχιμάγειρος, καὶ ἤγαγεν αὐτοὺς πρὸς τὸν βασιλέα Βαβυλῶνος εἰς Ῥεβλαθά.
\VS{21}Καὶ ἔπαισεν αὐτοὺς ὁ βασιλεὺς Βαβυλῶνος, καὶ ἐθανάτωσεν αὐτοὺς εἰς Ῥεβλαθὰ ἐν γῇ Αἰμάθ· καὶ ἀπῳκίσθη Ἰούδας ἐπάνωθεν τῆς γῆς αὐτοῦ.
\par }{\PP \VS{22}Καὶ ὁ λαὸς ὁ καταλειφθεὶς ἐν τῇ γῇ Ἰούδα οὓς κατέλιπε Ναβουχοδονόσορ βασιλεὺς Βαβυλῶνος, καὶ κατέστησεν ἐπʼ αὐτῶν τὸν Γοδολίαν υἱὸν Ἀχικὰμ υἱὸν Σαφάν.
\VS{23}Καὶ ἤκουσαν πάντες οἱ ἄρχοντες τῆς δυνάμεως αυτοὶ καὶ οἱ ἄνδρες αὐτῶν, ὅτι κατέστησε βασιλεὺς Βαβυλῶνος τὸν Γοδολίαν, καὶ ἦλθον πρὸς Γοδολίαν εἰς Μασσηφὰθ, καὶ Ἰσμαὴλ υἱὸς Ναθανίου, καὶ Ἰωνὰ υἱὸς Καρὴθ, καὶ Σαραίας υἱὸς Θαναμὰθ ὁ Νετωφαθίτης, καὶ Ἰεζονίας υἱὸς τοῦ Μαχαθὶ, αὐτοὶ καὶ οἱ ἄνδρες αὐτῶν·
\VS{24}Καὶ ὤμοσεν Γοδολίας αὐτοῖς, καὶ τοῖς ἀνδράσιν αὐτῶν, καὶ εἶπεν αὐτοῖς, μὴ φοβεῖσθε πάροδον τῶν Χαλδαίων, καθίσατε ἐν τῇ γῇ, καὶ δουλεύσατε τῷ βασιλεῖ Βαβυλῶνος, καὶ καλῶς ἔσται ὑμῖν.
\par }{\PP \VS{25}Καὶ ἐγενήθη ἐν τῷ ἑβδόμῳ μηνὶ ἦλθεν Ἰσμαὴλ υἱὸς Ναθανίου υἱοῦ Ἐλισαμὰ ἐκ τοῦ σπέρματος τῶν βασιλέων, καὶ δέκα ἄνδρες μετʼ αὐτοῦ, καὶ ἐπάταξε τὸν Γοδολίαν καὶ ἀπέθανε, καὶ τοὺς Ἰουδαίους καὶ τοὺς Χαλδαίους, οἳ ἦσαν μετʼ αὐτοῦ ἐν Μασσηφά.
\VS{26}Καὶ ἀνέστη πᾶς ὁ λαὸς ἀπὸ μικροῦ ἕως μεγάλου καὶ οἱ ἄρχοντες τῶν δυνάμεων, καὶ εἰσῆλθον εἰς Αἴγυπτον, ὅτι ἐφοβήθησαν ἀπὸ προσώπου τῶν Χαλδαίων.
\par }{\PP \VS{27}Καὶ ἐγενήθη ἐν τῷ τριακοστῷ καὶ ἑβδόμῳ ἔτει τῆς ἀποικίας τοῦ Ἰωαχὶμ βασιλέως Ἰούδα, ἐν τῷ δωδεκάτῳ μηνὶ, ἑβδόμῃ καὶ εἰκάδι τοῦ μηνὸς, ὕψωσεν Εὐιαλμαρωδὲκ βασιλεὺς Βαβυλῶνος ἐν τῷ ἐνιαυτῷ τῆς βασιλείας αὐτοῦ τὴν κεφαλὴν Ἰωαχὶμ τοῦ βασιλέως Ἰούδα, καὶ ἐξήγαγεν αὐτὸν ἐξ οἴκου φυλακῆς αὐτοῦ.
\VS{28}Καὶ ἐλάλησε μετʼ αὐτοῦ ἀγαθὰ, καὶ ἔδωκε τὸν θρόνον αὐτοῦ ἐπάνωθεν τῶν θρόνων τῶν βασιλέων τῶν μετʼ αὐτοῦ ἐν Βαβυλῶνι.
\VS{29}Καὶ ἠλλοίωσε τὰ ἱμάτια τῆς φυλακῆς αὐτοῦ, καὶ ἤσθιεν ἄρτον διαπαντὸς ἐνώπιον αὐτοῦ πάσας τὰς ἡμέρας τῆς ζωῆς αὐτοῦ.
\VS{30}Καὶ ἡ ἑστιατορία αὐτοῦ ἑστιατορία διαπαντὸς ἐδόθη αὐτῷ ἐξ οἴκου τοῦ βασιλέως, λόγον ἡμέρας ἐν τῇ ἡμέρᾳ αὐτοῦ, πάσας τὰς ἡμέρας τῆς ζωῆς αὐτοῦ.
\par }