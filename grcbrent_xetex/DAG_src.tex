\NormalFont\ShortTitle{ΔΑΝΙΗΛ (Ελληνικά)}
{\MT ΔΑΝΙΗΛ

\par }\ChapOne{1}{\PP \VerseOne{1}ἘΝ ἔτει τρίτῳ τῆς βασιλείας Ἰωακεὶμ βασιλέως Ἰούδα, ἦλθε Ναβουχοδονόσορ ὁ βασιλεὺς Βαβυλῶνος εἰς Ἱερουσαλὴμ, καὶ ἐπολιόρκει αὐτήν.
\VS{2}Καὶ ἔδωκε Κύριος ἐν χειρὶ αὐτοῦ τὸν Ἰωακεὶμ βασιλέα Ἰούδα, καὶ ἀπὸ μέρους τῶν σκευῶν οἴκου τοῦ Θεοῦ· καὶ ἤνεγκεν αὐτὰ εἰς γῆν Σενναὰρ οἴκου τοῦ θεοῦ αὐτοῦ, καὶ τὰ σκεύη εἰσήνεγκεν εἰς τὸν οἶκον θησαυροῦ τοῦ θεοῦ αὐτοῦ.
\VS{3}Καὶ εἶπεν ὁ βασιλεὺς τῷ Ἀσφανὲζ τῷ ἀρχιευνούχῳ αὐτοῦ, εἰσαγαγεῖν ἀπὸ τῶν υἱῶν τῆς αἰχμαλωσίας Ἰσραὴλ, καὶ ἀπὸ τοῦ σπέρματος τῆς βασιλείας, καὶ ἀπὸ τῶν φορθομμὶν,
\VS{4}νεανίσκους, οἷς οὐκ ἔστιν ἐν αὐτοῖς μῶμος, καὶ καλοὺς τῇ ὄψει, καὶ συνιέντας ἐν πάσῃ σοφίᾳ, καὶ γινώσκοντας γνῶσιν, καὶ διανοουμένους φρόνησιν, καὶ οἷς ἐστιν ἰσχὺς ἐν αὐτοῖς ἑστάναι ἐν τῷ οἴκῳ ἐνώπιον τοῦ βασιλέως, καὶ διδάξαι αὐτοὺς γράμματα καὶ γλῶσσαν Χαλδαίων.
\par }{\PP \VS{5}Καὶ διέταξεν αὐτοῖς ὁ βασιλεὺς τὸ τῆς ἡμέρας καθʼ ἡμέραν, ἀπὸ τῆς τραπέζης τοῦ βασιλέως, καὶ ἀπὸ τοῦ οἴνου τοῦ ποτοῦ αὐτοῦ, καὶ θρέψαι αὐτοὺς ἔτη τρία, καὶ μετὰ ταῦτα στῆναι ἐνώπιον τοῦ βασιλέως.
\par }{\PP \VS{6}Καὶ ἐγένετο ἐν αὐτοῖς ἐκ τῶν υἱῶν Ἰούδα, Δανιὴλ, καὶ Ἀνανίας, καὶ Ἀζαρίας, καὶ Μισαήλ.
\VS{7}Καὶ ἐπέθηκεν αὐτοῖς ὁ ἀρχιευνοῦχος ὀνόματα· τῷ Δανιὴλ Βαλτάσαρ, καὶ τῷ Ἀνανίᾳ Σεδρὰχ, καὶ τῷ Μισαὴλ Μισὰχ, καὶ τῷ Ἀζαρίᾳ Ἀβδεναγώ.
\VS{8}Καὶ ἔθετο Δανιὴλ εἰς τὴν καρδίαν αὐτοῦ, ὡς οὐ μὴ ἁλισγηθῇ ἐν τῇ τραπέζῃ τοῦ βασιλέως, καὶ ἐν τῷ οἴνῳ ἀπὸ τοῦ ποτοῦ αὐτοῦ· καὶ ἠξίωσε τὸν ἀρχιευνοῦχον, ὡς οὐ μὴ ἁλισγηθῇ.
\VS{9}Καὶ ἔδωκεν ὁ Θεὸς τὸν Δανιὴλ εἰς ἔλεον καὶ οἰκτιρμὸν ἐνώπιον τοῦ ἀρχιευνούχου.
\VS{10}Καὶ εἶπεν ὁ ἀρχιευνοῦχος τῷ Δανιὴλ, φοβοῦμαι ἐγὼ τὸν κύριόν μου τὸν βασιλέα, τὸν ἐκτάξαντα τὴν βρῶσιν ὑμῶν καὶ τὴν πόσιν ὑμῶν, μήποτε ἴδῃ τὰ πρόσωπα ὑμῶν σκυθρωπὰ παρὰ τὰ παιδάρια τὰ συνήλικα ὑμῶν, καὶ καταδικάσητε τὴν κεφαλήν μου τῷ βασιλεῖ.
\VS{11}Καὶ εἶπε Δανιὴλ πρὸς Ἀμελσὰδ, ὃν κατέστησεν ὁ ἀρχιευνοῦχος ἐπὶ Δανιὴλ, Ἀνανίαν, Μισαὴλ, Ἀζαρίαν,
\VS{12}πείρασον δὴ τοὺς παῖδάς σου ἡμέρας δέκα, καὶ δότωσαν ἡμῖν ἀπὸ τῶν σπερμάτων, καὶ φαγώμεθα, καὶ ὕδωρ πιώμεθα,
\VS{13}καὶ ὀφθήτωσαν ἐνώπιόν σου αἱ ἰδέαι ἡμῶν, καὶ αἱ ἰδέαι τῶν παιδαρίων τῶν ἐσθόντων τὴν τράπεζαν τοῦ βασιλέως, καὶ καθὼς ἐὰν ἴδῃς, ποίησον μετὰ τῶν παίδων σου.
\par }{\PP \VS{14}Καὶ εἰσήκουσεν αὐτῶν, καὶ ἐπείρασεν αὐτοὺς ἡμέρας δέκα.
\VS{15}Καὶ μετὰ τὸ τέλος τῶν δέκα ἡμερῶν, ὡράθησαν αἱ ἰδέαι αὐτῶν ἀγαθαὶ καὶ ἰσχυραὶ ταῖς σαρξὶν ὑπὲρ τὰ παιδάρια τὰ ἔσθοντα τὴν τράπεζαν τοῦ βασιλέως.
\VS{16}Καὶ ἐγένετο Ἀμελσὰδ ἀναιρούμενος τὸ δεῖπνον αὐτῶν, καὶ τὸν οἶνον τοῦ πόματος αὐτῶν, καὶ ἐδίδου αὐτοῖς σπέρματα.
\par }{\PP \VS{17}Καὶ τὰ παιδάρια ταῦτα οἱ τέσσαρες αὐτοὶ, ἔδωκεν αὐτοῖς ὁ Θεὸς σύνεσιν καὶ φρόνησιν ἐν πάσῃ γραμματικῇ καὶ σοφίᾳ· καὶ Δανιὴλ συνῆκεν ἐν πάσῃ ὁράσει καὶ ἐνυπνίοις.
\VS{18}Καὶ μετὰ τὸ τέλος τῶν ἡμερῶν, ὧν εἶπεν ὁ βασιλεὺς εἰσαγαγεῖν αὐτοὺς, καὶ εἰσήγαγεν αὐτοὺς ὁ ἀρχιευνοῦχος ἐναντίον Ναβουχοδονόσορ.
\VS{19}Καὶ ἐλάλησε μετʼ αὐτῶν ὁ βασιλεύς· καὶ οὐχ εὑρέθησαν ἐκ πάντων αὐτῶν ὅμοιοι Δανιὴλ, καὶ Ἀνανίᾳ, καὶ Μισαὴλ, καὶ Ἀζαρίᾳ· καὶ ἔστησαν ἐνώπιον τοῦ βασιλέως.
\VS{20}Καὶ ἐν παντὶ ῥήματι σοφίας καὶ ἐπιστήμης ὧν ἐζήτησε παρʼ αὐτῶν ὁ βασιλεύς, εὗρεν αὐτοὺς δεκαπλασίονας παρὰ πάντας τοὺς ἐπαοιδοὺς καὶ τοὺς μάγους τοὺς ὄντας ἐν πάσῃ τῇ βασιλείᾳ αὐτοῦ.
\VS{21}Καὶ ἐγένετο Δανιὴλ ἕως ἔτους ἑνὸς Κύρου τοῦ βασιλέως.

\par }\Chap{2}{\PP \VerseOne{1}Ἐν τῷ ἔτει τῷ δευτέρῳ τῆς βασιλείας, ἐνυπνιάσθη Ναβουχοδονόσορ ἐνύπνιον, καὶ ἐξέστη τὸ πνεῦμα αὐτοῦ, καὶ ὁ ὕπνος αὐτοῦ ἐγένετο ἀπʼ αὐτοῦ.
\VS{2}Καὶ εἶπεν ὁ βασιλεὺς καλέσαι τοὺς ἐπαοιδοὺς, καὶ τοὺς μάγους, καὶ τοὺς φαρμακοὺς, καὶ τοὺς Χαλδαίους, τοῦ ἀναγγεῖλαι τῷ βασιλεῖ τὰ ἐνύπνια αὐτοῦ· καὶ ἦλθαν, καὶ ἔστησαν ἐνώπιον τοῦ βασιλέως.
\par }{\PP \VS{3}Καὶ εἶπεν αὐτοῖς ὁ βασιλεὺς, ἑνυπνιάσθην, καὶ ἐξέστη τὸ πνεῦμά μου, τοῦ γνῶναι τὸ ἐνύπνιον.
\VS{4}Καὶ ἐλάλησαν οἱ Χαλδαῖοι τῷ βασιλεῖ Συριστὶ, βασιλεῦ, εἰς τοὺς αἰῶνας ζῆθι· σὺ εἰπὸν τὸ ἐνύπνιον τοῖς παισί σου, καὶ τὴν σύγκρισιν ἀναγγελοῦμεν.
\VS{5}Ἀπεκρίθη ὁ βασιλεὺς τοῖς Χαλδαίοις, ὁ λόγος ἀπʼ ἐμοῦ ἀπέστη· ἐὰν μὴ γνωρίσητέ μοι τὸ ἐνύπνιον καὶ τὴν σύγκρισιν, εἰς ἀπώλειαν ἔσεσθε, καὶ οἱ οἶκοι ὑμῶν διαρπαγήσονται.
\VS{6}Ἐὰν δὲ τὸ ἐνύπνιον καὶ τὴν σύγκρισιν αὐτοῦ γνωρίσητέ μοι, δόματα καὶ δωρεὰς καὶ τιμὴν πολλὴν λήψεσθε παρʼ ἐμοῦ· πλὴν τὸ ἐνύπνιον καὶ τὴν σύγκρισιν αὐτοῦ ἀπαγγείλατέ μοι.
\VS{7}Ἀπεκρίθησαν δεύτερον, καὶ εἶπαν, ὁ βασιλεὺς εἰπάτω τὸ ἐνύπνιον τοῖς παισὶν αὐτοῦ, καὶ τὴν σύγκρισιν ἀναγγελοῦμεν.
\par }{\PP \VS{8}Καὶ ἀπεκρίθη ὁ βασιλεὺς, καὶ εἶπεν, ἐπʼ ἀληθείας οἶδα ἐγὼ, ὅτι καιρὸν ὑμεῖς ἐξαγοράζετε· καθότι ἴδετε, ὅτι ἀπέστη ἀπʼ ἐμοῦ τὸ ῥῆμα.
\VS{9}Ἐὰν οὖν τὸ ἐνύπνιον μὴ ἀναγγείλητέ μοι, οἶδα ὅτι ῥῆμα ψευδὲς καὶ διεφθαρμένον συνέθεσθε εἰπεῖν ἐνώπιόν μου, ἕως οὗ ὁ καιρὸς παρέλθῃ· τὸ ἐνύπνιόν μου εἴπατέ μοι, καὶ γνώσομαι ὅτι καὶ τὴν σύγκρισιν αὐτοῦ ἀναγγελεῖτέ μοι.
\VS{10}Ἀπεκρίθησαν οἱ Χαλδαῖοι ἐνώπιον τοῦ βασιλέως, καὶ λέγουσιν, οὐκ ἔστιν ἄνθρωπος ἐπὶ τῆς ξηρᾶς, ὅστις τὸ ῥῆμα τοῦ βασιλέως δυνήσεται γνωρίσαι, καθότι πᾶς βασιλεὺς μέγας καὶ ἄρχων ῥῆμα τοιοῦτον οὐκ ἐπερωτᾷ ἐπαοιδὸν μάγον καὶ Χαλδαῖον·
\VS{11}Ὅτι ὁ λόγος ὃν ὁ βασιλεὺς ἐπερωτᾷ, βαρύς, καὶ ἕτερος οὐκ ἔστιν ὃς ἀναγγελεῖ αὐτὸν ἐνώπιον τοῦ βασιλέως, ἀλλʼ οἱ θεοὶ, ὧν οὐκ ἔστιν ἡ κατοικία μετὰ πάσης σαρκός.
\par }{\PP \VS{12}Τότε ὁ βασιλεὺς ἐν θυμῷ καὶ ὀργῇ εἶπεν ἀπολέσαι πάντας τοὺς σοφοὺς Βαβυλῶνος.
\VS{13}Καὶ τὸ δόγμα ἐξῆλθε, καὶ οἱ σοφοὶ ἀπεκτέννοντο· καὶ ἐζήτησαν Δανιὴλ καὶ τοὺς φίλους αὐτοῦ ἀνελεῖν.
\par }{\PP \VS{14}Τότε Δανιὴλ ἀπεκρίθη βουλὴν καὶ γνώμην τῷ Ἀριὼχ τῷ ἀρχιμαγείρῳ τοῦ βασιλέως, ὃς ἐξῆλθεν ἀναιρεῖν τοὺς σοφοὺς Βαβυλῶνος,
\VS{15}ἄρχων τοῦ βασιλέως, περὶ τίνος ἐξῆλθεν ἡ γνώμη ἡ ἀναιδὴς ἐκ προσώπου τοῦ βασιλέως; ἐγνώρισε δὲ ὁ Ἀριὼχ τὸ ῥῆμα τῷ Δανιήλ.
\VS{16}Καὶ Δανιὴλ ἠξίωσε τὸν βασιλέα ὅπως χρόνον δῷ αὐτῷ, καὶ τὴν σύγκρισιν αὐτοῦ ἀναγγελῇ τῷ βασιλεῖ.
\VS{17}Καὶ εἰσῆλθε Δανιὴλ εἰς τὸν οἶκον αὐτοῦ καὶ τῷ Ἀνανίᾳ καὶ τῷ Μισαὴλ καὶ τῷ Ἀζαρίᾳ τοῖς φίλοις αὐτοῦ τὸ ῥῆμα ἐγνώρισε.
\VS{18}Καὶ οἰκτιρμοὺς ἐζήτουν παρὰ τοῦ Θεοῦ τοῦ οὐρανοῦ ὑπὲρ τοῦ μυστηρίου τούτου, ὅπως ἂν μὴ ἀπόλωνται Δανιὴλ καὶ οἱ φίλοι αὐτοῦ μετὰ τῶν ἐπιλοίπων σοφῶν Βαβυλῶνος.
\par }{\PP \VS{19}Τότε τῷ Δανιὴλ ἐν ὁράματι τῆς νυκτὸς τὸ μυστήριον ἀπεκαλύφθη· καὶ εὐλόγησε τὸν Θεὸν τοῦ οὐρανοῦ Δανιὴλ, καὶ εἶπεν,
\par }{\PP \VS{20}Εἴη τὸ ὄνομα τοῦ Θεοῦ εὐλογημένον ἀπὸ τοῦ αἰῶνος καὶ ἕως τοῦ αἰῶνος, ὅτι ἡ σοφία καὶ ἡ σύνεσις αὐτοῦ ἐστι.
\VS{21}Καὶ αὐτὸς ἀλλοιοῖ καιροὺς καὶ χρόνους, καθιστᾷ βασιλεῖς, καὶ μεθιστᾷ, διδοὺς σοφίαν τοῖς σοφοῖς, καὶ φρόνησιν τοῖς εἰδόσι σύνεσιν,
\VS{22}αὐτὸς ἀποκαλύπτει βαθέα καὶ ἀπόκρυφα, γινώσκων τὰ ἐν τῷ σκότει, καὶ τὸ φῶς μετʼ αὐτοῦ ἐστι·
\VS{23}Σοὶ ὁ Θεὸς τῶν πατέρων μου ἐξομολογοῦμαι καὶ αἰνῶ, ὅτι σοφίαν καὶ δύναμιν δέδωκάς μοι, καὶ ἐγνώρισάς μοι ἃ ἠξιώσαμεν παρὰ σοῦ, καὶ τὸ ὅραμα τοῦ βασιλέως ἐγνώρισάς μοι.
\par }{\PP \VS{24}Καὶ ἦλθε Δανιὴλ πρὸς Ἀριὼχ, ὃν κατέστησεν ὁ βασιλεὺς ἀπολέσαι τοὺς σοφοὺς Βαβυλῶνος, καὶ εἶπεν αὐτῷ, τοὺς σοφοὺς Βαβυλῶνος μὴ ἀπολέσῃς, εἰσάγαγε δέ με ἐνώπιον τοῦ βασιλέως, καὶ τὴν σύγκρισιν τῷ βασιλεῖ ἀναγγελῶ.
\VS{25}Τότε Ἀριὼχ ἐν σπουδῇ εἰσήγαγε τὸν Δανιὴλ ἐνώπιον τοῦ βασιλέως, καὶ εἶπεν αὐτῷ, εὕρηκα ἄνδρα ἐκ τῶν υἱῶν τῆς αἰχμαλωσίας τῆς Ἰουδαίας, ὅστις τὸ σύγκριμα τῷ βασιλεῖ ἀναγγελεῖ.
\VS{26}Καὶ ἀπεκρίθη ὁ βασιλεὺς, καὶ εἶπε τῷ Δανιὴλ, οὗ τὸ ὄνομα Βαλτάσαρ, εἰ δύνασαί μοι ἀναγγεῖλαι τὸ ἐνύπνιον ὃ ἴδον, καὶ τὴν σύγκρισιν αὐτοῦ;
\par }{\PP \VS{27}Καὶ ἀπεκρίθη Δανιὴλ ἐνώπιον τοῦ βασιλέως, καὶ εἶπε, τὸ μυστήριον ὃ ὁ βασιλεὺς ἐπερωτᾷ, οὐκ ἔστι σοφῶν, μάγων, ἐπαοιδῶν, γαζαρηνῶν ἀναγγεῖλαι τῷ βασιλεῖ·
\VS{28}Ἀλλʼ ἤ ἐστι Θεὸς ἐν οὐρανῷ ἀποκαλύπτων μυστήρια, καὶ ἐγνώρισε τῷ βασιλεῖ Ναβουχοδονόσορ, ἃ δεῖ γενέσθαι ἐπʼ ἐσχάτων τῶν ἡμερῶν· τὸ ἐνύπνιόν σου καὶ αἱ ὁράσεις τῆς κεφαλῆς σου ἐπὶ τῆς κοίτης σου, τοῦτό ἐστι,
\VS{29}βασιλεῦ· οἱ διαλογισμοί σου ἐπὶ τῆς κοίτης σου ἀνέβησαν τί δεῖ γενέσθαι μετὰ ταῦτα· καὶ ὁ ἀποκαλύπτων μυστήρια ἐγνώρισέ σοι ἃ δεῖ γενέσθαι.
\VS{30}Καὶ ἐμοὶ δὲ οὐκ ἐν σοφίᾳ τῇ οὔσῃ ἐν ἐμοὶ παρὰ πάντας τοὺς ζῶντας τὸ μυστήριον τοῦτο ἀπεκαλύφθη, ἀλλʼ ἕνεκεν τοῦ τὴν σύγκρισιν τῷ βασιλεῖ γνωρίσαι, ἵνα τοὺς διαλογισμοὺς τῆς καρδίας σου γνῷς.
\par }{\PP \VS{31}Σὺ βασιλεῦ ἐθεώρεις, καὶ ἰδοὺ εἰκὼν μία, μεγάλη ἡ εἰκὼν ἐκείνη, καὶ ἡ πρόσοψις αὐτῆς ὑπερφερὴς, ἑστῶσα πρὸ προσώπου σου, καὶ ἡ ὅρασις αὐτῆς φοβερά.
\VS{32}Εἰκὼν, ἧς ἡ κεφαλὴ χρυσίου χρηστοῦ, αἱ χεῖρες καὶ τὸ στῆθος καὶ οἱ βραχίονες αὐτῆς ἀργυροῖ, ἡ κοιλία καὶ οἱ μηροὶ χαλκοῖ,
\VS{33}αἱ κνῆμαι σιδηραῖ, οἱ πόδες μέρος μέν τι σιδηροῦν, καὶ μέρος δέ τι ὀστράκινον.
\VS{34}Ἐθεώρεις ἕως ἀπεσχίσθη λίθος ἐξ ὄρους ἄνευ χειρῶν, καὶ ἐπάταξε τὴν εἰκόνα ἐπὶ τοὺς πόδας τοὺς σιδηροῦς καὶ ὀστρακίνους, καὶ ἐλέπτυνεν αὐτοὺς εἰς τέλος.
\VS{35}Τότε ἐλεπτύνθησαν εἰσάπαξ τὸ ὄστρακον, ὁ σίδηρος, ὁ χαλκὸς, ὁ ἄργυρος, ὁ χρυσός· καὶ ἐγένετο ὡσεὶ κονιορτὸς ἀπὸ ἅλωνος θερινῆς· καὶ ἐξῇρεν αὐτὰ τὸ πλῆθος τοῦ πνεύματος, καὶ τόπος οὐχ εὑρέθη αὐτοῖς· καὶ ὁ λίθος ὁ πατάξας τὴν εἰκόνα, ἐγενήθη ὄρος μέγα, καὶ ἐπλήρωσε πᾶσαν τὴν γῆν.
\VS{36}Τοῦτό ἐστι τὸ ἐνύπνιον, καὶ τὴν σύγκρισιν αὐτοῦ ἐροῦμεν ἐνώπιον τοῦ βασιλέως.
\par }{\PP \VS{37}Σὺ βασιλεῦ βασιλεὺς βασιλέων, ᾧ ὁ Θεὸς τοῦ οὐρανοῦ βασιλείαν ἰσχυρὰν καὶ κραταιὰν καὶ ἔντιμον ἔδωκεν
\VS{38}ἐν παντὶ τόπῳ, ὅπου κατοικοῦσιν οἱ υἱοὶ τῶν ἀνθρώπων· θηρία τε ἀγροῦ, καὶ πετεινὰ οὐρανοῦ, καὶ ἰχθύας τῆς θαλάσσης ἔδωκεν ἐν τῇ χειρί σου, καὶ κατέστησέ σε κύριον πάντων· σὺ εἶ ἡ κεφαλὴ ἡ χρυσῆ.
\VS{39}Καὶ ὀπίσω σου ἀναστήσεται βασιλεία ἑτέρα ἥττων σου, καὶ βασιλεία τρίτη, ἥτις ἐστὶν ὁ χαλκὸς, ἣ κυριεύσει πάσης τῆς γῆς,
\VS{40}καὶ βασιλεία τετάρτη, ἥτις ἔσται ἰσχυρὰ ὡς σίδηρος· ὃν τρόπον ὁ σίδηρος λεπτύνει καὶ δαμάζει πάντα, οὕτως πάντα λεπτυνεῖ καὶ δαμάσει.
\VS{41}Καὶ ὅτι εἶδες τοὺς πόδας, καὶ τοὺς δακτύλους, μέρος μέν τι ὀστράκινον, μέρος δέ τι σιδηροῦν, βασιλεία διῃρημένη ἔσται, καὶ ἀπὸ τῆς ῥίζης τῆς σιδηρᾶς ἔσται ἐν αὐτῇ, ὃν τρόπον εἶδες τὸν σίδηρον ἀναμεμιγμένον τῷ ὀστράκῳ.
\VS{42}Καὶ οἱ δάκτυλοι τῶν ποδῶν μέρος μέν τι σιδηροῦν, μέρος δέ τι ὀστράκινον, μέρος τι τῆς βασιλείας ἔσται ἰσχυρὸν, καὶ ἀπʼ αὐτῆς ἔσται συντριβόμενον.
\VS{43}Ὅτι εἶδες τὸν σίδηρον ἀναμεμιγμένον τῷ ὀστράκῳ, συμμιγεῖς ἔσονται ἐν σπέρματι ἀνθρώπων, καὶ οὐκ ἔσονται προσκολλώμενοι οὗτος μετὰ τούτου, καθὼς ὁ σίδηρος οὐκ ἀναμίγνυται μετὰ τοῦ ὀστράκου.
\par }{\PP \VS{44}Καὶ ἐν ταῖς ἡμέραις τῶν βασιλέων ἐκείνων, ἀναστήσει ὁ Θεὸς τοῦ οὐρανοῦ βασιλείαν, ἥτις εἰς τοὺς αἰῶνας οὐ διαφθαρήσεται, καὶ ἡ βασιλεία αὐτοῦ λαῷ ἑτέρῳ οὐχ ὑπολειφθήσεται, λεπτυνεῖ καὶ λικμήσει πάσας τὰς βασιλείας, καὶ αὕτη ἀναστήσεται εἰς τοὺς αἰῶνας·
\VS{45}Ὃν τρόπον εἶδες, ὅτι ἀπὸ ὄρους ἐτμήθη λίθος ἄνευ χειρῶν, καὶ ἐλέπτυνε τὸ ὄστρακον, τὸν σίδηρον, τὸν χαλκόν, τὸν ἄργυρον, τὸν χρυσόν· ὁ Θεὸς ὁ μέγας ἐγνώρισε τῷ βασιλεῖ ἃ δεῖ γενέσθαι μετὰ ταῦτα· καὶ ἀληθινὸν τὸ ἐνύπνιον, καὶ πιστὴ ἡ σύγκρισις αὐτοῦ.
\par }{\PP \VS{46}Τότε ὁ βασιλεὺς Ναβουχοδονόσορ ἔπεσεν ἐπὶ πρόσωπον, καὶ τῷ Δανιὴλ προσεκύνησε, καὶ μαναὰ καὶ εὐωδίας εἶπε σπεῖσαι αὐτῷ.
\VS{47}Καὶ ἀποκριθεὶς ὁ βασιλεὺς, εἶπε τῷ Δανιήλ, ἐπʼ ἀληθείας ὁ Θεὸς ὑμῶν, αὐτός ἐστι Θεὸς θεῶν, καὶ Κύριος τῶν βασιλέων, ὁ ἀποκαλύπτων μυστήρια, ὅτι ἠδυνάσθης ἀποκαλύψαι τὸ μυστήριον τοῦτο.
\VS{48}Καὶ ἐμεγάλυνεν ὁ βασιλεὺς τὸν Δανιὴλ, καὶ δόματα μεγάλα καὶ πολλὰ ἔδωκεν αὐτῷ, καὶ κατέστησεν αὐτὸν ἐπὶ πάσης χώρας Βαβυλῶνος, καὶ ἄρχοντα σατραπῶν ἐπὶ πάντας τοὺς σοφοὺς Βαβυλῶνος.
\VS{49}Καὶ Δανιὴλ ᾐτήσατο παρὰ τοῦ βασιλέως, καὶ κατέστησεν ἐπὶ τὰ ἔργα τῆς χώρας Βαβυλῶνος τὸν Σεδρὰχ, Μισὰχ, καὶ Ἀβδεναγώ· καὶ Δανιὴλ ἦν ἐν τῇ αὐλῇ τοῦ βασιλέως.

\par }\Chap{3}{\PP \VerseOne{1}Ἔτους ὀκτωκαιδεκάτου Ναβουχοδονόσορ ὁ βασιλεὺς ἐποίησεν εἰκόνα χρυσῆν, ὕψος αὐτῆς πήχεων ἑξήκοντα, εὖρος αὐτῆς πήχεων ἕξ· καὶ ἔστησεν αὐτὴν ἐν πεδίῳ Δεειρᾷ, ἐν χώρᾳ Βαβυλῶνος.
\VS{2}Καὶ ἀπέστειλε συναγαγεῖν τοὺς ὑπάτους, καὶ τοὺς στρατηγοὺς, καὶ τοὺς τοπάρχας, ἡγουμένους, καὶ τυράννους, καὶ τοὺς ἐπʼ ἐξουσιῶν, καὶ πάντας τοὺς ἄρχοντας τῶν χωρῶν, ἐλθεῖν εἰς τὰ ἐγκαίνια τῆς εἰκόνος·
\VS{3}Καὶ συνήχθησαν οἱ τοπάρχαι, ὕπατοι, στρατηγοὶ, ἡγούμενοι, τύραννοι μεγάλοι, οἱ ἐπʼ ἐξουσιῶν, καὶ πάντες οἱ ἄρχοντες τῶν χωρῶν, εἰς τὸν ἐγκαινισμὸν τῆς εἰκόνος, ἧς ἔστησε Ναβουχοδονόσορ ὁ βασιλεύς· καὶ εἱστήκεισαν ἐνώπιον τῆς εἰκόνος.
\par }{\PP \VS{4}Καὶ ὁ κήρυξ ἐβόα ἐν ἰσχύϊ, ὑμῖν λέγεται λαοῖς, φυλαὶ, γλῶσσαι,
\VS{5}ᾗ ἂν ὥρᾳ ἀκούσητε φωνῆς σάλπιγγος, σύριγγός τε, καὶ κιθάρας, σαμβύκης τε, καὶ ψαλτηρίου, καὶ παντὸς γένους μουσικῶν, πίπτοντες προσκυνεῖτε τῇ εἰκόνι τῇ χρυσῇ ᾗ ἔστησε Ναβουχοδονόσορ ὁ βασιλεύς.
\VS{6}Καὶ ὃς ἂν μὴ πεσὼν προσκυνήσῃ, αὐτῇ τῇ ὥρᾳ ἐμβληθήσεται εἰς τὴν κάμινον τοῦ πυρὸς τὴν καιομένην.
\VS{7}Καὶ ἐγένετο ὅταν ἤκουον οἱ λαοὶ τῆς φωνῆς τῆς σάλπιγγος, σύριγγός τε, καὶ κιθάρας, σαμβύκης τε, καὶ ψαλτηρίου, καὶ παντὸς γένους μουσικῶν, πίπτοντες πάντες οἱ λαοὶ, φυλαὶ, γλῶσσαι, προσεκύνουν τῇ εἰκόνι τῇ χρυσῇ ἣν ἔστησε Ναβουχοδονόσορ ὁ βασιλεύς.
\par }{\PP \VS{8}Τότε προσήλθοσαν ἄνδρες Χαλδαῖοι, καὶ διέβαλον τοὺς Ἰουδαίους τῷ βασιλεῖ·
\VS{9}βασιλεῦ, εἰς τοὺς αἰῶνας ζῆθι.
\VS{10}Σὺ βασιλεῦ ἔθηκας δόγμα, πάντα ἄνθρωπον ὃς ἂν ἀκούσῃ τῆς φωνῆς τῆς σάλπιγγος, σύριγγός τε, καὶ κιθάρας, σαμβύκης, καὶ ψαλτηρίου, καὶ παντὸς γένους μουσικῶν,
\VS{11}καὶ μὴ πεσὼν προσκυνήσῃ τῇ εἰκόνι τῇ χρυσῇ, ἐμβληθήσηται εἰς τὴν κάμινον τοῦ πυρὸς τὴν καιομένην.
\VS{12}Εἰσὶν ἄνδρες Ἰουδαῖοι, οὓς κατέστησας ἐπὶ τὰ ἔργα τῆς χώρας Βαβυλῶνος, Σεδρὰχ, Μισὰχ, Ἀβδεναγὼ, οἳ οὐχ ὑπήκουσαν βασιλεῦ τῷ δόγματί σου, τοῖς θεοῖς σου οὐ λατρεύουσι, καὶ τῇ εἰκόνι τῇ χρυσῇ τᾗ ἔστησας οὐ προσκυνοῦσι.
\par }{\PP \VS{13}Τότε Ναβουχοδονόσορ ἐν θυμῷ καὶ ὀργῇ εἶπεν ἀγαγεῖν τὸν Σεδρὰχ, Μισὰχ, καὶ Ἀβδεναγώ· καὶ ἤχθησαν ἐνώπιον τοῦ βασιλέως.
\VS{14}Καὶ ἀπεκρίθη Ναβουχοδονόσορ, καὶ εἶπεν αὐτοῖς, εἶ ἀληθῶς Σεδρὰχ, Μισὰχ, Ἀβδεναγὼ, τοῖς θεοῖς μου οὐ λατρεύετε, καὶ τῇ εἰκόνι τῇ χρυσῇ ᾗ ἔστησα οὐ προσκυνεῖτε;
\VS{15}Νῦν οὖν εἰ ἔχετε ἑτοίμως, ἵνα ὡς ἂν ἀκούσητε τῆς φωνῆς τῆς σάλπιγγος, σύριγγός τε, καὶ κιθάρας, σαμβύκης τε, καὶ ψαλτηρίου, καὶ συμφωνίας, καὶ παντὸς γένους μουσικῶν, πεσόντες προσκυνήσητε τῇ εἰκόνι τῇ χρυσῇ ᾗ ἐποίησα· ἐὰν δὲ μὴ προσκυνήσητε, αὐτῇ τῇ ὥρᾳ ἐμβληθήσεσθε εἰς τὴν κάμινον τοῦ πυρὸς τὴν καιομένην· καὶ τίς ἐστι Θεὸς ὃς ἐξελεῖται ὑμᾶς ἐκ χειρός μου;
\par }{\PP \VS{16}Καὶ ἀπεκρίθησαν Σεδρὰχ, Μισὰχ, Ἀβδεναγὼ, λέγοντες τῷ βασιλεῖ Ναβουχοδονόσορ, οὐ χρείαν ἔχομεν ἡμεῖς περὶ τοῦ ῥήματος τούτου ἀποκριθῆναί σοι.
\VS{17}Ἔστι γὰρ Θεὸς ἡμῶν ἐν οὐρανοῖς, ᾧ ἡμεῖς λατρεύομεν, δυνατὸς ἐξελέσθαι ἡμᾶς ἐκ τῆς καμίνου τοῦ πυρὸς τῆς καιομένης, καὶ ἐκ τῶν χειρῶν σου βασιλεῦ ῥύσεται ἡμᾶς.
\VS{18}Καὶ ἐὰν μὴ, γνωστὸν ἔστω σοι, βασιλεῦ, ὅτι τοῖς θεοῖς σου οὐ λατρεύομεν, καὶ τῇ εἰκόνι ᾗ ἔστησας οὐ προσκυνοῦμεν.
\par }{\PP \VS{19}Τότε Ναβουχοδονόσορ ἐπλήσθη θυμοῦ, καὶ ἡ ὄψις τοῦ προσώπου αὐτοῦ ἠλλοιώθη ἐπὶ Σεδρὰχ, Μισὰχ, καὶ Ἀβδεναγὼ, καὶ εἶπεν ἐκκαῦσαι τὴν κάμινον ἐπταπλασίως, ἕως οὗ εἰς τέλος ἐκκαῇ.
\VS{20}Καὶ ἄνδρας ἰσχυροὺς ἰσχύϊ εἶπε, πεδήσαντας τὸν Σεδρὰχ, Μισὰχ, καὶ Ἀβδεναγὼ, ἐμβαλεῖν εἰς τὴν κάμινον τοῦ πυρὸς τὴν καιομένην.
\VS{21}Τότε οἱ ἄνδρες ἐκεῖνοι ἐπεδήθησαν σὺν τοῖς σαραβάροις αὐτῶν, καὶ τιάραις, καὶ περικνημίσι, καὶ ἐβλήθησαν εἰς τὸ μέσον τῆς καμίνου τοῦ πυρὸς τῆς καιομένης,
\VS{22}ἐπεὶ τὸ ῥῆμα τοῦ βασιλέως ὑπερίσχυε· καὶ ἡ κάμινος ἐξεκαύθη ἐκ περισσοῦ.
\VS{23}Καὶ οἱ τρεῖς οὗτοι Σεδρὰχ, Μισὰχ, καὶ Ἀβδεναγὼ, ἔπεσον εἰς μέσον τῆς καμίνου τῆς καιομένης πεπεδημένοι,
\VS{24}καὶ περιεπάτουν ἐν μέσῳ τῆς φλογὸς, ὑμνοῦντες τὸν Θεὸν, καὶ εὐλογοῦντες τὸν Κύριον.
\par }{\PP \VS{25}Καὶ συστὰς Ἀζαρίας προσηύξατο οὕτως· καὶ ἀνοίξας τὸ στομά αὐτοῦ ἐν μέσῳ τοῦ πυρὸς, εἶπεν,
\par }{\PP \VS{26}Εὐλογητὸς εἶ Κύριε ὁ Θεὸς τῶν πατέρων ἡμῶν, καὶ αἰνετὸς, καὶ δεδοξασμένον τὸ ὄνομά σου εἰς τοὺς αἰῶνας.
\VS{27}Οτι δίκαιος εἶ ἐπὶ πᾶσιν οἶς ἐποίησας, καὶ πάντα τὰ ἔργα σου ἀληθινὰ, καὶ εὐθεῖαι αἱ ὁδοί σου, καὶ πᾶσαι αἱ κρίσεις σου ἀλήθεια.
\par }{\PP \VS{28}Καὶ κρίματα ἀληθείας ἐποίησας κατὰ πάντα ἃ ἐπήγαγες ἡμῖν, καὶ ἐπὶ τὴν πόλιν τὴν ἁγίαν τὴν τῶν πατέρων ἡμῶν Ἱερουσαλήμ· ὅτι ἐν ἀληθείᾳ καὶ κρίσει ἐπήγαγες ταῦτα πάντα διὰ τὰς ἁμαρτίας ἡμῶν.
\VS{29}Ὅτι ἡμάρτομεν καὶ ἠνομήσαμεν ἀποστῆναι ἀπὸ σοῦ, καὶ ἐξημάρτομεν ἐν πᾶσι, καὶ τῶν ἐντολῶν σου οὐκ ἠκούσαμεν,
\VS{30}οὐδὲ συνετηρήσαμεν, οὐδὲ ἐποιήσαμεν καθὼς ἐνετείλω ἡμῖν, ἵνα εὖ ἡμῖν γένηται.
\VS{31}Καὶ πάντα ὅσα ἐπήγαγες ἡμῖν, καὶ πάντα ὅσα ἐποίησας ἡμῖν, ἐν ἀληθινῇ κρίσει ἐποίησας.
\par }{\PP \VS{32}Καὶ παρέδωκας ἡμᾶς εἰς χεῖρας ἐχθρῶν ἀνόμων, καὶ ἐχθίστων ἀποστατῶν, καὶ βασιλεῖ ἀδίκῳ καὶ πονηροτάτῳ παρὰ πᾶσαν τὴν γῆν.
\VS{33}Καὶ νῦν οὐκ ἔστιν ἡμῖν ἀνοῖξαι τὸ στόμα ἡμῶν· αἰσχύνη καὶ ὄνειδος ἐγενήθημεν τοῖς δούλοις σου, καὶ τοῖς σεβομένοις σε.
\par }{\PP \VS{34}Μὴ δὴ παραδῴης ἡμᾶς εἰς τέλος διὰ τὸ ὄνομά σου, καὶ μὴ διασκεδάσῃς τὴν διαθήκην σου,
\VS{35}καὶ μὴ ἀποστήσῃς τὸ ἔλεός σου ἀφʼ ἡμῶν, διὰ Ἁβραὰμ τὸν ἠγαπημένον ὑπὸ σοῦ, καὶ διὰ Ἰσαὰκ τὸν δοῦλόν σου, καὶ Ἰσραὴλ τὸν ἅγιόν σου,
\VS{36}οἷς ἐλάλησας πληθῦναι τὸ σπέρμα αὐτῶν, ὡς τὰ ἄστρα τοῦ οὐρανοῦ, καὶ ὡς τὴν ἄμμον τὴν παρὰ τὸ χεῖλος τῆς θαλάσσης.
\VS{37}Ὅτι, δέσποτα, ἐσμικρύνθημεν παρὰ πάντα τὰ ἔθνη, καί ἐσμὲν ταπεενοὶ ἐν πάσῃ τῇ γῇ σήμερον, διὰ τὰς ἁμαρτίας ἡμῶν.
\VS{38}Καὶ οὐκ ἔστιν ἐν τῷ καιρῷ τούτῳ ἄρχων καὶ προφήτης καὶ ἡγούμενος, οὐδὲ ὁλοκαύτωσις, οὐδὲ θυσία, οὐδὲ προσφορὰ, οὐδὲ θυμίαμα, οὐδὲ τόπος τοῦ καρπῶσαι ἐναντίον σου, καὶ εὑρεῖν ἕλεος.
\par }{\PP \VS{39}Ἀλλʼ ἐν ψυχῇ συντετριμμένῃ, καὶ πνεύματι ταπεινώσεως προσδεχθείημεν, ὡς ἐν ὁλοκαυτώσει κριῶν καὶ ταύρων, καὶ ἐν μυριάσιν ἀρνῶν πιόνων,
\VS{40}οὕτως γενέσθω ἡ θυσία ἡμῶν ἐνώπιόν σου σήμερον, καὶ ἐκτελέσαι ὄπισθέν σου· ὅτι οὐκ ἔσται αἰσχύνη τοῖς πεποιθόσιν ἐπὶ σοί.
\par }{\PP \VS{41}Καὶ νῦν ἐξακολουθοῦμεν ἐν ὅλῃ καρδίᾳ, καὶ φοβούμεθά σε, καὶ ζητοῦμεν τὸ πρόσωπόν σου. Μὴ καταισχύνῃς ἡμᾶς,
\VS{42}ἀλλὰ ποίησον μεθʼ ἡμῶν κατὰ τὴν ἐπιείκειάν σου, καὶ κατὰ τὸ πλῆθος τοῦ ἐλέους σου.
\par }{\PP \VS{43}Καὶ ἐξελοῦ ἡμᾶς κατὰ τὰ θαυμάσιά σου, καὶ δὸς δόξαν τῷ ὀνόματί σου, Κύριε·
\VS{44}καὶ ἐντραπείησαν πάντες οἱ ἐνδεικνύμενοι τοῖς δούλοις σου κακὰ, καὶ καταισχυνθείησαν ἀπὸ πάσης τῆς δυναστείας, καὶ ἡ ἰσχὺς αὐτῶν συντριβείη,
\VS{45}καὶ γνώτωσαν ὅτι σὺ εἶ Κύριος, Θεὸς μόνος, καὶ ἔνδοξος ἐφʼ ὅλην τὴν οἰκουμένην.
\par }{\PP \VS{46}Καὶ οὐ διέλιπον οἱ ἐμβάλλοντες αὐτοὺς ὑπηρέται τοῦ βασιλέως, καίοντες τὴν κάμινον νάφθαν καὶ πίσσαν καὶ στιππύον καὶ κληματίδα.
\VS{47}Καὶ διεχεῖτο ἡ φλὸξ ἐπάνω τῆς καμίνου ἐπὶ πήχεις τεσσαρακονταεννέα.
\VS{48}Καὶ διώδευσε, καὶ ἐνεπύρισεν οὓς εὗρε περὶ τὴν κάμινον τῶν Χαλδαίων.
\par }{\PP \VS{49}Ὁ δὲ ἄγγελος Κυρίου συγκατέβη ἅμα τοῖς περὶ τὸν Αζαρίαν εἰς τὴν κάμινον, καὶ ἐξετίναξε τὴν φλόγα τοῦ πυρὸς ἐκ τῆς καμίνου,
\VS{50}καὶ ἐποίησε τὸ μέσον τῆς καμίνου, ὡς πνεῦμα δρόσου διασυρίζον· καὶ οὐχ ἥψατο αὐτῶν τὸ καθόλου τὸ πῦρ; καὶ οὐκ ἐλύπησεν, οὐδὲ παρηνώχλησεν αὐτοῖς.
\par }{\PP \VS{51}Τότε οἱ τρεῖς ὡς ἐξ ἑνὸς στόματος ὕμνουν, καὶ ἐδόξαζον, καὶ ηὐλόγουν τὸν Θεὸν ἐν τῇ καμίνῳ, λέγοντες,
\par }{\PP \VS{52}Εὐλογητὸς εἶ Κύριε ὁ Θεὸς τῶν πατέρων ἡμῶν, καὶ αἰνετὸς, καὶ ὑπερυψούμενος εἰς τοὺς αἰῶνας. Καὶ εὐλογημένον τὸ ὄνομα τῆς δόξης σου τὸ ἅγιον, καὶ ὑπεραινετὸν καὶ ὑπερυψούμενον εἰς πάντας τοὺς αἰῶνας.
\par }{\PP \VS{53}Εὐλογημένος εἶ ἐν τῷ ναῷ τῆς ἁγίας δόξης σου, καὶ ὑπερυμνητὸς καὶ ὑπερένδοξος εἰς τοὺς αἰῶνας.
\VS{54}Εὐλογημένος εἶ ὁ ἐπιβλέπων ἀβύσσους, καθήμενος ἐπὶ χερουβὶμ, καὶ αἰνετὸς καὶ ὑπερυψούμενος εἰς τοὺς αἰῶνας.
\VS{55}Εὐλογημένος εἶ ἐπὶ θρόνου τῆς βασιλείας σου, καὶ ὑπερυμνητὸς καὶ ὑπερυμνούμενος εἰς τοῦς αἰῶνας.
\VS{56}Εὐλογητὸς εἶ ἐν τῷ στερεώματι τοῦ οὐρανοῦ, καὶ ὑμνητὸς καὶ δεδοξασμένος εἰς τοὺς αἰῶνας.
\par }{\PP \VS{57}Εὐλογεῖτε πάντα τὰ ἔργα Κυρίου τὸν Κύριον, ὑμνεῖτε καὶ ὑπερυψοῦτε αὐτὸν εἰς τοὺς αἰῶνας.
\VS{58}Εὐλογεῖτε οὐρανοὶ τὸν Κύριον, ὑμνεῖτε καὶ ὑπερυψοῦτε αὐτὸν εἰς τοὺς αἰῶνας.
\VS{59}Εὐλογεῖτε ἄγγελοι Κυρίου τὸν Κύριον, ὑμνεῖτε καὶ ὑπερυψοῦτε αὐτὸν εἰς τοὺς αἰῶνας.
\VS{60}Εὐλογεῖτε ὕδατα καὶ πάντα τὰ ὑπεράνω τοῦ οὐρανοῦ τὸν Κύριον, ὑμνεῖτε καὶ ὑπερυψοῦτε αὐτὸν εἰς τοὺς αἰῶνας.
\VS{61}Εὐλογείτω πᾶσα ἡ δύναμις Κυρίου τὸν Κύριον, ὑμνεῖτε καὶ ὑπερυψοῦτε αὐτὸν εἰς τοὺς αἰῶνας.
\par }{\PP \VS{62}Εὐλογεῖτε ἥλιος καὶ σελήνη τὸν Κύριον, ὑμνεῖτε καὶ ὑπερυψοῦτε αὐτὸν εἰς τοὺς αἰῶνας.
\VS{63}Εὐλογεῖτε ἄστρα τοῦ οὐρανοῦ τὸν Κύριον, ὑμνεῖτε καὶ ὑπερυψοῦτε αὐτὸν εἰς τοὺς αἰῶνας.
\VS{64}Εὐλογείτω πᾶς ὄμβρος καὶ δρόσος τὸν Κύριον, ὑμνεῖτε καὶ ὑπερυψοῦτε αὐτὸν εἰς τοὺς αἰῶνας.
\VS{65}Εὐλογεῖτε πάντα τὰ πνεύματα τὸν Κύριον, ὑμνεῖτε καὶ ὑπερυψοῦτε αὐτὸν εἰς τοὺς αἰῶνας.
\VS{66}Εὐλογεῖτε πῦρ καὶ καῦμα τὸν Κύριον, ὑμνεῖτε καὶ ὑπερυψοῦτε αὐτὸν εἰς τοὺς αἰῶνας.
\par }{\PP \VS{71}Εὐλογεῖτε νύκτες καὶ ἡμέραι τὸν Κύριον, ὑμνεῖτε, καὶ ὑπερυψοῦτε αὐτὸν εἰς τοὺς αἰῶνας.
\VS{72}Εὐλογεῖτε φῶς καὶ σκότος τὸν Κύριον, ὑμνεῖτε καὶ ὑπερυψοῦτε αὐτὸν εἰς τοὺς αἰῶνας.
\VS{72a}Εὐλογεῖτε ψύχος καὶ καῦμα τὸν Κύριον, ὑμνεῖτε καὶ ὑπερυψοῦτε αὐτὸν εἰς τοὺς αἰῶνας.
\VS{72b}Εὐλογεῖτε πάχναι καὶ χιόνες τὸν Κύριον, ὑμνεῖτε καὶ ὑπερυψοῦτε αὐτὸν εἰς τοὺς αἰῶνας.
\VS{73}Εὐλογεῖτε ἀστραπαὶ καὶ νεφέλαι τὸν Κύριον, ὑμνεῖτε καὶ ὑπερυψοῦτε αὐτὸν εἰς τοὺς αἰῶνας.
\par }{\PP \VS{74}Εὐλογείτω ἡ γῆ τὸν Κύριον, ὑμνείτω καὶ ὑπερυψούτω αὐτὸν εἰς τοὺς αἰῶνας.
\VS{75}Εὐλογεῖτε ὄρη καὶ βουνοὶ τὸν Κύριον, ὑμνεῖτε καὶ ὑπερυψοῦτε αὐτὸν εἰς τοὺς αἰῶνας.
\VS{76}Εὐλογεῖτε πάντα τὰ φυόμενα ἐν τῇ γῇ τὸν Κύριον, ὑμνεῖτε καὶ ὑπερυψοῦτε αὐτὸν εἰς τοὺς αἰῶνας.
\par }{\PP \VS{77}Εὐλογεῖτε θάλασσα καὶ ποταμοὶ τὸν Κύριον, ὑμνεῖτε καὶ ὑπερυψοῦτε αὐτὸν εἰς τοὺς αἰῶνας.
\VS{78}Εὐλογεῖτε αἱ πηγαὶ τὸν Κύριον, ὑμνεῖτε καὶ ὑμερυψοῦτε αὐτὸν εἰς τοὺς αἰῶνας.
\VS{79}Εὐλογεῖτε κήτη καὶ πάντα τὰ κινούμενα ἐν τοῖς ὕδασι τὸν Κύριον, ὑμνεῖτε καὶ ὑπερυψοῦτε αὐτὸν εἰς τοὺς αἰῶνας.
\VS{80}Εὐλογεῖτε πάντα τὰ πετεινὰ τοῦ οὐρανοῦ τὸν Κύριον, ὑμνεῖτε καὶ ὑπερυψοῦτε αὐτὸν εἰς τοὺς αἰῶνας.
\VS{81}Εὐλογεῖτε πάντα τὰ θηρία καὶ τὰ κτήνη τὸν Κύριον, ὑμνεῖτε καὶ ὑπερυψοῦτε αὐτὸν εἰς τοὺς αἰῶνας.
\par }{\PP \VS{82}Εὐλογεῖτε υἱοὶ τῶν ἀνθρώπων τὸν Κύριον, ὑμνεῖτε καὶ ὑπερυψοῦτε αὐτὸν εἰς τοὺς αἰῶνας.
\VS{83}Εὐλογεῖτε Ἰσραὴλ τὸν Κύριον, ὑμνεῖτε καὶ ὑπερυψοῦτε αὐτὸν εἰς τοὺς αἰῶνας.
\par }{\PP \VS{84}Εὐλογεῖτε ἱερεῖς τὸν Κύριον, ὑμνεῖτε καὶ ὑπερυψοῦτε αὐτὸν εἰς τοὺς αἰῶνας.
\VS{85}Εὐλογεῖτε δοῦλοι τὸν Κύριον, ὑμνεῖτε καὶ ὑπερυψοῦτε αὐτὸν εἰς τοὺς αἰῶνας.
\VS{86}Εὐλογεῖτε πνεύματα καὶ ψυχαὶ δικαίων τὸν Κύριον, ὑμνεῖτε καὶ ὑπερυψοῦτε αὐτὸν εἰς τοὺς αἰῶνας.
\VS{87}Εὐλογεῖτε ὅσιοι καὶ ταπεινοὶ τῇ καρδίᾳ τὸν Κύριον, ὑμνεῖτε καὶ ὑπερυψοῦτε αὐτὸν εἰς τοὺς αἰῶνας.
\par }{\PP \VS{88}Εὐλογεῖτε Ἀνανία, Ἀζαρία, Μισαὴλ τὸν Κύριον, ὑμνεῖτε καὶ ὑπερυψοῦτε αὐτὸν εἰς τοὺς αἰῶνας· ὅτι ἐξείλετο ἡμᾶς ἐξ ᾅδου, καὶ ἐκ χειρὸς θανάτου ἔσωσεν ἡμᾶς· καὶ ἐῤῥύσατο ἡμᾶς ἐκ μέσου καμίνου καιομένης φλογὸς, καὶ ἐκ μέσου πυρὸς ἐῤῥύσατο ἡμᾶς.
\VS{89}Ἐξομολογεῖσθε τῷ Κυρίῳ ὅτι χρηστὸς, ὅτι εἰς τὸν αἰῶνα τὸ ἔλεος αὐτοῦ.
\par }{\PP \VS{90}Εὐλογεῖτε πάντες οἱ σεβόμενοι τὸν Κύριον τὸν Θεὸν τῶν θεῶν, ὑμνεῖτε καὶ ἐξομολογεῖσθε, ὅτι εἰς τὸν αἰῶνα τὸ ἔλεος αὐτοῦ.
\par }{\PP \VS{91}Καὶ Ναβουχοδονόσορ ἤκουσεν ὑμνούντων αὐτῶν, καὶ ἐθαύμασε, καὶ ἐξανέστη ἐν σπουδῇ, καὶ εἶπε τοῖς μεγιστᾶσιν αὐτοῦ, οὐχὶ ἄνδρας τρεῖς ἐβάλομεν εἰς τὸ μέσον τοῦ πυρὸς πεπεδημένους; καὶ εἶπον τῷ βασιλεῖ, ἀληθῶς βασιλεῦ.
\VS{92}Καὶ εἶπεν ὁ βασιλεὺς, ὁ δὲ ἐγὼ ὁρῶ ἄνδρας τέσσαρας λελυμένους, καὶ περιπατοῦντας ἐν μέσῳ τοῦ πυρὸς, καὶ διαφθορὰ οὐκ ἔστιν ἐν αὐτοῖς, καὶ ἡ ὅρασις τοῦ τετάρτου ὁμοία υἱῷ Θεοῦ.
\VS{93}Τότε προσῆλθε Ναβουχοδονόσορ πρὸς τὴν θύραν τῆς καμίνου τοῦ πυρὸς τῆς καιομένης, καὶ εἶπε, Σεδρὰχ, Μισὰχ, Ἀβδεναγὼ, οἱ δοῦλοι τοῦ Θεοῦ τοῦ ὑψίστου, ἐξέλθετε καὶ δεῦτε· καὶ ἐξῆλθον Σεδρὰχ, Μισὰχ, Ἀβδεναγὼ, ἐκ μέσου τοῦ πυρός.
\VS{94}Καὶ συνάγονται οἱ σατράπαι, καὶ οἱ στρατηγοὶ, καὶ οἱ τοπάρχαι, καὶ οἱ δυνάσται τοῦ βασιλέως, καὶ ἐθεώρουν τοὺς ἄνδρας, ὅτι οὐκ ἐκυρίευσε τὸ πῦρ τοῦ σώματος αὐτῶν, καὶ ἡ θρὶξ τῆς κεφαλῆς αὐτῶν οὐκ ἐφλογίσθη, καὶ τὰ σαράβαρα αὐτῶν οὐκ ἠλλοιώθη, καὶ ὀσμὴ πυρὸς οὐκ ἦν ἐν αὐτοῖς.
\par }{\PP \VS{95}Καὶ ἀπεκρίθη Ναβουχοδονόσορ ὁ βασιλεὺς, καὶ εἶπεν, εὐλογητὸς ὁ Θεὸς τοῦ Σεδρὰχ, Μισὰχ, Ἀβδεσαγὼ, ὃς ἀπέστειλε τὸν ἄγγελον αὐτοῦ, καὶ ἐξείλατο τοὺς παῖδας αὐτοῦ, ὅτι ἐπεποίθεισαν ἐπʼ αὐτῷ· καὶ τὸ ῥῆμα τοῦ βασιλέως ἠλλοίωσαν, καὶ παρέδωκαν τὰ σὼματα αὐτῶν εἰς πῦρ, ὅπως μὴ λατρεύσωσι μηδὲ προσκυνήσωσι παντὶ θεῷ, ἀλλʼ ἢ τῷ Θεῷ αὐτῶν.
\VS{96}Καὶ ἐγὼ ἐκτίθεμαι τὸ δόγμα· πᾶς λαὸς, φυλὴ, γλῶσσα, ἣ ἐὰν εἴπῃ βλασφημίαν κατὰ τοῦ Θεοῦ Σεδρὰχ, Μισὰχ, Ἀβδεναγὼ, εἰς ἀπώλειαν ἔσονται, καὶ οἱ οἶκοι αὐτῶν εἰς διαρπαγὴν, καθότι οὐκ ἔστι Θεὸς ἕτερος ὅστις δυνήσεται ῥύσασθαι οὕτως.
\VS{97}Τότε ὁ βασιλεὺς κατεύθυνε τὸν Σεδρὰχ, Μισὰχ, Ἀβδεναγὼ, ἐν τῇ χώρᾳ Βαβυλῶνος, καὶ ηὔξησεν αὐτοὺς, καὶ ἠξίωσεν αὐτοὺς ἡγεῖσθαι πάντων τῶν Ἰουδαίων, τῶν ἐν τῇ βασιλείᾳ αὐτοῦ.

\par }\Chap{4}{\PP \VerseOne{1}Ναβουχοδονόσορ ὁ βασιλεὺς πᾶσι τοῖς λαοῖς, φυλαῖς, καὶ γλώσσαις, τοῖς οἰκοῦσιν ἐν πάσῃ τῇ γῇ, ἐρήνη ὑμῖν πληθυνθείη.
\VS{2}Τὰ σημεῖα καὶ τὰ τέρατα ἃ ἐποίησε μετʼ ἐμοῦ ὁ Θεὸς ὁ ὕψιστος,
\VS{3}ἤρεσεν ἐναντίον ἐμοῦ ἀναγγεῖλαι ὑμῖν, ὡς μεγάλα καὶ ἰσχυρὰ, ἡ βασιλεία αὐτοῦ βασιλεία αἰώνιος, καὶ ἡ ἐξουσία αὐτοῦ εἰς γενεὰν καὶ γενεάν.
\par }{\PP \VS{4}Ἐγὼ Ναβουχοδονόσορ εὐθηνῶν ἤμην ἐν τῷ οἴκῳ μου, καὶ εὐθαλῶν.
\VS{5}Ἐνύπνιον ἴδον, καὶ ἐφοβέρισέ με, καὶ ἐταράχθην ἐπὶ τῆς κοίτης μου, καὶ αἱ ὁράσεις τῆς κεφαλῆς μου ἐτάραξάν με.
\VS{6}Καὶ διʼ ἐμοῦ ἐτέθη δόγμα τοῦ εἰσαγαγεῖν ἐνώπιόν μου πάντας τοὺς σοφοὺς Βαβυλῶνος, ὅπως τὴν σύγκρισιν τοῦ ἐνυπνίου γνωρίσωσί μοι.
\VS{7}Καὶ εἰσπορεύοντο οἱ ἐπαοιδοὶ, μάγοι, γαζαρηνοὶ, Χαλδαῖοι· καὶ τὸ ἐνύπνιον ἐγὼ εἶπα ἐνώπιον αὐτῶν· καὶ τὴν σύγκρισιν αὐτοῦ οὐκ ἐγνώρισάν μοι,
\VS{8}ἕως ἦλθε Δανιὴλ, οὗ τὸ ὄνομα Βαλτάσαρ κατὰ τὸ ὄνομα τοῦ θεοῦ μου, ὃς πνεῦμα Θεοῦ ἅγιον ἐν ἑαυτῷ ἔχει. ᾧ εἶπα,
\par }{\PP \VS{9}Βαλτάσαρ ὁ ἄρχων τῶν ἐπαοιδῶν, ὃν ἐγὼ ἔγνων ὅτι πνεῦμα Θεοῦ ἅγιον ἐν σοὶ, καὶ πᾶν μυστήριον οὐκ ἀδυνατεῖ σε, ἄκουσον τὴν ὅρασιν τοῦ ἐνυπνίου μου, οὗ ἴδον, καὶ τὴν σύγκρισιν αὐτοῦ εἰπόν μοι.
\VS{10}Ἐπὶ τῆς κοίτης μου ἐθεώρουν, καὶ ἰδοὺ δένδρον ἐν μέσῳ τῆς γῆς, καὶ τὸ ὕψος αὐτοῦ πολύ.
\VS{11}Ἐμεγαλύνθη τὸ δένδρον καὶ ἴσχυσε, καὶ τὸ ὕψος αὐτοῦ ἔφθασεν ἕως τοῦ οὐρανοῦ, καὶ τὸ κῦτος αὐτοῦ εἰς τὸ πέρας ἁπάσης τῆς γῆς,
\VS{12}τὰ φύλλα αὐτοῦ ὡραῖα, καὶ ὁ καρπὸς αὐτοῦ πολὺς, καὶ τροφὴ πάντων ἐν αὐτῷ, καὶ ὑποκάτω αὐτοῦ κατεσκήνουν τὰ θηρία τὰ ἄγρια, καὶ ἐν τοῖς κλάδοις αὐτοῦ κατῴκουν τὰ ὄρνεα τοῦ οὐρανοῦ, καὶ ἐξ αὐτοῦ ἐτρέφετο πᾶσα σάρξ.
\par }{\PP \VS{13}Ἐθεώρουν ἐν ὁράματι τῆς νυκτὸς ἐπὶ τῆς κοίτης μου, καὶ ἰδοὺ εἲρ, καὶ ἅγιος ἀπʼ οὐρανοῦ κατέβη,
\VS{14}καὶ ἐφώνησεν ἐν ἰσχύϊ, καὶ οὕτως εἶπεν, ἐκκόψατε τὸ δένδρον, καὶ ἐκτίλατε τοὺς κλάδους αὐτοῦ, καὶ ἐκτινάξατε τὰ φύλλα αὐτοῦ, καὶ διασκορπίσατε τὸν καρπὸν αὐτοῦ· σαλευθήτωσαν τὰ θηρία ὑποκάτωθεν αὐτοῦ, καὶ τὰ ὄρνεα ἀπὸ τῶν κλάδων αὐτοῦ.
\VS{15}Πλὴν τὴν φυὴν τῶν ῥιζῶν αὐτοῦ ἐν τῇ γῇ ἐάσατε, καὶ ἐν δεσμῷ σιδηρῷ καὶ χαλκῷ, καὶ ἐν τῇ χλόῃ τῇ ἔξω, καὶ ἐν τῇ δρόσῳ τοῦ οὐρανοῦ κοιτασθήσεται, καὶ μετὰ τῶν θηρίων ἡ μερὶς αὐτοῦ ἐν τῷ χόρτῳ τῆς γῆς.
\VS{16}Ἡ καρδία αὐτοῦ ἀπὸ τῶν ἀνθρώπων ἀλλοιωθήσεται, καὶ καρδία θηρίου δοθήσεται αὐτῷ, καὶ ἑπτὰ καιροὶ ἀλλαγήσονται ἐπʼ αὐτόν.
\VS{17}Διὰ συνκρίματος εἲρ ὁ λόγος, καὶ ῥῆμα ἁγίων τὸ ἐπερώτημα, ἵνα γνῶσιν οἱ ζῶντες, ὅτι Κύριός ἐστιν ὁ ὕψιστος τῆς βασιλείας τῶν ἀνθρώπων, καὶ ᾧ ἐὰν δόξῃ δώσει αὐτὴν, καὶ ἐξουδένωμα ἀνθρώπων ἀναστήσει ἐπʼ αὐτήν.
\VS{18}Τοῦτο τὸ ἐνύπνιον ὃ ἴδον ἐγὼ Ναβουχοδονόσορ ὁ βασιλεὺς, καὶ σύ Βαλτάσαρ τὸ σύγκριμα εἰπὸν, ὅτι πάντες οἱ σοφοὶ τῆς βασιλείας μου οὐ δύνανται τὸ σύγκριμα αὐτοῦ δηλῶσαί μοι· σὺ δὲ Δανιὴλ δύνασαι, ὅτι πνεῦμα Θεοῦ ἅγιον ἐν σοί.
\par }{\PP \VS{19}Τότε Δανιὴλ, οὗ τὸ ὄνομα Βαλτάσαρ, ἀπηνεώθη ὡσεὶ ὥραν μίαν, καὶ οἱ διαλογισμοὶ αὐτοῦ συνετάρασσον αὐτόν· καὶ ἀπεκρίθη Βαλτάσαρ, καὶ εἶπε, κύριε, τὸ ἐνύπνιον ἔστω τοῖς μισοῦσί σε, καὶ ἡ σύγκρισις αὐτοῦ τοῖς ἐχθροῖς σου.
\VS{20}Τὸ δένδρον ὃ εἶδες τὸ μεγαλυνθὲν καὶ τὸ ἰσχυκὸς, οὗ τὸ ὓψος ἔφθανεν εἰς τὸν οὐρανὸν, καὶ τὸ κῦτος αὐτοῦ εἰς πᾶσαν τὴν γῆν,
\VS{21}καὶ τὰ φύλλα αὐτοῦ εὐθαλῆ, καὶ ὁ καρπὸς αὐτοῦ πολύς, καὶ τροφὴ πᾶσιν ἐν αὐτῷ, ὑποκάτω αὐτοῦ κατῴκουν τὰ θηρία τὰ ἄγρια, καὶ ἐν τοῖς κλάδοις αὐτοῦ κατεσκήνουν τὰ ὄρνεα τοῦ οὐρανοῦ,
\VS{22}σὺ εἶ, βασιλεῦ, ὅτι ἐμεγαλύνθης καὶ ἴσχυσας, καὶ ἡ μεγαλωσύνη σου ἐμεγαλύνθη, καὶ ἔφθασεν εἰς τὸν οὐρανὸν, καὶ ἡ κυρεία σου εἰς τὰ πέρατα τῆς γῆς.
\VS{23}Καὶ ὅτι εἶδεν ὁ βασιλεὺς εἲρ, καὶ ἅγιον καταβαίνοντα ἀπὸ τοῦ οὐρανοῦ, καὶ εἶπεν, ἐκτίλατε τὸ δένδρον, καὶ διαφθείρατε αὐτὸ, πλὴν τὴν φυὴν τῶν ῥιζῶν αὐτοῦ ἐν τῇ γῇ ἐάσατε, καὶ ἐν δεσμῷ σιδηρῷ καὶ ἐν χαλκῷ, καὶ ἐν τῇ χλόῃ τῇ ἔξω, καὶ ἐν τῇ δρόσῳ τοῦ οὐρανοῦ αὐλισθήσεται, καὶ μετὰ θηρίων ἀγρίων ἡ μερὶς αὐτοῦ, ἕως οὗ ἑπτὰ καιροὶ ἀλλοιωθῶσιν ἐπʼ αὐτόν·
\VS{24}Τοῦτο ἡ σύγκρισις αὐτοῦ βασιλεῦ, καὶ σύγκριμα ὑψίστου ἐστὶν, ὃ ἔφθασεν ἐπὶ τὸν κύριόν μου τὸν βασιλέα,
\VS{25}καὶ σὲ ἐκδιώξουσιν ἀπὸ τῶν ἀνθρώπων, καὶ μετὰ θηρίων ἀγρίων ἔσται ἡ κατοικία σου, καὶ χόρτον ὡς βοῦν ψωμιοῦσί σε, καὶ ἀπὸ τῆς δρόσου τοῦ οὐρανοῦ αὐλισθήσῃ, καὶ ἑπτὰ καιροὶ ἀλλαγήσονται ἐπὶ σὲ, ἕως οὗ γνῷς ὅτι κυριεύει ὁ ὕψιστος τῆς βασιλείας τῶν ἀνθρώπων, καὶ ᾧ ἂν δόξῃ δώσει αὐτήν.
\VS{26}Καὶ ὅτι εἶπαν, ἐάσατε τὴν φυὴν τῶν ῥιζῶν τοῦ δένδρου· ἡ βασιλεία σου σοὶ μένει, ἀφʼ ἧς ἂν γνῷς τὴν ἐξουσίαν τὴν οὐράνιον.
\VS{27}Διατοῦτο, βασιλεῦ, ἡ βουλή μου ἀρεσάτω σοι, καὶ τὰς ἁμαρτίας σου ἐν ἐλεημοσύναις λύτρωσαι, καὶ τὰς ἀδικίας, ἐν οἰκτιρμοῖς πενήτων· ἴσως ἔσται μακρόθυμος τοῖς παραπτώμασί σου ὁ Θεός.
\par }{\PP \VS{28}Ταῦτα πάντα ἔφθασεν ἐπὶ Ναβουχοδονόσορ τὸν βασιλέα.
\VS{29}Μετὰ δωδεκάμηνον, ἐπὶ τῷ ναῷ τῆς βασιλείας αὐτοῦ ἐν Βαβυλῶνι περιπατῶν,
\VS{30}ἀπεκρίθη ὁ βασιλεὺς, καὶ εἶπεν, οὐχ αὕτη ἐστὶ Βαβυλὼν ἡ μεγάλη, ἣν ἐγὼ ᾠκοδόμησα εἰς οἶκον βασιλείας, ἐν τῷ κράτει τῆς ἰσχύος μου, εἰς τιμὴν τῆς δόξης μου;
\par }{\PP \VS{31}Ἔτι τοῦ λόγου ἐν τῷ στόματι τοῦ βασιλέως ὄντος, φωνὴ ἀπʼ οὐρανοῦ ἐγένετο, σοὶ λέγουσι Ναβουχοδονόσορ βασιλεῦ, ἡ βασιλεία παρῆλθεν ἀπὸ σοῦ,
\VS{32}καὶ ἀπὸ τῶν ἀνθρώπων σε ἐκδιώκουσι, καὶ μετὰ θηρίων ἀγρίων ἡ κατοικία σου, καὶ χόρτον ὡς βοῦν ψωμιοῦσί σε, καὶ ἑπτὰ καιροὶ ἀλλαγήσονται ἐπὶ σὲ, ἕως γνῷς ὅτι κυριεύει ὁ ὕψιστος τῆς βασιλείας τῶν ἀνθρώπων, καὶ ᾧ ἂν δόξῃ δώσει αὐτήν.
\par }{\PP \VS{33}Αὐτῇ τῇ ὥρᾳ ὁ λόγος συνετελέσθη ἐπὶ Ναβουχοδονόσορ, καὶ ἀπὸ τῶν ἀνθρώπων ἐξεδιώχθη, καὶ χόρτον ὡς βοῦς ἤσθιε, καὶ ἀπὸ τῆς δρόσου τοῦ οὐρανοῦ τὸ σῶμα αὐτοῦ ἐβάφη, ἕως αἱ τρίχες αὐτοῦ ὡς λεόντων ἐμεγαλύνθησαν, καὶ οἱ ὄνυχες αὐτοῦ ὡς ὁρνέων.
\par }{\PP \VS{34}Καὶ μετὰ τὸ τέλος τῶν ἡμερῶν ἐγὼ Ναβουχοδονόσορ τοὺς ὀφθαλμούς μου εἰς τὸν οὐρανὸν ἀνέλαβον, καὶ αἱ φρένες μου ἐπʼ ἐμὲ ἐπεστράφησαν, καὶ τῷ ὑψίστῳ ηὐλόγησα, καὶ τῷ ζῶντι εἰς τὸν αἰῶνα ᾔνεσα, καὶ ἐδόξασα, ὅτι ἡ ἐξουσία αὐτοῦ ἐξουσία αἰώνιος, καὶ ἡ βασιλεία αὐτοῦ εἰς γενεὰν καὶ γενεὰν,
\VS{35}καὶ πάντες οἱ κατοικοῦντες τὴν γῆν ὡς οὐδὲν ἐλογίσθησαν· καὶ κατὰ τὸ θέλημα αὐτοῦ ποιεῖ ἐν τῇ δυνάμει τοῦ οὐρανοῦ, καὶ ἐν τῇ κατοικίᾳ τῆς γῆς· καὶ οὐκ ἔστιν ὃς ἀντιποιήσεται τῇ χειρὶ αὐτοῦ, καὶ ἐρεῖ αὐτῷ, τί ἐποίησας;
\VS{36}Αὐτῷ τῷ καιρῷ αἱ φρένες μου ἐπεστράφησαν ἐπʼ ἐμέ, καὶ εἰς τὴν τιμὴν τῆς βασιλείας μου ἦλθον· καὶ ἡ μορφή μου ἐπέστρεψεν ἐπʼ ἐμέ, καὶ οἱ τύραννοί μου, καὶ οἱ μεγιστᾶνές μου ἐζήτουν με, καὶ ἐπὶ τὴν βασιλείαν μου ἐκραταιώθην, καὶ μεγαλωσύνη περισσοτέρα προσετέθη μοι.
\par }{\PP \VS{37}Νῦν οὖν ἐγὼ Ναβουχοδονόσορ αἰνῶ καὶ ὑπερυψῶ καὶ δοξάζω τὸν βασιλέα τοῦ οὐρανοῦ, ὅτι πάντα τὰ ἔργα αὐτοῦ ἀληθινὰ, καὶ αἱ τρίβοι αὐτοῦ κρίσεις, καὶ πάντας τοὺς πορευομένους ἐν ὑπερηφανίᾳ δύναται ταπεινῶσαι.

\par }\Chap{5}{\PP \VerseOne{1}Βαλτάσαρ ὁ βασιλεὺς ἐποίησε δεῖπνον μέγα τοῖς μεγιστᾶσιν αὐτοῦ χιλίοις, καὶ κατέναντι τῶν χιλίων ὁ οἶνος,
\VS{2}καὶ πίνων Βαλτάσαρ εἶπεν ἐν τῇ γεύσει τοῦ οἴνου, τοῦ ἐνεγκεῖν τὰ σκεύη τὰ χρυσᾶ καὶ τὰ ἀργυρᾶ, ἃ ἐξήνεγκε Ναβουχοδονόσορ ὁ πατὴρ αὐτοῦ ἐκ τοῦ ναοῦ τοῦ ἐν Ἰερουσαλὴμ, καὶ πιέτωσαν ἐν αὐτοῖς ὁ βασιλεὺς, καὶ οἱ μεγιστᾶνες αὐτοῦ, καὶ αἱ παλλακαὶ αὐτοῦ, καὶ αἱ παράκοιτοι αὐτοῦ.
\VS{3}Καὶ ἠνέχθησαν τὰ σκεύη τὰ χρυσᾶ καὶ τὰ ἀργυρᾶ, ἃ ἐξήνεγκεν ἐκ τοῦ ναοῦ τοῦ Θεοῦ τοῦ ἐν Ἰερουσαλὴμ, καὶ ἔπινον ἐν αὐτοῖς ὁ βασιλεὺς, καὶ οἱ μεγιστᾶνες αὐτοῦ, καὶ αἱ παλλακαὶ αὐτοῦ, καὶ αἱ παράκοιτοι αὐτοῦ.
\VS{4}Ἔπινον οἶνον· καὶ ᾔνεσαν τοὺς θεοὺς τοὺς χρυσοῦς, καὶ ἀργυροῦς, καὶ χαλκοῦς, καὶ σιδηροῦς, καὶ ξυλίνους, καὶ λιθίνους.
\par }{\PP \VS{5}Ἐν αὐτῇ τῇ ὥρᾳ ἐξῆλθον δάκτυλοι χειρὸς ἀνθρώπου, καὶ ἔγραφον κατέναντι τῆς λαμπάδος ἐπὶ τὸ κονίαμα τοῦ τοίχου τοῦ οἴκου τοῦ βασιλέως, καὶ ὁ βασιλεὺς ἐθεώρει τοὺς ἀστραγάλους τῆς χειρὸς τῆς γραφούσης.
\VS{6}Τότε τοῦ βασιλέως ἡ μορφὴ ἠλλοιώθη, καὶ οἱ διαλογισμοὶ αὐτοῦ συνετάρασσον αὐτὸν, καὶ οἱ σύνδεσμοι τῆς ὀσφύος αὐτοῦ διελύοντο, καὶ τὰ γόνατα αὐτοῦ συνεκροτοῦντο.
\VS{7}Καὶ ἐβόησεν ὁ βασιλεὺς ἐν ἰσχύϊ, τοῦ εἰσαγαγεῖν μάγους, Χαλδαίους, γαζαρηνούς· καὶ εἶπε τοῖς σοφοῖς Βαβυλῶνος, ὃς ἂν ἀναγνῷ τὴν γραφὴν ταύτην, καὶ τὴν σύγκρισιν γνωρίσῃ μοι, πορφύραν ἐνδύσεται, καὶ ὁ μανιάκης ὁ χρυσοῦς ἐπὶ τὸν τράχηλον αὐτοῦ, καὶ τρίτος ἐν τῇ βασιλείᾳ μου ἄρξει.
\VS{8}Καὶ εἰσεπορεύοντο πάντες οἱ σοφοὶ τοῦ βασιλέως, καὶ οὐκ ἠδύναντο τὴν γραφὴν ἀναγνῶναι, οὐδὲ τὴν σύγκρισιν γνωρίσαι τῷ βασιλεῖ.
\VS{9}Καὶ ὁ βασιλεὺς Βαλτάσαρ ἐταράχθη, καὶ ἡ μορφὴ αὐτοῦ ἠλλοιώθη ἐν αὐτῷ, καὶ οἱ μεγιστᾶνες αὐτοῦ συνεταράσσοντο.
\par }{\PP \VS{10}Καὶ εἰσῆλθεν ἡ βασίλισσα εἰς τὸν οἶκον τοῦ πότου, καὶ εἶπε, βασιλεῦ, εἰς τὸν αἰῶνα ζῆθι· μὴ ταρασσέτωσάν σε οἱ διαλογισμοί σου, καὶ ἡ μορφή σου μὴ ἀλλοιούσθω.
\VS{11}Ἔστιν ἀνὴρ ἐν τῇ βασιλείᾳ σου, ἐν ᾧ πνεῦμα Θωοῦ· καὶ ἐν ταῖς ἡμέραις τοῦ πατρός σου, γρηγόρησις καὶ σύνεσις εὑρέθη ἐν αὐτῷ, καὶ ὁ βασιλεὺς Ναβουχοδονόσορ ὁ πατήρ σου ἄρχοντα ἐπαοιδῶν, μάγων, Χαλδαίων, γαζαρηνῶν, κατέστησεν αὐτὸν,
\VS{12}ὅτι πνεῦμα περισσὸν ἐν αὐτῷ, καὶ φρόνησις καὶ σύνεσις ἐν αὐτῷ, συγκρίνων ἐνύπνια, καὶ ἀναγγέλλων κρατούμενα, καὶ λύων συνδέσμους, Δανιήλ· καὶ ὁ βασιλεὺς ἐπέθηκεν ὄνομα αὐτῷ, Βαλτάσαρ· νῦν οὖν κληθήτω, καὶ τὴν σύγκρισιν αὐτοῦ ἀναγγελεῖ σοι.
\par }{\PP \VS{13}Τότε Δανιὴλ εἰσήχθη ἐνώπιον τοῦ βασιλέως· καὶ εἶπεν ὁ βασιλεὺς τῷ Δανιὴλ, σὺ εἶ Δανιὴλ, ὁ ἀπὸ τῶν υἱῶν τῆς αἰχμαλωσίας τῆς Ἰουδαίας, ἧς ἤγαγεν ὁ βασιλεὺς ὁ πατήρ μου;
\VS{14}Ἤκουσα περὶ σοῦ ὅτι πνεῦμα Θεοῦ ἐν σοί, καὶ γρηγόρησις, καὶ σύνεσις, καὶ σοφία περισσὴ εὑρέθη ἐν σοί.
\VS{15}Καὶ νῦν εἰσῆλθον ἐνώπιόν μου οἱ σοφοὶ, μάγοι, γαζαρηνοὶ, ἵνα τὴν γραφὴν ταύτην ἀναγνῶσι, καὶ τὴν σύγκρισιν γνωρίσωσί μοι, καὶ οὐκ ἠδυνήθησαν ἀναγγεῖλαί μοι.
\VS{16}Καὶ ἐγὼ ἤκουσα περὶ σοῦ, ὅτι δύνασαι κρίματα συγκρῖναι· νῦν οὖν ἐὰν δυνηθῇς τὴν γραφὴν ἀναγνῶναι, καὶ τὴν σύγκρισιν αὐτῆς γνωρίσαι μοι, πορφύραν ἐνδύσῃ, καὶ ὁ μανιάκης ὁ χρυσοῦς ἔσται ἐπὶ τῷ τραχήλῳ σου, καὶ τρίτος ἐν τῇ βασιλείᾳ μου ἄρξεις.
\par }{\PP \VS{17}Καὶ εἶπε Δανιὴλ ἐνώπιον τοῦ βασιλέως, τὰ δόματά σου σοὶ ἔστω, καὶ τὴν δωρεὰν τῆς οἰκίας σου ἑτέρῳ δὸς, ἐγὼ δὲ τὴν γραφὴν ἀναγνώσομαι, καὶ τὴν σύγκρισιν αὐτῆς γνωρίσω σοι,
\VS{18}βασιλεῦ· ὁ Θεὸς ὁ ὕψιστος τὴν βασιλείαν, καὶ τὴν μεγαλωσύνην, καὶ τὴν τιμὴν, καὶ τὴν δόξαν ἔδωκε Ναβουχοδονόσορ τῷ πατρί σου·
\VS{19}Καὶ ἀπὸ τῆς μεγαλωσύνης ἧς ἔδωκεν αὐτῷ πάντες οἱ λαοὶ, φυλαὶ, γλῶσσαι ἦσαν τρέμοντες καὶ φοβούμενοι ἀπὸ προσώπου αὐτοῦ· οὓς ἠβούλετο αὐτὸς ἀνῄρει, καὶ οὓς ἠβούλετο αὐτὸς ἔτυπτε, καὶ οὓς ἠβούλετο αὐτὸς ὕψου, καὶ οὓς ἠβούλετο αὐτὸς ἐταπείνου.
\VS{20}Καὶ ὅτε ὑψώθη ἡ καρδία αὐτοῦ, καὶ τὸ πνεῦμα αὐτοῦ ἐκραταιώθη τοῦ ὑπερηφανεύσασθαι, κατηνέχθη ἀπὸ τοῦ θρόνου τῆς βασιλείας, καὶ ἡ τιμὴ ἀφῃρέθη ἀπʼ αὐτοῦ,
\VS{21}καὶ ἀπὸ τῶν ἀνθρώπων ἐξεδιώχθη, καὶ ἡ καρδία αὐτοῦ μετὰ τῶν θηρίων ἐδόθη, καὶ μετὰ τῶν ὀνάγρων ἡ κατοικία αὐτοῦ, καὶ χόρτον ὡς βοῦν ἐψώμιζον αὐτὸν, καὶ ἀπὸ τῆς δρόσου τοῦ οὐρανοῦ τὸ σῶμα αὐτοῦ ἐβάφη, ἕως οὗ ἔγνω ὅτι κυριεύει ὁ Θεὸς ὕψιστος τῆς βασιλείας τῶν ἀνθρώπων, καὶ ᾧ ἂν δόξῃ δώσει αὐτήν.
\par }{\PP \VS{22}Καὶ σὺ οὖν ὁ υἱὸς αὐτοῦ Βαλτάσαρ οὐκ ἐταπείνωσας τὴν καρδίαν σου κατενώπιον τοῦ Θεοῦ· οὐ πάντα ταῦτα ἔγνως;
\VS{23}Καὶ ἐπὶ τὸν Κύριον Θεὸν τοῦ οὐρανοῦ ὑψώθης, καὶ τὰ σκεύη τοῦ οἴκου αὐτοῦ ἤνεγκαν ἐνώπιόν σου, καὶ σὺ καὶ οἱ μεγιστᾶνές σου, καὶ αἱ παλλακαί σου, καὶ αἱ παράκοιτοί σου οἶνον ἐπίνετε ἐν αὐτοῖς· καὶ τοὺς θεοὺς τοὺς χρυσοῦς, καὶ ἀργυροῦς, καὶ χαλκοῦς, καὶ σιδηροῦς, καὶ ξυλίνους, καὶ λιθίνους, οἳ οὐ βλέπουσι, καὶ οἳ οὐκ ἀκούουσι, καὶ οὐ γινώσκουσιν, ᾔνεσας, καὶ τὸν Θεὸν οὗ ἡ πνοή σου ἐν χειρὶ αὐτοῦ καὶ πᾶσαι αἱ ὁδοί σου, αὐτὸν οὐκ ἐδόξασας.
\VS{24}Διατοῦτο ἐκ προσώπου αὐτοῦ ἀπεστάλη ἀστράγαλος χειρὸς, καὶ τὴν γραφὴν ταύτην ἐνέταξε.
\par }{\PP \VS{25}Καὶ αὕτη ἡ γραφὴ ἐντεταγμένη, Μανὴ, Θεκὲλ, Φάρες.
\VS{26}Τοῦτο τὸ σύγκριμα τοῦ ῥήματος· Μανὴ, ἐμέτρησεν ὁ Θεὸς τὴν βασιλείαν σου, καὶ ἐπλήρωσεν αὐτήν.
\VS{27}Θεκὲλ, ἐστάθη ἐν ζυγῷ, καὶ εὑρέθη ὑστεροῦσα.
\VS{28}Φάρες, διήρηται ἡ βασιλεία σου, καὶ ἐδόθη Μήδοις καὶ Πέρσαις.
\par }{\PP \VS{29}Καὶ εἶπε Βαλτάσαρ, καὶ ἐνέδυσαν τον Δανιὴλ πορφύραν, καὶ τὸν μανιάκην τὸν χρυσοῦν περιέθηκαν περὶ τὸν τράχηλον αὐτοῦ, καὶ ἐκήρυξε περὶ αὐτοῦ, εἶναι αὐτὸν ἄρχοντα τρίτον ἐν τῇ βασιλείᾳ.
\VS{30}Ἐν αὐτῇ τῇ νυκτὶ ἀνῃρέθη Βαλτάσαρ ὁ βασιλεὺς ὁ Χαλδαῖος,

\Chap{6}\VerseOne{1}καὶ Δαρεῖος ὁ Μῆδος παρέλαβε τὴν βασιλείαν, ὢν ἐτῶν ἑξήκοντα δύο.
\par }{\PP \VS{2}Καὶ ἤρεσεν ἐνώπιον Δαρείου, καὶ κατέστησεν ἐπὶ τῆς βασιλείας σατράπας ἑκατὸν εἴκοσι, τοῦ εἶναι αὐτοὺς ἐν ὅλῃ τῇ βασιλείᾳ αὐτοῦ,
\VS{3}καὶ ἐπάνω αὐτῶν τακτικοὺς τρεῖς, ὃς ἦν Δανιὴλ εἷς ἐξ αὐτῶν, τοῦ ἀποδιδόναι αὐτοῖς τοὺς σατράπας λόγον, ὅπως ὁ βασιλεὺς μὴ ἐνοχλῆται.
\VS{4}Καὶ ἦν Δανιὴλ ὑπὲρ αὐτοὺς, ὅτι πνεῦμα περισσὸν ἐν αὐτῷ, καὶ ὁ βασιλεὺς κατέστησεν αὐτὸν ἐφʼ ὅλης τῆς βασιλείας αὐτοῦ.
\par }{\PP \VS{5}Καὶ οἱ τακτικοὶ καὶ οἱ σατράπαι ἐζήτουν πρόφασιν εὑρεῖν κατὰ Δανιήλ· καὶ πᾶσαν πρόφασιν καὶ παράπτωμα καὶ ἀμπλάκημα οὐχ εὗρον κατʼ αὐτοῦ, ὅτι πιστὸς ἦν.
\VS{6}Καὶ εἶπον οἱ τακτικοὶ, οὐχ εὑρήσομεν κατὰ Δανιὴλ πρόφασιν, εἰ μὴ ἐν νομίμοις Θεοῦ αὐτοῦ.
\par }{\PP \VS{7}Τότε οἱ τακτικοὶ καὶ οἱ σατράπαι παρέστησαν τῷ βασιλεῖ, καὶ εἶπαν αὐτῷ, Δαρεῖε βασιλεῦ, εἰς τοὺς αἰῶνας ζῆθι.
\VS{8}Συνεβουλεύσαντο πάντες οἱ ἐπὶ τῆς βασιλείας σου στρατηγοὶ καὶ σατράπαι, ὕπατοι καὶ τοπάρχαι, τοῦ στῆσαι στάσει βασιλικῇ, καὶ ἐνισχῦσαι ὁρισμόν, ὅπως ὃς ἂν αἰτήσῃ αἴτημα παρὰ παντὸς θεοῦ καὶ ἀνθρώπου, ἕως ἡμερῶν τριάκοντα, ἀλλʼ ἢ παρὰ σοῦ βασιλεῦ, ἐμβληθήσεται εἰς τὸν λάκκον τῶν λεόντων.
\VS{9}Νῦν οὖν βασιλεῦ στῆσον τὸν ὁρισμὸν, καὶ ἔκθες γραφὴν, ὅπως μὴ ἀλλοιωθῇ τὸ δόγμα Περσῶν καὶ Μήδων.
\VS{10}Τότε ὁ βασιλεὺς Δαρεῖος ἐπέταξε γραφῆναι τὸ δόγμα.
\par }{\PP \VS{11}Καὶ Δανιὴλ ἡνίκα ἔγνω ὅτι ἐνετάγη τὸ δόγμα, εἰσῆλθεν εἰς τὸν οἶκον αὐτοῦ· καὶ αἱ θυρίδες ἀνεῳγμέναι αὐτῷ ἐν τοῖς ὑπερῴοις αὐτοῦ κατέναντι Ἱερουσαλὴμ, καὶ καιροὺς τρεῖς τῆς ἡμέρας ἦν κάμπτων ἐπὶ τὰ γόνατα αὐτοῦ, καὶ προσευχόμενος καὶ ἐξομολογούμενος ἐναντίον τοῦ Θεοῦ αὐτοῦ, καθὼς ἦν ποιῶν ἔμπροσθεν.
\par }{\PP \VS{12}Τότε οἱ ἄνδρες ἐκεῖνοι παρετήρησαν, καὶ εὗρον τὸν Δανιὴλ ἀξιοῦντα καὶ δεόμενον τοῦ Θεοῦ αὐτοῦ.
\VS{13}Καὶ προσελθόντες λέγουσι τῷ βασιλεῖ, βασιλεῦ, οὐχ ὁρισμὸν ἔταξας, ὅπως πᾶς ἄνθρωπος ὃς ἂν αἰτήσῃ παρὰ παντὸς θεοῦ καὶ ἀνθρώπου αἴτημα, ἕως ἡμερῶν τριάκοντα, ἀλλʼ ἢ παρὰ σοῦ βασιλεῦ, ἐμβληθήσεται εἰς τὸν λάκκον τῶν λεόντων; καὶ εἶπεν ὁ βασιλεὺς, ἀληθινὸς ὁ λόγος, καὶ τὸ δόγμα Μήδων καὶ Περσῶν οὐ παρελεύσεται.
\VS{14}Τότε ἀπεκρίθησαν καὶ λέγουσιν ἐνώπιον τοῦ βασιλέως, Δανιὴλ, ὁ ἀπὸ τῶν υἱῶν τῆς αἰχμαλωσίας τῆς Ἰουδαίας, οὐχ ὑπετάγη τῷ δόγματί σου· καὶ καιροὺς τρεῖς τῆς ἡμέρας αἰτεῖ παρὰ τοῦ Θεοῦ αὐτοῦ τὰ αἰτήματα αὐτοῦ.
\VS{15}Τότε ὁ βασιλεὺς, ὡς τὸ ῥῆμα ἤκουσε, πολὺ ἐλυπήθη ἐπʼ αὐτῷ, καὶ περὶ τοῦ Δανιὴλ ἠγωνίσατο τοῦ ἐξελέσθαι αὐτὸν, καὶ ἕως ἑσπέρας ἦν ἀγωνιζόμενος τοῦ ἐξελέσθαι αὐτόν.
\par }{\PP \VS{16}Τότε οἱ ἄνδρες ἐκεῖνοι λέγουσι τῷ βασιλεῖ, γνῶθι βασιλεῦ, ὅτι τὸ δόγμα Μήδοις καὶ Πέρσαις, τοῦ πᾶν ὁρισμὸν καὶ στάσιν ἣν ἂν ὁ βασιλεὺς στήσῃ, οὐ δεῖ παραλλάξαι.
\VS{17}Τότε ὁ βασιλεὺς εἶπε· καὶ ἢγαγον τὸν Δανιὴλ, καὶ ἐνέβαλον αὐτὸν εἰς τὸν λάκκον τῶν λεόντων· καὶ εἶπεν ὁ βασιλεὺς τῷ Δανιὴλ, ὁ Θεός σου, ᾧ σὺ λατρεύεις ἐνδελεχῶς, αὐτὸς ἐξελεῖταί σε.
\VS{18}Καὶ ἤνεγκαν λίθον, καὶ ἐπέθηκαν ἐπὶ τὸ στόμα τοῦ λάκκου, καὶ ἐσφραγίσατο ὁ βασιλεὺς ἐν τῷ δακτυλίῳ αὐτοῦ, καὶ ἐν τῷ δακτυλίῳ τῶν μεγιστάνων αὐτοῦ, ὅπως μὴ ἀλλοιωθῇ πρᾶγμα ἐν τῷ Δανιήλ.
\VS{19}Καὶ ἀπῆλθεν ὁ βασιλεὺς εἰς τὸν οἶκον αὐτοῦ, καὶ ἐκοιμήθη ἄδειπνος, καὶ ἐδέσματα οὐκ εἰσήνεγκαν αὐτῷ· καὶ ὁ ὕπνος ἀπέστη ἀπʼ αὐτοῦ· καὶ ἔκλεισεν ὁ Θεὸς τὰ στόματα τῶν λεόντων, καὶ οὐ παρηνώχλησαν τῷ Δανιήλ.
\par }{\PP \VS{20}Τότε ὁ βασιλεὺς ἀνέστη τὸ πρωῒ ἐν τῷ φωτὶ, καὶ ἐν σπουδῇ ἦλθεν ἐπὶ τὸν λάκκον τῶν λεόντων.
\VS{21}Καὶ ἐν τῷ ἐγγίζειν αὐτὸν τῷ λάκκῳ, ἐβοήσε φωνῇ ἰσχυρᾷ, Δανιὴλ, ὁ δοῦλος τοῦ Θεοῦ τοῦ ζῶντος, ὁ Θεός σου ᾧ σὺ λατρεύεις ἐνδελεχῶς, εἰ ἠδυνήθη ἐξελέσθαί σε ἐκ στόματος τῶν λεόντων;
\VS{22}Καὶ εἶπε Δανιὴλ τῷ βασιλεῖ, βασιλεῦ, εἰς τοὺς αἰῶνας ζῆθι.
\VS{23}Ὁ Θεός μου ἀπέστειλε τὸν ἄγγελον αὐτοῦ, καὶ ἐνέφραξε τὰ στόματα τῶν λεόντων, καὶ οὐκ ἐλυμήναντό με, ὅτι κατέναντι αὐτοῦ εὐθύτης εὑρέθη ἐμοὶ, καὶ ἐνώπιον δὲ σου βασιλεῦ παράπτωμα οὐκ ἐποίησα.
\VS{24}Τότε ὁ βασιλεὺς πολὺ ἠγαθύνθη ἐπʼ αὐτῷ, καὶ τὸν Δανιὴλ εἶπεν ἀνενέγκαι ἐκ τοῦ λάκκου· καὶ ἀνηνέχθη Δανιὴλ ἐκ τοῦ λάκκου· καὶ πᾶσα διαφθορὰ οὐχ εὑρέθη ἐν αὐτῷ, ὅτι ἐπίστευσεν ἐν τῷ Θεῷ αὐτοῦ.
\par }{\PP \VS{25}Καὶ εἶπεν ὁ βασιλεὺς, καὶ ἠγάγοσαν τοὺς ἄνδρας τοὺς διαβαλόντας τὸν Δανιὴλ, καὶ εἰς τὸν λάκκον τῶν λεόντων ἐνεβλήθησαν αὐτοὶ, καὶ οἱ υἱοὶ αὐτῶν, καὶ αἱ γυναῖκες αὐτῶν· καὶ οὐκ ἔφθασαν εἰς τὸ ἔδαφος τοῦ λάκκου, ἕως οὗ ἐκυρίευσαν αὐτῶν οἱ λέοντες, καὶ πάντα τὰ ὀστᾶ αὐτῶν ἐλέπτυναν.
\par }{\PP \VS{26}Τότε Δαρεῖος ὁ βασιλεὺς ἔγραψε πᾶσι τοῖς λαοῖς, φυλαῖς, γλώσσαις, τοῖς οἰκοῦσιν ἐν πάσῃ τῇ γῇ· εἰρήνη ὑμῖν πληθυνθείη.
\VS{27}Ἐκ προσώπου μου ἐτέθη δόγμα τοῦτο ἐν πάσῃ ἀρχῇ τῆς βασιλείας μου, εἶναι τρέμοντας καὶ φοβουμένους ἀπὸ προσώπου τοῦ Θεοῦ Δανιήλ, ὅτι αὐτός ἐστι Θεὸς ζῶν, καὶ μένων εἰς τοὺς αἰῶνας, καὶ ἡ βασιλεία αὐτοῦ οὐ διαφθαρήσεται, καὶ ἡ κυρεία αὐτοῦ ἕως τέλους·
\VS{28}Ἀντιλαμβάνεται καὶ ῥύεται, καὶ ποιεῖ σημεῖα καὶ τέρατα ἐν τῷ οὐρανῷ καὶ ἐπὶ τῆς γῆς, ὅστις ἐξείλατο τὸν Δανιὴλ ἐκ χειρὸς τῶν λεόντων.
\VS{29}Καὶ Δανιὴλ κατηύθυνεν ἐν τῇ βασιλείᾳ Δαρείου, καὶ ἐν τῇ βασιλείᾳ Κύρου τοῦ Πέρσου.

\par }\Chap{7}{\PP \VerseOne{1}Ἐν ἔτει τρώτῳ τῷ Βαλτάσαρ βασιλέως Χαλδαίων, Δανιὴλ ἐνύπνιον εἶδε, καὶ αἱ ὁράσεις τῆς κεφαλῆς αὐτοῦ ἐπὶ τῆς κοίτης αὐτοῦ· καὶ τὸ ἐνύπνιον αὐτοῦ ἔγραψεν.
\par }{\PP \VS{2}Ἐγὼ Δανιὴλ ἐθεώρουν· καὶ ἰδοὺ οἱ τέσσαρες ἄνεμοι τοῦ οὐρανοῦ προσέβαλον εἰς τὴν θάλασσαν τὴν μεγάλην·
\VS{3}Καὶ τέσσαρα θηρία μεγάλα ἀνέβαινον ἐκ τῆς θαλάσσης, διαφέροντα ἀλλήλων.
\VS{4}Τὸ πρῶτον ὡσεὶ λέαινα, καὶ πτερὰ αὐτῆς ὡς ἀετοῦ· ἐθεώρουν ἕως οὗ ἐξετίλη τὰ πτερὰ αὐτῆς· καὶ ἐξῄρθη ἀπὸ τῆς γῆς, καὶ ἐπὶ ποδῶν ἀνθρώπου ἐστάθη, καὶ καρδία ἀνθρώπου ἐδόθη αὐτῇ.
\VS{5}Καὶ ἰδοὺ θηρίον δεύτερον ὅμοιον ἄρκτῳ, καὶ εἰς μέρος ἓν ἐστάθη, καὶ τρεῖς πλευραὶ ἐν τῷ στόματι αὐτῆς, ἀναμέσον τῶν ὀδόντων αὐτῆς· καὶ οὕτως ἔλεγον αὐτῇ, ἀνάστηθι, φάγε σάρκας πολλάς.
\VS{6}Ὀπίσω τούτου ἐθεώρουν, καὶ ἰδοὺ θηρίον ἕτερον ὡσεὶ παρδάλις· καὶ αὐτῇ πτερὰ τέσσαρα πετεινοῦ ὑπεράνω αὐτῆς, καὶ τέσσαρες κεφαλαὶ τῷ θηρίῳ, καὶ ἐξουσία ἐδόθη αὐτῇ.
\VS{7}Ὀπίσω τούτου ἐθεώρουν, καὶ ἰδοὺ θηρίον τέταρτον φοβερὸν καὶ ἔκθαμβον, καὶ ἰσχυρὸν περισσῶς, καὶ οἱ ὀδόντες αὐτοῦ σιδηροῖ, ἐσθίον, καὶ λεπτύνον, καὶ τὰ ἐπίλοιπα τοῖς ποσὶν αὐτοῦ συνεπάτει, καὶ αὐτὸ διάφορον περισσῶς παρὰ πάντα τὰ θηρία τὰ ἔμπροσθεν αὐτοῦ· καὶ κέρατα δέκα αὐτῷ.
\VS{8}Προσενόουν τοῖς κέρασιν αὐτοῦ, καὶ ἰδοὺ κέρας ἕτερον μικρὸν ἀνέβη ἐν μέσῳ αὐτῶν, καὶ τρία κέρατα τῶν ἔμπροσθεν αὐτοῦ ἐξεῤῥιζώθη ἀπὸ προσώπου αὐτοῦ· καὶ ἰδοὺ ὀφθαλμοὶ, ὡσεὶ ὀφθαλμοὶ ἀνθρώπου ἐν τῷ κέρατι τούτῳ, καὶ στόμα λαλοῦν μεγάλα.
\par }{\PP \VS{9}Ἐθεώρουν ἕως ὅτου οἱ θρόνοι ἐτέθησαν, καὶ παλαιὸς ἡμερῶν ἐκάθητο, καὶ τὸ ἔνδυμα αὐτοῦ λευκὸν ὡσεὶ χιὼν, καὶ ἡ θρὶξ τῆς κεφαλῆς αὐτοῦ ὡσεὶ ἔριον καθαρόν, ὁ θρόνος αὐτοῦ φλὸξ πυρός, οἱ τροχοὶ αὐτοῦ πῦρ φλέγον.
\VS{10}Ποταμὸς πυρὸς εἷλκεν ἔμπροσθεν αὐτοῦ· χίλιαι χιλιάδες ἐλειτούργουν αὐτῷ, καὶ μύριαι μυριάδες παρειστήκεισαν αὐτῷ· κριτήριον ἐκάθισε, καὶ βίβλοι ἠνεῴχθησαν.
\VS{11}Ἐθεώρουν τότε ἀπὸ φωνῆς τῶν λόγων τῶν μεγάλων, ὧν τὸ κέρας ἐκεῖνο ἐλάλει, ἕως ἀνῃρέθη τὸ θηρίον, καὶ ἀπώλετο, καὶ τὸ σῶμα αὐτοῦ ἐδόθη εἰς καῦσιν πυρός.
\VS{12}Καὶ τῶν λοιπῶν θηρίων μετεστάθη ἡ ἀρχὴ, καὶ μακρότης ζωῆς ἐδόθη αὐτοῖς ἕως καιροῦ καὶ καιροῦ.
\par }{\PP \VS{13}Ἐθεώρουν ἐν ὁράματι τῆς νυκτός, καὶ ἰδοὺ μετὰ τῶν νεφελῶν τοῦ οὐρανοῦ, ὡς υἱὸς ἀνθρώπου ἐρχόμενος, καὶ ἕως τοῦ παλαιοῦ τῶν ἡμερῶν ἔφθασε, καὶ προσηνέχθη αὐτῷ.
\VS{14}Καὶ αὐτῷ ἐδόθη ἡ ἀρχὴ καὶ ἡ τιμὴ καὶ ἡ βασιλεία, καὶ πάντες οἱ λαοὶ, φυλαὶ, καὶ γλῶσσαι αὐτῷ δουλεύσουσιν· ἡ ἐξουσία αὐτοῦ, ἐξουσία αἰώνιος, ἥτις οὐ παρελεύσεται, καὶ ἡ βασιλεία αὐτοῦ οὐ διαφθαρήσεται.
\par }{\PP \VS{15}Ἔφριξε τὸ πνεῦμά μου ἐν τῇ ἕξει μου, ἐγὼ Δανιὴλ, καὶ αἱ ὁράσεις τῆς κεφαλῆς μου ἐτάρασσόν με.
\VS{16}Καὶ προσῆλθον ἑνὶ τῶν ἑστηκότων, καὶ τὴν ἀκρίβειαν ἐζήτουν παρʼ αὐτοῦ μαθεῖν περὶ πάντων τούτων· καὶ εἶπέ μοι τὴν ἀκρίβειαν, καὶ τὴν σύγκρισιν τῶν λόγων ἐγνώρισέ μοι.
\VS{17}Ταῦτα τὰ θηρία τὰ τέσσαρα, τέσσαρες βασιλεῖαι ἀναστήσονται ἐπὶ τῆς γῆς, αἳ ἀρθήσονται·
\VS{18}καὶ παραλήψονται τὴν βασιλείαν ἅγιοι ὑψίστου, καὶ καθέξουσιν αὐτὴν ἕως αἰῶνος τῶν αἰώνων.
\par }{\PP \VS{19}Καὶ ἐζήτουν ἀκριβῶς περὶ τοῦ θηοίου τοῦ τετάρτου· ὃτι ἦν διαφέρον παρὰ πᾶν θηρίον, φοβερὸν περισσῶς, οἱ ὀδόντες αὐτοῦ σιδηροῖ, καὶ ὄνυχες αὐτοῦ χαλκοῖ, ἐσθίον, καὶ λεπτῦνον, καὶ τὰ ἐπίλοιπα τοῖς ποσὶν αὐτοῦ συνεπάτει·
\VS{20}Καὶ περὶ τῶν κεράτων αὐτοῦ τῶν δέκα τῶν ἐν τῇ κεφαλῇ αὐτοῦ, καὶ τοῦ ἐτέρου τοῦ ἀναβάντος, καὶ ἐκτινάξαντος τῶν πρώτων, ᾧ οἱ ὀφθαλμοὶ καὶ στόμα λαλοῦν μεγάλα, καὶ ἡ ὅρασις αὐτοῦ μείζων τῶν λοιπῶν.
\VS{21}Ἐθεώρουν, καὶ τὸ κέρας ἐκεῖνο ἐποίει πόλεμον μετὰ τῶν ἁγίων, καὶ ἴσχυσε πρὸς αὐτοὺς,
\VS{22}ἕως οὗ ἦλθεν ὁ παλαιὸς ἡμερῶν, καὶ τὸ κρίμα ἔδωκεν ἁγίοις ὑψίστου· καὶ ὁ καιρὸς ἔφθασε, καὶ τὴν βασιλείαν κατέσχον οἱ ἅγιοι.
\VS{23}Καὶ εἶπε, τὸ θηρίον τὸ τέταρτον, βασιλεία τετάρτη ἔσται ἐν τῇ γῇ, ἥτις ὑπερέξει πάσας τὰς βασιλείας, καὶ καταφάγεται πᾶσαν τὴν γῆν, καὶ συμπατήσει αὐτὴν καὶ κατακόψει.
\VS{24}Καὶ τὰ δέκα κέρατα αὐτοῦ, δέκα βασιλεῖς ἀναστήσονται, καὶ ὀπίσω αὐτῶν ἀναστήσεται ἕτερος, ὃς ὑπεροίσει κακοῖς πάντας τοὺς ἔμπροσθεν, καὶ τρεῖς βασιλεῖς ταπεινώσει,
\VS{25}καὶ λόγους πρὸς τὸν ὕψιστον λαλήσει, καὶ τοὺς ἁγίους ὑψίστου παλαιώσει, καὶ ὑπονοήσει τοῦ ἀλλοιῶσαι καιροὺς καὶ νόμον, καὶ δοθήσεται ἐν χειρὶ αὐτοῦ ἕως καιροῦ καὶ καιρῶν καί γε ἥμισυ καιροῦ.
\VS{26}Καὶ τὸ κριτήριον ἐκάθισε, καὶ τὴν ἀρχὴν μεταστήσουσι τοῦ ἀφανίσαι, καὶ τοῦ ἀπολέσαι ἕως τέλους.
\VS{27}Καὶ ἡ βασιλεία καὶ ἡ ἐξουσία καὶ ἡ μεγαλωσύνη τῶν βασιλέων τῶν ὑποκάτω παντὸς τοῦ οὐρανοῦ, ἐδόθη ἁγίοις ὑψίστου· καὶ ἡ βασιλεία αὐτοῦ, βασιλεία αἰώνιος, καὶ πᾶσαι αἱ ἀρχαὶ αὐτῷ δουλεύσουσι καὶ ὑπακούσονται.
\par }{\PP Ἕως ὧδε τὸ πέρας τοῦ λόγου·
\VS{28}ἐγὼ Δανιήλ, οἱ διαλογισμοί μου ἐπὶ πολὺ συνετάρασσόν με, καὶ ἡ μορφή μου ἠλλοιώθη, καὶ τὸ ῥῆμα ἐν τῇ καρδίᾳ μου διετήρησα.

\par }\Chap{8}{\PP \VerseOne{1}Ἐν ἔτει τρίτῳ τῆς βασιλείας Βαλτάσαρ τοῦ βασιλέως ὅρασις ὤφθη πρὸς μέ· ἐγὼ Δανιὴλ μετὰ τὴν ὀφθεῖσάν μοι τὴν ἀρχήν,
\VS{2}καὶ ἤμην ἐν Σούσοις τῇ βάρει, ἥ ἐστιν ἐν χώρᾳ Αἰλάμ· καὶ ἤμην ἐπὶ τοῦ Οὐβάλ.
\VS{3}Καὶ ᾖρα τοὺς ὀφθαλμούς μου, καὶ ἴδον· καὶ ἰδοὺ κριὸς εἷς ἑστηκὼς πρὸ τοῦ Οὐβάλ· καὶ αὐτῷ κέρατα ὑψηλά· καὶ τὸ ἓν ὑψηλότερον τοῦ ἑτέρου, καὶ τὸ ὑψηλὸν ἀνέβαινεν ἐπʼ ἐσχάτων.
\VS{4}Καὶ ἴδον τὸν κριὸν κερατίζοντα κατὰ θάλασσαν, καὶ Βοῤῥᾶν, καὶ Νότον· καὶ πάντα τὰ θηρία οὐ στήσεται ἐνώπιον αὐτοῦ· καὶ οὐκ ἦν ὁ ἐξαιρούμενος ἐκ χειρὸς αὐτοῦ, καὶ ἐποίησε κατὰ τὸ θέλημα αὐτοῦ καὶ ἐμεγαλύνθη.
\par }{\PP \VS{5}Καὶ ἐγὼ ἤμην συνιῶν, καὶ ἰδοὺ τράγος αἰγῶν ἤρχετο ἀπὸ Λιβὸς ἐπὶ πρόσωπον πάσης τῆς γῆς, καὶ οὐκ ἦν ἁπτόμενος τῆς γῆς· καὶ τῷ τράγῳ κέρας μέσον τῶν ὀφθαλμῶν αὐτοῦ.
\VS{6}Καὶ ἦλθεν ἕως τοῦ κριοῦ τοῦ τὰ κέρατα ἔχοντος, οὗ ἴδον ἑστὼς ἐνώπιον τοῦ Οὐβάλ, καὶ ἔδραμε πρὸς αὐτὸν ἐν ὁρμῇ τῆς ἰσχύος αὐτοῦ.
\VS{7}Καὶ ἴδον αὐτὸν φθάνοντα ἕως τοῦ κριοῦ, καὶ ἐξηγριάνθη πρὸς αὐτόν, καὶ ἔπαισε τὸν κριὸν, καὶ συνέτριψεν ἀμφότερα τὰ κέρατα αὐτοῦ· καὶ οὐκ ἦν ἰσχὺς τῷ κριῷ, τοῦ στῆναι ἐνώπιον αὐτοῦ· καὶ ἔῤῥιψεν αὐτὸν ἐπὶ τὴν γῆν, καὶ συνεπάτησεν αὐτόν, καὶ οὐκ ἦν ὁ ἐξαιρούμενος τὸν κριὸν ἐκ χειρὸς αὐτοῦ.
\par }{\PP \VS{8}Καὶ ὁ τράγος τῶν αἰγῶν ἐμεγαλύνθη ἕως σφόδρα· καὶ ἐν τῷ ἰσχῦσαι αὐτὸν, συνετρίβη τὸ κέρας αὐτοῦ τὸ μέγα· καὶ ἀνέβη ἕτερα τέσσαρα ὑποκάτω αὐτοῦ εἰς τοὺς τέσσαρας ἀνέμους τοῦ οὐρανοῦ.
\VS{9}Καὶ ἐκ τοῦ ἑνὸς αὐτῶν ἐξῆλθε κέρας ἓν ἰσχυρόν, καὶ ἐμεγαλύνθη περισσῶς πρὸς τὸν Νότον, καὶ πρὸς τὴν δύναμιν,
\VS{10}καὶ ἐμεγαλύνθη ἕως τῆς δυνάμεως τοῦ οὐρανοῦ· καὶ ἔπεσεν ἐπὶ τὴν γῆν ἀπὸ τῆς δυνάμεως τοῦ οὐρανοῦ καὶ ἀπὸ τῶν ἄστρων, καὶ συνεπάτησαν αὐτά.
\VS{11}Καὶ ἕως οὗ ὁ ἀρχιστράτηγος ῥύσηται τὴν αἰχμαλωσίαν, καὶ διʼ αὐτὸν θυσία ἐταράχθη, καὶ κατευωδώθη αὐτῷ· καὶ τὸ ἅγιον ἐρημωθήσεται.
\VS{12}Καὶ ἐδόθη ἐπὶ τὴν θυσίαν ἁμαρτία, καὶ ἐῤῥρίφη χαμαὶ ἡ δικαιοσύνη· καὶ ἐποίησε, καὶ εὐωδώθη.
\VS{13}Καὶ ἤκουσα ἑνὸς ἁγίου λαλοῦντος· καὶ εἶπεν εἷς ἅγιος τῷ φελμουνὶ τῷ λαλοῦντι, ἕως πότε ἡ ὅρασις στήσεται, ἡ θυσία ἡ ἀρθεῖσα, καὶ ἡ ἁμαρτία ἐρημώσεως ἡ δοθεῖσα, καὶ τὸ ἅγιον καὶ ἡ δύναμις συμπατηθήσεται;
\VS{14}Καὶ εἶπεν αὐτῷ, ἕως ἑσπέρας καὶ πρωῒ ἡμέραι δισχίλιαι καὶ τετρακόσιαι, καὶ καθαρισθήσεται τὸ ἅγιον.
\par }{\PP \VS{15}Καὶ ἐγένετο ἐν τῷ ἰδεῖν με, ἐγὼ Δανιὴλ, τὴν ὅρασιν, καὶ ἐζήτουν σύνεσιν, καὶ ἰδοὺ ἔστη ἐνώπιον ἐμοῦ ὡς ὅρασις ἀνδρός.
\VS{16}Καὶ ἤκουσα φωνὴν ἀνδρὸς ἀναμέσον τοῦ Οὐβὰλ, καὶ ἐκάλεσε, καὶ εἶπε, Γαβριὴλ συνέτισον ἐκεῖνον τὴν ὅρασιν.
\VS{17}Καὶ ἦλθε, καὶ ἔστη ἐχόμενος τῆς στάσεώς μου· καὶ ἐν τῷ ἐλθεῖν αὐτὸν ἐθαμβήθην, καὶ πίπτω ἐπὶ πρόσωπόν μου· καὶ εἶπε πρὸς μὲ, σύνες υἱὲ ἀνθρώπου· ἔτι γὰρ εἰς καιροῦ πέρας ἡ ὅρασις·
\VS{18}Καὶ ἐν τῷ λαλεῖν αὐτὸν μετʼ ἐμοῦ, πίπτω ἐπὶ πρόσωπόν μου ἐπὶ τὴν γῆν, καὶ ἥψατό μου, καὶ ἔστησέ με ἐπὶ πόδας,
\VS{19}καὶ εἶπεν, ἰδοὺ ἐγὼ γνωρίζω σοι τὰ ἐσόμενα ἐπʼ ἐσχάτων τῆς ὀργῆς· ἔτι γὰρ εἰς καιροῦ πέρας ἡ ὅρασις.
\par }{\PP \VS{20}Ὁ κριὸς ὃν εἶδες, ὁ ἔχων τὰ κέρατα, βασιλεὺς Μήδων καὶ Περσῶν.
\VS{21}Ὁ τράγος τῶν αἰγῶν, βασιλεὺς Ἑλλήνων· καὶ τὸ κέρας τὸ μέγα ὃ ἦν ἀναμέσον τῶν ὀφθαλμῶν αὐτοῦ, αὐτός ἐστιν ὁ βασιλεὺς ὁ πρῶτος.
\VS{22}Καὶ τοῦ συντριβέντος οὗ ἔστησαν τέσσαρα κέρατα ὑποκάτω, τέσσαρες βασιλεῖς ἐκ τοῦ ἔθνους αὐτοῦ ἀναστήσονται, καὶ οὐκ ἐν τῇ ἰσχύϊ αὐτῶν.
\VS{23}Καὶ ἐπʼ ἐσχάτων τῆς βασιλείας αὐτῶν, πληρουμένων τῶν ἁμαρτιῶν αὐτῶν, ἀναστήσεται βασιλεὺς ἀναιδὴς προσώπῳ, καὶ συνιῶν προβλήματα·
\VS{24}καὶ κραταιὰ ἡ ἰσχὺς αὐτοῦ, καὶ θαυμαστὰ διαφθερεῖ, καὶ κατευθυνεῖ, καὶ ποιήσει, καὶ διαφθερεῖ ἰσχυροὺς, καὶ λαὸν ἅγιον.
\VS{25}Καὶ ζυγὸς τοῦ κλοιοῦ αὐτοῦ κατευθυνεῖ, δόλος ἐν τῇ χειρὶ αὐτοῦ, καὶ ἐν καρδίᾳ αὐτοῦ μεγαλυνθήσεται, καὶ δόλῳ διαφθερεῖ πολλοὺς, καὶ ἐπὶ ἀπωλείας πολλῶν στήσεται· καὶ ὡς ὠὰ χειρὶ συντρίψει.
\VS{26}Καὶ ἡ ὅρασις τῆς ἑσπέρας καὶ τῆς πρωΐας τῆς ῥηθείσης ἀληθῶς ἐστι· καὶ σὺ σφράγισον τὴν ὅρασιν, ὅτι εἰς ἡμέρας πολλάς.
\par }{\PP \VS{27}Καὶ ἐγὼ Δανιὴλ ἐκοιμήθην, καὶ ἐμαλακίσθην, καὶ ἀνέστην, καὶ ἐποίουν τὰ ἔργα τοῦ βασιλέως, καὶ ἐθαύμαζον τὴν ὅρασιν, καὶ οὐκ ἦν ὁ συνιῶν.

\par }\Chap{9}{\PP \VerseOne{1}Ἐν τῷ πρώτῳ ἔτει Δαρείου τοῦ υἱοῦ Ἀσουήρου, ἀπὸ τοῦ σπέρματος τῶν Μήδων, ὃς ἐβασίλευσεν ἐπὶ βασιλείαν Χαλδαίων,
\VS{2}ἐγὼ Δανιὴλ συνῆκα ἐν ταῖς βίβλοις τὸν ἀριθμὸν τῶν ἐτῶν, ὃς ἐγενήθη λόγος Κυρίου πρὸς Ἱερεμίαν τὸν προφήτην, εἰς συμπλήρωσιν ἐρημώσεως Ἱερουσαλὴμ ἑβδομήκοντα ἔτη.
\par }{\PP \VS{3}Καὶ ἔδωκα τὸ πρόσωπόν μου πρὸς Κύριον τὸν Θεὸν, τοῦ ἐκζητῆσαι προσευχὴν καὶ δεήσεις ἐν νηστείαις καὶ σάκκῳ.
\VS{4}Καὶ προσευξάμην πρὸς Κύριον τὸν Θεόν μου, καὶ ἐξωμολογησάμην, καὶ εἶπα, Κύριε ὁ Θεὸς ὁ μέγας καὶ θαυμαστὸς, ὁ φυλάσσων τὴν διαθήκην σου, καὶ τὸ ἔλεός τοῖς ἀγαπῶσί σε σου καὶ τοῖς φυλάσσουσι, τὰς ἐντολάς σου,
\VS{5}ἡμάρτομεν, ἠδικήσαμεν, ἠνομήσαμεν, καὶ ἀπέστημεν καὶ ἐξεκλίναμεν ἀπὸ τῶν ἐντολῶν σου, καὶ ἀπὸ τῶν κριμάτων σου,
\VS{6}καὶ οὐκ εἰσηκούσαμεν τῶν δούλων σου τῶν προφητῶν οἳ ἐλάλουν ἐν τῷ ὀνόματί σου πρὸς τοὺς βασιλεῖς ἡμῶν, καὶ ἄρχοντας ἡμῶν, καὶ πατέρας ἡμῶν, καὶ πρὸς πάντα τὸν λαὸν τῆς γῆς.
\VS{7}Σοὶ Κύριε ἡ δικαιοσύνη καὶ ἡμῖν ἡ αἰσχύνη τοῦ προσώπου, ὡς ἡ ἡμέρα αὕτη, ἀνδρὶ Ἰούδα, καὶ τοῖς ἐνοικοῦσιν ἐν Ἱερουσαλὴμ, καὶ παντὶ Ἰσραὴλ, τοῖς ἐγγὺς καὶ τοῖς μακρὰν ἐν πάσῃ τῇ γῇ, οὗ διέσπειρας αὐτοὺς ἐκεῖ, ἐν ἀθεσίᾳ αὐτῶν ᾗ ἠθέτησαν.
\VS{8}Ἐν σοὶ Κύριέ ἐστιν ἡμῶν ἡ δικαιοσύνη, καὶ ἡμῖν ἡ αἰσχύνη τοῦ προσώπου, καὶ τοῖς βασιλεῦσιν ἡμῶν, καὶ τοῖς ἄρχουσιν ἡμῶν, καὶ τοῖς πατράσιν ἡμῶν, οἵτινες ἡμάρτομεν.
\VS{9}Σοὶ Κυρί τῷ Θεῷ ἡμῶν οἱ οἰκτιρμοὶ καὶ οἱ ἱλασμοὶ, ὅτι ἀπέστημεν,
\VS{10}καὶ οὐκ εἰσηκούσαμεν τῆς φωνῆς Κυρίου τοῦ Θεοῦ ἡμῶν, πορεύεσθαι ἐν τοῖς νόμοις αὐτοῦ, οἷς ἔδωκε κατὰ πρόσωπον ἡμῶν ἐν χερσὶ τῶν δούλων αὐτοῦ τῶν προφητῶν.
\par }{\PP \VS{11}Καὶ πᾶς Ἰσραὴλ παρέβησαν τὸν νόμον σου, καὶ ἐξέκλιναν τοῦ μὴ ἀκοῦσαι τῆς φωνῆς σου· καὶ ἐπῆλθεν ἐφʼ ἡμᾶς ἡ κατάρα, καὶ ὁ ὅρκος ὁ γεγραμμένος ἐν νόμῳ Μωυσέως δούλου τοῦ Θεοῦ, ὅτι ἡμάρτομεν αὐτῷ.
\VS{12}Καὶ ἔστησε τοὺς λόγους αὐτοῦ οὓς ἐλάλησεν ἐφʼ ἡμᾶς, καὶ ἐπὶ τοὺς κριτὰς ἡμῶν, οἳ ἔκρινον ἡμᾶς, ἐπαγαγεῖν ἐφʼ ἡμᾶς κακὰ μεγάλα, οἷα οὐ γέγονεν ὑποκάτω παντὸς τοῦ οὐρανοῦ, κατὰ τὰ γενόμενα ἐν Ἱερουσαλὴμ,
\VS{13}καθὼς γέγραπται ἐν τῷ νόμῳ Μωυσῇ· πάντα τὰ κακὰ τοῦτα ἦλθεν ἐφʼ ἡμᾶς· καὶ οὐκ ἐδεήθημεν τοῦ προσώπου Κυρίου τοῦ Θεοῦ ἡμῶν, ἀποστρέψαι ἀπὸ τῶν ἀδικιῶν ἡμῶν, καὶ τοῦ συνιέναι ἐν πάσῃ ἀληθείᾳ σου.
\VS{14}Καὶ ἐγρηγόρησε Κύριος, καὶ ἐπήγαγεν αὐτὰ ἐφʼ ἡμᾶς, ὅτι δίκαιος Κύριος ὁ Θεὸς ἡμῶν ἐπὶ πᾶσαν τὴν ποίησιν αὐτοῦ ἣν ἐποίησε, καὶ οὐκ εἰσηκούσαμεν τῆς φωνῆς αὐτοῦ.
\VS{15}Καὶ νῦν Κύριε ὁ Θεὸς ἡμῶν, ὃς ἐξήγαγες τὸν λαόν σου ἐκ γῆς Αἰγύπτου ἐν χειρὶ κραταιᾷ, καὶ ἐποίησας σεαυτῷ ὄνομα ὡς ἡ ἡμέρα αὕτη, ἡμάρτομεν, ἠνομήσαμεν.
\par }{\PP \VS{16}Κύριε ἐν πᾶσιν ἐλεημοσύνη σου, ἀποστραφήτω δὴ ὁ θυμός σου, καὶ ἡ ὀργή σου ἀπὸ τῆς πόλεώς σου Ἱερουσαλὴμ ὄρους ἁγίου σου, ὅτι ἡμάρτομεν, καὶ ἐν ταῖς ἀδικίαις ἡμῶν, καὶ τῶν πατέρων ἡμῶν, Ἱερουσαλὴμ, καὶ ὁ λαός σου εἰς ὀνειδισμὸν ἐγένετο ἐν πᾶσι τοῖς περικύκλῳ ἡμῶν.
\VS{17}Καὶ νῦν εἰσάκουσον Κύριε ὁ Θεὸς ἡμῶν τῆς προσευχῆς τοῦ δούλου σου καὶ τῶν δεήσεων αὐτοῦ, καὶ ἐπίφανον τὸ πρόσωπόν σου ἐπὶ τὸ ἁγίασμά σου τὸ ἔρημον, ἔνεκέν σου Κύριε,
\VS{18}κλῖνον ὁ Θεός μου τὸ οὖς σου, καὶ ἄκουσον· ἄνοιξον τοὺς ὀφθαλμούς σου, καὶ ἴδε τὸν ἀφανισμὸν ἡμῶν, καὶ τῆς πόλεως σου ἐφʼ ἧς ἐπικέκληται τὸ ὄνομά σου ἐπʼ αὐτῆς· ὅτι οὐκ ἐπὶ ταῖς δικαιοσύναις ἡμῶν ῥιπτοῦμεν τὸν οἰκτιρμὸν ἡμῶν ἐνώπιόν σου, ἀλλʼ ἐπὶ τοὺς οἰκτιρμούς σου τοὺς πολλοὺς Κύριε.
\VS{19}Εἰσάκουσον Κύριε, ἱλάσθητι Κύριε, πρόσχες Κύριε· μὴ χρονίσῃς ἕνεκέν σου ὁ Θεός μου, ὅτι τὸ ὄνομά σου ἐπικέκληται ἐπὶ τὴν πόλιν σου, καὶ ἐπὶ τὸν λαόν σου.
\par }{\PP \VS{20}Καὶ ἔτι ἐμοῦ λαλοῦντος, καὶ προσευχομένου, καὶ ἐξαγορεύοντος τὰς ἁμαρτίας μου, καὶ τὰς ἁμαρτίας τοῦ λαοῦ μου Ἰσραὴλ, καὶ ῥιπτοῦντος τὸν ἔλεόν μου ἐναντίον τοῦ Κυρίου τοῦ Θεοῦ μου περὶ τοῦ ὄρους τοῦ ἁγίου,
\VS{21}καὶ ἔτι ἐμοῦ λαλοῦντος ἐν τῇ προσευχῇ, καὶ ἰδοὺ ἀνὴρ Γαβριὴλ, ὃν ἴδον ἐν τῇ ὁράσει ἐν τῇ ἀρχῆ, πετόμενος, καὶ ἥψατό μου, ὡσεὶ ὥραν θυσίας ἑσπερινῆς.
\VS{22}Καὶ συνέτισέ με, καὶ ἐλάλησε μετʼ ἐμοῦ, καὶ εἶπε, Δανιὴλ, νῦν ἐξῆλθον συμβιβάσαι σε σύνεσιν
\VS{23}ἐν ἀρχῇ τῆς δεήσεώς σου, ἐξῆλθε λόγος, καὶ ἐγὼ ἦλθον τοῦ ἀναγγεῖλαί σοι, ὅτι ἀνὴρ ἐπιθυμιῶν εἶ σὺ, καὶ ἐννοήθητι ἐν τῷ ῥήματι, καὶ σύνες ἐν τῇ ὀπτασίᾳ.
\par }{\PP \VS{24}Ἑβδομήκοντα ἑβδομάδες συνετμήθησαν ἐπὶ τὸν λαόν σου, καὶ ἐπὶ τὴν πόλιν τὴν ἁγίαν, τοῦ συντελεσθῆναι ἁμαρτίαν, καὶ τοῦ σφραγίσαι ἁμαρτίας, καὶ ἀπαλεῖψαι τὰς ἀδικίας, καὶ τοῦ ἐξιλάσασθαι ἀδικίας, καὶ τοῦ ἀγαγεῖν δικαιοσύνην αἰώνιον, καὶ τοῦ σφραγίσαι ὅρασιν καὶ προφήτην, καὶ τοῦ χρίσαι ἅγιον ἁγίων.
\par }{\PP \VS{25}Καὶ γνώσῃ καὶ συνήσεις ἀπὸ ἐξόδου λόγου τοῦ ἀποκριθῆναι, καὶ τοῦ οἰκοδομῆσαι Ἱερουσαλὴμ, ἓως Χριστοῦ ἡγουμένου ἑβδομάδες ἑπτὰ, καὶ ἑβδομάδες ἑξηκονταδύο· καὶ ἐπιστρέψει, καὶ οἰκοδομηθήσεται πλατεία, καὶ τεῖχος, καὶ ἐκκενωθήσονται οἱ καιροί.
\par }{\PP \VS{26}Καὶ μετὰ τὰς ἑβδομάδας τὰς ἑξηκονταδύο, ἐξολοθρενθήσεται χρίσμα, καὶ κρίμα οὐκ ἔστιν ἐν αὐτῷ· καὶ τὴν πόλιν, καὶ τὸ ἅγιον διαφθερεῖ σὺν τῷ ἡγουμένῳ τῷ ἐρχομένῳ, ἐκκοπήσονται ἐν κατακλυσμῷ, καὶ ἕως τέλους πολέμου συντετμημένου τάξει, ἀφανισμοίς.
\par }{\PP \VS{27}Καὶ δυναμώσει διαθήκην πολλοῖς ἑβδομὰς μία· καὶ ἐν τῷ ἡμίσει τῆς ἑβδομάδος ἀρθήσεταί μου θυσία καὶ σπονδὴ, καὶ ἐπὶ τὸ ἱερὸν βδέλυγμα τῶν ἐρημώσεων, καὶ ἕως τῆς συντελείας καιροῦ συντέλεια δοθήσεται ἐπὶ τὴν ἐρήμωσιν.

\par }\Chap{10}{\PP \VerseOne{1}Ἐν ἔτει τρίτῳ Κύρου βασιλέως Περσῶν λόγος ἀπεκαλύφθη τῷ Δανιὴλ, οὗ τὸ ὄνομα ἐπεκλήθη Βαλτάσαρ· καὶ ἀληθινὸς ὁ λόγος, καὶ δύναμις μεγάλη καὶ σύνεσις ἐδόθη αὐτῷ ἐν τῇ ὀπτασίᾳ.
\VS{2}Ἐν ταῖς ἡμέραις ἐκείναις ἐγὼ Δανιὴλ ἤμην πενθῶν τρεῖς ἑβδομάδας ἡμερῶν,
\VS{3}ἄρτον ἐπιθυμιῶν οὐκ ἔφαγον, καὶ κρέας καὶ οἶνος οὐκ εἰσῆλθεν εἰς τὸ στόμα μου, καὶ ἄλειμμα οὐκ ἠλειψάμην, ἕως πληρώσεως τριῶν ἑβδομάδων ἡμερῶν.
\par }{\PP \VS{4}Ἐν ἡμέρᾳ εἰκοστῇ τετάρτῃ τοῦ μηνὸς τοῦ πρώτου, καὶ ἐγὼ ἤμην ἐχόμενα τοῦ ποταμοῦ τοῦ μεγάλου, αὐτός ἐστι Τίγρις Ἐδδεκέλ.
\VS{5}Καὶ ᾖρα τοὺς ὀφθαλμούς μου, καὶ ἴδον, καὶ ἰδοὺ ἀνὴρ εἷς ἐνδεδυμένος βαδδὶν, καὶ ἡ ὀσφὺς αὐτοῦ περιεζωσμένη ἐν χρυσίῳ Ὠφὰζ,
\VS{6}καὶ τὸ σῶμα αὐτοῦ ὡσεὶ Θαρσὶς, καὶ τὸ πρόσωπον αὐτοῦ ὡς ἡ ὅρασις ἀστραπῆς, καὶ οἱ ὀφθαλμοὶ αὐτοῦ ὡσεὶ λαμπάδες πυρὸς, καὶ οἱ βραχίονες αὐτοῦ καὶ τὰ σκέλη ὡς ὅρασις χαλκοῦ στίλβοντος, καὶ ἡ φωνὴ τῶν λόγων αὐτοῦ ὡς φωνὴ ὄχλου.
\VS{7}Καὶ ἴδον ἐγὼ Δανιὴλ μόνος τὴν ὀπτασίαν, καὶ οἱ ἄνδρες οἱ μετʼ ἐμοῦ οὐκ ἴδον τὴν ὀπτασίαν, ἀλλʼ ἢ ἔκστασις μεγάλη ἐπέπεσεν ἐπʼ αὐτοὺς, καὶ ἔφυγον ἐν φόβῳ.
\VS{8}Καὶ ἐγὼ ὑπελείφθην μόνος, καὶ ἴδον τὴν ὀπτασίαν τὴν μεγάλην ταύτην, καὶ οὐχ ὑπελείφθη ἐν ἐμοὶ ἰσχὺς, καὶ ἡ δόξα μου μετεστράφη εἰς διαφθοράν· καὶ οὐκ ἐκράτησα ἰσχύος.
\VS{9}Καὶ ἤκουσα τὴν φωνὴν τῶν λόγων αὐτοῦ· καὶ ἐν τῷ ἀκοῦσαί με αὐτοῦ, ἤμην κατανενυγμένος, καὶ τὸ πρόσωπόν μου ἐπὶ τὴν γῆν.
\par }{\PP \VS{10}Καὶ ἰδοὺ χεὶρ ἁπτομένη μου, καὶ ἤγειρέ με ἐπὶ τὰ γόνατά μου.
\VS{11}Καὶ εἶπε πρὸς μὲ, Δανιὴλ ἀνὴρ ἐπιθυμιῶν, σύνες ἐν τοῖς λόγοις οἷς ἐγὼ λαλῶ πρὸς σὲ, καὶ στῆθι ἐπὶ τῇ στάσει σου, ὅτι νῦν ἀπεστάλην πρὸς σέ· καὶ ἐν τῷ λαλῆσαι αὐτὸν πρὸς μὲ τὸν λόγον τοῦτον, ἀνέστην ἔντρομος.
\VS{12}Καὶ εἶπε πρὸς μὲ, μὴ φοβοῦ Δανιὴλ, ὅτι ἀπὸ τῆς πρώτης ἡμέρας ἧς ἔδωκας τὴν καρδίαν σου τοῦ συνεῖναι, καὶ κακωθῆναι ἐναντίον Κυρίου τοῦ Θεοῦ σου, ἠκούσθησαν οἱ λόγοι σου, καὶ ἐγὼ ἦλθον ἐν τοῖς λόγοις σου.
\VS{13}Καὶ ὁ ἄρχων βασιλείας Περσῶν εἱστήκει ἐξεναντίας μου εἴκοσι καὶ μίαν ἡμέραν· καὶ ἰδοὺ Μιχαὴλ εἷς τῶν ἀρχόντων ἦλθε βοηθῆσαί μοι, καὶ αὐτὸν κατέλιπον ἐκεῖ μετὰ τοῦ ἄρχοντος βασιλείας Περσῶν,
\VS{14}καὶ ἦλθον συνετίσαι σε ὅσα ἀπαντήσεται τῷ λαῷ σου ἐπʼ ἐσχάτων τῶν ἡμερῶν, ὅτι ἔτι ἡ ὅρασις εἰς ἡμέρας.
\VS{15}Καὶ ἐν τῷ λαλῆσαι αὐτὸν μετʼ ἐμοῦ κατὰ τοὺς λόγους τούτους, ἔδωκα τὸ πρόσωπόν μου ἐπὶ τὴν γῆν, καὶ κατενύγην.
\par }{\PP \VS{16}Καὶ ἰδοὺ ὡς ὁμοίωσις υἱοῦ ἀνθρώπου ἥψατο τῶν χειλέων μου· καὶ ἤνοιξα τὸ στόμα μου, καὶ ἐλάλησα, καὶ εἶπα πρὸς τὸν ἑστῶτα ἐναντίον μου, κύριε, ἐν τῇ ὀπτασίᾳ σου ἐστράφη τὰ ἐντός μου ἐν ἐμοὶ, καὶ οὐκ ἔσχον ἰσχύν.
\VS{17}Καὶ πῶς δυνήσεται ὁ παῖς σου κύριε λαλῆσαι μετὰ τοῦ κυρίου μου τούτου; καὶ ἐγὼ, ἀπὸ τοῦ νῦν οὐ στήσεται ἐν ἐμοὶ ἰσχὺς, καὶ πνεῦμα οὐχ ὑπελείφθη ἐν ἐμοί.
\VS{18}Καὶ προσέθετο, καὶ ἥψατό μου ὡς ὅρασις ἀνθρώπου, καὶ ἐνίσχυσέ με,
\VS{19}καὶ εἶπέ μοι, μὴ φοβοῦ ἀνὴρ ἐπιθυμιῶν, εἰρήνη σοι· ἀνδρίζου καὶ ἴσχυε· καὶ ἐν τῷ λαλῆσαι αὐτὸν μετʼ ἐμοῦ, ἴσχυσα, καὶ εἶπα λαλείτω ὁ κύριός μου, ὅτι ἐνίσχυσάς με.
\par }{\PP \VS{20}Καὶ εἶπεν, εἰ οἶδας, ἱνατί ἦλθον πρὸς σέ; καὶ νῦν ἐπιστρέψω τοῦ πολεμῆσαι μετὰ τοῦ ἄρχοντος Περσῶν· καὶ ἐγὼ εἰσεπορευόμην, καὶ ὁ ἄρχων τῶν Ἑλλήνων ἤρχετο.
\VS{21}Ἀλλʼ ἢ ἀναγγελῶ σοι τὸ ἐντεταγμένον ἐν γραφῇ ἀληθείας, καὶ οὐκ ἔστιν εἷς ἀντεχόμενος μετʼ ἐμοῦ περὶ τούτων, ἀλλʼ ἢ Μιχαὴλ ὁ ἄρχων ὑμῶν.

\par }\Chap{11}{\PP \VerseOne{1}Καὶ ἐγὼ ἐν ἔτει πρώτῳ Κύρου ἔστην εἰς κράτος καὶ ἰσχύν.
\par }{\PP \VS{2}Καὶ νῦν ἀλήθειαν ἀναγγελῶ σοι· ἰδοὺ ἔτι τρεῖς βασιλεῖς ἀναστήσονται ἐν τῇ Περσίδι, καὶ ὁ τέταρτος πλουτήσει πλοῦτον μέγαν παρὰ πάντας· καὶ μετὰ τὸ κρατῆσαι αὐτὸν τοῦ πλούτου αὐτοῦ, ἐπαναστήσεται πάσαις βασιλείαις Ἑλλήνων.
\par }{\PP \VS{3}Καὶ ἀναστήσεται βασιλεὺς δυνατὸς, καὶ κυριεύσει κυρείας πολλῆς, καὶ ποιήσει κατὰ τὸ θέλημα αὐτοῦ.
\par }{\PP \VS{4}Καὶ ὡς ἂν στῇ ἡ βασιλεία αὐτοῦ, συντριβήσεται, καὶ διαιρεθήσεται εἰς τοὺς τέσσαρας ἀνέμους τοῦ οὐρανοῦ, καὶ οὐκ εἰς τὰ ἔσχατα αὐτοῦ, οὐδὲ κατὰ τὴν κυρείαν αὐτοῦ, ἣν ἐκυρίευσεν, ὅτι ἐκτιλήσεται ἡ βασιλεία αὐτοῦ, καὶ ἑτέροις ἐκτὸς τούτων.
\par }{\PP \VS{5}Καὶ ἐνισχύσει ὁ βασιλεὺς τοῦ Νότου· καὶ εἷς τῶν ἀρχόντων αὐτῶν ἐνισχύσει ἐπʼ αὐτὸν, καὶ κυριεύσει κυρίαν πολλήν.
\VS{6}Καὶ μετὰ τὰ ἔτη αὐτοῦ συμμιγήσονται, καὶ θυγάτηρ βασιλέως τοῦ Νότου εἰσελεύσεται πρὸς βασιλέα τοῦ Βοῤῥᾶ, τοῦ ποιῆσαι συνθήκας μετʼ αὐτοῦ, καὶ οὐ κρατήσει ἰσχύος βραχίονος, καὶ οὐ στήσεται τὸ σπέρμα αὐτοῦ, καὶ παραδοθήσεται αὕτη, καὶ οἱ φέροντες αὐτὴν, καὶ ἡ νεάνις, καὶ ὁ κατισχύων αὐτὴν ἐν τοῖς καιροῖς.
\par }{\PP \VS{7}Ἀναστήσεται ἐκ τοῦ ἄνθους τῆς ῥίζης αὐτῆς, τῆς ἑτοιμασίας αὐτοῦ, καὶ ἥξει πρὸς τὴν δύναμιν, καὶ εἰσελεύσεται εἰς τὰ ὑποστηρίγματα τοῦ βασιλέως τοῦ Βοῤῥᾶ, καὶ ποιήσει ἐν αὐτοῖς, καὶ κατισχύσει.
\VS{8}Καί γε τοὺς θεοὺς αὐτῶν μετὰ τῶν χωνευτῶν αὐτῶν, πᾶν σκεῦος ἐπιθυμητὸν αὐτῶν, ἀργυρίου καὶ χρυσίου, μετὰ αἰχμαλωσίας οἴσει εἰς Αἴγυπτον, καὶ αὐτὸς στήσεται ὑπὲρ βασιλέα τοῦ Βοῤῥᾶ.
\VS{9}Καὶ εἰσελεύσεται εἰς τὴν βασιλείαν τοῦ βασιλέως τοῦ Νότου, καὶ ἀναστρέψει εἰς τὴν γῆν αὐτοῦ.
\par }{\PP \VS{10}Καὶ οἱ υἱοὶ αὐτοῦ συνάξουσιν ὄχλον ἀναμέσον πολλῶν· καὶ ἐλεύσεται ἐρχόμενος καὶ κατακλύζων, καὶ παρελεύσεται, καὶ καθίεται, καὶ συμπροσπλακήσεται ἕως τῆς ἰσχύος αὐτοῦ.
\VS{11}Καὶ ἀγριανθήσεται βασιλεὺς τοῦ Νότου, καὶ ἐξελεύσεται, καὶ πολεμήσει μετὰ τοῦ βασιλέως τοῦ Βοῤῥᾶ, καὶ στήσει ὄχλον πολύν, καὶ παραδοθήσεται ὁ ὄχλος ἐν χειρὶ αὐτοῦ,
\VS{12}καὶ λήψεται τὸν ὄχλον, καὶ ὑψωθήσεται ἡ καρδία αὐτοῦ, καὶ καταβαλεῖ μυριάδας, καὶ οὐ κατισχύσει.
\VS{13}Καὶ ἐπιστρέψει ὁ βασιλεὺς τοῦ Βοῤῥᾶ, καὶ ἄξει ὄχλον πολὺν ὑπὲρ τὸν πρότερον· καὶ εἰς τὸ τέλος τῶν καιρῶν ἐνιαυτῶν ἐπελεύσεται εἰσόδια ἐν δυνάμει μεγάλῃ, καὶ ἐν ὑπάρξει πολλῇ.
\par }{\PP \VS{14}Καὶ ἐν τοῖς καιροῖς ἐκείνοις πολλοὶ ἐπαναστήσονται ἐπὶ βασιλέα τοῦ Νότου, καὶ οἱ υἱοὶ τῶν λοιμῶν τοῦ λαοῦ σου ἐπαρθήσονται, τοῦ στῆσαι ὅρασιν, καὶ ἀσθενήσουσι.
\VS{15}Καὶ εἰσελεύσεται βασιλεὺς τοῦ Βοῤῥᾶ, καὶ ἐκχεεῖ πρόσχωμα, καὶ συλλήψεται πόλεις ὀχυρὰς, καὶ οἱ βραχίονες τοῦ βασιλέως τοῦ Νότου στήσονται, καὶ ἀναστήσονται οἱ ἐκλεκτοὶ αὐτοῦ, καὶ οὐκ ἔσται ἰσχὺς τοῦ στῆναι.
\VS{16}Καὶ ποιήσει ὁ εἰσπορευόμενος πρὸς αὐτὸν κατὰ τὸ θέλημα αὐτοῦ, καὶ οὐκ ἔστιν ἑστὼς κατὰ πρόσωπον αὐτοῦ· καὶ στήσεται ἐν τῇ γῇ τοῦ Σαβεὶ, καὶ τελεσθήσεται ἐν τῇ χειρὶ αὐτοῦ.
\par }{\PP \VS{17}Καὶ τάξει τὸ πρόσωπον αὐτοῦ εἰσελθεῖν ἐν ἰσχύϊ πάσης τῆς βασιλείας αὐτοῦ, καὶ εὐθεῖα πάντα μετʼ αὐτοῦ ποιήσει· καὶ θυγατέρα τῶν γυναικῶν δώσει αὐτῷ διαφθεῖραι αὐτὴν, καὶ οὐ μὴ παραμείνῃ, καὶ οὐκ αὐτῷ ἔσται.
\VS{18}Καὶ ἐπιστρέψει τὸ πρόσωπον αὐτοῦ εἰς τὰς νήσους, καὶ συλλήψεται πολλὰς, καὶ καταπαύσει ἄρχοντας ὀνειδισμοῦ αὐτῶν, πλὴν ὀνειδισμὸς αὐτοῦ ἐπιστρέψει αὐτῷ.
\VS{19}Καὶ ἐπιστρέψει τὸ πρόσωπον αὐτοῦ εἰς τὴν ἰσχὺν τῆς γῆς αὐτοῦ, καὶ ἀσθενήσει, καὶ πεσεῖται, καὶ οὐκ εὑρεθήσεται.
\par }{\PP \VS{20}Καὶ ἀναστήσεται ἐκ τῆς ῥίζης αὐτοῦ φυτὸν τῆς βασιλείας ἐπὶ τὴν ἑτοιμασίαν αὐτοῦ παραβιβάζων, πράσσων δόξαν βασιλείας· καὶ ἐν ταῖς ἡμέραις ἐκείναις ἔτι συντριβήσεται, καὶ οὐκ ἐν προσώποις, οὐδὲ ἐν πολέμῳ.
\par }{\PP \VS{21}Στήσεται ἐπὶ τὴν ἑτοιμασίαν αὐτοῦ, ἐξουδενώθη, καὶ οὐκ ἔδωκαν ἐπʼ αὐτὸν δόξαν βασιλείας· καὶ ἥξει ἐν εὐθηνίᾳ, καὶ κατισχύσει βασιλείας ἐν ὀλισθήμασι,
\VS{22}καὶ βραχίονες τοῦ κατακλύζοντος κατακλυσθήσονται ἀπὸ προσώπου αὐτοῦ, καὶ συντριβήσονται, καὶ ἡγούμενος διαθήκης·
\VS{23}Καὶ ἀπὸ τῶν συναναμίξεων πρὸς αὐτὸν ποιήσει δόλον, καὶ ἀναβήσεται, καὶ ὑπερισχύσει αὐτοὺς ἐν ὀλίγῳ ἔθνει.
\VS{24}Καὶ ἐν εὐθηνίᾳ, καὶ ἐν πίοσι χώραις ἥξει, καὶ ποιήσει ἃ οὐκ ἐποίησαν οἱ πατέρες αὐτοῦ καὶ τατέρες τῶν πατέρων αὐτοῦ· προνομὴν καὶ σκῦλα καὶ ὕπαρξιν αὐτοῖς διασκορπιεῖ, καὶ ἐπʼ Αἴγυπτον λογιεῖται λογισμοὺς καὶ ἕως καιροῦ.
\VS{25}Καὶ ἐξεγερθήσεται ἡ ἰσχὺς αὐτοῦ, καὶ ἡ καρδία αὐτοῦ ἐπὶ βασιλέα τοῦ Νότου ἐν δυνάμει μεγάλῃ· καὶ ὁ βασιλεὺς τοῦ Νότου συνάψει πόλεμον ἐν δυνάμει μεγάλῃ καὶ ἰσχυρᾷ σφόδρα, καὶ οὐ στήσονται, ὅτι λογιοῦνται ἐπʼ αὐτὸν λογισμοὺς,
\VS{26}καὶ φάγονται τὰ δέοντα αὐτοῦ, καὶ συντρίψουσιν αὐτὸν, καὶ δυνάμεις κατακλύσει, καὶ πεσοῦνται τραυματίαι πολλοί.
\par }{\PP \VS{27}Καὶ ἀμφότεροι οἱ βασιλεῖς, αἱ καρδίαι αὐτῶν εἰς πονηρίαν, καὶ ἐπὶ τραπέζῃ μιᾷ ψευδῆ λαλήσουσι, καὶ οὐ κατευθυνεῖ, ὅτι ἔτι πέρας εἰς καιρόν.
\VS{28}Καὶ ἐπιστρέψει εἰς τὴν γῆν αὐτοῦ ἐν ὑπάρξει πολλῇ, καὶ ἡ καρδία αὐτοῦ ἐπὶ διαθήκην ἁγίαν, καὶ ποιήσει, καὶ ἐπιστρέψει εἰς τὴν γῆν αὐτοῦ.
\par }{\PP \VS{29}Εἰς τὸν καιρὸν ἐπιστρέψει, καὶ ἥξει ἐν τῷ Νότῳ, καὶ οὐκ ἔσται ὡς ἡ πρώτη καὶ ἡ ἐσχάτη.
\VS{30}Καὶ εἰσελεύσονται ἐν αὐτῷ οἱ ἐκπορευόμενοι Κίτιοι, καὶ ταπεινωθήσεται, καὶ ἐπιστρέψει, καὶ θυμωθήσεται ἐπὶ διαθήκην ἁγίαν· καὶ ποιήσει, καὶ ἐπιστρέψει, καὶ συνήσει ἐπὶ τοὺς καταλιπόντας διαθήκην ἁγίαν.
\par }{\PP \VS{31}Καὶ σπέρματα ἐξ αὐτοῦ ἀναστήσονται, καὶ βεβηλώσουσι τὸ ἁγίασμα τῆς δυναστείας, καὶ μεταστήσουσι τὸν ἐνδελεχισμὸν, καὶ δώσουσι βδέλυγμα ἠφανισμένον.
\VS{32}Καὶ οἱ ἀνομοῦντες διαθήκην ἐπάξουσιν ἐν ὀλισθήμασι· καὶ λαὸς γινώσκοντες Θεὸν αὐτοῦ κατισχύσουσι, καὶ ποιήσουσι,
\VS{33}καὶ οἱ συνετοὶ τοῦ λαοῦ συνήσουσιν εἰς πολλὰ, καὶ ἀσθενήσουσιν ἐν ῥομφαίᾳ, καὶ ἐν φλογὶ, καὶ ἐν αἰχμαλωσίᾳ, καὶ ἐν διαρπαγῇ ἡμερῶν.
\VS{34}Καὶ ἐν τῷ ἀσθενῆσαι αὐτοὺς, βοηθήσονται βοήθειαν μικρὰν, καὶ προστεθήσονται πρὸς αὐτοὺς πολλοὶ ἐν ὀλισθήμασι.
\par }{\PP \VS{35}Καὶ ἀπὸ τῶν συνιέντων ἀσθενήσουσι, τοῦ πυρῶσαι αὐτοὺς, καὶ τοῦ ἐκλέξασθαι, καὶ τοῦ ἀποκαλυφθῆναι ἕως καιροῦ πέρας, ὅτι ἔτι εἰς καιρόν.
\par }{\PP \VS{36}Καὶ ποιήσει κατὰ τὸ θέλημα αὐτοῦ· καὶ ὁ βασιλεὺς ὑψωθήσεται καὶ μεγαλυνθήσεται ἐπὶ πάντα θεόν, καὶ λαλήσει ὑπέρογκα, καὶ κατευθυνεῖ μέχρις οὗ συντελεσθῇ ἡ ὀργὴ, εἰς γὰρ συντέλειαν γίνεται.
\VS{37}Καὶ ἐπὶ πάντος θεοὺς τῶν πατέρων αὐτοῦ οὐ συνήσει, καὶ ἐπιθυμία γυναικῶν, καὶ ἐπὶ πᾶν θεὸν οὐ συνήσει, ὅτι ἐπὶ πάντας μεγαλυνθήσεται.
\VS{38}Καὶ θεὸν Μαωζεὶμ ἐπὶ τόπου αὐτοῦ δοξάσει, καὶ θεὸν ὃν οὐκ ἔγνωσαν οἱ πατέρες αὐτοῦ, δοξάσει ἐν χρυσῷ καὶ ἀργύρῳ καὶ λίθῳ τιμίῳ, καὶ ἐν ἐπιθυμήμασι.
\VS{39}Καὶ ποιήσει τοῖς ὀχυρώμασι τῶν καταφυγῶν μετὰ θεοῦ ἀλλοτρίου, καὶ πληθυνεῖ δόξαν, καὶ ὑποτάξει αὐτοῖς πολλοὺς, καὶ γῆν διελεῖ ἐν δώροις.
\par }{\PP \VS{40}Καὶ ἐν καιροῦ πέρατι συγκερατισθήσεται μετὰ τοῦ βασιλέως τοῦ Νότου· καὶ συναχθήσεται ἐπʼ αὐτὸν βασιλεὺς τοῦ Βοῤῥᾶ ἐν ἅρμασι καὶ ἐν ἱππεῦσι καὶ ἐν ναυσὶ πολλαῖς, καὶ εἰσελεύσονται εἰς τὴν γῆν, καὶ συντρίψει, καὶ παρελεύσεται,
\VS{41}καὶ εἰσελεύσεται εἰς τὴν γῆν τοῦ Σαβαεὶμ, καὶ πολλοὶ ἀσθενήσουσι· καὶ οὗτοι διασωθήσονται ἐκ χειρὸς αὐτοῦ, Ἐδὼμ, καὶ Μωὰβ, καὶ ἀρχὴ υἱῶν Ἀμμών.
\VS{42}Καὶ ἐκτενεῖ τὴν χεῖρα ἐπὶ τὴν γῆν, καὶ γῆ Αἰγύπτου οὐκ ἔσται εἰς σωτηρίαν.
\VS{43}Καὶ κυριεύσει ἐν τοῖς ἀποκρύφοις τοῦ χρυσοῦ καὶ τοῦ ἀργύρου, καὶ ἐν πᾶσιν ἐπιθυμητοῖς Αἰγύπτου, καὶ Λιβύων, καὶ Αἰθιόπων, ἐν τοῖς ὀχυρώμασιν αὐτῶν.
\VS{44}Καὶ ἀκοαὶ καὶ σπουδαὶ ταράξουσιν αὐτὸν ἐξ ἀνατολῶν καὶ ἀπὸ Βοῤῥᾶ· καὶ ἥξει ἐν θυμῷ πολλῷ, τοῦ ἀφανίσαι πολλούς.
\VS{45}Καὶ πήξει τὴν σκηνὴν αὐτοῦ Ἐφαδανῶ, ἀναμέσον τῶν θαλασσῶν εἰς ὄρος Σαβαεὶν ἅγιον, ἥξει ἕως μέρους αὐτοῦ, καὶ οὐκ ἔστιν ὁ ῥυόμενος αὐτόν.

\par }\Chap{12}{\PP \VerseOne{1}Καὶ ἐν τῷ καιρῷ ἐκείνῳ ἀναστήσεται Μιχαὴλ ὁ ἄρχων ὁ μέγας, ὁ ἑστηκὼς ἐπὶ τοὺς υἱοὺς τοῦ λαοῦ σου· καὶ ἔσται καιρὸς θλίψεως, θλίψις οἵα οὐ γέγονεν ἀφʼ οὗ γεγένηται ἔθνος ἐν τῇ γῇ, ἕως τοῦ καιροῦ ἐκείνου· ἐν τῷ καιρῷ ἐκείνῳ σωθήσεται ὁ λαός σου πᾶς ὁ γεγραμμένος ἐν τῇ βίβλῳ.
\VS{2}Καὶ πολλοὶ τῶν καθευδόντων ἐν γῆς χώματι ἐξεγερθήσονται, οὗτοι εἰς ζωὴν αἰώνιον, καὶ οὗτοι εἰς ὀνειδισμὸν καὶ εἰς αἰσχύνην αἰώνιον.
\VS{3}Καὶ οἱ συνιέντες λάμψουσιν ὡς ἡ λαμπρότης τοῦ στερεώματος, καὶ ἀπὸ τῶν δικαίων τῶν πολλῶν ὡς οἱ ἀστέρες εἰς τοὺς αἰῶνας, καὶ ἔτι.
\par }{\PP \VS{4}Καὶ σὺ Δανιὴλ ἔμφραξον τοὺς λόγους, καὶ σφράγισον τὸ βιβλίον ἕως καιροῦ συντελείας, ἕως διδαχθῶσι πολλοὶ καὶ πληθυνθῇ ἡ γνῶσις.
\par }{\PP \VS{5}Καὶ ἴδον ἐγὼ Δανιὴλ, καὶ ἰδοὺ δύο ἕτεροι εἱστήκεισαν, εἷς ἐντεῦθεν τοῦ χειλους τοῦ ποταμοῦ, καὶ εἷς ἐντεῦθεν τοῦ χείλους τοῦ ποταμοῦ.
\VS{6}Καὶ εἶπε τῷ ἀνδρὶ τῷ ἐνδεδυμένῳ τὰ βαδδὶν, ὃς ἦν ἐπάνω τοῦ ὕδατος τοῦ ποταμοῦ, ἕως πότε τὸ πέρας ὧν εἴρηκας τῶν θαυμασίων;
\VS{7}Καὶ ἤκουσα τοῦ ἀνδρὸς τοῦ ἐνδεδυμένον τὰ βαδδὶν, ὃς ἦν ἐπάνω τοῦ ὕδατος τοῦ ποταμοῦ· καὶ ὕψωσε τὴν δεξιὰν αὐτοῦ καὶ τὴν ἀριστερὰν αὐτοῦ εἰς τὸν οὐρανὸν, καὶ ὤμοσεν ἐν τῷ ζῶντι εἰς τὸν αἰῶνα, ὅτι εἰς καιρὸν καιρῶν καὶ ἥμισυ καιροῦ, ἐν τῷ συντελεσθῆναι διασκορπισμὸν γνώσονται πάντα ταῦτα.
\par }{\PP \VS{8}Καὶ ἐγὼ ἤκουσα, καὶ οὐ συνῆκα· καὶ εἶπα, Κύριε, τί τὰ ἔσχατα τούτων;
\VS{9}Καὶ εἶπε, δεῦρο Δανιὴλ, ὅτι ἐμπεφραγμένοι καὶ ἐσφραγισμένοι οἱ λόγοι ἕως καιροῦ πέρας.
\VS{10}Ἐκλεγῶσι, καὶ ἐκλευκανθῶσι, καὶ πυρωθῶσι, καὶ ἁγιασθῶσι πολλό· καὶ ἀνομήσωσιν ἄνομοι, καὶ οὐ συνήσουσι πάντες ἄνομοι, καὶ οἱ νοήμονες συνήσουσι.
\VS{11}Καὶ ἀπὸ καιροῦ παραλλάξεως τοῦ ἐνδελεχισμοῦ, καὶ δοθήσεται τὸ βδέλυγμα ἐρημώσεως, ἡμέραι χίλιαι διακόσιαι ἐννενήκοντα.
\VS{12}Μακάριος ὁ ὑπομένων καὶ φθάσας εἰς ἡμέρας χιλίας τριακοσίας τριακονταπέντε.
\VS{13}Καὶ σὺ δεῦρο, καὶ ἀναπαύου· ἔτι γὰρ ἡμέραι καὶ ὧραι εἰς ἀναπλήρωσιν συντελείας, καὶ ἀναστήσῃ εἰς τὸν κλῆρόν σου εἰς συντέλειαν ἡμερῶν.
\par }