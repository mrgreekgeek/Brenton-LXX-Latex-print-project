\NormalFont\ShortTitle{ΓΕΝΕΣΙΣ}
{\MT ΓΕΝΕΣΙΣ

\par }\ChapOne{1}{\PP \VerseOne{1}ἘΝ ἀρχῇ ἐποίησεν ὁ Θεὸς τὸν οὐρανὸν καὶ τὴν γῆν.
\VS{2}Ἡ δὲ γῆ ἦν ἀόρατος καὶ ἀκατασκεύαστος, καὶ σκότος ἐπάνω τῆς ἀβύσσου· καὶ πνεῦμα Θεοῦ ἐπεφέρετο ἐπάνω τοῦ ὕδατος.
\VS{3}Καὶ εἶπεν ὁ Θεὸς, γενηθήτω φῶς· καὶ ἐγένετο φῶς.
\VS{4}Καὶ εἶδεν ὁ Θεὸς τὸ φῶς, ὅτι καλόν· καὶ διεχώρισεν ὁ Θεὸς ἀνὰ μέσον τοῦ φωτὸς, καὶ ἀνὰ μέσον τοῦ σκότους.
\VS{5}Καὶ ἐκάλεσεν ὁ Θεὸς τὸ φῶς ἡμέραν, καὶ τὸ σκότος ἐκάλεσε νύκτα. Καὶ ἐγένετο ἑσπέρα, καὶ ἐγένετο πρωῒ, ἡμέρα μία.
\par }{\PP \VS{6}Καὶ εἶπεν ὁ Θεὸς, γενηθήτω στερέωμα ἐν μέσῳ τοῦ ὕδατος· καὶ ἔστω διαχωρίζον ἀνὰ μέσον ὕδατος καὶ ὕδατος· καὶ ἐγένετο οὕτως.
\VS{7}Καὶ ἐποίησεν ὁ Θεὸς τὸ στερέωμα· καὶ διεχώρισεν ὁ Θεὸς ἀνὰ μέσον τοῦ ὕδατος, ὃ ἦν ὑποκάτω τοῦ στερεώματος, καὶ ἀνὰ μέσον τοῦ ὕδατος, τοῦ ἐπάνω τοῦ στερεώματος.
\VS{8}Καὶ ἐκάλεσεν ὁ Θεὸς τὸ στερέωμα οὐρανόν· καὶ εἶδεν ὁ Θεὸς ὅτι καλόν· καὶ ἐγένετο ἑσπέρα, καὶ ἐγένετο πρωῒ, ἡμέρα δευτέρα.
\par }{\PP \VS{9}Καὶ εἶπεν ὁ Θεὸς, συναχθήτω τὸ ὕδωρ τὸ ὑποκάτω τοῦ οὐρανοῦ εἰς συναγωγὴν μίαν, καὶ ὀφθήτω ἡ ξηρά· καὶ ἐγένετο οὕτως· καὶ συνήχθη τὸ ὕδωρ τὸ ὑποκάτω τοῦ οὐρανοῦ εἰς τὰς συναγωγὰς αὐτῶν, καὶ ὤφθη ἡ ξηρά.
\VS{10}Καὶ ἐκάλεσεν ὁ Θεὸς τὴν ξηρὰν, γῆν· καὶ τὰ συστήματα τῶν ὑδάτων ἐκάλεσε θαλάσσας· καὶ εἶδεν ὁ Θεὸς ὅτι καλόν.
\VS{11}Καὶ εἶπεν ὁ Θεὸς, βλαστησάτω ἡ γῆ βοτάνην χόρτου, σπεῖρον σπέρμα κατὰ γένος καὶ καθʼ ὁμοιότητα, καὶ ξύλον κάρπιμον ποιοῦν καρπὸν, οὗ τὸ σπέρμα αὐτοῦ ἐν αὐτῷ κατὰ γένος ἐπὶ τῆς γῆς· καὶ ἐγένετο οὕτως.
\VS{12}Καὶ ἐξήνεγκεν ἡ γῆ βοτάνην χόρτου, σπεῖρον σπέρμα κατὰ γένος καὶ καθʼ ὁμοιότητα, καὶ ξύλον κάρπιμον ποιοῦν καρπὸν, οὗ τὸ σπέρμα αὐτοῦ ἐν αὐτῷ κατὰ γένος ἐπὶ τῆς γῆς· καὶ εἶδεν ὁ Θεὸς ὅτι καλόν.
\VS{13}Καὶ ἐγένετο ἑσπέρα, καὶ ἐγένετο πρωῒ, ἡμέρα τρίτη.
\par }{\PP \VS{14}Καὶ εἶπεν ὁ Θεὸς, γενηθήτωσαν φωστῆρες ἐν τῷ στερεώματι τοῦ οὐρανοῦ εἰς φαῦσιν ἐπὶ τῆς γῆς, τοῦ διαχωρίζειν ἀνὰ μέσον τῆς ἡμέρας καὶ ἀνὰ μέσον τῆς νυκτός· καὶ ἔστωσαν εἰς σημεῖα, καὶ εἰς καιροὺς, καὶ εἰς ἡμέρας, καὶ εἰς ἐνιαυτούς.
\VS{15}Καὶ ἔστωσαν εἰς φαῦσιν ἐν τῷ στερεώματι τοῦ οὐρανοῦ, ὥστε φαίνειν ἐπὶ τῆς γῆς· καὶ ἐγένετο οὕτως.
\VS{16}Καὶ ἐποίησεν ὁ Θεὸς τοὺς δύο φωστῆρας τοὺς μεγάλους· τὸν φωστῆρα τὸν μέγαν εἰς ἀρχὰς τῆς ἡμέρας, καὶ τὸν φωστῆρα τὸν ἐλάσσω εἰς ἀρχὰς τῆς νυκτὸς, καὶ τοὺς ἀστέρας.
\VS{17}Καὶ ἔθετο αὐτοὺς ὁ Θεὸς ἐν τῷ στερεώματι τοῦ οὐρανοῦ, ὥστε φαίνειν ἐπὶ τῆς γῆς,
\VS{18}καὶ ἄρχειν τῆς ἡμέρας καὶ τῆς νυκτὸς, καὶ διαχωρίζειν ἀνὰ μέσον τοῦ φωτὸς, καὶ ἀνὰ μέσον τοῦ σκότους· καὶ εἶδεν ὁ Θεὸς ὅτι καλόν.
\VS{19}Καὶ ἐγένετο ἑσπέρα καὶ ἐγένετο πρωῒ, ἡμέρα τετάρτη.
\par }{\PP \VS{20}Καὶ εἶπεν ὁ Θεὸς, ἐξαγαγέτω τὰ ὕδατα ἑρπετὰ ψυχῶν ζωσῶν, καὶ πετεινὰ πετόμενα ἐπὶ τῆς γῆς κατὰ τὸ στερέωμα τοῦ οὐρανοῦ· καὶ ἐγένετο οὕτως.
\VS{21}Καὶ ἐποίησεν ὁ Θεὸς τὰ κήτη τὰ μεγάλα, καὶ πᾶσαν ψυχὴν ζώων ἑρπετῶν, ἃ ἐξήγαγε τὰ ὕδατα κατὰ γένη αὐτῶν, καὶ πᾶν πετεινὸν πτερωτὸν κατὰ γένος· καὶ εἶδεν ὁ Θεὸς ὅτι καλά.
\VS{22}Καὶ εὐλόγησεν αὐτὰ ὁ Θεὸς, λέγων, αὐξάνεσθε καὶ πληθύνεσθε, καὶ πληρώσατε τὰ ὕδατα ἐν ταῖς θαλάσσαις, καὶ τὰ πετεινὰ πληθυνέσθωσαν ἐπὶ τῆς γῆς.
\VS{23}Καὶ ἐγένετο ἑσπέρα, καὶ ἐγένετο πρωῒ, ἡμέρα πέμπτη.
\par }{\PP \VS{24}Καὶ εἶπεν ὁ Θεὸς, ἐξαγαγέτω ἡ γῆ ψυχὴν ζῶσαν κατὰ γένος, τετράποδα, καὶ ἑρπετὰ, καὶ θηρία τῆς γῆς κατὰ γένος· καὶ ἐγένετο οὕτως.
\VS{25}Καὶ ἐποίησεν ὁ Θεὸς τὰ θηρία τῆς γῆς κατὰ γένος, καὶ τὰ κτήνη κατὰ γένος αὐτῶν, καὶ πάντα τὰ ἑρπετὰ τῆς γῆς κατὰ γένος· καὶ εἶδεν ὁ Θεὸς ὅτι καλά.
\par }{\PP \VS{26}Καὶ εἶπεν ὁ Θεός, Ποιήσωμεν ἄνθρωπον κατʼ εἰκόνα ἡμετέραν καὶ καθʼ ὁμοίωσιν· καὶ ἀρχέτωσαν τῶν ἰχθύων τῆς θαλάσσης, καὶ τῶν πετεινῶν τοῦ οὐρανοῦ, καὶ τῶν κτηνῶν, καὶ πάσης τῆς γῆς, καὶ πάντων τῶν ἑρπετῶν τῶν ἑρπόντων ἐπὶ τῆς γῆς.
\VS{27}Καὶ ἐποιήσεν ὁ Θεὸς τὸν ἄνθρωπον· κατʼ εἰκόνα Θεοῦ ἐποίησεν αὐτόν· ἄρσεν καὶ θῆλυ ἐποίησεν αὐτούς.
\VS{28}Καὶ εὐλόγησεν αὐτοὺς ὁ Θεὸς, λέγων, αὐξάνεσθε καὶ πληθύνεσθε, καὶ πληρώσατε τὴν γῆν, καὶ κατακυριεύσατε αὐτῆς· καὶ ἄρχετε τῶν ἰχθύων τῆς θαλάσσης, καὶ τῶν πετεινῶν τοῦ οὐρανοῦ, καὶ πάντων τῶν κτηνῶν, καὶ πάσης τῆς γῆς, καὶ πάντων τῶν ἑρπετῶν τῶν ἑρπόντων ἐπὶ τῆς γῆς.
\VS{29}Καὶ εἶπεν ὁ Θεός, Ἰδοὺ δέδωκα ὑμῖν πάντα χόρτον σπόριμον σπεῖρον σπέρμα, ὅ ἐστιν ἐπάνω πάσης τῆς γῆς· καὶ πᾶν ξύλον, ὃ ἔχει ἐν ἑαυτῷ καρπὸν σπέρματος σπορίμου, ὑμῖν ἔσται εἰς βρῶσιν,
\VS{30}καὶ πᾶσι τοῖς θηρίοις τῆς γῆς, καὶ πᾶσι τοῖς πετεινοῖς τοῦ οὐρανοῦ, καὶ παντὶ ἑρπετῷ ἕρποντι ἐπὶ τῆς γῆς, ὃ ἔχει ἐν ἑαυτῷ ψυχὴν ζωῆς, καὶ πάντα χόρτον χλωρὸν εἰς βρῶσιν· καὶ ἐγένετο οὕτως.
\VS{31}Καὶ εἶδεν ὁ Θεὸς τὰ πάντα, ὅσα ἐποίησε, καὶ ἰδοὺ καλὰ λίαν· καὶ ἐγένετο ἑσπέρα, καὶ ἐγένετο πρωῒ, ἡμέρα ἕκτη.

\Chap{2}\VerseOne{1}Καὶ συνετελέσθησαν ὁ οὐρανὸς καὶ ἡ γῆ, καὶ πᾶς ὁ κόσμος αὐτῶν.
\par }{\PP \VS{2}Καὶ συνετέλεσεν ὁ Θεὸς ἐν τῇ ἡμέρᾳ τῇ ἕκτῃ τὰ ἔργα αὐτοῦ, ἃ ἐποίησε· καὶ κατέπαυσε τῇ ἡμέρᾳ τῇ ἑβδόμῃ ἀπὸ πάντων τῶν ἔργων αὐτοῦ, ὧν ἐποίησε.
\VS{3}Καὶ εὐλόγησεν ὁ Θεὸς τὴν ἡμέραν τὴν ἑβδόμην, καὶ ἡγίασεν αὐτήν, ὅτι ἐν αὐτῇ κατέπαυσεν ἀπὸ πάντων τῶν ἔργων αὐτοῦ, ὧν ἤρξατο ὁ Θεὸς ποιῆσαι.
\par }{\PP \VS{4}Αὕτη ἡ βίβλος γενέσεως οὐρανοῦ καὶ γῆς, ὅτε ἐγένετο, ᾗ ἡμέρᾳ ἐποίησε Κύριος ὁ Θεὸς τὸν οὐρανὸν καὶ τὴν γῆν,
\VS{5}καὶ πᾶν χλωρὸν ἀγροῦ πρὸ τοῦ γενέσθαι ἐπὶ τῆς γῆς, καὶ πάντα χόρτον ἀγροῦ πρὸ τοῦ ἀνατεῖλαι· οὐ γὰρ ἔβρεξεν ὁ Θεὸς ἐπὶ τὴν γῆν, καὶ ἄνθρωπος οὐκ ἦν ἐργάζεσθαι αὐτήν.
\VS{6}Πηγὴ δὲ ἀνέβαινεν ἐκ τῆς γῆς, καὶ ἐπότιζε πᾶν τὸ πρόσωπον τῆς γῆς.
\VS{7}Καὶ ἔπλασεν ὁ Θεὸς τὸν ἄνθρωπον, χοῦν ἀπὸ τῆς γῆς· καὶ ἐνεφύσησεν εἰς τὸ πρόσωπον αὐτοῦ πνοὴν ζωῆς, καὶ ἐγένετο ὁ ἄνθρωπος εἰς ψυχὴν ζῶσαν.
\par }{\PP \VS{8}Καὶ ἐφύτευσεν ὁ Θεὸς παράδεισον ἐν Ἐδὲμ κατὰ ἀνατολάς· καὶ ἔθετο ἐκεῖ τὸν ἄνθρωπον, ὃν ἔπλασε.
\VS{9}Καὶ ἐξανέτειλεν ὁ Θεὸς ἔτι ἐκ τῆς γῆς πᾶν ξύλον ὡραῖον εἰς ὅρασιν, καὶ καλὸν εἰς βρῶσιν, καὶ τὸ ξύλον τῆς ζωῆς ἐν μέσῳ τοῦ παραδείσου, καὶ τὸ ξύλον τοῦ εἰδέναι γνωστὸν καλοῦ καὶ πονηροῦ.
\VS{10}Ποταμὸς δὲ ἐκπορεύεται ἐξ Ἐδὲμ ποτίζειν τὸν παράδεισον· ἐκεῖθεν ἀφορίζεται εἰς τέσσαρας ἀρχάς.
\VS{11}Ὄνομα τῷ ἑνὶ, Φισῶν· οὗτος ὁ κυκλῶν πᾶσαν τὴν γῆν Εὐιλάτ· ἐκεῖ οὗ ἐστι τὸ χρυσίον.
\VS{12}Τὸ δὲ χρυσίον τῆς γῆς ἐκείνης καλόν· καὶ ἐκεῖ ἐστιν ὁ ἄνθραξ, καὶ ὁ λίθος ὁ πράσινος.
\VS{13}Καὶ ὄνομα τῷ ποταμῷ τῷ δευτέρῳ, Γεῶν· οὗτος ὁ κυκλῶν πᾶσαν τὴν γὴν Αἰθιοπίας.
\VS{14}Καὶ ὁ ποταμὸς ὁ τρίτος, Τίγρις· οὗτος ὁ προπορευόμενος κατέναντι Ἀσσυρίων· ὁ δὲ ποταμὸς ὁ τέταρτος, Εὐφράτης.
\VS{15}Καὶ ἔλαβε Κύριος ὁ Θεὸς τὸν ἄνθρωπον ὃν ἔπλασε, καὶ ἔθετο αὐτὸν ἐν τῷ παραδείσῳ τῆς τρυφῆς, ἐργάζεσθαι αὐτὸν καὶ φυλάσσειν.
\VS{16}Καὶ ἐνετείλατο Κύριος ὁ Θεὸς τῷ Ἀδὰμ, λέγων, ἀπὸ παντὸς ξύλου τοῦ ἐν τῷ παραδείσῳ βρώσει φαγῇ.
\VS{17}Ἀπὸ δὲ τοῦ ξύλου τοῦ γινώσκειν καλὸν καὶ πονηρὸν, οὐ φάγεσθε ἀπʼ αὐτοῦ· ᾗ δʼ ἂν ἡμέρᾳ φάγητε ἀπʼ αὐτοῦ, θανάτῳ ἀποθανεῖσθε.
\par }{\PP \VS{18}Καὶ εἶπε Κύριος ὁ Θεὸς, οὐ καλὸν εἶναι τὸν ἄνθρωπον μόνον· ποιήσωμεν αὐτῷ βοηθὸν κατʼ αὐτόν.
\VS{19}Καὶ ἔπλασεν ὁ Θεὸς ἔτι ἐκ τῆς γῆς πάντα τὰ θηρία τοῦ ἀγροῦ, καὶ πάντα τὰ πετεινὰ τοῦ οὐρανοῦ· καὶ ἤγαγεν αὐτὰ πρὸς τὸν Ἀδὰμ, ἰδεῖν τί καλέσει αὐτά· καὶ πᾶν ὃ ἐὰν ἐκάλεσεν αὐτὸ Ἀδὰμ ψυχὴν ζῶσαν, τοῦτο ὄνομα αὐτῷ.
\VS{20}Καὶ ἐκάλεσεν Ἀδὰμ ὀνόματα πᾶσι τοῖς κτήνεσι, καὶ πᾶσι τοῖς πετεινοῖς τοῦ οὐρανοῦ, καὶ πᾶσι τοῖς θηρίοις τοῦ ἀγροῦ· τῷ δὲ Ἀδὰμ οὐχ εὑρέθη βοηθὸς ὅμοιος αὐτῷ.
\VS{21}Καὶ ἐπέβαλεν ὁ Θεὸς ἔκστασιν ἐπὶ τὸν Ἀδὰμ, καὶ ὕπνωσε· καὶ ἔλαβε μίαν τῶν πλευρῶν αὐτοῦ, καὶ ἀνεπλήρωσε σάρκα ἀντʼ αὐτῆς.
\VS{22}Καὶ ᾠκοδόμησεν ὁ Θεὸς τὴν πλευρὰν, ἣν ἔλαβεν ἀπὸ τοῦ Ἀδὰμ εἰς γυναῖκα· καὶ ἤγαγεν αὐτὴν πρὸς τὸν Ἀδάμ.
\VS{23}Καὶ εἶπεν Ἀδάμ· τοῦτο νῦν ὀστοῦν ἐκ τῶν ὀστέων μου, καὶ σὰρξ ἐκ τῆς σαρκός μου· αὕτη κληθήσεται γυνὴ, ὅτι ἐκ τοῦ ἀνδρὸς αὐτῆς ἐλήφθη.
\VS{24}Ἕνεκεν τούτου καταλείψει ἄνθρωπος τὸν πατέρα αὐτοῦ καὶ τὴν μητέρα, καὶ προσκολληθήσεται πρὸς τὴν γυναῖκα αὐτοῦ· καὶ ἔσονται οἱ δύο εἰς σάρκα μίαν.
\VS{25}Καὶ ἦσαν οἱ δύο γυμνοὶ, ὅ, τε Ἀδὰμ καὶ ἡ γυνὴ αὐτοῦ, καὶ οὐκ ᾐσχύνοντο.

\par }\Chap{3}{\PP \VerseOne{1}Ὁ δὲ ὄφις ἦν φρονιμώτατος πάντων τῶν θηρίων τῶν ἐπὶ τῆς γῆς, ὧν ἐποίησε Κύριος ὁ Θεός· καὶ εἶπεν ὁ ὄφις τῇ γυναικὶ, τί ὅτι εἶπεν ὁ Θεός, οὐ μὴ φάγητε ἀπὸ παντὸς ξύλου τοῦ παραδείσου;
\VS{2}Καὶ εἶπεν ἡ γυνὴ τῷ ὄφει, ἀπὸ καρποῦ τοῦ ξύλου τοῦ παραδείσου φαγούμεθα·
\VS{3}Ἀπὸ δὲ τοῦ καρποῦ τοῦ ξύλου, ὅ ἐστιν ἐν μέσῳ τοῦ παραδείσου, εἶπεν ὁ Θεός, οὐ φάγεσθε ἀπʼ αὐτοῦ, οὐδὲ μὴ ἅψησθε αὐτοῦ, ἵνα μὴ ἀποθάνητε.
\VS{4}Καὶ εἶπεν ὁ ὄφις τῇ γυναικί· οὐ θανάτῳ ἀποθανεῖσθε·
\VS{5}Ἤδει γὰρ ὁ Θεὸς, ὅτι ᾗ ἂν ἡμέρᾳ φάγητε ἀπʼ αὐτοῦ, διανοιχθήσονται ὑμῶν οἱ ὀφθαλμοί, καὶ ἔσεσθε ὡς θεοί, γινώσκοντες καλὸν καὶ πονηρόν.
\VS{6}Καὶ εἶδεν ἡ γυνὴ, ὅτι καλὸν τὸ ξύλον εἰς βρῶσιν, καὶ ὅτι ἀρεστὸν τοῖς ὀφθαλμοῖς ἰδεῖν, καὶ ὡραῖόν ἐστι τοῦ κατανοῆσαι· καὶ λαβοῦσα ἀπὸ τοῦ καρποῦ αὐτοῦ, ἔφαγε· καὶ ἔδωκε καὶ τῷ ἀνδρὶ αὐτῆς μετʼ αὐτῆς, καὶ ἔφαγον.
\VS{7}Καὶ διηνοίχθησαν οἱ ὀφθαλμοὶ τῶν δύο, καὶ ἔγνωσαν ὅτι γυμνοὶ ἦσαν· καὶ ἔῤῥαψαν φύλλα συκῆς, καὶ ἐποίησαν ἑαυτοῖς περιζώματα.
\VS{8}Καὶ ἤκουσαν τὴς φωνὴς Κυρίου τοῦ Θεοῦ περιπατοῦντος ἐν τῷ παραδείσῳ τὸ δειλινόν· καὶ ἐκρύβησαν ὅ, τε Ἀδὰμ καὶ ἡ γυνὴ αὐτοῦ ἀπὸ προσώπου Κυρίου τοῦ Θεοῦ ἐν μέσῳ τοῦ ξύλου τοῦ παραδείσου.
\VS{9}Καὶ ἐκάλεσεν Κύριος ὁ Θεὸς τὸν Ἀδὰμ, καὶ εἶπεν αὐτῷ· Ἀδὰμ ποῦ εἶ;
\VS{10}Καὶ εἶπεν αὐτῷ· τὴς φωνῆς σου ἤκουσα περιπατοῦντος ἐν τῷ παραδείσῳ, καὶ ἐφοβήθην ὅτι γυμνός εἰμι, καὶ ἐκρύβην.
\VS{11}Καὶ εἶπεν αὐτῷ ὁ Θεὸς, τὶς ἀνήγγειλέ σοι ὅτι γυμνὸς εἶ, εἰ μὴ ἀπὸ τοῦ ξύλου, οὗ ἐνετειλάμην σοι τούτου μόνου μὴ φαγεῖν, ἀπʼ αὐτοῦ ἔφαγες;
\VS{12}Καὶ εἶπεν ὁ Ἀδάμ· ἡ γυνή, ἣν ἔδωκας μετʼ ἐμοῦ, αὕτη μοι ἔδωκεν ἀπὸ τοῦ ξύλου, καὶ ἔφαγον.
\VS{13}Καὶ εἶπε Κύριος ὁ Θεὸς τῇ γυναικί· τί τοῦτο ἐποιήσας; καὶ εἶπεν ἡ γυνὴ, ὁ ὄφις ἠπάτησέ με, καὶ ἔφαγον.
\par }{\PP \VS{14}Καὶ εἶπε Κύριος ὁ Θεὸς τῷ ὄφει· ὅτι ἐποίησας τοῦτο, ἐπικατάρατος σὺ ἀπὸ πάντων τῶν κτηνῶν, καὶ ἀπὸ πάντων τῶν θηρίων τῶν ἐπὶ τῆς γῆς· ἐπὶ τῷ στήθει σου καὶ τῇ κοιλίᾳ πορεύσῃ, καὶ γῆν φαγῃ πάσας τὰς ἡμέρας τῆς ζωῆς σου.
\VS{15}Καὶ ἔχθραν θήσω ἀνὰ μέσον σοῦ καὶ ἀνὰ μέσον τῆς γυναικὸς, καὶ ἀνὰ μέσον τοῦ σπέρματός σου, καὶ ἀνὰ μέσον τοῦ σπέρματος αὐτῆς· αὐτός σοῦ τηρήσει κεφαλὴν, καὶ σὺ τηρήσεις αὐτοῦ πτέρναν.
\VS{16}Καὶ τῇ γυναικὶ εἶπε· πληθύνων πληθυνῶ τὰς λύπας σου, καὶ τὸν στεναγμόν σου· ἐν λύπαις τέξῃ τέκνα, καὶ πρὸς τὸν ἄνδρα σου ἡ ἀποστροφή σου· καὶ αὐτός σου κυριεύσει.
\VS{17}Τῷ δὲ Ἀδὰμ εἶπεν· ὅτι ἤκουσας τῆς φωνῆς τῆς γυναικός σου, καὶ ἔφαγες ἀπὸ τοῦ ξύλου, οὗ ἐνετειλάμην σοι τούτου μόνου μὴ φαγεῖν, ἀπʼ αὐτοῦ ἔφαγες, ἐπικατάρατος ἡ γῆ ἐν τοῖς ἔργοις σου· ἐν λύπαις φάγῃ αὐτὴν πάσας τὰς ἡμέρας τῆς ζωῆς σου.
\VS{18}Ἀκάνθας καὶ τριβόλους ἀνατελεῖ σοι, καὶ φαγῇ τὸν χόρτον τοῦ ἀγροῦ.
\VS{19}Ἐν ἱδρῶτι τοῦ προσώπου σου φαγῃ τὸν ἄρτον σου, ἕως τοῦ ἀποστρέψαι σε εἰς τὴν γῆν ἐξ ἧς ἐλήμφθης· ὅτι γῆ εἶ, καὶ εἰς γῆν ἀπελεύσῃ.
\VS{20}Καὶ ἐκάλεσεν Ἀδὰμ τὸ ὄνομα τῆς γυναικὸς αὐτοῦ Ζωή, ὅτι μήτηρ πάντων τῶν ζώντων.
\VS{21}Καὶ ἐποίησε Κύριος ὁ Θεὸς τῷ Ἀδὰμ, καὶ τῇ γυναικὶ αὐτοῦ χιτῶνας δερματίνους, καὶ ἐνέδυσεν αὐτούς.
\par }{\PP \VS{22}Καὶ εἶπεν ὁ Θεός, ἰδοὺ Ἀδὰμ γέγονεν ὡς εἷς ἐξ ἡμῶν, τοῦ γινώσκειν καλὸν καὶ πονηρόν· καὶ νῦν μή ποτε ἐκτείνῃ τὴν χεῖρα αὐτοῦ, καὶ λάβῃ τοῦ ξύλου τῆς ζωῆς καὶ φάγῃ, καὶ ζήσεται εἰς τὸν αἰῶνα.
\VS{23}Καὶ ἐξαπέστειλεν αὐτὸν Κύριος ὁ Θεὸς ἐκ τοῦ παραδείσου τῆς τρυφῆς, ἐργάζεσθαι τὴν γῆν ἐξ ἧς ἐλήμφθη.
\VS{24}Καὶ ἐξέβαλεν τὸν Ἀδὰμ, καὶ κατῴκισεν αὐτὸν ἀπέναντι τοῦ παραδείσου τῆς τρυφῆς· καὶ ἔταξε τὰ χερουβὶμ· καὶ τὴν φλογίνην ῥομφαίαν τὴν στρεφομένην, φυλάσσειν τὴν ὁδὸν τοῦ ξύλου τῆς ζωῆς.

\par }\Chap{4}{\PP \VerseOne{1}Ἀδὰμ δὲ ἔγνω Εὔαν τὴν γυναῖκα αὐτοῦ, καὶ συλλαβοῦσα ἔτεκε τὸν Κάϊν· καὶ εἶπεν, ἐκτησάμην ἄνθρωπον διὰ τοῦ Θεοῦ.
\VS{2}Καὶ προσέθηκε τεκεῖν τὸν ἀδελφὸν αὐτοῦ τὸν Ἄβελ· καὶ ἐγένετο Ἄβελ ποιμὴν προβάτων, Κάϊν δὲ ἦν ἐργαζόμενος τὴν γῆν.
\VS{3}Καὶ ἐγένετο μεθʼ ἡμέρας ἤνεγκε Κάϊν ἀπὸ τῶν καρπῶν τῆς γῆς θυσίαν τῷ Κυρίῳ·
\VS{4}Καὶ Ἄβελ ἤνεγκε καὶ αὐτὸς ἀπὸ τῶν πρωτοτόκων τῶν προβάτων αὐτοῦ, καὶ ἀπὸ τῶν στεάτων αὐτῶν· καὶ ἐπεῖδεν ὁ Θεὸς ἐπὶ Ἄβελ, καὶ ἐπὶ τοῖς δώροις αὐτοῦ.
\VS{5}Ἐπὶ δὲ Κάϊν, καὶ ἐπὶ ταῖς θυσίαις αὐτοῦ, οὐ προσέσχε· καὶ ἐλυπήθη Κάϊν λίαν, καὶ συνέπεσε τῷ προσώπῳ αὐτοῦ.
\VS{6}Καὶ εἶπε Κύριος ὁ Θεὸς τῷ Κάϊν, ἵνα τί περίλυπος ἐγένου, καὶ ἵνα τί συνέπεσε τὸ πρόσωπόν σου;
\VS{7}Οὐκ ἐὰν ὀρθῶς προσενέγκῃς, ὀρθῶς δὲ μὴ διέλῃς, ἥμαρτες; ἡσυχασον· πρός σὲ ἡ ἀποστροφὴ αὐτοῦ, καὶ σὺ ἄρξεις αὐτοῦ.
\par }{\PP \VS{8}Καὶ εἶπεν Κάϊν πρὸς Ἄβελ τὸν ἀδελφὸν αὐτοῦ, διέλθωμεν εἰς τὸ πεδίον· καὶ ἐγένετο ἐν τῷ εἶναι αὐτοὺς ἐν τῷ πεδίῳ, ἀνέστη Κάϊν ἐπὶ Ἄβελ τὸν ἀδελφὸν αὐτοῦ, καὶ ἀπέκτεινεν αὐτόν.
\VS{9}Καὶ εἶπε Κύπιος ὁ Θεὸς πρὸς Κάϊν· ποῦ ἔστιν Ἄβελ ὁ ἀδελφός σου; καὶ εἶπεν, οὐ γινώσκω· μὴ φύλαξ τοῦ ἀδελφοῦ μου εἰμὶ ἐγώ;
\VS{10}Καὶ εἶπε Κύριος, τί πεποίηκας; φωνὴ αἵματος τοῦ ἀδελφοῦ σου βοᾷ πρός με ἐκ τῆς γῆς.
\VS{11}Καὶ νῦν ἐπικατάρατος σὺ ἀπὸ τῆς γῆς, ἣ ἔχανε τὸ στόμα αὐτῆς δέξασθαι τὸ αἷμα τοῦ ἀδελφοῦ σου ἐκ τῆς χειρός σου.
\VS{12}Ὅτε ἐργᾷ τὴν γῆν, καὶ οὐ προσθήσει τὴν ἰσχὺν αὐτῆς δοῦναί σοι· στένων καὶ τρέμων ἐσῃ ἐπὶ τῆς γῆς.
\VS{13}Καὶ εἶπε Κάϊν πρὸς Κύριον τὸν Θεὸν, μείζων ἡ αἰτία μου τοῦ ἀφεθῆναί με.
\VS{14}Εἰ ἐκβάλλεις με σήμερον ἀπὸ προσώπου τῆς γῆς, καὶ ἀπὸ τοῦ προσώπου σου κρυβήσομαι, καὶ ἔσομαι στένων καὶ τρέμων ἐπὶ τῆς γῆς, καὶ ἔσται πᾶς ὁ εὑρίσκων με, ἀποκτενεῖ με.
\VS{15}Καὶ εἴπεν αὐτῷ Κύριος ὁ Θεὸς, οὐχ οὕτω· πᾶς ὁ ἀποκτείνας Κάϊν, ἑπτὰ ἐκδικούμενα παραλύσει. Καὶ ἔθετο Κύριος ὁ Θεὸς σημεῖον τῷ Κάϊν, τοῦ μὴ ἀνελεῖν αὐτὸν πάντα τὸν εὑρίσκοντα αὐτόν.
\VS{16}Ἐξῆλθεν δὲ Κάϊν ἀπὸ προσώπου τοῦ Θεοῦ, καὶ ᾤκησεν ἐν γῇ Ναὶδ κατέναντι Ἐδέμ.
\par }{\PP \VS{17}Καὶ ἔγνω Κάϊν τὴν γυναῖκα αὐτοῦ· καὶ συλλαβοῦσα ἔτεκε τὸν Ἐνώχ. Καὶ ἦν οἰκοδομῶν πόλιν· καὶ ἐπῳνόμασε τὴν πόλιν ἐπὶ τῷ ὀνόματι τοῦ υἱοῦ αὐτοῦ, Ἐνώχ.
\VS{18}Ἐγενήθη δὲ τῷ Ἐνὼχ Γαϊδάδ· καὶ Γαϊδὰδ ἐγέννησε τὸν Μαλελεὴλ· καὶ Μαλελεὴλ ἐγέννησε τὸν Μαθουσάλα· καὶ Μαθουσάλα ἐγέννησε τὸν Λάμεχ.
\par }{\PP \VS{19}Καὶ ἔλαβεν ἑαυτῷ Λάμεχ δύο γυναῖκας· ὄνομα τῇ μιᾷ, Ἀδά· καὶ ὄνομα τῇ δευτέρᾳ, Σελλά.
\VS{20}Καὶ ἔτεκεν Ἀδὰ τὸν Ἰωβήλ· οὗτος ἦν πατὴρ οἰκούντων ἐν σκηναῖς κτηνοτρόφων.
\VS{21}Καὶ ὄνομα τῷ ἀδελφῷ αὐτοῦ, Ἰουβάλ· οὗτος ἦν ὁ καταδείξας ψαλτήριον καὶ κιθάραν.
\VS{22}Σελλὰ δὲ καὶ αὐτὴ ἔτεκε τὸν Θόβελ· καὶ ἦν σφυροκόπος χαλκεὺς χαλκοῦ καὶ σιδήρου. ἀδελφὴ δὲ Θόβελ, Νοεμά.
\VS{23}Εἶπε δὲ Λάμεχ ταῖς ἑαυτοῦ γυναιξίν, Ἀδὰ καὶ Σελλὰ, ἀκούσατέ μου τῆς φωνῆς, γυναῖκες Λάμεχ, ἐνωτίσασθέ μου τοὺς λόγους· ὅτι ἄνδρα ἀπέκτεινα εἰς τραῦμα ἐμοὶ, καὶ νεανίσκον εἰς μώλωπα ἐμοί.
\VS{24}Ὅτι ἑπτάκις ἐκδεδίκηται ἐκ Κάϊν· ἐκ δὲ Λάμεχ, ἑβδομηκοντάκις ἑπτά.
\par }{\PP \VS{25}Ἔγνω δὲ Ἀδὰμ Εὔαν τὴν γυναῖκα αὐτοῦ· καὶ συλλαβοῦσα ἔτεκεν υἱόν· καὶ ἐπωνόμασε τὸ ὄνομα αὐτοῦ Σὴθ, λέγουσα, ἐξανέστησε γάρ μοι ὁ Θεὸς σπέρμα ἕτερον ἀντὶ Ἄβελ, ὃν ἀπέκτεινε Κάϊν.
\VS{26}Καὶ τῷ Σὴθ ἐγένετο υἱός· ἐπωνόμασε δὲ τὸ ὄνομα αὐτοῦ, Ἑνώς· οὗτος ἤλπισεν ἐπικαλεῖσθαι τὸ ὄνομα Κυρίου τοῦ Θεοῦ.

\par }\Chap{5}{\PP \VerseOne{1}Αὕτη ἡ βίβλος γενέσεως ἀνθρώπων· ᾗ ἡμέρᾳ ἐποίησεν ὁ Θεὸς τὸν Ἀδὰμ, κατʼ εἰκόνα Θεοῦ ἐποίησεν αὐτόν·
\VS{2}Ἄρσεν καὶ θῆλυ ἐποίησεν αὐτούς· καὶ εὐλόγησεν αὐτούς· καὶ ἐπωνόμασε τὸ ὄνομα αὐτοῦ Ἀδὰμ, ᾗ ἡμέρᾳ ἐποίησεν αὐτούς.
\VS{3}Ἔζησεν δὲ Ἀδὰμ τριάκοντα καὶ διακόσια ἔτη, καὶ ἐγέννησε κατὰ τὴν ἰδέαν αὐτοῦ, καὶ κατὰ τὴν εἰκόνα αὐτοῦ, καὶ ἐπωνόμασε τὸ ὄνομα αὐτοῦ, Σήθ.
\VS{4}Ἐγένοντο δὲ αἱ ἡμέραι Ἀδὰμ, ἃς ἔζησε μετὰ τὸ γεννῆσαι αὐτὸν τὸν Σὴθ, ἔτη ἑπτακόσια· καὶ ἐγέννησεν υἱοὺς καὶ θυγατέρας.
\VS{5}Καὶ ἐγένοντο πᾶσαι αἱ ἡμέραι Ἀδὰμ, ἃς ἔζησε, τριάκοντα καὶ ἐννακόσια ἔτη· καὶ ἀπέθανεν.
\VS{6}Ἔζησε δὲ Σὴθ πέντε καὶ διακόσια ἔτη· καὶ ἐγέννησε τὸν Ἐνώς.
\VS{7}Καὶ ἔζησε Σὴθ μετὰ τὸ γεννῆσαι αὐτὸν τὸν Ἐνὼς, ἑπτὰ ἔτη καὶ ἑπτακόσια· καὶ ἐγέννησεν υἱοὺς καὶ θυγατέρας.
\VS{8}Καὶ ἐγένοντο πᾶσαι αἱ ἡμέραι Σὴθ δώδεκα καὶ ἐννακόσια ἔτη· καὶ ἀπέθανε.
\VS{9}Καὶ ἔζησεν Ἐνὼς ἔτη ἑκατὸν ἐννεήκοντα· καὶ ἐγέννησε τὸν Καϊνᾶν.
\VS{10}Καὶ ἔζησεν Ἐνὼς μετὰ τὸ γεννῆσαι αὐτὸν τὸν Καϊνᾶν, πεντεκαίδεκα ἔτη καὶ ἑπτκόσια· καὶ ἐγέννησεν υἱοὺς καὶ θυγατέρας.
\VS{11}Καὶ ἐγένοντο πᾶσαι αἱ ἡμέραι Ἐνὼς πέντε ἔτη καὶ ἐννακόσια· καὶ ἀπέθανε.
\VS{12}Καὶ ἔζησεν Καϊνᾶν ἑβδομήκοντα καὶ ἑκατὸν ἔτη· καὶ ἐγέννησε τὸν Μαλελεήλ.
\VS{13}Καὶ ἔζησε Καϊνᾶν μετὰ τὸ γεννῆσαι αὐτὸν τὸν Μαλελεὴλ, τεσσεράκοντα καὶ ἑπτακόσια ἔτη· καὶ ἐγέννησεν υἱοὺς καὶ θυγατέρας.
\VS{14}Καὶ ἐγένοντο πᾶσαι αἱ ἡμέραι Καϊνᾶν δέκα ἔτη καὶ ἐννακόσια· καὶ ἀπέθανε.
\par }{\PP \VS{15}Καὶ ἔζησε Μαλελεὴλ πέντε καὶ ἑξήκοντα καὶ ἑκατὸν ἔτη· καὶ ἐγέννησε τὸν Ἰάρεδ.
\VS{16}Καὶ ἔζησε Μαλελεὴλ μετὰ τὸ γεννῆσαι αὐτὸν τὸν Ἰάρεδ, ἔτη τριάκοντα καὶ ἑπτακόσια· καὶ ἐγέννησεν υἱοὺς καὶ θυγατέρας.
\VS{17}Καὶ ἐγένοντο πᾶσαι αἱ ἡμέραι Μαλελεὴλ, ἔτη πέντε καὶ ἐννενήκοντα καὶ ὀκτακόσια· καὶ ἀπέθανε.
\VS{18}Καὶ ἔζησεν Ἰάρεδ δύο καὶ ἑξήκοντα ἔτη καὶ ἑκατὸν· καὶ ἐγέννησε τὸν Ἐνώχ.
\VS{19}Καὶ ἔζησεν Ἰάρεδ μετὰ τὸ γεννῆσαι αὐτὸν τὸν Ἐνὼχ, ὀκτακόσια ἔτη· καὶ ἐγέννησεν υἱοὺς καὶ θυγατέρας.
\VS{20}Καὶ ἐγένοντο πᾶσαι αἱ ἡμέραι Ἰάρεδ, δύο καὶ ἑξήκοντα καὶ ἐννακόσια ἔτη· καὶ ἀπέθανε.
\VS{21}Καὶ ἔζησεν Ἐνὼχ πέντε καὶ ἑξήκοντα καὶ ἑκατὸν ἔτη· καὶ ἐγέννησε τὸν Μαθουσάλα.
\VS{22}Εὐηρέστησε δὲ Ἐνὼχ τῷ Θεῷ μετὰ τὸ γεννῆσαι αὐτὸν τὸν Μαθουσάλα, διακόσια ἔτη· καὶ ἐγέννησεν υἱοὺς καὶ θυγατέρας.
\VS{23}Καὶ ἐγένοντο πᾶσαι αἱ ἡμέραι Ἐνὼχ, πέντε καὶ ἑξήκοντα καὶ τριακόσια ἔτη.
\VS{24}Καὶ εὐηρέστησεν Ἐνὼχ τῷ Θεῷ· καὶ οὐχ εὑρίσκετο, ὅτι μετέθηκεν αὐτὸν ὁ Θεός.
\VS{25}Καὶ ἔζησε Μαθουσάλα ἑπτὰ ἔτη καὶ ἑξήκοντα καὶ ἑκατόν· καὶ ἐγέννησε τὸν Λάμεχ.
\VS{26}Καὶ ἔζησε Μαθουσάλα μετὰ τὸ γεννῆσαι αὐτὸν τὸν Λάμεχ, δύο καὶ ὀκτακόσια ἔτη· καὶ ἐγέννησεν υἱοὺς καὶ θυγατέρας.
\VS{27}Καὶ ἐγένοντο πᾶσαι αἱ ἡμέραι Μαθουσάλα ἃς ἔζησεν, ἐννέα καὶ ἑξήκοντα καὶ ἐννακόσια ἔτη· καὶ ἀπέθανε.
\VS{28}Καὶ ἔζησε Λάμεχ ὀκτὼ καὶ ὀγδοήκοντα καὶ ἑκατὸν ἔτη· καὶ ἐγέννησεν υἱόν.
\VS{29}Καὶ ἐπωνόμασε τὸ ὄνομα αὐτοῦ Νῶε, λέγων, οὗτος διαναπαύσει ἡμᾶς ἀπὸ τῶν ἔργων ἡμῶν, καὶ ἀπὸ τῶν λυπῶν τῶν χειρῶν ἡμῶν, καὶ ἀπὸ τῆς γῆς, ἧς κατηράσατο Κύριος ὁ Θεός.
\VS{30}Καὶ ἔζησε Λάμεχ μετὰ τὸ γεννῆσαι αὐτὸν τὸν Νῶε, πεντακόσια καὶ ἑξήκοντα καὶ πέντε ἔτη· καὶ ἐγέννησεν υἱοὺς καὶ θυγατέρας.
\VS{31}Καὶ ἐγένοντο πᾶσαι αἱ ἡμέραι Λάμεχ, ἑπτακόσια καὶ πεντήκοντα τρία ἔτη· καὶ ἀπέθανε.
\VS{32}Καὶ ἦν Νῶε ἐτῶν πεντακοσίων· καὶ ἐγέννησε τρεῖς υἱοὺς, τὸν Σὴμ, τὸν Χὰμ, τὸν Ἰάφεθ.

\par }\Chap{6}{\PP \VerseOne{1}Καὶ ἐγένετο ἡνίκα ἤρξαντο οἱ ἄνθρωποι πολλοὶ γίνεσθαι ἐπὶ τῆς γῆς, καὶ θυγατέρες ἐγεννήθησαν αὐτοῖς.
\VS{2}Ἰδόντες δὲ υἱοὶ τοῦ Θεοῦ τὰς θυγατέρας τῶν ἀνθρώπων, ὅτι καλαί εἰσιν, ἔλαβον ἑαυτοῖς γυναῖκας ἀπὸ πασῶν, ὧν ἐξελέξαντο.
\VS{3}Καὶ εἶπε Κύριος ὁ Θεὸς, οὐ μὴ καταμείνῃ τὸ πνεῦμά μου ἐν τοῖς ἀνθρώποις τούτοις εἰς τὸν αἰῶνα, διὰ τὸ εἶναι αὐτοὺς σάρκας· ἔσονται δὲ αἱ ἡμέραι αὐτῶν, ἑκατὸν εἴκοσιν ἔτη.
\VS{4}Οἱ δὲ γίγαντες ἦσαν ἐπὶ τῆς γῆς ἐν ταῖς ἡμέραις ἐκείναις, καὶ μετʼ ἐκεῖνο, ὡς ἂν εἰσεπορεύοντο οἱ υἱοὶ τοῦ Θεοῦ πρὸς τὰς θυγατέρας τῶν ἀνθρώπων, καὶ ἐγεννῶσαν αὐτοῖς· ἐκεῖνοι ἦσαν οἱ γίγαντες οἱ ἀπʼ αἰῶνος, οἱ ἄνθρωποι οἱ ὀνομαστοί.
\par }{\PP \VS{5}Ἰδὼν δὲ Κύριος ὁ Θεὸς, ὅτι ἐπληθύνθησαν αἱ κακίαι τῶν ἀνθρώπων ἐπὶ τῆς γῆς, καὶ πᾶς τις διανοεῖται ἐν τῇ καρδίᾳ αὐτοῦ ἐπιμελῶς ἐπὶ τὰ πονηρὰ πάσας τὰς ἡμέρας·
\VS{6}Καὶ ἐνεθυμήθη ὁ Θεὸς, ὅτι ἐποίησε τὸν ἄνθρωπον ἐπὶ τῆς γῆς, καὶ διενοήθη.
\VS{7}Καὶ εἶπεν ὁ Θεὸς, ἀπαλείψω τὸν ἄνθρωπον, ὃν ἐποίησα, ἀπὸ προσώπου τῆς γῆς, ἀπὸ ἀνθρώπου ἕως κτήνους, καὶ ἀπὸ ἑρπετῶν ἕως πετεινῶν τοῦ οὐρανοῦ· ὅτι ἐνεθυμήθην, ὅτι ἐποίησα αὐτούς.
\par }{\PP \VS{8}Νῶε δὲ εὗρε χάριν ἐναντίον Κυρίου τοῦ Θεοῦ.
\VS{9}Αὗται δὲ αἱ γενέσεις Νῶε. Νῶε ἄνθρωπος δίκαιος, τέλειος ὢν ἐν τῇ γενεᾷ αὐτοῦ, τῷ Θεῷ εὐηρέστησε Νῶε.
\VS{10}Ἐγέννησε δὲ Νῶε τρεῖς υἱοὺς, τὸν Σὴμ, τὸν Χὰμ, τὸν Ἰάφεθ.
\VS{11}Ἐφθάρη δὲ ἡ γῆ ἐναντίον τοῦ Θεοῦ, καὶ ἐπλήσθη ἡ γῆ ἀδικίας.
\VS{12}Καὶ εἶδε Κύριος ὁ Θεὸς τὴν γῆν, καὶ ἦν κατεφθαρμένη· ὅτι κατέφθειρε πᾶσα σὰρξ τὴν ὁδὸν αὐτοῦ ἐπὶ τῆς γῆς.
\VS{13}Καὶ εἶπε Κύριος ὁ Θεὸς τῷ Νῶε, καιρὸς παντὸς ἀνθρώπου ἥκει ἐναντίον μου, ὅτι ἐπλήσθη ἡ γῆ ἀδικίας ἀπʼ αὐτῶν· καὶ ἰδοὺ ἐγὼ καταφθείρω αὐτοὺς καὶ τὴν γῆν.
\par }{\PP \VS{14}Ποίησον οὖν σεαυτῷ κιβωτὸν ἐκ ξύλων τετραγώνων· νοσσιὰς ποιήσεις τὴν κιβωτόν· καὶ ἀσφαλτώσεις αὐτὴν ἔσωθεν καὶ ἔξωθεν τῇ ἀσφάλτῳ.
\VS{15}Καὶ οὕτω ποιήσεις τὴν κιβωτόν· τριακοσίων πήχεων τὸ μῆκος τῆς κιβωτοῦ, καὶ πεντήκοντα πήχεων τὸ πλάτος, καὶ τριάκοντα πήχεων τὸ ὕψος αὐτῆς.
\VS{16}Ἐπισυνάγων ποιήσεις τὴν κιβωτὸν, καὶ εἰς πῆχυν συντελέσεις αὐτὴν ἄνωθεν· τὴν δὲ θύραν τῆς κιβωτοῦ ποιήσεις ἐκ πλαγίων· κατάγαια διώροφα καὶ τριώροφα ποιήσεις αὐτήν.
\VS{17}Ἐγὼ δὲ ἰδοὺ ἐπάγω τὸν κατακλυσμὸν, ὕδωρ ἐπὶ τὴν γῆν, καταφθεῖραι πᾶσαν σάρκα, ἐν ᾗ ἐστι πνεῦμα ζωῆς ὑποκάτω τοῦ οὐρανοῦ· καὶ ὅσα ἂν ᾖ ἐπὶ τῆς γῆς, τελευτήσει.
\par }{\PP \VS{18}Καὶ στήσω τὴν διαθήκην μου μετά σου· εἰσελεύσῃ δὲ εἰς τὴν κιβωτὸν σὺ, καὶ οἱ υἱοί σου, καὶ ἡ γυνή σου, καὶ αἱ γυναῖκες τῶν υἱῶν σου μετά σου.
\VS{19}Καὶ ἀπὸ πάντων τῶν κτηνῶν, καὶ ἀπὸ πάντων τῶν ἑρπετῶν, καὶ ἀπὸ πάντων τῶν θηρίων, καὶ ἀπὸ πάσης σαρκὸς δύο δύο ἀπὸ πάντων εἰσάξεις εἰς τὴν κιβωτὸν, ἵνα τρέφῃς μετὰ σεαυτοῦ· ἄρσεν καὶ θῆλυ ἔσονται.
\VS{20}Ἀπὸ πάντων τῶν ὀρνέων τῶν πετεινῶν κατὰ γένος, καὶ ἀπὸ πάντων τῶν κτηνῶν κατὰ γένος, καὶ ἀπὸ πάντων τῶν ἑρπετῶν τῶν ἑρπόντων ἐπὶ τῆς γῆς κατὰ γένος αὐτῶν, δύο δύο ἀπὸ πάντων εἰσελεύσονται πρὸς σὲ τρέφεσθαι μετά σου, ἄρσεν καὶ θῆλυ.
\VS{21}Σὺ δὲ λήψῃ σεαυτῷ ἀπὸ πάντων τῶν βρωμάτων ἃ ἔδεσθε, καὶ συνάξεις πρὸς σεαυτὸν, καὶ ἔσται σοι καὶ ἐκείνοις φαγεῖν.
\VS{22}Καὶ ἐποίησε Νῶε πάντα ὅσα ἐνετείλατο αὐτῷ Κύριος ὁ Θεὸς, οὕτως ἐποίησε.

\par }\Chap{7}{\PP \VerseOne{1}Καὶ εἶπε Κύριος ὁ Θεὸς πρὸς Νῶε, εἴσελθε σὺ καὶ πᾶς ὁ οἶκός σου εἰς τὴν κιβωτὸν, ὅτι σὲ εἶδον δίκαιον ἐναντίον μου ἐν τῇ γενεᾷ ταύτῃ.
\VS{2}Ἀπὸ δὲ τῶν κτηνῶν τῶν καθαρῶν εἰσάγαγε πρὸς σὲ ἑπτὰ ἑπτὰ ἄρσεν καὶ θῆλυ, ἀπὸ δὲ τῶν κτηνῶν τῶν μὴ καθαρῶν δύο δύο ἄρσεν καὶ θῆλυ.
\VS{3}Καὶ ἀπὸ τῶν πετεινῶν τοῦ οὐρανοῦ τῶν καθαρῶν ἑπτὰ ἑπτὰ ἄρσεν καὶ θῆλυ, καὶ ἀπὸ πάντων τῶν πετεινῶν τῶν μὴ καθαρῶν δύο δύο ἄρσεν καὶ θῆλυ, διαθρέψαι σπέρμα ἐπὶ πᾶσαν τὴν γῆν.
\VS{4}Ἔτι γὰρ ἡμερῶν ἑπτὰ ἐγὼ ἐπάγω ὑετὸν ἐπὶ τὴν γῆν, τεσσαράκοντα ἡμέρας καὶ τεσσαράκοντα νύκτας· καὶ ἐξαλείψω πᾶν τὸ ἀνάστημα, ὃ ἐποίησα ἀπὸ προσώπου πάσης τῆς γῆς.
\VS{5}Καὶ ἐποίησε Νῶε πάντα, ὅσα ἐνετείλατο αὐτῷ Κύριος ὁ Θεός.
\VS{6}Νῶε δὲ ἦν ἐτῶν ἑξακοσίων, καὶ ὁ κατακλυσμὸς τοῦ ὕδατος ἐγένετο ἐπὶ τῆς γῆς.
\VS{7}Εἰσῆλθε δὲ Νῶε καὶ οἱ υἱοὶ αὐτοῦ, καὶ ἡ γυνὴ αὐτοῦ, καὶ αἱ γυναῖκες τῶν υἱῶν αὐτοῦ μετʼ αὐτοῦ εἰς τὴν κιβωτὸν, διὰ τὸ ὕδωρ τοῦ κατακλυσμοῦ.
\VS{8}Καὶ ἀπὸ τῶν πετεινῶν τῶν καθαρῶν, καὶ ἀπὸ τῶν πετεινῶν τῶν μὴ καθαρῶν, καὶ ἀπὸ τῶν κτηνῶν τῶν καθαρῶν, καὶ ἀπὸ τῶν κτηνῶν τῶν μὴ καθαρῶν, καὶ ἀπὸ πάντων τῶν ἑρπόντων ἐπὶ τῆς γῆς,
\VS{9}δύο δύο εἰσῆλθον πρὸς Νῶε εἰς τὴν κιβωτὸν ἄρσεν καὶ θῆλυ, καθὰ ἐνετείλατο ὁ Θεὸς τῷ Νῶε.
\VS{10}Καὶ ἐγένετο μετὰ τὰς ἑπτὰ ἡμέρας, καὶ τὸ ὕδωρ τοῦ κατακλυσμοῦ ἐγένετο ἐπὶ τῆς γῆς.
\VS{11}Ἐν τῷ ἑξακοσιοστῷ ἔτει ἐν τῇ ζωῇ τοῦ Νῶε, τοῦ δευτέρου μηνὸς, ἑβδόμῃ καὶ εἰκάδι τοῦ μηνὸς, τῇ ἡμέρᾳ ταύτῃ ἐῤῥάγησαν πᾶσαι αἱ πηγαὶ τῆς ἀβύσσου, καὶ οἱ καταῤῥάκται τοῦ οὐρανοῦ ἠνεῴχθησαν.
\VS{12}Καὶ ἐγένετο ὁ ὑετὸς ἐπὶ τῆς γῆς τεσσαράκοντα ἡμέρας καὶ τεσσαράκοντα νύκτας.
\VS{13}Ἐν τῇ ἡμέρᾳ ταύτῃ εἰσῆλθε Νῶε, Σὴμ, Χὰμ, Ἰάφεθ, οἱ υἱοὶ Νῶε, καὶ ἡ γυνὴ Νῶε, καὶ αἱ τρεῖς γυναῖκες τῶν υἱῶν αὐτοῦ μετʼ αὐτοῦ, εἰς τὴν κιβωτόν.
\VS{14}Καὶ πάντα τὰ θηρία κατὰ γένος, καὶ πάντα τὰ κτήνη κατὰ γένος, καὶ πᾶν ἑρπετὸν κινούμενον ἐπὶ τῆς γῆς κατὰ γένος, καὶ πᾶν ὄρνεον πετεινὸν κατὰ γένος αὐτοῦ,
\VS{15}εἰσῆλθον πρὸς Νῶε εἰς τὴν κιβωτὸν, δύο δύο ἄρσεν καὶ θῆλυ ἀπὸ πάσης σαρκὸς, ἐν ᾧ ἐστι πνεῦμα ζωῆς.
\VS{16}Καὶ τὰ εἰσπορευόμενα ἄρσεν καὶ θῆλυ ἀπὸ πάσης σαρκὸς εἰσῆλθε, καθὰ ἐνετείλατο ὁ Θεὸς τῷ Νῶε· καὶ ἔκλεισε Κύριος ὁ Θεὸς τὴν κιβωτὸν ἔξωθεν αὐτοῦ.
\par }{\PP \VS{17}Καὶ ἐγένετο ὁ κατακλυσμὸς τεσσαράκοντα ἡμέρας καὶ τεσσαράκοντα νύκτας ἐπὶ τῆς γῆς· καὶ ἐπεπληθύνθη τὸ ὕδωρ· καὶ ἐπῇρε τὴν κιβωτὸν, καὶ ὑψώθη ἀπὸ τῆς γῆς.
\VS{18}Καὶ ἐπεκράτει τὸ ὕδωρ, καὶ ἐπληθύνετο σφόδρα ἐπὶ τῆς γῆς· καὶ ἐπεφέρετο ἡ κιβωτὸς ἐπάνω τοῦ ὕδατος.
\VS{19}Τὸ δὲ ὕδωρ ἐπεκράτει σφόδρα σφόδρα ἐπὶ τῆς γῆς· καὶ ἐκάλυψε πάντα τὰ ὄρη τὰ ὑψηλὰ, ἃ ἦν ὑποκάτω τοῦ οὐρανοῦ.
\VS{20}Πεντεκαίδεκα πήχεις ὑπεράνω ὑψώθη τὸ ὕδωρ· καὶ ἐπεκάλυψε πάντα τὰ ὄρη τὰ ὑψηλά.
\VS{21}Καὶ ἀπέθανε πᾶσα σὰρξ κινουμένη ἐπὶ τῆς γῆς τῶν πετεινῶν, καὶ τῶν κτηνῶν, καὶ τῶν θηρίων· καὶ πᾶν ἑρπετὸν κινούμενον ἐπὶ τῆς γῆς, καὶ πᾶς ἄνθρωπος.
\VS{22}Καὶ πάντα ὅσα ἔχει πνοὴν ζωῆς, καὶ πᾶν ὃ ἦν ἐπὶ τῆς ξηρᾶς, ἀπέθανε.
\VS{23}Καὶ ἐξήλειψε πᾶν τὸ ἀνάστημα, ὃ ἦν ἐπὶ προσώπου τῆς γῆς, ἀπὸ ἀνθρώπου ἕως κτήνους, καὶ ἑρπετῶν, καὶ τῶν πετεινῶν τοῦ οὐρανοῦ· καὶ ἐξηλείφησαν ἀπὸ τῆς γῆς· καὶ κατελείφθη μόνος Νῶε, καὶ οἱ μετʼ αὐτοῦ ἐν τῇ κιβωτῷ.
\VS{24}Καὶ ὑψώθη τὸ ὕδωρ ἐπὶ τῆς γῆς ἡμέρας ἑκατὸν πεντήκοντα.

\par }\Chap{8}{\PP \VerseOne{1}Καὶ ἀνεμνήσθη ὁ Θεὸς τοῦ Νῶε, καὶ πάντων τῶν θηρίων, καὶ πάντων τῶν κτηνῶν, καὶ πάντων τῶν πετεινῶν, καὶ πάντων τῶν ἑρπετῶν τῶν ἑρπόντων, ὅσα ἦν μετʼ αὐτοῦ ἐν τῇ κιβωτῷ· καὶ ἐπήγαγεν ὁ Θεὸς πνεῦμα ἐπὶ τὴν γῆν, καὶ ἐκόπασε τὸ ὕδωρ.
\VS{2}Καὶ ἐπεκαλύφθησαν αἱ πηγαὶ τῆς ἀβύσσου, καὶ οἱ καταῤῥάκται τοῦ οὐρανοῦ, καὶ συνεσχέθη ὁ ὑετὸς ἀπὸ τοῦ οὐρανοῦ.
\VS{3}Καὶ ἐνεδίδου τὸ ὕδωρ πορευόμενον ἀπὸ τῆς γῆς· καὶ ἠλαττονοῦτο τὸ ὕδωρ μετὰ πεντήκοντα καὶ ἑκατὸν ἡμέρας.
\VS{4}Καὶ ἐκάθισεν ἡ κιβωτὸς ἐν μηνὶ τῷ ἑβδόμῳ, ἑβδόμῃ καὶ εἰκάδι τοῦ μηνὸς, ἐπὶ τὰ ὄρη τὰ Ἀραράτ.
\VS{5}Τὸ δὲ ὕδωρ ἠλαττονοῦτο ἕως τοῦ δεκάτου μηνός. Καὶ ἐν τῷ δεκάτῳ μηνὶ, τῇ πρώτῃ τοῦ μηνὸς, ὤφθησαν αἱ κεφαλαὶ τῶν ὀρέων.
\VS{6}Καὶ ἐγένετο μετὰ τεσσαράκοντα ἡμέρας ἠνέῳξε Νῶε τὴν θυρίδα τῆς κιβωτοῦ, ἣν ἐποίησε.
\VS{7}Καὶ ἀπέστειλε τὸν κόρακα· καὶ ἐξελθὼν, οὐκ ἀνέστρεψεν ἕως τοῦ ξηρανθῆναι τὸ ὕδωρ ἀπὸ τῆς γῆς.
\VS{8}Καὶ ἀπέστειλε τὴν περιστερὰν ὀπίσω αὐτοῦ, ἰδεῖν εἰ κεκόπακε τὸ ὕδωρ ἀπὸ τῆς γῆς.
\VS{9}Καὶ οὐχ εὑροῦσα ἡ περιστερὰ ἀνάπαυσιν τοῖς ποσὶν αὐτῆς, ἀνέστρεψε πρὸς αὐτὸν εἰς τὴν κιβωτὸν, ὅτι ὕδωρ ἦν ἐπὶ πᾶν τὸ πρόσωπον τῆς γῆς· καὶ ἐκτείνας τὴν χεῖρα ἔλαβεν αὐτὴν, καὶ εἰσήγαγεν αὐτὴν πρὸς ἑαυτὸν εἰς τὴν κιβωτόν.
\VS{10}Καὶ ἐπισχὼν ἔτι ἡμέρας ἑπτὰ ἑτέρας, πάλιν ἐξαπέστειλε τὴν περιστερὰν ἐκ τῆς κιβωτοῦ.
\VS{11}Καὶ ἀνέστρεψε πρὸς αὐτὸν ἡ περιστερὰ τὸ πρὸς ἑσπέραν· καὶ εἶχε φύλλον ἐλαίας κάρφος ἐν τῷ στόματι αὐτῆς· καὶ ἔγνω Νῶε, ὅτι κεκόπακε τὸ ὕδωρ ἀπὸ τῆς γῆς.
\VS{12}Καὶ ἐπισχὼν ἔτι ἡμέρας ἑπτὰ ἑτέρας, πάλιν ἐξαπέστειλε τὴν περιστερὰν, καὶ οὐ προσέθετο τοῦ ἐπιστρέψαι πρὸς αὐτὸν ἔτι.
\VS{13}Καὶ ἐγένετο ἐν τῷ ἑνὶ καὶ ἑξακοσιοστῷ ἔτει ἐν τῇ ζωῇ τοῦ Νῶε, τοῦ πρώτου μηνὸς, μιᾷ τοῦ μηνὸς, ἐξέλιπε τὸ ὕδωρ ἀπὸ τῆς γῆς. Καὶ ἀπεκάλυψε Νῶε τὴν στέγην τῆς κιβωτοῦ, ἣν ἐποίησε· καὶ εἶδεν ὅτι ἐξέλιπε τὸ ὕδωρ ἀπὸ προσώπου τῆς γῆς.
\VS{14}Ἐν δὲ τῷ δευτέρῳ μηνὶ ἐξηράνθη ἡ γῆ, ἑβδόμῃ καὶ εἰκάδι τοῦ μηνός.
\par }{\PP \VS{15}Καὶ εἶπε Κύριος ὁ Θεὸς πρὸς Νῶε, λέγων,
\VS{16}Ἔξελθε ἐκ τῆς κιβωτοῦ σὺ, καὶ ἡ γυνή σου, καὶ οἱ υἱοί σου, καὶ αἱ γυναῖκες τῶν υἱῶν σου μετὰ σοῦ,
\VS{17}Καὶ πάντα τὰ θηρία ὅσα ἐστὶ μετὰ σοῦ, καὶ πᾶσα σὰρξ ἀπὸ πετεινῶν ἕως κτηνῶν, καὶ πᾶν ἑρπετὸν κινούμενον ἐπὶ τῆς γῆς, ἐξάγαγε μετὰ σεαυτοῦ. καὶ αὐξάνεσθε καὶ πληθύνεσθε ἐπὶ τῆς γῆς.
\VS{18}Καὶ ἐξῆλθε Νῶε, καὶ ἡ γυνὴ αὐτοῦ, καὶ οἱ υἱοὶ αὐτοῦ, καὶ αἱ γυναῖκες τῶν υἱῶν αὐτοῦ μετʼ αὐτοῦ·
\VS{19}Καὶ πάντα τὰ θηρία, καὶ πάντα τὰ κτήνη, καὶ πᾶν πετεινὸν, καὶ πᾶν ἑρπετὸν κινούμενον ἐπὶ τῆς γῆς κατὰ γένος αὐτῶν, ἐξήλθοσαν ἐκ τῆς κιβωτοῦ.
\par }{\PP \VS{20}Καὶ ᾠκοδόμησε Νῶε θυσιαστήριον τῷ Κυρίῳ· καὶ ἔλαβεν ἀπὸ πάντων τῶν κτηνῶν τῶν καθαρῶν, καὶ ἀπὸ πάντων τῶν πετεινῶν τῶν καθαρῶν, καὶ ἀνήνεγκεν εἰς ὁλοκάρπωσιν ἐπὶ τὸ θυσιαστήριον.
\VS{21}Καὶ ὠσφράνθη Κύριος ὁ Θεὸς ὀσμὴν εὐωδίας. Καὶ εἶπε Κύριος ὁ Θεὸς διανοηθείς, οὐ προσθήσω ἔτι καταράσασθαι τὴν γῆν διὰ τὰ ἔργα τῶν ἀνθρώπων· ὅτι ἔγκειται ἡ διάνοια τοῦ ἀνθρώπου ἐπιμελῶς ἐπὶ τὰ πονηρὰ ἐκ νεότητος αὐτοῦ· οὐ προσθήσω οὖν ἔτι πατάξαι πᾶσαν σάρκα ζῶσαν, καθὼς ἐποίησα.
\VS{22}Πάσας τὰς ἡμέρας τῆς γῆς, σπέρμα καὶ θερισμὸς, ψύχος καὶ καῦμα, θέρος καὶ ἔαρ, ἡμέραν καὶ νύκτα, οὐ καταπαύσουσι.

\par }\Chap{9}{\PP \VerseOne{1}Καὶ εὐλόγησεν ὁ Θεὸς τὸν Νῶε, καὶ τοὺς υἱοὺς αὐτοῦ· καὶ εἶπεν αὐτοῖς· αὐξάνεσθε καὶ πληθύνεσθε, καὶ πληρώσατε τὴν γῆν, καὶ κατακυριεύσατε αὐτῆς.
\VS{2}Καὶ ὁ τρόμος, καὶ ὁ φόβος ὑμῶν, ἔσται ἐπὶ πᾶσι τοῖς θηρίοις τῆς γῆς, ἐπὶ πάντα τὰ πετεινὰ τοῦ οὐρανοῦ, καὶ ἐπὶ πάντα τὰ κινούμενα ἐπὶ τῆς γῆς, καὶ ἐπὶ πάντας τοὺς ἰχθύας τῆς θαλάσσης· ὑπὸ χεῖρας ὑμῖν δέδωκα.
\VS{3}Καὶ πᾶν ἑρπετὸν, ὅ ἐστι ζῶν, ὑμῖν ἔσται εἰς βρῶσιν· ὡς λάχανα χόρτου δέδωκα ὑμῖν τὰ πάντα.
\VS{4}Πλὴν κρέας ἐν αἵματι ψυχῆς οὐ φάγεσθε.
\VS{5}Καὶ γὰρ τὸ ὑμέτερον αἷμα τῶν ψυχῶν ὑμῶν ἐκ χειρὸς πάντων τῶν θηρίων ἐκζητήσω αὐτό· καὶ ἐκ χειρὸς ἀνθρώπου ἀδελφοῦ ἐκζητήσω τὴν ψυχὴν τοῦ ἀνθρώπου.
\VS{6}Ὁ ἐκχέων αἷμα ἀνθρώπου, ἀντὶ τοῦ αἵματος αὐτοῦ ἐκχυθήσεται, ὅτι ἐν εἰκόνι Θεοῦ ἐποίησα τὸν ἄνθρωπον.
\VS{7}Ὑμεῖς δὲ αὐξάνεσθε, καὶ πληθύνεσθε, καὶ πληρώσατε τὴν γῆν, καὶ κατακυριεύσατε αὐτῆς.
\par }{\PP \VS{8}Καὶ εἶπεν ὁ Θεὸς τῷ Νῶε καὶ τοῖς υἱοῖς αὐτοῦ, μετʼ αὐτοῦ λέγων,
\VS{9}καὶ ἰδοὺ ἐγὼ ἀνίστημι τὴν διαθήκην μου ὑμῖν, καὶ τῷ σπέρματι ὑμῶν μεθʼ ὑμᾶς,
\VS{10}καὶ πάσῃ ψυχῇ ζώσῃ μεθʼ ὑμῶν, ἀπὸ ὀρνέων, καὶ ἀπὸ κτηνῶν· καὶ πᾶσι τοῖς θηρίοις τῆς γῆς, ὅσα ἐστὶ μεθʼ ὑμῶν ἀπὸ πάντων τῶν ἐξελθόντων ἐκ τῆς κιβωτοῦ.
\VS{11}Καὶ στήσω τὴν διαθήκην μου πρὸς ὑμᾶς· καὶ οὐκ ἀποθανεῖται πᾶσα σὰρξ ἔτι ἀπὸ τοῦ ὕδατος τοῦ κατακλυσμοῦ· καὶ οὐκ ἔτι ἔσται κατακλυσμὸς ὕδατος, καταφθεῖραι πᾶσαν τὴν γῆν.
\VS{12}Καὶ εἶπε Κύριος ὁ Θεὸς πρὸς Νῶε· τοῦτο τὸ σημεῖον τῆς διαθήκης, ὃ ἐγὼ δίδωμι ἀνὰ μέσον ἐμοῦ καὶ ὑμῶν, καὶ ἀνὰ μέσον πάσης ψυχῆς ζώσης, ἥ ἐστι μεθʼ ὑμῶν εἰς γενεὰς αἰωνίους.
\VS{13}Τὸ τόξον μου τίθημι ἐν τῇ νεφέλῃ, καὶ ἔσται εἰς σημεῖον διαθήκης ἀνὰ μέσον ἐμοῦ καὶ τῆς γῆς.
\VS{14}Καὶ ἔσται ἐν τῷ συννεφεῖν με νεφέλας ἐπὶ τὴν γῆν, ὀφθήσεται τὸ τόξον ἐν τῇ νεφέλῃ.
\VS{15}Καὶ μνησθήσομαι τῆς διαθήκης μου, ἥ ἐστιν ἀνὰ μέσον ἐμοῦ καὶ ὑμῶν, καὶ ἀνὰ μέσον πάσης ψυχῆς ζώσης ἐν πάσῃ σαρκί· καὶ οὐκ ἔσται ἔτι τὸ ὕδωρ εἰς κατακλυσμὸν, ὥστε ἐξαλεῖψαι πᾶσαν σάρκα.
\VS{16}Καὶ ἔσται τὸ τόξον μου ἐν τῇ νεφέλῃ· καὶ ὄψομαι τοῦ μνησθῆναι διαθήκην αἰώνιον ἀνὰ μέσον ἐμοῦ καὶ τῆς γῆς, καὶ ἀνὰ μέσον ψυχῆς ζώσης ἐν πάσῃ σαρκὶ, ἥ ἐστιν ἐπὶ τῆς γῆς.
\VS{17}Καὶ εἶπεν ὁ Θεὸς τῷ Νῶε, τοῦτο τὸ σημεῖον τῆς διαθήκης, ἧς διεθέμην ἀνὰ μέσον ἐμοῦ, καὶ ἀνὰ μέσον πάσης σαρκὸς, ἥ ἐστιν ἐπὶ τῆς γῆς.
\par }{\PP \VS{18}Ἦσαν δὲ οἱ υἱοὶ Νῶε, οἱ ἐξελθόντες ἐκ τῆς κιβωτοῦ, Σὴμ, Χὰμ, Ἰάφεθ. Χὰμ δὲ ἦν πατὴρ Χαναάν.
\VS{19}Τρεῖς οὗτοί εἰσιν υἱοὶ Νῶε· ἀπὸ τούτων διεσπάρησαν ἐπὶ πᾶσαν τὴν γῆν.
\VS{20}Καὶ ἤρξατο Νῶε ἄνθρωπος γεωργὸς γῆς, καὶ ἐφύτευσεν ἀμπελῶνα.
\VS{21}Καὶ ἔπιεν ἐκ τοῦ οἴνου, καὶ ἐμεθύσθη, καὶ ἐγυμνώθη ἐν τῷ οἴκῳ αὐτοῦ.
\VS{22}Καὶ εἶδε Χὰμ ὁ πατὴρ Χαναὰν τὴν γύμνωσιν τοῦ πατρὸς αὐτοῦ, καὶ ἐξελθὼν ἀνήγγειλε τοῖς δυσὶν ἀδελφοῖς αὐτοῦ ἔξω.
\VS{23}Καὶ λαβόντες Σὴμ καὶ Ἰάφεθ τὸ ἱμάτιον, ἐπέθεντο ἐπὶ τὰ δύο νῶτα αὐτῶν, καὶ ἐπορεύθησαν ὀπισθοφανῶς, καὶ συνεκάλυψαν τὴν γύμνωσιν τοῦ πατρὸς αὐτῶν· καὶ τὸ πρόσωπον αὐτῶν ὀπισθοφανῶς, καὶ τὴν γύμνωσιν τοῦ πατρὸς αὐτῶν οὐκ εἶδον.
\VS{24}Ἐξένηψε δὲ Νῶε ἀπὸ τοῦ οἴνου, καὶ ἔγνω ὅσα ἐποίησεν αὐτῷ ὁ υἱὸς αὐτοῦ ὁ νεώτερος.
\VS{25}Καὶ εἶπεν, ἐπικατάρατος Χαναὰν παῖς· οἰκέτης ἔσται τοῖς ἀδελφοῖς αὐτοῦ.
\VS{26}Καὶ εἶπεν, εὐλογητὸς Κύριος ὁ Θεὸς τοῦ Σήμ· καὶ ἔσται Χαναὰν παῖς οἰκέτης αὐτοῦ.
\VS{27}Πλατύναι ὁ Θεὸς τῷ Ἰάφεθ, καὶ κατοικησάτω ἐν τοῖς οἴκοις τοῦ Σήμ· καὶ γενηθήτω Χαναὰν παῖς αὐτοῦ.
\par }{\PP \VS{28}Ἔζησε δὲ Νῶε μετὰ τὸν κατακλυσμὸν ἔτη τριακόσια πεντήκοντα.
\VS{29}Καὶ ἐγένοντο πᾶσαι αἱ ἡμέραι Νῶε ἐννακόσια πεντήκοντα ἔτη· καὶ ἀπέθανεν.

\par }\Chap{10}{\PP \VerseOne{1}Αὗται δὲ αἱ γενέσεις τῶν υἱῶν Νῶε, Σὴμ, Χὰμ, Ἰάφεθ· καὶ ἐγεννήθησαν αὐτοῖς υἱοὶ μετὰ τὸν κατακλυσμόν.
\par }{\PP \VS{2}Υἱοὶ Ἰάφεθ, Γαμὲρ, καὶ Μαγὼγ, καὶ Μαδοὶ, καὶ Ἰωύαν, καὶ Ἐλισὰ, καὶ Θοβὲλ, καὶ Μοσὸχ, καὶ Θείρας.
\VS{3}Καὶ υἱοὶ Γαμὲρ, Ἀσχανὰζ, καὶ Ῥιφὰθ, καὶ Θοργαμά.
\VS{4}Καὶ υἱοὶ Ἰωύαν, Ἐλισὰ, καὶ Θάρσεις, Κήτιοι, Ῥόδὶοι.
\VS{5}Ἐκ τούτων ἀφωρίσθησαν νῆσοι τῶν ἐθνῶν ἐν τῇ γῇ αὐτῶν· ἕκαστος κατὰ γλῶσσαν ἐν ταῖς φυλαῖς αὐτῶν, καὶ ἐν τοῖς ἔθνεσιν αὐτῶν.
\par }{\PP \VS{6}Υἱοὶ δὲ Χὰμ, Χοὺς, καὶ Μεσραῒν, Φοὺδ, καὶ Χαναάν.
\VS{7}Υἱοὶ δὲ Χοὺς, Σαβὰ, καὶ Εὐϊλὰ, καὶ Σαβαθὰ, καὶ Ῥεγμὰ, καὶ Σαβαθακά· υἱοὶ δὲ Ῥεγμὰ, Σαβὰ, καὶ Δαδάν.
\VS{8}Χοὺς δὲ ἐγέννησε τὸν Νεβρώδ· οὗτος ἤρξατο εἶναι γίγας ἐπὶ τῆς γῆς.
\VS{9}Οὗτος ἦν γίγας κυνηγὸς ἐναντίον Κυρίου τοῦ Θεοῦ· διὰ τοῦτο ἐροῦσιν, ὡς Νεβρὼδ γίγας κυνηγὸς ἐναντίον Κυρίου.
\VS{10}Καὶ ἐγένετο ἀρχὴ τῆς βασιλείας αὐτοῦ Βαβυλὼν, καὶ Ὀρὲχ, καὶ Ἀρχὰδ, καὶ Χαλάννη, ἐν τῇ γῇ Σεναάρ.
\VS{11}Ἐκ τῆς γῆς ἐκείνης ἐξῆλθεν Ἀσσούρ· καὶ ᾠκοδόμησε τὴν Νινευῒ, καὶ τὴν Ῥοωβὼθ πόλιν, καὶ τὴν Χαλὰχ,
\VS{12}καὶ τὴν Δασὴ ἀνὰ μέσον Νινευῒ, καὶ ἀνὰ μέσον Χαλάχ· αὕτη ἡ πόλις μεγάλη.
\VS{13}Καὶ Μεσραῒν ἐγέννησε τοὺς Λουδιεὶμ, καὶ τοὺς Νεφθαλεὶμ, καὶ τοὺς Ἐνεμετιεὶμ, καὶ τοὺς Λαβιεὶμ,
\VS{14}καὶ τοὺς Πατροσωνιεὶμ, καὶ τοὺς Χασμωνιεὶμ, ὅθεν ἐξῆλθε Φυλιστιεὶμ, καὶ τοὺς Γαφθοριείμ.
\VS{15}Χαναὰν δὲ ἐγέννησε τὸν Σιδῶνα πρωτότοκον αὐτοῦ, καὶ τὸν Χετταῖον,
\VS{16}καὶ τὸν Ἰεβουσαῖον, καὶ τὸν Ἀμοῤῥαῖον, καὶ τὸν Γεργεσαῖον,
\VS{17}καὶ τὸν Εὐαῖον, καὶ τὸν Ἀρουκαῖον, καὶ τὸν Ἀσενναῖον,
\VS{18}καὶ τον Ἀράδιον, καὶ τὸν Σαμαραῖον, καὶ τὸν Ἀμαθί. Καὶ μετὰ τοῦτο διεσπάρησαν αἱ φυλαὶ τῶν Χαναναίων.
\VS{19}Καὶ ἐγένετο τὰ ὅρια τῶν Χαναναίων ἀπὸ Σιδῶνος ἕως ἐλθεῖν εἰς Γεραρὰ καὶ Γαζὰν, ἕως ἐλθεῖν ἕως Σοδόμων καὶ Γομόῤῥας, Ἀδαμὰ καὶ Σεβωῒμ ἕως Δασά.
\VS{20}Οὗτοι υἱοὶ Χὰμ, ἐν ταῖς φυλαῖς αὐτῶν, κατὰ γλώσσας αὐτῶν, ἐν ταῖς χώραις αὐτῶν, καὶ ἐν τοῖς ἔθνεσιν αὐτῶν.
\par }{\PP \VS{21}Καὶ τῷ Σὴμ ἐγεννήθη καὶ αὐτῷ πατρὶ πάντων τῶν υἱῶν Ἕβερ, ἀδελφῷ Ἰάφεθ τοῦ μείζονος.
\VS{22}Υἱοὶ Σὴμ, Ἐλὰμ, καὶ Ἀσσοὺρ, καὶ Ἀρφαξὰδ, καὶ Λοὺδ, καὶ Ἀρὰμ, καὶ Καϊνᾶν.
\VS{23}Καὶ υἱοὶ Ἀρὰμ, Οὒζ, καὶ Οὒλ, καὶ Γατὲρ, καὶ Μοσόχ.
\VS{24}Καὶ Ἀρφαξὰδ ἐγέννησε τὸν Καϊνᾶν, καὶ Καϊνᾶν ἐγέννησε τὸν Σαλά· Σαλὰ δὲ ἐγέννησε τὸν Ἕβερ.
\VS{25}Καὶ τῷ Ἕβερ ἐγεννήθησαν δύο υἱοί· ὄνομα τῷ ἑνὶ, Φαλὲγ, ὅτι ἐν ταῖς ἡμέραις αὐτοῦ διεμερίσθη ἡ γῆ· καὶ ὄνομα τῷ ἀδελφῷ αὐτοῦ Ἰεκτάν.
\VS{26}Ἰεκτὰν δὲ ἐγέννησε τὸν Ἐλμωδὰδ, καὶ Σαλὲθ, καὶ τὸν Σαρμὼθ, καὶ Ἰαρὰχ,
\VS{27}καὶ Ὁδοῤῥὰ, καὶ Αἰβὴλ, καὶ Δεκλὰ, καὶ Εὐὰλ,
\VS{28}καὶ Ἀβιμαὲλ, καὶ Σαβὰ,
\VS{29}καὶ Οὐφεὶρ, καὶ Εὑεϊλὰ, καὶ Ἰωβάβ· πάντες οὗτοι υἱοὶ Ἰεκτάν.
\VS{30}Καὶ ἐγένετο ἡ κατοίκησις αὐτῶν, ἀπὸ Μασσῆ ἕως ἐλθεῖν εἰς Σαφηρὰ ὄρος ἀνατολῶν.
\VS{31}Οὗτοι υἱοὶ Σὴμ, ἐν ταῖς φυλαῖς αὐτῶν, κατὰ γλώσσας αὐτῶν, ἐν ταῖς χώραις αὐτῶν, καὶ ἐν τοῖς ἔθνεσιν αὐτῶν.
\VS{32}Αὗται αἱ φυλαὶ υἱῶν Νῶε κατὰ γενέσεις αὐτῶν, κατὰ ἔθνη αὐτῶν· ἀπὸ τούτων διεσπάρησαν νῆσοι τῶν ἐθνῶν ἐπὶ τῆς γῆς μετὰ τὸν κατακλυσμόν.

\par }\Chap{11}{\PP \VerseOne{1}Καὶ ἦν πᾶσα ἡ γῆ χεῖλος ἓν, καὶ φωνὴ μία πᾶσι.
\VS{2}Καὶ ἐγένετο ἐν τῷ κινῆσαι αὐτοὺς ἀπὸ ἀνατολῶν, εὗρον πεδίον ἐν γῇ Σεναὰρ, καὶ κατῴκησαν ἐκεῖ.
\VS{3}Καὶ εἶπεν ἄνθρωπος τῷ πλησίον αὐτοῦ, δεῦτε πλινθεύσωμεν πλίνθους, καὶ ὀπτήσωμεν αὐτὰς πυρί· καὶ ἐγένετο αὐτοῖς ἡ πλίνθος εἰς λίθον, καὶ ἄσφαλτος ἦν αὐτοῖς ὁ πηλός.
\VS{4}Καὶ εἶπαν, δεῦτε οἰκοδομήσωμεν ἑαυτοῖς πόλιν καὶ πύργον, οὗ ἔσται ἡ κεφαλὴ ἕως τοῦ οὐρανοῦ, καὶ ποιήσωμεν ἑαυτοῖς ὄνομα, πρὸ τοῦ διασπαρῆναι ἡμᾶς ἐπὶ προσώπου πάσης τῆς γῆς.
\VS{5}Καὶ κατέβη Κύριος ἰδεῖν τὴν πόλιν καὶ τὸν πύργον, ὃν ᾠκοδόμησαν οἱ υἱοὶ τῶν ἀνθρώπων.
\VS{6}Καὶ εἶπε Κύριος, ἰδοὺ γένος ἓν, καὶ χεῖλος ἓν πάντων, καὶ τοῦτο ἤρξαντο ποιῆσαι, καὶ νῦν οὐκ ἐκλείψει ἀπʼ αὐτῶν πάντα ὅσα ἂν ἐπιθῶνται ποιεῖν.
\VS{7}Δεῦτε, καὶ καταβάντες συγχέωμεν αὐτῶν ἐκεῖ τὴν γλῶσσαν, ἵνα μὴ ἀκούσωσιν ἕκαστος τὴν φωνὴν τοῦ πλησίον.
\VS{8}Καὶ διέσπειρεν αὐτοὺς Κύριος ἐκεῖθεν ἐπὶ πρόσωπον πάσης τῆς γῆς· καὶ ἐπαύσαντο οἰκοδομοῦντες τῆν πόλιν καὶ τὸν πύργον.
\VS{9}Διὰ τοῦτο ἐκλήθη τὸ ὄνομα αὐτῆς, Σύγχυσις, ὅτι ἐκεῖ συνέχεε Κύριος τὰ χείλη πάσης τῆς γῆς, καὶ ἐκεῖθεν διέσπειρεν αὐτοὺς Κύριος ἐπὶ πρόσωπον πάσης τῆς γῆς.
\par }{\PP \VS{10}Καὶ αὗται αἱ γενέσεις Σήμ· καὶ ἦν Σὴμ υἱὸς ἑκατὸν ἐτῶν, ὅτε ἐγέννησε τὸν Ἀρφαξὰδ, δευτέρου ἔτους μετὰ τὸν κατακλυσμόν.
\VS{11}Καὶ ἔζησε Σὴμ, μετὰ τὸ γεννῆσαι αὐτὸν τὸν Ἀρφαξὰδ, ἔτη πεντακόσια, καὶ ἐγέννησεν υἱοὺς καὶ θυγατέρας, καὶ ἀπέθανε.
\VS{12}Καὶ ἔζησεν Ἀρφαξὰδ ἑκατὸν τριακονταπέντε ἔτη, καὶ ἐγέννησε τὸν Καϊνᾶν.
\VS{13}Καὶ ἔζησεν Ἀρφαξὰδ, μετὰ τὸ γεννῆσαι αὐτὸν τὸν Καϊνᾶν, ἔτη τετρακόσια, καὶ ἐγέννησεν υἱοὺς καὶ θυγατέρας, καὶ ἀπέθανε. Καὶ ἔζησε Καϊνᾶν ἑκατὸν καὶ τριάκοντα ἔτη, καὶ ἐγέννησε τὸν Σαλά· καὶ ἔξησε Καϊνᾶν, μετὰ τὸ γεννῆσαι αὐτὸν τὸν Σαλὰ, ἔτη τριακόσια τριάκοντα, καὶ ἐγέννησεν υἱοὺς καὶ θυγατέρας, καὶ ἀπέθανε.
\VS{14}Καὶ ἔζησε Σαλὰ ἑκατὸν τριάκοντα ἔτη, καὶ ἐγέννησε τὸν Ἕβερ.
\VS{15}Καὶ ἔζησε Σαλὰ μετὰ τὸ γεννῆσαι αὐτὸν τὸν Ἕβερ, τριακόσια τριάκοντα ἔτη, καὶ ἐγέννησεν υἱοὺς καὶ θυγατέρας· καὶ ἀπέθανε.
\VS{16}Καὶ ἔζησεν Ἕβερ ἑκατὸν τριάκοντα τέσσαρα ἔτη, καὶ ἐγέννησε τὸν Φαλέγ.
\VS{17}Καὶ ἔξησεν Ἕβερ, μετὰ τὸ γεννῆσαι αὐτὸν τὸν Φαλὲγ, ἔτη διακόσια ἑβδομήκοντα, καὶ ἐγέννησεν υἱοὺς καὶ θυγατέρας, καὶ ἀπέθανε.
\VS{18}Καὶ ἔζησε Φαλὲγ τριάκοντα καὶ ἑκατὸν ἔτη, καὶ ἐγέννησε τὸν Ῥαγαῦ.
\VS{19}Καὶ ἔζησε Φαλὲγ, μετὰ τὸ γεννῆσαι αὐτὸν τὸν Ῥαγαῦ, ἐννέα καὶ διακόσια ἔτη, καὶ ἐγέννησεν υἱοὺς και θυγατέρας, καὶ ἀπέθανε.
\VS{20}Καὶ ἔζησε Ῥαγαὺ ἑκατὸν τριάκοντα καὶ δύο ἔτη, καὶ ἐγέννησε τὸν Σερούχ.
\VS{21}Καὶ ἔζησε Ῥαγαῦ, μετὰ τὸ γεννῆσαι αὐτὸν τὸν Σεροὺχ, διακόσια ἑπτὰ ἔτη, καὶ ἐγέννησεν υἱοὺς καὶ θυγατέρας, καὶ ἀπέθανε.
\VS{22}Καὶ ἔζησε Σεροὺχ ἑκατὸν τριάκοντα ἔτη, καὶ ἐγέννησε τὸν Ναχώρ.
\VS{23}Καὶ ἔζησε Σεροὺχ, μετὰ τὸ γεννῆσαι αὐτὸν τὸν Ναχὼρ, ἔτη διακόσια, καὶ ἐγέννησεν υἱοὺς καὶ θυγατέρας, καὶ ἀπέθανε.
\VS{24}Καὶ ἔζησε Ναχὼρ ἔτη ἑκατὸν ἑβδομηκονταεννέα, καὶ ἐγέννησε τὸν Θάῤῥα.
\VS{25}Καὶ ἔζησε Ναχὼρ, μετὰ τὸ γεννῆσαι αὐτὸν τὸν Θάῤῥα, ἔτη ἑκατὸν εἰκοσιπὲντε, καὶ ἐγεννησεν υἱοὺς καὶ θυγατέρας, καὶ ἀπέθανε.
\VS{26}Καὶ ἔζησε Θάῤῥα ἑβδομήκοντα ἔτη, καὶ ἐγέννησε τὸν Ἄβραμ, καὶ τὸν Ναχὼρ, καὶ τὸν Ἀῤῥάν.
\par }{\PP \VS{27}Αὗται δὲ αἱ γενέσεις Θάῤῥα· Θάῤῥα ἐγέννησε τὸν Ἅβραμ, καὶ τὸν Ναχὼρ, καὶ τὸν Ἀῤῥάν· καὶ Ἀῤῥὰν ἐγέννησε τὸν Λώτ.
\VS{28}Καὶ ἀπέθανεν Ἀῤῥὰν ἐνώπιον Θάῤῥα τοῦ πατρὸς αὐτοῦ ἐν τῇ γῇ ᾗ ἐγενήθη, ἐν τῇ χώρᾳ τῶν Χαλδαίων.
\VS{29}Καὶ ἔλαβον Ἅβραμ καὶ Ναχὼρ ἑαυτοῖς γυναῖκας· ὄνομα τῇ γυναικὶ Ἅβραμ, Σάρα, καὶ ὄνομα τῇ γυναικὶ Ναχὼρ, Μελχά, θυγάτηρ Ἀῤῥάν· καὶ πατὴρ Μελχὰ, καὶ πατὴρ Ἰεσχά.
\VS{30}Καὶ ἦν Σάρα στεῖρα, καὶ οὐκ ἐτεκνοποίει.
\VS{31}Καὶ ἔλαβε Θάῤῥα τὸν Ἅβραμ υἱὸν αὐτοῦ, καὶ τὸν Λὼτ υἱὸν Ἀῤῥάν, υἱὸν τοῦ υἱοῦ αὐτοῦ, καὶ τὴν Σάραν τὴν νύμφην αὐτοῦ, γυναῖκα Ἅβραμ τοῦ υἱοῦ αὐτοῦ, καὶ ἐξήγαγεν αὐτοὺς ἐκ τῆς χώρας τῶν Χαλδαίων, πορευθῆναι εἰς γῆν Χαναάν· καὶ ἦλθον ἕως Χαῤῥὰν, καὶ κατῴκησεν ἐκεῖ.
\VS{32}Καὶ ἐγένοντο πᾶσαι αἱ ἡμέραι Θάῤῥα ἐν γῇ Χαῤῥὰν, διακόσια πέντε ἔτη· καὶ ἀπέθανε Θάῤῥα ἐν Χαῤῥάν.

\par }\Chap{12}{\PP \VerseOne{1}Καὶ εἶπε Κύριος τῷ Ἅβραμ, ἔξελθε ἐκ τῆς γῆς σου, καὶ ἐκ τῆς συγγενείας σου, καὶ ἐκ τοῦ οἴκου τοῦ πατρός σου, καὶ δεῦρο εἰς τὴν γῆν, ἣν ἄν σοι δείξω.
\VS{2}Καὶ ποιήσω σε εἰς ἔθνος μέγα, καὶ εὐλογήσω σε, καὶ μεγαλυνῶ τὸ ὄνομά σου, καὶ ἔσῃ εὐλογημένος.
\VS{3}Καὶ εὐλογήσω τοὺς εὐλογοῦντάς σε, καὶ τοὺς καταρωμένους σε καταράσομαι, καὶ ἐνευλογηθήσονται ἐν σοὶ πᾶσαι αἱ φυλαὶ τῆς γῆς.
\VS{4}Καὶ ἐπορεύθη Ἅβραμ, καθάπερ ἐλάλησεν αὐτῷ Κύριος, καὶ ᾤχετο μετʼ αὐτοῦ Λώτ· Ἅβραμ δὲ ἦν ἐτῶν ἑβδομηκονταπέντε, ὅτε ἐξῆλθεν ἐκ Χαῤῥάν.
\VS{5}Καὶ ἔλαβεν Ἅβραμ Σάραν τὴν γυναῖκα αὐτοῦ, καὶ τὸν Λὼτ υἱὸν τοῦ ἀδελφοῦ αὐτοῦ, καὶ πάντα τὰ ὑπάρχοντα αὐτῶν ὅσα ἐκτήσαντο, καὶ πᾶσαν ψυχὴν ἣν ἐκτήσαντο, ἐκ Χαῤῥάν, καὶ ἐξήλθοσαν πορευθῆναι εἰς γῆν Χανάαν.
\VS{6}Καὶ διώδευσεν Ἅβραμ τὴν γῆν εἰς τὸ μῆκος αὐτῆς ἕως τοῦ τόπου Συχέμ, ἐπὶ τὴν δρῦν τὴν ὑψηλήν· οἱ δὲ Χαναναῖοι τότε κατῴκουν τὴν γῆν.
\VS{7}Καὶ ὤφθη Κύριος τῷ Ἅβραμ, καὶ εἶπεν αὐτῷ, τῷ σπέρματί σου δώσω τὴν γῆν ταύτην· καὶ ᾠκοδόμησεν ἐκεῖ Ἅβραμ θυσιαστήριον Κυρίῳ τῷ ὀφθέντι αὐτῷ.
\VS{8}Καὶ ἀπέστη ἐκεῖθεν εἰς τὸ ὄρος κατὰ ἀνατολὰς Βαιθήλ· καὶ ἔστησεν ἐκεῖ τὴν σκηνὴν αὐτοῦ ἐν Βαιθὴλ κατὰ θάλασσαν, καὶ Ἀγγαὶ κατὰ ἀνατολάς· καὶ ᾠκοδόμησεν ἐκεῖ θυσιαστήριον τῷ Κυρίῳ, καὶ ἐπεκαλέσατο ἐπὶ τῷ ὀνόματι Κυρίου.
\VS{9}Καὶ ἀπῇρεν Ἅβραμ, καὶ πορευθεὶς ἐστρατοπέδευσεν ἐν τῇ ἐρήμῳ.
\par }{\PP \VS{10}Καὶ ἐγένετο λιμὸς ἐπὶ τῆς γῆς· καὶ κατέβη Ἅβραμ εἰς Αἴγυπτον παροικῆσαι ἐκεῖ, ὅτι ἐνίσχυσεν ὁ λιμὸς ἐπὶ τῆς γῆς.
\VS{11}Ἐγένετο δὲ ἡνίκα ἤγγισεν Ἅβραμ εἰσελθεῖν εἰς Αἴγυπτον, εἶπεν Ἅβραμ Σάρα τῇ γυναικὶ, γινώσκω ἐγὼ, ὅτι γυνὴ εὐπρόσωπος εἶ.
\VS{12}Ἔσται οὖν ὡς ἂν ἴδωσί σε οἱ Αἰγύπτιοι, ἐροῦσιν ὅτι γυνὴ αὐτοῦ ἐστιν αὐτὴ, καὶ ἀποκτενοῦσί με, σὲ δὲ περιποιήσονται.
\VS{13}Εἶπον οὖν, ὅτι ἀδελφὴ αὐτοῦ εἰμι, ὅπως ἄν εὖ μοι γένηται διὰ σὲ, καὶ ζήσεται ἡ ψυχή μου ἕνεκέν σου.
\VS{14}Ἐγένετο δὲ, ἡνίκα εἰσῆλθεν Ἅβραμ εἰς Αἴγυπτον, ἰδόντες οἱ Αἰγύπτιοι τὴν γυναῖκα αὐτοῦ, ὅτι καλὴ ἦν σφόδρα.
\VS{15}Καὶ ἴδον αὐτὴν οἱ ἄρχοντες Φαραὼ, καὶ ἐπῄνεσαν αὐτὴν πρὸς Φαραὼ, καὶ εἰσήγαγον αὐτὴν εἰς τὸν οἶκον Φαραώ.
\VS{16}Καὶ τῷ Ἅβραμ εὖ ἐχρήσαντο διʼ αὐτήν· καὶ ἐγένοντο αὐτῷ πρόβατα, καὶ μόσχοι, καὶ ὄνοι, καὶ παῖδες, καὶ παιδίσκαι, καὶ ἡμίονοι, καὶ κάμηλοι.
\VS{17}Καὶ ἤτασεν ὁ Θεὸς τὸν Φαραὼ ἐτασμοῖς μεγάλοις καὶ πονηροῖς, καὶ τὸν οἶκον αὐτοῦ, περὶ Σάρας τῆς γυναικὸς Ἅβραμ.
\VS{18}Καλέσας δὲ Φαραὼ τὸν Ἅβραμ, εἶπεν, τί τοῦτο ἐποίησάς μοι, ὅτι οὐκ ἀπήγγειλάς μοι, ὅτι γυνή σου ἐστίν;
\VS{19}Ἱνατί εἶπας ὅτι ἀδελφή μου ἐστίν; καὶ ἔλαβον αὐτὴν ἐμαυτῷ γυναῖκα· καὶ νῦν ἰδοὺ ἡ γυνή σου ἔναντί σου, λαβὼν ἀπότρεχε.
\VS{20}Καὶ ἐνετείλατο Φαραὼ ἀνδράσι περὶ Ἅβραμ συμπροπέμψαι αὐτὸν, καὶ τὴν γυναῖκα αὐτοῦ, καὶ πάντα ὅσα ἦν αὐτῷ.

\par }\Chap{13}{\PP \VerseOne{1}Ἀνέβη δὲ Ἅβραμ ἐξ Αἰγύπτου αὐτὸς, καὶ ἡ γυνὴ αὐτοῦ, καὶ πάντα τὰ αὐτοῦ, καὶ Λὼτ μετʼ αὐτοῦ, εἰς τὴν ἔρημον.
\VS{2}Ἅβραμ δὲ ἦν πλούσιος σφόδρα κτήνεσι, καὶ ἀργυρίῳ, καὶ χρυσίῳ.
\VS{3}Καὶ ἐπορεύθη ὅθεν ἦλθεν εἰς τὴν ἔρημον ἕως Βαιθὴλ, ἕως τοῦ τόπου οὗ ἦν ἡ σκηνὴ αὐτοῦ τὸ πρότερον, ἀνὰ μέσον Βαιθὴλ καὶ ἀνὰ μέσον Ἀγγαί,
\VS{4}εἰς τὸν τόπον τοῦ θυσιαστηρίου, οὗ ἐποίησεν ἐκεῖ τὴν ἀρχὴν, καὶ ἐπεκαλέσατο ἐκεῖ Ἅβραμ τὸ ὄνομα τοῦ Κυρίου.
\VS{5}Καὶ Λὼτ τῷ συμπορευομένῳ μετὰ Ἅβραμ ἦν πρόβατα, καὶ βόες, καὶ σκηναί.
\VS{6}Καὶ οὐκ ἐχώρει αὐτοὺς ἡ γῆ κατοικεῖν ἅμα, ὅτι ἦν τὰ ὑπάρχοντα αὐτῶν πολλά· καὶ οὐκ ἐχώρει αὐτοὺδ ἡ γῆ κατοικεῖν ἅμα.
\VS{7}Καὶ ἐγενετο μάχη ἀνὰ μέσον τῶν ποιμένων τῶν κτηνῶν τοῦ Ἅβραμ, καὶ ἀνὰ μέσον τῶν ποιμένων τῶν κτηνῶν τοῦ Λώτ· οἱ δὲ Χαναναῖοι καὶ οἱ Φερεζαῖοι τότε κατῴκουν τὴν γῆν.
\VS{8}Εἶπε δὲ Ἅβραμ τῷ Λὼτ, μὴ ἔστω μάχη ἀνὰ μέσον ἐμοῦ καὶ σοῦ, καὶ ἀνὰ μέσον τῶν ποιμένων μου καὶ ἀνὰ μέσον τῶν ποιμένων σοῦ, ὅτι ἄνθρωποι ἀδελφοὶ ἐσμὲν ἡμεῖς.
\VS{9}Οὐκ ἰδοὺ πᾶσα ἡ γῆ ἐναντίον σου ἐστί; διαχωρίσθητι ἀπʼ ἐμοῦ· εἰ σὺ εἰς ἀριστερὰ, ἐγὼ εἰς δεξιά· εἰ δὲ σὺ εἰς δεξιὰ, ἐγὼ εἰς ἀριστερά.
\VS{10}Καὶ ἐπάρας Λὼτ τοὺς ὀφθαλμοὺς αὐτοῦ, ἐπεῖδε πᾶσαν τὴν περίχωρον τοῦ Ἰορδάνου, ὅτι πᾶσα ἦν ποτιζομένη, πρὸ τοῦ καταστρέψαι τὸν Θεὸν Σόδομα καὶ Γόμοῤῥα, ὡς ὁ παράδεισος τοῦ Θεοῦ, καὶ ὡς ἡ γῆ Αἰγύπτου, ἕως ἐλθεῖν εἰς Ζόγορα.
\VS{11}Καὶ ἐξελέξατο ἑαυτῷ Λὼτ πᾶσαν τὴν περίχωρον τοῦ Ἰορδάνου· καὶ ἀπῇρε Λὼτ ἀπὸ ἀνατολῶν· καὶ διεχωρίσθησαν ἕκαστος ἀπὸ τοῦ ἀδελφοῦ αὐτοῦ.
\VS{12}Ἅβραμ δὲ κατῴκησεν ἐν γῇ Χαναάν· Λὼτ δὲ κατῴκησεν ἐν πόλει τῶν περιχώρων, καὶ ἐσκήνωσεν ἐν Σοδόμοις.
\VS{13}Οἱ δὲ ἄνθρωποι οἱ ἐν Σοδόμοις πονηροὶ καὶ ἁμαρτωλοὶ ἐναντίον τοῦ Θεοῦ σφόδρα.
\VS{14}Ὁ δὲ Θεὸς εἶπε τῷ Ἅβραμ μετὰ τὸ διαχωρισθῆναι τὸν Λὼτ ἀπʼ αὐτοῦ, ἀνάβλεψον τοῖς ὀφθαλμοῖς σου, καὶ ἴδε ἀπὸ τοῦ τόπου οὗ νῦν σὺ εἶ πρὸς βοῤῥὰν καὶ λίβα καὶ ἀνατολὰς καὶ θάλασσαν·
\VS{15}ὅτι πᾶσαν τὴν γῆν, ἣν σὺ ὁρᾷς, σοὶ δώσω αὐτὴν καὶ τῷ σπέρματί σου ἕως αἰῶνος.
\VS{16}Καὶ ποιήσω τὸ σπέρμα σου, ὡς τὴν ἄμμον τῆς γῆς· εἰ δύναταί τις ἐξαριθμῆσαι τὴν ἄμμον τῆς γῆς, καὶ τὸ σπέρμα σου ἐξαριθμηθήσεται.
\VS{17}Ἀναστὰς διόδευσον τὴν γῆν εἴς τε τὸ μῆκος αὐτῆς καὶ εἰς τὸ πλάτος· ὅτι σοι δώσω αὐτὴν καὶ τῷ σπέρματί σου εἰς τὸν αἰῶνα.
\VS{18}Καὶ ἀποσκηνώσας Ἅβραμ, ἐλθὼν κατῴκησε παρὰ τὴν δρῦν τὴν Μαμβρῆ, ἣ ἦν ἐν Χεβρὼμ, καὶ ᾠκοδόμησεν ἐκεῖ θυσιαστήριον τῷ Κυρίῳ.

\par }\Chap{14}{\PP \VerseOne{1}Ἐγένετο δὲ ἐν τῇ βασιλείᾳ τῇ Ἀμαρφὰλ βασιλέως Σενναὰρ, καὶ Ἀριὼχ βασιλέως Ἑλλασὰρ, Χοδολλογομὸρ βασιλεὺς Ἐλὰμ, καὶ Θαργὰλ βασιλεὺς ἐθνῶν,
\VS{2}ἐποίησαν πόλεμον μετὰ Βαλλὰ βασιλέως Σοδόμων, καὶ μετὰ Βαρσὰ βασιλέως Γομόῤῥας, καὶ μετὰ Σενναὰρ βασιλέως Ἀδαμὰ, καὶ μετὰ Συμοβὸρ βασιλέως Σεβωεὶμ, καὶ βασιλέως Βαλάκ· αὕτη ἐστὶ Σηγώρ.
\VS{3}Πάντες οὗτοι συνεφώνησαν ἐπὶ τὴν φάραγγα τὴν ἁλυκήν· αὕτη ἡ θάλασσα τῶν ἁλῶν.
\VS{4}Δώδεκα ἔτη αὐτοὶ ἐδούλευσαν τῷ Χοδολλογομόρ· τῷ δὲ τρισκαιδεκάτῳ ἔτει ἀπέστησαν.
\VS{5}Ἐν δὲ τῷ τεσσαρεσκαιδεκάτῳ ἔτει ἦλθε Χοδολλογομὸρ καὶ οἱ βασιλεῖς μετʼ αὐτοῦ, καὶ κατέκοψαν τοὺς γίγαντας τοὺς ἐν Ἀσταρὼθ, καὶ Καρναῒν, καὶ ἔθνη ἰσχυρὰ ἅμα αὐτοῖς, καὶ τοὺς Ὀμμαίους τοὺς ἐν Σαυῇ τῇ πόλει.
\VS{6}Καὶ τοὺς Χοῤῥαίους τοὺς ἐν τοῖς ὄρεσι Σηεὶρ, ἕως τῆς τερεβίνθου τῆς Φαρὰν, ἥ ἐστιν ἐν τῇ ἐρήμῳ.
\VS{7}Καὶ ἀναστρέψαντες ἦλθον ἐπὶ τὴν πηγὴν τῆς κρίσεως· αὕτη ἐστὶ Κάδης· καὶ κατέκοψαν πάντας τοὺς ἄρχοντας Ἀμαλὴκ, καὶ τοὺς Ἀμοῤῥαίους τοὺς κατοικοῦντας ἐν ʼΑσασονθαμὰρ
\VS{8}Ἐξῆλθε δὲ βασιλεὺς Σοδόμων, καὶ βασιλεὺς Γομόῤῥας, καὶ βασιλεὺς Ἀδαμὰ, καὶ βασιλεὺς Σεβωεὶμ, καὶ βασιλεὺς Βαλάκ· αὕτη ἐστὶ Σηγώρ· καὶ παρετάξαντο αὐτοῖς εἰς πόλεμον ἐν τῇ κοιλάδι, τῇ ἁλυκῇ,
\VS{9}πρὸς Χοδολλογομὸρ βασιλέα Ἐλὰμ, καὶ Θαπγὰλ βασιλέα ἐθνῶν, καὶ Ἀμαρφὰλ βασιλέα Σενναὰρ, καὶ Ἀριὼχ βασιλέα Ἑλλασὰρ, οἱ τέσσαρες βασιλεῖς πρὸς τοὺς πέντε.
\VS{10}Ἡ δὲ κοιλὰς ἡ ἁλυκὴ, φρέατα ἀσφάλτου· ἔφυγε δὲ βασιλεὺς Σοδόμων καὶ βασιλεὺς Γομόῤῥας, καὶ ἐνέπεσαν ἐκεῖ· οἱ δὲ καταλειφθέντες εἰς τὴν ὀρεινὴν ἔφυγον.
\VS{11}Ἔλαβον δὲ τὴν ἵππον πᾶσαν τὴν Σοδόμων καὶ Γομόῤῥας, καὶ πάντα τὰ βρώματα αὐτῶν, καὶ ἀπῆλθον.
\VS{12}Ἔλαβον δὲ καὶ τὸν Λὼτ τὸν υἱὸν τοῦ ἀδελφοῦ Ἅβραμ, καὶ τὴν ἀποσκευὴν αὐτοῦ, καὶ ἀπῴχοντο· ἦν γὰρ κατοικῶν ἐν Σοδόμοις.
\par }{\PP \VS{13}Παραγενόμενος δὲ τῶν ἀνασωθέντων τις ἀπήγγειλεν Ἅβραμ τῷ περάτῃ· αὐτὸς δὲ κατῴκει παρὰ τῇ δρυῒ τῇ Μαμβρῇ Ἀμοῤῥαίου τοῦ ἀδελφοῦ Ἐσχὼλ, καὶ τοῦ ἀδελφοῦ Αὐνὰν, οἳ ἦσαν συνωμόται τοῦ Ἅβραμ.
\VS{14}Ἀκούσας δὲ Ἅβραμ ὅτι ᾐχμαλώτευται Λὼτ ὁ ἀδελφοῦς αὐτοῦ, ἠρίθμησε τοὺς ἰδίους οἰκογενεῖς αὐτοῦ τριακοσίους δέκα καὶ ὀκτώ· καὶ κατεδίωξεν ὀπίσω αὐτῶν ἕως Δάν.
\VS{15}Καὶ ἐπέπεσεν ἐπʼ αὐτοὺς τὴν νύκτα αὐτὸς, καὶ οἱ παῖδες αὐτοῦ, καὶ ἐπάταξεν αὐτοὺς, καὶ κατεδίωξεν αὐτοὺς ἕως Χοβὰ, ἥ ἐστιν ἐν ἀριστερᾷ Δαμασκοῦ.
\VS{16}Καὶ ἀπέστρεψε πᾶσαν τὴν ἵππον Σοδόμων· καὶ Λὼτ τὸν ἀδελφιδοῦν αὐτοῦ ἀπέστρεψε, καὶ πάντα τὰ ὑπάρχοντα αὐτοῦ, καὶ τὰς γυναῖκας, καὶ τὸν λαόν.
\VS{17}Ἐξῆλθε δὲ βασιλεὺς Σοδόμων εἰς συνάντησιν αὐτῷ, μετὰ τὸ ὑποστρέψαι αὐτὸν ἀπὸ τῆς κοπῆς τοῦ Χοδολλογομὸρ, καὶ τῶν βασιλέων τῶν μετʼ αὐτοῦ εἰς τὴν κοιλάδα τοῦ Σαβύ· τοῦτο ἦν τὸ πεδίον τῶν βασιλέων.
\par }{\PP \VS{18}Καὶ Μελχισεδὲκ βασιλεὺς Σαλὴμ ἐξήνεγκεν ἄρτους καὶ οἶνον· ἦν δὲ ἱερεὺς τοῦ Θεοῦ τοῦ ὑψίστου.
\VS{19}Καὶ εὐλόγησε τὸν Ἅβραμ, καὶ εἶπεν, εὐλογημένος Ἅβραμ τῷ Θεῷ τῷ ὑψίστῳ, ὃς ἔκτισε τὸν οὐρανὸν καὶ τὴν γῆν.
\VS{20}Καὶ εὐλογητὸς ὁ Θεὸς ὁ ὕψιστος, ὃς παρέδωκε τοὺς ἐχθρούς σου ὑποχειρίους σοι· καὶ ἔδωκεν αὐτῷ Ἅβραμ δεκάτην ἀπὸ πάντων.
\VS{21}Εἶπε δὲ βασιλεὺς Σοδόμων πρὸς Ἅβραμ, δός μοι τοὺς ἄνδρας, τὴν δὲ ἵππον λάβε σεαυτῷ.
\VS{22}Εἶπε δὲ Ἅβραμ πρὸς τὸν βασιλέα Σοδόμων, ἐκτενῶ τὴν χεῖρά μου πρὸς Κύπιον τὸν Θεὸν τὸν ὕψιστον, ὃς ἔκτισε τὸν οὐρανὸν καὶ τὴν γῆν,
\VS{23}εἰ ἀπὸ σπαρτίου ἕως σφυρωτῆρος ὑποδήματος λήψομαι ἀπὸ πάντων τῶν σῶν, ἵνα μὴ εἴπῃς, ὅτι ἐγὼ ἐπλούτισα τὸν Ἅβραμ.
\VS{24}Πλὴν ὧν ἔφαγον οἱ νεανίσκοι, καὶ τῆς μερίδος τῶν ἀνδρῶν τῶν συμπορευθέντων μετʼ ἐμοῦ Ἐσχὼλ, Αὐνᾶν, Μαμβρῆ· οὗτοι λήψονται μερίδα.

\par }\Chap{15}{\PP \VerseOne{1}Μετὰ δὲ τὰ ῥήματα ταῦτα ἐγενήθη ῥῆμα Κυρίου πρὸς Ἅβραμ ἐν ὁράματι, λέγων, μὴ φοβοῦ Ἅβραμ· ἐγὼ ὑπερασπίζω σου· ὁ μισθός σου πολὺς ἔσται σφόδρα.
\VS{2}Δέγει δὲ Ἅβραμ, Δέσποτα Κύριε, τί μοι δώσεις; ἐγὼ δὲ ἀπολύομαι ἄτεκνος· ὁ δὲ υἱὸς Μασὲκ τῆς οἰκογενοῦς μου, οὗτος Δαμασκὸς Ἐλιέζερ.
\VS{3}Καὶ εἶπεν Ἅβραμ, ἐπειδὴ ἐμοὶ οὐκ ἔδωκας σπέρμα, ὁ δὲ οἰκογενής μου κληρονομήσει με.
\VS{4}Καὶ εὐθὺς φωνὴ Κυρίου ἐγένετο πρὸς αὐτὸν, λέγουσα, οὐ κληρονομήσει σε οὗτος· ἀλλʼ ὃς ἐξελεύσεται ἐκ σοῦ, οὗτος κληρονομήσει σε.
\VS{5}Ἐξήγαγε δὲ αὐτὸν ἔξω, καὶ εἶπεν αὐτῷ, ἀνάβλεψον δὴ εἰς τὸν οὐρανὸν, καὶ ἀρίθμησον τοὺς ἀστέρας, εἰ δυνήσῃ ἐξαριθμῆσαι αὐτούς· καὶ εἶπεν, οὕτως ἔσται τὸ σπέρμα σου.
\VS{6}Καὶ ἐπίστευσεν Ἅβραμ τῷ Θεῷ, καὶ ἐλογίσθη αὐτῷ εἰς δικαιοσύνην.
\VS{7}Εἶπε δὲ πρὸς αὐτὸν, ἐγὼ ὁ Θεὸς ὁ ἐξαγαγών σε ἐκ χώρας Χαλδαίων, ὥστε δοῦναί σοι τὴν γῆν ταύτην κληρονομῆσαι.
\VS{8}Εἶπε δέ, Δέσποτα Κύριε, κατὰ τί γνώσομαι, ὅτι κληρονομήσω αὐτήν;
\VS{9}Εἶπε δὲ αὐτῷ, λάβε μοι δάμαλιν τριετίζουσαν, καὶ αἶγα τριετίζουσαν, καὶ κριὸν τριετίζοντα, καὶ τρυγόνα, καὶ περιστεράν.
\VS{10}Ἔλαβε δὲ αὐτῷ πάντα ταῦτα, καὶ διεῖλεν αὐτὰ μέσα, καὶ ἔθηκεν αὐτὰ ἀντιπρόσωπα ἀλλήλοις· τὰ δὲ ὄρνεα οὐ διεῖλε.
\VS{11}Κατέβη δὲ ὄρνεα ἐπὶ τὰ σώματα, ἐπὶ τὰ διχοτομήματα αὐτῶν· καὶ συνεκάθισεν αὐτοῖς Ἅβραμ.
\VS{12}Περὶ δὲ ἡλίου δυσμὰς ἔκστασις ἐπέπεσε τῷ Ἅβραμ, καὶ ἰδοὺ φόβος σκοτεινὸς μέγας ἐπιπίπτει αὐτῷ.
\VS{13}Καὶ ἐῤῥέθη πρὸς Ἅβραμ· γινώσκων γνώσῃ ὅτι πάροικον ἔσται τὸ σπέρμα σου ἐν γῇ οὐκ ἰδίᾳ, καὶ δουλώσουσιν αὐτοὺς, καὶ κακώσουσιν αὐτοὺς, καὶ ταπεινώσουσιν αὐτοὺς, τετρακόσια ἔτη.
\VS{14}Τὸ δὲ ἔθνος, ᾧ ἐὰν δουλεύσωσι, κρινῶ ἐγώ· μετὰ δὲ ταῦτα, ἐξελεύσονται ὧδε μετὰ ἀποσκευῆς πολλῆς.
\VS{15}Σὺ δὲ ἀπελεύσῃ πρὸς τοὺς πατέρας σου ἐν εἰρήνῃ, τραφεὶς ἐν γήρᾳ καλῷ.
\VS{16}Τετάρτῃ δὲ γενεᾷ ἀποστραφήσονται ὧδε· οὔπω γὰρ ἀναπεπλήρωνται αἱ ἁμαρτίαι τῶν Ἀμοῤῥαίων ἕως τοῦ νῦν.
\VS{17}Ἐπεὶ δὲ ὁ ἥλιος ἐγένετο πρὸς δυσμὰς, φλὸξ ἐγένετο· καὶ ἰδοὺ κλίβανος καπνιζόμενος καὶ λαμπάδες πυρός, αἳ διῆλθον ἀνὰ μέσον τῶν διχοτομημάτων τούτων.
\VS{18}Ἐν τῇ ἡμέρᾳ ἐκείνῃ διέθετο Κύριος τῷ Ἅβραμ διαθήκην, λέγων, τῷ σπέρματί σου δώσω τὴν γῆν ταύτην, ἀπὸ τοῦ ποταμοῦ Αἰγύπτου ἕως τοῦ ποταμοῦ τοῦ μεγάλου Εὐφράτου·
\VS{19}Τοὺς Κεναίους, καὶ τοὺς Κενεζαίους, καὶ τοὺς Κεδμωναίους,
\VS{20}καὶ τοὺς Χετταίους, καὶ τοὺς Φερεζαίους, καὶ τοὺς ʼΡαφαεὶν,
\VS{21}καὶ τοὺς Ἀμοῤῥαίους, καὶ τοὺς Χαναναίους, καὶ τοὺς Εὐαίους, καὶ τοὺς Γεργεσαίους, καὶ τοὺς Ἰεβουσαίους.

\par }\Chap{16}{\PP \VerseOne{1}Σάρα δὲ ἡ γυνὴ Ἅβραμ οὐκ ἔτικτεν αὐτῷ· ἦν δὲ αὐτῇ παιδίσκη Αἰγυπτία, ᾗ ὄνομα Ἄγαρ.
\VS{2}Εἶπε δὲ Σάρα πρὸς Ἅβραμ, ἰδοὺ συνέκλεισέ με Κύριος τοῦ μὴ τίκτειν· εἴσελθε οὖν πρὸς τὴν παιδίσκην μου, ἵνα τεκνοποιήσωμαι ἐξ αὐτῆς· ὑπήκουσελ δὲ Ἅβραμ τῆς φωνῆς Σάρας.
\VS{3}Καὶ λαβοῦσα Σάρα ἡ γυνὴ Ἅβραμ Ἄγαρ τὴν Αἰγυπτίαν τὴν ἑαυτῆς παιδίσκην, μετὰ δέκα ἔτη τοῦ οἰκῆσαι Ἅβραμ ἐν γῇ Χαναὰν, ἔδωκεν αὐτὴν τῷ Ἅβραμ ἀνδρὶ αὐτῆς αὐτῷ γυναῖκα.
\VS{4}Καὶ εἰσῆλθε πρὸς Ἄγαρ, καὶ συνέλαβε· καὶ εἶδεν ὅτι ἐν γαστρὶ ἔχει, καὶ ἠτιμάσθη ἡ κυρία ἐναντίον αὐτῆς.
\VS{5}Εἶπε δὲ Σάρα πρὸς Ἅβραμ, ἀδικοῦμαι ἐκ σοῦ· ἐγὼ δέδωκα τὴν παιδίσκην μου εἰς τὸν κόλπον σου, ἰδοῦσα δὲ ὅτι ἐν γαστρὶ ἔχει, ἠτιμάσθην ἐναντίον αὐτῆς. κρίναι ὁ Θεὸς ἀνὰ μέσον ἐμοῦ καὶ σου.
\VS{6}Εἶπε δὲ Ἅβραμ πρὸς Σάραν, ἰδοὺ ἡ παιδίσκη σου ἐν ταῖς χερσί σου, χρῶ αὐτῇ ὡς ἄν σοι ἀρεστὸν ᾖ. καὶ ἐκάκωσεν αὐτὴν Σάρα, καὶ ἀπέδρα ἀπὸ προσώπου αὐτῆς.
\par }{\PP \VS{7}Εὗρε δὲ αὐτὴν ἄγγελος Κυρίου ἐπὶ τῆς πηγῆς τοῦ ὕδατος ἐν τῇ ἐρήμῳ, ἐπὶ τῆς πηγῆς ἐν τῇ ὁδῷ Σούρ.
\VS{8}Καὶ εἶπεν αὐτῇ ὁ ἄγγελος Κυρίου, Ἄγαρ παιδίσκη Σάρας, πόθεν ἔρχῃ; καὶ ποῦ πορεύῃ; καὶ εἶπεν· ἀπὸ προσώπου Σάρας τῆς κυρίας μου ἐγὼ ἀποδιδράσκω.
\VS{9}Εἶπε δὲ αὐτῇ ὁ ἄγγελος Κυρίου, ἀποστράφηθι πρὸς τὴν κυρίαν σου, καὶ ταπεινώθητι ὑπὸ τὰς χεῖρας αὐτῆς.
\VS{10}Καὶ εἶπεν αὐτῇ ὁ ἄγγελος Κυρίου, πληθύνων πληθυνῶ τὸ σπέρμα σου, καὶ οὐκ ἀριθμηθήσεται ὑπὸ τοῦ πλήθους.
\VS{11}Καὶ εἶπεν αὐτῇ ὁ ἄγγελος Κυρίου, ἰδοὺ σὺ ἐν γαστρὶ ἔχεις, καὶ τέξῃ υἱὸν, καὶ καλέσεις τὸ ὄνομα αὐτοῦ Ἰσμαὴλ, ὅτι ἐπήκουσε Κύριος τῇ ταπεινώσει σου.
\VS{12}Οὗτος ἔσται ἄγροικος ἄνθρωπος· αἱ χεῖρες αὐτοῦ ἐπὶ πάντας, καὶ αἱ χεῖρες πάντων ἐπʼ αὐτόν· καὶ κατὰ πρόσωπον πάντων τῶν ἀδελφῶν αὐτοῦ κατοικήσει.
\VS{13}Καὶ ἐκάλεσε τὸ ὄνομα Κυρίου τοῦ λαλοῦντος πρὸς αὐτὴν, σὺ ὁ Θεὸς ὁ ἐτιδών με· ὅτι εἶπε, καὶ γὰρ ἐνώπιον εἶδον ὀφθέντα μοι.
\VS{14}Ἕνεκεν τούτου ἐκάλεσε τὸ φρέαρ, φρέαρ οὗ ἐνώπιον εἶδον· ἰδοὺ ἀνὰ μέσον Κάδης καὶ ἀνὰ μέσον Βαράδ.
\VS{15}Καὶ ἔτεκεν Ἄγαρ τῷ Ἅβραμ υἱὸν, καὶ ἐκάλεσεν Ἅβραμ τὸ ὄνομα τοῦ υἱοῦ αὐτοῦ, ὃν ἔτεκεν αὐτῷ Ἄγαρ, Ἰσμαήλ.
\VS{16}Ἅβραμ δὲ ἦν ἐτῶν ὀγδοηκονταὲξ, ἡνίκα ἔτεκεν Ἄγαρ τῷ Ἅβραμ τὸν Ἰσμαήλ.

\par }\Chap{17}{\PP \VerseOne{1}Ἐγένετο δὲ Ἅβραμ ἐτῶν ἐννενηκονταεννέα. Καὶ ὤφθη Κύριος τῷ Ἅβραμ, καὶ εἶπεν αὐτῷ, ἐγώ εἰμι ὁ Θεός σου· εὐαρέστει ἐνώπιον ἐμοῦ, καὶ γίνου ἄμεμπτος.
\VS{2}Καὶ θήσομαι τὴν διαθήκην μου ἀνὰ μέσον ἐμοῦ, καὶ ἀνὰ μέσον σου, καὶ πληθυνῶ σε σφόδρα.
\VS{3}Καὶ ἔπεσεν Ἅβραμ ἐπὶ πρόσωπον αὐτοῦ.
\VS{4}Καὶ ἐλάλησεν αὐτῷ ὁ Θεὸς, λέγων, Καὶ ἐγὼ ἰδοὺ ἡ διαθήκη μου μετὰ σοῦ· καὶ ἔσῃ πατὴρ πλήθους ἐθνῶν.
\VS{5}Καὶ οὐ κληθήσεται ἔτι τὸ ὄνομά σου Ἅβραμ, ἀλλʼ ἔσται τὸ ὄνομά σου Ἁβραὰμ, ὅτι πατέρα πολλῶν ἐθνῶν τέθεικά σε.
\VS{6}Καὶ αὐξανῶ σε σφόδρα σφόδρα, καὶ θήσω σε εἰς ἔθνη· καὶ βασιλεῖς ἐκ σοῦ ἐξελεύσονται.
\VS{7}Καὶ στήσω τὴν διαθήκην μου ἀνὰ μέσον σου, καὶ ἀνὰ μέσον τοῦ σπέρματός σου μετὰ σὲ εἰς τὰς γενεὰς αὐτῶν, εἰς διαθήκην αἰώνιον εἶναί σου Θεὸς, καὶ τοῦ σπέρματός σου μετὰ σέ.
\VS{8}Καὶ δώσω σοι καὶ τῷ σπέρματί σου μετὰ σὲ τὴν γῆν, ἣν παροικεῖς, πᾶσαν τὴν γῆν Χαναὰν, εἰς κατάσχεσιν αἰώνιον· καὶ ἔσομαι αὐτοῖς εἰς Θεόν.
\VS{9}Καὶ εἶπεν ὁ Θεὸς πρὸς Ἁβραὰμ, σὺ δὲ τὴν. διαθήκην μου διατηρήσεις, σὺ καὶ τὸ σπέρμα σου μετὰ σὲ εἰς τὰς γενεὰς αὐτῶν.
\VS{10}Καὶ αὕτη ἡ διαθήκη, ἣν διατηρήσεις, ἀνὰ μέσον ἐμοῦ καὶ ὑμῶν, καὶ ἀνὰ μέσον τοῦ σπέρματός σου μετὰ σὲ εἰς τὰς γενεὰς αὐτῶν· περιτμηθήσεται ὑμῶν πᾶν ἀρσενικόν.
\VS{11}Καὶ περιτμηθήσεσθε τὴν σάρκα τῆς ἀκροβυστίας ὑμῶν, καὶ ἔσται εἰς σημεῖον διαθήκης ἀνὰ μέσον ἐμοῦ καὶ ὑμῶν.
\VS{12}Καὶ παιδίον ὀκτὼ ἡμερῶν περιτμηθήσεται ὑμῖν, πᾶν ἀρσενικὸν εἰς τὰς γενεὰς ὑμῶν· καὶ οἰκογενὴς καὶ ὁ ἀργυρώνητος ἀπὸ παντὸς υἱοῦ ἀλλοτρίου, ὃς οὐκ ἔστιν ἐκ τοῦ σπέρματός σου,
\VS{13}Περιτομῇ περιτμηθήσεται ὁ οἰκογενὴς τῆς οἰκίας σου, καὶ ὁ ἀργυρώνητος· καὶ ἔσται ἡ διαθήκη μου ἐπὶ τῆς σαρκὸς ὑμῶν εἰς διαθήκην αἰώνιον.
\VS{14}Καὶ ἀπερίτμητος ἄρσην, ὃς οὐ περιτμηθήσεται τὴν σάρκα τῆς ἀκροβυστίας αὐτοῦ τῇ ἡμέρᾳ τῇ ὀγδόῃ, ἐξολοθρευθήσεται ἡ ψυχὴ ἐκείνη ἐκ τοῦ γένους αὐτῆς, ὅτι τὴν διαθήκην μου διεσκέδασε.
\VS{15}Καὶ εἶπεν ὁ Θεὸς τῷ Ἁβραὰμ, Σάρα ἡ γυνή σου, οὐ κληθήσεται τὸ ὄνομα αὐτῆς Σάρα, Σάῤῥα ἔσται τὸ ὄνομα αὐτῆς.
\VS{16}Εὐλογήσω δὲ αὐτὴν, καὶ δώσω σοι ἐξ αὐτῆς τέκνον, καὶ εὐλογήσω αὐτὸ, καὶ ἔσται εἰς ἔθνη, καὶ βασιλεῖς ἐθνῶν ἐξ αὐτοῦ ἔσονται.
\VS{17}Καὶ ἔπεσεν Ἁβραὰμ ἐπὶ πρόσωπον αὐτοῦ, καὶ ἐγέλασε· καὶ εἶπεν ἐν τῇ διανοίᾳ αὐτοῦ, λέγων, εἰ τῷ ἑκατονταετεῖ γενήσεται υἱός; καὶ εἰ ἡ Σάῤῥα ἐννενήκοντα ἐτῶν τέξεται;
\VS{18}Εἶπε δὲ Ἁβραὰμ πρὸς τὸν Θεόν· Ἰσμαὴλ οὗτος ζήτω ἐναντίον σου.
\VS{19}Εἶπε δὲ ὁ Θεὸς πρὸς Ἁβραὰμ, ναί· ἰδοὺ Σάῤῥα ἡ γυνή σου τέξεταί σοι υἱὸν, καὶ καλέσεις τὸ ὄνομα αὐτοῦ Ἰσαάκ· καὶ στήσω τὴν διαθήκην μου πρὸς αὐτὸν, εἰς διαθήκην αἰώνιον, εἶναι αὐτῷ Θεὸς καὶ τῷ σπέρματι αὐτοῦ μετʼ αὐτόν.
\VS{20}Περὶ δὲ Ἰσμαὴλ ἰδοὺ ἐπήκουσά σου· καὶ ἰδοὺ εὐλόγηκα αὐτὸν, καὶ αὐξανῶ αὐτὸν, καὶ πληθυνῶ αὐτὸν σφόδρα δώδεκα ἔθνη γεννήσει, καὶ δώσω αὐτὸν εἰς ἔθνος μέγα.
\VS{21}Τὴν δὲ διαθήκην μου στήσω πρὸς Ἰσαὰκ, ὃν τέξεταί σοι Σάῤῥα εἰς τὸν καιρὸν τοῦτον, ἐν τῷ ἐνιαυτῷ τῷ ἑτέρῳ.
\VS{22}Συνετέλεσε δὲ λαλῶν πρὸς αὐτὸν, καὶ ἀνέβη ὁ Θεὸς ἀπὸ Ἁβραάμ.
\par }{\PP \VS{23}Καὶ ἔλαβεν Ἁβραὰμ Ἰσμαὴλ τὸν υἱὸν ἑαυτοῦ, καὶ πάντας τοὺς οἰκογενεῖς αὐτοῦ, καὶ πάντας τοὺς ἀργυρωνήτους, καὶ πᾶν ἄρσεν τῶν ἀνδρῶν τῶν ἐν τῷ οἴκῳ Ἁβραὰμ, καὶ περιέτεμε τὰς ἀκροβυστίας αὐτῶν, ἐν τῷ καιρῷ τῆς ἡμέρας ἐκείνης, καθὰ ἐλάλησεν αὐτῷ ὁ Θεός.
\VS{24}Ἁβραὰμ δὲ ἐννενηκονταεννέα ἦν ἐτῶν, ἡνίκα περιετέμετο τὴν σάρκα τῆς ἀκροβυστίας αὐτοῦ.
\VS{25}Ἰσμαὴλ δὲ ὁ υἱὸς αὐτοῦ ἦν ἐτῶν δεκατριῶν, ἡνίκα περιετέμετο τὴν σάρκα τῆς ἀκροβυστίας αὐτοῦ.
\VS{26}Ἐν δὲ τῷ καιρῷ τῆς ἡμέρας ἐκείνης, περιετμήθη Ἁβραὰμ, καὶ Ἰσμαὴλ ὁ υἱὸς αὐτοῦ,
\VS{27}καὶ πάντες οἱ ἄνδρες τοῦ οἴκου αὐτοῦ, καὶ οἱ οἰκογενεῖς αὐτοῦ, καὶ οἱ ἀργυρώνητοι ἐξ ἀλλογενῶν ἐθνῶν.

\par }\Chap{18}{\PP \VerseOne{1}Ὤφθη δὲ αὐτῷ ὁ Θεὸς πρὸς τῇ δρυῒ τῇ Μαμβρῇ, καθημένου αὐτοῦ ἐπὶ τῆς θύρας τῆς σκηνῆς αὐτοῦ μεσημβρίας.
\VS{2}Ἀναβλέψας δὲ τοῖς ὀφθαλμοῖς αὐτοῦ εἶδε, καὶ ἰδοὺ τρεῖς ἄνδρες εἱστήκεισαν ἐπάνω αὐτοῦ· καὶ ἰδὼν, προσέδραμεν εἰς συνάντησιν αὐτοῖς ἀπὸ τῆς θύρας τῆς σκηνῆς αὐτοῦ, καὶ προσεκύνησεν ἐπὶ τὴν γῆν.
\VS{3}Καὶ εἶπε, Κύριε, εἰ ἄρα εὗρον χάριν ἐναντίον σου, μὴ παρέλθῃς τὸν παῖδά σου.
\VS{4}Ληφθήτω δὴ ὕδωρ, καὶ νιψάτωσαν τοὺς πόδας ὑμῶν, καὶ καταψύξατε ὑπὸ τὸ δένδρον.
\VS{5}Καὶ λήψομαι ἄρτον, καὶ φάγεσθε. Καὶ μετὰ τοῦτο παρελεύσεσθε εἰς τὴν ὁδὸν ὑμῶν, οὗ ἕνεκεν ἐξεκλίνατε πρὸς τὸν παῖδα ὑμῶν. Καὶ εἶπεν, οὕτω ποίησον, καθὼς εἴρηκας.
\VS{6}Καὶ ἔσπευσεν Ἁβραὰμ ἐπὶ τὴν σκηνὴν πρὸς Σάῤῥαν, καὶ εἶπεν αὐτῇ, σπεῦσον, καὶ φύρασον τρία μέτρα σεμιδάλεως, καὶ ποίησον ἐγκρυφίας.
\VS{7}Καὶ εἰς τὰς βόας ἔδραμεν Ἁβραὰμ, καὶ ἔλαβεν ἁπαλὸν μοσχάριον καὶ καλὸν, καὶ ἔδωκε τῷ παιδὶ, καὶ ἐτάχυνε τοῦ ποιῆσαι αὐτό.
\VS{8}Ἔλαβε δὲ βούτυρον, καὶ γάλα, καὶ τὸ μοσχάριον ὃ ἐποίησε, καὶ παρέθηκεν αὐτοῖς, καὶ ἔφαγον· αὐτὸς δὲ παρειστήκει αὐτοῖς ὑπὸ τὸ δένδρον.
\par }{\PP \VS{9}Εἶπε δὲ πρὸς αὐτὸν, ποῦ Σάῤῥα ἡ γυνή σου; ὁ δὲ ἀποκριθεὶς εἶπεν, ἰδοὺ ἐν τῇ σκηνῇ.
\VS{10}Εἶπε δὲ, ἐπαναστρέφων ἥξω πρὸς σὲ κατὰ τὸν καιρὸν τοῦτον εἰς ὥρας, καὶ ἕξει υἱὸν Σάῤῥα ἡ γυνή σου. Σάῤῥα δὲ ἤκουσε πρὸς τῇ θύρᾳ τῆς σκηνῆς οὖσα ὄπισθεν αὐτοῦ.
\VS{11}Ἁβραὰμ δὲ καὶ Σάῤῥα πρεσβύτεροι προβεβηκότες ἡμερῶν· ἐξέλιπε δὲ τῇ Σάῤῥᾳ γίνεσθαι τὰ γυναικεια.
\VS{12}Ἐγέλασε δὲ Σάῤῥα ἐν ἑαυτῇ λέγουσα, οὔπω μέν μοι γέγονεν ἕως τοῦ νῦν· ὁ δὲ κύριός μου πρεσβύτερος.
\VS{13}Καὶ εἶπε Κύριος πρὸς Ἁβραὰμ, τί ὅτι ἐγέλασε Σάῤῥα ἐν ἑαυτῇ, λέγουσα, ἆρά γε ἀληθῶς τέξομαι; ἐγὼ δὲ γεγήρακα.
\VS{14}Μὴ ἀδυνατήσει παρὰ τῷ Θεῷ ῥῆμα; εἰς τὸν καιρὸν τοῦτον ἀναστρέψω πρὸς σὲ εἰς ὥρας, καὶ ἔσται τῇ Σάῤῥᾳ υἱός.
\VS{15}Ἠρνήσατο δὲ Σάῤῥα, λέγουσα, οὐκ ἐγέλασα· ἐφοβήθη γάρ. Καὶ εἶπεν αὐτῇ, οὐχὶ, ἀλλὰ ἐγέλασας.
\par }{\PP \VS{16}Ἐξαναστάντες δὲ ἐκεῖθεν οἱ ἄνδρες κατέβλεψαν ἐπὶ πρόσωπον Σοδόμων καὶ Γομόῤῥας. Ἁβραὰμ δὲ συνεπορεύετο μετʼ αὐτῶν, συμπροπέμπων αὐτούς.
\VS{17}Ὁ δὲ Κύριος εἶπε, οὐ μὴ κρύψω ἐγὼ ἀπὸ Ἁβραὰμ τοῦ παιδός μου ἃ ἐγὼ ποιῶ.
\VS{18}Ἁβραὰμ δὲ γινόμενος ἔσται εἰς ἔθνος μέγα καὶ πολὺ, καὶ ἐνευλογηθήσονται ἐν αὐτῷ πάντα τὰ ἔθνη τῆς γῆς.
\VS{19}Ἤδειν γὰρ ὅτι συντάξει τοῖς υἱοῖς αὐτοῦ, καὶ τῷ οἴκῳ αὐτοῦ μετʼ αὐτὸν, καὶ φυλάξουσι τὰς ὁδοὺς Κυρίου, ποιεῖν δικαιοσύνην καὶ κρίσιν, ὅπως ἂν ἐπαγάγῃ Κύριος ἐπὶ Ἁβραὰμ πάντα ὅσα ἐλάλησε πρὸς αὐτόν.
\VS{20}Εἶπε δὲ Κύριος, κραυγὴ Σοδόμων καὶ Γομόῤῥας πεπλήθυνται πρὸς μὲ, καὶ αἱ ἁμαρτίαι αὐτῶν μεγάλαι σφόδρα.
\VS{21}Καταβὰς οὖν ὄψομαι, εἰ κατὰ τὴν κραυγὴν αὐτῶν τὴν ἐρχομενεην πρὸς μὲ, συντελοῦνται· εἰ δὲ μὴ, ἵνα γνῶ.
\VS{22}Καὶ ἀποστρέψαντες ἐκεῖθεν οἱ ἄνδρες, ἦλθον εἰς Σόδομα· Ἁβραὰμ δὲ ἔτι ἦν ἑστηκὼς ἐναντίον Κυρίου.
\VS{23}Καὶ ἐγγίσας Ἁβραὰμ, εἶπε, μὴ συναπολέσῃς δίκαιον μετὰ ἀσεβοῦς· καὶ ἔσται ὁ δίκαιος ὡς ὁ ἀσεβής.
\VS{24}Ἐὰν ὦσι πεντήκοντα δίκαιοι ἐν τῇ πόλει, ἀπολεῖς αὐτούς; οὐκ ἀνήσεις πάντα τὸν τόπον ἕνεκεν τῶν πεντήκοντα δικαίων, ἐὰν ὦσιν ἐν αὐτῇ;
\VS{25}Μηδαμῶς σὺ ποιήσεις ὡς τὸ ῥῆμα τοῦτο, τοῦ ἀποκτεῖναι δίκαιον μετὰ ἀσεβοῦς, καὶ ἔσται ὁ δίκαιος ὡς ὁ ἀσεβής· μηδαμῶς· ὁ κρίνων πᾶσαν τὴν γῆν, οὐ ποιήσεις κρίσιν;
\VS{26}Εἶπε δὲ Κύριος, ἐὰν ὦσιν ἐν Σοδόμοις πεντήκοντα δίκαιοι ἐν τῇ πόλει, ἀφήσω ὅλην τὴν πόλιν, καὶ πάντα τὸν τόπον διʼ αὐτούς.
\VS{27}Καὶ ἀποκριθεὶς Ἁβραὰμ εἶπε, νῦν ἠρξάμην λαλῆσαι πρὸς τὸν Κύριόν μου· ἐγὼ δὲ εἰμὶ γῆ καὶ σποδός.
\VS{28}Ἐὰν δὲ ἐλαττονωθῶσιν οἱ πεντήκοντα δίκαιοι εἰς τεσσαρακονταπέντε, ἀπολεῖς ἕνεκεν τῶν πέντε πᾶσαν τὴν πόλιν; καὶ εἶπεν, οὐ μὴ ἀπολέσω, ἐὰν εὕρω ἐκεῖ τεσσαρακονταπέντε.
\VS{29}Καὶ προσέθηκεν ἔτι λαλῆσαι πρὸς αὐτὸν, καὶ εἶπεν, ἐὰν δὲ εὑρεθῶσιν ἐκεῖ τεσσαράκοντα· καὶ εἶπεν, οὐ μὴ ἀπολέσω ἕνεκεν τῶν τεσσαράκοντα.
\VS{30}Καὶ εἶπε, μή τι Κύριε ἐὰν λαλήσω; ἐὰν δὲ εὑρεθῶσιν ἐκεῖ τριάκοντα; καὶ εἶπεν, οὐ μὴ ἀπολέσω ἕνεκεν τῶν τριάκοντα.
\VS{31}Καὶ εἶπεν, ἐπειδὴ ἔχω λαλῆσαι πρὸς τὸν Κύριον, ἐὰν δὲ εὑρεθῶσιν ἐκεῖ εἴκοσι; καὶ εἶπεν, οὐ μὴ ἀπολέσω, ἐὰν εὕρω ἐκεῖ εἴκοσι.
\VS{32}Καὶ εἶπε, μή τι Κύριε ἐὰν λαλήσω ἔτι ἅπαξ; ἐὰν δὲ εὑρεθῶσιν ἐκεῖ δέκα; καὶ εἶπεν, οὐ μὴ ἀπολέσω ἕνεκεν τῶν δέκα.
\VS{33}Ἀπῆλθε δὲ ὁ Κύριος, ὡς ἐπαύσατο λαλῶν τῷ Ἁβραάμ· καὶ Ἁβραὰμ ἀπέστρεψεν εἰς τὸν τόπον αὐτοῦ.

\par }\Chap{19}{\PP \VerseOne{1}Ἦλθον δε οἱ δύο ἄγγελοι εἰς Σόδομα ἑσπέρας. Λὼτ δὲ ἐκάθητο παρὰ τὴν πύλην Σοδόμων· ἰδὸν δὲ Λὼτ, ἐξανέστη εἰς συνάντησιν αὐτοῖς, καὶ προσεκύνησε τῷ προσώπῳ ἐπὶ τὴν γῆν.
\VS{2}Καὶ εἶπεν, ἰδοὺ, Κύριοι, ἐκκλίνατε εἰς τὸν οἶκον τοῦ παιδὸς ὑμῶν, καὶ καταλύσατε, καὶ νίψασθε τοὺς πόδας ὑμῶν, καὶ ὀρθρίσαντες ἀπελεύσεσθε εἰς τὴν ὁδὸν ὑμῶν. Καὶ εἶπαν, οὐχὶ, ἀλλʼ ἐν τῇ πλατείᾳ καταλύσομεν.
\VS{3}Καὶ κατεβιάσατο αὐτοὺς, καὶ ἐξέκλιναν πρὸς αὐτὸν, καὶ εἰσῆλθον εἰς τὸν οἶκον αὐτοῦ· καὶ ἐποίησεν αὐτοῖς πότον, καὶ ἀζύμους ἔπεψεν αὐτοῖς, καὶ ἔφαγον.
\VS{4}Πρὸ τοῦ κοιμηθῆναι δὲ, οἱ ἄνδρες τῆς πόλεως, οἱ Σοδομῖται περιεκύκλωσαν τὴν οἰκίαν, ἀπὸ νεανίσκου ἕως πρεσβυτέρου, ἅπας ὁ λαὸς ἅμα.
\VS{5}Καὶ ἐξεκαλοῦντο τὸν Λὼτ, καὶ ἔλεγον πρὸς αὐτὸν, ποῦ εἰσιν οἱ ἄνδρες οἱ εἰσελθόντες πρὸς σὲ τὴν νύκτα; ἐξάγαγε αὐτοὺς πρὸς ἡμᾶς, ἵνα συγγενώμεθα αὐτοῖς.
\VS{6}Ἐξῆλθε δὲ Λὼτ πρὸς αὐτοὺς πρὸς τὸ πρόθυρον, τὴν δὲ θύραν προσέῳξεν ὀπίσω αὐτοῦ.
\VS{7}Εἶπε δὲ πρὸς αὐτοὺς, μηδαμῶς ἀδελφοὶ μὴ πονηρεύσησθε.
\VS{8}Εἰσὶ δέ μοι δύο θυγατέρες, αἳ οὐκ ἔγνωσαν ἄνδρα· ἐξάξω αὐτὰς πρὸς ὑμᾶς, καὶ χρᾶσθε αὐταῖς καθὰ ἂν ἀρέσκοι ὑμῖν· μόνον εἰς τοὺς ἄνδρας τούτους μὴ ποιήσητε ἄδικον, οὗ εἵνεκεν εἰσῆλθον ὑπὸ τὴν σκέπην τῶν δοκῶν μου.
\VS{9}Εἶπαν δὲ αὐτῷ, ἀπόστα ἐκεῖ· εἰσῆλθες παροικεῖν, μὴ καὶ κρίσιν κρίνειν; νῦν οὖν σε κακώσωμεν μᾶλλον ἢ ἐκείνους. Καὶ παρεβιάζοντο τὸν ἄνδρα τὸν Λὼτ σφόδρα, καὶ ἤγγισαν συντρίψαι τὴν θύραν.
\VS{10}Ἐκτείναντες δὲ οἱ ἄνδρες τὰς χεῖρας εἰσεσπάσαντο τὸν Λὼτ πρὸς ἑαυτοὺς εἰς τὸν οἶκον, καὶ τὴν θύραν τοῦ οἴκου ἀπέκλεισαν.
\VS{11}Τοὺς δὲ ἄνδρας τοὺς ὄντας ἐπὶ τῆς θύρας τοῦ οἴκου ἐπάταξαν ἐν ἀορασίᾳ ἀπὸ μικροῦ ἕως μεγάλου· καὶ παρελύθησαν ζητοῦντες τὴν θύραν.
\VS{12}Εἶπαν δὲ οἱ ἄνδρες πρὸς τὸν Λὼτ, εἰσί σοι ὧδε γαμβροὶ, ἢ υἱοὶ, ἢ θυγατέρες; ἢ εἴτις σοι ἄλλος ἐστὶν ἐν τῇ πόλει, ἐξάγαγε ἐκ τοῦ τόπου τούτου,
\VS{13}Ὅτι ἡμεῖς ἀπόλλυμεν τὸν τόπον τοῦτον· ὅτι ὑψώθη ἡ κραυγὴ αὐτῶν ἔναντι Κυρίου, καὶ ἀπέστειλεν ἡμᾶς Κύριος ἐκτρίψαι αὐτήν.
\VS{14}Ἐξῆλθε δὲ Λῶτ, καὶ ἐλάλησε πρὸς τοὺς γαμβροὺς αὐτοῦ τοὺς εἰληφότας τὰς θυγατέρας αὐτοῦ, καὶ εἶπεν, ἀνάστητε, καὶ ἐξέλθετε ἐκ τοῦ τόπου τούτου, ὅτι ἐκτρίβει Κύριος τὴν πόλιν· ἔδοξε δὲ γελοιάζειν ἐναντίον τῶν γαμβρῶν αὐτοῦ.
\VS{15}Ἡνίκα δὲ ὄρθρος ἐγένετο, ἐσπούδαζον οἱ ἄγγελοι τὸν Λὼτ, λέγοντες, ἀναστὰς λάβε τὴν γυναῖκά σου, καὶ τὰς δύο θυγατέρας σου, ἃς ἔχεις, καὶ ἔξελθε, ἵνα μὴ καὶ σὺ συναπόλῃ ταῖς ἀνομίαις τῆς πόλεως.
\VS{16}Καὶ ἐταράχθησαν, καὶ ἐκράτησαν οἱ ἄγγελοι τῆς χειρὸς αὐτοῦ, καὶ τῆς χειρὸς τῆς γυναικὸς αὐτοῦ, καὶ τῶν χειρῶν τῶν δύο θυγατέρων αὐτοῦ, ἐν τῷ φείσασθαι Κύριον αὐτοῦ.
\par }{\PP \VS{17}Καὶ ἐγένετο ἡνίκα ἐξήγαγον αὐτοὺς ἔξω, καὶ εἶπαν, σώζων σῶζε τὴν σεαυτοῦ ψυχήν· μὴ περιβλέψῃ εἰς τὰ ὀπίσω, μηδὲ στῇς ἐν πάσῃ τῇ περιχώρῳ· εἰς τὸ ὄρος σώζου, μή ποτε συμπαραληφθῇς.
\VS{18}Εἶπε δὲ Λὼτ πρὸς αὐτοὺς, δέομαι
\VS{19}Κύριε, ἐπειδὴ εὗρεν ὁ παῖς σου ἔλεος ἐναντίον σου, καὶ ἐμεγάλυνας τὴν δικαιοσύνην σου, ὃ ποιεῖς ἐπʼ ἐμὲ, τοῦ ζῆν τὴν ψυχήν μου· ἐγὼ δὲ οὐ δυνήσομαι διασωθῆναι εἰς τὸ ὄρος, μή ποτε καταλάβῃ με τὰ κακὰ, καὶ ἀποθάνω.
\VS{20}Ἰδοὺ πόλις αὕτη ἐγγὺς τοῦ καταφυγεῖν με ἐκεῖ, ἥ ἐστι μικρά· καὶ ἐκεῖ διασωθήσομαι· οὐ μικρά ἐστι; καὶ ζήσεται ἡ ψυχή μου ἕνεκέν σου.
\VS{21}Καὶ εἶπεν αὐτῷ, ἰδοὺ ἐθαύμασά σου τὸ πρόσωπον καὶ ἐπὶ τῷ ῥήματι τούτῳ, τοῦ μὴ καταστρέψαι τὴν πόλιν περὶ ἧς ἐλάλησας.
\VS{22}Σπεῦσον οὖν τοῦ σωθῆναι ἐκεῖ, οὐ γὰρ δυνήσομαι ποιῆσαι πρᾶγμα, ἕως τοῦ ἐλθεῖν σε ἐκεῖ. διὰ τοῦτο ἐκάλεσε τὸ ὄνομα τῆς πόλεως ἐκείνης, Σηγώρ.
\VS{23}Ὁ ἥλιος ἐξῆλθεν ἐπὶ τὴν γῆν, καὶ Λὼτ εἰσῆλθεν εἰς Σηγώρ.
\VS{24}Καὶ Κύριος ἔβρεξεν ἐπὶ Σόδομα καὶ Γόμοῤῥα θεῖον καὶ πῦρ παρὰ Κυρίου ἐξ οὐρανοῦ.
\VS{25}Καὶ κατέστρεψε τὰς πόλεις ταύτας, καὶ πᾶσαν τὴν περίχωρον, καὶ πάντας τοὺς κατοικοῦντας ἐν ταῖς πόλεσι, καὶ τὰ ἀνατέλλοντα ἐκ τῆς γῆς.
\VS{26}Καὶ ἐπέβλεψεν ἡ γυνὴ αὐτοῦ εἰς τὰ ὀπίσω, καὶ ἐγένετο στήλη ἁλός.
\VS{27}Ὤρθρισε δὲ Ἁβραὰμ τῷ πρωῒ εἰς τὸν τόπον, οὗ εἱστήκει ἐναντίον Κυρίου.
\VS{28}Καὶ ἐπέβλεψεν ἐπὶ πρόσωπον Σοδόμων καὶ Γομόῤῥας, καὶ ἐπὶ πρόσωπον τῆς περιχώρου, καὶ εἶδε, καὶ ἰδοὺ ἀνέβαινεν φλὸξ ἐκ τῆς γῆς, ὡσεὶ ἀτμὶς καμίνου.
\VS{29}Καὶ ἐγένετο ἐν τῷ ἐκτρίψαι τὸν Θεὸν πάσας τὰς πόλεις τῆς περιοίκου, ἐμνήσθη ὁ Θεὸς τοῦ Ἁβραάμ· καὶ ἐξαπέστειλε τὸν Λὼτ ἐκ μέσου τῆς καταστροφῆς, ἐν τῷ καταστρέψαι Κύριον τὰς πόλεις, ἐν αἷς κατῴκει ἐν αὐταῖς Λώτ.
\par }{\PP \VS{30}Ἀνέβη δὲ Λὼτ ἐκ Σηγὼρ, καὶ ἐκάθητο ἐν τῷ ὄρει αὐτὸς, καὶ αἱ δύο θυγατέρες αὐτοῦ μετʼ αὐτοῦ· ἐφοβήθη γὰρ κατοικῆσαι ἐν Σηγώρ· καὶ κατῴκησεν ἐν τῷ σπηλαίῳ αὐτὸς, καὶ αἱ δύο θυγατέρες αὐτοῦ μετʼ αὐτοῦ.
\VS{31}Εἶπε δὲ ἡ πρεσβυτέρα πρὸς τὴν νεωτέραν, ὁ πατὴρ ἡμῶν πρεσβύτερος, καὶ οὐδείς ἐστιν ἐπὶ τῆς γῆς, ὃς εἰσελεύσεται πρὸς ἡμᾶς, ὡς καθήκει πάσῃ τῇ γῇ.
\VS{32}Δεῦρο καὶ ποτίσωμεν τὸν πατέρα ἡμῶν οἶνον, καὶ κοιμηθῶμεν μετʼ αὐτοῦ, καὶ ἐξαναστήσωμεν ἐκ τοῦ πατρὸς ἡμῶν σπέρμα.
\VS{33}Ἐπότισαν δὲ τὸν πατέρα αὐτῶν οἶνον ἐν τῇ νυκτὶ ἐκείνῃ, καὶ εἰσελθοῦσα ἡ πρεσβυτέρα ἐκοιμήθη μετὰ τοῦ πατρὸς αὐτῆς ἐν τῇ νυκτὶ ἐκείνῃ· καὶ οὐκ ᾔδει ἐν τῷ κοιμηθῆναι αὐτὸν, καὶ ἐν τῷ ἀναστῆναι.
\VS{34}Ἐγένετο δὲ ἐν τῇ ἐπαύριον, καὶ εἶπεν ἡ πρεσβυτέρα πρὸς τὴν νεωτέραν, ἰδοὺ ἐκοιμήθην χθὲς μετὰ τοῦ πατρὸς ἡμῶν· ποτίσωμεν αὐτὸν οἶνον καὶ ἐν τῇ νυκτὶ ταύτῃ, καὶ εἰσελθοῦσα κοιμήθητι μετʼ αὐτοῦ, καὶ ἐξαναστήσωμεν ἐκ τοῦ πατρὸς ἡμῶν σπέρμα.
\VS{35}Ἐπότισαν δὲ καὶ ἐν τῇ νυκτὶ ἐκείνῃ τὸν πατέρα αὐτῶν οἶνον, καὶ εἰσελθοῦσα ἡ νεωτέρα ἐκοιμήθη μετὰ τοῦ πατρὸς αὐτῆς· καὶ οὐκ ᾔδει ἐν τῷ κοιμηθῆναι αὐτὸν, καὶ ἀναστῆναι.
\VS{36}Καὶ συνέλαβον αἱ δύο θυγατέρες Λὼτ ἐκ τοῦ πατρὸς αὐτῶν.
\VS{37}Καὶ ἔτεκεν ἡ πρεσβυτέρα υἱὸν, καὶ ἐκάλεσε τὸ ὄνομα αὐτοῦ Μωὰβ, λέγουσα, ἐκ τοῦ πατρός μου· οὗτος πατὴρ Μωαβιτῶν ἕως τῆς σήμερον ἡμέρας.
\VS{38}Ἔτεκε δὲ καὶ ἡ νεωτέρα υἱὸν, καὶ ἐκάλεσε τὸ ὄνομα αὐτοῦ Ἀμμὰν, λέγουσα, υἱὸς γένους μου· οὗτος πατὴρ Ἀμμανιτῶν ἕως τῆς σήμερον ἡμέρας.

\par }\Chap{20}{\PP \VerseOne{1}Καὶ ἐκίνησεν ἐκεῖθεν Ἁβραὰμ εἰς γῆν πρὸς Λίβα· καὶ ᾤκησεν ἀνὰ μέσον Κάδης, καὶ ἀνὰ μέσον Σούρ· καὶ παρῴκησεν ἐν Γεράροις.
\VS{2}Εἶπε δὲ Ἁβραὰμ περὶ Σάῤῥας τῆς γυναικὸς αὐτοῦ, ὅτι ἀδελφή μου ἐστίν· ἐφοβήθη γὰρ εἰπεῖν ὅτι γυνή μου ἐστὶ, μή ποτε ἀποκτείνωσιν αὐτὸν οἱ ἄνδρες τῆς πόλεως διʼ αὐτήν· ἀπέστειλε δὲ Ἀβιμέλεχ βασιλεὺς Γεράρων, καὶ ἔλαβε τὴν Σάῤῥαν.
\VS{3}Καὶ εἰσῆλθεν ὁ Θεὸς πρὸς Ἀβιμέλεχ ἐν ὕπνῳ τὴν νύκτα, καὶ εἶπεν, ἰδοὺ σὺ ἀποθνήσκεις περὶ τῆς γυναικὸς, ἧς ἔλαβες· αὕτη δέ ἐστι συνῳκηκυῖα ἀνδρί.
\VS{4}Ἀβιμέλεχ δὲ οὐχ ἥψατο αὐτῆς· καὶ εἶπε, Κύριε, ἔθνος ἀγνοοῦν καὶ δίκαιον ἀπολεῖς;
\VS{5}Οὐκ αὐτός μοι εἶπεν, ἀδφή μου ἐστί; καὶ αὕτη μοι εἶπεν, ἀδελφός μου ἐστίν; ἐν καθαρᾷ καρδίᾳ καὶ ἐν δικαιοσύνῃ χειρῶν ἐποίησα τοῦτο.
\VS{6}Εἶπε δὲ αὐτῷ ὁ Θεὸς καθʼ ὕπνον, κᾀγὼ ἔγνων ὅτι ἐν καθαρᾷ καρδίᾳ ἐποίησας τοῦτο, καὶ ἐφεισάμην σου τοῦ μὴ ἁμαρτεῖν σε εἰς ἐμέ· ἕνεκα τούτου οὐκ ἀφῆκά σε ἅψασθαι αὐτῆς.
\VS{7}Νῦν δὲ ἀπόδος τὴν γυναῖκα τῷ ἀνθρώπῳ, ὅτι προφήτης ἐστὶ, καὶ προσεύξεται περὶ σοῦ, καὶ ζήσῃ· εἰ δὲ μὴ ἀποδίδως, γνώσῃ ὅτι ἀποθανῇ σὺ καὶ πάντα τὰ σὰ.
\VS{8}Καὶ ὤρθρισεν Ἀβιμέλεχ τῷ πρωῒ, καὶ ἐκάλεσε πάντας τοὺς παῖδας αὐτοῦ, καὶ ἐλάλησε πάντα τὰ ῥήματα ταῦτα εἰς τὰ ὦτα αὐτῶν· ἐφοβήθησαν δὲ πάντες οἱ ἄνθρωποι σφόδρα.
\VS{9}Καὶ ἐκάλεσεν Ἀβιμέλεχ τὸν Ἁβραὰμ καὶ εἶπεν αὐτῷ, τί τοῦτο ἐποίησας ἡμῖν; μήτι ἡμάρτομεν εἰς σὲ, ὅτι ἐπήγαγες ἐπʼ ἐμὲ καὶ ἐπὶ τὴν βασιλείαν μου ἁμαρτίαν μεγάλην; ἔργον ὃ οὐδεὶς ποιήσει, πεποίηκάς μοι.
\VS{10}Εἶπε δὲ Ἀβιμέλεχ τῷ Ἁβραὰμ, τί ἐνιδὼν ἐποίησας τοῦτο;
\VS{11}Εἶπε δὲ Ἁβραὰμ, εἶπα γὰρ, ἄρα οὐκ ἔστι θεοσέβεια ἐν τῷ τόπῳ τούτῳ, ἐμέ τε ἀποκτενοῦσιν ἕνεκεν τῆς γυναικός μου.
\VS{12}Καὶ γὰρ ἀληθῶς, ἀδελφή μου ἐστὶν ἐκ πατρὸς, ἀλλʼ οὐκ ἐκ μητρός· ἐγενήθη δέ μοι εἰς γυναῖκα.
\VS{13}Ἐγένετο δὲ ἡνίκα ἐξήγαγέ με ὁ Θεὸς ἐκ τοῦ οἴκου τοῦ πατρός μου, καὶ εἶπα αὐτῇ, ταύτην τὴν δικαιοσύνην ποιήσεις εἰς ἐμὲ, εἰς πάντα τόπον οὗ ἐὰν εἰσέλθωμεν ἐκεῖ, εἶπον ἐμὲ, ὅτι ἀδελφός μου ἐστίν.
\VS{14}Ἔλαβε δὲ Ἀβιμέλεχ χίλια δίδραγμα, καὶ πρόβατα, καὶ μόσχους, καὶ παῖδας, καὶ παιδίσκας, καὶ ἔδωκε τῷ Ἁβραάμ· καὶ ἀπέδωκεν αὐτῷ Σάῤῥαν τὴν γυναῖκα αὐτοῦ.
\VS{15}Καὶ εἶπεν Ἀβιμέλεχ τῷ Ἁβραὰμ, ἰδοὺ ἡ γῆ μου ἐναντίον σου· οὗ ἄν σοι ἀρέσκῃ, κατοίκει.
\VS{16}Τῇ δὲ Σάῤῥᾳ εἶπεν, ἰδοὺ δέδωκα χίλια δίδραγμα τῷ ἀδελφῷ σου· ταῦτα ἔσται σοι εἰς τιμὴν τοῦ προσώπου σου, καὶ πάσαις ταῖς μετὰ σοῦ· καὶ πάντα ἀλήθευσον.
\VS{17}Προσηύξατο δὲ Ἁβραὰμ πρὸς τὸν Θεὸν, καὶ ἰάσατο ὁ Θεὸς τὸν Ἀβιμέλεχ, καὶ τὴν γυναῖκα αὐτοῦ, καὶ τὰς παιδίσκας αὐτοῦ· καὶ ἔτεκον.
\VS{18}Ὅτι συγκλείων συνέκλεισε Κύριος ἔξωθεν πᾶσαν μήτραν ἐν τῷ οἴκῳ Ἀβιμέλεχ, ἕνεκεν Σάῤῥας τῆς γυναικὸς Ἁβραάμ.

\par }\Chap{21}{\PP \VerseOne{1}Καὶ Κύριος ἐπεσκέψατο τὴν Σάῤῥαν, καθὰ εἶπε· καὶ ἐποίησε Κύριος τῇ Σάῥῥᾳ, καθὰ ἐλάλησε.
\VS{2}Καὶ συλλαβοῦσα ἔτεκε τῷ Ἁβραὰμ υἱὸν εἰς τὸ γῆρας, εἰς τὸν καιρὸν καθὰ ἐλάλησεν αὐτῷ Κύριος.
\VS{3}Καὶ ἐκάλεσεν Ἁβραὰμ τὸ ὄνομα τοῦ υἱοῦ αὐτοῦ τοῦ γενομένου αὐτῷ, ὃν ἔτεκεν αὐτῷ Σάῤῥα, Ἰσαάκ·
\VS{4}Περιέτεμε δὲ Ἁβραὰμ τὸν Ἰσαὰκ τῇ ἡμέρᾳ τῇ ὀγδόῃ, καθὰ ἐνετείλατο αὐτῷ ὁ Θεός.
\VS{5}Καὶ Ἁβραὰμ ἦν ἑκατὸν ἐτῶν, ηνίκα ἐγένετο αὐτῷ Ἰσαὰκ ὁ υἱὸς αὐτοῦ.
\VS{6}Εἶπε δὲ Σάῤῥα, γέλωτά μοι ἐποίησε Κύριος· ὃς γὰρ ἂν ἀκούσῃ συγχαρεῖταί μοι.
\VS{7}Καὶ εἶπε τίς ἀναγγελεῖ τῷ Ἁβραὰμ ὅτι θηλάζει παιδίον Σάῤῥα; ὅτι ἔτεκον υἱὸν ἐν τῷ γήρᾳ μου.
\VS{8}Καὶ ηὐξήθη τὸ παιδίον, καὶ ἀπεγαλακτίσθη· καὶ ἐποίησεν Ἁβραὰμ δοχὴν μεγάλην, ᾗ ἡμέρᾳ ἀπεκγαλακτίσθη Ἰσαὰκ ὁ υἱὸς αὐτοῦ.
\VS{9}Ἰδοῦσα δὲ Σάῥῥα τὸν υἱὸν Ἄγαρ τῆς Αἰγυπτίας, ὃς ἐγένετο τῷ Ἁβραὰμ, παίζοντα μετὰ Ἰσαὰκ τοῦ υἱοῦ αὐτῆς,
\VS{10}καὶ εἶπε τῷ Ἁβραὰμ, ἔκβαλε τὴν παιδίσκην ταύτην, καὶ τὸν υἱὸν αὐτῆς· οὐ γὰρ μὴ κληρονομήσει ὁ υἱὸς τῆς παιδίσκης ταύτης μετὰ τοῦ υἱοῦ μου Ἰσαάκ.
\VS{11}Σκληρὸν δὲ ἐφάνη τὸ ῥῆμα σφόδρα ἐνατίον Ἁβραὰμ περὶ τοῦ υἱοῦ αὐτοῦ.
\VS{12}Εἶπε δὲ ὁ Θεὸς τῷ Ἁβραὰμ, μὴ σκληρὸν ἔστω ἐναντίον σου περὶ τοῦ παιδίου, καὶ περὶ τῆς παιδίσκης· πάντα ὅσα ἂν εἴπῃ σοι Σάῤῥα, ἄκουε τῆς φωνῆς αὐτῆς· ὅτι ἐν Ἰσαὰκ κληθήσεταί σοι σπέρμα.
\VS{13}Καὶ τὸν υἱὸν δὲ τῆς παιδίσκης ταύτης εἰς ἔθνος μέγα ποιήσω αὐτὸν, ὅτι σπέρμα σόν ἐστιν.
\VS{14}Ἀνέστη δὲ Ἁβραὰμ τὸ πρωῒ, καὶ ἔλαβεν ἄρτους καὶ ἀσκὸν ὕδατος, καὶ ἔδωκεν τῇ Ἄγαρ· καὶ ἐπέθηκεν ἐπὶ τὸν ὦμον αὐτῆς τὸ παιδίον, καὶ ἀπέστειλεν αὐτήν· Ἀπελθοῦσα δὲ ἐπλανᾶτο κατὰ τὴν ἔρημον, κατὰ τὸ φρέαρ τοῦ ὅρκου.
\VS{15}Ἐξέλιπε δὲ τὸ ὕδωρ ἐκ τοῦ ἀσκου· καὶ ἔῤῥιψε τὸ παιδίον ὑποκάτω μιᾶς ἐλάτης·
\VS{16}Ἀπελθοῦσα δὲ ἐκάθητο ἀπέναντι αὐτοῦ μακρόθεν, ὡσεὶ τόξου βολήν· εἶπε γὰρ, οὐ μὴ ἴδω τὸν θάνατον τοῦ παιδίου μου. καὶ ἐκάθισεν ἀπέναντι αὐτοῦ· ἀναβοῆσαν δὲ τὸ παιδίον ἔκλαυσεν.
\VS{17}Εἰσήκουσε δὲ ὁ Θεὸς τῆς φωνῆς τοῦ παιδίου ἐκ τοῦ τόπου οὗ ἦν· καὶ ἐκάλεσεν ἄγγελος Θεοῦ τὴν Ἄγαρ ἐκ τοῦ οὐρανοῦ, καὶ εἶπεν αὐτῇ, τί ἐστιν Ἄγαρ; μὴ φοβοῦ· ἐπακήκοε γὰρ ὁ Θεὸς τῆς φωνῆς τοῦ παιδίου ἐκ τοῦ τόπου οὗ ἐστιν.
\VS{18}Ἀνάστηθι καὶ λάβε τὸ παιδίον, καὶ κράτησον τῇ χειρί σου αὐτό· εἰς γὰρ ἔθνος μέγα ποιήσω αὐτό.
\VS{19}Καὶ ἀνέῳξεν ὁ Θεὸς τοὺς ὀφθαλμοὺς αὐτῆς· καὶ εἶδε φρέαρ ὕδατος ζῶντος, καὶ ἐπορεύθη, καὶ ἔπλησε τὸν ἀσκὸν ὕδατος, καὶ ἐπότισε τὸ παιδίον.
\VS{20}Καὶ ἦν ὁ Θεὸς μετὰ τοῦ παιδίου· καὶ ηὐξήθη, καὶ κατῴκησεν ἐν τῇ ἐρήμῳ· ἐγένετο δὲ τοξότης.
\VS{21}Καὶ κατῴκησεν ἐν τῇ ἐρήμῳ· καὶ ἔλαβεν αὐτῷ ἡ μήτηρ γυναῖκα ἐκ Φαρὰν Αἰγύπτου.
\par }{\PP \VS{22}Ἐγένετο δὲ ἐν τῷ καιρῷ ἐκείνῳ, καὶ εἶπεν Ἀβιμέλεχ, καὶ Ὁχοζὰθ ὁ νυμφαγωγὸς αὐτοῦ, καὶ Φιχὸλ ὁ ἀρχιστράτηγος τῆς δυνάμεως αὐτοῦ, πρὸς Ἁβραὰμ, λέγων, ὁ Θεὸς μετὰ σοῦ ἐν πᾶσιν, οἷς ἐὰν ποιῇς.
\VS{23}Νῦν οὖν ὄμοσόν μοι τὸν Θεὸν μὴ ἀδικήσειν με, μηδὲ τὸ σπέρμα μου, μηδὲ τὸ ὄνομά μου· ἀλλὰ κατὰ τὴν δικαιοσύνην ἣν ἐποίησα μετὰ σοῦ, ποιήσεις μετʼ ἐμοῦ, καὶ τῇ γᾗ, ᾗ σὺ παρῴκησας ἐν αὐτῇ.
\VS{24}Καὶ εἶπεν Ἁβραὰμ, ἐγὼ ὀμοῦμαι.
\VS{25}Καὶ ἤλεγξεν Ἁβραὰμ τὸν Ἀβιμέλεχ περὶ τῶν φρεάτων τοῦ ὕδατος, ὧν ἀφείλοντο οἱ παῖδες τοῦ Ἀβιμέλεχ.
\VS{26}Καὶ εἶπεν αὐτῷ Ἀβιμέλεχ, οὐκ ἔγνων τίς ἐποίησέ σοι τὸ ῥῆμα τοῦτο· οὐδὲ σύ μοι ἀπήγγειλας, οὐδὲ ἐγὼ ἤκουσα, ἀλλʼ ἢ σήμερον.
\VS{27}Καὶ ἔλαβεν Ἁβραὰμ πρόβατα καὶ μόσχους, καὶ ἔδωκε τῷ Ἀβιμέλεχ· καὶ διέθεντο ἀμφότεροι διαθήκην.
\VS{28}Καὶ ἔστησεν Ἁβραὰμ, ἑπτὰ ἀμνάδας προβάτων μόνας.
\VS{29}Καὶ εἶπεν Ἀβιμέλεχ τῷ Ἁβραὰμ, τί εἰσιν αἱ ἑπτὰ ἀμνάδες τῶν προβάτων τούτων, ἃς ἔστησας μόνας;
\VS{30}Καὶ εἶπεν Ἁβραὰμ, ὅτι τὰς ἑπτὰ ἀμνάδας λήψῃ παρʼ ἐμοῦ, ἵνα ὦσι μοι εἰς μαρτύριον, ὅτι ἐγὼ ὤρυξα τό φρέαρ τοῦτο.
\VS{31}Διὰ τοῦτο ἐπωνόμασε τὸ ὄνομα τοῦ τόπου ἐκείνου, Φρέαρ ὁρκισμοῦ· ὅτι ἐκεῖ ὤμοσαν ἀμφότεροι.
\VS{32}Καὶ διέθεντο διαθήκην ἐν τῷ φρέατι τοῦ ὁρκισμου· ἀνέστη δὲ Ἀβιμέλεχ, Ὁχοζὰθ ὁ νυμφαγωγὸς αὐτοῦ, καὶ Φίχολ ὁ ἀρχιστράτηγος τῆς δυνάμεως αὐτοῦ, καὶ ἐπέστρεψαν εἰς τὴν γῆν τῶν Φυλιστιείμ.
\VS{33}Καὶ ἐφύτευσεν Ἁβραὰμ ἄρουραν ἐπὶ τῷ φρέατι τοῦ ὅρκου· καὶ ἐπεκαλέσατο ἐκεῖ τὸ ὄνομα Κυρίου, Θεὸς αἰώνιος.
\VS{34}Παρῴκησε δὲ Ἁβραὰμ ἐν τῇ γῇ τῶν Φυλιστιεὶμ ἡμέρας πολλάς.

\par }\Chap{22}{\PP \VerseOne{1}Καὶ ἐγένετο μετὰ τὰ ῥήματα ταῦτα ὁ Θεὸς ἐπείρασε τὸν Ἁβραὰμ, καὶ εἶπεν αὐτῷ, Ἁβραὰμ, Ἁβραάμ· καὶ εἶπεν, ἰδοὺ ἐγώ.
\VS{2}Καὶ εἶπε, λάβε τὸν υἱόν σου τὸν ἀγαπητὸν, ὃν ἠγάπησας, τὸν Ἰσαὰκ, καὶ πορεύθητι εἰς τὴν γῆν τὴν ὑψηλὴν, καὶ ἀνένεγκε αὐτὸν ἐκεῖ εἰς ὁλοκάρπωσιν ἐφʼ ἓν τῶν ὀρέων ὧν ἄν σοι εἴπω.
\VS{3}Ἀναστὰς δὲ Ἁβραὰμ τὸ πρωῒ, ἐπέσαξε τὴν ὄνον αὐτοῦ· παρέλαβε δὲ μεθʼ ἑαυτοῦ δύο παῖδας, καὶ Ἰσαὰκ τὸν υἱὸν αὐτοῦ· καὶ σχίσας ξύλα εἰς ὁλοκάρπωσιν, ἀναστὰς ἐπορεύθη, καὶ ἦλθεν ἐπὶ τὸν τόπον, ὃν εἶπεν αὐτῷ ὁ Θεὸς, τῇ ἡμέρᾳ τῇ τρίτῃ.
\VS{4}Καὶ ἀναβλέψας Ἁβραὰμ τοῖς ὀφθαλμοῖς αὐτοῦ, εἶδε τὸν τόπον μακρόθεν.
\VS{5}Καὶ εἶπεν Ἁβραὰμ τοῖς παισὶν αὐτοῦ, καθίσατε αὐτοῦ μετὰ τῆς ὄνου· ἐγὼ δὲ καὶ τὸ παιδάριον διελευσόμεθα ἕως ὧδε· καὶ προσκυνήσαντες ἀναστρέψομεν πρὸς ὑμᾶς.
\VS{6}Ἔλαβε δὲ Ἁβραὰμ τὰ ξύλα τῆς ὁλοκαρπώσεως, καὶ ἐπέθηκεν Ἰσαὰκ τῷ υἱῷ αὐτοῦ· ἔλαβε δὲ μετὰ χεῖρας καὶ τὸ πῦρ καὶ τὴν μάχαιραν, καὶ ἐπορεύθησαν οἱ δύο ἅμα.
\VS{7}Εἶπε δὲ Ἰσαὰκ πρὸς Ἁβραὰμ τὸν πατέρα αὐτοῦ, πάτερ· ὁ δὲ εἶπε, τί ἐστι, τέκνον; εἶπε, δὲ, ἰδοὺ τὸ πῦρ καὶ τὰ ξύλα, ποῦ ἐστὶ τὸ πρόβατον τὸ εἰς ὁλοκάρπωσιν;
\VS{8}Εἶπε δὲ Ἁβραὰμ, ὁ Θεὸς ὄψεται ἑαυτῷ πρόβατον εἰς ὁλοκάρπωσιν, τέκνον. πορευθέντες δὲ ἀμφότεροι ἅμα,
\VS{9}ἦλθον ἐπὶ τὸν τόπον, ὃν εἶπεν αὐτῷ ὁ Θεός· καὶ ᾠκοδόμησεν ἐκεῖ Ἁβραὰμ τὸ θυσιαστήριον, καὶ ἐπέθηκε τὰ ξύλα· καὶ συμποδίσας Ἰσαὰκ τὸν υἱὸν αὐτοῦ, ἐπέθηκεν αὐτὸν ἐπὶ τὸ θυσιαστήριον ἐπάνω τῶν ξύλων.
\VS{10}Καὶ ἐξέτεινεν Ἁβραὰμ τὴν χεῖρα αὐτοῦ λαβεῖν τὴν μάχαιραν, σφάξαι τὸν υἱὸν αὐτοῦ.
\VS{11}Καὶ ἐκάλεσεν αὐτὸν Ἄγγελος Κυρίου ἐκ τοῦ οὐρανοῦ, καὶ εἶπεν, Ἁβραὰμ, Ἁβραάμ· ὁ δὲ εἶπεν, ἰδοὺ ἐγώ.
\VS{12}Καὶ εἶπε, μὴ ἐπιβάλῃς τὴν χεῖρά σου ἐπὶ τὸ παιδάριον, μηδὲ ποιήσῃς αὐτῷ μηδέν· νῦν γὰρ ἔγνων, ὅτι φοβῇ σὺ τὸν Θεόν· καὶ οὐκ ἐφείσω τοῦ υἱοῦ σου τοῦ ἀγαπητοῦ διʼ ἐμέ.
\VS{13}Καὶ ἀναβλέψας Ἁβραὰμ τοῖς ὀφθαλμοῖς αὐτοῦ εἶδε, καὶ ἰδοὺ κριὸς εἷς κατεχόμενος ἐν φυτῷ Σαβὲκ τῶν κεράτων. Καὶ ἐπορεύθη Ἁβραὰμ, καὶ ἔλαβε τὸν κριὸν, καὶ ἁνήνεγκεν αὐτὸν εἰς ὁλοκάρπωσιν ἀντὶ Ἰσαὰκ τοῦ υἱοῦ αὐτοῦ.
\par }{\PP \VS{14}Καὶ ἐκάλεσεν Ἁβραὰμ τὸ ὄνομα τοῦ τόπου ἐκείνου, Κύριος εἶδεν· ἵνα εἴπωσιν σήμερον, ἐν τῷ ὄρει Κύριος ὤφθη.
\VS{15}Καὶ ἐκάλεσεν Ἄγγελος Κυρίου τὸν Ἁβραὰμ δεύτερον ἐκ τοῦ οὐρανοῦ,
\VS{16}λέγων, Κατʼ ἐμαυτοῦ ὤμοσα, λέγει Κύριος, οὗ εἵνεκεν ἐποίησας τὸ ῥῆμα τοῦτο, καὶ οὐκ ἐφείσω τοῦ υἱοῦ σου τοῦ ἀγαπτοῦ διʼ ἐμὲ,
\VS{17}Ἦ μὴν εὐλογῶν εὐλογήσω σε, καὶ πληθύνων πληθυνῶ τὸ σπέρμα σου, ὡς τοὺς ἀστέρας τοῦ οὐρανοῦ, καὶ ὡς τὴν ἄμμον τὴν παρὰ τὸ χεῖλος τῆς θαλάσσης· καὶ κληρονομήσει τὸ σπέρμα σου τὰς πόλεις τῶν ὑπεναντίων.
\VS{18}Καὶ ἐνευλογηθήσονται ἐν τῷ σπέρματί σου πάντα τὰ ἔθνη τῆς γῆς, ἀνθʼ ὧν ὑπήκουσας τῆς ἐμῆς φωνῆς.
\VS{19}Ἀπεστράφη δὲ Ἁβραὰμ πρὸς τοὺς παῖδας αὐτοῦ· καὶ ἀναστάντες ἐπορεύθησαν ἅμα ἐπὶ τὸ φρέαρ τοῦ ὅρκου. Καὶ κατῴκησεν Ἁβραὰμ ἐπὶ τὸ φρέαρ τοῦ ὅρκου.
\par }{\PP \VS{20}Ἐγένετο δὲ μετὰ τὰ ῥήματα ταῦτα, καὶ ἀνηγγέλη τῷ Ἁβραὰμ, λέγοντες, ἰδοὺ τέτοκε Μελχὰ καὶ αὐτὴ υἱοὺς τῷ Ναχὼρ τῷ ἀδελφῷ σου,
\VS{21}τὸν Οὒζ πρωτότοκον, καὶ τὸν Βαὺξ ἀδελφὸν αὐτοῦ, καὶ τὸν Καμουὴλ πατέρα Σύρων,
\VS{22}καὶ τὸν Χαζὰδ, καὶ Ἀζαῦ, καὶ τὸν Φαλδὲς, καὶ τὸν Ἰελδὰφ, καὶ τὸν Βαθουήλ.
\VS{23}Βαθουὴλ δὲ ἐγέννησε τὴν Ῥεβέκκαν. ὀκτὼ οὗτοι υἱοὶ, οὓς ἔτεκε Μελχὰ τῷ Ναχὼρ τῷ ἀδελφῷ Ἁβραάμ.
\VS{24}Καὶ ἡ παλλακὴ αὐτοῦ, ᾗ ὄνομα Ῥεύμα, ἔτεκε καὶ αὐτὴ τὸν Ταβὲκ, καὶ τὸν Ταὰμ, καὶ τὸν Τοχός, καὶ τὸν Μοχά.

\par }\Chap{23}{\PP \VerseOne{1}Ἐγένετο δὲ ἡ ζωὴ Σάῤῥας, ἔτη ἑκατὸν εἰκοσιεπτά.
\VS{2}Καὶ ἀπέθανε Σάῤῥα ἐν πόλει Ἀρβὸκ, ἥ ἐστιν ἐν τῷ κοιλώματι· αὕτη ἔστι Χεβρὼν ἐν τῇ γῇ Χαναάν ἦλθε δὲ Ἁβραὰμ κόψασθαι Σάῤῥαν, καὶ πενθῆσαι.
\VS{3}Καὶ ἀνέστη Ἁβραὰμ ἀπὸ τοῦ νεκροῦ αὐτοῦ· καὶ εἶπεν Ἁβραὰμ τοῖς υἱοῖς τοῦ Χὲτ, λέγων,
\VS{4}Πάροικος καὶ παρεπίδημος ἐγώ εἰμι μεθʼ ὑμῶν· δότε μοι οὖν κτῆσιν τάφου μεθʼ ὑμῶν, καὶ θάψω τὸν νεκρόν μου ἀπʼ ἐμοῦ.
\VS{5}Ἀπεκρίθησαν δὲ οἱ υἱοὶ Χὲτ πρὸς Ἁβραὰμ, λέγοντες, μὴ, κύριε.
\VS{6}Ἄκουσον δὲ ἡμῶν· βασιλεὺς παρὰ Θεοῦ σὺ εἶ ἐν ἡμῖν· ἐν τοῖς ἐκλεκτοῖς μνημείοις ἡμῶν θάψον τὸν νεκρόν σου· οὐδεὶς γὰρ ἡμῶν οὐ μὴ κωλύσει τὸ μνημεῖον αὐτοῦ ἀπὸ σοῦ, τοῦ θάψαι τὸν νεκρόν σου ἐκεῖ.
\VS{7}Ἀναστὰς δὲ Ἁβραὰμ προσεκύνησε τῷ λαῷ τῆς γῆς, τοῖς υἱοῖς τοῦ Χέτ.
\VS{8}Καὶ ἐλάλησε πρὸς αὐτοὺς Ἁβραὰμ, λέγων, εἰ ἔχετε τῇ ψυχῇ ὑμῶν, ὥστε θάψαι τὸν νεκρόν μου ἀπὸ προσώπου μου, ἀκούσατέ μου, καὶ λαλήσατε περὶ ἐμοῦ Ἐφρὼν τῷ τοῦ Σαάρ.
\VS{9}Καὶ δότω μοι τὸ σπήλαιον τὸ διπλοῦν, ὅ ἐστιν αὐτῷ, τὸ ὂν ἐν μέρει τοῦ ἀγροῦ αὐτοῦ· ἀργυρίου τοῦ ἀξίου δότε μοι αὐτὸ ἐν ὑμῖν εἰς κτῆσιν μνημείου.
\VS{10}Ἐφρὼν δὲ ἐκάθητο ἐν μέσῳ τῶν υἱῶν Χέτ· ἀποκριθεὶς δὲ Ἐφρὼν ὁ Χετταῖος πρὸς Ἁβραὰμ εἶπεν, ἀκουόντων τῶν υἱῶν Χὲτ, καὶ τῶν εἰσπορευομένων εἰς τὴν πόλιν πάντων, λέγων,
\VS{11}Παρʼ ἐμοὶ γενοῦ, κύριε, καὶ ἄκουσόν μου· τὸν ἀγρὸν, καὶ τὸ σπήλαιον τὸ ἐν αὐτῷ, σοὶ δίδωμι· ἐναντίον πάντων τῶν πολιτῶν μου δέδωκά σοι· θάψον τὸν νεκρόν σου.
\VS{12}Καὶ προσεκύνησεν Ἁβραὰμ ἐναντίον τοῦ λαοῦ τῆς γῆς.
\VS{13}Καὶ εἶπε τῷ Ἐφρὼν εἰς τὰ ὦτα ἐναντίον τοῦ λαοῦ τῆς γῆς, ἐπειδὴ πρὸς ἐμοῦ εἶ, ἄκουσόν μου· τὸ ἀργύριον τοῦ ἀγροῦ λάβε παρʼ ἐμοῦ, καὶ θάψω τὸν νεκρόν μου ἐκεῖ.
\VS{14}Ἀπεκρίθη δὲ Ἐφρὼν τῷ Ἁβραὰμ, λέγων,
\VS{15}Οὐχὶ, κύριε· ἀκήκοα γὰρ, γῆ τετρακοσίων διδράχμων ἀργύριου· ἀλλὰ τί ἂν εἴη τοῦτο ἀνὰ μέσον ἐμοῦ καὶ σοῦ; σὺ δὲ τὸν νεκρόν σου θάψον.
\VS{16}καὶ ἤκουσεν Ἁβραὰμ τοῦ Ἐφρών· καὶ ἀπεκατέστησεν Ἁβραὰμ τῷ Ἐφρὼν τὸ ἀργύριον, ὃ ἐλάλησεν εἰς τὰ ὦτα τῶν υἱῶν Χὲτ, τετρακόσια δίδραχμα ἀργυρίου δοκίμου ἐμπόροις.
\VS{17}Καὶ ἔστη ὁ ἀγρὸς Ἐφρών, ὃς ἦν ἐν τῷ διπλῷ σπηλαίῳ, ὅς ἐστι κατὰ πρόσωπον Μαμβρῆ, ὁ ἀγρὸς καὶ τὸ σπήλαιον, ὃ ἦν ἐν αὐτῷ, καὶ πᾶν δένδρον, ὃ ἦν ἐν τῷ ἀγρῷ, καὶ πᾶν ὅ ἐστιν ἐν τοῖς ὁρίοις αὐτοῦ κύκλῳ,
\VS{18}τῷ Ἁβραὰμ, εἰς κτῆσιν ἐναντίον τῶν υἱῶν Χὲτ, καὶ πάντων τῶν εἰσπορευομένων εἰς τὴν πόλιν.
\VS{19}Μετὰ ταῦτα ἔθαψεν Ἁβραὰμ Σάῤῥαν τὴν γυναῖκα αὐτοῦ ἐν τῷ σπηλαίῳ τοῦ ἀγροῦ τῷ διπλῷ, ὅ ἐστιν ἀπέναντι Μαμβρῆ· αὕτη ἐστὶ Χεβρὼν ἐν τῇ γῇ Χαναάν.
\VS{20}Καὶ ἐκυρώθη ὁ ἀγρὸς καὶ τὸ σπήλαιον ὃ ἦν ἐν αὐτῷ τῷ Ἁβραὰμ εἰς κτῆσιν τάφου, παρὰ τῶν υἱῶν Χέτ.

\Chap{24}\VerseOne{1}Καὶ Ἁβραὰμ ἦν πρεσβύτερος προβεβηκὼς ἡμερῶν· καὶ Κύριος ηὐλόγησε τὸν Ἁβραὰμ κατὰ πάντα.
\par }{\PP \VS{2}Καὶ εἶπεν Ἁβραὰμ τῷ παιδὶ αὐτοῦ τῷ πρεσβυτέρῳ τῆς οἰκίας αὐτοῦ, τῷ ἄρχοντι πάντων τῶν αὐτοῦ, θὲς τὴν χεῖρά σου ὑπὸ τὸν μηρόν μου.
\VS{3}Καὶ ἐξορκιῶ σε Κύριον τὸν Θεὸν τοῦ οὐρανοῦ καὶ τὸν Θεὸν τῆς γῆς, ἵνα μὴ λάβῃς γυναῖκα τῷ υἱῷ μου Ἰσαὰκ ἀπὸ τῶν θυγατέρων τῶν Χαναναίων, μεθʼ ὧν ἐγὼ οἰκῶ ἐν αὐτοις.
\VS{4}Ἀλλʼ ἢ εἰς τὴν γῆν μου, οὗ ἐγεννήθην, πορεύσῃ, καὶ εἰς τὴν φυλήν μου, καὶ λήψῃ γυναῖκα τῷ υἱῷ μου Ἰσαὰκ ἐκεῖθεν.
\VS{5}Εἶπε δὲ πρὸς αὐτὸν ὁ παῖς, μή ποτε οὐ βούληται ἡ γυνὴ πορευθῆναι μετʼ ἐμοῦ ὀπίσω εἰς τὴν γῆν ταύτην, ἀποστρέψω τὸν υἱόν σου εἰς τὴν γῆν, ὅθεν ἐξῆλθες ἐκεῖθεν;
\VS{6}Εἶπε δὲ πρὸς αὐτὸν Ἁβραάμ, πρόσεχε σεαυτῷ μὴ ἀποστρέψῃς τὸν υἱόν μου ἐκεῖ.
\VS{7}Κύριος ὁ Θεὸς τοῦ οὐρανοῦ καὶ ὁ Θεὸς τῆς γῆς, ὃς ἔλαβέ με ἐκ τοῦ οἴκου τοῦ πατρός μου, καἰ ἐκ τῆς γῆς ἧς ἐγεννήθην, ὃς ἐλάλησέ μοι, καὶ ὃς ὤμοσέ μοι, λέγων, σοὶ δώσω τὴν γῆν ταύτην καὶ τῷ σπέρματί σου, αὐτὸς ἀποστελεῖ τὸν Ἄγγελον αὐτοῦ ἔμπροσθέν σου, καὶ λήψῃ γυναῖκα τῷ υἱῷ μου ἐκεῖθεν.
\VS{8}Ἐὰν δὲ μὴ θέλῃ ἡ γυνὴ πορευθῆναι μετὰ σοῦ εἰς τὴν γῆν ταύτην, καθαρὸς ἔσῃ ἀπὸ τοῦ ὅρκου μου· μόνον τὸν υἱόν μου μὴ ἀποστρέψῃς ἐκεῖ.
\VS{9}Καὶ ἔθηκεν ὁ παῖς τὴν χεῖρα αὐτοῦ ὑπὸ τὸν μηρὸν Ἁβραὰμ τοῦ κυρίου αὐτοῦ, καὶ ὤμοσεν αὐτῷ περὶ τοῦ ῥήματος τούτου.
\VS{10}Καὶ ἔλαβεν ὁ παῖς δέκα καμήλους ἀπὸ τῶν καμήλων τοῦ κυρίου αὐτοῦ, καὶ ἀπὸ πάντων τῶν ἀγαθῶν τοῦ κυρίου αὐτοῦ μεθʼ ἑαυτοῦ· καὶ ἀναστὰς ἐπορεύθη εἰς τὴν Μεσοποταμίαν εἰς τὴν πόλιν Ναχώρ.
\VS{11}Καὶ ἐκοίμησε τὰς καμήλους ἔξω τῆς πόλεως παρὰ τὸ φρέαρ τοῦ ὕδατος τὸ πρὸς ὀψέ, ἡνίκα ἐκπορεύονται αἱ ὑδρευόμεναι.
\par }{\PP \VS{12}Καὶ εἶπε, Κύριε ὁ Θεὸς τοῦ κυρίου μου Ἁβραάμ, εὐόδωσον ἐναντίον ἐμοῦ σήμερον, καὶ ποίησον ἔλεος μετὰ τοῦ κυρίου μου Ἁβραάμ.
\VS{13}Ἰδοὺ ἐγὼ ἕστηκα ἐπὶ τῆς πηγῆς τοῦ ὕδατος· αἱ δὲ θυγατέρες τῶν οἰκούντων τὴν πόλιν ἐκπορεύονται ἀντλῆσαι ὕδωρ.
\VS{14}Καὶ ἔσται ἡ παρθένος ᾗ ἂν ἐγὼ εἴπω, ἐπίκλινον τὴν ὑδρίαν σου, ἵνα πίω, καὶ εἴπῃ μοι, πίε σύ, καὶ τὰς καμήλους σου ποτιῶ, ἕως ἂν παύσωνται πίνουσαι, ταύτην ἡτοίμασας τῷ παιδί σου τῷ Ἰσαάκ· καὶ ἐν τούτῳ γνώσομαι, ὅτι ἐποίησας ἔλεος μετὰ τοῦ κυρίου μου Ἁβραάμ.
\par }{\PP \VS{15}Καὶ ἐγένετο πρὸ τοῦ συντελέσαι αὐτὸν λαλοῦντα ἐν τῇ διανοίᾳ αὐτοῦ, καὶ ἰδοὺ Ῥεβέκκα ἐξεπορεύετο ἡ τεχθεῖσα Βαθουήλ, υἱῷ Μελχὰς τῆς γυναικὸς Ναχώρ, ἀδελφοῦ δὲ Ἁβραάμ, ἔχουσα τὴν ὑδρίαν ἐπὶ τῶν ὤμων αὐτῆς.
\VS{16}Ἡ δὲ παρθένος ἦν καλὴ τῇ ὄψει σφόδρα· παρθένος ἦν, ἀνὴρ οὐκ ἔγνω αὐτήν· καταβᾶσα δὲ ἐπὶ τὴν πηγὴν, ἔπλησε τὴν ὑδρίαν αὐτῆς, καὶ ἀνέβη.
\VS{17}Ἐπέδραμε δὲ ὁ παῖς εἰς συνάντησιν αὐτῆς, καὶ εἶπε, Πότισόν με μικρὸν ὕδωρ ἐκ τῆς ὑδρίας σου.
\VS{18}Ἡ δὲ εἶπε, πίε, κύριε· καὶ ἔσπευσε καὶ καθεῖλε τὴν ὑδρίαν ἐπὶ τὸν βραχίονα αὐτῆς, καὶ ἐπότισεν αὐτὸν, ἕων ἐπαύσατο πίνων.
\VS{19}Καὶ εἶπε, καὶ ταῖς καμήλοις σου ὑδρεύσομαι, ἕως ἂν πᾶσαι πίωσι.
\VS{20}Καὶ ἔσπευσε καὶ ἐξεκένωσε τὴν ὑδρίαν εἰς τὸ ποτιστήριον· καὶ ἔδραμεν ἐπὶ τὸ φρέαρ ἀντλῆσαι πάλιν· καὶ ὑδρεύσατο πάσαις ταῖς καμήλοις.
\VS{21}Ὁ δὲ ἄνθρωπος κατεμάνθανεν αὐτήν· καὶ παρεσιώπα τοῦ γνῶναι εἰ εὐώδωκε Κύριος τὴν ὁδὸν αὐτοῦ, ἢ οὔ.
\VS{22}Ἐγένετο δὲ ἡνίκα ἐπαύσαντο πᾶσαι αἱ κάμηλοι πίνουσαι, ἔλαβεν ὁ ἄνθρωπος ἐνώτια χρυσᾶ ἀνὰ δραχμὴν ὁλκῆς, καὶ δύο ψέλλια ἐπὶ τὰς χεῖρας αὐτῆς, δέκα χρυσῶν ὁλκὴ αὐτῶν.
\VS{23}Καὶ ἐπηρώτησεν αὐτὴν, καὶ εἶπε, θυγάτηρ τίνος εἶ; ἀνάγγειλόν μοι, εἰ ἔστι παρὰ τῷ πατρί σου τόπος ἡμῖν του καταλῦσαι.
\VS{24}Ἡ δὲ εἶπεν αὐτῷ, θυγάτηρ Βαθουήλ εἰμι τοῦ Μελχάς, ὃν ἔτεκε τῷ Ναχώρ.
\VS{25}Καὶ εἶπεν αὐτῷ, Καὶ ἄχυρα καὶ χορτάσματα πολλὰ παρʼ ἡμῖν, καὶ τόπος τοῦ καταλῦσαι.
\VS{26}Καὶ εὐδοκήσας ὁ ἄνθρωπος προσεκύνησε τῷ Κυρίῳ
\VS{27}Καὶ εἶπεν, εὐλογητὸς Κύριος ὁ Θεὸς τοῦ κυρίου μου Ἁβραάμ, ὃς οὐκ ἐγκατέλειπε τὴν δικαιοσύνην αὐτοῦ, καὶ τὴν ἀλήθειαν, ἀπὸ τοῦ κυρίου μου· ἐμὲ τʼ εὐώδωκε Κύριος εἰς οἶκον τοῦ ἀδελφοῦ τοῦ κυρίου μου.
\VS{28}Καὶ δραμοῦσα ἡ παῖς ἀνήγγειλεν εἰς τὸν οἶκον τῆς μητρὸς αὐτῆς, κατὰ τὰ ῥήματα ταῦτα.
\VS{29}Τῇ δὲ Ῥεβέκκᾷ ἀδελφὸς ἦν, ᾧ ὄνομα Λάβαν· καὶ ἔδραμε Λάβαν πρὸς τὸν ἄνθρωπον ἔξω ἐπὶ τὴν πηγήν.
\VS{30}Καὶ ἐγένετο ἡνίκα εἶδε τὰ ἐνώτια, καὶ τὰ ψέλλια ἐν ταῖς χερσὶ τῆς ἀδελφῆς αὐτοῦ, καὶ ὅτε ἤκουσε τὰ ῥήματα Ῥεβέκκας τῆς ἀδελφῆς αὐτοῦ, λεγούσης, οὕτω λελάληκέ μοι ὁ ἄνθρωπος, καὶ ἦλθε πρὸς τὸν ἄνθρωπον, ἑστηκότος αὐτοῦ ἐπὶ τῶν καμήλων ἐπὶ τῆς πηγῆς.
\VS{31}Καὶ εἶπεν αὐτῷ, δεῦρο εἴσελθε, εὐλογητὸς Κυροίυ· ἱνατί ἕστηκας ἔξω; ἐγὼ δὲ ἡτοίμασα τὴν οἰκίαν, καὶ τόπον ταῖς καμήλοις.
\VS{32}Εἰσῆλθε δὲ ὁ ἄνθρωπος εἰς τὴν οἰκίαν, καὶ ἀπέσαξε τὰς καμήλους· καὶ ἔδωκεν ἄχυρα καὶ χορτάσματα ταῖς καμήλοις, καὶ ὕδωρ νίψασθαι τοῖς ποσὶν αὐτοῦ, καὶ τοῖς ποσὶ τῶν ἀνδρῶν τῶν μετʼ αὐτοῦ.
\VS{33}Καὶ παρέθηκεν αὐτοῖς ἄρτους φαγεῖν· καὶ εἶπεν, οὐ μὴ φάγω, ἕως τοῦ λαλῆσαί με τὰ ῥήματά μου· καὶ εἶπεν, λάλησον.
\par }{\PP \VS{34}Καὶ εἶπε, παῖς Ἁβραὰμ ἐγώ εἰμι.
\VS{35}Κύριος δὲ ηὐλόγησε τὸν κύριόν μου σφόδρα, καὶ ὑψώθη· καὶ ἔδωκεν αὐτῷ πρόβατα, καὶ μόσχους, καὶ ἀργύριον, καὶ χρυσίον, παῖδας, καὶ παιδίσκας, καμήλους, καὶ ὄνους.
\VS{36}Καὶ ἔτεκε Σάῤῥα ἡ γυνὴ τοῦ κυρίου μου υἱὸν ἕνα τῷ κυρίῳ μου μετὰ τὸ γηράσαι αὐτόν· καὶ ἔδωκεν αὐτῷ ὅσα ἦν αὐτῷ.
\VS{37}Καὶ ὥρκισέ με ὁ κύριός μου, λέγων, οὐ λήμψῃ γυναῖκα τῷ υἱῷ μου ἀπὸ τῶν θυγατέρων τῶν Χαναναίων, ἐν οἷς ἐγὼ παροικῶ ἐν τῇ γῇ αὐτῶν.
\VS{38}Ἀλλʼ εἰς τὸν οἶκον τοῦ πατρός μου πορεύσῃ, καὶ εἰς τὴν φυλήν μου, καὶ λήψῃ γυναῖκα τῷ υἱῷ μου ἐκεῖθεν.
\VS{39}Εἶπα δὲ τῷ κυρίῳ μου, μήποτε οὐ πορεύσεται ἡ γυνὴ μετʼ ἐμοῦ.
\VS{40}Καὶ εἶπέ μοι, Κύριος ὁ Θεὸς ᾧ εὐηρέστησα ἐναντίον αὐτοῦ, αὐτὸς ἐξαποστελεῖ τὸν Ἀγγελον αὐτοῦ μετὰ σοῦ, καὶ εὐοδώσει τὴν ὁδόν σου· καὶ λήψῃ γυναῖκα τῷ υἱῷ μου ἐκ τῆς φυλῆς μου, καὶ ἐκ τοῦ οἴκου τοῦ πατρός μου.
\VS{41}Τότε ἀθῷος ἔσῃ ἀπὸ τῆς ἀρᾶς μου· ἡνίκα γὰρ ἐὰν ἔλθῃς εἰς τὴν φυλήν μου, καὶ μή σοι δῶσι, καὶ ἔσῃ ἀθῷος ἀπὸ τοῦ ὁρκισμοῦ μου.
\VS{42}Καὶ ἐλθὼν σήμερον ἐπὶ τὴν πηγὴν εἶπα, Κύριε ὁ Θεὸς τοῦ κυρίου μου Ἁβραὰμ, εἰ σὺ εὐοδοῖς τὴν ὁδόν μου, ἐν ᾗ νῦν ἐγὼ πορεύομαι ἐν αὐτῇ,
\VS{43}ἰδοὺ ἐγὼ ἐφέστηκα ἐπὶ τῆς πηγῆς τοῦ ὕδατος, καὶ αἱ θυγατέρες τῶν ἀνθρώπων τῆς πόλεως ἐκπορεύονται ἀντλῆσαι ὕδωρ· καὶ ἔσται ἡ παρθένος, ᾗ ἂν ἐγὼ εἴπω, πότισόν με ἐκ τῆς ὑδρίας σου μικρὸν ὕδωρ,
\VS{44}καὶ εἴπῃ μοι, καὶ σὺ πίε, καὶ ταῖς καμήλοις σου ὑδρεύσομαι, αὕτη ἡ γυνὴ ἣν ἡτοίμασε Κύριος τῷ ἑαυτοῦ θεράποντι Ἰσαάκ· καὶ ἐν τούτῳ γνώσομαι, ὅτι πεποίηκας ἔλεος τῷ κυρίῳ μου Ἁβραάμ.
\VS{45}Καὶ ἐγένετο πρὸ τοῦ συντελέσαι με λαλοῦντα ἐν τῇ διανοίᾳ μου, εὐθὺς Ῥεβέκκα ἐξεπορεύετο, ἔχουσα τὴν ὑδρίαν ἐπὶ τῶν ὤμων· καὶ κατέβη ἐπὶ τὴν πηγὴν, καὶ ὑδρεύσατο· εἶπα δὲ αὐτῇ, πότισόν με.
\VS{46}Καὶ σπεύσασα καθεῖλε τὴν ὑδρίαν ἐπὶ τὸν βραχίονα αὐτῆς ἀφʼ ἑαυτῆς, καὶ εἶπε, πίε σὺ, καὶ τὰς καμήλους σου ποτιῶ· καὶ ἔπιον, καὶ τὰς καμήλους ἐπότισε.
\VS{47}Καὶ ἠρώτησα αὐτὴν, καὶ εἶπα, θυγάτηρ τίνος εἶ, ἀναγγειλόν μοι· ἡ δὲ ἔφη, θυγάτηρ Βαθουὴλ εἰμὶ υἱοῦ τοῦ Ναχὼρ, ὃν ἔτεκεν αὐτῷ Μελχά· καὶ περιέθηκα αὐτῇ τὰ ἐνώτια, καὶ τὰ ψέλλια περὶ τὰς χεῖρας αὐτῆς.
\VS{48}Καὶ εὐδοκήσας προσεκύνησα τῷ Κυρίῳ, καὶ εὐλόγησα Κύριον τὸν Θεὸν τοῦ κυρίου μου Ἁβραὰμ, ὃς εὐώδωσέ με ἐν ὁδῷ ἀληθείας λαβεῖν τὴν θυγατέρα τοῦ ἀδελφοῦ τοῦ κυρίου μου τῷ υἱῷ αὐτοῦ.
\VS{49}Εἰ οὖν ποιεῖτε ὑμεῖς ἔλεος καὶ δικαιοσύνην πρὸς τὸν κύριόν μου· εἰ δὲ μὴ, ἀπαγγείλατέ μοι, ἵνα ἐπιστρέψω εἰς δεξιὰν ἤ ἀριστεράν.
\par }{\PP \VS{50}Ἀποκριθεὶς δὲ Λάβαν καὶ Βαθουὴλ εἶπαν, παρὰ κυρίου ἐξῆλθε τὸ πρᾶγμα τοῦτο· οὐ δυνησόμεθά σοι ἀντειπεῖν κακὸν ἢ καλόν.
\VS{51}Ἰδοὺ Ῥεβέκκα ἐνώπιόν σου· λαβὼν ἀπότρεχε· καὶ ἔστω γυνὴ τῷ υἱῷ τοῦ κυρίου σου, καθὰ ἐλάλησε Κύριος.
\VS{52}Ἐγένετο δὲ ἐν τῷ ἀκοῦσαι τὸν παῖδα τοῦ Ἁβραὰμ τῶν ῥημάτων αὐτῶν, προσεκύνησεν ἐπὶ τὴν γῆν τῷ κυρίῳ.
\VS{53}καὶ ἐξενέγκας ὁ παῖς σκεύη ἀργυρᾶ καὶ χρυσᾶ καὶ ἱματισμὸν, ἔδωκε τῇ Ῥεβέκκᾳ· καὶ δῶρα ἔδωκε τῷ ἀδελφῷ αὐτῆς, καὶ τῇ μητρὶ αὐτῆς.
\VS{54}Καὶ ἔφαγον καὶ ἔπιον καὶ αὐτὸς καὶ οἱ ἄνδρες οἱ μετʼ αὐτοῦ ὄντες, καὶ ἐκοιμήθησαν· καὶ ἀναστὰς τὸ πρωῒ εἶπεν, ἐκπέμψατέ με, ἵνα ἀπέλθω πρὸς τὸν κύριόν μου.
\VS{55}Εἶπαν δὲ οἱ ἀδελφοὶ αὐτῆς, καὶ ἡ μήτηρ, μεινάτω ἡ παρθένος μεθʼ ἡμῶν ἡμέρας ὡσεὶ δέκα, καὶ μετὰ ταῦτα ἀπελεύσεται.
\VS{56}Ὁ δὲ εἶπε πρὸς αὐτοὺς, μὴ κατέχετέ με· καὶ Κύριος εὐώδωσε τὴν ὁδόν μου ἐν ἐμοί· ἐκπέμψατέ με, ἵνα ἀπέλθω πρὸς τὸν κύριόν μου.
\VS{57}Οἱ δὲ εἶπαν, Καλέσωμεν τὴν παῖδα, καὶ ἐρωτήσωμεν τὸ στόμα αὐτῆς.
\VS{58}Καὶ ἐκάλεσαν τὴν Ῥεβέκκαν, καὶ εἶπαν αὐτῇ, πορεύσῃ μετὰ τοῦ ἀνθρώπου τούτου; ἡ δὲ εἶπε, πορεύσομαι.
\VS{59}Καὶ ἐξέπεμψαν Ῥεβέκκαν τὴν ἀδελφὴν αὐτῶν, καὶ τὰ ὑπάρχοντα αὐτῆς, καὶ τὸν παῖδα τοῦ Ἁβραὰμ, καὶ τοὺς μετʼ αὐτοῦ.
\VS{60}Καὶ εὐλόγησαν Ῥεβέκκαν, καὶ εἶπαν αὐτῇ, ἀδελφὴ ἡμῶν εἶ, γίνου εἰς χιλιάδας μυριάδων, καὶ κληρονομησάτω τὸ σπέρμα σου τὰς πόλεις τῶν ὑπεναντίων.
\VS{61}Ἀναστᾶσα δὲ Ῥεβέκκα καὶ αἱ ἅβραι αὐτῆς, ἐπέβησαν ἐπὶ τὰς καμήλους, καὶ ἐπορεύθησαν μετὰ τοῦ ἀνθρώπου· καὶ ἀναλαβὼν ὁ παῖς τὴν Ῥεβέκκαν ἀπῆλθεν.
\par }{\PP \VS{62}Ἰσαὰκ δὲ διεπορεύετο διὰ τῆς ἐρήμου κατὰ τὸ φρέαρ τῆς ὁράσεως· αὐτὸς δὲ κατῴκει ἐν τῇ γῇ τῇ πρὸς Λίβα.
\VS{63}Καὶ ἐξῆλθεν Ἰσαὰκ ἀδολεσχῆσαι εἰς τὸ πεδίον τὸ πρὸς δείλης, καὶ ἀναβλέψας τοῖς ὀφθαλμοῖς αὐτοῦ εἶδε καμήλους ἐρχομένας.
\VS{64}Καὶ ἀναβλέψασα Ῥεβέκκα τοῖς ὀφθαλμοῖς εἶδε τὸν Ἰσαάκ· καὶ κατεπήδησεν ἀπὸ τῆς καμήλου.
\VS{65}Καὶ εἶπε τῷ παιδὶ, τίς ἐστιν ὁ ἄνθρωπος ἐκεῖνος ὁ πορευόμενος ἐν τῷ πεδίῳ εἰς συνάντησιν ἡμῖν; εἶπε δὲ ὁ παῖς, οὗτός ἐστιν ὁ κύριός μου· ἡ δὲ λαβοῦσα τὸ θέριστρον, περιεβάλετο.
\VS{66}Καὶ διηγήσατο ὁ παῖς τῷ Ἰσαὰκ πάντα τὰ ῥήματα, ἃ ἐποίησεν.
\VS{67}Εἰσῆλθε δὲ Ἰσαὰκ εἰς τὸν οἶκον τῆς μητρὸς αὐτοῦ, καὶ ἔλαβε τὴν Ῥεβέκκαν, καὶ ἐγένετο αὐτοῦ γυνὴ, καὶ ἠγάπησεν αὐτήν· καὶ παρεκλήθη Ἰσαὰκ περὶ Σάῤῥας τῆς μητρὸς αὐτοῦ.

\par }\Chap{25}{\PP \VerseOne{1}Προσθέμενος δὲ Ἁβραὰμ ἔλαβε γυναῖκα, ᾗ ὄνομα Χεττούρα.
\VS{2}Ἔτεκε δὲ αὐτῷ τὸν Ζομβρᾶν, καὶ τὸν Ἰεζὰν, καὶ τὸν Μαδὰλ, καὶ τὸν Μαδιὰμ, καὶ τὸν Ἰεσβὼκ, καὶ τὸν Σωίε.
\VS{3}Ἰεζὰν δὲ ἐγέννησε τὸν Σαβὰ, καὶ τὸν Δεδάν· υἱοὶ δὲ Δεδὰν Ἀσσουριεὶμ, καὶ Λατουσιεὶμ, καὶ Λαωμείμ.
\VS{4}Υἱοὶ δὲ Μαδιὰμ Γεφὰρ, καὶ Ἀφεὶρ, καὶ Ἐνὼχ, καὶ Ἀβειδὰ, καὶ Ἐλδαγά· πάντες οὗτοι ἦσαν υἱοὶ Χεττούρας.
\VS{5}Ἔδωκε δὲ Ἁβραὰμ πάντα τὰ ὑπάρχοντα αὐτοῦ Ἰσαὰκ τῷ υἱῷ αὐτοῦ.
\VS{6}Καὶ τοῖς υἱοῖς τῶν παλλακῶν αὐτοῦ ἔδωκεν Ἁβραὰμ δόματα, καὶ ἐξαπέστειλεν αὐτοὺς ἀπὸ Ἰσαὰκ τοῦ υἱοῦ αὐτοῦ, ἔτι ζῶντος αὐτοῦ, πρὸς ἀνατολὰς εἰς γῆν ἀνατολῶν.
\VS{7}Ταῦτα δὲ τὰ ἔτη ἡμερῶν τῆς ζωῆς Ἁβραὰμ ὅσα ἔζησεν, ἑκατὸν ἑβδομηκονταπεντε ἔτη.
\VS{8}Καὶ ἐκλείπων ἀπέθανεν Ἁβραὰμ ἐν γήρᾳ καλῷ πρεσβύτης, καὶ πλήρης ἡμερῶν, καὶ προσετέθη πρὸς τὸν λαὸν αὐτοῦ.
\VS{9}Καὶ ἔθαψαν αὐτὸν Ἰσαὰκ καὶ Ἰσμαὴλ οἱ υἱοὶ αὐτοῦ εἰς τὸ σπήλαιον τὸ διπλοῦν, εἰς τὸν ἀγρὸν Ἐφρων τοῦ Σαὰρ τοῦ Χετταίου, ὅς ἐστιν ἀπέναντι Μαμβρῆ,
\VS{10}τὸν ἀγρὸν καὶ τὸ σπήλαιον, ὃ ἐκτήσατο Ἁβραὰμ παρὰ τῶν υἱῶν τοῦ Χέτ· ἐκεῖ ἔθαψαν Ἁβραὰμ, καὶ Σάῤῥαν τὴν γυναῖκα αὐτοῦ.
\VS{11}Ἐγένετο δὲ μετὰ τὸ ἀποθανεῖν Ἁβραὰμ, εὐλόγησεν ὁ Θεὸς τὸν Ἰσαὰκ υἱὸν αὐτοῦ· καὶ κατῴκησεν Ἰσαὰκ παρὰ τὸ φρέαρ τῆς ὁράσεως.
\VS{12}Αὗται δὲ αἱ γενέσεις Ἰσμαὴλ τοῦ υἱοῦ Ἁβραὰμ, ὃν ἔτεκεν Ἄγαρ ἡ Αἰγυπτία, ἡ παιδίσκη Σάῤῥας, τῷ Ἁβραάμ.
\VS{13}Καὶ ταῦτα τὰ ὀνόματα τῶν υἱῶν Ἰσμαὴλ, κατʼ ὀνόματα τῶν γενεῶν αὐτοῦ· πρωτότοκος Ἰσμαὴλ, καὶ Ναβαϊὼθ, καὶ Κηδὰρ, καὶ Ναβδεὴλ, καὶ Μασσὰμ,
\VS{14}καὶ Μασμὰ, καὶ Δουμὰ, καὶ Μασσῆ,
\VS{15}καὶ Χοδδὰν, καὶ Θαιμὰν, καὶ Ἰετοὺρ, καὶ Ναφὲς, καὶ Κεδμά.
\VS{16}οὗτοί εἰσιν οἱ υἱοὶ Ἰσμαὴλ, καὶ ταῦτα τὰ ὀνόματα αὐτῶν ἐν ταῖς σκηναῖς αὐτῶν, καὶ ἐν ταῖς ἐπαύλεσιν αὐτῶν· δώδεκα ἄρχοντες κατὰ ἔθνη αὐτῶν.
\VS{17}Καὶ ταῦτα τὰ ἔτη τῆς ζωῆς Ἰσμαὴλ, ἑκατὸν τριακονταεπτὰ ἔτη· καὶ ἐκλείπων ἀπέθανε, καὶ προσετέθη πρὸς τὸ γένος αὐτοῦ.
\VS{18}Κατῴκησε δὲ ἀπὸ Εὐϊλὰτ ἕως Σοὺρ, ἥ ἐστι κατὰ πρόσωπον Αἰγύπτου ἕως ἐλθεῖν πρὸς Ἀσσυρίους· κατὰ πρόσωπον πάντων τῶν ἀδελφῶν αὐτοῦ κατῴκησε.
\par }{\PP \VS{19}Καὶ αὗται αἱ γενέσεις Ἰσαὰκ τοῦ υἱοῦ Ἁβραάμ· Ἁβραάμ ἐγέννησε τὸν Ἰσαάκ.
\VS{20}Ἦν δὲ Ἰσαὰκ ἐτῶν τεσσαράκοντα ὅτε ἔλαβε τὴν Ῥεβέκκαν θυγατέρα Βαθουὴλ τοῦ Σύρου ἐκ τῆς Μεσοποταμίας Συρίας, ἀδελφὴν Λάβαν τοῦ Σύρου, ἑαυτῷ εἰς γυναῖκα.
\VS{21}Ἐδέετο δὲ Ἰσαὰκ Κυρίου περὶ Ῥεβέκκας τῆς γυναικὸς αὐτοῦ, ὅτι στεῖρα ἦν· ἐπήκουσε δὲ αὐτοῦ ὁ Θεὸς, καὶ συνέλαβεν ἐν γαστρὶ Ῥεβέκκα ἡ γυνὴ αὐτοῦ.
\VS{22}Ἐσκίρτων δὲ τὰ παιδία ἐν αὐτῇ· εἶπε δὲ, εἰ οὕτω μοι μέλλει γίνεσθαι, ἵνα τί μοι τοῦτο; ἐπορεύθη δὲ πυθέσθαι παρὰ Κυρίου.
\VS{23}Καὶ εἶπε Κύριος αὐτῇ, δύο ἔθνη ἐν γαστρί σου εἰσὶ, καὶ δύο λαοὶ ἐκ τῆς κοιλίας σου διασταλήσονται· καὶ λαὸς λαοῦ ὑπερέξει, καὶ ὁ μείζων δουλεύσει τῷ ἐλάσσονι.
\VS{24}Καὶ ἐπληρώθησαν αἱ ἡμέραι τοῦ τεκεῖν αὐτήν· καὶ τῇδε ἦν δίδυμα ἐν τῇ κοιλίᾳ αὐτῆς.
\VS{25}Ἐξῆλθε δὲ ὁ πρωτότοκος πυῤῥάκης· ὅλος, ὡσεὶ δορὰ, δασύς· ἐπωνόμασε δὲ τὸ ὄνομα αὐτοῦ, Ἡσαῦ.
\VS{26}Καὶ μετὰ τοῦτο ἐξῆλθεν ὁ ἀδελφὸς αὐτοῦ, καὶ ἡ χεὶρ αὐτοῦ ἐπειλημμένη τῆς πτέρνης Ἡσαῦ· καὶ ἐκάλεσε τὸ ὄνομα αὐτοῦ, Ἰακώβ. Ἰσαὰκ δὲ ἦν ἐτῶν ἑξήκοντα, ὅτε ἔτεκεν αὐτοὺς Ῥεβέκκα.
\VS{27}Ηὐξήθησαν δὲ οἱ νεανίσκοι· καὶ ἦν Ἡσαῦ ἄνθρωπος εἰδὼς κυνηγεῖν, ἄγροικος· Ἰακὼβ δὲ ἄνθρωπος ἄπλαστος, οἰκῶν οἰκίαν.
\VS{28}Ἠγάπησε δὲ Ἰσαὰκ τὸν Ἡσαῦ, ὅτι ἡ θήρα αὐτοῦ βρῶσις αὐτῷ· Ῥεβέκκα δὲ ἠγάπα τὸν Ἰακώβ.
\par }{\PP \VS{29}Ἥψησε δὲ Ἰακὼβ ἕψημα· ἦλθε δὲ Ἡσαῦ ἐκ τοῦ πεδίου ἐκλείπων.
\VS{30}Καὶ εἶπεν Ἡσαῦ τῷ Ἰακὼβ, γεῦσόν με ἀπὸ τοῦ ἑψήματος πυῤῥου τούτου, ὅτι ἐκλείπω· διὰ τοῦτο ἐκλήθη τὸ ὄνομα αὐτοῦ, Ἐδώμ.
\VS{31}Εἶπε δὲ Ἰακὼβ τῷ Ἡσαῦ, ἀπόδου μοι σήμερον τὰ πρωτοτόκιά σου ἐμοί.
\VS{32}Καὶ εἶπεν Ἡσαῦ, ἰδοὺ ἐγὼ πορεύομαι τελευτᾷν· καὶ ἵνα τί μοι ταῦτα τὰ πρωτοτόκια;
\VS{33}Καὶ εἶπεν αὐτῷ Ἰακὼβ, ὄμοσόν μοι σήμερον· καὶ ὤμοσεν αὐτῷ· ἀπέδοτο δὲ Ἡσαῦ τὰ πρωτοτόκια τῷ Ἰακώβ.
\VS{34}Ἰακὼβ δὲ ἔδωκε τῷ Ἠσαῦ ἄρτον, καὶ ἕψημα φακοῦ· καὶ ἔφαγε καὶ ἔπιε, καὶ ἀναστὰς ᾤχετο· καὶ ἐφαύλισεν Ἡσαῦ τὰ πρωτοτόκια.

\par }\Chap{26}{\PP \VerseOne{1}Ἐγένετο δὲ λιμὸς ἐπὶ τῆς γῆς, χωρὶς τοῦ λιμοῦ τοῦ πρότερον, ὃς ἐγένετο ἐν τῷ καιρῷ τοῦ Ἁβραάμ· ἐπορεύθη δὲ Ἰσαὰκ πρὸς Ἀβιμέλεχ βασιλέα Φυλιστιεὶμ εἰς Γέραρα.
\VS{2}Ὤφθη δὲ αὐτῷ Κύριος, καὶ εἶπε, μὴ καταβῇς εἰς Αἴγυπτον· κατοίκησον δὲ ἐν τῇ γῇ, ᾗ ἄν σοι εἴπω.
\VS{3}Καὶ παροίκει ἐν τῇ γῇ ταύτῃ, καὶ ἔσομαι μετὰ σοῦ, καὶ εὐλογήσω σε· σοὶ γὰρ καὶ τῷ σπέρματί σου δώσω πᾶσαν τὴν γῆν ταύτην· καὶ στήσω τὸν ὅρκον μου, ὅν ὤμοσα τῷ Ἁβραὰμ τῷ πατρί σου.
\VS{4}Καὶ πληθυνῶ τὸ σπέρμα σου, ὡς τοὺς ἀστέρας τοῦ οὐρανοῦ· καὶ δώσω τῷ σπέρματί σου πᾶσαν τὴν γῆν ταύτην· καὶ εὐλογηθήσονται ἐν τῷ σπέρματί σου πάντα τὰ ἔθνη τῆς γῆς.
\VS{5}Ἀνθʼ ὧν ὑπήκουσεν Ἁβραὰμ ὁ πατήρ σου τῆς ἐμῆς φωνῆς, καὶ ἐφύλαξε τὰ προστάγματά μου, καὶ τὰς ἐντολάς μου, καὶ τὰ δικαιώματά μου, καὶ τὰ νόμιμά μου.
\VS{6}Κατῴκησε δὲ Ἰσαὰκ ἐν Γεράροις.
\VS{7}Ἐπηρώτησαν δὲ οἱ ἄνδρες τοῦ τόπου περὶ Ῥεβέκκας τῆς γυναικὸς αὐτοῦ, καὶ εἶπεν, ἀδελφή μου ἐστίν· ἐφοβήθη γὰρ εἰπεῖν, ὅτι γυνή μου ἐστὶ, μή ποτε ἀποκτείνωσιν αὐτὸν οἱ ἄνδρες τοῦ τόπου περὶ Ῥεβέκκας, ὅτι ὡραία τῇ ὄψει ἦν.
\VS{8}Ἐγένετο δὲ πολυχρόνιος ἐκεῖ· καὶ παρακύψας Ἀβιμέλεχ ὁ βασιλεὺς Γεράρων διὰ τῆς θυρίδος, εἶδε τὸν Ἰσαὰκ παίζοντα μετὰ Ῥεβέκκας τῆς γυναικὸς αὐτοῦ.
\VS{9}Ἐκάλεσε δὲ Ἀβιμέλεχ τὸν Ἰσαὰκ, καὶ εἶπεν αὐτῷ, ἆρά γε γυνή σου ἐστί; τί ὅτι εἶπας, ἀδελφή μου ἐστίν; εἶπε δὲ αὐτῷ Ἰσαὰκ, εἶπα γὰρ, μή ποτε ἀποθάνω διʼ αὐτήν.
\VS{10}Εἶπε δὲ αὐτῷ Ἀβιμέλεχ, τί τοῦτο ἐποίησας ἡμῖν; μικροῦ ἐκοιμήθη τις ἐκ τοῦ γένους μου μετὰ τῆς γυναικός σου, καὶ ἐπήγαγες ἂν ἐφʼ ἡμᾶς ἄγνοιαν.
\VS{11}Συνέταξε δὲ Ἀβιμέλεχ παντὶ τῷ λαῷ αὐτοῦ, λέγων, πᾶς ὁ ἁψάμενος τοῦ ἀνθρώπου τούτου καὶ τῆς γυναικὸς αὐτοῦ, θανάτῳ ἔνοχος ἔσται.
\VS{12}Ἔσπειρε δὲ Ἰσαὰκ ἐν τῇ γῇ ἐκείνῃ, καὶ εὗρεν ἐν τῷ ἐνιαυτῷ ἐκείνῳ ἑκατοστεύουσαν κριθήν· εὐλόγησε δὲ αὐτὸν Κύριος.
\VS{13}Καὶ ὑψώθη ὁ ἄνθρωπος, καὶ προβαίνων μείζων ἐγένετο, ἕως οὗ μέγας ἐγένετο σφόδρα.
\VS{14}Ἐγένετο δὲ αὐτῷ κτήνη προβάτων, καὶ κτήνη βοῶν, καὶ γεώργια πολλά. ἐζήλωσαν δὲ αὐτὸν οἱ Φυλιστιείμ.
\VS{15}Καὶ πάντα τὰ φρέατα, ἃ ὤρυξαν οἱ παῖδες τοῦ πατρὸς αὐτοῦ ἐν τῷ χρόνῳ τοῦ πατρὸς αὐτοῦ, ἐνέφραξαν αὐτὰ οἱ Φυλιστιεὶμ, καὶ ἔπλησαν αὐτὰ γῆς.
\VS{16}Εἶπε δὲ Ἀβιμέλεχ πρὸς Ἰσαὰκ, ἄπελθε ἀφʼ ἡμῶν, ὅτι δυνατώτερος ἡμῶν ἐγένου σφόδρα.
\VS{17}Καὶ ἀπῆλθεν ἐκεῖθεν Ἰσαάκ· καὶ κατέλυσεν ἐν τῇ φάραγγι Γεράρων, καὶ κατῴκησεν ἐκεῖ.
\par }{\PP \VS{18}Καὶ πάλιν Ἰσαὰκ ὤρυξε τὰ φρέατα τοῦ ὕδατος, ἃ ὤρυξαν οἱ παῖδες Ἁβραὰμ τοῦ πατρὸς αὐτοῦ, καὶ ἐνέφραξαν αὐτὰ οἱ Φυλιστιεὶμ μετὰ τὸ ἀποθανεῖν Ἁβραὰμ τὸν πατέρα αὐτοῦ· καὶ ἐπωνόμασεν αὐτοῖς ὀνόματα κατὰ τὰ ὀνόματα, ἃ ὠνόμασεν ὁ πατὴρ αὐτοῦ.
\VS{19}Καὶ ὤρυξαν οἱ παῖδες Ἰσαὰκ ἐν τῇ φάραγγι Γεράρων· καὶ εὗρον ἐκεῖ φρέαρ ὕδατος ζῶντος.
\VS{20}Καὶ ἐμαχέσαντο οἱ ποιμένες Γεράρων μετὰ τῶν ποιμένων Ἰσαὰκ, φάσκοντες αὐτῶν εἶναι τὸ ὕδωρ· καὶ ἐκάλεσαν τὸ ὄνομα τοῦ φρέατος, Ἀδικία· ἠδίκησαν γὰρ αὐτόν.
\VS{21}Ἀπᾴρας δὲ ἐκεῖθεν ὤρυξε φρέαρ ἕτερον· ἐκρίνοντο δὲ καὶ περὶ ἐκείνου· καὶ ἐπωνόμασε τὸ ὄνομα αὐτοῦ, Ἐχθρία.
\VS{22}Ἀπᾴρας δὲ ἐκεῖθεν ὤρυξε φρέαρ ἕτερον· καὶ οὐκ ἐμαχέσαντο περὶ αὐτοῦ· καὶ ἐπωνόμασε τὸ ὄνομα αὐτοῦ, Εὐρυχωρία, λέγων, διότι νῦν ἐπλάτυνε Κύριος ἡμῖν, καὶ ηὔξησεν ἡμᾶς ἐπὶ τῆς γῆς.
\par }{\PP \VS{23}Ἀνέβη δὲ ἐκεῖθεν ἐπὶ τὸ φρέαρ τοῦ ὅρκου.
\VS{24}Καὶ ὤφθη αὐτῷ Κύριος ἐν τῇ νυκτὶ ἐκείνῃ, καὶ εἶπεν, ἐγώ εἰμι ὁ Θεὸς Ἁβραὰμ τοῦ πατρός σου· μὴ φοβοῦ, μετὰ σοῦ γάρ εἰμι, καὶ εὐλογήσω σε, καὶ πληθυνῶ τὸ σπέρμα σου διʼ Ἁβραὰμ τὸν πατέρα σου.
\VS{25}Καὶ ᾠκοδόμησεν ἐκεῖ θυσιαστήριον, καὶ ἐπεκαλέσατο τὸ ὄνομα Κυρίου, καὶ ἔπηξεν ἐκεῖ τὴν σκηνὴν αὐτοῦ· ὤρυξαν δὲ ἐκεῖ οἱ παῖδες Ἰσαὰκ φρέαρ ἐν τῇ φάραγγι Γεράρων.
\VS{26}Καὶ Ἀβιμέλεχ ἐπορεύθη πρὸς αὐτὸν ἀπὸ Γεράρων, καὶ Ὁχοζὰθ ὁ νυμφαγωγὸς αὐτοῦ, καὶ Φιχὼλ ὁ ἀρχιστράτηγος τῆς δυνάμεως αὐτοῦ.
\VS{27}Καὶ εἶπεν αὐτοῖς Ἰσαὰκ, ἵνα τί ἤλθετε πρός με; ὑμεῖς δὲ ἐμισήσατέ με, καὶ ἐξαπεστείλατέ με ἀφʼ ὑμῶν.
\VS{28}Οἱ δὲ εἶπαν, ἰδόντες ἑωράκαμεν ὅτι ἦν Κύριος μετὰ σοῦ· καὶ εἴπαμεν, γενέσθω ἀρὰ ἀνὰ μέσον ἡμῶν καὶ ἀνὰ μέσον σου, καὶ διαθησόμεθα μετὰ σοῦ διαθήκην,
\VS{29}Μὴ ποιήσαι μεθʼ ἡμῶν κακὸν, καθότι οὐκ ἐβδελυξάμεθά σε ἡμεῖς, καὶ ὃν τρόπον ἐχρησάμεθά σοι καλῶς, καὶ ἐξαπεστείλαμέν σε μετʼ εἰρήνης· καὶ νῦν εὐλογημένος σὺ ὑπὸ Κυρίου.
\VS{30}Καὶ ἐποίησεν αὐτοῖς δοχὴν, καὶ ἔφαγον καὶ ἔπιον.
\VS{31}Καὶ ἀναστάντες τὸ πρωῒ, ὤμοσεν ἕκαστος τῷ πλησίον· καὶ ἐξαπέστειλεν αὐτοὺς Ἰσαάκ· καὶ ἀπῴχοντο ἀπʼ αὐτοῦ μετὰ σωτηρίας.
\VS{32}Ἐγένετο δὲ ἐν τῇ ἡμέρᾳ ἐκείνῃ, καὶ παραγενόμενοι οἱ παῖδες Ἰσαὰκ ἀπήγγειλαν αὐτῷ περὶ τοῦ φρέατος οὗ ὤρυξαν, καὶ εἶπαν, οὐχ εὕρομεν ὕδωρ.
\VS{33}Καὶ ἐκάλεσεν αὐτὸ, Ὅρκος· διὰ τοῦτο ἐκάλεσεν ὄνομα τῇ πόλει ἐκείνῃ, Φρέαρ Ὅρκου, ἕως τῆς σήμερον ἡμέρας.
\par }{\PP \VS{34}Ἦν δὲ Ἡσαῦ ἐτῶν τεσσαράκοντα, καὶ ἔλαβε γυναῖκα Ἰουδὶθ, θυγατέρα Βεὼχ τοῦ Χετταίου, καὶ τὴν Βασεμὰθ, θυγατέρα Ἑλὼν Χετταίου.
\VS{35}Καὶ ἦσαν ἐρίζουσαι τῷ Ἰσαὰκ καὶ τῇ Ῥεβέκκᾳ.

\par }\Chap{27}{\PP \VerseOne{1}Ἐγένετο δὲ μετὰ τὸ γηράσαι τὸν Ἰσαὰκ, καὶ ἠμβλύνθησαν οἱ ὀφθαλμοὶ αὐτοῦ τοῦ ὁρᾷν, καὶ ἐκάλεσεν Ἡσαῦ τὸν υἱὸν αὐτοῦ τὸν πρεσβύτερον, καὶ εἶπεν αὐτῷ, υἱέ μου· καὶ εἶπεν, ἰδοὺ ἐγώ.
\VS{2}Καὶ εἶπεν, ἰδοὺ γεγήρακα, καὶ οὐ γινώσκω τὴν ἡμέραν τῆς τελευτῆς μου.
\VS{3}Νῦν οὖν λάβε τὸ σκεῦός σου, τήν τε φαρέτραν, καὶ τὸ τόξον, καὶ ἔξελθε εἰς τὸ πεδίον, καὶ θήρευσόν μοι θήραν.
\VS{4}Καὶ ποίησόν μοι ἐδέσματα, ὡς φιλῶ ἐγὼ, καὶ ἔνεγκέ μοι, ἵνα φάγω, ὅπως εὐλογήσῃ σε ἡ ψυχή μου πρὶν ἀποθανεῖν με.
\VS{5}Ῥεβέκκα δὲ ἤκουσε λαλοῦντος Ἰσαὰκ πρὸς Ἡσαῦ τὸν υἱὸν αὐτοῦ· ἐπορεύθη δὲ Ἡσαῦ εἰς τὸ πεδίον θηρεῦσαι θήραν τῷ πατρὶ αὐτοῦ.
\VS{6}Ῥεβέκκα δὲ εἶπε πρὸς τὸν Ἰακὼβ τὸν υἱὸν αὐτῆς τὸν ἐλάσσω, ἴδε, ἤκουσα τοῦ πατρός σου λαλοῦντος πρὸς Ἡσαῦ τὸν ἀδελφόν σου, λέγοντος,
\VS{7}Ἔνεγκόν μοι θήραν, καὶ ποίησόν μοι ἐδέσματα, ἵνα φαγὼν εὐλογήσω σε ἐναντίον Κυρίου πρὸ τοῦ ἀποθανεῖν με.
\VS{8}Νῦν οὖν, υἱέ μου, ἄκουσόν μου, καθὰ ἐγώ σοι ἐντέλλομαι.
\VS{9}Καὶ πορευθεὶς εἰς τὰ πρόβατα, λάβε μοι ἐκεῖθεν δύο ἐρίφους ἁπαλοὺς καὶ καλοὺς, καὶ ποιήσω αὐτοὺς ἐδέσματα τῷ πατρί σου, ὡς φιλεῖ.
\VS{10}Καὶ εἰσοίσεις τῷ πατρί σου, καὶ φάγεται, ὅπως εὐλογήσῃ σε ὁ πατήρ σου πρὸ τοῦ ἀποθανεῖν αὐτόν.
\VS{11}Εἶπε δὲ Ἰακὼβ πρὸς Ῥεβέκκαν τὴν μητέρα αὐτοῦ, ἔστιν Ἡσαῦ ὁ ἀδελφός μου ἀνὴρ δασὺς, ἐγὼ δὲ ἀνὴρ λεῖος.
\VS{12}Μή ποτε ψηλαφήσῃ με ὁ πατὴρ, καὶ ἔσομαι ἐναντίον αὐτοῦ ὡς καταφρονῶν, καὶ ἐπάξω ἐπʼ ἐμαυτὸν κατάραν, καὶ οὐκ εὐλογίαν.
\VS{13}Εἶπε δὲ αὐτῷ ἡ μήτηρ, ἐπʼ ἐμὲ ἡ κατάρα σου, τέκνον· μόνον ἐπάκουσόν μου τῆς φωνῆς, καὶ πορευθεὶς ἔνεγκέ μοι.
\VS{14}Πορευθεὶς δὲ ἔλαβε, καὶ ἤνεγκε τῇ μητρί· καὶ ἐποίησεν ἡ μήτηρ αὐτοῦ ἐδέσματα, καθὰ ἐφίλει ὁ πατὴρ αὐτοῦ.
\par }{\PP \VS{15}Καὶ λαβοῦσα Ῥεβέκκα τὴν στολὴν Ἡσαῦ τοῦ υἱοῦ αὐτῆς τοῦ πρεσβυτέρου τὴν καλὴν, ἣ ἦν παρʼ αὐτῇ ἐν τῷ οἴκῳ, ἐνέδυσεν αὐτὴν Ἰακὼβ τὸν υἱὸν αὐτῆς τὸν νεώτερον.
\VS{16}Καὶ τὰ δέρματα τῶν ἐρίφων περιέθηκεν ἐπὶ τοὺς βραχίονας αὐτοῦ, καὶ ἐπὶ τὰ γυμνὰ τοῦ τραχήλου αὐτοῦ.
\VS{17}Καὶ ἔδωκε τὰ ἐδέσματα, καὶ τοὺς ἄρτους οὓς ἐποίησεν, εἰς τὰς χεῖρας Ἰακὼβ τοῦ υἱοῦ αὐτῆς.
\VS{18}Καὶ εἰσήνεγκε τῷ πατρὶ αὐτοῦ· εἶπε δὲ, πάτερ· ὁ δὲ εἶπεν, ἰδοὺ ἐγώ· τίς εἶ σὺ, τέκνον;
\VS{19}Καὶ εἶπεν Ἰακὼβ τῷ πατρὶ, ἐγὼ Ἡσαῦ ὁ πρωτότοκός σου πεποίηκα καθὰ ἐλάλησάς μοι· ἀναστὰς κάθισον, καὶ φάγε ἀπὸ τῆς θήρας μου, ὅπως εὐλογήσῃ με ἡ ψυχή σου.
\VS{20}Εἶπε δὲ Ἰσαὰκ τῷ υἱῷ αὐτοῦ, τί τοῦτο, ὃ ταχὺ εὗρες, ὦ τέκνον; ὁ δὲ εἶπεν, ὃ παρέδωκε Κύριος ὁ Θεός σου ἐναντίον μου.
\VS{21}Εἶπε δὲ Ἰσαὰκ τῷ Ἰακὼβ, ἔγγισόν μοι, καὶ ψηλαφήσω σε, τέκνον, εἰ σὺ εἶ ὁ υἱός μου Ἡσαῦ, ἢ οὔ.
\VS{22}Ἤγγισε δὲ Ἰακὼβ πρὸς Ἰσαὰκ τὸν πατέρα αὐτοῦ· καὶ ἐψηλάφησεν αὐτὸν, καὶ εἶπεν, ἡ μὲν φωνὴ, φωνὴ Ἰακὼβ, αἱ δὲ χεῖρες, χεῖρες Ἡσαῦ.
\VS{23}Καὶ οὐκ ἐπέγνω αὐτὸν, ἦσαν γὰρ αἱ χεῖρες αὐτοῦ, ὡς αἱ χεῖρες Ἡσαῦ τοῦ ἀδελφοῦ αὐτοῦ, δασεῖαι· καὶ εὐλόγησεν αὐτὸν,
\VS{24}καὶ εἶπε, σὺ εἶ ὁ υἱός μου Ἡσαῦ; ὁ δὲ εἶπεν, ἐγώ.
\VS{25}Καὶ εἶπε, προσάγαγέ μοι, καὶ φάγομαι ἀπὸ τῆς θήρας σου, τέκνον, ἵνα εὐλογήσῃ σε ἡ ψυχή μου· καὶ προσήνεγκεν αὐτῷ, καὶ ἔφαγε· καὶ εἰσήνεγκεν αὐτῷ οἶνον, καὶ ἔπιε.
\VS{26}Καὶ εἴπεν αὐτῷ Ἰσαὰκ ὁ πατὴρ αὐτοῦ, ἔγγισόν μοι, καὶ φίλησόν με, τέκνον.
\VS{27}Καὶ ἐγγίσας ἐφίλησεν αὐτόν· καὶ ὠσφράνθη τὴν ὀσμὴν τῶν ἱματίων αὐτοῦ, καὶ εὐλόγησεν αὐτὸν, καὶ εἶπεν, ἰδοὺ ὀσμὴ τοῦ υἱοῦ μου, ὡς ὀσμὴ ἀγροῦ πλήρους, ὃν εὐλόγησε Κύριος.
\VS{28}Καὶ δῴη σοι ὁ Θεὸς ἀπὸ τῆς δρόσου τοῦ οὐρανοῦ, καὶ ἀπὸ τῆς πιότητος τῆς γῆς, καὶ πλῆθος σίτου καὶ οἴνου.
\VS{29}Καὶ δουλευσάτωσάν σοι ἔθνη, καὶ προσκυνησάτωσάν σοι ἄρχοντες· καὶ γίνου κύριος τοῦ ἀδελφοῦ σου, καὶ προσκυνήσουσί σοι οἱ υἱοὶ τοῦ πατρός σου· ὁ καταρώμενός σε, ἐπικατάρατος· ὁ δὲ εὐλογῶν σε, εὐλογημένος.
\par }{\PP \VS{30}Καὶ ἐγένετο μετὰ τὸ παύσασθαι Ἰσαὰκ εὐλογοῦντα Ἰακὼβ τὸν υἱὸν αὐτοῦ, καὶ ἐγένετο, ὡς ἂν ἐξῆλθεν Ἰακὼβ ἀπὸ προσώπου Ἰσαὰκ τοῦ πατρὸς αὐτοῦ, καὶ Ἡσαῦ ὁ ἀδελφὸς αὐτοῦ ἦλθεν ἀπὸ τῆς θήρας.
\VS{31}Καὶ ἐποίησε καὶ αὐτὸς ἐδέσματα, καὶ προσήνεγκε τῷ πατρὶ αὐτοῦ· καὶ εἶπε τῷ πατρὶ, ἀναστήτω ὁ πατήρ μου, καὶ φαγέτω ἀπὸ τῆς θήρας τοῦ υἱοῦ αὐτοῦ, ὅπως εὐλογήσῃ με ἡ ψυχή σου.
\VS{32}Καὶ εἶπεν αὐτῷ Ἰσαὰκ ὁ πατὴρ αὐτοῦ, τίς εἶ σύ; ὁ δὲ εἶπεν, ἐγώ εἰμι ὁ υἱός σου ὁ πρωτότοκος Ἡσαῦ.
\VS{33}Ἐξέστη δὲ Ἰσαὰκ ἔκστασιν μεγάλην σφόδρα, καὶ εἶπε, τίς οὖν ὁ θηρεύσας μοι θήραν καὶ εἰσενέγκας μοι, καὶ ἔφαγον ἀπὸ πάντων πρὸ τοῦ ἐλθεῖν σε; καὶ εὐλόγησα αὐτὸν, καὶ εὐλογημένος ἔσται.
\VS{34}Ἐγένετο δὲ ἡνίκα ἤκουσεν Ἡσαῦ τὰ ῥήματα τοῦ πατρὸς αὐτοῦ Ἰσαὰκ, ἀνεβόησε φωνὴν μεγάλην καὶ πικρὰν σφόδρα· καὶ εἶπεν, εὐλόγησον δὴ κᾀμὲ, πάτερ.
\VS{35}Εἶπε δὲ αὐτῷ, ἐλθὼν ὁ ἀδελφός σου μετὰ δόλου ἔλαβε τὴν εὐλογίαν σου.
\VS{36}Καὶ εἶπε, δικαίως ἐκλήθη τὸ ὄνομα αὐτοῦ Ἰακὼβ, ἐπτέρνικε γάρ με ἰδοὺ δεύτερον τοῦτο· τά τε πρωτοτόκιά μου εἴληφε, καὶ νῦν ἔλαβε τὴν εὐλογίαν μου· καὶ εἶπεν Ἡσαῦ τῷ πατρὶ αὐτοῦ, οὐχ ὑπελίπου μοι εὐλογίαν, πάτερ;
\VS{37}Ἀποκριθεὶς δὲ Ἰσαὰκ εἶπε τῷ Ἡσαῦ, εἰ κύριον αὐτὸν πεποίηκά σου, καὶ πάντας τοὺς ἀδελφοὺς αὐτοῦ πεποίηκα αὐτοῦ οἰκέτας· σίτῳ καὶ οἴνῳ ἐστήριξα αὐτόν· σοὶ δὲ τί ποιήσω, τέκνον;
\VS{38}Εἶπε δὲ Ἡσαῦ πρὸς τὸν πατέρα αὐτοῦ, μὴ εὐλογία μία σοι ἔστι, πάτερ; εὐλόγησον δὴ κᾀμὲ, πάτερ· κατανυχθέντος δὲ Ἰσαὰκ, ἀνεβόησε φωνῇ Ἡσαῦ, καὶ ἔκλαυσεν.
\VS{39}Ἀποκοιθεὶς δὲ Ἰσαὰκ ὁ πατὴρ αὐτοῦ εἶπεν αὐτῷ, ἰδοὺ ἀπὸ τῆς πιότητος τῆς γῆς ἔσται ἡ κατοίκησίς σου, καὶ ἀπὸ τῆς δρόσου τοῦ οὐρανοῦ ἄνωθεν.
\VS{40}Καὶ ἐπὶ τῇ μαχαίρᾳ σου ζήσῃ, καὶ τῷ ἀδελφῷ σου δουλεύσεις· ἔσται δὲ ἡνίκα ἐὰν καθέλῃς καὶ ἐκλύσῃς τὸν ζυγὸν αὐτοῦ ἀπὸ τοῦ τραχήλου σου.
\par }{\PP \VS{41}Καὶ ἐνεκότει Ἡσαῦ τῷ Ἰακὼβ περὶ τῆς εὐλογίας, ἧς εὐλόγησεν αὐτὸν ὁ πατὴρ αὐτοῦ· εἶπε δὲ Ἡσαῦ ἐν τῇ διανοίᾳ αὐτοῦ, ἐγγισάτωσαν αἱ ἡμέραι τοῦ πένθους τοῦ πατρός μου, ἵνα ἀποκτείνω Ἰακὼβ τὸν ἀδελφόν μου.
\VS{42}Ἀπηγγέλη δὲ Ῥεβέκκᾳ τὰ ῥήματα Ἡσαῦ τοῦ υἱοῦ αὐτῆς τοῦ πρεσβυτέρου· καὶ πέμψασα ἐκάλεσεν Ἰακὼβ τὸν υἱὸν αὐτῆς τὸν νεώτερον, καὶ εἶπεν αὐτῷ, ἰδοὺ Ἡσαῦ ὁ ἀδελφός σου ἀπειλεῖ σοι τοῦ ἀποκτεῖναί σε.
\VS{43}Νῦν οὖν, τέκνον, ἄκουσόν μου τῆς φωνῆς, καὶ ἀναστὰς ἀπόδραθι εἰς τὴν Μεσοποταμίαν πρὸς Λάβαν τὸν ἀδελφόν μου εἰς Χαῤῥάν.
\VS{44}Καὶ οἴκησον μετʼ αὐτοῦ ἡμέρας τινὰς, ἕως τοῦ ἀποστρέψαι τὸν θυμὸν,
\VS{45}καὶ τὴν ὀργὴν τοῦ ἀδελφοῦ σου ἀπὸ σοῦ, καὶ ἐπιλάθηται ἃ πεποίηκας αὐτῷ· καὶ ἀποστείλασα μεταπέμψομαί σε ἐκεῖθεν, μή ποτε ἀποτεκνωθῶ ἀπὸ τῶν δύο ὑμῶν ἐν ἡμέρᾳ μιᾷ.
\VS{46}Εἶπε δὲ Ῥεβέκκα πρὸς Ἰσαὰκ, προσώχθικα τῇ ζωῇ μου διὰ τὰς θυγατέρας τῶν υἱῶν Χέτ· εἰ λήψεται Ἰακὼβ γυναῖκα ἀπὸ τῶν θυγατέρων τῆς γῆς ταύτης, ἵνα τί μοι τὸ ζῇν;

\par }\Chap{28}{\PP \VerseOne{1}Προσκαλεσάμενος δὲ Ἰσαὰκ τὸν Ἰακὼβ, εὐλόγησεν αὐτὸν, καὶ ἐνετείλατο αὐτῷ, λέγων, οὐ λήψῃ γυναῖκα ἐκ τῶν θυγατέρων τῶν Χαναναίων.
\VS{2}Ἀναστὰς ἀπόδραθι εἰς τὴν Μεσοποταμίαν, εἰς τὸν οἶκον Βαθουὴλ τοῦ πατρὸς τῆς μητρός σου, καὶ λάβε σεαυτῷ ἐκεῖθεν γυναῖκα ἐκ τῶν θυγατέρων Λάβαν τοῦ ἀδελφοῦ τῆς μητρός σου.
\VS{3}Ὁ δὲ Θεός μου εὐλογήσαι σε, καὶ αὐξήσαι σε, καὶ πληθύναι σε· καὶ ἔσῃ εἰς συναγωγὰς ἐθνῶν.
\VS{4}Καὶ δῴη σοι τὴν εὐλόγιαν Ἁβραὰμ τοῦ πατρός μου, σοὶ καὶ τῷ σπέρματί σου μετὰ σὲ, κληρονομῆσαι τὴν γῆν τῆς παροικήσεώς σου, ἣν ἔδωκεν ὁ Θεὸς τῷ Ἁβραάμ.
\VS{5}Καὶ ἀπέστειλεν Ἰσαὰκ τὸν Ἰακώβ· καὶ ἐπορεύθη εἰς τὴν Μεσοποταμίαν πρὸς Λάβαν τὸν υἱὸν Βαθουὴλ τοῦ Σύρου, ἀδελφὸν Ῥεβέκκας τῆς μητρὸς Ἰακὼβ καὶ Ἡσαῦ.
\par }{\PP \VS{6}Ἴδε δὲ Ἡσαῦ ὅτι εὐλόγησεν Ἰσαὰκ τὸν Ἰακὼβ, καὶ ἀπέστειλεν εἰς τὴν Μεσοποταμίαν Συρίας, λαβεῖν ἑαυτῷ γυναῖκα ἐκεῖθεν, ἐν τῷ εὐλογεῖν αὐτόν· καὶ ἐνετείλατο αὐτῷ, λέγων, οὐ λήψῃ γυναῖκα ἐκ τῶν θυγατέρων τῶν Χαναναίων.
\VS{7}Καὶ ἤκουσεν Ἰακὼβ τοῦ πατρὸς καὶ τῆς μητρὸς αὐτοῦ· καὶ ἐπορεύθη εἰς τὴν Μεσοποταμίαν Συρίας.
\VS{8}Ἰδὼν δὲ καὶ Ἡσαῦ ὅτι πονηραί εἰσιν αἱ θυγατέρες Χαναὰν ἐναντίον Ἰσαὰκ τοῦ πατρὸς αὐτοῦ,
\VS{9}ἐπορεύθη Ἡσαῦ πρὸς Ἰσμαήλ· καὶ ἔλαβε τὴν Μαελὲθ, θυγατέρα Ἰσμαὴλ τοῦ υἱοῦ Ἁβραὰμ, ἀδελφὴν Ναβεὼθ, πρὸς ταῖς γυναιξὶν αὐτοῦ γυναῖκα.
\par }{\PP \VS{10}Καὶ ἐξῆλθεν Ἰακὼβ ἀπὸ τοῦ φρέατος τοῦ ὅρκου, καὶ ἐπορεύθη εἰς Χαῤῥάν.
\VS{11}Καὶ ἀπήντησε τόπῳ, καὶ ἐκοιμήθη ἐκεῖ, ἔδυ γὰρ ὁ ἥλιος· καὶ ἔλαβεν ἀπὸ τῶν λίθων τοῦ τόπου, καὶ ἔθηκε πρὸς κεφαλῆς αὐτοῦ· καὶ ἐκοιμήθη ἐν τῷ τόπῳ ἐκείνῳ.
\VS{12}Καὶ ἐνυπνιάσθη· καὶ ἰδοὺ κλίμαξ ἐστηριγμένη ἐν τῇ γῇ, ἧς ἡ κεφαλὴ ἀφικνεῖτο εἰς τὸν οὐρανόν· καὶ οἱ ἄγγελοι τοῦ θεοῦ ἀνέβαινον καὶ κατέβαινον ἐπʼ αὐτῇ.
\VS{13}Ὁ δὲ Κύριος ἐπεστήρικτο ἐπʼ αὐτῆς· καὶ εἶπεν, ἐγώ εἰμι ὁ Θεὸς Ἁβραὰμ τοῦ πατρός σου, καὶ ὁ Θεὸς Ἰσαάκ· μὴ φοβοῦ· ἡ γῆ ἐφʼ ἧς σὺ καθεύδεις ἐπʼ αὐτῆς, σοὶ δώσω αὐτὴν, καὶ τῷ σπέρματί σου.
\VS{14}Καὶ ἔσται τὸ σπέρμα σου ὡς ἡ ἄμμος τῆς γῆς, καὶ πλατυνθήσεται ἐπὶ θάλασσαν, καὶ Λίβα, καὶ Βοῤῥὰν, καὶ ἐπὶ ἀνατολάς· καὶ ἐνευλογηθήσονται ἐν σοὶ πᾶσαι αἱ φυλαὶ τῆς γῆς, καὶ ἐν τῷ σπέρματί σου.
\VS{15}Καὶ ἰδοὺ ἐγώ εἰμι μετὰ σοῦ, διαφυλάσσων σε ἐν τῇ ὁδῷ πάσῃ, οὗ ἂν πορευθῇς· καὶ ἀποστρέψω σε εἰς τὴν γῆν ταύτην· ὅτι οὐ μή σε ἐγκαταλίπω, ἕως τοῦ ποιῆσαί με πάντα ὅσα ἐλάλησά σοι.
\VS{16}Καὶ ἐξηγέρθη Ἰακὼβ ἐκ τοῦ ὕπνου αὐτοῦ, καὶ εἶπεν, ὅτι ἔστι Κύριος ἐν τῷ τόπῳ τούτῳ, ἐγὼ δὲ οὐκ ᾔδειν.
\VS{17}Καὶ ἐφοβήθη, καὶ εἶπεν, ὡς φοβερὸς ὁ τόπος οὗτος· οὐκ ἔστι τοῦτο ἀλλʼ ἢ οἶκος Θεοῦ, καὶ αὕτη ἡ πύλη τοῦ οὐρανοῦ.
\VS{18}Καὶ ἀνέστη Ἰακὼβ τὸ πρωῒ, καὶ ἔλαβε τὸν λίθον, ὃν ὑπέθηκεν ἐκεῖ πρὸς κεφαλῆς αὐτοῦ, καὶ ἔστησεν αὐτὸν στήλην, καὶ ἐπέχεεν ἔλαιον ἐπὶ τὸ ἄκρον αὐτῆς.
\VS{19}Καὶ ἐκάλεσε τὸ ὄνομα τοῦ τόπου ἐκείνου, οἶκος Θεοῦ· καὶ Οὐλαμλοὺζ ἦν ὄνομα τῇ πόλει τὸ πρότερον.
\VS{20}Καὶ ηὔξατο Ἰακὼβ εὐχὴν, λέγων, ἐὰν ᾖ Κύριος ὁ Θεὸς μετʼ ἐμοῦ, καὶ διαφυλάξῃ με ἐν τῇ ὁδῷ ταύτῃ, ᾗ ἐγὼ πορεύομαι, καὶ δῷ μοι ἄρτον φαγεῖν, καὶ ἱμάτιον περιβαλέσθαι,
\VS{21}καὶ ἀποστρέψῃ με μετὰ σωτηρίας εἰς τὸν οἶκον τοῦ πατρός μου, καὶ ἔσται Κύριός μοι εἰς Θεόν.
\VS{22}Καὶ ὁ λίθος οὗτος, ὃν ἔστησα στήλην, ἔσται μοι οἶκος Θεοῦ· καὶ πάντων ὧν ἐάν μοι δῷς, δεκάτην ἀποδεκατώσω αὐτά σοι.

\par }\Chap{29}{\PP \VerseOne{1}Καὶ ἐξᾴρας Ἰακὼβ τοὺς πόδας ἐπορεύθη εἰς γῆν ἀνατολῶν, πρὸς Λάβαν τὸν υἱὸν Βαθουὴλ τοῦ Σύρου, ἀδελφὸν δὲ Ῥεβέκκας, μητρὸς Ἰακὼβ καὶ Ἡσαῦ.
\VS{2}Καὶ ὁρᾷ, καὶ ἰδοὺ φρέαρ ἐν τῷ πεδίῳ· ἦσαν δὲ ἐκεῖ τρία ποίμνια προβάτων ἀναπαυόμενα ἐπʼ αὐτοῦ· ἐκ γὰρ τοῦ φρέατος ἐκείνου ἐπότιζον τὰ ποίμνια· λίθος δὲ ἦν μέγας ἐπὶ τῷ στόματι τοῦ φρέατος.
\VS{3}Καὶ συνήγοντο ἐκεῖ πάντα τὰ ποίμνια· καὶ ἀπεκύλιον τὸν λίθον ἀπὸ τοῦ στόματος τοῦ φρέατος, καὶ ἐπότιζον τὰ πρόβατα, καὶ ἀπεκαθίστων τὸν λίθον ἐπὶ τὸ στόμα τοῦ φρέατος εἰς τὸν τόπον αὐτοῦ.
\VS{4}Εἶπε δὲ αὐτοῖς Ἰακὼβ, ἀδελφοὶ, πόθεν ἐστὲ ὑμεῖς; οἱ δὲ εἶπαν, ἐκ Χαῤῥὰν ἐσμέν.
\VS{5}Εἶπε δὲ αὐτοῖς, γινώσκετε Λάβαν τὸν υἱὸν Ναχώρ; οἱ δὲ εἶπαν, γινώσκομεν·
\VS{6}Εἶπε δὲ αὐτοῖς, ὑγιαίνει; οἱ δὲ εἶπαν, ὑγιαίνει· καὶ ἰδοὺ Ῥαχὴλ ἡ θυγάτηρ αὐτοῦ ἤρχετο μετὰ τῶν προβάτων.
\VS{7}Καὶ εἶπεν Ἰακὼβ, ἔτι ἐστὶν ἡμέρα πολλὴ· οὔπω ὥρα συναχθῆναι τὰ κτήνη· ποτίσαντες τὰ πρόβατα, ἀπελθόντες βόσκετε.
\VS{8}Οἱ δὲ εἶπαν, οὐ δυνησόμεθα, ἕως τοῦ συναχθῆναι πάντας τοὺς ποιμένας, καὶ ἀποκυλίσουσι τὸν λίθον ἀπὸ τοῦ στόματος τοῦ φρέατος, καὶ ποτιοῦμεν τὰ πρόβατα.
\VS{9}Ἔτι αὐτοῦ λαλοῦντος αὐτοῖς, καὶ ἰδοὺ Ῥαχὴλ ἡ θυγάτηρ Λάβαν ἤρχετο μετὰ τῶν προβάτων τοῦ πατρὸς αὐτῆς· αὐτὴ γὰρ ἔβοσκε τὰ πρόβατα τοῦ πατρὸς αὐτῆς.
\VS{10}Ἐγένετο δὲ ὡς εἶδεν Ἰακὼβ τὴν Ῥαχὴλ τὴν θυγατέρα Λάβαν, τοῦ ἀδελφοῦ τῆς μητρὸς αὐτοῦ, καὶ τὰ πρόβατα Λάβαν τοῦ ἀδελφοῦ τῆς μητρὸς αὐτοῦ, καὶ προσελθὼν Ἰακὼβ ἀπεκύλισε τὸν λίθον ἀπὸ τοῦ στόματος τοῦ φρέατος, καὶ ἐπότιζε τὰ πρόβατα Λάβαν τοῦ ἀδελφοῦ τῆς μητρὸς αὐτοῦ.
\VS{11}Καὶ ἐφίλησεν Ἰακὼβ τὴν Ῥαχὴλ, καὶ βοήσας τῇ φωνῇ αὐτοῦ ἔκλαυσε.
\VS{12}Καὶ ἀπήγγειλε τῇ Ῥαχὴλ, ὅτι ἀδελφὸς τοῦ πατρὸς αὐτῆς ἐστι, καὶ ὅτι υἱὸς Ῥεβέκκας ἐστί· καὶ δραμοῦσα ἀπήγγειλε τῷ πατρὶ αὐτῆς κατὰ τὰ ῥήματα ταῦτα.
\VS{13}Ἐγένετο δὲ ὡς ἤκουσε Λάβαν τὸ ὄνομα Ἰακὼβ τοῦ υἱοῦ τῆς ἀδελφῆς αὐτοῦ, ἔδραμεν εἰς συνάντησιν αὐτῷ, καὶ περιλαβὼν αὐτὸν ἐφίλησε, καὶ εἰσήγαγεν αὐτὸν εἰς τὸν οἶκον αὐτοῦ· καὶ διηγήσατο τῷ Λάβαν πάντας τοὺς λόγους τούτους.
\VS{14}Καὶ εἶπεν αὐτῷ Λάβαν, ἐκ τῶν ὀστῶν μου καὶ ἐκ τῆς σαρκός μου εἶ σύ· καὶ ἦν μετʼ αὐτοῦ μῆνα ἡμερῶν.
\par }{\PP \VS{15}Εἶπε δὲ Λάβαν τῷ Ἰακὼβ, ὅτι γὰρ ἀδελφός μου εἶ, οὐ δουλεύσεις μοι δωρεάν· ἀπάγγειλόν μοι τίς ὁ μισθός σου ἐστί;
\VS{16}Τῷ δὲ Λάβαν ἦσαν δύο θυγατέρες· ὄνομα τῇ μείζονι, Λεία, καὶ ὄνομα τῇ νεωτέρᾳ, Ῥαχήλ.
\VS{17}Οἱ δὲ ὀφθάλμοὶ Λείας, ἀσθενεῖς· Ῥαχῆλ δὲ ἦν καλὴ τῷ εἴδει, καὶ ὡραία τῇ ὄψει σφάδρα.
\VS{18}Ἠγάπησε δὲ Ἰακὼβ τὴν Ῥαχήλ· καὶ εἶπε, δουλεύσω σοι ἑπτὰ ἔτη περὶ τῆς Ῥαχὴλ τῆς θυγατρός σου τῆς νεωτέρας.
\VS{19}Εἶπε δὲ αὐτῷ Λάβαν, βέλτιον δοῦναί με αὐτήν σοι, ἢ δοῦναί με αὐτὴν ἀνδρὶ ἑτέρῳ· οἴκησον μετʼ ἐμοῦ.
\VS{20}Καὶ ἐδούλευσεν Ἰακὼβ περὶ Ῥαχὴλ ἑπτὰ ἔτη· καὶ ἤσαν ἐναντίον αὐτοῦ ὡς ἡμέραι ὀλίγαι, παρὰ τὸ ἀγαπᾷν αὐτὸν αὐτήν.
\VS{21}Εἶπε δὲ Ἰακὼβ τῷ Λάβαν, δός μοι τὴν γυναῖκά μου, πεπλήρωνται γὰρ αἱ ἡμέραι ὅπως εἰσέλθω πρὸς αὐτήν.
\VS{22}Συνήγαγε δὲ Λάβαν πάντας τοὺς ἄνδρας τοῦ τόπου, καὶ ἐποίησε γάμον.
\VS{23}Καὶ ἐγένετο ἑσπέρα, καὶ λαβὼν Λείαν τὴν θυγατέρα αὐτοῦ, εἰσήγαγεν πρὸς Ἰακὼβ, καὶ εἰσῆλθε πρὸς αὐτὴν Ἰακώβ.
\VS{24}Ἔδωκε δὲ Λάβαν Λείᾳ τῇ θυγατρὶ αὐτοῦ Ζελφὰν τὴν παιδίσκην αὐτοῦ, αὐτῇ παιδίσκην.
\VS{25}Ἐγένετο δὲ πρωῒ, καὶ ἰδοὺ ἦν Λεία· εἶπε δὲ Ἰακὼβ τῷ Λάβαν, τί τοῦτο ἐποίησάς μοι; οὐ περὶ Ῥαχὴλ ἐδούλευσα παρὰ σοι; καὶ ἱνατί παρελογίσω με;
\VS{26}Ἀπεκρίθη δὲ Λάβαν, οὐκ ἔστιν οὕτως ἐν τῷ τόπῳ ἡμῶν, δοῦναι τὴν νεωτέραν πρινὴ τὴν πρεσβυτέραν.
\VS{27}Συντέλεσον οὖν τὰ ἕβδομα ταύτης, καὶ δώσω σοι καὶ ταύτην ἀντὶ τῆς ἐργασίας, ἧς ἐργᾷ παρʼ ἐμοὶ ἔτι ἑπτὰ ἔτη ἕτερα.
\VS{28}Ἐποίησε δὲ Ἰακὼβ οὕτως, καὶ ἀνεπλήρωσε τὰ ἕβδομα ταύτης· καὶ ἔδωκεν αὐτῷ Λάβαν Ῥαχὴλ τὴν θυγατέρα αὐτοῦ αὐτῷ γυναῖκα.
\VS{29}Ἔδωκε δὲ Λάβαν τῇ θυγατρὶ αὐτοῦ Βαλλὰν τὴν παιδίσκην αὐτοῦ, αὐτῇ παιδίσκην.
\VS{30}Καὶ εἰσῆλθε πρὸς Ῥαχήλ· ἠγάπησε δὲ Ῥαχὴλ μᾶλλον ἢ Λείαν· καὶ ἐδούλευσεν αὐτῷ ἑπτὰ ἔτη ἕτερα.
\par }{\PP \VS{31}Ἰδὼν δὲ Κύριος ὁ Θεὸς ὅτι ἐμισεῖτο Λεία, ἤνοιξε τὴν μήτραν αὐτῆς· Ῥαχὴλ δὲ ἦν στεῖρα.
\VS{32}Καὶ συνέλαβε Λεία, καὶ ἔτεκεν υἱὸν τῷ Ἰακώβ· ἐκάλεσε δὲ τὸ ὄνομα αὐτοῦ Ῥουβὴν, λέγουσα, διότι εἶδέ μου Κύριος τὴν ταπείνωσιν, καὶ ἔδωκέ μοι υἱόν· νῦν οὖν ἀγαπήσει με ὁ ἀνήρ μου.
\VS{33}Καὶ συνέλαβε πάλιν, καὶ ἔτεκεν υἱὸν δεύτερον τῷ Ἰακὼβ, καὶ εἶπεν, ὅτι ἤκουσε Κύριος ὅτι μισοῦμαι, καὶ προσέδωκέ μοι καὶ τοῦτον· καὶ ἐκάλεσε τὸ ὄνομα αὐτοῦ, Συμεών.
\VS{34}Καὶ συνέλαβεν ἔτι, καὶ ἔτεκεν υἱὸν, καὶ εἶπεν, ἐν τῷ νῦν καιρῷ πρὸς ἐμοῦ ἔσται ὁ ἀνήρ μου, τέτοκα γὰρ αὐτῷ τρεῖς υἱούς· διὰ τοῦτο ἐκάλεσε τὸ ὄνομα αὐτοῦ, Λευεί.
\VS{35}Καὶ συλλαβοῦσα ἔτι ἔτεκεν υἱὸν, καὶ εἶπε, νῦν ἔτι τοῦτο ἐξομολογήσομαι Κυρίῳ· διὰ τοῦτο ἐκάλεσε τὸ ὄνομα αὐτοῦ, Ἰούδαν· καὶ ἔστη τοῦ τίκτειν.

\par }\Chap{30}{\PP \VerseOne{1}Ἰδοῦσα δὲ Ῥαχὴλ, ὅτι οὐ τέτοκε τῷ Ἱακώβ· καὶ ἐζήλωσε Ῥαχὴλ τὴν ἀδελφὴν αὐτῆς· καὶ εἶπε τῷ Ἰακὼβ, δός μοι τέκνα· εἰ δὲ μὴ, τελευτήσω ἐγώ.
\VS{2}Θυμωθεὶς δὲ Ἰακὼβ τῇ Ῥαχὴλ εἶπεν αὐτῇ, μὴ ἀντὶ Θεοῦ ἐγώ εἰμι, ὃς ἐστέρησέ σε καρπὸν κοιλίας;
\VS{3}Εἶπε δὲ Ῥαχὴλ τῷ Ἰακὼβ, ἰδοὺ ἡ παιδίσκη μου Βαλλά· εἴσελθε πρὸς αὐτήν· καὶ τέξεται ἐπὶ τῶν γονάτων μου, καὶ τεκνοποιήσομαι κᾀγὼ ἐξ αὐτῆς.
\VS{4}Καὶ ἔδωκεν αὐτῷ Βαλλὰν τὴν παιδίσκην αὐτῆς, αὐτῷ γυναῖκα· καὶ εἰσῆλθε πρὸς αὐτὴν Ἰακώβ.
\VS{5}Καὶ συνέλαβε Βαλλὰ ἡ παιδίσκη Ῥαχὴλ, καὶ ἔτεκε τῷ Ἰακὼβ υἱόν.
\VS{6}Καὶ εἶπε Ῥαχὴλ, ἔκρινέ μοι ὁ Θεὸς, καὶ ἐπήκουσε τῆς φωνῆς μου, καὶ ἔδωκε μοι υἱόν· διὰ τοῦτο ἐκάλεσε τὸ ὄνομα αὐτοῦ, Δάν.
\VS{7}Καὶ συνέλαβεν ἔτι Βαλλὰ ἡ παιδίσκη Ῥαχὴλ, καὶ ἔτεκεν υἱὸν δεύτερον τῷ Ἰακώβ.
\VS{8}Καὶ εἶπε Ῥαχὴλ, συναντελάβετό μου ὁ Θεὸς, καὶ συνανεστράφην τῇ ἀδελφῇ μου, καὶ ἠδυνάσθην· καὶ ἐκάλεσε τὸ ὄνομα αὐτοῦ, Νεφθαλεί.
\VS{9}Εἶδε δὲ Λεία ὅτι ἔστη τοῦ τίκτειν· καὶ ἔλαβε Ζελφὰν τὴν παιδίσκην αὐτῆς, καὶ ἔδωκεν αὐτὴν τῷ Ἰακὼβ γυναῖκα· καὶ εἰσῆλθε πρὸς αὐτήν.
\VS{10}Καὶ συνέλαβε Ζελφὰ ἡ παιδίσκη Λείας, καὶ ἔτεκε τῷ Ἰακὼβ υἱόν.
\VS{11}Καὶ εἶπε Λεία, ἐν τύχῃ· καὶ ἐπωνόμασε τὸ ὄνομα αὐτοῦ, Γάδ.
\VS{12}Καὶ συνέλαβεν ἔτι Ζελφὰ ἡ παιδίσκη Λείας, καὶ ἔτεκε τῷ Ἰακὼβ υἱὸν δεύτερον.
\VS{13}Καὶ εἶπε Λεία, μακαρία ἐγὼ, ὅτι μακαριοῦσί με αἱ γυναῖκες· καὶ ἐκάλεσε τὸ ὄνομα αὐτοῦ, Ἀσήρ.
\VS{14}Ἐπορεύθη δὲ Ῥουβὴν ἐν ἡμέρᾳ θερισμοῦ πυρῶν, καὶ εὗρε μῆλα μανδραγορῶν ἐν τῷ ἀγρῷ, καὶ ἤνεγκεν αὐτὰ πρὸς Λείαν τὴν μητέρα αὐτοῦ· εἶπε δὲ Ῥαχὴλ τῇ Λείᾳ τῇ ἀδελφῇ αὐτῆς, δός μοι τῶν μανδραγορῶν τοῦ υἱοῦ σου.
\VS{15}Εἶπε δὲ Λεία, οὐχ ἱκανόν σοι ὅτι ἔλαβες τὸν ἄνδρα μου; μὴ καὶ τοὺς μανδραγόρας τοῦ υἱοῦ μου λήψῃ; εἶπε δὲ Ῥαχὴλ, οὐχ οὕτως· κοιμηθήτω μετὰ σοῦ τὴν νύκτα ταύτην ἀντὶ τῶν μανδραγορῶν τοῦ υἱοῦ σου.
\VS{16}Εἰσῆλθεν δὲ Ἰακὼβ ἐξ ἀγροῦ ἑσπέρας· καὶ ἐξῆλθε Λεία εἰς συνάντησιν αὐτῷ, καὶ εἶπε, πρὸς ἐμὲ εἰσελεύσῃ σήμερον· μεμίσθωμαι γάρ σε ἀντὶ τῶν μανδραγορῶν τοῦ υἱοῦ μου· καὶ ἐκοιμήθη μετʼ αὐτῆς τὴν νύκτα ἐκείνην.
\VS{17}Καὶ ἐπήκουσεν ὁ Θεὸς Λείας· καὶ συλλαβοῦσα ἔτεκε τῷ Ἰακὼβ υἱὸν πέμπτον.
\VS{18}Καὶ εἶπε Λεία, δέδωκέ μοι ὁ Θεὸς τὸν μισθόν μου, ἀνθʼ οὗ ἔδωκα τὴν παιδίσκην μου τῷ ἀνδρί μου· καὶ ἐκάλεσε τὸ ὄνομα αὐτοῦ, Ἰσσάχαρ, ὅ ἐστι μισθός.
\VS{19}Καὶ συνέλαβεν ἔτι Λεία, καὶ ἔτεκεν υἱὸν ἕκτον τῷ Ἰακώβ.
\VS{20}Καὶ εἶπε Λεία, δεδώρηται ὁ Θεός μοι δῶρον καλὸν ἐν τῷ νῦν καιρῷ· αἱρετιεῖ με ὁ ἀνήρ μου, τέτοκα γὰρ αὐτῷ υἱοὺς ἕξ· καὶ ἐκάλεσε τὸ ὄνομα αὐτοῦ, Ζαβουλών.
\VS{21}Καὶ μετὰ τοῦτο ἔτεκε θυγατέρα, καὶ ἐκάλεσε τὸ ὄνομα αὐτῆς, Δεῖνα.
\VS{22}Ἐμνήσθη δὲ ὁ Θεὸς τῆς Ῥαχὴλ, καὶ ἔπήκουσεν αὐτῆς ὁ Θεός· καὶ ἀνέῳξεν αὐτῆς τὴν μήτραν.
\VS{23}Καὶ συλλαβοῦσα ἔτεκε τῷ Ἰακὼβ υἱόν· εἶπε δὲ Ῥαχὴλ, ἀφεῖλεν ὁ Θεός μου τὸ ὄνειδος.
\VS{24}Καὶ ἐκάλεσε τὸ ὄνομα αὐτοῦ Ἰωσὴφ, λέγουσα, προσθέτω ὁ Θεός μοι υἱὸν ἕτερον.
\par }{\PP \VS{25}Ἐγένετο δὲ ὡς ἔτεκε Ῥαχὴλ τὸν Ἰωσὴφ, εἶπεν Ἰακὼβ τῷ Λάβαν, ἀπόστειλόν με, ἵνα ἀπέλθω εἰς τὸν τόπον μου, καὶ εἰς τὴν γῆν μου.
\VS{26}Ἀπόδος τὰς γυναῖκας μου, καὶ τὰ παιδία μου, περὶ ὧν δεδούλευκά σοι, ἵνα ἀπέλθω· σὺ γὰρ γινώσκεις τὴν δουλείαν, ἣν δεδούλευκά σοι.
\VS{27}Εἶπε δὲ αὐτῷ Λάβαν, εἰ εὗρον χάριν ἐναντίον σου, οἰωνισάμην ἄν· εὐλόγησε γάρ με ὁ Θεὸς ἐπὶ τῇ σῇ εἰσόδῳ.
\VS{28}Διάστειλον τὸν μισθόν σου πρός με, καὶ δώσω.
\VS{29}Εἶπε δὲ Ἰακὼβ, σὺ γινώσκεις ἃ δεδούλευκά σοι, καὶ ὅσα ἦν κτήνη σου μετʼ ἐμοῦ.
\VS{30}Μικρὰ γὰρ ἦν ὅσα σοι ἐναντίον ἐμοῦ, καὶ ηὐξήθη εἰς πλῆθος· καὶ εὐλόγησέ σε Κύριος ὁ Θεὸς ἐπὶ τῷ ποδί μου· νῦν οὖν πότε ποιήσω κᾀγὼ ἐμαυτῷ οἶκον;
\VS{31}Καὶ εἶπεν αὐτῷ Λάβαν, τί σοι δώσω; Εἶπε δὲ αὐτῷ Ἰακὼβ, οὐ δώσεις μοι οὐθὲν, ἐὰν ποιήσῃς μοι τὸ ῥῆμα τοῦτο, πάλιν ποιμανῶ τὰ πρόβατά σου, καὶ φυλάξω.
\VS{32}Παρελθέτω πάντα τὰ πρόβατά σου σήμερον, καὶ διαχώρισον ἐκεῖθεν πᾶν πρόβατον φαιὸν ἐν τοῖς ἄρνασι, καὶ πᾶν διάλευκον καὶ ῥαντὸν ἐν ταῖς αἰξὶν, ἔσται μοι μισθός.
\VS{33}Καὶ ἐπακούσεταί μοι ἡ δικαιοσύνη μου ἐν τῇ ἡμέρᾳ τῇ ἐπαύριον, ὅτι ἐστὶν ὁ μισθός μου ἐνώπιόν σου· πᾶν ὃ ἐὰν μὴ ᾖ ῥαντὸν καὶ διάλευκον ἐν ταῖς αἰξὶ, καὶ φαιὸν ἐν τοῖς ἄρνασι, κεκλεμμένον ἔσται παρʼ ἐμοί.
\VS{34}Εἶπε δὲ αὐτῷ Λάβαν, ἔστω κατὰ τὸ ῥῆμά σου.
\VS{35}Καὶ διέστειλεν ἐν τῇ ἡμέρᾳ ἐκείνῃ τοὺς τράγους τοὺς ῥαντοὺς καὶ τοὺς διαλεύκους, καὶ πάσας τὰς αἶγας τὰς ῥαντὰς καὶ τὰς διαλεύκους, καὶ πᾶν ὃ ἦν φαιὸν ἐν τοῖς ἄρνασι, καὶ πᾶν ὃ ἦν λευκὸν ἐν αὐτοῖς, καὶ ἔδωκε διὰ χειρὸς τῶν υἱῶν αὐτοῦ.
\VS{36}Καὶ ἀπέστησεν ὁδὸν τριῶν ἡμερῶν, καὶ ἀνὰ μέσον αὐτῶν καὶ ἀνὰ μέσον Ἰακώβ· Ἰακὼβ δὲ ἐποίμαινε τὰ πρόβατα Λάβαν τὰ ὑπολειφθέντα.
\VS{37}Ἔλαβε δὲ ἑαυτῷ Ἰακὼβ ῥάβδον στυρακίνην χλωρὰν καὶ καρυΐνην καὶ πλατάνου· καὶ ἐλέπισεν αὐτὰς Ἰακὼβ λεπίσματα λευκά· καὶ περισύρων τὸ χλωρὸν, ἐφαίνετο ἐπὶ ταῖς ῥάβδοις τὸ λευκὸν, ὃ ἐλέπισε, ποικίλον.
\VS{38}Καὶ παρέθηκε τὰς ῥάβδους, ἃς ἐλέπισεν, ἐν τοῖς ληνοῖς τῶν ποτιστηρίων τοῦ ὕδατος, ἵνα ὡς ἂν ἔλθωσι τὰ πρόβατα πιεῖν, ἐνώπιον τῶν ῥάβδων ἐλθόντων αὐτῶν εἰς τὸ πιεῖν, ἐγκισσήσωσι τὰ πρόβατα εἰς τὰς ῥάβδους.
\VS{39}Καὶ ἐνεκίσσων τὰ πρόβατα εἰς τὰς ῥάβδους· καὶ ἔτικτον τὰ πρόβατα διάλευκα καὶ ποικίλα καὶ σποδοειδῆ ῥαντά.
\VS{40}Τοὺς δὲ ἀμνοὺς διέστειλεν Ἰακὼβ, καὶ ἔστησεν ἐναντίον τῶν προβάτων κριὸν διάλευκον, καὶ πᾶν ποικίλον ἐν τοῖς ἀμνοῖς· καὶ διεχώρισεν ἑαυτῷ ποίμνια καθʼ ἑαυτὸν, καὶ οὐκ ἔμιξεν αὐτὰ εἰς τὰ πρόβατα Λάβαν.
\VS{41}Ἐγένετο δὲ ἐν τῷ καιρῷ ᾧ ἐνεκίσσων τὰ πρόβατα ἐν γαστρὶ λαμβάνοντα, ἔθηκεν Ἰακὼβ τὰς ῥάβδους ἐναντίον τῶν προβάτων ἐν τοῖς ληνοῖς, τοῦ ἐγκισσῆσαι αὐτὰ κατὰ τὰς ῥάβδους.
\VS{42}Ἡνίκα δʼ ἂν ἔτεκε τὰ πρόβατα, οὐκ ἐτίθει· ἐγένετο δὲ τὰ μὲν ἄσημα τοῦ Λάβαν, τὰ δὲ ἐπίσημα τοῦ Ἰακώβ.
\VS{43}Καὶ ἐπλούτησεν ὁ ἄνθρωπος σφόδρα σφόδρα· καὶ ἐγένετο αὐτῷ κτήνη πολλὰ, καὶ βόες, καὶ παῖδες, καὶ παιδίσκαι, καὶ κάμηλοι, καὶ ὄνοι.

\par }\Chap{31}{\PP \VerseOne{1}Ἤκουσε δὲ Ἰακὼβ τὰ ῥήματα τῶν υἱῶν Λάβαν, λεγόντων, εἴληφεν Ἰακὼβ πάντα τὰ τοῦ πατρὸς ἡμῶν, καὶ ἐκ τῶν τοῦ πατρὸς ἡμῶν πεποίηκε πᾶσαν τὴν δόξαν ταύτην.
\VS{2}Καὶ εἶδεν Ἰακὼβ τὸ πρόσωπον τοῦ Λάβαν, καὶ ἰδοὺ οὐκ ἦν πρὸς αὐτὸν ὡσεὶ χθὲς καὶ τρίτην ἡμέραν.
\VS{3}Εἶπε δὲ Κύριος πρὸς Ἰακὼβ, ἀποστρέφου εἰς τὴν γῆν τοῦ πατρός σου, καὶ εἰς τὴν γενεάν σου, καὶ ἔσομαι μετὰ σοῦ.
\VS{4}Ἀποστείλας δὲ Ἰακὼβ ἐκάλεσε Λείαν καὶ Ῥαχὴλ εἰς τὸ πεδίον, οὗ ἦν τὰ ποίμνια.
\VS{5}Καὶ εἶπεν αὐταῖς, ὁρῶ ἐγὼ τὸ πρόσωπον τοῦ πατρὸς ὑμῶν, ὅτι οὐκ ἔστι πρὸς ἐμοῦ, ὡς ἐχθὲς καὶ τρίτην ἡμέραν· ὁ δὲ Θεὸς τοῦ πατρός μου ἦν μετʼ ἐμοῦ.
\VS{6}Καὶ αὐταὶ δὲ οἴδατε, ὅτι ἐν πάσῃ τῇ ἰσχύϊ μου δεδούλευκα τῷ πατρὶ ὑμῶν.
\VS{7}Ὁ δὲ πατὴρ ὑμῶν παρεκρούσατό με, καὶ ἤλλαξε τὸν μισθόν μου τῶν δέκα ἀμνῶν· καὶ οὐκ ἔδωκεν αὐτῷ ὁ Θεὸς κακοποιῆσαί με.
\VS{8}Ἐὰν οὕτως εἴπῃ, τὰ ποικίλα ἔσται σου μισθὸς, καὶ τέξεται πάντα τὰ πρόβατα ποικίλα· ἐὰν δὲ εἴπῃ, τὰ λευκὰ ἔσται σου μισθὸς, καὶ τέξεται πάντα τὰ πρόβατα λευκά.
\VS{9}Καὶ ἀφείλετο ὁ Θεὸς πάντα τὰ κτήνη τοῦ πατρὸς ὑμῶν, καὶ ἔδωκέ μοι αὐτά.
\VS{10}Καὶ ἐγένετο ἡνίκα ἐνεκίσσων τὰ πρόβατα ἐν γαστρὶ λαμβάνοντα, καὶ εἶδον τοῖς ὀφθαλμοῖς μου ἐν τῷ ὕπνῳ· καὶ ἰδοὺ οἱ τράγοι καὶ οἱ κριοὶ ἀναβαίνοντες ἐπὶ τὰ πρόβατα καὶ τὰς αἶγας, διάλευκοι καὶ ποικίλοι καὶ σποδοειδεῖς ῥαντοί.
\VS{11}Καὶ εἶπέ μοι ὁ Ἄγγελος τοῦ Θεοῦ καθʼ ὕπνον, Ἰακώβ· ἐγὼ δὲ εἶπα, τί ἐστι;
\VS{12}Καὶ εἶπεν, ἀνάβλεψον τοῖς ὀφθαλμοῖς σου, καὶ ἴδε τοὺς τράγους καὶ τοὺς κριοὺς ἀναβαίνοντας ἐπὶ τὰ πρόβατα καὶ τὰς αἶγας διαλεύκους καὶ ποικίλους καὶ σποδοειδεῖς ῥαντούς· ἑώρακα γὰρ ὅσα σοι Λάβαν ποιεῖ.
\VS{13}Ἐγώ εἰμι ὁ Θεὸς ὁ ὀφθείς σοι ἐν τόπῳ Θεοῦ, οὗ ἤλειψάς μοι ἐκεῖ στήλην, καὶ ηὔξω μοι ἐκεῖ εὐχήν· νῦν οὖν ἀνάστηθι, καὶ ἔξελθε ἐκ τῆς γῆς ταύτης, καὶ ἄπελθε εἰς τὴν γῆν τῆς γενέσεώς σου, καὶ ἔσομαι μετὰ σοῦ.
\VS{14}Καὶ ἀποκριθεῖσαι Ῥαχὴλ καὶ Λεία εἶπαν αὐτῷ, μὴ ἔστιν ἡμῖν ἔτι μερὶς ἢ κληρονομία ἐν τῷ οἴκῳ τοῦ πατρὸς ἡμῶν;
\VS{15}Οὐχ ὡς αἱ ἀλλότριαι λελογίσμεθα αὐτῷ; πέπρακε γὰρ ἡμᾶς, καὶ καταβρώσει κατέφαγε τὸ ἀργύριον ἡμῶν.
\VS{16}Πάντα τὸν πλοῦτον καὶ τὴν δόξαν, ἣν ἀφείλετο ὁ Θεὸς τοῦ πατρὸς ἡμῶν, ἡμῖν ἔσται καὶ τοῖς τέκνοις ἡμῶν· νῦν οὖν ὅσα σοι εἴρηκεν ὁ Θεὸς, ποίει.
\VS{17}Ἀναστὰς δὲ Ἰακὼβ ἔλαβε τὰς γυναῖκας αὐτοῦ καὶ τὰ παιδία αὐτοῦ ἐπὶ τὰς καμήλους·
\VS{18}Καὶ ἀπήγαγε πάντα τὰ ὑπάρχοντα αὐτῷ, καὶ πᾶσαν τὴν ἀποσκευὴν αὐτοῦ, ἣν περιεποιήσατο ἐν τῇ Μεσοποταμίᾳ, καὶ πάντα τὰ αὐτοῦ, ἀπελθεῖν πρὸς Ἰσαὰκ τὸν πατέρα αὐτοῦ εἰς γῆν Χαναάν.
\VS{19}Λάβαν δὲ ᾤχετο κεῖραι τὰ πρόβατα αὐτοῦ· ἔκλεψε δὲ Ῥαχὴλ τὰ εἴδωλα τοῦ πατρὸς αὐτῆς.
\VS{20}Ἔκρυψε δὲ Ἰακὼβ Λάβαν τὸν Σύρον, τοῦ μὴ ἀναγγεῖλαι αὐτῷ, ὅτι ἀποδιδράσκει.
\VS{21}Καὶ ἀπέδρα αὐτὸς, καὶ τὰ αὐτοῦ πάντα, καὶ διέβη τὸν ποταμὸν, καὶ ὥρμησεν εἰς τὸ ὄρος Γαλαάδ.
\VS{22}Ἀνηγγέλη δὲ Λάβαν τῷ Σύρῳ τῇ ἡμέρᾳ τῇ τρίτῃ, ὅτι ἀπέδρα Ἰακώβ.
\VS{23}Καὶ παραλαβὼν τοὺς ἀδελφοὺς αὐτοῦ μεθʼ ἑαντοῦ, ἐδίωξεν ὀπίσω αὐτοῦ ὁδὸν ἡμερῶν ἑπτά· καὶ κατέλαβεν αὐτὸν ἐν τῷ ὄρει Γαλαάδ.
\VS{24}Ἦλθε δὲ ὁ Θεὸς πρὸς Λάβαν τὸν Σύρον καθʼ ὕπνον τὴν νύκτα, καὶ εἶπεν αὐτῷ, Φύλαξαι σεαυτὸν μή ποτε λαλήσῃς μετὰ Ἰακὼβ πονηρά.
\VS{25}Καὶ κατέλαβε Λάβαν τὸν Ἰακώβ· Ἰακὼβ δὲ ἔπηξεν τὴν σκηνὴν αὐτοῦ ἐν τῷ ὄρει· Λάβαν δὲ ἔστησε τοὺς ἀδελφοὺς αὐτοῦ ἐν τῷ ὄρει Γαλαάδ.
\VS{26}Εἶπε δὲ Λάβαν τῷ Ἰακὼβ, τί ἐποίησας; ἱνατί κρυφῇ ἀπέδρας, καὶ ἐκλοποφόρησάς με, καὶ ἀπήγαγες τὰς θυγατέρας μου, ὡς αἰχμαλώτιδας μαχαίρᾳ;
\VS{27}Καὶ εἰ ἀνήγγειλάς μοι, ἐξαπέστειλα ἄν σε μετʼ εὐφροσύνης, καὶ μετὰ μουσικῶν, καὶ τυμπάνων, καὶ κιθάρας.
\VS{28}Καὶ οὐκ ἠξιώθην καταφιλῆσαι τὰ παιδία μου, καὶ τὰς θυγατέρας μου· νῦν δὲ ἀφρόνως ἔπραξας.
\VS{29}Καὶ νῦν ἰσχύει ἡ χείρ μου κακοποιῆσαί σε· ὁ δὲ Θεὸς τοῦ πατρός σου χθὲς εἶπε πρός με, λέγων, Φύλαξαι σεαυτὸν μή ποτε λαλήσῃς μετὰ Ἰακὼβ πονηρά.
\VS{30}Νῦν οὖν πεπόρευσαι· ἐπιθυμίᾳ γὰρ ἐπεθύμησας ἀπελθεῖν εἰς τὸν οἶκον τοῦ πατρός σου· ἱνατί ἔκλεψας τοὺς θεούς μου;
\VS{31}Ἀποκριθεὶς δὲ Ἰακὼβ εἶπε τῷ Λάβαν, ὅτι ἐφοβήθην· εἶπα γὰρ, μή ποτε ἀφέλῃ τὰς θυγατέρας σου ἀπʼ ἐμοῦ, καὶ πάντα τὰ ἐμά.
\VS{32}Καὶ εἶπεν Ἰακὼβ, παρʼ ᾧ ἂν εὕρῃς τοὺς θεούς σου, οὐ ζήσεται ἐναντίον τῶν ἀδελφῶν ἡμῶν· ἐπίγνωθι τί ἐστι παρʼ ἐμοὶ τῶν σῶν, καὶ λάβε· καὶ οὐκ ἐπέγνω παρʼ αὐτῷ οὐθέν· οὐκ ᾔδει δὲ Ἰακὼβ, ὅτι Ῥαχὴλ ἡ γυνὴ αὐτοῦ ἔκλεψεν αὐτούς.
\VS{33}Εἰσελθὼν δὲ Λάβαν ἠρεύνησεν εἰς τὸν οἶκον Λείας, καὶ οὐχ εὗρεν· καὶ ἐξῆλθεν ἐκ τοῦ οἴκου Λείας, καὶ ἠρεύνησε τὸν οἶκον Ἰακὼβ, καὶ ἐν τῷ οἴκῳ τῶν δύο παιδισκῶν, καὶ οὐχ εὗρεν· εἰσῆλθε δὲ καὶ εἰς τὸν οἶκον Ῥαχήλ.
\VS{34}Ῥαχὴλ δὲ ἔλαβε τὰ εἴδωλα, καὶ ἐνέβαλεν αὐτὰ εἰς τὰ σάγματα τῆς καμήλου, καὶ ἐπεκάθισεν αὐτοῖς.
\VS{35}Καὶ εἶπε τῷ πατρὶ αὐτῆς, μὴ βαρέως φέρε, κύριε· οὐ δυνάμαι ἀναστῆναι ἐνώπιόν σου, ὅτι τὰ κατʼ ἐθισμὸν τῶν γυναικῶν μοι ἐστίν· ἠρεύνησε Λάβαν ἐν ὅλῳ τῷ οἴκῳ, καὶ οὐχ εὗρε τὰ εἴδωλα.
\VS{36}Ὠργίσθη δὲ Ἰακὼβ, καὶ ἐμαχέσατο τῷ Λάβαν· ἀποκριθεὶς δὲ Ἰακὼβ εἶπε τῷ Λάβαν, τί τὸ ἀδίκημά μου; καὶ τί τὸ ἁμάρτημά μου, ὅτι κατεδίωξας ὀπίσω μου,
\VS{37}καὶ ὅτι ἠρεύνησας πάντα τὰ σκεύη τοῦ οἴκου μου; τί εὗρες ἀπὸ πάντων τῶν σκευῶν τοῦ οἴκου σου; θὲς ὧδε ἐνώπιον τῶν ἀδελφῶν σου καὶ τῶν ἀδελφῶν μου, καὶ ἐλεγξάτωσαν ἀνὰ μέσον τῶν δύο ἡμῶν.
\VS{38}Ταῦτά μοι εἴκοσι ἔτη ἐγώ εἰμι μετὰ σοῦ· τὰ πρόβατά σου καὶ αἱ αἶγές σου οὐκ ἠτεκνώθησαν· κριοὺς τῶν προβάτων σου οὐ κατέφαγον.
\VS{39}Θηριάλωτον οὐκ ἐνήνοχά σοι· ἐγὼ ἀπετίννυον παρʼ ἐμαυτοῦ κλέμματα ἡμέρας, καὶ κλέμματα νυκτός.
\VS{40}Ἐγενόμην τῆς ἡμέρας συγκαιόμενος τῷ καύματι, καὶ τῷ παγετῷ τῆς νυκτός· καὶ ἀφίστατο ὁ ὕπνος μου ἀπὸ τῶν ὀφθαλμῶν μου.
\VS{41}Ταῦτά μοι εἴκοσι ἔτη ἐγώ εἰμι ἐν τῇ οἰκίᾳ σου· ἐδούλευσά σοι δεκατέσσαρα ἔτη ἀντὶ τῶν δύο θυγατέρων σου, καὶ ἓξ ἔτη ἐν τοῖς προβάτοις σου, καὶ παρελογίσω τὸν μισθόν μου δέκα ἀμνάσιν.
\VS{42}Εἰ μὴ ὁ Θεὸς τοῦ πατρός μου Ἁβραὰμ, καὶ ὁ φόβος Ἰσαὰκ, ἦν μοι, νῦν ἂν κενόν με ἐξαπέστειλας· τὴν ταπείνωσίν μου, καὶ τὸν κόπον τῶν χειρῶν μου, εἶδεν ὁ Θεός· καὶ ἤλεγξέ σε χθές.
\par }{\PP \VS{43}Ἀποκριθεὶς δὲ Λάβαν εἶπε τῷ Ἰακὼβ, αἱ θυγατέρες, θυγατέρες μου, καὶ υἱοὶ, υἱοί μου, καὶ τὰ κτήνη, κτήνη μου· καὶ πάντα ὅσα σὺ ὁρᾷς, ἐμά ἐστι, καὶ τῶν θυγατέρων μου· τί ποιήσω ταύταις σήμερον ἢ τοῖς τέκνοις αὐτῶν, οἷς ἔτεκον;
\VS{44}Νῦν οὖν δεῦρο διαθῶμαι διαθήκην ἐγώ τε καὶ σύ· καὶ ἔσται εἰς μαρτύριον ἀνὰ μέσον ἐμοῦ καὶ σοῦ· εἶπε δὲ αὐτῷ, ἰδοὺ οὐθεὶς μεθʼ ἡμῶν ἐστιν· ἴδε ὁ Θεὸς μάρτυς ἀνὰ μέσον ἐμοῦ καὶ σοῦ.
\VS{45}Λαβὼν δὲ Ἰακὼβ λίθον, ἔστησεν αὐτὸν στήλην.
\VS{46}Εἶπε δὲ Ἰακὼβ τοῖς ἀδελφοῖς αὐτοῦ, συλλέγετε λίθους· καὶ συνέλεξαν λίθους, καὶ ἐποίησαν βουνόν· καὶ ἔφαγον ἐκεῖ ἐπὶ τοῦ βουνοῦ· καὶ εἶπεν αὐτῷ Λάβαν, ὁ βουνὸς οὗτος μαρτυρεῖ ἀνὰ μέσον ἐμοῦ καὶ σοῦ σήμερον.
\VS{47}Καὶ ἐκάλεσεν αὐτὸν Λάβαν, βουνὸς τῆς μαρτυρίας· Ἰακὼβ δὲ ἐκάλεσεν αὐτὸν, βουνὸς μάρτυς.
\VS{48}Εἶπε δὲ Λάβαν τῷ Ἰακὼβ, ἰδοὺ ὁ βουνὸς οὗτος καὶ ἡ στήλη, ἣν ἔστησα ἀνὰ μέσον ἐμοῦ καὶ σοῦ· μαρτυρεῖ ὁ βουνὸς οὗτος, καὶ μαρτυρεῖ ἡ στήλη αὕτη· διὰ τοῦτο ἐκλήθη τὸ ὄνομα, βουνὸς μαρτυρεῖ.
\VS{49}Καὶ ἡ ὅρασις, ἣν εἶπεν, ἐπίδοι ὁ Θεὸς ἀνὰ μέσον ἐμοῦ καὶ σοῦ· ὅτι ἀποστησόμεθα ἕτερος ἀφʼ ἑτέρου.
\VS{50}Εἰ ταπεινώσεις τὰς θυγατέρας μου, εἰ λάβῃς γυναῖκας πρὸς ταῖς θυγατράσι μου, ὅρα, οὐθεὶς μεθʼ ἡμῶν ἐστιν ὁρῶν· Θεὸς μάρτυς μεταξὺ ἐμοῦ καὶ μεταξὺ σοῦ.
\VS{50a}Καὶ εἶπε Λάβαν τῷ Ἰακὼβ, ἰδοὺ ὁ βουνὸς οὗτος καὶ μάρτυς ἡ στήλη αὕτη.
\VS{52}Ἐάν τε γὰρ ἐγὼ μὴ διαβῶ πρός σε, μήτε σὺ διαβῇς πρός με τὸν βουνὸν τοῦτον καὶ τὴν στήλην ταύτην ἐπὶ κακίᾳ.
\VS{53}Ὁ Θεὸς Ἁβραὰμ καὶ ὁ Θεὸς Ναχὼρ κρίναι ἀνὰ μέσον ἡμῶν· καὶ ὤμοσεν Ἰακὼβ κατὰ τοῦ φόβου τοῦ πατρὸς αὐτοῦ Ἰσαάκ.
\VS{54}Καὶ ἔθυσεν θυσίαν ἐν τῷ ὄρει· καὶ ἐκάλεσε τοὺς ἀδελφοὺς αὐτοῦ, καὶ ἔφαγον καὶ ἔπιον, καὶ ἐκοιμήθησαν ἐν τῷ ὄρει.

\Chap{32}\VerseOne{1}Ἀναστὰς δὲ Λάβαν τὸ πρωῒ, κατεφίλησε τοὺς υἱοὺς καὶ τὰς θυγατέρας αὐτοῦ, καὶ εὐλόγησεν αὐτούς· καὶ ἀποστραφεὶς Λάβαν ἀπῆλθεν εἰς τὸν τόπον αὐτοῦ.
\par }{\PP \VS{2}Καὶ Ἰακὼβ ἀπῆλθεν εἰς τὴν ὁδὸν ἑαυτοῦ· καὶ ἀναβλέψας εἶδε παρεμβολὴν Θεοῦ παρεμβεβληκυῖαν· καὶ συνήντησαν αὐτῷ οἱ Ἄγγελοι τοῦ Θεοῦ.
\VS{3}Εἶπε δὲ Ἰακὼβ, ἡνίκα εἶδεν αὐτοὺς, παρεμβολὴ Θεοῦ αὕτη· καὶ ἐκάλεσε τὸ ὄνομα τοῦ τόπου ἐκείνου, Παρεμβολαί.
\par }{\PP \VS{4}Ἀπέστειλε δὲ Ἰακὼβ ἀγγέλους ἔμπροσθεν αὐτοῦ πρὸς Ἡσαῦ τὸν ἀδελφὸν αὐτοῦ εἰς γῆν Σηεὶρ, εἰς χώραν Ἐδώμ.
\VS{5}Καὶ ἐνετείλατο αὐτοῖς, λέγων, οὕτως ἐρεῖτε τῷ κυρίῳ μου Ἡσαῦ· οὕτως λέγει ὁ παῖς σου Ἰακώβ· μετὰ Λάβαν παρῴκησα, καὶ ἐχρόνισα ἕως τοῦ νῦν.
\VS{6}Καὶ ἐγένοντό μοι βόες, καὶ ὄνοι, καὶ πρόβατα, καὶ παῖδες, καὶ παιδίσκαι· καὶ ἀπέστειλα ἀναγγεῖλαι τῷ κυρίῳ μου Ἡσαῦ, ἵνα εὕρῃ ὁ παῖς σου χάριν ἐναντίον σου.
\VS{7}Καὶ ἀνέστρεψαν οἱ ἄγγελοι πρὸς Ἰακὼβ, λέγοντες, ἤλθομεν πρὸς τὸν ἀδελφόν σου Ἡσαυ· καὶ ἰδοὺ αὐτὸς ἔρχεται εἰς συνάντησίν σου, καὶ τετρακόσιοι ἄνδρες μεθʼ αὐτοῦ.
\VS{8}Ἐφοβήθη δὲ Ἰακὼβ σφόδρα, καὶ ἠπορεῖτο· καὶ διεῖλε τὸν λαὸν τὸν μεθʼ ἑαυτοῦ, καὶ τοὺς βόας, καὶ τὰς καμήλους, καὶ τὰ πρόβατα, εἰς δύο παρεμβολάς.
\VS{9}Καὶ εἶπεν Ἰακὼβ, ἐὰν ἔλθῃ Ἡσαῦ εἰς παρεμβολὴν μίαν, καὶ κόψῃ αὐτὴν, ἔσται ἡ παρεμβολὴ ἡ δευτέρα εἰς τὸ σώζεσθαι.
\VS{10}Εἶπε δὲ Ἰακὼβ, ὁ Θεὸς τοῦ πατρός μου Ἁβραὰμ, καὶ ὁ Θεὸς τοῦ πατρός μου Ἰσαὰκ, Κύριε σὺ ὁ εἰπών μοι, ἀπότρεχε εἰς τὴν γῆν τῆς γενέσεώς σου, καὶ εὖ σε ποιήσω·
\VS{11}Ἱκανούσθω μοι ἀπὸ πάσης δικαιοσύνης, καὶ ἀπὸ πάσης ἀληθείας, ἧς ἐποίησας τῷ παιδί σου· ἐν γὰρ τῇ ῥάβδῳ μου ταύτῃ διέβην τὸν Ἰορδάνην τοῦτον· νυνὶ δὲ γέγονα εἰς δύο παρεμβολάς.
\VS{12}Ἐξελοῦ με ἐκ χειρὸς τοῦ ἀδελφοῦ μου, ἐκ χειρὸς Ἡσαῦ· ὅτι φοβοῦμαι ἐγὼ αὐτὸν, μή ποτε ἐλθὼν πατάξῃ με, καὶ μητέρα ἐπὶ τέκνοις.
\VS{13}Σὺ δὲ εἶπας, εὐ σε ποιήσω, καὶ θήσω τὸ σπέρμα σου ὡς τὴν ἄμμον τῆς θαλάσσης, ἣ οὐκ ἀριθμηθήσεται ὑπὸ τοῦ πλήθους.
\VS{14}Καὶ ἐκοιμήθη ἐκεῖ τὴν νύκτα ἐκείνην· καὶ ἔλαβεν ὧν ἔφερεν δῶρα· καὶ ἐξαπέστειλεν Ἡσαῦ τῷ ἀδελφῷ αὐτοῦ,
\VS{15}αἶγας διακοσίας, τράγους εἴκοσι, πρόβατα διακόσια, κριοὺς εἴκοσι,
\VS{16}καμήλους θηλαζούσας καὶ τὰ παιδία αὐτῶν τριάκοντα, βόας τεσσαράκοντα, ταύρους δέκα, ὄνους εἴκοσι, καὶ πώλους δέκα.
\VS{17}Καὶ ἔδωκεν αὐτὰ τοῖς παισὶν αὐτοῦ ποίμνιον κατὰ μόνας· εἶπε δὲ τοῖς παισὶν αὐτοῦ, προπορεύεσθε ἔμπροσθέν μου, καὶ διάστημα ποιεῖτε ἀνὰ μέσον ποίμνης καὶ ποίμνης.
\VS{18}Καὶ ἐνετείλατο τῷ πρώτῳ, λέγων, ἐάν σοι συναντήσῃ Ἡσαῦ ὁ ἀδελφός μου, καὶ ἐρωτᾷ σε, λέγων, τίνος εἶ; καὶ ποῦ πορεύῃ; καὶ τίνος ταῦτα τὰ προπορευόμενά σου;
\VS{19}Ἐρεῖς, τοῦ παιδός σου Ἰακώβ· δῶρα ἀπέσταλκε τῷ κυρίῳ μου Ἡσαῦ· καὶ ἰδοὺ αὐτὸς ὀπίσω ἡμῶν.
\VS{20}Καὶ ἐνετείλατο τῷ πρώτῳ, καὶ τῷ δευτέρῳ, καὶ τῷ τρίτῳ, καὶ πᾶσι τοῖς προπορευομένοις ὀπίσω τῶν ποιμνίων τούτων, λέγων, κατὰ τὸ ῥῆμα τοῦτο λαλήσατε Ἡσαῦ ἐν τῷ εὑρεῖν ὑμᾶς αὐτόν·
\VS{21}Καὶ ἐρεῖτε, ἰδοὺ ὁ παῖς σου Ἰακὼβ παραγίνεται ὀπίσω ἡμῶν· εἶπε γὰρ, ἐξιλάσομαι τὸ πρόσωπον αὐτοῦ ἐν τοῖς δώροις τοῖς προπορευομένοις αὐτοῦ, καὶ μετὰ τοῦτο ὄψομαι τὸ πρόσωπον αὐτοῦ· ἴσως γὰρ προσδέξεται τὸ πρόσωπόν μου.
\VS{22}Καὶ προεπορεύετο τὰ δῶρα κατὰ πρόσωπον αὐτοῦ· αὐτὸς δὲ ἐκοιμήθη τὴν νύκτα ἐκείνην ἐν τῇ παρεμβολῇ.
\VS{23}Ἀναστὰς δὲ τὴν νύκτα ἐκείνην, ἔλαβε τὰς δύο γυναῖκας, καὶ τὰς δύο παιδίσκας, καὶ τὰ ἕνδεκα παιδία αὐτοῦ, καὶ διέβη τὴν διάβασιν τοῦ Ἰαβώχ.
\VS{24}Καὶ ἔλαβεν αὐτοὺς, καὶ διέβη τὸν χειμάῤῥουν, καὶ διεβίβασε πάντα τὰ αὐτοῦ.
\par }{\PP \VS{25}Ὑπελείφθη δὲ Ἰακὼβ μόνος· καὶ ἐπάλαιεν ἄνθρωπος μετʼ αὐτοῦ ἕως πρωΐ.
\VS{26}Εἶδε δὲ ὅτι οὐ δύναται πρὸς αὐτόν· καὶ ἥψατο τοῦ πλάτους τοῦ μηροῦ αὐτοῦ, καὶ ἐνάρκησε τὸ πλάτος τοῦ μηροῦ Ἰακὼβ ἐν τῷ παλαίειν αὐτὸν μετʼ αὐτοῦ.
\VS{27}Καὶ εἶπεν αὐτῷ, ἀπόστειλόν με, ἀνέβη γὰρ ὁ ὄρθρος. ὁ δὲ εἶπεν, οὐ μή σε ἀποστείλω, ἐὰν μή με εὐλογήσῃς.
\VS{28}Εἶπε δὲ αὐτῷ, τί τὸ ὄνομά σου ἐστίν; ὁ δὲ εἶπεν, Ἰακώβ.
\VS{29}Καὶ εἶπεν αὐτῷ, οὐ κληθήσεται ἔτι τὸ ὄνομά σου Ἰακὼβ, ἀλλʼ Ἰσραὴλ ἔσται τὸ ὄνομά σου· ὅτι ἐνίσχυσας μετὰ Θεοῦ, καὶ μετὰ ἀνθρώπων δυνατὸς ἔσῃ.
\VS{30}Ἠρώτησε δὲ Ἰακὼβ, καὶ εἶπεν, ἀνάγγειλόν μοι τὸ ὄνομά σου· καὶ εἶπεν, ἱνατί τοῦτο ἐρωτᾷς σὺ τὸ ὄνομά μου; καὶ εὐλόγησεν αὐτὸν ἐκεῖ.
\VS{31}Καὶ ἐκάλεσεν Ἰακὼβ τὸ ὄνομα τοῦ τόπου ἐκείνου, εἶδος Θεοῦ· εἶδον γὰρ Θεὸν πρόσωπον πρὸς πρὸσωπον, καὶ ἐσώθη μου ἡ ψυχή.
\VS{32}Ἀνέτειλεν δὲ αὐτῷ ὁ ἥλιος, ἡνίκα παρῆλθε τὸ εἶδος τοῦ Θεοῦ· αὐτὸς δὲ ἐπέσκαζε τῷ μηρῷ αὐτοῦ.
\VS{33}Ἕνεκεν τούτου οὐ μὴ φάγωσιν υἱοὶ Ἰσραὴλ τὸ νεῦρον, ὃ ἐνάρκησεν, ὅ ἐστιν ἐπὶ τοῦ πλάτους τοῦ μηροῦ, ἕως τῆς ἡμέρας ταύτης, ὅτι ἥψατο τοῦ πλάτους τοῦ μηροῦ Ἰακὼβ τοῦ νεύρου, ὃ ἐνάρκησεν.

\par }\Chap{33}{\PP \VerseOne{1}Ἀναβλέψας δὲ Ἰακὼβ τοῖς ὀφθαλμοῖς αὐτοῦ εἶδε· καὶ ἰδοὺ Ἡσαῦ ὁ ἀδελφὸς αὐτοῦ ἐρχόμενος, καὶ τετρακόσιοι ἄνδρες μετʼ αὐτοῦ· καὶ διεῖλεν Ἰακὼβ τὰ παιδία ἐπὶ Λείαν, καὶ ἐπὶ Ῥαχὴλ, καὶ τὰς δύο παιδίσκας.
\VS{2}Καὶ ἔθετο τὰς δύο παιδίσκας καὶ τοὺς υἱοὺς αὐτῶν ἐν πρώτοις, καὶ Λείαν καὶ τὰ παιδία αὐτῆς ὀπίσω, καὶ Ῥαχὴλ καὶ Ἰωσὴφ ἐσχάτους.
\VS{3}Αὐτὸς δὲ προῆλθεν ἔμπροσθεν αὐτῶν· καὶ προσεκύνησεν ἐπὶ τὴν γῆν ἑπτάκις, ἕως τοῦ ἐγγίσαι τῷ ἀδελφῷ αὐτοῦ.
\VS{4}Καὶ προσέδραμεν Ἡσαῦ εἰς συνάντησιν αὐτῷ· καὶ περιλαβὼν αὐτὸν προσέπεσεν ἐπὶ τὸν τράχηλον αὐτοῦ, καὶ κατεφίλησεν αὐτόν· καὶ ἔκλαυσαν ἀμφότεροι.
\VS{5}Καὶ ἀναβλέψας Ἡσαῦ εἶδε τὰς γυναῖκας καὶ τὰ παιδία· καὶ εἶπε, τί ταῦτά σοι ἐστίν; ὁ δὲ εἶπε, τὰ παιδία, οἷς ἠλέησεν ὁ Θεὸς τὸν παῖδά σου.
\VS{6}Καὶ προσήγγισαν αἱ παιδίσκαι καὶ τὰ τέκνα αὐτῶν, καὶ προσεκύνησαν.
\VS{7}Καὶ προσήγγισε Λεία καὶ τὰ τέκνα αὐτῆς, καὶ προσεκύνησαν· καὶ μετὰ ταῦτα προσήγγισε Ῥαχὴλ καὶ Ἰωσὴφ, καὶ προσεκύνησαν.
\VS{8}Καὶ εἶπε, τί ταῦτά σοι ἐστὶν, πᾶσαι αἱ παρεμβολαὶ αὗται, αἷς ἀπήντηκα; ὁ δὲ εἶπεν, ἵνα εὕρῃ ὁ παῖς σου χάριν ἐναντίον σου, κύριε.
\VS{9}Εἶπε δὲ Ἡσαῦ, ἔστι μοι πολλὰ, ἀδελφέ· ἔστω σοι τὰ σά.
\VS{10}Εἶπε δὲ Ἰακὼβ, εἰ εὓρον χάριν ἐναντίον σου, δέξαι τὰ δῶρα διὰ τῶν ἐμῶν χειρῶν· ἕνεκεν τούτου εἶδον τὸ πρόσωπόν σου, ὡς ἄν τις ἴδοι πρόσωπον Θεοῦ, καὶ εὐδοκήσεις με.
\VS{11}Λάβε τὰς εὐλογίας μου, ἃς ἤνεγκά σοι, ὅτι ἠλέησέ με ὁ Θεὸς, καὶ ἔστι μοι πάντα· καὶ ἐβιάσατο αὐτὸν, καὶ ἔλαβε.
\VS{12}Καὶ εἶπεν, ἀπάραντες πορευσώμεθα ἐπʼ εὐθεῖαν.
\VS{13}Εἶπε δὲ αὐτῷ, ὁ κύριός μου γινώσκει, ὅτι τὰ παιδία ἁπαλώτερα, καὶ τὰ πρόβατα καὶ αἱ βόες λοχεύονται ἐπʼ ἐμέ· ἐὰν οὖν καταδιώξω αὐτὰ ἡμέραν μίαν, ἀποθανοῦνται πάντα τὰ κτήνη.
\VS{14}Προελθέτω ὁ κύριός μου ἔμπροσθεν τοῦ παιδὸς αὐτοῦ· ἐγὼ δὲ ἐνισχύσω ἐν τῇ ὁδῷ κατὰ σχολὴν τῆς πορεύσεως τῆς ἐναντίον μου, καὶ κατὰ πόδα τῶν παιδαρίων, ἕως τοῦ ἐλθεῖν με πρὸς τὸν κύριόν μου εἰς Σηείρ.
\VS{15}Εἶπε δὲ Ἡσαῦ, καταλείψω μετὰ σοῦ ἀπὸ τοῦ λαοῦ τοῦ μετʼ ἐμοῦ· ὁ δὲ εἶπεν, ἱνατί τοῦτο; ἱκανὸν ὅτι εὗρον χάριν ἐναντίον σου, κύριε.
\VS{16}Ἀπέστρεψε δὲ Ἡσαῦ ἐν τῇ ἡμέρᾳ ἐκείνῃ εἰς τὴν ὁδὸν αὐτοῦ εἰς Σηείρ.
\VS{17}Καὶ Ἰακὼβ ἀπαίρει εἰς σκηνὰς, καὶ ἐποίησεν ἑαυτῷ ἐκεῖ οἰκίας, καὶ τοῖς κτήνεσιν αὐτοῦ ἐποίησε σκηνάς· διὰ τοῦτο ἐκάλεσε τὸ ὄνομα τοῦ τόπου ἐκείνου, Σκηναί.
\par }{\PP \VS{18}Καὶ ἦλθεν Ἰακὼβ εἰς Σαλὴμ, πόλιν Σηκίμων, ἥ ἐστιν ἐν γῇ Χαναὰν, ὅτε ἐπανῆλθεν ἐκ τῆς Μεσοποταμίας Συρίας· καὶ παρενέλαβε κατὰ πρόσωπον τῆς πόλεως.
\VS{19}Καὶ ἐκτήσατο τὴν μερίδα τοῦ ἀγροῦ, οὗ ἔστησεν ἐκεῖ τὴν σκηνὴν αὐτοῦ, παρὰ Ἐμμὼρ πατρὸς Συχὲμ, ἑκατὸν ἀμνῶν.
\VS{20}Καὶ ἔστησεν ἐκεῖ θυσιαστήριον, καὶ ἐπεκαλέσατο τὸν Θεὸν Ἰσραήλ.

\par }\Chap{34}{\PP \VerseOne{1}Ἐξῆλθε δὲ Δείνα, ἡ θυγάτηρ Λείας, ἣν ἔτεκε τῷ Ἰακώβ, καταμαθεῖν τὰς θυγατέρας τῶν ἐγχωρίων.
\VS{2}Καὶ εἶδεν αὐτὴν Συχὲμ ὁ υἱὸς Ἐμμὼρ ὁ Εὐαῖος, ὁ ἄρχων τῆς γῆς· καὶ λαβὼν αὐτὴν, ἐκοιμήθη μετʼ αὐτῆς, καὶ ἐταπείνωσεν αὐτήν.
\VS{3}Καὶ προσέσχε τῇ ψυχῇ Δείνας τῆς θυγατρὸς Ἰακώβ· καὶ ἠγάπησε τὴν παρθένον· καὶ ἐλάλησε κατὰ τὴν διάνοιαν τῆς παρθένου αυτῇ.
\VS{4}Εἶπε Συχὲμ πρὸς Ἐμμὼρ τὸν πατέρα αὐτοῦ, λέγων, λάβε μοι τὴν παῖδα ταύτην εἰς γυναῖκα.
\VS{5}Ἰακὼβ δὲ ἤκουσεν, ὅτι ἐμίανεν ὁ υἱὸς Ἐμμὼρ Δείναν τὴν θυγατέρα αὐτοῦ· οἱ δὲ υἱοὶ αὐτοῦ ἦσαν μετὰ τῶν κτηνῶν αὐτοῦ ἐν τῷ πεδίῳ· παρεσιώπησε δὲ Ἰακὼβ, ἕως τοῦ ἐλθεῖν αὐτούς.
\VS{6}Ἐξῆλθε δὲ Ἐμμὼρ ὁ πατὴρ Συχὲμ πρὸς Ἰακὼβ, λαλῆσαι αὐτῷ.
\VS{7}Οἱ δὲ υἱοὶ Ἰακὼβ ἦλθον ἐκ τοῦ πεδίου· ὡς δὲ ἤκουσαν, κατενύγησαν οἱ ἄνδρες, καὶ λυπηρὸν ἦν αὐτοῖς σφόδρα· ὅτι ἄσχημον ἐποίησεν ἐν Ἰσραὴλ, κοιμηθεὶς μετὰ τῆς θυγατρὸς Ἰακώβ· καὶ οὐχ οὕτως ἔσται.
\VS{8}Καὶ ἐλάλησεν Ἐμμὼρ αὐτοῖς, λέγων, Συχὲμ ὁ υἱός μου προείλετο τῇ ψυχῇ τὴν θυγατέρα ὑμῶν· δότε οὖν αὐτὴν αὐτῷ γυναῖκα,
\VS{9}καὶ ἐπιγαμβρεύσασθε ἡμῖν· τὰς θυγατέρας ὑμῶν δότε ἡμῖν, καὶ τὰς θυγατέρας ἡμῶν λάβετε τοῖς υἱοῖς ὑμῶν.
\VS{10}Καὶ ἐν ἡμῖν κατοικεῖτε· καὶ ἡ γῆ ἰδοὺ πλατεῖα ἐναντίον ὑμῶν· κατοικεῖτε, καὶ ἐμπορεύεσθε ἐπʼ αὐτῆς, καὶ ἐγκτᾶσθε ἐν αὐτῇ.
\VS{11}Εἶπε δὲ Συχὲμ πρὸς τὸν πατέρα αὐτῆς, καὶ πρὸς τοὺς ἀδελφοὺς αὐτῆς, εὕροιμι χάριν ἐναντίον ὑμῶν· καὶ ὃ ἐὰν εἴπητε, δώσομεν.
\VS{12}Πληθύνατε τὴν φερνὴν σφόδρα, καὶ δώσω καθότι ἂν εἴπητέ μοι, καὶ δώσετέ μοι τὴν παῖδα ταύτην εἰς γυναῖκα.
\par }{\PP \VS{13}Ἀπεκρίθησαν δὲ οἱ υἱοὶ Ἰακὼβ τῷ Συχὲμ, καὶ Ἐμμὼρ τῷ πατρὶ αὐτοῦ, μετὰ δόλου· καὶ ἐλάλησαν αὐτοῖς, ὅτι ἐμίαναν Δείναν τὴν ἀδελφὴν αὐτῶν.
\VS{14}Καὶ εἶπαν αὐτοῖς Συμεὼν καὶ Λευὶ οἱ ἀδελφοὶ Δείνας, οὐ δυνησόμεθα ποιῆσαι τὸ ῥῆμα τοῦτο, δοῦναι τὴν ἀδελφὴν ἡμῶν ἀνθρώπῳ, ὃς ἔχει ἀκροβυστίαν· ἔστι γὰρ ὄνειδος ἡμῖν.
\VS{15}Μόνον ἐν τούτῳ ὁμοιωθησόμεθα ὑμῖν, καὶ κατοικήσομεν ἐν ὑμῖν, ἐὰν γένησθε ὡς ἡμεῖς καὶ ὑμεῖς, ἐν τῷ περιτμηθῆναι ὑμῶν πᾶν ἀρσενικόν.
\VS{16}Καὶ δώσομεν τὰς θυγατέρας ἡμῶν ὑμῖν, καὶ ἀπὸ τῶν θυγατέρων ὑμῶν ληψόμεθα ἡμῖν γυναῖκας, καὶ οἰκήσομεν παρʼ ὑμῖν, καὶ ἐσόμεθα ὡς γένος ἕν.
\VS{17}Ἐὰν δὲ μὴ εἰσακούσητε ἡμῶν τοῦ περιτεμέσθαι, λαβόντες τὴν θυγατέρα ἡμῶν ἀπελευσόμεθα.
\VS{18}Καὶ ἤρεσαν οἱ λόγοι ἐναντίον Ἐμμὼρ, καὶ ἐναντίον Συχὲμ τοῦ υἱοῦ Ἐμμώρ.
\VS{19}Καὶ οὐκ ἐχρόνισεν ὁ νεανίσκος τοῦ ποιῆσαι τὸ ῥῆμα τοῦτο· ἐνέκειτο γὰρ τῇ θυγατρὶ Ἰακώβ· αὐτὸς δὲ ἦν ἐνδοξότατος πάντων τῶν ἐν τῷ οἴκῳ τοῦ πατρὸς αὐτοῦ.
\VS{20}Ἦλθε δὲ Ἐμμὼρ καὶ Συχὲμ ὁ υἱὸς αὐτοῦ πρὸς τὴν πύλην τῆς πόλεως αὐτῶν, καὶ ἐλάλησαν πρὸς τοὺς ἄνδρας τῆς πόλεως αὐτῶν, λέγοντες,
\VS{21}Οἱ ἄνθρωποι οὗτοι εἰρήνικοί εἰσι, μεθʼ ἡμῶν οἰκείτωσαν επὶ τῆς γῆς, καὶ ἐμπορευέσθωσαν αὐτήν· ἡ δὲ γῆ ἰδοὺ πλατεῖα ἐναντίον αὐτῶν· τὰς θυγατέρας αὐτῶν ληψόμεθα ἡμῖν γυναῖκας, καὶ τὰς θυγατέρας ἡμῶν δώσομεν αὐτοῖς.
\VS{22}Ἐν τούτῳ μόνον ὁμοιωθήσονται ἡμῖν οἱ ἄνθρωποι τοῦ κατοικεῖν μεθʼ ἡμῶν, ὥστε εἶναι λαὸν ἕνα, ἐν τῷ περιτεμέσθαι ἡμῶν πᾶν ἀρσενικὸν, καθὰ καὶ αὐτοὶ περιτέτμηνται.
\VS{23}Καὶ τὰ κτήνη αὐτῶν, καὶ τὰ τετράποδα, καὶ τὰ ὑπάρχοντα αὐτῶν, οὐχ ἡμῶν ἔσται; μόνον ἐν τούτῳ ὁμοιωθῶμεν αὐτοῖς, καὶ οἰκήσουσι μεθʼ ἡμῶν.
\VS{24}Καὶ εἰσήκουσαν Ἐμμὼρ καὶ Συχὲμ τοῦ υἱοῦ αὐτοῦ πάντες οἱ ἐμπορευόμενοι τὴν πύλην τῆς πόλεως αὐτῶν· καὶ περιετέμοντο τὴν σάρκα τῆς ἀκροβυστίας αὐτῶν πᾶς ἄρσην.
\par }{\PP \VS{25}Ἐγένετο δὲ ἐν τῇ ἡμέρᾳ τῇ τρίτῃ, ὅτε ἦσαν ἐν τῷ πόνῳ, ἔλαβον οἱ δύο υἱοὶ Ἰακὼβ Συμεὼν καὶ Λευὶ, ἀδελφοὶ Δείνας, ἕκαστος τὴν μάχαιραν αὐτοῦ, καὶ εἰσῆλθον εἰς τὴν πόλιν ἀσφαλὼς, καὶ ἀπέκτειναν πᾶν ἀρσενικόν.
\VS{26}Τόν τε Ἐμμὼρ καὶ Συχὲμ τὸν υἱὸν αὐτοῦ ἀπέκτειναν ἐν στόματι μαχαίρας· καὶ ἔλαβον τὴν Δείναν ἐκ τοῦ οἴκου τοῦ Συχὲμ, καὶ ἐξῆλθον.
\VS{27}Οἱ δὲ υἱοὶ Ἰακὼβ εἰσῆλθον ἐπὶ τοὺς τραυματίας, καὶ διήρπασαν τὴν πόλιν, ἐν ᾗ ἐμίαναν Δείναν τὴν ἀδελφὴν αὐτῶν.
\VS{28}Καὶ τὰ πρόβατα αὐτῶν, καὶ τοὺς βόας αὐτῶν, καὶ τοὺς ὄνους αὐτῶν, ὅσα τε ἦν ἐν τῇ πόλει, καὶ ὅσα ἦν ἐν τῷ πεδίῳ, ἔλαβον.
\VS{29}Καὶ πάντα τὰ σώματα αὐτῶν, καὶ πᾶσαν τὴν ἀποσκευὴν αὐτῶν, καὶ τὰς γυναῖκας αὐτῶν ἠχμαλώτευσαν· καὶ διήρπασαν ὅσα τε ἦν ἐν τῇ πόλει, καὶ ὅσα ἦν ἐν ταῖς οἰκίαις.
\VS{30}Εἶπε δὲ Ἰακὼβ πρὸς Συμεὼν καὶ Λευὶ, μισητόν με πεποιήκατε, ὥστε πονηρόν με εἶναι πᾶσι τοῖς κατοικοῦσι τὴν γῆν, ἔν τε τοῖς Χαναναίοις, καὶ ἐν τοῖς Φερεζαίοις· ἐγὼ δὲ ὀλιγοστός εἰμι ἐν ἀριθμῷ· καὶ συναχθέντες ἐπʼ ἐμὲ συγκόψουσί με, καὶ ἐκτριβήσομαι ἐγὼ, καὶ ὁ οἶκός μου.
\VS{31}Οἱ δὲ εἶπαν, ἀλλʼ ὡσεὶ πόρνῃ χρήσονται τῇ ἀδελφῇ ἡμῶν;

\par }\Chap{35}{\PP \VerseOne{1}Εἶπε δὲ ὁ Θεὸς πρὸς Ἰακὼβ, ἀναστὰς ἀνάβηθι εἰς τὸν τόπον Βαιθὴλ, καὶ οἴκει ἐκεῖ· καὶ ποίησον ἐκεῖ θυσιαστήριον τῷ Θεῷ τῷ ὀφθέντι σοι, ἐν τῷ ἀποδιδράσκειν σε ἀπὸ προσώπου Ἡσαῦ τοῦ ἀδελφοῦ σου.
\VS{2}Εἶπε δὲ Ἰακὼβ τῷ οἴκῳ αὐτοῦ, καὶ πᾶσι τοῖς μετʼ αὐτοῦ, ἄρατε τοὺς θεοὺς τοὺς ἀλλοτρίους τοὺς μεθʼ ὑμῶν ἐκ μέσου ὑμῶν, καὶ καθαρίσθητε, καὶ ἀλλάξατε τὰς στολὰς ὑμῶν.
\VS{3}Καὶ ἀναστάντες ἀναβῶμεν εἰς Βαιθὴλ, καὶ ποιήσωμεν ἐκεῖ θυσιαστήριον τῷ Θεῷ τῷ ἐπακούσαντί μου ἐν ἡμέρᾳ θλίψεως, ὃς ἦν μετʼ ἐμοῦ, καὶ διέσωσέ με ἐν τῇ ὁδῷ, ᾗ ἐπορεύθην.
\VS{4}Καὶ ἔδωκαν τῷ Ἰακὼβ τοὺς θεοὺς τοὺς ἀλλοτρίους, οἳ ἦσαν ἐν ταῖς χερσὶν αὐτῶν, καὶ τὰ ἐνώτια τὰ ἐν τοῖς ὠσὶν αὐτῶν· καὶ κατέκρυψεν αὐτὰ Ἰακὼβ ὑπὸ τὴν τερέβινθον τὴν ἐν Σηκίμοις· καὶ ἀπώλεσεν αὐτὰ, ἕως τῆς σήμερον ἡμέρας.
\VS{5}Καὶ ἐξῇρεν Ἰσραὴλ ἐκ Σηκίμων· καὶ ἐγένετο φόβος Θεοῦ ἐπὶ τὰς πόλεις τὰς κύκλῳ αὐτῶν, καὶ οὐ κατεδίωξαν ὀπίσω τῶν υἱῶν Ἰσραήλ.
\VS{6}Ἦλθε δὲ Ἰακὼβ εἰς Λουζὰ ἥ ἐστιν ἐν γῇ Χαναὰν, ἥ ἐστι Βαιθὴλ, αὐτὸς, καὶ πᾶς ὁ λαὸς, ὃς ἦν μετʼ αὐτοῦ.
\VS{7}Καὶ ᾠκοδόμησεν ἐκεῖ θυσιαστήριον, καὶ ἐκάλεσε τὸ ὄνομα τοῦ τόπου, Βαιθήλ· ἐκεῖ γὰρ ἐφάνη αὐτῷ ὁ Θεὸς, ἐν τῷ ἀποδιδράσκειν αὐτὸν ἀπὸ προσώπου Ἡσαῦ τοῦ ἀδελφοῦ αὐτοῦ.
\par }{\PP \VS{8}Ἀπέθανε δὲ Δεβόῤῥα, ἡ τρόφος Ῥεβέκκας, καὶ ἐτάφη κατώτερον Βαιθὴλ ὑπὸ τὴν βάλανον· καὶ ἐκάλεσεν Ἰακὼβ τὸ ὄνομα αὐτῆς, βάλανος πένθους.
\VS{9}Ὤφθη δὲ ὁ Θεὸς τῷ Ἰακὼβ ἔτι ἐν Λουζᾷ, ὅτε παρεγένετο ἐκ Μεσοποταμίας τῆς Συρίας· καὶ εὐλόγησεν αὐτὸν ὁ Θεὸς.
\VS{10}Καὶ εἶπεν αὐτῷ ὁ Θεὸς, τὸ ὄνομά σου οὐ κληθήσεται ἔτι Ἰακὼβ, ἀλλʼ Ἰσραὴλ ἔσται τὸ ὄνομά σου· καὶ ἐκάλεσε τὸ ὄνομα αὐτοῦ Ἰσραήλ.
\VS{11}Εἶπε δὲ αὐτῷ ὁ Θεὸς, ἐγὼ ὁ Θεός σου· αὐξάνου, καὶ πληθύνου· ἔθνη καὶ συναγωγαὶ ἐθνῶν ἔσονται ἐκ σοῦ, καὶ βασιλεῖς ἐκ τῆς ὀσφύος σου ἐξελεύσονται.
\VS{12}Καὶ τὴν γῆν, ἣν ἔδωκα Ἁβραὰμ καὶ Ἰσαὰκ, σοὶ δέδωκα αὐτήν· σοὶ ἔσται· καὶ τῷ σπέρματί σου μετὰ σὲ δώσω τῆν γῆν ταύτην.
\VS{13}Ἀνέβη δὲ ὁ Θεὸς ἀπʼ αὐτοῦ ἐκ τοῦ τόπου, οὗ ἐλάλησε μετʼ αὐτοῦ.
\VS{14}Καὶ ἔστησεν Ἰακὼβ στήλην ἐν τῷ τόπῳ, ᾧ ἐλάλησε μετʼ αὐτοῦ ὁ Θεὸς, στήλην λιθίνην· καὶ ἔσπεισεν ἐπʼ αὐτὴν σπονδὴν, καὶ ἐπέχεεν ἐπʼ αὐτὴν ἔλαιον.
\VS{15}Καὶ ἐκάλεσεν Ἰακὼβ τὸ ὄνομα τοῦ τόπου, ἐν ᾧ ἐλάλησε μετʼ αὐτοῦ ἐκεῖ ὁ Θεὸς, Βαιθήλ.
\VS{16}Ἀπάρας δὲ Ἰακὼβ ἐκ Βαιθὴλ, ἔπηξε τὴν σκηνὴν αὐτοῦ ἐπέκεινα τοῦ πύργου Γαδέρ· ἐγένετο δὲ ἡνίκα ἤγγισεν εἰς Χαβραθὰ τοῦ ἐλθεῖν εἰς τὴν Ἐφραθᾶ, ἔτεκε Ῥαχήλ· καὶ ἐδυστόκησεν ἐν τῷ τοκετῷ.
\VS{17}Ἐγένετο δὲ ἐν τῷ σκληρὼς αὐτὴν τίκτειν, εἶπεν αὐτῇ ἡ μαῖα, θάρσει, καὶ γὰρ οὗτός σοι ἐστὶν υἱός.
\VS{18}Ἐγένετο δὲ ἐν τῷ ἀφιέναι αὐτὴν τὴν ψυχὴν, ἀπέθνησκε γὰρ, ἐκάλεσε τὸ ὄνομα αὐτοῦ, υἱὸς ὀδύνης μου· ὁ δὲ πατὴρ ἐκάλεσεν τὸ ὄνομα αὐτοῦ, Βενιαμίν.
\VS{19}Ἀπέθανε δὲ Ῥαχὴλ, καὶ ἐτάφη ἐν τῇ ὁδῷ τοῦ ἱπποδρόμου Ἐφραθᾶ· αὕτη ἐστὶ Βηθλεέμ.
\VS{20}Καὶ ἔστησεν Ἰακὼβ στήλην ἐπὶ τοῦ μνημείου αὐτῆς· αὕτη ἐστὶν ἡ στήλη ἐπὶ τοῦ μνημείου Ῥαχὴλ ἕως τῆς ἡμέρας ταύτης.
\VS{22}Ἐγένετο δὲ ἡνίκα κατῴκησεν Ἰσραὴλ ἐν τῇ γῇ ἐκείνῃ, ἐπορεύθη Ῥουβὴν, καὶ ἐκοιμήθη μετὰ Βαλλὰς, τῆς παλλακῆς τοῦ πατρὸς αὐτοῦ Ἰακώβ· καὶ ἤκουσεν Ἰσραὴλ, καὶ πονηρὸν ἐφάνη ἐναντίον αὐτοῦ.
\par }{\PP Ἦσαν δὲ οἱ υἱοὶ Ἰακὼβ, δώδεκα.
\VS{23}Υἱοὶ Λείας, πρωτότοκος Ἰακὼβ, Ῥουβὴν, Συμεὼν, Λευὶ, Ἰούδας, Ἰσσάχαρ, Ζαβουλών.
\VS{24}Υἱοὶ δὲ Ῥαχὴλ, Ἰωσὴφ, καὶ Βενιαμίν.
\VS{25}Υἱοὶ δὲ Βαλλᾶς παιδίσκης Ῥαχὴλ, Δαν, καὶ Νεφθαλείμ.
\VS{26}Υἱοὶ δὲ Ζελφᾶς παιδίσκης Λείας, Γὰδ, καὶ Ἀσήρ· οὗτοι υἱοὶ Ἰακὼβ, οἳ ἐγένοντο αὐτῷ ἐν Μεσοποταμίᾳ τῆς Συρίας.
\VS{27}Ἦλθε δὲ Ἰακὼβ πρὸς Ἰσαὰκ τὸν πατέρα αὐτοῦ εἰς Μαμβρῆ, εἰς πόλιν τοῦ πεδίου· αὕτη ἐστὶ Χεβρὼν ἐν γῇ Χαναὰν, οὗ παρῴκησεν Ἁβραὰμ καὶ Ἰσαάκ.
\VS{28}Ἐγένοντο δὲ αἱ ἡμέραι Ἰσαὰκ, ἃς ἔζησεν, ἔτη ἑκατὸν ὀγδοήκοντα.
\VS{29}Καὶ ἐκλείπων Ἰσαὰκ ἀπέθανε, καὶ προσετέθη πρὸς τὸ γένος αὐτοῦ πρεσβύτερος καὶ πλήρης ἡμερῶν· καὶ ἔθαψαν αὐτὸν Ἡσαῦ καὶ Ἰακὼβ οἱ υἱοὶ αὐτοῦ.

\par }\Chap{36}{\PP \VerseOne{1}Αὗται δὲ αἱ γενέσεις Ἡσαῦ· αὐτός ἐστιν Ἐδώμ.
\VS{2}Ἡσαῦ δὲ ἔλαβε τὰς γυναῖκας ἑαυτῷ ἀπὸ τῶν θυγατέρων τῶν Χαναναίων· τὴν Ἀδὰ, θυγατέρα Αἰλὼμ τοῦ Χετταίου· καὶ τὴν Ὀλιβεμὰ, θυγατέρα Ἀνὰ τοῦ υἱοῦ Σεβεγὼν τοῦ Εὐαίου.
\VS{3}Καὶ τὴν Βασεμὰθ, θυγατέρα Ἰσμαὴλ, ἀδελφὴν Ναβαιώθ.
\VS{4}Ἔτεκε δὲ αὐτῷ Ἀδὰ τὸν Ἑλιφάς· καὶ Βασεμὰθ ἔτεκε τὸν Ῥαγουήλ.
\VS{5}Καὶ Ὀλιβεμὰ ἔτεκε τὸν Ἰεοὺς, καὶ τὸν Ἰεγλὸμ, καὶ τὸν Κορέ· οὗτοι υἱοὶ Ἡσαῦ, οἳ ἐγένοντο αὐτῷ ἐν γῇ Χαναάν.
\VS{6}Ἔλαβε δὲ Ἡσαῦ τὰς γυναῖκας αὐτοῦ, καὶ τοὺς υἱοὺς αὐτοῦ, καὶ τὰς θυγατέρας αὐτοῦ, καὶ πάντα τὰ σώματα τοῦ οἴκου αὐτοῦ, καὶ πάντα τὰ ὑπάρχοντα αὐτοῦ, καὶ πάντα τὰ κτήνη, καὶ πάντα ὅσα ἐκτήσατο, καὶ πάντα ὅσα περιεποιήσατο ἐν γῇ Χαναάν· καὶ ἐπορεύθη Ἡσαῦ ἐκ τῆς γῆς Χαναὰν ἀπὸ προσώπου Ἰακὼβ τοῦ ἀδελφοῦ αὐτοῦ.
\VS{7}Ἦν γὰρ αὐτῶν τὰ ὑπάρχοντα πολλὰ, τοῦ οἰκεῖν ἅμα· καὶ οὐκ ἠδύνατο ἡ γῆ τῆς παροικήσεως αὐτῶν φέρειν αὐτοὺς, ἀπὸ τοῦ πλήθους τῶν ὑπαρχόντων αὐτῶν.
\VS{8}Κατῴκησε δὲ Ἡσαῦ ἐν τῷ ὄρει Σηείρ· Ἡσαῦ αὐτός ἐστιν Ἐδώμ.
\VS{9}Αὗται δὲ αἱ γενέσεις Ἡσαῦ πατρὸς Ἐδὼμ ἐν τῷ ὄρει Σηείρ.
\VS{10}Καὶ ταῦτα τὰ ὀνόματα τῶν υἱῶν Ἡσαῦ· Ἑλιφὰς υἱὸς Ἀδὰς γυναικὸς Ἡσαῦ· καὶ Ῥαγουὴλ υἱὸς Βασεμὰθ γυναικὸς Ἡσαῦ.
\VS{11}Ἐγένοντο δὲ Ἑλιφὰς υἱοὶ, Θαιμὰν, Ὠμὰρ, Σωφὰρ, Γοθὼμ, καὶ Κενέζ.
\VS{12}Θαμνὰ δὲ ἦν παλλακὴ Ἑλιφὰς τοῦ υἱοῦ Ἡσαῦ· καὶ ἔτεκε τῷ Ἑλιφὰς τὸν Ἀμαλήκ· οὗτοι υἱοὶ Ἀδὰς γυναικὸς Ἡσαῦ.
\VS{13}Οὗτοι δὲ υἱοὶ Ῥαγουὴλ, Ναχὼθ, Ζαρὲ, Σομὲ, καὶ Μοζέ· οὗτοι ἦσαν υἱοὶ Βασεμὰθ γυναικὸς Ἡσαῦ.
\VS{14}Οὗτοι δὲ υἱοὶ Ὀλιβεμὰς θυγατρὸς Ἀνὰ τοῦ υἱοῦ Σεβεγὼν, γυναικὸς Ἡσαῦ· ἔτεκε δὲ τῷ Ἡσαῦ τὸν Ἰεοὺς, καὶ τὸν Ἰεγλὸμ, καὶ τὸν Κορέ.
\VS{15}Οὗτοι ἡγεμόνες υἱοὶ Ἡσαῦ· υἱοὶ Ἑλιφὰς πρωτοτόκου Ἡσαῦ· ἡγεμὼν Θαιμὰν, ἡγεμὼν Ὠμὰρ, ἡγεμὼν Σωφὰρ, ἡγεμὼν Κενὲζ,
\VS{16}ἡγεμὼν Κορὲ, ἡγεμὼν Γοθὼμ, ἡγεμὼν Ἀμαλήκ· οὗτοι ἡγεμόνες Ἑλιφὰς ἐν γῇ Ἰδουμαίᾳ· οὗτοι υἱοὶ Ἀδάς.
\VS{17}Καὶ οὗτοι υἱοὶ Ῥαγουὴλ υἱοῦ Ἡσαῦ· ἡγεμὼν Ναχὼθ, ἡγεμὼν Ζαρὲ, ἡγεμὼν Σομὲ, ἡγεμὼν Μοζέ· οὗτοι ἡγεμόνες Ῥαγουὴλ ἐν γῇ Ἐδώμ· οὗτοι υἱοὶ Βασεμὰθ γυναικὸς Ἡσαῦ.
\VS{18}Οὗτοι δὲ υἱοὶ Ὀλιβεμὰς γυναικὸς Ἡσαῦ· ἡγεμὼν Ἰεοὺς, ἡγεμὼν Ἰεγλὸμ, ἡγεμὼν Κορέ· οὗτοι ἡγεμόνες Ὀλιβεμὰς θυγατρὸς Ἀνὰ γυναικὸς Ἡσαῦ.
\VS{19}Οὗτοι υἱοὶ Ἡσαῦ, καὶ οὗτοι ἡγεμόνες αὐτῶν· οὗτοί εἰσιν υἱοὶ Ἐδώμ.
\VS{20}Οὗτοι δὲ υἱοὶ Σηεὶρ τοῦ Χοῤῥαίου, τοῦ κατοικοῦντος τὴν γῆν· Λωτὰν, Σωβὰλ, Σεβεγὼν, Ἀνὰ,
\VS{21}καὶ Δησὼν, καὶ Ἀσὰρ, καὶ Ῥισών· οὗτοι ἡγεμόνες τοῦ Χοῤῥαίου, τοῦ υἱοῦ Σηεὶρ ἐν τῇ γῇ Ἐδώμ.
\VS{22}Ἐγένοντο δὲ υἱοὶ Λωτάν· Χοῤῥὶ, καὶ Αἱμάν· ἀδελφὴ δὲ Λωτὰν, Θαμνά.
\VS{23}Οὗτοι δὲ υἱοὶ Σωβάλ· Γωλὰμ, καὶ Μαναχὰθ, καὶ Γαιβὴλ, καὶ Σωφὰρ, καὶ Ὠμάρ.
\VS{24}Καὶ οὗτοι υἱοὶ Σεβεγὼν, Ἀϊὲ, καὶ Ἀνά· οὗτός ἐστιν Ἀνὰ, ὃς εὗρε τὸν Ἰαμεὶν ἐν τῇ ἐρήμῳ, ὅτε ἔνεμε τὰ ὑποζύγια Σεβεγὼν τοῦ πατρὸς αὐτοῦ·
\VS{25}Οὗτοι δὲ υἱοὶ Ἀνά· Δησὼν, καὶ Ὀλιβεμὰ θυγάτηρ Ἀνά.
\VS{26}Οὗτοι δὲ υἱοὶ Δησών· Ἀμαδὰ, καὶ Ἀσβὰν, καὶ Ἰθρὰν, καὶ Χαῤῥάν.
\VS{27}Οὗτοι δὲ υἱοὶ Ἀσάρ· Βαλαὰμ, καὶ Ζουκὰμ, καὶ Ἰουκάμ.
\VS{28}Οὗτοι δὲ υἱοὶ Ῥισὼν, Ὧς, καὶ Ἀράν.
\VS{29}Οὗτοι δὲ ἡγεμόνες Χοῤῥί· ἡγεμὼν Λωτὰν, ἡγεμὼν Σωβὰλ, ἡγεμὼν Σεβεγὼν, ἡγεμὼν Ἀνὰ,
\VS{30}ἡγεμὼν Δησὼν, ἡγεμὼν Ἀσὰρ, ἡγεμὼν Ῥισών· οὗτοι ἡγεμόνες Χοῤῥὶ ἐν ταῖς ἡγεμονίαις αὐτῶν ἐν γῇ Ἐδώμ.
\par }{\PP \VS{31}Καὶ οὗτοι οἱ βασιλεῖς οἱ βασιλεύσαντες ἐν Ἐδὼμ, πρὸ τοῦ βασιλεῦσαι βασιλέα ἐν Ἰσραήλ.
\VS{32}Καὶ ἐβασίλευσεν ἐν Ἐδὼμ Βαλὰκ υἱὸς Βεώρ· καὶ ὄνομα τῇ πόλει αὐτοῦ, Δενναβά.
\VS{33}Ἀπέθανε δὲ Βαλὰκ, καὶ ἐβασίλευσεν ἀντʼ αὐτοῦ Ἰωβὰβ υἱὸς Ζαρὰ ἐκ Βοσόῤῥας.
\VS{34}Ἀπέθανε δὲ Ἰωβὰβ, καὶ ἐβασίλευσεν ἀντʼ αὐτοῦ Ἀσὼμ ἐκ τῆς γῆς Θαιμανών.
\VS{35}Ἀπέθανε δὲ Ἀσὼμ, καὶ ἐβασίλευσεν ἀντʼ αὐτοῦ Ἀδὰδ υἱὸς Βαρὰδ ὁ ἐκκόψας Μαδιὰμ ἐν τῷ πεδίῳ Μωάβ· καὶ ὄνομα τῇ πόλει αὐτοῦ Γετθαίμ.
\VS{36}Ἀπέθανε δὲ Ἀδὰδ, καὶ ἐβασίλευσεν ἀντʼ αὐτοῦ Σαμαδὰ ἐκ Μασσεκκάς.
\VS{37}Ἀπέθανε δὲ Σαμαδὰ, καὶ ἐβασίλευσεν ἀντʼ αὐτοῦ Σαοὺλ ἐκ Ῥοωβὼθ τῆς παρὰ ποταμόν.
\VS{38}Ἀπέθανε δὲ Σαοὺλ, καὶ ἐβασίλευσεν ἀντʼ αὐτοῦ Βαλλενὼν υἱὸς Ἀχοβώρ.
\VS{39}Ἀπέθανε δὲ Βαλλενὼν υἱὸς Ἀχοβὼρ, καὶ ἐβασίλευσεν ἀντʼ αὐτοῦ Ἀρὰδ υἱὸς Βαράδ· καὶ ὄνομα τῇ πόλει αὐτοῦ Φογώρ· ὄνομα δὲ τῇ γυναικὶ αὐτοῦ Μετεβεὴλ, θυγάτηρ Ματραῒθ, υἱοῦ Μαιζοώβ.
\VS{40}Ταῦτα τὰ ὀνόματα τῶν ἡγεμόνων Ἡσαῦ, ἐν ταῖς φυλαῖς αὐτῶν, κατὰ τόπον αὐτῶν, ἐν ταῖς χώραις αὐτῶν, καὶ ἐν τοῖς ἔθνεσιν αὐτῶν· ἡγεμὼν Θαμνὰ, ἡγεμὼν Γωλὰ, ἡγεμὼν Ἰεθὲρ,
\VS{41}ἡγεμὼν Ὁλιβεμὰς, ἡγεμὼν Ἡλὰς, ἡγεμὼν Φινὼν,
\VS{42}ἡγεμὼν Κενὲζ, ἡγεμὼν Θαιμὰν, ἡγεμὼν Μαζὰρ,
\VS{43}ἡγεμὼν Μαγεδιὴλ, ἡγεμὼν Ζαφωίν· οὗτοι ἡγεμόνες Ἐδὼμ, ἐν ταῖς κατῳκοδομημέναις ἐν τῇ γῇ τῆς κτήσεως αὐτῶν· οὗτος Ἡσαῦ πατὴρ Ἐδώμ.

\par }\Chap{37}{\PP \VerseOne{1}Κατῴκει δὲ Ἰακὼβ ἐν τῇ γῇ, οὗ παρῴκησεν ὁ πατὴρ αὐτοῦ ἐν γῇ Χαναάν· αὗται δὲ αἱ γενέσεις Ἰακώβ.
\VS{2}Ἰωσὴφ δὲ δέκα καὶ ἑπτὰ ἐτῶν ἦν, ποιμαίνων τὰ πρόβατα τοῦ πατρὸς αὐτοῦ μετὰ τῶν ἀδελφῶν αὐτοῦ, ὢν νέος, μετὰ τῶν υἱῶν Βαλλᾶς, καὶ μετὰ τῶν υἱῶν Ζελφᾶς, τῶν γυναικῶν τοῦ πατρὸς αὐτοῦ· κατήνεγκαν δὲ Ἰωσὴφ ψόγον πονηρὸν πρὸς Ἰσραὴλ τὸν πατέρα αὐτῶν.
\VS{3}Ἰακὼβ δὲ ἠγάπα τὸν Ἰωσὴφ παρὰ πάντας τοὺς υἱοὺς αὐτοῦ, ὅτι υἱὸς γήρως ἦν αὐτῷ· ἐποίησε δὲ αὐτῷ χιτῶνα ποικίλον.
\VS{4}Ἰδόντες δὲ οἱ ἀδελφοὶ αὐτοῦ, ὅτι αὐτὸν ὁ πατὴρ φιλεῖ ἐκ πάντων τῶν υἱῶν αὐτοῦ, ἐμίσησαν αὐτὸν, καὶ οὐκ ἠδύναντο λαλεῖν αὐτῷ οὐδὲν εἰρηνικόν.
\VS{5}Ἐνυπνιασθεὶς δὲ Ἰωσὴφ ἐνύπνιον, ἀπήγγειλεν αὐτὸ τοῖς ἀδελφοῖς αὐτοῦ.
\VS{6}Καὶ εἶπεν αὐτοῖς, ἀκούσατε τοῦ ἐνυπνίου τούτου, οὗ ἐνυπνιάσθην.
\VS{7}Ὤμην ὑμᾶς δεσμεύειν δράγματα ἐν μέσῳ τῷ πεδίῳ· καὶ ἀνέστη τὸ ἐμὸν δράγμα, καὶ ὠρθώθη· περιστραφέντα δὲ τὰ δράγματα ὑμῶν, προσεκύνησαν τὸ ἐμὸν δράγμα.
\VS{8}Εἶπαν δὲ αὐτῷ οἱ ἀδελφοὶ αὐτοῦ, μὴ βασιλεύων βασιλεύσεις ἐφʼ ἡμᾶς, ἢ κυριεύων κυριεύσεις ἡμῶν, καὶ προσέθεντο ἔτι μισεῖν αὐτὸν ἕνεκεν τῶν ἐνυπνίων αὐτοῦ, καὶ ἕνεκεν τῶν ῥημάτων αὐτοῦ.
\VS{9}Εἶδε δὲ ἐνύπνιον ἕτερον, καὶ διηγήσατο αὐτὸ τῷ πατρὶ αὐτοῦ, καὶ τοῖς ἀδελφοῖς αὐτοῦ· καὶ εἶπεν, ἰδοὺ ἐνυπνιασάμην ἐνύπνιον ἕτερον· ὥσπερ ὁ ἥλιος, καὶ ἡ σελήνη, καὶ ἕνδεκα ἀστέρες προσεκύνουν με.
\VS{10}Καὶ ἐπετίμησεν αὐτῷ ὁ πατὴρ αὐτοῦ, καὶ εἶπεν αὐτῷ, τί τὸ ἐνύπνιον τοῦτο, ὃ ἐνυπνιάσθης; ἆρά γε ἐλθόντες ἐλευσόμεθα ἐγώ τε καὶ ἡ μήτηρ σου καὶ οἱ ἀδελφοί σου προσκυνῆσαί σοι ἐπὶ τὴν γῆν;
\VS{11}Ἐζήλωσαν δὲ αὐτὸν οἱ ἀδελφοὶ αὐτοῦ· ὁ δὲ πατὴρ αὐτοῦ διετήρησε τὸ ῥῆμα.
\VS{12}Ἐπορεύθησαν δὲ οἱ ἀδελφοὶ αὐτοῦ βόσκειν τὰ πρόβατα τοῦ πατρὸς αὐτῶν εἰς Συχέμ.
\VS{13}Καὶ εἶπεν Ἰσραὴλ πρὸς Ἰωσὴφ, οὐχὶ οἱ ἀδελφοί σου ποιμαίνουσιν εἰς Συχέμ; δεῦρο ἀποστείλω σε πρὸς αὐτούς· εἶπε δὲ αὐτῷ, ἰδοὺ ἐγώ.
\VS{14}Εἶπε δὲ αὐτῷ Ἰσραὴλ, πορευθεὶς ἴδε, εἰ ὑγιαίνουσιν οἱ ἀδελφοί σου, καὶ τὰ πρόβατα, καὶ ἀνάγγειλόν μοι· καὶ ἀπέστειλεν αὐτὸν ἐκ τῆς κοιλάδος τῆς Χεβρών· καὶ ἦλθεν εἰς Συχέμ.
\VS{15}Καὶ εὗρεν αὐτὸν ἄνθρωπος πλανώμενον ἐν τῷ πεδίῳ· ἠρώτησε δὲ αὐτὸν ὁ ἄνθρωπος, λέγων, τί ζητεῖς;
\VS{16}Ὁ δὲ εἶπε, τοὺς ἀδελφούς μου ζητῶ· ἀπάγγειλόν μοι ποῦ βόσκουσιν.
\VS{17}Εἶπε δὲ αὐτῷ ὁ ἄνθρωπος, ἀπῄρκασιν ἐντεῦθεν· ἤκουσα γὰρ αὐτῶν λεγόντων, πορευθῶμεν εἰς Δωθαείμ· καὶ ἐπορεύθη Ἰωσὴφ κατόπισθε τῶν ἀδελφῶν αὐτοῦ, καὶ εὗρεν αὐτοὺς ἐν Δωθαείμ.
\par }{\PP \VS{18}Προεῖδον δὲ αὐτὸν μακρόθεν πρὸ τοῦ ἐγγίσαι αὐτὸν πρὸς αὐτούς· καὶ ἐπονηρεύοντο τοῦ ἀποκτεῖναι αὐτόν.
\VS{19}Εἶπε δὲ ἕκαστος πρὸς τὸν ἀδελφὸν αὐτοῦ, ἰδοὺ ὁ ἐνυπνιαστὴς ἐκεῖνος ἔρχεται.
\VS{20}Νῦν οὖν δεῦτε ἀποκτείνωμεν αὐτὸν, καὶ ῥίψωμεν αὐτὸν εἰς ἕνα τῶν λάκκων· καὶ ἐροῦμεν, θηρίον πονηρὸν κατέφαγεν αὐτόν· καὶ ὀψόμεθα, τί ἔσται τὰ ἐνύπνια αὐτοῦ.
\VS{21}Ἀκούσας δὲ Ῥουβὴν, ἐξείλετο αὐτὸν ἐκ τῶν χειρῶν αὐτῶν· καὶ εἶπεν, οὐ πατάξωμεν αὐτὸν εἰς ψυχήν.
\VS{22}Εἶπε δὲ αὐτοῖς Ῥουβὴν, μὴ ἐκχέητε αἷμα· ἐμβάλλετε αὐτὸν εἰς ἕνα τῶν λάκκων τούτων τῶν ἐν τῇ ἐρήμῳ, χεῖρα δὲ μὴ ἐπενέγκητε αὐτῷ· ὅπως ἐξέληται αὐτὸν ἐκ τῶν χειρῶν αὐτῶν, καὶ ἀποδῷ αὐτὸν τῷ πατρὶ αὐτοῦ.
\VS{23}Ἐγένετο δὲ ἡνίκα ἦλθεν Ἰωσὴφ πρὸς τοὺς ἀδελφοὺς αὐτοῦ, ἐξέδυσαν Ἰωσὴφ τὸν χιτῶνα τὸν ποικίλον τὸν περὶ αὐτόν.
\VS{24}Καὶ λαβόντες αὐτὸν, ἔῤῥιψαν εἰς τὸν λάκκον· ὁ δὲ λάκκος κενὸς, ὕδωρ οὐκ εἶχε.
\VS{25}Ἐκάθισαν δὲ φαγεῖν ἄρτον· καὶ ἀναβλέψαντες τοῖς ὀφθαλμοῖς εἶδον, καὶ ἰδοὺ ὁδοιπόροι Ἰσμαηλῖται ἤρχοντο ἐκ Γαλαάδ· καὶ αἱ κάμηλοι αὐτῶν ἔγεμον θυμιαμάτων καὶ ῥητίνης καὶ στακτῆς. ἐπορεύοντο δὲ καταγαγεῖν εἰς Αἴγυπτον.
\par }{\PP \VS{26}Εἶπε δὲ Ἰούδας πρὸς τοὺς ἀδελφοὺς αὐτοῦ, τί χρήσιμον, ἐὰν ἀποκτείνωμεν τὸν ἀδελφὸν ἡμῶν, καὶ κρύψωμεν τὸ αἷμα αὐτοῦ;
\VS{27}Δεῦτε ἀποδώμεθα αὐτὸν τοῖς Ἰσμαηλίταις τούτοις· αἱ δὲ χεῖρες ἡμῶν μὴ ἔστωσαν ἐπʼ αὐτὸν, ὅτι ἀδελφὸς ἡμῶν καὶ σὰρξ ἡμῶν ἐστίν. Ἤκουσαν δὲ οἱ ἀδελφοὶ αὐτοῦ.
\VS{28}Καὶ παρεπορεύοντο οἱ ἄνθρωποι οἱ Μαδιηναῖοι ἔμποροι, καὶ ἐξείλκυσαν καὶ ἀνεβίβασαν τὸν Ἰωσὴφ ἐκ τοῦ λάκκου· καὶ ἀπέδοντο τὸν Ἰωσὴφ τοῖς Ἰσμαηλίταις εἴκοσι χρυσῶν. Καὶ κατήγαγον τὸν Ἰωσὴφ εἰς Αἴγυπτον.
\VS{29}Ἀνέστρεψε δὲ Ῥουβὴν ἐπὶ τὸν λάκκον, καὶ οὐχ ὁρᾷ τὸν Ἰωσὴφ ἐν τῷ λάκκῳ· καὶ διέῤῥηξε τὰ ἱμάτια αὐτοῦ.
\VS{30}Καὶ ἐπέστρεψε πρὸς τοὺς ἀδελφοὺς αὐτοῦ, καὶ εἶπε, τὸ παιδάριον οὐκ ἔστιν· ἐγὼ δὲ ποῦ πορεύομαι ἔτι;
\VS{31}Λαβόντες δὲ τὸν χιτῶνα τοῦ Ἰωσὴφ, ἔσφαξαν ἔριφον αἰγῶν, καὶ ἐμόλυναν τὸν χιτῶνα τῷ αἵματι.
\VS{32}Καὶ ἀπέστειλαν τὸν χιτῶνα τὸν ποικίλον, καὶ εἰσήνεγκαν τῷ πατρὶ αὐτῶν· καὶ εἶπαν, τοῦτον εὕρομεν, ἐπίγνωθι εἰ χιτὼν τοῦ υἱοῦ σου ἐστὶν, ἢ οὔ.
\VS{33}Καὶ ἐπέγνω αὐτὸν, καὶ εἶπε, χιτὼν τοῦ υἱοῦ μου ἐστί· θηρίον πονηρὸν κατέφαγεν αὐτόν· θηρίον ἥρπασε τὸν Ἰωσήφ.
\VS{34}Διέῤῥηξε δὲ Ἰακὼβ τὰ ἱμάτια αὐτοῦ, καὶ ἐπέθετο σάκκον ἐπὶ τὴν ὀσφῦν αὐτοῦ, καὶ ἐπένθει τὸν υἱὸν αὐτοῦ ἡμέρας πολλάς.
\VS{35}Συνήχθησαν δὲ πάντες οἱ υἱοὶ αὐτοῦ καὶ αἱ θυγατέρες, καὶ ἦλθον παρακαλέσαι αὐτόν· καὶ οὐκ ἤθελε παρακαλεῖσθαι, λέγων, ὅτι καταβήσομαι πρὸς τὸν υἱόν μου πενθῶν εἰς ᾅδου· καὶ ἔκλαυσεν αὐτὸν ὁ πατὴρ αὐτοῦ.
\VS{36}Οἱ δὲ Μαδιηναῖοι ἀπέδοντο τὸν Ἰωσὴφ εἰς Αἴγυπτον τῷ Πετεφρῇ τῷ σπάδοντι Φαραὼ ἀρχιμαγείρῳ.

\par }\Chap{38}{\PP \VerseOne{1}Ἐγένετο δὲ ἐν τῷ καιρῷ ἐκείνῳ, κατέβη Ἰούδας ἀπὸ τῶν ἀδελφῶν αὐτοῦ, καὶ ἀφίκετο ἕως πρὸς ἄνθρωπον τινὰ Ὀδολλαμίτην, ᾧ ὄνομα Εἰράς.
\VS{2}Καὶ εἶδεν ἐκεῖ Ἰούδας θυγατέρα ἀνθρώπου Χαναναίου, ᾗ ὄνομα Σαυά· καὶ ἔλαβεν αὐτὴν, καὶ εἰσῆλθε πρὸς αὐτήν.
\VS{3}Καὶ συλλαβοῦσα ἔτεκεν υἱὸν, καὶ ἐκάλεσε τὸ ὄνομα αὐτοῦ, Ἤρ.
\VS{4}Καὶ συλλαβοῦσα ἔτεκεν υἱὸν ἔτι, καὶ ἐκάλεσε τὸ ὄνομα αὐτοῦ, Αὐνάν.
\VS{5}Καὶ προσθεῖσα ἔτεκεν υἱὸν, καὶ ἐκάλεσε τὸ ὄνομα αὐτοῦ, Σηλώμ· αὕτη δὲ ἦν ἐν Χασβὶ, ἡνίκα ἔτεκεν αὐτούς.
\VS{6}Καὶ ἔλαβεν Ἰούδας γυναῖκα Ἢρ τῷ πρωτοτόκῳ αὐτοῦ, ᾗ ὄνομα Θάμαρ.
\VS{7}Ἐγένετο δὲ Ἢρ πρωτότοκος Ἰούδα πονηρὸς ἔναντι Κυρίου· καὶ ἀπέκτεινεν αὐτὸν ὁ Θεός.
\VS{8}Εἶπε δὲ Ἰούδας τῷ Αὐνάν· εἴσελθε πρὸς τὴν γυναῖκα τοῦ ἀδελφοῦ σου, καὶ ἐπιγάμβρευσαι αὐτὴν, καὶ ἀνάστησον σπέρμα τῷ ἀδελφῷ σου.
\VS{9}Γνοὺς δὲ Αὐνὰν, ὅτι οὐκ αὐτῷ ἔσται τὸ σπέρμα, ἐγένετο ὅταν εἰσήρχετο πρὸς τὴν γυναῖκα τοῦ ἀδελφοῦ αὐτου, ἐξέχεεν ἐπὶ τὴν γῆν, τοῦ μὴ δοῦναι σπέρμα τῷ ἀδελφῷ αὐτοῦ.
\VS{10}Πονηρὸν δὲ ἐφάνη ἐναντίον τοῦ Θεοῦ, ὅτι ἐποίησε τοῦτο· καὶ ἐθανάτωσε καὶ τοῦτον.
\par }{\PP \VS{11}Εἶπε δὲ Ἰούδας Θάμαρ τῇ νύμφῃ αὐτοῦ, κάθου χήρα ἐν τῷ οἴκῳ τοῦ πατρός σου, ἕως μέγας γένηται Σηλὼμ ὁ υἱός μου· εἶπε γάρ, μή ποτε ἀποθάνῃ καὶ οὗτος, ὥσπερ καὶ οἱ ἀδελφοὶ αὐτοῦ. Ἀπελθοῦσα δὲ Θάμαρ ἐκάθητο ἐν τῷ οἴκῳ τοῦ πατρὸς αὐτῆς.
\VS{12}Ἐπληθύνθησαν δὲ αἱ ἡμέραι, καὶ ἀπέθανε Σαυὰ ἡ γυνὴ Ἰούδα· καὶ παρακληθεὶς Ἰούδας ἀνέβη ἐπὶ τοὺς κείροντας τὰ πρόβατα αὐτοῦ, αὐτὸς καὶ Εἰρὰς ὁ ποιμὴν αὐτοῦ ὁ Ὀδολλαμίτης εἰς Θαμνά.
\VS{13}Καὶ ἀπηγγέλε Θάμαρ τῇ νύμφῃ αὐτοῦ, λέγοντες, ἰδοὺ ὁ πενθερός σου ἀναβαίνει εἰς Θαμνὰ, κεῖραι τὰ πρόβατα αὐτοῦ.
\VS{14}Καὶ περιελομένη τὰ ἱμάτια τῆς χηρεύσεως ἀφʼ ἑαυτῆς, περιέβαλε τὸ θέριστρον, καὶ ἐκαλλωπίσατο, καὶ ἐκάθισε πρὸς ταῖς πύλαις Αἰνὰν, ἥ ἐστιν ἐν παρόδῳ Θαμνά· ἴδε γὰρ ὅτι μέγας γέγονε Σηλὼμ, αὐτὸς δὲ οὐκ ἔδωκεν αὐτὴν αὐτῷ γυναῖκα.
\VS{15}Καὶ ἰδὼν αὐτὴν Ἰούδας ἔδοξεν αὐτὴν πόρνην εἶναι· κατεκαλύψατο γὰρ τὸ πρόσωπον αὐτῆς καὶ οὐκ ἐπέγνω αὐτήν.
\VS{16}Ἐξέκλινε δὲ πρὸς αὐτὴν τὴν ὁδόν· καὶ εἶπεν αὐτῇ, ἔασόν με εἰσελθεῖν πρός σε· οὐ γὰρ ἔγνω, ὅτι νύμφη αὐτοῦ ἐστίν· ἡ δὲ εἶπε, τί μοι δώσεις, ἐὰν εἰσέλθῃς πρός με;
\VS{17}Ὁ δὲ εἶπεν, ἐγώ σοι ἀποστελλῶ ἔριφον αἰγῶν ἐκ τῶν προβάτων μον· ἡ δὲ εἶπεν, ἐὰν δῷς μοι ἀῤῥαβῶνα, ἕως τοῦ ἀποστεῖλαί σε.
\VS{18}Ὁ δὲ εἶπε, τίνα τὸν ἀῤῥαβῶνά σοι δώσω; ἡ δὲ εἶπε, τὸν δακτύλιόν σου, καὶ τὸν ὁρμίσκον, καὶ τὴν ῥάβδον τὴν ἐν τῇ χειρίσου. Καὶ ἔδωκεν αὐτῇ, καὶ εἰσῆλθε πρὸς αὐτήν· καὶ ἐν γαστρὶ ἔλαβεν ἐξ αὐτοῦ.
\VS{19}Καὶ ἀναστᾶσα ἀπῆλθε, καὶ περιείλετο τὸ θέριστρον αὐτῆς ἀφʼ ἑαυτῆς, καὶ ἐνεδύσατο τὰ ἱμάτια τῆς χηρεύσεως αὐτῆς.
\VS{20}Ἀπέστειλε δὲ Ἰούδας τὸν ἔριφον ἐξ αἰγῶν ἐν χειρὶ τοῦ ποιμένος αὐτοῦ τοῦ Ὀδολλαμείτου, κομίσασθαι παρὰ τῆς γυναικὸς τὸν ἀῤῥαβῶνα· καὶ οὐχ εὗρεν αὐτήν.
\VS{21}Ἐπηρώτησε δὲ τοὺς ἄνδρας τοὺς ἐκ τοῦ τόπου, ποῦ ἐστιν ἡ πόρνη ἡ γενομένη ἐν Αἰνὰν ἐπὶ τῆς ὁδοῦ; καὶ εἶπαν, οὐκ ἦν ἐνταῦθα πόρνη.
\VS{22}Καὶ ἀπεστράφη πρὸς Ἰούδαν, καὶ εἶπεν, οὐχ εὗρον· καὶ οἱ ἄνθρωποι οἱ ἐκ τοῦ τόπου λέγουσι, μὴ εἶναι ὧδε πόρνην.
\VS{23}Εἶπε δὲ Ἰούδας, ἐχέτω αὐτά· ἀλλὰ μή ποτε καταγελασθῶμεν· ἐγὼ μὲν ἀπέσταλκα τὸν ἔριφον τοῦτον, σὺ δὲ οὐχ εὕρηκας.
\VS{24}Ἐγένετο δὲ μετὰ τρίμηνον ἀνηγγέλη τῷ Ἰούδα, λέγοντες, ἐκπεπόρνευκε Θάμαρ ἡ νύμφη σου, καὶ ἰδοὺ ἐν γαστρὶ ἔχει ἐκ πορνείας· Εἶπε δὲ Ἰούδας, ἐξαγάγετε αὐτὴν, καὶ κατακαυθήτω.
\VS{25}Αὐτὴ δὲ ἀγομένη ἀπέστειλε πρὸς τὸν πενθερὸν αὐτὴς, λέγουσα, ἐκ τοῦ ἀνθρώπου οὕτινος ταῦτά ἐστιν, ἐγὼ ἐν γαστρὶ ἔχω· καὶ εἶπεν, ἐπίγνωθι τίνος ὁ δακτύλιος, καὶ ὁ ὁρμίσκος καὶ ἡ ῥάβδος αὕτη.
\VS{26}Ἐπέγνω δὲ Ἰούδας, καὶ εἶπε, δεδικαίωται Θάμαρ ἢ ἐγώ· οὗ ἕνεκεν οὐκ ἔδωκα αὐτὴν Σηλὼμ τῷ υἱῷ μου· Καὶ οὐ προσέθετο ἔτι τοῦ γνῶναι αὐτήν.
\VS{27}Ἐγένετο δὲ ἡνίκα ἔτικτε, καὶ τῇδε ἦν δίδυμα ἐν τῇ γαστρὶ αὐτῆς.
\VS{28}Ἐγένετο δὲ ἐν τῷ τίκτειν αὐτὴν, ὁ εἷς προεξήνεγκεν τὴν χεῖρα· λαβοῦσα δὲ ἡ μαῖα, ἔδησεν ἐπὶ τὴν χεῖρα αὐτοῦ κόκκινον, λέγουσα, οὗτος ἐξελεύσεται πρότερος.
\VS{29}Ὡς δὲ ἐπισυνήγαγε τὴν χεῖρα, καὶ εὐθὺς ἐξῆλθεν ὁ ἀδελφὸς αὐτοῦ· ἡ δὲ εἶπε, τί διεκόπη διὰ σὲ φραγμός; καὶ ἐκάλεσε τὸ ὄνομα αὐτοῦ, Φαρές.
\VS{30}Καὶ μετὰ τοῦτο ἐξῆλθεν ὁ ἀδελφὸς αὐτοῦ, ἐφʼ ᾧ ἦν ἐπὶ τῇ χειρὶ αὐτοῦ τὸ κόκκινον· καὶ ἐκάλεσε τὸ ὄνομα αὐτοῦ, Ζαρά.

\par }\Chap{39}{\PP \VerseOne{1}Ἰωσὴφ δὲ κατήχθη εἰς Αἴγυπτον· καὶ ἐκτήσατο αὐτὸν Πετεφρὴς ὁ εὐνοῦχος Φαραὼ, ὁ ἀρχιμάγειρος, ἀνὴρ Αἰγύπτιος, ἐκ χειρῶν τῶν Ἰσμαηλιτῶν, οἳ κατήγαγον αὐτὸν ἐκεῖ.
\VS{2}Καὶ ἦν Κύριος μετὰ Ἰωσήφ· καὶ ἦν ἀνὴρ ἐπιτυγχάνων· καὶ ἐγένετο ἐν τῷ οἴκῳ παρὰ τῷ κυρίῳ αὐτοῦ τῷ Αἰγυπτίῳ.
\VS{3}Ἤδει δὲ ὁ κύριος αὐτοῦ, ὅτι ὁ Κύριος ἦν μετʼ αὐτοῦ, καὶ ὅσα ἐὰν ποιῇ, Κύριος εὐοδοῖ ἐν ταῖς χερσὶν αὐτοῦ.
\VS{4}Καὶ εὗρεν Ἰωσὴφ χάριν ἐναντίον τοῦ κυρίου αὐτοῦ, καὶ εὐηρέστησεν αὐτῷ. Καὶ κατέστησε αὐτὸν ἐπὶ τοῦ οἴκου αὐτοῦ· καὶ πάντα ὅσα ἦν αὐτῷ, ἔδωκε διὰ χειρὸς Ἰωσήφ.
\VS{5}Ἐγένετο δὲ μετὰ τὸ καταστῆναι αὐτὸν ἐπὶ τοῦ οἴκου αὐτοῦ, καὶ ἐπὶ πάντα ὅσα ἦν αὐτῷ, καὶ ηὐλόγησε Κύριος τὸν οἶκον τοῦ Αἰγυπτίου διὰ Ἰωσήφ· καὶ ἐγενήθη εὐλογία Κυρίου ἐν πᾶσι τοῖς ὑπάρχουσιν αὐτῷ ἐν τῷ οἴκῳ, καὶ ἐν τῷ ἀγρῷ αὐτοῦ.
\VS{6}Καὶ ἐπέτρεψε πάντα ὅσα ἦν αὐτῷ, εἰς χεῖρας Ἰωσήφ· καὶ οὐκ ᾔδει τῶν καθʼ αὑτὸν οὐδὲν, πλὴν τοῦ ἄρτου, οὗ ἤσθιεν αὐτός. Καὶ ἦν Ἰωσὴφ καλὸς τῷ εἴδει, καὶ ὡραῖος τῇ ὄψει σφόδρα.
\VS{7}Καὶ ἐγένετο μετὰ τὰ ῥήματα ταῦτα, καὶ ἐπέβαλεν ἡ γυνὴ τοῦ κυρίου αὐτοῦ τοὺς ὀφθαλμοὺς αὐτῆς ἐπὶ Ἰωσήφ· καὶ εἶπεν, κοιμήθητι μετʼ ἐμοῦ.
\VS{8}Ὁ δὲ οὐκ ἤθελεν· εἶπε δὲ τῇ γυναικὶ τοῦ κυρίου αὐτοῦ, εἰ ὁ κύριός μου οὐ γινώσκει διʼ ἐμὲ οὐδὲν ἐν τῷ οἴκῳ αὐτοῦ, καὶ πάντα ὅσα ἐστὶν αὐτῷ ἔδωκεν εἰς τὰς χεῖράς μου,
\VS{9}καὶ οὐχ ὑπερέχει ἐν τῇ οἰκίᾳ ταύτῆ οὐθὲν ἐμοῦ, οὐδὲ ὑπεξῄρηται ἀπʼ ἐμοῦ οὐδὲν, πλὴν σοῦ, διὰ τὸ σὲ γυναῖκα αὐτοῦ εἶναι, καὶ πῶς ποιήσω τὸ ῥῆμα τὸ πονηρὸν τοῦτο, καὶ ἁμαρτήσομαι ἐναντίον τοῦ Θεοῦ;
\VS{10}Ἡνίκα δὲ ἐλάλει τῷ Ἰωσὴφ ἡμέραν ἐξ ἡμέρας, καὶ οὐχ ὑπήκουεν αὐτῇ καθεύδειν μετʼ αὐτῆς, τοῦ συγγενέσθαι αὐτῇ.
\VS{11}Ἐγένετο δὲ τοιαύτη τις ἡμέρα, καὶ εἰσῆλθεν Ἰωσὴφ εἰς τὴν οἰκίαν ποιεῖν τὰ ἔργα αὐτοῦ, καὶ οὐθεὶς ἦν τῶν ἐν τῇ οἰκίᾳ ἔσω.
\VS{12}Καὶ ἐπεσπάσατο αὐτὸν τῶν ἱματίων, λέγουσα, κοιμήθητι μετʼ ἐμοῦ· καὶ καταλιπὼν τὰ ἱμάτια αὐτοῦ ἐν ταῖς χερσὶν αὐτῆς ἔφυγε, καὶ ἐξῆλθεν ἔξω.
\VS{13}Καὶ ἐγένετο ὡς εἶδεν ὅτι καταλιπὼν τὰ ἱμάτια αὐτοῦ ἐν ταῖς χερσὶν αὐτῆς ἔφυγε, καὶ ἐξῆλθεν ἔξω,
\VS{14}καὶ ἐκάλεσε τοὺς ὄντας ἐν τῇ οἰκίᾳ, καὶ εἶπεν αὐτοῖς, λέγουσα, ἴδετε, εἰσήγαγε ἡμῖν παῖδα Ἐβραῖον, ἐμπαίζειν ἡμῖν· εἰσῆλθε πρός με, λέγων, κοιμήθητι μετʼ ἐμοῦ· καὶ ἐβόησα φωνῇ μεγάλῃ.
\VS{15}Ἐν δὲ τῷ ἀκοῦσαι αὐτὸν, ὅτι ὕψωσα τὴν φωνήν μου καὶ ἐβόησα, καταλιπὼν τὰ ἱμάτια αὐτοῦ παρʼ ἐμοὶ ἔφυγε, καὶ ἐξῆλθεν ἔξω.
\VS{16}Καὶ καταλιμπάνει τὰ ἱμάτια παρʼ ἑαυτῇ, ἕως ἦλθεν ὁ κύριος εἰς τὸν οἶκον αὐτοῦ.
\VS{17}Καὶ ἐλάλησεν αὐτῷ κατὰ τὰ ῥήματα ταῦτα, λέγουσα, εἰσῆλθε πρός με ὁ παῖς ὁ Ἑβραῖος, ὃν εἰσήγαγες πρὸς ἡμᾶς, ἐμπαῖξαί μοι· καὶ εἶπέ μοι, κοιμηθήσομαι μετὰ σοῦ.
\VS{18}Ὡς δὲ ἤκοῦσεν, ὅτι ὕψωσα τὴν φωνήν μου καὶ ἐβόησα, καταλιπὼν τὰ ἱμάτια αὐτοῦ παρʼ ἐμοὶ ἔφυγε, καὶ ἐξῆλθεν ἔξω.
\VS{19}Ἐγένετο δὲ, ὡς ἤκουσεν ὁ κύριος τὰ ῥήματα τῆς γυναικὸς αὐτοῦ, ὅσα ἐλάλησε πρὸς αὐτὸν, λέγουσα, οὕτως ἐποίησέ μοι ὁ παῖς σου, καὶ ἐθυμώθη ὀργῇ.
\par }{\PP \VS{20}Καὶ λαβὼν ὁ κύριος Ἰωσὴφ, ἐνέβαλε αὐτὸν εἰς τὸ ὀχύρωμα, εἰς τὸν τόπον ἐν ᾧ οἱ δεσμῶται τοῦ βασιλέως κατέχονται ἐκεῖ ἐν τῷ ὀχυρώματι.
\VS{21}Καὶ ἦν Κύριος μετὰ Ἰωσὴφ, καὶ κατέχεεν αὐτοῦ ἔλεος· καὶ ἔδωκεν αὐτῷ χάριν ἐναντίον τοῦ ἀρχιδεσμοφύλακος.
\VS{22}Καὶ ἔδωκεν ὁ ἀρχιδεσμοφύλαξ τὸ δεσμωτήριον διὰ χειρὸς Ἰωσὴφ, καὶ πάντας τοὺς ἀπηγμένους ὅσοι ἐν τῷ δεσμωτηρίῳ, καὶ πάντα ὅσα ποιοῦσιν ἐκεῖ, αὐτὸς ἦν ποιῶν.
\VS{23}Οὐκ ἦν ὁ ἀρχιδεσμοφύλαξ τοῦ δεσμωτηρίου γινώσκον διʼ αὐτὸν οὐθέν· πάντα γὰρ ἦν διὰ χειρὸς Ἰωσὴφ, διὰ τὸ τὸν Κύριον μετʼ αὐτοῦ εἶναι· καὶ ὅσα αὐτὸς ἐποίει, ὁ Κύριος εὐώδο ἐν ταῖς χερσὶν αὐτοῦ.

\par }\Chap{40}{\PP \VerseOne{1}Ἐγένετο δὲ μετὰ τὰ ῥήματα ταῦτα, ἥμαρτεν ὁ ἀρχιοινοχόος τοῦ βασιλέως Αἰγύπτου, καὶ ὁ ἀρχισιτοποιὸς, τῷ κυρίῳ αὐτῶν βασιλεῖ Αἰγύπτου.
\VS{2}Καὶ ὠργίσθη Φαραὼ ἐπὶ τοῖς δυσὶν εὐνούχοις αὐτοῦ, ἐπὶ τῷ ἀρχιοινοχόῳ, καὶ ἐπὶ τῷ ἀρχισιτοποιῷ·
\VS{3}Καὶ ἔθετο αὐτοὺς ἐν φυλακῇ εἰς τὸ δεσμωτήριον, εἰς τὸν τόπον, οὗ Ἰωσὴφ ἀπῆκτο ἐκεῖ.
\VS{4}Καὶ συνέστησεν ὁ ἀρχιδεσμώτης τῷ Ἰωσὴφ αὐτούς· καὶ παρέστη αὐτοῖς· ἦσαν δὲ ἡμέρας ἐν τῇ φυλακῇ.
\VS{5}Καὶ εἶδον ἀμφότεροι ἐνύπνιον ἐν μιᾷ νυκτί· ἡ δὲ ὅρασις τοῦ ἐνυπνίου τοῦ ἀρχιοινοχόου καὶ ἀρχισιτοποιοῦ, οἳ ἦσαν τῷ βασιλεῖ Αἰγύπτου, οἱ ὄντες ἐν τῷ δεσμωτηρίῳ, ἦν αὕτη.
\VS{6}Εἰσῆλθε πρὸς αὐτοὺς Ἰωσὴφ τὸ πρωῒ, καὶ εἶδεν αὐτοὺς, καὶ ἦσαν τεταραγμένοι.
\VS{7}Καὶ ἠρώτα τοὺς εὐνούχους Φαραὼ, οἳ ἦσαν μετʼ αὐτοῦ ἐν τῇ φυλακῇ παρὰ τῷ κυρίῳ αὐτοῦ, λέγων, τί ὅτι τὰ πρόσωπα ὑμῶν σκυθρωπὰ σήμερον;
\VS{8}Οἱ δὲ εἶπαν αὐτῷ, ἐνύπνιον εἴδομεν, καὶ ὁ συγκρίνων οὐκ ἔστιν αὐτό· εἶπε δὲ αὐτοῖς Ἰωσὴφ, οὐχὶ διὰ τοῦ Θεοῦ ἡ διασάφησις αὐτῶν ἐστι; διηγήσασθε οὖν μοὶ.
\VS{9}Καὶ διηγήσατο ὁ ἀρχιοινοχόος τὸ ἐνύπνιον αὐτοῦ τῷ Ἰωσήφ· καὶ εἶπεν, ἐν τῷ ὕπνῳ μου ἦν ἄμπελος ἐναντίον μου.
\VS{10}Ἐν δὲ τῇ ἀμπέλῳ τρεῖς πυθμένες, καὶ αὐτὴ θάλλουσα, ἀνενηνοχυῖα βλαστούς· πέπειροι οἱ βότρυες σταφυλῆς.
\VS{11}Καὶ τὸ ποτήριον Φαραὼ ἐν τῇ χειρί μου· καὶ ἔλαβον τὴν σταφυλὴν, καὶ ἐξέθλιψα αὐτὴν εἰς τὸ ποτήριον, καὶ ἔδωκα τὸ ποτήριον εἰς τὴν χεῖρα Φαραώ.
\VS{12}Καὶ εἶπεν αὐτῷ Ἰωσὴφ, τοῦτο ἡ σύγκρίσις αὐτοῦ· οἱ τρεῖς πυθμένες, τρεῖς ἡμέραι εἰσίν.
\VS{13}Ετι τρεῖς ἡμέραι, καὶ μνησθήσεται Φαραὼ τῆς ἀρχῆς σου, καὶ ἀποκαταστήσει σε ἐπὶ τὴν ἀρχιοινοχοΐαν σου, καὶ δώσεις τὸ ποτήριον Φαραὼ εἰς τὴν χεῖρα αὐτοῦ κατὰ τὴν ἀρχήν σου τὴν προτέραν, ὡς ἦσθα οἰνοχοῶν.
\VS{14}Ἀλλὰ μνήσθητί μου διὰ σεαυτοῦ, ὅταν εὖ γενηταί σοι· καὶ ποιήσεις ἐν ἐμοὶ ἔλεος· καὶ μνησθήσῃ περὶ ἐμοῦ πρὸς Φαραὼ, καὶ ἐξάξεις με ἐκ τοῦ ὀχυρώματος τούτου.
\VS{15}Ὅτι κλοπῇ ἐκλάπην ἐκ γῆς Ἑβραίων, καὶ ὧδε οὐκ ἐποίησα οὐδὲν, ἀλλʼ ἐνέβαλόν με εἰς τὸν λάκκον τοῦτον.
\VS{16}Καὶ εἶδεν ὁ ἀρχισιτοποιὸς ὅτι ὀρθῶς συνέκριεν· καὶ εἶπε τῷ Ἰωσὴφ, κᾀγὼ εἶδον ἐνύπνιον· καὶ ᾤμην τρία κανᾶ χονδριτῶν αἴρειν ἐπὶ τῆς κεφαλῆς μου·
\VS{17}Ἐν δὲ κανῷ τῷ ἐπάνω ἀπὸ πάντων τῶν γενῶν, ὧν Φαραὼ ἐσθίει, ἔργον σιτοποιοῦ, καὶ τὰ πετεινὰ τοῦ οὐρανου κατήσθιεν αὐτὰ ἀπὸ τοῦ κανοῦ τοῦ ἐπάνω τῆς κεφαλῆς μου.
\VS{18}Ἀποκριθεὶς δὲ Ἰωσὴφ εἶπεν αὐτῷ, αὕτη ἡ σύγκρισις αὐτοῦ· τὰ τρία κανᾶ, τρεῖς ἡμέραι εἰσίν·
\VS{19}Ἔτι τριῶν ἡμερῶν, καὶ ἀφελεῖ Φαραὼ τὴν κεφαλήν σου ἀπὸ σου· καὶ κρεμάσει σε ἐπὶ ξύλου, καὶ φάγεται τὰ ὄρνεα τοῦ οὐρανοῦ τὰς σάρκας σου ἀπὸ σοῦ.
\VS{20}Ἐγένετο δὲ ἐν τῇ ἡμέρᾳ τῇ τρίτῃ, ἡμέρα γενέσεως ἦν Φαραὼ, καὶ ἐποίει πότον πᾶσι τοῖς παισὶν αὐτοῦ· καὶ ἐμνήσθη τῆς ἀρχῆς τοῦ οἰνοχόου καὶ τῆς ἀρχῆς τοῦ σιτοποιοῦ ἐν μέσῳ τῶν παίδων αὐτοῦ.
\VS{21}Καὶ ἀποκατέστησε τὸν ἀρχιοινοχόον ἐπὶ τὴν ἀρχὴν αὐτοῦ· καὶ ἔδωκε τὸ ποτήριον εἰς τὴν χεῖρα Φαραώ.
\VS{22}Τὸν δὲ ἀρχισιτοποιὸν ἐκρέμασεν, καθὰ συνέκρινεν αὐτοῖς Ἰωσήφ.
\VS{23}Καὶ οὐκ ἐμνήσθη ὁ ἀρχιοινοχόος τοῦ Ἰωσὴφ, ἀλλαʼ ἐπελάθετο αὐτοῦ.

\par }\Chap{41}{\PP \VerseOne{1}Ἐγένετο δὲ μετὰ δύο ἔτη ἡμερῶν, Φαραὼ εἶδεν ἐνύπνιον· ᾤετο ἑστάναι ἐπὶ τοῦ ποταμοῦ.
\VS{2}Καὶ ἰδοὺ ὥσπερ ἐκ τοῦ ποταμοῦ ἀνέβαινον ἐπτὰ βόες, καλαὶ τῷ εἴδει, καὶ ἐκλεκταὶ ταῖς σαρξὶ, καὶ ἐβόσκοντο ἐν τῷ Ἄχει.
\VS{3}Ἄλλαι δὲ ἑπτὰ βόες ἀνέβαινον μετὰ ταύτας ἐκ τοῦ ποταμοῦ, αἰσχραὶ τῷ εἴδει, καὶ λεπταὶ ταῖς σαρξὶ, καὶ ἐνέμοντο παρὰ τὰς βόας ἐπὶ τὸ χεῖλος τοῦ ποταμοῦ.
\VS{4}Καὶ κατέφαγον αἱ ἑπτὰ βόες αἱ αἰσχραὶ καὶ λεπταὶ ταῖς σαρξὶ τὰς ἑπτὰ βόας τὰς καλὰς τῷ εἴδει καὶ τὰς ἐκλεκτὰς ταῖς σαρξί· ἠγέρθη δὲ Φαραώ.
\VS{5}Καὶ ἐνυπνιάσθη τὸ δεύτερον· καὶ ἰδοὺ ἑπτὰ στάχυες ἀνέβαινον ἐν τῷ πυθμένι ἑνὶ ἐκλεκτοὶ καὶ καλοί.
\VS{6}Καὶ ἰδοὺ ἑπτὰ στάχυες λεπτοὶ καὶ ἀνεμόφθοροι ἀνεφύοντο μετʼ αὐτούς.
\VS{7}Καὶ κατέπιον οἱ ἑπτὰ στάχυες οἱ λεπτοὶ καὶ ἀνεμόφθοροι τοὺς ἑπτὰ στάχυας τοὺς ἐκλεκτοὺς καὶ τοὺς πλήρεις· ἠγέρθη δὲ Φαραὼ, καὶ ἦν ἐνύπνιον.
\VS{8}Ἐγένετο δὲ πρωῒ, καὶ ἐταράχθη ἡ ψυχὴ αὐτοῦ, καὶ ἀποστείλας ἐκάλεσε πάντας τοὺς ἐξηγητὰς Αἰγύπτου, καὶ πάντας τοὺς σοφοὺς αὐτῆς· καὶ διηγήσατο αὐτοῖς Φαραὼ τὸ ἐνύπνιον αὐτοῦ, καὶ οὐκ ἦν ὁ ἀπαγγέλλων αὐτὸ τῷ Φαραώ.
\VS{9}Καὶ ἐλάλησεν ὁ ἀρχιοινοχόος πρὸς Φαραὼ, λέγων, τὴν ἁμαρτίαν μου ἀναμιμνήσκω σήμερον.
\VS{10}Φαραὼ ὠργίσθη τοῖς παισὶν αὐτοῦ, καὶ ἔθετο ἡμᾶς ἐν φυλακῇ, ἐν τῷ οἴκῳ τοῦ ἀρχιμαγείρου, ἐμέ τε καὶ τὸν ἀρχισιτοποιόν.
\VS{11}Καὶ εἴδομεν ἐνύπνιον ἀμφότεροι ἐν νυκτὶ μιᾷ ἐγὼ καὶ αὐτὸς, ἕκαστος κατὰ τὸ αὐτοῦ ἐνύπνιον εἴδομεν.
\VS{12}Ἦν δὲ ἐκεῖ μεθʼ ἡμῶν νεανίσκος παῖς Ἑβραῖος τοῦ ἀρχιμαγείρου, καὶ διηγησάμεθα αὐτῷ, καὶ συνέκρινεν ἡμῖν.
\VS{13}Ἐγενήθη δὲ, καθὼς συνέκρινεν ἡμῖν οὕτω καὶ συνέβη, ἐμέ τε ἀποκατασταθῆναι ἐπὶ τὴν ἀρχήν μου, ἐκεῖνον δὲ κρεμασθῆναι.
\VS{14}Ἀποστείλας δὲ Φαραὼ ἐκάλεσε τὸν Ἰωσήφ· καὶ ἐξήγαγον αὐτὸν ἀπὸ τοῦ ὀχυρώματος, καὶ ἐξύρησαν αὐτὸν, καὶ ἤλλαξαν τὴν στολὴν αὐτοῦ· καὶ ἦλθε πρὸς Φαραώ.
\VS{15}Εἶπε δὲ Φαραὼ πρὸς Ἰωσὴφ, ἐνύπνιον ἑώρακα, καὶ ὁ συγκρίνων οὐκ ἔστιν αὐτό· ἐγὼ δὲ ἀκήκοα περὶ σοῦ λεγόντων, ἀκούσαντά σε ἐνύπνια, συγκρῖναι αὐτά.
\VS{16}Ἀποκριθεὶς δὲ Ἰωσὴφ τῷ Φαραὼ εἶπεν, ἄνευ τοῦ Θεοῦ οὐκ ἀποκριθήσεται τὸ σωτήριον Φαραώ.
\VS{17}Ἐλάλησε δὲ Φαραὼ τῷ Ἰωσὴφ, λέγων, ἐν τῷ ὕπνῳ μου ᾤμην ἑστάναι παρὰ τὸ χεῖλος τοῦ ποταμοῦ.
\VS{18}Καὶ ὥσπερ ἐκ τοῦ ποταμοῦ ἀνέβαινον ἑπτὰ βόες καλαὶ τῷ εἴδει καὶ ἐκλεκταὶ ταῖς σαρξὶ, καὶ ἐνέμοντο ἐν τῷ Ἄχει.
\VS{19}Καὶ ἰδοὺ ἑπτὰ βόες ἕτεραι ἀνέβαινον ὀπίσω αὐτῶν ἐκ τοῦ ποταμοῦ, πονηραὶ καὶ αἰσχραὶ τῷ εἴδει, καὶ λεπταὶ ταῖς σαρξὶν, οἵας οὐκ εἶδον τοιαύτας ἐν ὅλῃ γῇ Αἰγύπτου αἰσχροτέρας.
\VS{20}Καὶ κατέφαγον αἱ ἑπτὰ βόες αἱ αἰσχραὶ καὶ λεπταὶ τὰς ἑπτὰ βόας τὰς πρώτας τὰς καλὰς καὶ τὰς ἐκλεκτάς.
\VS{21}Καὶ εἰσῆλθον εἰς τὰς κοιλίας αὐτῶν· καὶ οὑ διάδηλοι ἐγένοντο, ὅτι εἰσῆλθον εἰς τὰς κοιλίας αὐτῶν· καὶ αἱ ὄψεις αὐτῶν αἰσχραὶ, καθὰ καὶ τὴν ἀρχήν· ἐξεγερθεὶς δὲ ἐκοιμήθην.
\VS{22}Καὶ εἶδον πάλιν ἐν τῷ ὕπνῳ μου, καὶ ὥσπερ ἑπτὰ στάχυες ἀνέβαινον ἐν πυθμένι ἑνὶ πλήρεις καὶ καλοί·
\VS{23}Ἄλλοι δὲ ἑπτὰ στάχυες λεπτοὶ καὶ ἀνεμόφθοροι ἀνεφύοντο ἐχόμενοι αὐτῶν.
\VS{24}Καὶ κατέπιον οἱ ἑπτὰ στάχυες οἱ λεπτοὶ καὶ ἀνεμόφθοροι τοὺς ἑπτὰ στάχυας τοὺς καλοὺς καὶ τοὺς πλήρεις· εἶπα οὖν τοῖς ἐξῆγηταῖς, καὶ οὐκ ἦν ὁ ἀπαγγέλλων μοι αὐτό.
\par }{\PP \VS{25}Καὶ εἶπεν Ἰωσὴφ τῷ Φαραὼ, τὸ ἐνύπνιον Φαραὼ ἕν ἐστιν· ὅσα ὁ Θεὸς ποιεῖ, ἔδειξε τῷ Φαραώ.
\VS{26}Αἱ ἑπτὰ βόες αἱ καλαὶ, ἑπτὰ ἔτη ἐστί· καὶ οἱ ἑπτὰ στάχυες οἱ καλοὶ, ἑπτὰ ἔτη ἐστί· τὸ ἐνύπνιον Φαραὼ ἕν ἐστι.
\VS{27}Καὶ αἱ ἑπτὰ βόες αἱ λεπταὶ, αἱ ἀναβαίνουσαι ὀπίσω αὐτῶν, ἑπτὰ ἔτη ἐστί· καὶ οἱ ἑπτὰ στάχυες οἱ λεπτοὶ καὶ ἀνεμόφθοροι, ἑπτὰ ἔτη ἐστί· ἔσονται ἑπτὰ ἔτη λιμοῦ.
\VS{28}Τὸ δὲ ῥῆμα ὃ εἴρηκα Φαραὼ, ὅσα ὁ Θεὸς ποιεῖ, ἔδειξε τῷ Φαραώ.
\VS{29}Ἰδοὺ ἑπτὰ ἔτη ἔρχεται εὐθηνία πολλὴ ἐν πάσῃ γῇ Αἰγύπτου.
\VS{30}Ἥξει δὲ ἑπτὰ ἔτη λιμοῦ μετὰ ταῦτα· καὶ ἐπιλήσονται τῆς πλησμονῆς τῆς ἐσομένης ἐν ὅλῃ Αἰγύπτῳ· καὶ ἀναλώσει ὁ λιμὸς τῆν γῆν.
\VS{31}Καὶ οὐκ ἐπιγνωσθήσεται ἡ εὐθηνία ἐπὶ τῆς γῆς ἀπὸ τοῦ λιμοῦ τοῦ ἐσομένου μετὰ ταῦτα· ἰσχυρὸς γὰρ ἔσται σφόδρα.
\VS{32}Περὶ δὲ τοῦ δευτερῶσαι τὸ ἐνύπνιον Φαραὼ δὶς, ὅτι ἀληθὲς ἔσται τὸ ῥῆμα τὸ παρὰ τοῦ Θεοῦ· καὶ ταχυνεῖ ὁ Θεὸς τοῦ ποιῆσαι αὐτό.
\VS{33}Νῦν οὖν σκέψαι ἄνθρωπον φρόνιμον καὶ συνετὸν, καὶ κατάστησον αὐτὸν ἐπὶ γῆς Αἰγύπτου.
\VS{34}Καὶ ποιησάτω Φαραὼ καὶ καταστησάτω τοπάρχας ἐπὶ τῆς γῆς· καὶ ἀποπεμπτωσάτωσαν πάντα τὰ γεννήματα τῆς γῆς Αἰγύπτου τῶν ἑπτὰ ἐτῶν τῆς εὐθηνίας,
\VS{35}καὶ συναγαγέτωσαν πάντα τὰ βρώματα τῶν ἑπτὰ ἐτῶν τῶν ἐρχομένων τῶν καλῶν τούτων· καὶ συναχθήτω ὁ σῖτος ὑπὸ χεῖρα Φαραώ· βρώματα ἐν ταῖς πόλεσι φυλαχθήτω.
\VS{36}Καὶ ἔσται τὰ βρώματα τὰ πεφυλαγμένα τῇ γῇ εἰς τὰ ἑπτὰ ἔτη τοῦ λιμοῦ, ἃ ἔσονται ἐν γῇ Αἰγύπτου, καὶ οὐκ ἐκτριβήσεται ἡ γῆ ἐν τῷ λιμῷ.
\VS{37}Ἤρεσε δὲ τὸ ῥῆμα ἐναντίον Φαραὼ, καὶ ἐναντίον πάντων τῶν παίδων αὐτοῦ.
\par }{\PP \VS{38}Καὶ εἶπε Φαραὼ πᾶσι τοῖς παισὶν αὐτοῦ, μῆ εὑρήσομεν ἄνθρωπον τοιοῦτον, ὃς ἔχει πνεῦμα Θεοῦ ἐν αὐτῷ;
\VS{39}Εἶπε δὲ Φαραὼ τῷ Ἰωσὴφ, ἐπειδὴ ἔδειξεν ὁ Θεός σοι πάντα ταῦτα, οὐκ ἔστιν ἄνθρωπος φρονιμώτερος καὶ συνετώτερός σου.
\VS{40}Σὺ ἔσῃ ἐπὶ τῷ οἴκῳ μου, καὶ ἐπὶ τῷ στόματί σου ὑπακούσεται πᾶς ὁ λαός μου· πλὴν τὸν θρόνον ὑπερέξω σου ἐγώ.
\VS{41}Εἶπε δὲ Φαραὼ τῷ Ἰωσὴφ, ἰδοὺ καθίστημί σε σήμερον ἐπὶ πάσῃ γῇ Αἰγύπτου.
\VS{42}Καὶ περιελόμενος Φαραὼ τὸν δακτύλιον ἀπὸ τῆς χειρὸς αὐτοῦ, περίεθηκεν αὐτὸν ἐπὶ τὴν χεῖρα Ἰωσὴφ, καὶ ἐνέδυσεν αὐτὸν στολὴν βυσσίνην, καὶ περιέθηκε κλοιὸν χρυσοῦν περὶ τὸν τράχηλον αὐτοῦ.
\VS{43}Καὶ ἀνεβίβασεν αὐτὸν ἐπὶ τὸ ἅρμα τὸ δεύτερον τῶν αὐτοῦ· καὶ ἐκήρυξεν ἔμπροσθεν αὐτοῦ κήρυξ· καὶ κατέστησεν αὐτὸν ἐφʼ ὅλης γῆς Αἰγύπτου.
\VS{44}Εἶπε δὲ Φαραὼ τῷ Ἰωσὴφ, ἐγὼ Φαραώ· ἄνευ σοῦ οὐκ ἐξαρεῖ οὐθεὶς τὴν χεῖρα αὐτοῦ ἐπὶ πάσης γῆς Αἰγύπτου.
\VS{45}Καὶ ἐκάλεσε Φαραὼ τὸ ὄνομα Ἰωσὴφ, Ψονθομφανήχ· καὶ ἔδωκεν αὐτῷ τὴν Ἀσενὲθ θυγατέρα Πετεφρῆ ἱερέως Ἡλιουπόλεως αὐτῷ εἰς γυναῖκα.
\VS{46}Ἰωσὴφ δὲ ἦν ἐτῶν τριάκοντα, ὅτε ἔστη ἐναντίον Φαραὼ βασιλέως Αἰγύπτου· ἐξῆλθε δὲ Ἰωσὴφ ἀπὸ προσώπου Φαραὼ, καὶ διῆλθε πᾶσαν γῆν Αἰγύπτου.
\VS{47}Καὶ ἐποίησεν ἡ γῆ ἐν τοῖς ἑπτὰ ἔτεσι τῆς εὐθηνίας δράγματα.
\VS{48}Καὶ συνήγαγε πάντα τὰ βρώματα τῶν ἑπτὰ ἐτῶν, ἐν οἷς ἦν ἡ εὐθηνία ἐν τῇ γῇ Αἰγύπτου· καὶ ἔθηκε τὰ βρώματα ἐν ταῖς πόλεσι· βρώματα τῶν πεδίων τῆς πόλεως τῶν κύκλῳ αὐτῆς ἔθηκεν ἐν αὐτῇ.
\VS{49}Καὶ συνήγαγεν Ἰωσὴφ σῖτον ὡσεὶ τὴν ἄμμον τῆς θαλάσσης πολὺν σφόδρα, ἕως οὐκ ἠδύνατο ἀριθμηθῆναι, οὐ γὰρ ἦν ἀριθμός.
\par }{\PP \VS{50}Τῷ δὲ Ἰωσὴφ ἐγένοντο υἱοὶ δύο πρὸ τοῦ ἐλθεῖν τὰ ἑπτὰ ἔτη τοῦ λιμοῦ, οὓς ἔτεκεν αὐτῷ Ἀσενὲθ ἡ θυγάτηρ Πετεφρῆ ἱερέως Ἡλιουπόλεως.
\VS{51}Ἐκάλεσε δὲ Ἰωσὴφ τὸ ὄνομα τοῦ πρωτοτόκου, Μανασσῆ· ὅτι ἐπιλαθέσθαι με ἐποίησεν ὁ Θεὸς πάντων τῶν πόνων μου, καὶ πάντων τῶν τοῦ πατρός μου·
\VS{52}Τὸ δὲ ὄνομα τοῦ δευτέρου ἐκάλεσεν, Ἐφραίμ· ὅτι ηὔξησέ με ὁ Θεὸς ἐν γῇ ταπεινώσεώς μου.
\VS{53}Παρῆλθον δὲ τὰ ἑπτὰ ἔτη τῆς εὐθηνίας, ἃ ἐγένοντο ἐν τῇ γῇ Αἰγύπτου.
\VS{54}Καὶ ἤρξατο τὰ ἑπτὰ ἔτη τοῦ λιμοῦ ἔρχεσθαι, καθὰ εἶπεν Ἰωσήφ· καὶ ἐγένετο λιμὸς ἐν πάσῃ τῇ γῇ· ἐν δὲ πάσῃ τῇ γῇ Αἰγύπτου ἦσαν ἄρτοι.
\VS{55}Καὶ ἐπείνασε πᾶσα ἡ γῆ Αἰγύπτου· ἔκραξε δὲ ὁ λαὸς πρὸς Φαραὼ περὶ ἄρτων· εἶπε δὲ Φαραὼ πᾶσι τοῖς Αἰγυπτίοις, πορεύεσθε πρὸς Ἰωσὴφ, καὶ ὃ ἐὰν εἴπῃ ὑμῖν, ποιήσατε.
\VS{56}Καὶ ὁ λιμὸς ἦν ἐπὶ προσώπου πάσης τῆς γῆς· ἀνέῳξε δὲ Ἰωσὴφ πάντας τοὺς σιτοβολῶνας, καὶ ἐπώλει πᾶσι τοῖς Αἰγυπτίοις.
\VS{57}Καὶ πᾶσαι αἱ χῶραι ἦλθον εἰς Αἴγυπτον, ἀγοράζειν πρὸς Ἰωσήφ· ἐπεκράτησε γὰρ ὁ λιμὸς ἐν πάσῃ τῇ γῇ·

\par }\Chap{42}{\PP \VerseOne{1}Ἰδὼν δὲ Ἰακὼβ ὅτι ἐστὶ πράσις ἐν Αἰγύπτῳ, εἶπε τοῖς υἱοῖς αὐτοῦ, ἱνατί ῥαθυμεῖτε;
\VS{2}Ἰδοὺ ἀκήκοα, ὅτι ἐστὶ σῖτος ἐν Αἰγύπτῳ· κατάβητε ἐκεὶ, καὶ πρίασθε ἡμῖν μικρὰ βρώματα, ἵνα ζήσωμεν καὶ μὴ ἀποθάνωμεν.
\par }{\PP \VS{3}Κατέβησαν δὲ οἱ ἀδελφοὶ Ἰωσὴφ οἱ δέκα, πρίασθαι σῖτον ἐξ Αἰγύπτου·
\VS{4}Τὸν δὲ Βενιαμὶν, τὸν ἀδελφὸν Ἰωσὴφ, οὐκ ἀπέστειλε μετὰ τῶν ἀδελφῶν αὐτοῦ· εἶπε γὰρ, μή ποτε συμβῇ αὐτῷ μαλακία.
\VS{5}Ἦλθον δὲ οἱ υἱοὶ Ἰσραὴλ ἀγοράζειν μετὰ τῶν ἐρχομένων· ἦν γὰρ ὁ λιμὸς ἐν γῇ Χαναάν.
\VS{6}Ἰωσὴφ δὲ ἦν ὁ ἄρχων τῆς γῆς· οὗτος ἐπώλει παντὶ τῷ λαῷ τῆς γῆς· ἐλθόντες δὲ οἱ ἀδελφοὶ Ἰωσὴφ προσεκύνησαν αὐτῷ ἐπὶ πρόσωπον ἐπὶ τὴν γῆν.
\VS{7}Ἰδὼν δὲ Ἰωσὴφ τοὺς ἀδελφοὺς αὐτοῦ, ἐπέγνω· καὶ ἠλλοτριοῦτο ἀπʼ αὐτῶν, καὶ ἐλάλησεν αὐτοῖς σκληρά· καὶ εἶπεν αὐτοῖς, πόθεν ἥκατε; οἱ δὲ εἶπον, ἐκ γῆς Χαναὰν, ἀγοράσαι βρώματα.
\VS{8}Ἐπέγνω δὲ Ἰωσὴφ τοὺς ἀδελφοὺς αὐτοῦ· αὐτοὶ δὲ οὐκ ἐπέγνωσαν αὐτόν·
\VS{9}Καὶ ἐμνήσθη Ἰωσὴφ τῶν ἐνυπνίων αὐτοῦ, ὧν εἶδεν αὐτός· καὶ εἶπεν αὐτοῖς, κατάσκοποί ἐστε, κατανοῆσαι τὰ ἴχνη τῆς χώρας ἥκατε.
\VS{10}Οἱ δὲ εἶπαν, οὐχὶ, κύριε· οἱ παῖδές σου ἤλθομεν πρίασθαι βρώματα.
\VS{11}Πάντες ἐσμὲν υἱοὶ ἑνὸς ἀνθρώπου· εἰρηνικοί ἐσμεν, οὐκ εἰσιν οἱ παῖδές σου κατάσκοποι.
\VS{12}Εἶπε δὲ αὐτοῖς, οὐχί· ἀλλὰ τὰ ἴχνη τῆς γῆς ἤλθετε ἰδεῖν.
\VS{13}Οἱ δὲ εἶπαν, δώδεκά ἐσμεν οἱ παῖδές σου ἀδελφοὶ ἐν γῇ Χαναάν· καὶ ἰδοὺ ὁ νεώτερος μετὰ τοῦ πατρὸς ἡμῶν σήμερον· ὁ δὲ ἕτερος οὐχ ὑπάρχει.
\VS{14}Εἶπε δὲ αὐτοῖς Ἰωσὴφ, τοῦτό ἐστιν ὃ εἴρηκα ὑμῖν, λέγων, ὅτι κατάσκοποί ἐστε.
\VS{15}Ἐν τούτῳ φανεῖσθε· νὴ τὴν ὑγίειαν Φαραὼ, οὐ μὴ ἐξέλθητε ἐντεῦθεν, ἐὰν μὴ ὁ ἀδελφὸς ὑμῶν ὁ νεώτερος ἔλθῃ ὧδε.
\VS{16}Ἀποστείλατε ἐξ ὑμῶν ἕνα, καὶ λάβετε τὸν ἀδελφὸν ὑμῶν· ὑμεῖς δὲ ἀπάχθητε ἕως τοῦ φανερὰ γενέσθαι τὰ ῥήματα ὑμῶν, εἰ ἀληθεύετε ἢ οὔ· εἰ δὲ μὴ, νὴ τὴν ὑγίειαν Φαραὼ, ἦ μὴν κατάσκοποί ἐστε.
\VS{17}Καὶ ἔθετο αὐτοὺς ἐν φυλακῇ ἡμέρας τρεῖς.
\VS{18}Εἶπε δὲ αὐτοῖς τῇ ἡμέρᾳ τῇ τρίτῃ, τοῦτο ποιήσατε, καὶ ζήσεσθε· τὸν Θεὸν γὰρ ἐγὼ φοβοῦμαι.
\VS{19}Εἰ εἰρηνικοί ἐστε, ἀδελφὸς ὑμῶν κατασχεθήτω εἷς ἐν τῇ φυλακῇ· αὐτοὶ δὲ βαδίσατε, καὶ ἀπαγάγετε τὸν ἀγορασμὸν τῆς σιτοδοσίας ὑμῶν.
\VS{20}Καὶ τὸν ἀδελφὸν ὑμῶν τὸν νεώτερον ἀγάγετε πρός με, καὶ πιστευθήσονται τὰ ῥήματα ὑμῶν· εἰ δὲ μὴ, ἀποθανεῖσθε. Ἐποίησαν δὲ οὕτως.
\VS{21}Καὶ εἶπεν ἕκαστος πρὸς τὸν ἀδελφὸν αὐτοῦ, ναὶ, ἐν ἁμαρτίαις γάρ ἐσμεν περὶ τοῦ ἀδελφοῦ ἡμῶν, ὅτι ὑπερίδομεν τὴν θλίψιν τῆς ψυχῆς αὐτοῦ, ὅτε κατεδέετο ἡμῶν, καὶ οὐκ εἰσηκούσαμεν αὐτοῦ· καὶ ἕνεκεν τούτου ἐπῆλθεν ἐφʼ ἡμᾶς ἡ θλίψις αὕτη.
\VS{22}Ἀποκριθεὶς δὲ Ῥουβὴν εἶπεν αὐτοῖς, οὐκ ἐλάλησα ὑμῖν, λέγων, μὴ ἀδικήσητε τὸ παιδάριον, καὶ οὐκ εἰσηκούσατέ μου; καὶ ἰδοὺ τὸ αἷμα αὐτοῦ ἐκζητεῖται.
\VS{23}Αὐτοὶ δὲ οὐκ ᾔδεισαν, ὅτι ἀκούει Ἰωσήφ· ὁ γὰρ ἑρμηνευτὴς ἀνὰ μέσον αὐτῶν ἦν·
\VS{24}Ἀποστραφεὶς δὲ ἀπʼ αὐτῶν ἔκλαυσεν Ἰωσήφ· καὶ πάλιν προσῆλθε πρὸς αὐτοὺς, καὶ εἶπεν αὐτοῖς· καὶ ἔλαβε τὸν Συμεὼν ἀπʼ αὐτῶν, καὶ ἔδησεν αὐτὸν ἐναντίον αὐτῶν.
\par }{\PP \VS{25}Ἐνετείλατο δὲ Ἰωσὴφ ἐμπλῆσαι τὰ ἀγγεῖα αὐτῶν σίτου, καὶ ἀποδοῦναι τὸ ἀργύριον αὐτῶν ἑκάστῳ εἰς τὸν σάκκον αὐτοῦ, καὶ δοῦναι αὐτοῖς ἐπισιτισμὸν εἰς τὴν ὁδόν· καὶ ἐγενήθη αὐτοῖς οὕτως.
\VS{26}Καὶ ἐπιθέντες τὸν σῖτον ἐπὶ τοῦς ὄνους αὐτῶν, ἀπῆλθον ἐκεῖθεν.
\VS{27}Λύσας δὲ εἷς τὸν μάρσιππον αὐτοῦ, δοῦναι χορτάσματα τοῖς ὄνοις αὐτοῦ, οὗ κατέλυσαν, καὶ εἶδε τὸν δεσμὸν τοῦ ἀργυρίου αὐτοῦ, καὶ ἦν ἐπάνω τοῦ στόματος τοῦ μαρσίππου.
\VS{28}Καὶ εἶπε τοῖς ἀδελφοῖς αὐτοῦ, ἀπεδόθη μοι τὸ ἀργύριον, καὶ ἰδοὺ τοῦτο ἐν τῷ μαρσίππῳ μου· καὶ ἐξέστη ἡ καρδία αὐτῶν, καὶ ἐταράχθησαν πρὸς ἀλλήλους, λέγοντες, τί τοῦτο ἐποίησεν ὁ Θεὸς ἡμῖν;
\VS{29}Ἦλθον δὲ πρὸς Ἰακὼβ τὸν πατέρα αὐτῶν εἰς γὴν Χαναὰν, καὶ ἀπήγγειλαν αὐτῷ πάντα τὰ συμβάντα αὐτοῖς, λέγοντες,
\VS{30}Λελάληκεν ὁ ἄνθρωπος ὁ κύριος τῆς γῆς πρὸς ἡμᾶς σκληρὰ, καὶ ἔθετο ἡμᾶς ἐν φυλακῇ, ὡς κατασκοπεύοντας τὴν γῆν.
\VS{31}Εἴπαμεν δὲ αὐτῷ, εἰρήνικοί ἐσμεν, οὐκ ἐσμὲν κατάσκοποι.
\VS{32}Δώδεκα ἀδελφοί ἐσμεν, υἱοὶ τοῦ πατρὸς ἡμῶν· ὁ εἷς οὐχ ὑπάρχει· ὁ δὲ μικρὸς μετὰ τοῦ πατρὸς ἡμῶν σήμερον ἐν γῇ Χαναάν.
\VS{33}Εἶπε δὲ ἡμῖν ὁ ἄνθρωπος ὁ κύριος τῆς γῆς, ἐν τούτῳ γνώσομαι, ὅτι εἰρηνικοί ἐστε· ἀδελφὸν ἕνα ἄφετε ὧδε μετʼ ἐμοῦ· τὸν δὲ ἀγορασμὸν τῆς σιτοδοσίας τοῦ οἴκου ὑμῶν λαβόντες ἀπέλθατε.
\VS{34}Καὶ ἀγάγετε πρός με τὸν ἀδελφὸν ὑμῶν τὸν νεώτερον· καὶ γνώσομαι ὅτι οὐ κατάσκοποί ἐστε, ἀλλʼ ὅτι εἰρηνικοί ἐστε· καὶ τὸν ἀδελφὸν ὑμῶν ἀποδώσω ὑμῖν, καὶ τῇ γῇ ἐμπορεύσεσθε.
\VS{35}Ἐγένετο δὲ ἐν τῷ κατακενοῦν αὐτοὺς τοὺς σάκκους αὐτῶν, καὶ ἦν ἑκάστου ὁ δεσμὸς τοῦ ἀργυρίου ἐν τῷ σάκκῳ αὐτῶν· καὶ εἶδον τοὺς δεσμοὺς τοῦ ἀργυρίου αὐτῶν αὐτοὶ, καὶ ὁ πατὴρ αὐτῶν, καὶ ἐφοβήθησαν.
\VS{36}Εἶπε δὲ αὐτοῖς Ἰακὼβ ὁ πατὴρ αὐτῶν, ἐμὲ ἠτεκνώσατε· Ἰωσὴφ οὐκ ἔστι, Συμεὼν οὐκ ἔστι, καὶ τὸν Βενιαμὶν λήψεσθε; ἐπʼ ἐμὲ ἐγένετο ταῦτα πάντα.
\VS{37}Εἶπε δὲ Ῥουβὴν τῷ πατρὶ αὐτῶν, λέγων, τοὺς δύο υἱούς μου ἀπόκτεινον, ἐὰν μὴ ἀγάγω αὐτὸν πρὸς σέ· δὸς αὐτὸν εἰς τὴν χεῖρά μου, κᾀγὼ ἀνάξω αὐτὸν πρὸς σέ.
\VS{38}Ὁ δὲ εἶπεν, οὐ καταβήσεται ὁ υἱός μου μεθʼ ὑμῶν, ὅτι ὁ ἀδελφὸς αὐτοῦ ἀπέθανε, καὶ αὐτὸς μόνος καταλέλειπται· καὶ συμβήσεται αὐτὸν μαλακισθῆναι ἐν τῇ ὁδῷ, ᾗ ἐὰν πορεύησθε, καὶ κατάξετέ μου τὸ γῆρας μετὰ λύπης εἰς ᾅδοῦ.

\par }\Chap{43}{\PP \VerseOne{1}Ὁ δὲ λιμὸς ἐνίσχυσεν ἐπὶ τῆς γῆς.
\VS{2}Ἐγένετο δὲ ἡνίκα συνετέλεσαν καταφαγεῖν τὸν σῖτον, ὃν ἤνεγκαν ἐξ Αἰγύπτου, καὶ εἶπεν αὐτοῖς ὁ πατὴρ αὐτῶν, πάλιν πορευθέντες πρίασθε ἡμῖν μικρὰ βρώματα.
\VS{3}Εἶπε δὲ αὐτῷ Ἰούδας, λέγων, διαμαρτυρίᾳ μεμαρτύρηται ἡμῖν ὁ ἄνθρωπος ὁ κύριος τῆς γῆς, λέγων, οὐκ ὄψεσθε τὸ πρόσωπόν μου, ἐὰν μὴ ὁ ἀδελφὸς ὑμῶν ὁ νεώτερος μεθʼ ὑμῶν ᾖ.
\VS{4}Εἰ μὲν οὖν ἀποστέλλῃς τὸν ἀδελφὸν ἡμῶν μεθʼ ἡμῶν, καταβησόμεθα, καὶ ἀγοράσομέν σοι βρώματα.
\VS{5}Εἰ δὲ μὴ ἀποστέλλῃς τὸν ἀδελφὸν ἡμῶν μεθʼ ἡμῶν, οὐ πορευσόμεθα· ὁ γὰρ ἄνθρωπος εἶπεν ἡμῖν, λέγων, οὐκ ὄψεσθέ μου τὸ πρόσωπον, ἐὰν μὴ ὁ ἀδελφὸς ὑμῶν ὁ νεώτερος μεθʼ ὑμῶν ᾖ.
\VS{6}Εἶπε δὲ Ἰσραὴλ, τί ἐκακοποιήσατέ με, ἀναγγείλαντες τῷ ἀνθρώπῳ ὅτι ἐστὶν ὑμῖν ἀδελφός;
\VS{7}Οἱ δὲ εἶπαν, ἐρωτῶν ἐπηρώτησεν ἡμᾶς ὁ ἄνθρωπος καὶ τὴν γενεὰν ἡμῶν, λέγων, εἰ ἔτι ὁ πατὴρ ὑμῶν ζῇ, καὶ εἰ ἔστιν ὑμῖν ἀδελφός· καὶ ἀπηγγείλαμεν αὐτῷ κατὰ τὴν ἐπερώτησιν ταύτην· μὴ ᾔδειμεν ὅτι ἐρεῖ ἡμῖν, ἀγάγετε τὸν ἀδελφὸν ὑμῶν;
\VS{8}Εἶπε δὲ Ἰούδας πρὸς Ἰσραὴλ τὸν πατέρα αὐτοῦ, ἀπόστειλον τὸ παιδάριον μετʼ ἐμοῦ· καὶ ἀναστάντες πορευσόμεθα, ἵνα ζῶμεν καὶ μὴ ἀποθάνωμεν καὶ ἡμεῖς, καὶ σὺ, καὶ ἡ ἀποσκευὴ ἡμῶν.
\VS{9}Ἐγὼ δὲ ἐκδέχομαι αὐτόν· ἐκ χειρός μου ζήτησον αὐτόν· ἐὰν μὴ ἀγάγω αὐτὸν πρός σε, καὶ στήσω αὐτὸν ἐναντίον σου, ἡμαρτηκὼς ἔσομαι εἰς σὲ πάσας τὰς ἡμέραν.
\VS{10}Εἰ μὴ γὰρ ἐβραδύναμεν, ἤδη ἂν ὑπεστρέψαμεν δίς.
\VS{11}Εἶπε δὲ αὐτοῖς Ἰσραὴλ ὁ πατὴρ αὐτῶν, εἰ οὕτως ἐστὶ, τοῦτο ποιήσατε· λάβετε ἀπὸ τῶν καρπῶν τῆς γῆς ἐν τοῖς ἀγγείοις ὑμῶν, καὶ καταγάγετε τῷ ἀνθρώπῳ δῶρα τῆς ῥητίνης, καὶ τοῦ μέλιτος, θυμίαμά τε καὶ στακτὴν, καὶ τερέινθον, καὶ κάρυα.
\VS{12}Καὶ τὸ ἀργύριον δισσὸν λάβετε ἐν ταῖς χερσὶν ὑμῶν· τὸ ἀργύριον τὸ ἀποστραφὲν ἐν τοῖς μαρσίπποις ὑμῶν ἀποστρέψατε μεθʼ ὑμῶν· μή ποτε ἀγνόημά ἐστι.
\VS{13}Καὶ τὸν ἀδελφὸν ὑμῶν λάβετε· καὶ ἀναστάντες κατάβητε πρὸς τὸν ἄνθρωπον.
\VS{14}Ὁ δὲ Θεός μου δώῃ ὑμῖν χάριν ἐναντίον τοῦ ἀνθρώπου καὶ ἀποστείλαι τὸν ἀδελφὸν ὑμῶν τὸν ἕνα, καὶ τὸν Βενιαμίν· ἐγὼ μὲν γὰρ καθάπερ ἠτέκνωμαι, ἠτέκνωμαι.
\par }{\PP \VS{15}Λαβόντες δὲ οἱ ἄνδρες τὰ δῶρα ταῦτα καὶ τὸ ἀργύριον διπλοῦν, ἔλαβον ἐν ταῖς χερσὶν αὐτῶν καὶ τὸν Βενιαμείν· καὶ ἀναστάντες κατέβησαν εἰς Αἴγυπτον· καὶ ἔστησαν ἐναντίον Ἰωσήφ.
\VS{16}Εἶδε δὲ Ἰωσὴφ αὐτοὺς, καὶ τὸν Βενιαμὶν τὸν ἀδελφὸν αὐτοῦ τὸν ὁμομήτριον· καὶ εἶπε τῷ ἐπὶ τῆς οἰκίας αὐτοῦ, εἰσάγαγε τοὺς ἀνθρώπους εἰς τὴν οἰκίαν, καὶ σφάξον θύματα, καὶ ἑτοίμασον· μετʼ ἐμοῦ γὰρ φάγονται οἱ ἄνθρωποι ἄρτους τὴν μεσημβρίαν.
\VS{17}Ἐποίησε δὲ ὁ ἄνθρωπος καθὰ εἶπεν Ἰωσήφ· καὶ εἰσήγαγε τοὺς ἀνθρώπους εἰς τὸν οἶκον Ἰωσήφ.
\VS{18}Ἰδόντες δὲ οἱ ἄνδρες ὅτι εἰσήχθησαν εἰς τὸν οἶκον τοῦ Ἰωσήφ, εἶπαν, διὰ τὸ ἀργύριον τὸ ἀποστραφὲν ἐν τοῖς μαρσίπποις ἡμῶν τὴν ἀρχὴν, ἡμεῖς εἰσαγόμεθα, τοῦ συκοφαντῆσαι ἡμᾶς καὶ ἐπιθέσθαι ἡμῖν, τοῦ λαβεῖν ἡμᾶς εἰς παῖδας, καὶ τοὺς ὄνους ἡμῶν.
\VS{19}Προσελθόντες δὲ πρὸς τὸν ἄνθρωπον τὸν ἐπὶ τοῦ οἴκου τοῦ Ἰωσὴφ, ἐλάλησαν αὐτῷ ἐν τῷ πυλῶνι τοῦ οἴκου,
\VS{20}λέγοντες, δεόμεθα, κύριε· κατέβηεν τὴν ἀρχὴν πρίασθαι βρώματα.
\VS{21}Ἐγένετο δὲ ἡνίκα ἤλθομεν εἰς τὸ καταλῦσαι, καὶ ἠνοίξαμεν τοὺς μαρσίππους ἡμῶν, καὶ τόδε τὸ ἀργύριον ἑκάστου ἐν τῷ μαρσίππῳ αὐτοῦ· τὸ ἀργύριον ἡμῶν ἐν σταθμῷ ἀπεστρέψαμεν νῦν ἐν ταῖς χερσὶν ἡμῶν.
\VS{22}Καὶ ἀργύριον ἕτερον ἠνέγκαμεν μεθʼ ἑαυτῶν, ἀγοράσαι βρώματα· οὐκ οἴδαμεν τίς ἐνέβαλεν τὸ ἀργύριον εἰς τοὺς μαρσίππους ἡμῶν.
\VS{23}Εἶπε δὲ αὐτοῖς, ἵλεως ὑμῖν, μὴ φοβεῖσθε· ὁ Θεὸς ὑμῶν, καὶ ὁ Θεὸς τῶν πατέρων ὑμῶν, ἔδωκεν ὑμῖν θησαυροὺς ἐν τοῖς μαρσίπποις ὑμῶν· καὶ τὸ ἀργύριον ὑμῶν εὐδοκιμοῦν ἀπέχω· καὶ ἐξήγαγε πρὸς αὐτοὺς τὸν Συμεών.
\VS{24}Καὶ ἤνεγκεν ὕδωρ νίψαι τοὺς πόδας αὐτῶν· καὶ ἔδωκε χορτάσματα τοῖς ὄνοις αὐτῶν.
\VS{25}Ἡτοίμασαν δὲ τὰ δῶρα, ἕως τοῦ ἐλθεῖν τὸν Ἰωσὴφ μεσημβρίας· ἤκουσαν γὰρ ὅτι ἐκεῖ μέλλει ἀριστᾷν.
\VS{26}Εἰσῆλθε δὲ Ἰωσὴφ εἰς τὴν οἰκίαν, καὶ προσήνεγκαν αὐτῷ τὰ δῶρα, ἃ εἶχον ἐν ταῖς χερσὶν αὐτῶν, εἰς τὸν οἶκον· καὶ προσεκύνησαν αὐτῷ ἐπὶ πρόσωπον ἐπὶ τὴν γῆν.
\VS{27}Ἠρώτησε δὲ αὐτοὺς, πῶς ἔχετε; καὶ εἶπεν αὐτοῖς, εἰ ὑγιαίνει ὁ πατὴρ ὑμῶν ὁ πρεσβύτης, ὃν εἴπατε; ἔτι ζῇ;
\VS{28}Οἱ δὲ εἶπαν, ὑγιαίνει ὁ παῖς σου ὁ πατὴρ ἡμῶν, ἔτι ζῇ. Καὶ εἶπεν, εὐλογημένος ὁ ἄνθρωπος ἐκεῖνος τῷ Θεῷ· καὶ κύψαντες προσεκύνησαν αὐτῷ.
\VS{29}Ἀναβλέψας δὲ τοῖς ὀφθαλμοῖς αὐτοῦ Ἰωσὴφ, εἶδε Βενιαμὶν τὸν ἀδελφὸν αὐτοῦ τὸν ὁμομήτριον· καὶ εἶπεν, οὗτος ὁ ἀδελφὸς ὑμῶν ὁ νεώτερος, ὃν εἴπατε πρός με ἀγαγεῖν; καὶ εἶπεν, ὁ Θεὸς ἐλεήσαι σε, τέκνον.
\VS{30}Ἐταράχθη δὲ Ἰωσήφ· συνεστρέφετο γὰρ τὰ ἔγκατα αὐτοῦ ἐπὶ τῷ ἀδελφῷ αὐτοῦ, καὶ ἐζήτει κλαῦσαι· εἰσελθὼν δὲ εἰς τὸ ταμεῖον, ἔκλαυσεν ἐκεῖ.
\par }{\PP \VS{31}Καὶ νιψάμενος τὸ πρόσωπον, ἐξελθὼν ἐνεκρατεύσατο· καὶ εἶπε, παράθετε ἄρτους.
\VS{32}Καὶ παρέθηκαν αὐτῷ μόνῳ, καὶ αὐτοῖς καθʼ ἑαυτούς, καὶ τοῖς Αἰγυπτίοις τοῖς συνδειπνοῦσι μετʼ αὐτοῦ καθʼ ἑαυτούς· οὐ γὰρ ἐδύναντο οἱ Αἰγύπτιοι συνεσθίειν μετὰ τῶν Ἐβραίων ἄρτους· βδέλυγμα γάρ ἐστι τοῖς Αἰγυπτίοις.
\VS{33}Ἐκάθισαν δὲ ἐναντίον αὐτοῦ, ὁ πρωτότοκος κατὰ τὰ πρεσβεῖα αὐτοῦ, καὶ ὁ νεώτερος κατὰ τὴν νεότητα αὐτοῦ· ἐξίσταντο δὲ οἱ ἄνθρωποι ἕκαστος πρὸς τὸν ἀδελφὸν αὐτοῦ.
\VS{34}Ἦραν δὲ μερίδα παρʼ αὐτοῦ πρὸς ἑαυτούς· ἐμεγαλύνθη δὲ ἡ μερὶς Βενιαμεὶν παρὰ τὰς μερίδας πάντων πενταπλασίως πρὸς τὰς ἐκείνων· ἔπιον δὲ καὶ ἐμεθύσθησαν μετʼ αὐτοῦ.

\Chap{44}\VerseOne{1}Καὶ ἐνετείλατο ὁ Ἰωσὴφ τῷ ὄντι ἐπὶ τῆς οἰκίας αὐτοῦ, λέγων, πλήσατε τοὺς μαρσίππους τῶν ἀνθρώπων βρωμάτων, ὅσα ἐὰν δύνωνται ἆραι· καὶ ἐμβάλετε ἑκάστου τὸ ἀργύριον ἐπὶ τοῦ στόματος τοῦ μαρσίππου.
\VS{2}Καὶ τὸ κόνδυ μου τὸ ἀργυροῦν ἐμβάλετε εἰς τὸν μάρσιππον τοῦ νεωτέρου, καὶ τὴν τιμὴν τοῦ σίτου αὐτοῦ· ἐγενήθη δὲ κατὰ τὸ ῥῆμα Ἰωσὴφ, καθὼς εἶπε.
\par }{\PP \VS{3}Τὸ πρωῒ διέφαυσε· καὶ οἱ ἄνθρωποι ἀπεστάλησαν, αὐτοὶ καὶ οἱ ὄνοι αὐτῶν.
\VS{4}Ἐξελθόντων δὲ αὐτῶν τὴν πόλιν, οὐκ ἀπέσχον μακράν· καὶ Ἰωσὴφ εἶπε τῷ ἐπὶ τῆς οἰκίας αὐτοῦ, ἀναστὰς ἐπιδίωξον ὀπίσω τῶν ἀνθρώπων, καὶ καταλήμψῃ αὐτοὺς, καὶ ἐρεῖς αὐτοῖς τί ὅτι ἀνταπεδώκατε πονηρὰ ἀντὶ καλῶν;
\VS{5}Ἱνατί ἐκλέψατέ μου τὸ κόνδυ τὸ ἀργυροῦν; οὐ τοῦτό ἐστιν, ἐν ᾧ πίνει ὁ κύριός μου; αὐτὸς δὲ οἰωνισμῷ οἰωνίζεται ἐν αὐτῷ. πονηρὰ συντετελέκατε ἃ πεποιήκατε.
\VS{6}Εὑρὼν δὲ αὐτοὺς, εἶπεν αὐτοῖς κατὰ τὰ ῥήματα ταῦτα.
\VS{7}Οἱ δὲ εἶπαν αὐτῷ, ἱνατί λαλεῖ ὁ κύριος κατὰ τὰ ῥήματα ταῦτα; μὴ γένοιτο τοῖς παισίν σου ποιῆσαι κατὰ τὸ ῥῆμα τοῦτο.
\VS{8}Εἰ τὸ μὲν ἀργύριον, ὃ εὕρομεν ἐν τοῖς μαρσίπποις ἡμῶν, ἀπεστρέψαμεν πρὸς σὲ ἐκ γῆς Χαναὰν, πῶς ἂν κλέψαιμεν ἐκ τοῦ οἴκου τοῦ κυρίου σου ἀργύριον ἢ χρυσίον;
\VS{9}Παρʼ ᾧ ἂν εὕρῃς τὸ κόνδυ τῶν παιδων σου, ἀποθνησκέτω· καὶ ἡμεῖς δὲ ἐσόμεθα παῖδες τῷ κυρίῳ ἡμῶν.
\VS{10}Ὁ δὲ εἶπε, καὶ νῦν, ὡς λέγετε, οὕτως ἔσται· παρʼ ᾧ ἂν εὑρεθῇ τὸ κόνδυ, ἔσται μου παῖς ὑμεῖς δὲ ἔσεσθε καθαροί.
\VS{11}Καὶ ἔσπευσαν, καὶ καθεῖλαν ἕκαστος τὸν μάρσιππον αὐτοῦ ἐπὶ τὴν γῆν· καὶ ἤνοιξαν ἕκαστος τὸν μάρσιππον αὐτοῦ.
\VS{12}Ἠρεύνησε δὲ ἀπὸ τοῦ πρεσβυτέρου ἀρξάμενος, ἕως ἦλθεν ἐπὶ τὸν νεώτερον. καὶ εὗρε τὸ κόνδυ ἐν τῷ μαρσίππῳ τοῦ Βενιαμίν.
\VS{13}Καὶ διέῤῥηξαν τὰ ἱμάτια αὐτῶν, καὶ ἐπέθηκαν ἕκαστος τὸν μαρσίππον αὐτοῦ ἐπὶ τὸν ὄνον αὐτοῦ, καὶ ἐπέστρεψαν εἰς τὴν πόλιν.
\par }{\PP \VS{14}Εἰσῆλθε δὲ Ἰούδας καὶ οἱ ἀδελφοὶ αὐτοῦ πρὸς Ἰωσὴφ ἔτι αὐτοῦ ὄντος ἐκεῖ, καὶ ἔπεσον ἐναντίον αὐτοῦ ἐπὶ τὴν γῆν.
\VS{15}Εἶπε δὲ αὐτοῖς Ἰωσὴφ, τί τὸ πρᾶγμα τοῦτο ἐποιήσατε; οὐκ οἴδατε ὅτι οἰωνισμῷ οἰωνιεῖται ὁ ἄνθρωπος, οἷος ἐγώ;
\VS{16}Εἶπε δὲ Ἰούδας, τί ἀντεροῦμεν τῷ κυρίῳ, ἢ τί λαλήσομεν, ἢ τί δικαιωθῶμεν; ὁ Θεὸς δὲ εὗρε τὴν ἀδικίαν τῶν παίδων σου· ἰδού ἐσμεν οἰκέται τῷ κυρίῳ ἡμῶν, καὶ ἡμεῖς, καὶ παρʼ ᾧ εὑρέθη τὸ κόνδυ.
\VS{17}Εἶπε δὲ Ἰωσὴφ, μή μοι γένοιτο ποιῆσαι τὸ ῥῆμα τοῦτο· ὁ ἄνθρωπος παρʼ ᾧ εὑρέθη τὸ κόνδυ, αὐτὸς ἔσται μου παῖς· ὑμεῖς δὲ ἀνάβητε μετὰ σωτηρίας πρὸς τὸν πατέρα ὑμῶν.
\VS{18}Ἐγγίσας δὲ αὐτῷ Ἰούδας εἶπε, δέομαι, κύριε· λαλησάτω ὁ παῖς σου ῥῆμα ἐναντίον σου, καὶ μὴ θυμωθῇς τῷ παιδί σου, ὅτι σὺ εἶ μετὰ Φαραώ.
\VS{19}Κύριε, σὺ ἠρώτησας τοὺς παῖδάς σου, λέγων, εἰ ἔχετε πατέρα ἢ ἀδελφόν.
\VS{20}Καὶ εἴπαμεν τῷ κυρίῳ, ἔστιν ἡμῖν πατὴρ πρεσβύτερος, καὶ παιδίον γήρως νεώτερον αὐτῷ, καὶ ὁ ἀδελφὸς αὐτοῦ ἀπέθανεν, αὐτὸς δὲ μόνος ὑπελείφθη τῇ μητρὶ αὐτοῦ, ὁ δὲ πατὴρ αὐτὸν ἠγάπησεν·
\VS{21}Εἶπας δὲ τοῖς παισί σου, καταγάγετε αὐτὸν πρὸς μέ, καὶ ἐπιμελοῦμαι αὐτοῦ.
\VS{22}Καὶ εἴπαμεν τῷ κυρίῳ, οὐ δυνήσεται τὸ παιδίον καταλιπεῖν τὸν πατέρα αὐτοῦ· ἐὰν δὲ καταλείπῃ τὸν πατέρα, ἀποθανεῖται.
\VS{23}Σὺ δὲ εἶπας τοῖς παισί σου, ἐὰν μὴ καταβῇ ὁ ἀδελφὸς ὑμῶν ὁ νεώτερος μεθʼ ὑμῶν, οὐ προσθήσεσθε ἰδεῖν τὸ πρόσωπόν μου.
\VS{24}Ἐγένετο δὲ ἡνίκα ἀνέβημεν πρὸς τὸν παῖδά σου πατέρα ἡμῶν, ἀπηγγείλαμεν αὐτῷ τὰ ῥήματα τοῦ κυρίου ἡμῶν.
\VS{25}Εἶπε δὲ ὁ πατὴρ ἡμῶν, βαδίσατε πάλιν καὶ ἀγοράσατε ἡμῖν μικρὰ βρώματα.
\VS{26}Ἡμεῖς δὲ εἴπομεν, οὐ δυνησόμεθα καταβῆναι· ἀλλʼ εἰ μὲν ὁ ἀδελφὸς ἡμῶν ὁ νεώτερος καταβαίνει μεθʼ ἡμῶν, καταβησόμεθα· οὐ γὰρ δυνησόμεθα ἰδεῖν τὸ πρόσωπον τοῦ ἀνθρώπου, τοῦ ἀδελφοῦ ἡμῶν τοῦ νεωτέρου μὴ ὄντος μεθʼ ἡμῶν.
\VS{27}Εἶπε δὲ ὁ παῖς σου πατὴρ ἡμῶν πρὸς ἡμᾶς, ὑμεῖς γινώσκετε ὅτι δύο ἔτεκέ μοι ἡ γυνὴ,
\VS{28}καὶ ἐξῆλθεν ὁ εἷς ἀπʼ ἐμοῦ· καὶ εἴπατε ὅτι θηριόβρωτος γέγονεν, καὶ οὐκ ἴδον αὐτὸν ἄχρι νῦν.
\VS{29}Ἐὰν οὖν λάβητε καὶ τοῦτον ἐκ τοῦ προσώπου μου, καὶ συμβῇ αὐτῷ μαλακία ἐν τῇ ὁδῷ, καὶ κατάξετέ μου τὸ γῆρας μετὰ λύπης εἰς ᾅδου.
\VS{30}Νῦν οὖν ἐὰν εἰσπορεύωμαι πρὸς τὸν παῖδά σου, πατέρα δὲ ἡμῶν, καὶ τὸ παιδίον μὴ ᾖ μεθʼ ἡμῶν, ἡ δὲ ψυχὴ αὐτοῦ ἐκκρέμαται ἐκ τῆς τούτου ψυχῆς,
\VS{31}καὶ ἔσται ἐν τῷ ἰδεῖν αὐτὸν μὴ ὂν τὸ παιδίον μεθʼ ἡμῶν, τελευτήσει, καὶ κατάξουσιν οἱ παῖδές σου τὸ γῆρας τοῦ παιδός σου, πατρὸς δὲ ἡμῶν, μετὰ λύπης εἰς ᾅδου.
\VS{32}Ὁ γὰρ παῖς σου παρὰ τοῦ πατρὸς ἐκδέδεκται τὸ παιδίον, λέγων, ἐὰν μὴ ἀγάγω αὐτὸν πρὸς σὲ, καὶ στήσω αὐτὸν ἐνώπιόν σου, ἡμαρτηκὼς ἔσομαι εἰς τὸν πατέρα πάσας τὰς ἡμέρας.
\VS{33}Νῦν οὖν παραμενῶ σοι παῖς ἀντὶ τοῦ παιδίου, οἰκέτης τοῦ κυρίου· τὸ δὲ παιδίον ἀναβήτω μετὰ τῶν ἀδελφῶν αὐτοῦ.
\VS{34}Πῶς γὰρ ἀναβήσομαι πρὸς τὸν πατέρα, τοῦ παιδίου μὴ ὄντος μεθʼ ἡμῶν; ἵνα μὴ ἴδω τὰ κακὰ, ἃ εὑρήσει τὸν πατέρα μου.

\par }\Chap{45}{\PP \VerseOne{1}Καὶ οὐκ ἠδύνατο Ἰωσὴφ ἀνέχεσθαι πάντων τῶν παρεστηκότων αὐτῷ, ἀλλʼ εἶπεν, ἐξαποστείλατε πάντας ἀπʼ ἐμοῦ· καὶ οὐ παρειστήκει οὐδεὶς τῷ Ἰωσὴφ, ἡνίκα ἀνεγνωρίζετο τοῖς ἀδελφοῖς αὐτοῦ.
\VS{2}Καὶ ἀφῆκε φωνὴν μετὰ κλαυθμοῦ· ἤκουσαν δὲ πάντες οἱ Αἰγύπτιοι, καὶ ἀκουστὸν ἐγένετο εἰς τὸν οἶκον Φαραώ.
\VS{3}Εἶπε δὲ Ἰωσὴφ πρὸς τοὺς ἀδελφοὺς αὐτοῦ, ἐγώ εἰμι Ἰωσήφ· ἔτι ὁ πατήρ μου ζῇ; καὶ οὐκ ἠδύναντο οἱ ἀδελφοὶ ἀποκριθῆναι αὐτῷ· ἐταράχθησαν γάρ.
\VS{4}Εἶπε δὲ Ἰωσὴφ πρὸς τοὺς ἀδελφοὺς αὐτοῦ, ἐγγίσατε πρὸς μέ· καὶ ἤγγισαν· καὶ εἶπεν, ἐγώ εἰμι Ἰωσὴφ ὁ ἀδελφὸς ὑμῶν, ὃν ἀπέδοσθε εἰς Αἴγυπτον.
\VS{5}Νῦν οὖν μὴ λυπεῖσθε, μηδὲ σκληρὸν ὑμῖν φανήτω, ὅτι ἀπέδοσθέ με ὧδε· εἰς γὰρ ζωὴν ἀπέστειλέ με ὁ Θεὸς ἔμπροσθεν ὑμῶν.
\VS{6}Τοῦτο γὰρ δεύτερον ἔτος λιμὸς ἐπὶ τῆς γῆς, καὶ ἔτι λοιπὰ πέντε ἔτη, ἐν οἷς οὐκ ἔσται ἀροτρίασις, οὐδὲ ἀμητός·
\VS{7}Ἀπέστειλε γάρ με ὁ Θεὸς ἔμπροσθεν ὑμῶν, ὑπολείπεσθαι ὑμῶν κατάλειμμα ἐπὶ τῆς γῆς, καὶ ἐκθρέψαι ὑμῶν κατάλειψιν μεγάλην.
\VS{8}Νῦν οὖν οὐχ ὑμεῖς με ἀπεστάλκατε ὧδε, ἀλλὰ ὁ Θεός· καὶ ἐποίησέ με ὡς πατέρα Φαραὼ, καὶ κύριον παντὸς τοῦ οἴκου αὐτοῦ, καὶ ἄρχοντα πάσης γῆς Αἰγύπτου.
\VS{9}Σπεύσαντες οὖν ἀνάβητε πρὸς τὸν πατέρα μου, καὶ εἴπατε αὐτῷ, τάδε λέγει ὁ υἱός σου Ἰωσήφ· ἐποίησέ με ὁ Θεὸς κύριον πάσης γῆς Αἰγύπτου· κατάβηθι οὖν πρός με, καὶ μὴ μείνῃς·
\VS{10}Καὶ κατοικήσεις ἐν γῇ Γεσὲμ Ἀραβίας· καὶ ἔσῃ ἐγγύς μου σὺ, καὶ οἱ υἱοί σου, καὶ οἱ υἱοὶ τῶν υἱῶν σου, τὰ πρόβατά σου, καὶ αἱ βόες σου, καὶ ὅσα σοι ἐστί.
\VS{11}Καὶ ἐκθρέψω σε ἐκεῖ· ἔτι γὰρ πέντε ἔτη λιμός· ἵνα μὴ ἐκτριβῇς σὺ, καὶ οἱ υἱοί σου, καὶ πάντα τὰ ὑπάρχοντά σου.
\VS{12}Ἰδοὺ οἱ ὀφθαλμοὶ ὑμῶν βλέπουσι, καὶ οἱ ὀφθαλμοὶ Βενιαμεὶν τοῦ ἀδελφοῦ μου, ὅτι τὸ στόμα μου τὸ λαλοῦν πρὸς ὑμᾶς.
\VS{13}Ἀπαγγείλατε οὖν τῷ πατρί μου πᾶσαν τὴν δόξαν μου τὴν ἐν Αἰγύπτῳ, καὶ ὅσα ἴδετε· καὶ ταχύναντες καταγάγετε τὸν πατέρα μου ὧδε.
\VS{14}Καὶ ἐπιπεσὼν ἐπὶ τὸν τράχηλον Βενιαμὶν τοῦ ἀδελφοῦ αὐτοῦ, ἔκλαυσεν ἐπʼ αὐτῷ· καὶ Βενιαμὶν ἔκλαυσεν ἐπὶ τῷ τραχήλῳ αὐτοῦ.
\VS{15}Καὶ καταφιλήσας πάντας τοὺς ἀδελφοὺς αὐτοῦ, ἔκλαυσεν ἐπʼ αὐτοῖς· καὶ μετὰ ταῦτα ἐλάλησαν οἱ ἀδελφοὶ αὐτοῦ πρὸς αὐτόν.
\par }{\PP \VS{16}Καὶ διεβοήθη ἡ φωνὴ εἰς τὸν οἶκον Φαραὼ, λέγοντες, ἥκασιν οἱ ἀδελφοὶ Ἰωσήφ· ἐχάρη δὲ Φαραὼ καὶ ἡ θεραπεία αὐτοῦ.
\VS{17}Εἶπε δὲ Φαραὼ πρὸς Ἰωσὴφ, εἰπον τοῖς ἀδελφοῖς σου, τοῦτο ποιήσατε, γεμίσατε τὰ φορεῖα ὑμῶν, καὶ ἀπέλθετε εἰς γῆν Χαναάν.
\VS{18}Καὶ ἀναλαβόντες τὸν πατέρα ὑμῶν, καὶ τὰ ὑπάρχοντα ὑμῶν, ἥκετε πρός με· καὶ δώσω ὑμῖν πάντων τῶν ἀγαθῶν Αἰγύπτου, καὶ φάγεσθε τὸν μυελὸν τῆς γῆς.
\VS{19}Σὺ δὲ ἔντειλαι ταῦτα· λαβεῖν αὐτοῖς ἁμάξας ἐκ γῆς Αἰγύπτου τοῖς παιδίοις ὑμῶν, καὶ ταῖς γυναιξὶν ὑμῶν· καὶ ἀναλαβόντες τὸν πατέρα ὑμῶν παραγίνεσθε.
\VS{20}Καὶ μὴ φείσησθε τοῖς ὀφθαλμοῖς τῶν σκευῶν ὑμῶν· τὰ γὰρ πάντα ἀγαθὰ Αἰγύπτου ὑμῖν ἔσται.
\VS{21}Ἐποίησαν δὲ οὕτως οἱ υἱοὶ Ἰσραήλ· ἔδωκε δὲ Ἰωσὴφ αὐτοῖς ἁμάξας κατὰ τὰ εἰρημένα ὑπὸ Φαραὼ τοῦ βασιλέως· καὶ ἔδωκεν αὐτοῖς ἐπισιτισμὸν εἰς τὴν ὁδόν·
\VS{22}Καὶ πᾶσιν ἔδωκε δισσὰς στολάς· τῷ δὲ Βενιαμὶν ἔδωκε τριακοσίους χρυσοὺς, καὶ πέντε ἐξαλλασσούσας στολάς.
\VS{23}Καὶ τῷ πατρὶ αὐτοῦ ἀπέστειλε κατὰ τὰ αὐτά· καὶ δέκα ὄνους, αἴροντας ἀπὸ πάντων τῶν ἀγαθῶν Αἰγύπτου, καὶ δέκα ἡμιόνους, αἰρούσας ἄρτους τῷ πατρὶ αὐτοῦ εἰς ὁδόν.
\VS{24}Ἐξαπέστειλε δὲ τοὺς ἀδελφοὺς αὐτοῦ, καὶ ἐπορεύθησαν· καὶ εἶπεν αὐτοῖς, μὴ ὀργίζεσθε ἐν τῇ ὁδῷ.
\VS{25}Καὶ ἀνέβησαν ἐξ Αἰγυπτου, καὶ ἦλθον εἰς γῆν Χαναὰν πρὸς Ἰακὼβ τὸν πατέρα αὐτῶν.
\VS{26}Καὶ ἀνήγγειλαν αὐτῷ λέγοντες, ὅτι ὁ υἱός σου Ἰωσὴφ ζῇ, καὶ αὐτὸς ἄρχει πάσης γῆς Αἰγύπτου· καὶ ἐξέστη τῇ διανοίᾳ Ἰακὼβ, οὐ γὰρ ἐπίστευσεν αὐτοῖς.
\VS{27}Ἐλάλησαν δὲ αὐτῷ πάντα τὰ ῥηθέντα ὑπὸ Ἰωσὴφ, ὅσα εἶπεν αὐτοῖς. Ἰδὼν δὲ τὰς ἁμάξας, ἃς ἀπέστειλεν Ἰωσὴφ ὥστε ἀναλαβεῖν αὐτὸν, ἀνεζωπύρησε τὸ πνεῦμα Ἰακὼβ τοῦ πατρὸς αὐτῶν.
\VS{28}Εἶπε δὲ Ἰσραὴλ, μέγα μοι ἐστὶν, εἰ ἔτι Ἰωσὴφ ὁ υἱός μου ζῇ· πορευθεὶς ὄψομαι αὐτὸν πρὸ τοῦ ἀποθανεῖν με.

\par }\Chap{46}{\PP \VerseOne{1}Ἀπᾴρας δὲ Ἰσραὴλ, αὐτὸς καὶ πάντα τὰ αὐτοῦ, ἦλθεν ἐπὶ τὸ φρέαρ τοῦ ὅρκου· καὶ ἔθυσε θυσίαν τῷ Θεῷ τοῦ πατρὸς αὐτοῦ Ἰσαάκ.
\VS{2}Εἶπε δὲ ὁ Θεὸς τῷ Ἰσραὴλ ἐν ὁράματι τῆς νυκτὸς, εἰπὼν, Ἰακὼβ, Ἰακώβ· ὁ δὲ εἶπε, τί ἐστιν;
\VS{3}Ὁ δὲ λέγει αὐτῷ, ἐγώ εἰμι ὁ Θεὸς τῶν πατέρων σου· μὴ φοβοῦ καταβῆναι εἰς Αἴγυπτον· εἰς γὰρ ἔθνος μέγα ποιήσω σε ἐκεῖ.
\VS{4}Καὶ ἐγὼ καταβήσομαι μετὰ σοῦ εἰς Αἴγυπτον, καὶ ἐγὼ ἀναβιβάσω σε εἰς τέλος· καὶ Ἰωσὴφ ἐπιβαλεῖ τὰς χεῖρας ἐπὶ τοὺς ὀφθαλμούς σου.
\VS{5}Ἀνέστη δὲ Ἰακὼβ ἀπὸ τοῦ φρέατος τοῦ ὅρκου· καὶ ἀνέλαβον οἱ υἱοὶ Ἰσραὴλ τὸν πατέρα αὐτῶν, καὶ τὴν ἀποσκευὴν, καὶ τὰς γυναῖκας αὐτῶν, ἐπὶ τὰς ἁμάξας, ἃς ἀπέστειλεν Ἰωσὴφ ἆραι αὐτόν.
\VS{6}Καὶ ἀναλαβόντες τὰ ὑπάρχοντα αὐτῶν, καὶ πᾶσαν τὴν κτῆσιν, ἣν ἐκτήσαντο ἐκ γῇ Χαναὰν, εἰσῆλθον εἰς Αἴγυπτον, Ἰακὼβ, καὶ πᾶν τὸ σπέρμα αὐτοῦ μετʼ αὐτοῦ.
\VS{7}Υἱοὶ, καὶ υἱοὶ τῶν υἱῶν αὐτοῦ μετʼ αὐτοῦ· θυγατέρες, καὶ θυγατέρες τῶν θυγατέρων αὐτοῦ· καὶ πᾶν τὸ σπέρμα αὐτοῦ ἤγαγεν εἰς Αἴγυπτον·
\VS{8}Ταῦτα δὲ τὰ ὀνόματα τῶν υἱῶν Ἰσραὴλ τῶν εἰσελθόντων εἰς Αἴγυπτον ἅμα Ἰακὼβ τῷ πατρὶ αὐτῶν. Ἰακὼβ καὶ οἱ υἱοὶ αὐτοῦ· πρωτότοκος Ἰακὼβ, Ῥουβήν.
\VS{9}γἱοὶ δὲ Ῥουβὴν, Ἑνὼχ, καὶ Φαλλὸς, Ἀσρὼν, καὶ Χαρμί.
\VS{10}Υἱοὶ δὲ Συμεὼν, Ἰεμουὴλ, καὶ Ἰαμεὶν, καὶ Ἀὼδ, καὶ Ἀχεὶν, καὶ Σαὰρ, καὶ Σαοὺλ υἱὸς τῆς Χανανίτιδος.
\VS{11}Υἱοὶ δὲ Λευὶ, Γηρσὼν, Κὰθ, καὶ Μεραρί.
\VS{12}Υἱοὶ δὲ Ἰούδα, Ἢρ, καὶ Αὐνὰν, καὶ Σηλὼμ, καὶ Φαρὲς, καὶ Ζαρά· ἀπέθανε δὲ Ἢρ καὶ Αὐνὰν ἐν γῇ Χαναάν· ἐγένοντο δὲ υἱοὶ Φαρὲς, Ἑσρὼν, καὶ Ἰεμουήλ.
\VS{13}Υἱοὶ δὲ Ἰσσάχαρ, Θωλὰ, καὶ Φουὰ, καὶ Ἀσοὺμ, καὶ Σαμβράν.
\VS{14}Υἱοὶ δὲ Ζαβουλὼν, Σερὲδ, καὶ Ἀλλὼν, καὶ Ἀχοήλ.
\VS{15}Οὗτοι υἱοὶ Λείας, οὓς ἔτεκε τῷ Ἰακὼβ ἐν Μεσοποταμίᾳ τῆς Συρίας, καὶ Δείναν τὴν θυγατέρα αὐτοῦ· πᾶσαι αἱ ψυχαί, υἱοὶ καὶ θυγατέρες, τριάκοντα τρεῖς.
\VS{16}Υἱοὶ δὲ Γάδ, Σαφὼν, καὶ Ἀγγὶς, καὶ Σαννὶς, καὶ Θασοβὰν, καὶ Ἀηδεὶς, καὶ Ἀροηδεὶς, καὶ Ἀρεηλείς.
\VS{17}Υἱοὶ δὲ Ἀσὴρ, Ἰεμνα, Ἰεσσουὰ, καὶ Ἰεοὺλ, καὶ βαριὰ, καὶ Σάρα ἀδελφὴ αὐτῶν. Υἱοὶ δὲ βαριὰ, Χοβὸρ, καὶ Μελχιΐλ.
\VS{18}Οὗτοι υἱοὶ Ζελφᾶς, ἣν ἔδωκε Λάβαν Λείᾳ τῇ θυγατρὶ αὐτοῦ, ἣ ἔτεκε τούτους τῷ Ἰακὼβ, δεκαὲξ ψυχάς.
\VS{19}Υἱοὶ δὲ Ῥαχὴλ γυναικὸς Ἰακὼβ, Ἰωσὴφ, καὶ Βενιαμείν.
\VS{20}Ἐγένοντο δὲ υἱοὶ Ἰωσὴφ ἐν γῇ Αἰγύπτου, οὓς ἔτεκεν αὐτῷ Ἀσενὲθ θυγάτηρ Πετεφρῆ ἱερέως Ἡλιουπόλεως, τὸν Μανασσῆ, καὶ τὸν Ἐφραίμ· ἐγένοντο δὲ υἱοὶ Μανασσῆ, οὓς ἔτεκεν αὐτῷ ἡ παλλακὴ ἡ Σύρα, τὸν Μαχίρ· Μαχὶρ δὲ ἐγέννησε τὸν Γαλαάδ· υἱοὶ δὲ Ἐφραὶμ ἀδελφοῦ Μανασσῆ, Σουταλαὰμ, καὶ Ταάμ· υἱοὶ δὲ Σουταλαὰμ, Ἐδώμ.
\VS{21}Υἱοὶ δὲ Βενιαμεὶν, Βαλὰ καὶ Βοχὸρ, καὶ Ἀσβήλ. Ἐγένοντο δὲ υἱοὶ Βαλὰ, Γηρὰ, καὶ Νοεμὰν, καὶ Ἀγχὶς, καὶ Ῥὼς, καὶ Μαμφίμ· Γηρὰ δὲ ἐγέννησε τὸν Ἀράδ.
\VS{22}Οὗτοι υἱοὶ Ῥαχὴλ, οὓς ἔτεκε τῷ Ἰακώβ· πᾶσαι αἱ ψυχαὶ δεκαοκτώ.
\VS{23}Υἱοὶ δὲ Δὰν, Ἀσόμ.
\VS{24}Καὶ υἱοὶ Νεφθαλὶ, Ἀσιὴλ, καὶ Γωνὶ, καὶ Ἰσσάαρ, καὶ Σολλήμ.
\VS{25}Οὗτοι υἱοὶ Βαλλὰς, ἣν ἔδωκε Λάβαν Ῥαχὴλ τῇ θυγατρὶ αὐτοῦ, ἣ ἔτεκε τούτους τῷ Ἰακὼβ, πᾶσαι αἱ ψυχαὶ ἑπτά.
\VS{26}Πᾶσαι δὲ ψυχαὶ αἱ εἰσελθοῦσαι μετὰ Ἰακὼβ εἰς Αἴγυπτον, οἱ ἐξελθόντες ἐκ τῶν μηρῶν αὐτοῦ, χωρὶς τῶν γυναικῶν υἱῶν Ἰακὼβ, πᾶσαι αἱ ψυχαὶ, ἑξηκονταέξ·
\VS{27}Υἱοὶ δὲ Ἰωσὴφ οἱ γενόμενοι αὐτῷ ἐν γῇ Αἰγύπτῳ, ψυχαὶ ἐννέα. Πᾶσαι ψυχαὶ οἴκου Ἰακὼβ, αἱ εἰσελθοῦσαι μετὰ Ἰακὼβ εἰς Αἴγυπτον, ψυχαὶ ἑβδομηκονταπέντε.
\par }{\PP \VS{28}Τὸν δὲ Ἰούδαν ἀπέστειλεν ἔμπροσθεν αὐτοῦ πρὸς Ἰωσὴφ, συναντῆσαι αὐτῷ καθʼ Ἡρώων πόλιν, εἰς γῆν Ῥαμεσσῆ.
\VS{29}Ζεύξας δὲ Ἰωσὴφ τὰ ἅρματα αὐτοῦ, ἀνέβη εἰς συνάντησιν Ἰσραὴλ τῷ πατρὶ αὐτοῦ, καθʼ Ἡρώων πόλιν· καὶ ὀφθεὶς αὐτῷ ἐπέπεσεν ἐπὶ τὸν τράχηλον αὐτοῦ, καὶ ἔκλαυσε κλαυθμῷ πίονι.
\VS{30}Καὶ εἶπεν Ἰσραήλ πρὸς Ἰωσὴφ, ἀποθανοῦμαι ἀπὸ τοῦ νῦν, ἐπεὶ ἑώρακα τὸ πρόσωπόν σου· ἔτι γὰρ σὺ ζῇς.
\VS{31}Εἶπε δὲ Ἰωσὴφ πρὸς τοὺς ἀδελφοὺς αὐτοῦ, ἀναβὰς ἀπαγγελῶ τῷ Φαραῷ, καὶ ἐρῶ αὐτῷ, οἱ ἀδελφοί μου, καὶ ὁ οἶκος τοῦ πατρός μου, οἳ ἦσαν ἐν γῇ χαναὰν, ἥκασι πρός με.
\VS{32}Οἱ δὲ ἄνδρες εἰσὶ ποιμένες· ἄνδρες γὰρ κτηνοτρόφοι ἦσαν· καὶ τὰ κτήνη, καὶ τοὺς βόας, καὶ πάντα τὰ αὐτῶν ἀγηόχασιν.
\VS{33}Ἐὰν οὖν καλέσῃ ὑμᾶς Φαραὼ, καὶ εἴπῃ ὑμῖν, τί τὸ ἔργον ὑμῶν ἐστι;
\VS{34}Ἐρεῖτε, ἄνδρες κτηνοτρόφοι ἐσμὲν οἱ παῖδές σου ἐκ παιδὸς ἕως τοῦ νῦν, καὶ ἡμεῖς, καὶ οἱ πατέρες ἡμῶν· ἵνα κατοικήσητε ἐν γῇ Γεσὲμ Ἀραβίας· βδέλυγμα γάρ ἐστιν Αἰγυπτίοις πᾶς ποιμὴν προβάτων.

\par }\Chap{47}{\PP \VerseOne{1}Ἐλθῶν δὲ Ἰωσὴφ ἀπήγγειλε τῷ Φαραῶ, λέγων, ὁ πατὴρ μου, καὶ οἱ ἀδελφοί μου, καὶ τὰ κτήνη, καὶ οἱ βόες αὐτῶν, καὶ πάντα τὰ αὐτῶν, ἦλθον ἐκ γῆς Χαναάν· καὶ ἰδού εἰσιν ἐν γῇ Γεσέμ.
\VS{2}Ἀπὸ δὲ τῶν ἀδελφῶν αὐτοῦ παρέλαβε πέντε ἄνδρας, καὶ ἔστησεν αὐτοὺς ἐναντίον Φαραώ.
\VS{3}Καὶ εἶπε Φαραὼ τοῖς ἀδελφοῖς Ἰωσὴφ, Τί τὸ ἔργον ὑμῶν; οἱ δὲ εἶπαν τῷ Φαραῷ, ποιμένες προβάτων οἱ παῖδές σου, καὶ ἡμεῖς καὶ οἱ πατέρες ἡμῶν.
\VS{4}Εἶπαν δὲ τῷ Φαραῷ, παροικεῖν ἐν τῇ γῇ ἥκαμεν, οὐ γάρ ἐστι νομὴ τοῖς κτήνεσι τῶν παιδων σου, ἐνίσχυσε γὰρ ὁ λιμὸς ἐν γῇ Χανάαν· νῦν οὖν κατοικήσομεν ἐν γῇ Γεσέμ. Εἶπε δὲ Φαραὼ τῷ Ἰωσὴφ, Κατοικείτωσαν ἐν γῇ Γεσέμ· εἰ δὲ ἐπίστῃ, ὅτι εἰσὶν ἐν αὐτοῖς ἄνδρες δυνατοὶ, κατάστησον αὐτοὺς ἄρχοντας τῶν ἐμῶν κτηνῶν. Ἦλθον δὲ εἰς Αἴγυπτον πρὸς Ἰωσὴφ Ἰακὼβ, καὶ οἱ υἱοὶ αὐτοῦ· καὶ ἤκουσε Φαραὼ βασιλεὺς Αἰγύπτου.
\VS{5}Καὶ εἶπε Φαραὼ πρὸς Ἰωσὴφ, λέγων, ὁ πατήρ σου, καὶ οἱ ἀδελφοί σου, ἥκασι πρὸς σέ.
\VS{6}Ἰδοὺ ἡ γῆ Αἰγύπτου ἐναντίον σου ἐστίν· ἐν τῇ βελτίστῃ γῇ κατοίκισον τὸν πατέρα σου, καὶ τοὺς ἀδελφούς σου.
\VS{7}Εἰσήγαγε δὲ Ἰωσὴφ Ἰακὼβ τὸν πατέρα αὐτοῦ, καὶ ἔστησεν αὐτὸν ἐναντίον Φαραώ· καὶ ηὐλόγησεν Ἰακὼβ τὸν Φαραώ.
\VS{8}Εἶπε δὲ Φαραὼ τῷ Ἰακὼβ, πόσα ἔτη ἡμερῶν τῆς ζωῆς σου;
\VS{9}Καὶ εἶπεν Ἰακὼβ τῷ Φαραῷ, αἱ ἡμέραι τῶν ἐτῶν τῆς ζωῆς μου, ἃς παροικῶ, ἑκατὸν τριάκοντα ἔτη· μικραὶ καὶ πονηραὶ γεγόνασιν αἱ ἡμέραι τῶν ἐτῶν τῆς ζωῆς μου· οὐκ ἀφίκοντο εἰς τὰς ἡμέρας τῶν ἐτῶν τῆς ζῶης τῶν πατέρων μου, ἃς ἡμέρας παρῴκησαν.
\VS{10}Καὶ εὐλογήσας Ἰακὼβ τὸν Φαραὼ, ἐξῆλθεν ἀπʼ αὐτοῦ.
\par }{\PP \VS{11}Καὶ κατῴκισεν Ἰωσὴφ τὸν πατέρα αὐτοῦ, καὶ τοὺς ἀδελφοὺς αὐτοῦ, καὶ ἔδωκεν αὐτοῖς κατάσχεσιν ἐν γῇ Αἰγύπτῳ, ἐν τῇ βελτίστῃ γῇ, ἐν γῇ Ῥαμεσσῆ, καθὰ προσέταξε Φαραώ.
\VS{12}Καὶ ἐσιτομέτρει Ἰωσὴφ τῷ πατρὶ αὐτοῦ, καὶ τοῖς ἀδελφοῖς, καὶ παντὶ τῷ οἴκῳ τοῦ πατρὸς αὐτοῦ, σῖτον κατὰ σῶμα.
\par }{\PP \VS{13}Σῖτος δὲ οὐκ ἦν ἐν πάσῃ τῇ γῇ, ἐνίσχυσε γὰρ ὁ λιμὸς σφόδρα· ἐξέλιπε δὲ ἡ γῆ Αἰγύπτου καὶ ἡ γῆ Χαναὰν ἀπὸ τοῦ λιμοῦ.
\VS{14}Συνήγαγε δὲ Ἰωσὴφ πᾶν τὸ ἀργύριον τὸ εὑρεθὲν ἐν γῇ Αἰγύπτου καὶ ἐν γῇ Χαναὰν, τοῦ σίτου, οὗ ἠγόραζον, καὶ ἐσιτομέτρει αὐτοῖς, καὶ εἰσήνεγκεν Ἰωσὴφ πᾶν τὸ ἀργύριον εἰς τὸν οἶκον Φαραώ.
\VS{15}Καὶ ἐξέλιπε πᾶν τὸ ἀργύριον ἐκ γῆς Αἰγύπτου καὶ ἐκ γῆς Χαναάν· ἦλθον δὲ πάντες οἱ Αἰγύπτιοι πρὸς Ἰωσὴφ, λέγοντες, δὸς ἡμῖν ἄρτους, καὶ ἱνατί ἀποθνήσκομεν ἐναντίον σου; ἐκλέλοιπε γὰρ τὸ ἀργύριον ἡμῶν.
\VS{16}Εἶπε δὲ αὐτοῖς Ἰωσὴφ, φέρετε τὰ κτήνη ὑμῶν, καὶ δώσω ὑμῖν ἄρτους, ἀντὶ τῶν κτηνῶν ὑμῶν, εἰ ἐκλέλοιπε τὸ ἀργύριον ὑμῶν.
\VS{17}Ἤγαγον δὲ τὰ κτήνη αὐτῶν πρὸς Ἰωσήφ· καὶ ἔδωκεν αὐτοῖς Ἰωσὴφ ἄρτους ἀντὶ τῶν ἵππων, καὶ ἀντὶ τῶν προβάτων, καὶ ἀντὶ τῶν βοῶν, καὶ ἀντὶ τῶν ὄνων· καὶ ἐξέθρεψεν αὐτοὺς ἐν ἄρτοις ἀντὶ πάντων τῶν κτηνῶν αὐτῶν ἐν τῷ ἐνιαυτῷ ἐκείνῳ.
\VS{18}Ἐξῆλθε δὲ τὸ ἔτος ἐκεῖνο, καὶ ἦλθον πρὸς αὐτὸν ἐν τῷ ἔτει τῷ δευτέρῳ, καὶ εἶπαν αὐτῷ, μή ποτε ἐκτριβῶμεν ἀπὸ τοῦ κυρίου ἡμῶν; εἰ γὰρ ἐκλέλοιπε τὸ ἀργύριον ἡμῶν, καὶ τὰ ὑπάρχοντα καὶ τὰ κτήνη πρός σε τὸν κύριον, καὶ οὐχ ὑπολέλειπται ἡμῖν ἐναντίον τοῦ κυρίου ἡμῶν, ἀλλʼ ἢ τὸ ἴδιον σῶμα καὶ ἡ γῆ ἡμῶν,
\VS{19}ἵνα οὖν μὴ ἀποθάνωμεν ἐναντίον σου, καὶ ἡ γῆ ἐρημωθῇ, κτῆσαι ἡμᾶς καὶ τὴν γῆν ἡμῶν ἀντὶ ἄρτων, καὶ ἐσόμεθα ἡμεῖς καὶ ἡ γῆ ἡμῶν παῖδες τῷ Φαραώ· δὸς σπέρμα, ἵνα σπείρωμεν, καὶ ζῶμεν καὶ μὴ ἀποθάνωμεν, καὶ ἡ γῆ οὐκ ἐρημωθήσεται.
\VS{20}Καὶ ἐκτήσατο Ἰωσὴφ πᾶσαν τὴν γῆν τῶν Αἰγυπτίων τῷ Φαραώ· ἀπέδοντο γὰρ οἱ Αἰγύπτιοι τὴν γῆν αὐτῶν τῷ Φαραώ· ἐπεκράτησε γὰρ αὐτῶν ὁ λιμός· καὶ ἐγένετο ἡ γῆ τῷ Φαραώ.
\VS{21}Καὶ τὸν λαὸν κατεδουλώσατο αὐτῷ εἰς παῖδας, ἀπʼ ἄκρων ὁρίων Αἰγύπτου ἕως τῶν ἄκρων,
\VS{22}χωρὶς τῆς γῆς τῶν ἱερέων μόνον· οὐκ ἐκτήσατο ταύτην Ἰωσήφ· ἐν δόσει γὰρ ἔδωκε δόμα τοῖς ἱερεῦσι Φαραὼ, καὶ ἤσθιον τὴν δόσιν, ἣν ἔδωκεν αὐτοῖς Φαραώ· διὰ τοῦτο οὐκ ἀπέδοντο τὴν γῆν αὐτῶν.
\VS{23}Εἶπε δὲ Ἰωσὴφ πᾶσι τοῖς Αἰγυπτίοις, ἰδοὺ κέκτημαι ὑμᾶς καὶ τὴν γῆν ὑμῶν σήμερον τῷ Φαραῷ· λάβετε ἑαυτοῖς σπέρμα, καὶ σπείρατε τὴν γῆν.
\VS{24}Καὶ ἔσται τὰ γεννήματα αὐτῆς· καὶ δώσετε τὸ πεμπτὸν μέρος τῷ Φαραώ· τὰ δὲ τέσσαρα μέρη ἔσται ὑμῖν αὐτοῖς εἰς σπέρμα τῇ γῇ, καὶ εἰς βρῶσιν ὑμῖν, καὶ πᾶσι τοῖς ἐν τοῖς οἴκοις ὑμῶν.
\VS{25}Καὶ εἶπαν, σέσωκας ἡμᾶς· εὕρομεν χάριν ἐναντίον τοῦ κυρίου ἡμῶν, καὶ ἐσόμεθα παῖδες τῷ Φαραώ.
\VS{26}Καὶ ἔθετο αὐτοῖς Ἰωσὴφ εἰς πρόσταγμα ἕως τῆς ἡμέρας ταύτης, ἐπὶ γῆς Αἰγύπτου τῷ Φαραὼ ἀποπεμπτοῦν, χωρὶς τῆς γῆς τῶν ἱερέων μόνον· οὐκ ἧν τῷ Φαραώ.
\par }{\PP \VS{27}Κατῶκησε δὲ Ἰσραὴλ ἐν γῇ Αἰγύπτῳ ἐπὶ γῆς Γεσὲμ, καὶ ἐκληρονόμησαν ἐπʼ αὐτῆς· καὶ ηὐξήθησαν καὶ ἐπληθύνθησαν σφόδρα.
\VS{28}Ἐπέζησε δὲ Ἰακὼβ ἐν γῇ Αἰγύπτῳ δεκαεπτὰ ἔτη· καὶ ἐγένοντο αἱ ἡμέραι Ἰακὼβ ἐνιαυτῶν τῆς ζωῆς αὐτοῦ ἑκατὸν τεσσαρακονταεπτὰ ἔτη.
\VS{29}Ἤγγισαν δὲ αἱ ἡμέραι Ἰσραὴλ τοῦ ἀποθανεῖν· καὶ ἐκάλεσε τὸν υἱὸν αὐτοῦ Ἰωσὴφ, καὶ εἶπεν αὐτῷ, εἰ εὕρηκα χάριν ἐναντίον σου, ὑπόθες τὴν χεῖρά σου ὑπὸ τὸν μηρόν μου, καὶ ποιήσεις ἐπʼ ἐμὲ ἐλεημοσύνην, καὶ ἀλήθειαν, τοῦ μή με θάψαι ἐν Αἰγύπτῳ·
\VS{30}Ἀλλὰ κοιμηθήσομαι μετὰ τῶν πατέρων μου· καὶ ἀρεῖς με ἐξ Αἰγύπτου, καὶ θάψεις με ἐν τῷ τάφῳ αὐτῶν· ὁ δὲ εἶπεν, ἐγὼ ποιήσω κατὰ τὸ ῥῆμά σου.
\VS{31}Εἶπε δὲ, ὄμοσόν μοι· καὶ ὤμοσεν αὐτῷ· καὶ προσεκύνησεν Ἰσραὴλ ἐπὶ τὸ ἄκρον τῆς ῥάβδου αὐτοῦ.

\par }\Chap{48}{\PP \VerseOne{1}Ἐγένετο δὲ μετὰ τὰ ῥήματα ταῦτα, καὶ ἀπηγγέλη τῷ Ἰωσὴφ, ὅτι ὁ πατήρ σου ἐνοχλεῖται· καὶ ἀναλαβὼν τοὺς δύο υἱοὺς αὐτοῦ τὸν Μανασσῆ καὶ τὸν Ἐφραὶμ, ἦλθε πρὸς Ἰακώβ.
\VS{2}Ἀπηγγέλη δὲ τῷ Ἰακὼβ, λέγοντες, ἰδοὺ ὁ υἱός σου Ἰωσὴφ ἔρχεται πρὸς σέ· καὶ ἐνισχύσας Ἰσραὴλ ἐκάθισεν ἐπὶ τὴν κλίνην.
\VS{3}Καὶ εἶπεν Ἰακὼβ τῷ Ἰωσὴφ, ὁ Θεός μου ὤφθη μοι ἐν Λουζᾷ ἐν γῇ Χαναὰν, καὶ εὐλόγησέ με,
\VS{4}καὶ εἶπέ μοι, ἰδοὺ ἐγὼ αὐξανῶ σε, καὶ πληθυνῶ σε, καὶ ποιήσω σε εἰς συναγωγὰς ἐθνῶν· καὶ δώσω σοι τὴν γῆν ταύτην, καὶ τῷ σπέρματί σου μετὰ σὲ, εἰς κατάσχεσιν αἰώνιον.
\VS{5}Νῦν οὖν οἱ δύο υἱοί σου, οἱ γενόμενοί σοι ἐν γῇ Αἰγύπτῳ πρὸ τοῦ με ἐλθεῖν πρός σε εἰς Αἴγυπτον, ἐμοί εἰσιν, Ἐφραὶμ καὶ Μανασσῆ· ὡς Ῥουβὴν καὶ Συμεὼν ἔσονταί μοι.
\VS{6}Τὰ δὲ ἔκγονα, ἃ ἐὰν γεννήσῃς μετὰ ταῦτα, ἔσονται ἐπὶ τῷ ὀνόματι τῶν ἀδελφῶν αὐτῶν· κληθήσονται ἐπὶ τοῖς ἐκείνων κλήροις.
\VS{7}Ἐγὼ δὲ ἡνίκα ἠρχόμην ἐκ Μεσοποταμίας τῆς Συρίας, ἀπέθανε Ῥαχὴλ ἡ μήτηρ σου ἐν γῇ Χαναὰν, ἐγγίζοντός μου κατὰ τὸν ἱππόδρομον Χαβραθὰ τῆς γῆς, τοῦ ἐλθεῖν Ἐφραθά· καὶ κατώρυξα αὐτὴν ἐν τῇ ὁδῷ τοῦ ἱπποδρόμου· αὕτη ἐστὶ Βηθλεέμ.
\par }{\PP \VS{8}Ἰδὼν δὲ Ἰσραὴλ τοὺς υἱοὺς Ἰωσὴφ, εἶπε, τίνες σοι οὗτοι;
\VS{9}Εἶπε δὲ Ἰωσὴφ τῷ πατρὶ αὐτοῦ, υἱοί μου εἰσὶν, οὓς ἔδωκε μοι ὁ Θεὸς ἐνταῦθα. Καὶ εἶπεν Ἰακὼβ, προσάγαγέ μοι αὐτοὺς, ἵνα εὐλογήσω αὐτούς.
\VS{10}Οἱ ὀφθαλμοὶ δὲ Ἰσραὴλ ἐβαρυώπησαν ἀπὸ τοῦ γήρως, καὶ οὐκ ἠδύνατο βλέπειν· καὶ ἤγγισεν αὐτοὺς πρὸς αὐτὸν, καὶ ἐφίλησεν αὐτοὺς, καὶ περιέλαβεν αὐτους.
\VS{11}Καὶ εἶπεν Ἰσραὴλ πρὸς Ἰωσὴφ, ἰδοὺ τοῦ προσώπου σου οὐκ ἐστερήθην, καὶ ἰδοὺ ἔδειξέ μοι ὁ Θεὸς καὶ τὸ σπέρμα σου.
\VS{12}Καὶ ἐξήγαγεν αὐτοὺς Ἰωσὴφ ἀπὸ τῶν γονάτων αὐτοῦ· καὶ προσεκύνησαν αὐτῷ ἐπὶ πρόσωπον ἐπὶ τῆς γῆς.
\VS{13}Λαβὼν δὲ Ἰωσὴφ τοὺς δύο υἱοὺς αὐτοῦ, τόν τε Ἐφραὶμ ἐν τῇ δεξιᾷ, ἐξ ἀριστερῶν δὲ Ἰσραὴλ, τὸν δὲ Μανασσῆ ἐξ ἀριστερῶν, ἐκ δεξιῶν δὲ Ἰσραὴλ, ἤγγισεν αὐτοὺς αὐτῷ.
\VS{14}Ἐκτείνας δὲ Ἰσραὴλ τὴν χεῖρα τὴν δεξιὰν, ἐπέβαλεν ἐπὶ τὴν κεφαλὴν Ἐφραὶμ, οὗτος δὲ ἦν ὁ νεώτερος, καὶ τὴν ἀριστερὰν ἐπὶ τὴν κεφαλὴν Μανασσῆ, ἐναλλὰξ τὰς χεῖρας.
\par }{\PP \VS{15}Καὶ εὐλόγησεν αὐτοὺς, καὶ εἶπεν, ὁ Θεὸς, ᾧ εὐηρέστησαν οἱ πατέρες μου ἐνώπιον αὐτοῦ, Ἁβραὰμ καὶ Ἰσαὰκ, ὁ Θεὸς ὁ τρέφων με ἐκ νεότητος ἕως τῆς ἡμέρας ταύτης,
\VS{16}ὁ Ἄγγελος ὁ ῥυόμενός με ἐκ πάντων τῶν κακῶν, εὐλογήσαι τὰ παιδία ταῦτα· καὶ ἐπικληθήσεται ἐν αὐτοις τὸ ὄνομά μου, καὶ τὸ ὄνομα τῶν πατέρων μου Ἁβραὰμ καὶ Ἰσαάκ· καὶ πληθυνθείησαν εἰς πλῆθος πολὺ ἐπὶ τῆς γῆς.
\VS{17}Ἰδὼν δὲ Ἰωσὴφ ὅτι ἐπέβαλεν ὁ πατὴρ αὐτοῦ τὴν χεῖρα τὴν δεξιὰν αὐτοῦ ἐπὶ τὴν κεφαλὴν Ἐφραὶμ, βαρὺ αὐτῷ κατεφάνη· καὶ ἀντελάβετο Ἰωσὴφ τῆς χειρὸς τοῦ πατρὸς αὐτοῦ, ἀφελεῖν αὐτὴν ἀπὸ τῆς κεφαλῆς Ἐφραὶμ ἐπὶ τὴν κεφαλὴν Μανασσῆ.
\VS{18}Εἶπε δὲ Ἰωσὴφ τῷ πατρὶ αὐτοῦ, οὐχ οὕτως, πατὴρ, οὗτος γὰρ ὁ πρωτότοκος· ἐπίθες τὴν δεξιάν σου ἐπὶ τὴν κεφαλὴν αὐτοῦ.
\VS{19}Καὶ οὐκ ἠθέλησεν, ἀλλʼ εἶπεν, οἶδα, τέκνον, οἶδα· καὶ οὗτος ἔσται εἰς λαὸν, καὶ οὗτος ὑψωθήσεται· ἀλλʼ ὁ ἀδελφὸς αὐτοῦ ὁ νεώτερος μείζον αὐτοῦ ἔσται, καὶ τὸ σπέρμα αὐτοῦ ἔσται εἰς πλῆθος ἐθνῶν.
\VS{20}Καὶ εὐλόγησεν αὐτοὺς ἐν τῇ ἡμέρᾳ ἐκείνῃ, λέγων, ἐν ὑμῖν εὐλογηθήσεται Ἰσραὴλ, λέγοντες, ποιήσαι σε ὁ Θεὸς ὡς Ἐφραὶμ καὶ ὡς Μανασσῆ· καὶ ἔθηκε τὸν Ἐφραὶμ ἔμπροσθεν τοῦ Μανασσῆ.
\VS{21}Εἶπε δὲ Ἰσραὴλ τῷ Ἰωσὴφ, ἰδοὺ ἐγὼ ἀποθνήσκω· καὶ ἔσται ὁ Θεὸς μεθʼ ὑμῶν, καὶ ἀποστρέψει ὑμᾶς εἰς τὴν γῆν τῶν πατέρων ὑμῶν.
\VS{22}Ἐγὼ δὲ δίδωμί σοι Σίκιμα ἐξαίρετον ὑπὲρ τοὺς ἀδελφούς σου, ἣν ἔλαβον ἐκ χειρὸς Ἀμοῤῥαίων ἐν μαχαίρᾳ μου καὶ τόξῳ.

\par }\Chap{49}{\PP \VerseOne{1}Ἐκάλεσε δὲ Ἰακὼβ τοὺς υἱοὺς αὐτοῦ, καὶ εἶπεν αὐτοῖς, συνάχθητε, ἵνα ἀναγγείλω ὑμῖν, τί ἀπαντήσει ὑμῖν ἐπʼ ἐσχάτων τῶν ἡμέρων.
\VS{2}Συνάχθητε, καὶ ἀκούσατέ μου, υἱοὶ Ἰακώβ· ἀκούσατε Ἰσραὴλ, ἀκούσατε τοῦ πατρὸς ὑμῶν.
\VS{3}Ῥουβὴν πρωτότοκός μου, σὺ ἰσχύς μου, καὶ ἀρχὴ τέκνων μου, σκληρὸς φέρεσθαι, καὶ σκληρὸς αὐθάδης.
\VS{4}Ἐξύβρισας ὡς ὕδωρ, μὴ ἐκζέσῃς, ἀνέβης γὰρ ἐπὶ τὴν κοίτην τοῦ πατρός σου· τότε ἐμίανας τὴν στρωμνὴν, οὗ ἀνέβης.
\VS{5}Συμεὼν καὶ Λευὶ ἀδελφοὶ συνετέλεσαν ἀδικίαν ἐξαιρέσεως αὐτῶν·
\VS{6}Εἰς βουλὴν αὐτῶν μὴ ἔλθοι ἡ ψυχή μου, καὶ ἐπὶ τῇ συστάσει αὐτῶν μὴ ἐρίσαι τὰ ἥπατά μου· ὅτι ἐν τῷ θυμῷ αὐτῶν ἀπέκτειναν ἀνθρώπους, καὶ ἐν τῇ ἐπιθυμίᾳ αὐτῶν ἐνευροκόπησαν ταῦρον.
\VS{7}Ἐπικατάρατος ὁ θυμὸς αὐτὼν, ὅτι αὐθάδης· καὶ ἡ μῆνις αὐτῶν, ὅτι ἐσκληρύνθη· διαμεριῷ αὐτοὺς ἐν Ἰακὼβ, καὶ διασπερῷ αὐτοὺς ἐν Ἰσραήλ.
\VS{8}Ἰούδα, σὲ αἰνέσαισαν οἱ ἀδελφοί σου· αἱ χεῖρές σου ἐπὶ νώτου τῶν ἐχθρῶν σου· προσκυνήσουσί σοι οἱ υἱοὶ τοῦ πατρός σου.
\VS{9}Σκύμνος λέοντος Ἰούδα· ἐκ βλαστοῦ, υἱέ μου, ἀνέβης· ἀναπεσὼν ἐκοιμήθης ὡς λέων καὶ ὡς σκύμνος· τίς ἐγερεῖ αὐτόν;
\VS{10}Οὐκ ἐκλείψει ἄρχων ἐξ Ἰούδα, καὶ ἡγούμενος ἐκ τῶν μηρῶν αὐτοῦ, ἕως ἐὰν ἔλθῃ τὰ ἀποκείμενα αὐτῷ· καὶ αὐτὸς προσδοκία ἐθνῶν.
\VS{11}Δεσμεύων πρὸς ἄμπελον τὸν πῶλον αὐτοῦ, καὶ τῇ ἕλικι τὸν πῶλον τῆς ὄνου αὐτοῦ, πλυνεῖ ἐν οἴνῳ τὴν στολὴν αὐτοῦ, καὶ ἐν αἵματι σταφυλῆς τὴν περιβολὴν αὐτοῦ.
\VS{12}Χαροποιοὶ οἱ ὀφθαλμοὶ αὐτοῦ ὑπὲρ οἶνον· καὶ λευκοὶ οἱ ὀδόντες αὐτοῦ ἢ γάλα.
\VS{13}Ζαβουλὼν παράλιος κατοικήσει καὶ αὐτὸς παρʼ ὅρμον πλοίων, καὶ παρατενεῖ ἕως Σιδῶνος.
\VS{14}Ἰσσάχαρ τὸ καλὸν ἐπεθύμησεν, ἀναπαυόμενος ἀνὰ μέσον τῶν κλήρων.
\VS{15}Καὶ ἰδὼν τὴν ἀνάπαυσιν ὅτι καλὴ, καὶ τὴν γῆν ὅτι πίων, ὑπέθηκε τὸν ὦμον αὐτοῦ εἰς τὸ πονεῖν, καὶ ἐγενήθη ἀνὴρ γεωργός.
\VS{16}Δὰν κρινεῖ τὸν λαὸν αὐτοῦ, ὡσεὶ καὶ μία φυλὴ ἐν Ἰσραήλ.
\VS{17}Καὶ γενηθητω Δὰν ὄφις ἐφʼ ὁδοῦ, ἐγκαθήμενος ἐπὶ τρίβου, δάκνων πτέρναν ἵππου· καὶ πεσεῖται ὁ ἱππεὺς εἰς τὰ ὀπίσω,
\VS{18}τὴν σωτηρίαν περιμένων Κυρίου.
\VS{19}Γὰδ, πειρατήριον πειρατεύσει αὐτόν· αὐτὸς δὲ πειράτεύσει αὐτὸν κατὰ πόδας.
\VS{20}Ἀσὴρ, πίων αὐτοῦ ὁ ἄρτος· καὶ αὐτὸς δώσει τρυφὴν ἄρχουσι.
\VS{21}Νεφθαλὶ στέλεχος ἀνειμένον, ἐπιδιδοὺς ἐν τῷ γεννήματι κάλλος.
\VS{22}Υἱὸς ηὐξημένος Ἰωσὴφ, υἱὸς ηὐξημένος μου ζηλωτὸς, υἱός μου νεώτατος· πρός με ἀνάστρεψον.
\VS{23}Εἰς ὃν διαβουλευόμενοι ἐλοιδόρουν, καὶ ἐνεῖχον αὐτῷ κύριοι τοξευμάτων.
\VS{24}Καὶ συνετρίβη μετὰ κράτους τὰ τόξα αὐτῶν· καὶ ἐξελύθη τὰ νεῦρα βραχιόνων χειρὸς αὐτῶν, διὰ χεῖρα δυνάστου Ἰακώβ· ἐκεῖθεν ὁ κατισχύσας Ἰσραὴλ παρὰ Θεοῦ τοῦ πατρός σου.
\VS{25}Καὶ ἐβοήθησέ σοι ὁ Θεὸς ὁ ἐμὸς, καὶ εὐλόγησέ σε εὐλογίαν οὐρανοῦ ἄνωθεν, καὶ εὐλογίαν γῆς ἐχούσης πάντα, εἵνεκεν εὐλογίας μαστῶν καὶ μήτρας,
\VS{26}εὐλογίας πατρός σου καὶ μητρός σου· ὑπερίσχυσεν ὑπὲρ εὐλογίας ὀρέων μονίμων, καὶ ἐπʼ εὐλογίαις θινῶν ἀενάων· ἔσονται ἐπὶ κεφαλὴν Ἰωσὴφ, καὶ ἐπὶ κορυφῆς ὧν ἡγήσατο ἀδελφῶν.
\VS{27}Βενιαμὶν λύκος ἅρπαξ, τὸ πρωϊνὸν ἔδεται ἔτι, καὶ εἰς τὸ ἑσπέρας δίδωσι τροφήν.
\par }{\MM \VS{28}Πάντες οὕτοι υἱοὶ Ἰακὼβ δώδεκα· καὶ ταῦτα ἐλάλησεν αὐτοῖς ὁ πατὴρ αὐτῶν· καὶ εὐλόγησεν αὐτούς· ἕκαστον κατὰ τὴν εὐλογίαν αὐτοῦ εὐλόγησεν αὐτούς.
\VS{29}Καὶ εἶπεν αὐτοῖς, ἐγὼ προστίθεμαι πρὸς τὸν ἐμὸν λαόν· θάψτέ με μετὰ τῶν πατέρων μου ἐν τῷ σπηλαίῳ, ὅ ἐστιν ἐν τῷ ἀγρῷ Ἐφρὼν τοῦ Χετταίου,
\VS{30}ἐν τῷ σπηλαίῳ τῷ διπλῷ, τῷ ἀπέναντι Μαμβρῆ, ἐν γῇ Χαναὰν, ὃ ἐκτήσατο Ἁβραὰμ τὸ σπήλαιον παρὰ Ἐφρὼν τοῦ Χετταίου ἐν κτήσει μνημείου.
\VS{31}Ἐκεῖ ἔθαψαν Ἁβραὰμ καὶ Σάῤῥαν τὴν γυναῖκα αὐτοῦ· ἐκεῖ ἔθαψαν Ἰσαὰκ καὶ Ῥεβέκκαν τὴν γυναῖκα αὐτοῦ· ἐκεῖ ἔθαψαν Λείαν·
\VS{32}Ἐν κτήσει τοῦ ἀγροῦ καὶ τοῦ σπηλαίου τοῦ ὄντος ἐν αὐτῷ, παρὰ τῶν υἱῶν Χέτ.
\VS{33}Καὶ κατέπαυσεν Ἰακὼβ ἐπιτάσσων τοῖς υἱοῖς αὐτοῦ· καὶ ἐξᾴρας τοὺς πόδας αὐτοῦ ἐπὶ τὴν κλίνην, ἐξέλιπε· καὶ προσετέθη πρὸς τὸν λαὸν αὐτοῦ.

\par }\Chap{50}{\PP \VerseOne{1}Καὶ ἐπιπεσὼν Ἰωσὴφ ἐπὶ πρόσωπον τοῦ πατρὸς αὐτοῦ ἔκλαυσεν αὐτὸν, καὶ ἐφίλησεν αὐτόν.
\VS{2}Καὶ προσέταξεν Ἰωσὴφ τοῖς παισὶν αὐτοῦ τοῖς ἐνταφιασταῖς, ἐνταφιάσαι τὸν πατέρα αὐτοῦ· καὶ ἐνεταφίασαν οἱ ἐνταφιασταὶ τὸν Ἰσραήλ.
\VS{3}Καὶ ἐπλήρωσαν αὐτοῦ τεσσαράκοντα ἡμέρας· οὕτω γὰρ καταριθμοῦνται αἱ ἡμέραι τῆς ταφῆς· καὶ ἐπένθησεν αὐτὸν Αἴγυπτος ἑβδομήκοντα ἡμέρας.
\VS{4}Ἐπεὶ δὲ παρῆλθον αἱ ἡμέραι τοῦ πένθους, ἐλάλησεν Ἰωσὴφ πρὸς τοὺς δυνάστας Φαραὼ, λέγων, εἰ εὗρον χάριν ἐναντίον ὑμῶν, λαλήσατε περὶ ἐμοῦ εἰς τὰ ὦτα Φαραὼ, λέγοντες,
\VS{5}ὁ πατήρ μου ὥρκισέ με, λέγων, ἐν τῷ μνημείῳ, ᾧ ὤρυξα ἐμαυτῷ ἐν γῇ Χαναὰν, ἐκεῖ με θάψεις· νῦν οὖν ἀναβὰς· θάψω τὸν πατέρα μου, καὶ ἐπανελεύσομαι·
\VS{6}Καὶ εἶπε Φαραὼ τῷ Ἰωσὴφ, ἀνάβηθι, θάψον τὸν πατέρα σου, καθάπερ ὥρκισέ σε.
\VS{7}Καὶ ἀνέβη Ἰωσὴφ θάψαι τὸν πατέρα αὐτοῦ· καὶ συνανέβησαν μετʼ αὐτοῦ πάντες οἱ παῖδες Φαραὼ, καὶ οἱ πρεσβύτεροι τοῦ οἴκου αὐτοῦ, καὶ πάντες οἱ πρεσβύτεροι τῆς γῆς Αἰγύπτου,
\VS{8}καὶ πᾶσα ἡ πανοικία Ἰωσὴφ, καὶ οἱ ἀδελφοὶ αὐτοῦ, καὶ πᾶσα ἡ οἰκία ἡ πατρικὴ αὐτοῦ, καὶ ἡ συγγένεια αὐτοῦ· καὶ τὰ πρόβατα, καὶ τοὺς βόας ὑπελίποντο ἐν γῇ Γεσέμ.
\VS{9}Καὶ συνανέβησαν μετʼ αὐτοῦ καὶ ἅρματα καὶ ἱππεῖς, καὶ ἐγένετο ἡ παρεμβολὴ μεγάλη σφόδρα.
\VS{10}Καὶ παρεγένοντο εἰς ἅλωνα Ἀτὰδ, ὅ ἐστι πέραν τοῦ Ἰορδάνου· καὶ ἐκόψαντο αὐτὸν κοπετὸν μέγαν καὶ ἰσχυρὸν σφόδρα· καὶ ἐποίησε τὸ πένθος τῷ πατρὶ αὐτοῦ ἑπτὰ ἡμέρας.
\VS{11}Καὶ εἶδον οἱ κάτοικοι τῆς γῆς Χαναὰν τὸ πένθος ἐπὶ ἅλωνι Ἀτὰδ, καὶ εἶπαν, πένθος μέγα τοῦτό ἐστι τοῖς Αἰγυπτίοις· διὰ τοῦτο ἐκάλεσε τὸ ὄνομα αὐτοῦ, Πένθος Αἰγύπτου, ὅ ἐστι πέραν τοῦ Ἰορδάνου.
\VS{12}Καὶ ἐποίησαν αὐτῷ οὕτως οἱ υἱοὶ αὐτοῦ.
\VS{13}Καὶ ἀνέλαβον αὐτὸν οἱ υἱοὶ αὐτοῦ εἰς γῆν Χαναάν· καὶ ἔθαψαν αὐτὸν εἰς τὸ σπήλαιον τὸ διπλοῦν, ὃ ἐκτήσατο Ἁβραὰμ τὸ σπήλαιον ἐν κτήσει μνημείου παρὰ Ἐφρὼν τοῦ Χετταίου, κατέναντι Μαμβρή.
\VS{14}Καὶ ὑπέστρεψεν Ἰωσὴφ εἰς Αἴγυπτον, αὐτὸς καὶ οἱ ἀδελφοὶ αὐτοῦ, καὶ οἱ συναναβάντες θάψαι τὸν πατέρα αὐτοῦ.
\par }{\PP \VS{15}Ἰδόντες δὲ οἱ ἀδελφοὶ Ἰωσὴφ, ὅτι τέθνηκεν ὁ πατὴρ αὐτῶν, εἶπαν, μή ποτε μνησικακήσῃ ἡμῖν Ἰωσὴφ, καὶ ἀνταπόδομα ἀνταποδῷ ἡμῖν πάντα τὰ κακὰ, ἃ ἐνεδειξάμεθα εἰς αὐτὸν.
\VS{16}Καὶ παραγενόμενοι πρὸς Ἰωσὴφ εἶπαν, ὁ πατήρ σου ὥρκισε πρὸ τοῦ τελευτῆσαι αὐτὸν, λέγων,
\VS{17}οὕτως εἴπατε Ἰωσήφ· ἄφες αὐτοῖς τὴν ἀδικίαν καὶ τὴν ἁμαρτίαν αὐτῶν, ὅτι πονηρά σοι ἐνεδείξαντο· καὶ νῦν δέξαι τὴν ἀδικίαν τῶν θεραπόντων τοῦ Θεοῦ τοῦ πατρός σου· καὶ ἔκλαυσεν Ἰωσὴφ λαλούντων αὐτῶν πρὸς αὐτόν.
\VS{18}Καὶ ἐλθόντες πρὸς αὐτὸν εἶπαν, οἵδε ἡμεῖς σοὶ οἰκέται.
\VS{19}Καὶ εἶπεν αὐτοῖς Ἰωσὴφ, μὴ φοβεῖσθε, τοῦ γὰρ Θεοῦ εἰμι ἐγώ.
\VS{20}Ὑμεῖς ἐβουλεύσασθε κατʼ ἐμοῦ εἰς πονηρὰ, ὁ δὲ Θεὸς ἐβουλεύσατο περὶ ἐμοῦ εἰς ἀγαθὰ, ὅπως ἂν γενηθῇ ὡς σήμερον, καὶ τραφῇ λαὸς πολύς.
\VS{21}Καὶ εἶπεν αὐτοῖς, μὴ φοβεῖσθε· ἐγὼ διαθρέψω ὑμᾶς, καὶ τὰς οἰκίας ὑμῶν· καὶ παρεκάλεσεν αὐτοὺς, καὶ ἐλάλησεν αὐτῶν εἰς τὴν καρδίαν.
\VS{22}Καὶ κατῴκησεν Ἰωσὴφ ἐν Αἰγύπτῳ, αὐτὸς καὶ οἱ ἀδελφοὶ αὐτοῦ, καὶ πᾶσα ἡ πανοικία τοῦ πατρὸς αὐτοῦ· καὶ ἔζησεν Ἰωσὴφ ἔτη ἑκατὸν δέκα.
\VS{23}Καὶ εἶδεν Ἰωσὴφ Ἐφραὶμ παιδία, ἕως τρίτης γενεᾶς· καὶ οἱ υἱοὶ Μαχεὶρ τοῦ υἱοῦ Μανασσῆ ἐτέχθησαν ἐπὶ μηρῶν Ἰωσήφ.
\VS{24}Καὶ εἶπεν Ἰωσὴφ τοῖς ἀδελφοῖς αὐτοῦ, λέγων, ἐγὼ ἀποθνήσκω· ἐπισκοπῇ δὲ ἐπισκέψεται ὁ Θεὸς ὑμᾶς, καὶ ἀνάξει ὑμᾶς ἐκ τῆς γῆς ταύτης εἰς τὴν γῆν, ἣν ὤμοσεν ὁ Θεὸς τοῖς πατράσιν ἡμῶν, Ἁβραὰμ, Ἰσαὰκ, καὶ Ἰακώβ.
\VS{25}Καὶ ὥρκισεν Ἰωσὴφ τοὺς υἱοὺς Ἰσραὴλ, λέγων, ἐν τῇ ἐπισκοπῇ ᾗ ἐπισκέψηται ὁ Θεὸς ὑμᾶς, καὶ συνανοίσετε τὰ ὀστᾶ μου ἐντεῦθεν μεθʼ ὑμῶν.
\VS{26}Καὶ ἐτελεύτησεν Ἰωσὴφ ἐτῶν ἑκατὸν δέκα· καὶ ἔθαψαν αὐτὸν, καὶ ἔθηκαν ἐν τῇ σορῷ ἐν Αἰγύπτῳ.
\par }