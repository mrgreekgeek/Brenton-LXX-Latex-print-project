\NormalFont\ShortTitle{ΜΑΚΚΑΒΑΙΩΝ Δʹ}
{\MT ΜΑΚΚΑΒΑΙΩΝ Δʹ

\par }\ChapOne{1}{\PP \VerseOne{1}ΦΙΛΟΣΟΦΩΤΑΤΟΝ λόγον ἐπιδείκνυσθαι μέλλων, εἰ αὐτοδέσποτός ἐστιν τῶν παθῶν ὁ εὐσεβὴς λογισμός· συμβουλεύσαιμʼ ἂν ὑμῖν ὀρθῶς, ὅπως προθύμως προσέχητε τῇ φιλοσοφίᾳ.
\VS{2}Καὶ γὰρ ἀναγκαῖος εἰς ἐπιστήμην παντὶ ὁ λόγος, καὶ ἄλλως τῆς μεγίστης ἀρετῆς ἀρετῆς, λέγω δὴ φρονήσεως, περιέχει ἔπαινον·
\par }{\PP \VS{3}Εἰ ἄρα τῶν σωφροσύνης κωλυτικῶν παθῶν ὁ λογισμὸς φαίνεται ἐπικρατεῖν, γαστριμαργίας τε καὶ ἐπιθυμίας·
\VS{4}ἀλλὰ καὶ τῶν τῆς δικαιοσύνης ἐμποδιστικῶν παθῶν κυριεύειν ἀναφαίνεται, οἷον κακοηθείας· καὶ τῶν τῆς ἀνδρείας ἐμποδιστικῶν παθῶν, θυμοῦ τε, καὶ πόνου καὶ φόβου.
\VS{5}Πῶς οὖν, ἴσως εἴποιεν ἄν τινες, εἰ τῶν παθῶν ὁ λογισμὸς κρατεῖ, λήθης καὶ ἀγνοίας οὐ δεσπόζει;
\VS{6}γελοῖον ἐπιχειροῦντες λέγειν· οὐ γὰρ τῶν ἑαυτοῦ παθῶν ὁ λογισμὸς κρατεῖ, ἀλλὰ τῶν τῆς δικαιοσύνης καὶ ἀνδρείας καὶ σωφροσύνης, καὶ φρονήσεως ἐναντίων· καὶ τούτων, οὐχ ὥστε αὐτὰ καταλῦσαι, ἀλλʼ ὥστε αὐτοῖς μὴ εἶξαι.
\par }{\PP \VS{7}Πολλαχόθεν μὲν οὖν καὶ ἀλλαχόθεν ἔχοιμʼ ἂν ὑμῖν ἐπιδεῖξαι, ὅτι αὐτοκράτωρ ἐστὶν τῶν παθῶν ὁ εὐσεβὴς λογισμός.
\VS{8}Πολὺ δὲ πλέον τοῦτο ἀποδείξαιμι ἀπὸ τῆς ἀνδραγαθείας τῶν ὑπὲρ ἀρετὴν ἀποθανόντων, Ἐλεαζάρου τε καὶ ἑπτὰ ἀδελφῶν καὶ τῆς τούτων μητρός.
\VS{9}Ἅπαντες γὰρ οὗτοι τῶν ἕως θανάτου πόνων ὑπεριδοντες, ὑπεριδόντες ἐπεδείξαντο ὅτι περικρατεῖ τῶν παθῶν ὁ λογισμός.
\par }{\PP \VS{10}Τῶν μὲν οὖν ἀρετῶν, ἔπεστί μοι ἐπαινεῖν τοὺς κατὰ τοῦτον τὸν καιρὸν ὑπὲρ τῆς καλοκᾳγαθίας ἀποθανόντας μετὰ τῆς μητρὸς ἄνδρας·
\VS{11}τῶν δὲ τιμῶν μακαρίσαιμʼ ἄν· θαυμασθέντες γὰρ ἐκεῖνοι οὐ μόνον ὑπὸ πάντων ἀνθρώπων ἐπὶ τῇ ἀνδρείᾳ καὶ τῇ ὑπομονῇ, ἀλλὰ καὶ ὑπὸ τῶν αἰκισαμένων, αἴτιοι κατέστησαν τοῦ καταλυθῆναι τὴν κατὰ τοῦ ἔθνους τυραννίδα, νικήσαντες τὸν τύραννον τῇ ὑπομονῇ, ὥστε διʼ αὐτῶν καθαρισθῆναι τὴν πατρίδα.
\par }{\PP \VS{12}Ἀλλὰ καὶ περὶ τούτου νῦν αὐτίκα δὴ λέγειν ἐξέσται, ἀρξαμένων τῆς ὑποθέσεως, ὥσπερ εἴωθα ποιεῖν, καὶ οὕτως εἰς τὸν περὶ αὐτῶν τρέψομαι λόγον, δόξαν διδοὺς τῷ πανσόφῳ Θεῷ.
\par }{\PP \VS{13}Ζητοῦμεν δὴ τοίνυν, εἰ αὐτοκράτωρ ἐστὶν παθῶν ὁ λογισμός.
\VS{14}Διακρίνωμεν δὲ, τί ποτέ ἐστιν λογισμός; καὶ τί πάθος; καὶ πόσαι παθῶν ἰδέαι; καὶ εἰ πάντων ἐπικρατεῖ τούτων ὁ λογισμός;
\par }{\PP \VS{15}Λογισμὸς μὲν δὴ τοίνυν ἐστὶν νοῦς μετὰ ὀρθοῦς βίου· πρωτιμῶν τὸν σοφίας λόγον.
\VS{16}Σοφία δὴ τοὶνυν ἐστὶν γνῶσις θείων καὶ ἀνθρωπίνων πραγμάτων, καὶ τῶν τούτων αἰτίων.
\VS{17}Αὕτη δὴ τοίνυν ἐστὶν ἠ τοῦ νόμου παιδσεία· διʼ ἧς τὰ θεῖα σεμνῶς, καὶ τὰ ἀνθρώπινα συμφερόντως μανθάνομεν.
\par }{\PP \VS{18}Τῆς δὲ σοφίας ἰδέαι καθεστᾶσιν, φρόνησις καὶ δικαιοσύνη καὶ ἀνδρεια καὶ σωφροσύνη.
\VS{19}Κυριωτάτη πάντων ἡ φρόνησις· ἐξ ἧς δὴ τῶν παθῶν ὁ λογισμὸς ἐπικρατεῖ.
\VS{20}Παθῶν δὲ φύσεις εἰσὶν αἱ περιεκτικώταται δύο, ἡδονή τε καὶ πόνος· τούτων δὲ ἐκάτερον καὶ περὶ τὴν ψυχὴν πέφυκεν.
\VS{21}Πολλαὶ δὲ καὶ περὶ τὴν ἡδονὴν καὶ τὸν πόνον παθῶν εἰσὶν ἀκολουθίαι.
\VS{22}Πρὸ μὲν οὖν τῆς ἡδονῆς ἐστιν ἐπιθυμία· μετὰ δὲ τὴν ἡδονὴν, χαρά.
\VS{23}Πρὸ δὲ τοῦ πόνου ἐστὶν φόβος· μετὰ δὲ τὸν πόνον, λύπη.
\par }{\PP \VS{24}Θυμὸς δὲ κοινὸν πάθος ἐστὶν ἡδονῆς καὶ πόνου, ἐὰν ἐννοηθῇ τις ὃτε αὐτῷ περιέπεσεν.
\VS{25}Ἐν δὲ τῇ ἡδονῇ ἐστιν καὶ ἡ κακοήθης διάθεσις, πολυτροπωτάτη πάντων τῶν παθῶν οὖσα.
\VS{26}Κατὰ μὲν ψυχῆς ἀλαζονεία, καὶ φιλαργυρία, καὶ φιλοδοξία, καὶ φιλονεικία, ἀπιστία καὶ βασκανία·
\VS{27}κατὰ δὲ τὸ σῶμα, παντοφαγία, καὶ λαιμαργία, καὶ νομοφαγία.
\par }{\PP \VS{28}Καθάπερ οὖν δυοῖν τοῦ σώματος καὶ τῆς ψυχῆς φυτῶν ὄντων ἡδονῆς τε καὶ πὸνου, πολλαὶ τούτων τῶν παθῶν εἰσιν παραφυάδες.
\VS{29}Ὧν ἕκαστος ὁ πανγέωργος λογισμὸς περικαθαίρων τε καὶ ἀποκνίζων, καὶ περιπλέκων, καὶ ἐπάρδων, καὶ πάντα τρόπον μεταχέων, ἐξημεροῖ τὰς τῶν ἠθῶν καὶ παθῶν ὕλας.
\VS{30}Ὁ γὰρ λογισμὸς τῶν μὲν ἀρετῶν ἐστιν ἡγεμῶν, τῶν δὲ παθῶν αὐτοκράτωρ. Ἐπιθεώρει γε τοίνυν πρῶτον διʼ αὐτῶν κωλυτικῶν τῆς σωφροσύνης ἔργων, ὅτι αὐτοδέσποτός ἐστιν τῶν παθῶν ὁ λογισμός.
\par }{\PP \VS{31}Σωφροσύνη δὴ τοίνυν ἐστὶν ἐπικράτεια τῶν ἐπιθυμιῶν.
\VS{32}Τῶν δὲ ἐπιθυμιῶν αἱ μέν εἰσιν ψυχικαὶ, αἱ δὲ σωματικαί· καὶ τούτων ἀμφοτέρων ὁ λογισμὸς ἐπικρατεῖν φαίνεται.
\VS{33}Ἐπεὶ πόθεν κινούμενοι πρὸς τὰς ἀπειρημένας τοοφὰς, ἀποτρεπόμεθα τὰς ἐξ ἑαυτῶν ἡδονάς; οὐχ ὅτι δύναται τῶν ὀρὲξεων ἐπικρατεῖν ὁ λογισμός; ἐγὼ μὲν οἶμαι.
\VS{34}Τοιγαροῦν ἐνύδρων ἐπιθυμοῦντες καὶ ὀρνέων καὶ τετραπόδων, παντοίων βρωμάτων τὼν ἀπηγορευμένων ἡμῖν κατὰ τὸν νόμον ἀπεχόμεθα διὰ τὴν τοῦ λογισμοῦ ἐπικράτειαν.
\VS{35}Ἀντέχεται γὰρ τὰ τῶν ὀρέξεων πάθη ὑπὸ τοῦ σώφρονος νοὸς ἀνακαμπτόμενα· καὶ φιλοτιμοῦνται πάντα τὰ τοῦ σώματος κινήματα ὑπὸ τοῦ λογισμοῦ.

\par }\Chap{2}{\PP \VerseOne{1}Καὶ τὶ θαυμαστὸν; εἰ αἱ τῆς ψυχῆς ἐπιθυμίαι πρὸς τὴν τοῦ κάλλους μετουσίαν ἀκυροῦνται.
\VS{2}Ταύτῃ γοῦν ὁ σώφρων Ἰωσὴφ ἐπαινεῖται, ὅτι τῷ λογισμῷ, διανοίᾳ περιεκράτησεν τῆς ἠδυπαθείας.
\VS{3}Νέος γὰρ ὢν καὶ ἀκμάζων πρὸς συνουσιασμὸν ἠκύρωσεν τῷ λογισμῷ τὸν τῶν παθῶν οἶστρον.
\par }{\PP \VS{4}Οὐ μόνον δὲ τὴν τῆς ἡδυπαθείας οἰστρηλασίαν ἐπικρατεῖν ὁ λογισμὸς φαίνεται, ἀλλὰ καὶ πάσης ἐπιθυμίας.
\VS{5}Λέγει γοῦν ὁ νόμος· οὐκ ἐπιθυμήσεις τὴν γυναῖκα τοῦ πλησίον σου, οὐδὲ ὅσα τῷ πλησίον σου ἐστίν.
\VS{6}Καίτοι ὅτε μὴ ἐπιθυμεῖν εἴρηκεν ἡμᾶς ὁ νόμος, πολὺ πλέον πείσαιμʼ ἂν ὑμᾶς, ὅτι τῶν ἐπιθυμιῶν κρατεῖν δύναται ὁ λογισμὸς, ὥσπερ καὶ τῶν κωλυτικῶν τῆς δικαιοσύνης παθῶν.
\VS{7}Ἐπεὶ τίνα τρόπον μονοφάγος τις ὢν τὸ ἦθος, καὶ γαστρίμαργος, καὶ μέθυσος, μεταπαιδεύεται, εἰ μὴ δῆλον, ὅτι κύριός ἐστιν τῶν παθῶν ὁ λογισμός;
\par }{\PP \VS{8}Αὐτίκα γοῦν τῷ νόμῳ πολιτευόμενος, κᾂν φιλάργυρός τις εἴη, βιάζεται τὸν ἑαυτοῦ τρόπον, τοῖς δεομένοις δανείζων χωρὶς τόκων, καὶ τὸ δάνειον τῶν ἑβδομάδων ἐντάσσων χρεοκοπούμενος.
\VS{9}Κᾂν φειδωλός τις ᾖ, ὑπὸ τοῦ νόμου κρατεῖται διὰ τὸν λογισμὸν, μήτε ἐπικαρπούμενος τοὺς ἀμητοὺς, μήτε ἐπιῤῥωγολογούμενος τοὺς ἀμπελῶνας, καὶ ἐπὶ τῶν ἐτέρων ἔστιν ἐπιγνῶναι τοῦτο, ὅτι τῶν παθῶν ἐστιν ὁ λογισμὸς κρατῶν.
\par }{\PP \VS{10}Ὁ γὰρ νόμος καὶ τῆς πρὸς γονεῖς εὐνοίας κρατεῖ, μὴ καταπροδιδοὺς τὴν ἀρετὴν διʼ αὐτούς·
\VS{11}καὶ τῆς προσγαμετῆς φιλίας ἐπικρατεῖ, διὰ παρανομίαν αὐτὴν ἀπελέγχων.
\VS{12}Καὶ τῆς τέκνων φιλίας κυριεύει, διὰ κακίαν αὐτῶν κολάζων, καὶ τῆς φίλων συνηθείας δεσπόζει, διὰ πονηρίας αὐτοὺς ἐξελέγχων.
\VS{13}Καὶ μὴ νομίσητε παράδοξον εἶναι, ὅπου καὶ ἔχθραν ὁ λογισμὸς ἐπινομίσητε παράδοξον εἶναι, ὅπου καὶ ἔχθραν ὁ λογισμὸς ἐπινομισητε παράδοξον εἶναι, ὅπου καὶ ἔχθραν ὁ λογισμὸς ἐπικρατεῖν δύναται διὰ τὸν νόμον,
\VS{14}μηδὲ δενδροτομῶν τὰ ἥμερα τῶν πολεμίων φυτὰ, τὰ δὲ τῶν ἐχθρῶν τοῖς ἀπολέσασιν διασώζων, καὶ τὰ πεπτωκότα συνεγείρων.
\par }{\PP \VS{15}Καὶ τῶν βιοτέρων δὲ παθῶν κρατεῖν ὁ λογισμὸς φαίνεται, φιλαρχίας, καὶ κενοδοξίας, καὶ ἀλαζονείας, καὶ μεγαλαυχίας, καὶ βασκανίας.
\VS{16}Πάντα γὰρ ταῦτα τὰ κακοήθη πάθη ὁ σώφρων νοῦς ἀπωθεῖται, ὥσπερ καὶ τὸν θυμόν· καὶ γὰρ τοῦτο δεσπόζει.
\par }{\PP \VS{17}Θυμούμενος γέ τοι Μωσῆς κατὰ Δαθὰν καὶ Ἀβειρῶν, οὐ θυμῷ τι κατʼ αὐτῶν ἐποίησεν, ἀλλὰ λογισμῷ τὸν θυμὸν διῄτησεν.
\VS{18}Δυνατὸς γὰρ ὁ σώφρων νοῦς, ὡς ἔφην, κατὰ τῶν παθῶι ἀριστεῦσαι, καὶ τὰ μὲν αὐτῶν μεταθεῖναι, τὰ δὲ καὶ ἀκυρῶσαι.
\VS{19}Ἐπεὶ διατί ὁ πάνσοφος ἡμῶν πατὴρ Ἰακὼβ τοὺς περὶ Συμεὼν καὶ Λευὶν αἰτιᾶται, μὴ λογισμῷ τοὺς Σικιμίτας ἐθνηδὸν ἀποσφάξαντας, λέγων, ἐπικατάρατος ὁ θυμὸς αὐτῶν;
\VS{20}Εἰ μὴ γὰρ ἐδύνετο τῶν θυμῶν ὁ λογισμὸς κρατεῖν, οὐκ ἂν εἶπεν οὑτως.
\par }{\PP \VS{21}Ὁπηνίκα γὰρ ὁ Θεὸς τὸν ἄνθρωπον κατεσκεύαζεν, τὰ πάθη αὐτοῦ καὶ τὰ ἤθη περιεφύτευσεν.
\VS{22}Καὶ τηνικαῦτα δὲ περὶ πάντων τὸν ἱερὸν ἡγεμόνα νοῦν διὰ τῶν αἰσθητηρίων ἐνεθρόνισεν·
\VS{23}καὶ τούτῳ νόμον ἔδωκεν, καθʼ ὃν πολιτευόμενος βασιλεύσει βασιλείαν σώφρονά τε, καὶ δικαίαν, καὶ ἀγαθὴν, καὶ ἀνδρείαν.
\VS{24}Πῶς οὖν, εἴποι τις ἂν, εἰ τῶν παθῶν ὁ λογισμὸς κρατεῖ, λήθης καὶ ἀγνοίας οὐ κρατεῖ;

\par }\Chap{3}{\PP \VerseOne{1}Ἐστὶ δὲ κομιδῆ γελοῖος ὁ λογισμός οὐ γὰρ τῶν ἑαυτοῦ παθῶν ὁ λογισμὸς ἐπικρατεῖν φαίνεται, ἀλλὰ τῶν σωματικῶν.
\VS{2}Οἷον ἐπιθυμίαν τις ὑμῶν οὐ δύναται ἐκκόψαι, ἀλλὰ μὴ δουλωθῆναι τῇ ἐπιθυμίᾳ δύναται ὁ λογισμὸς παρασχέσθαι.
\par }{\PP \VS{3}Θυμόν τις οὐ δύναται ἐκκόψαι ἡμῶν τῆς ψυχῆς, ἀλλὰ τῷ θυμῷ δυνατὸν βοηθῆσαι.
\VS{4}Κακοήθειάν τις ὑμῶν οὐ δύναται ἐκκόψαι, ἀλλὰ τὸ μὴ καμφθῆναι τῇ κακοηθείᾀ δυνατὸν ὁ λογισμὸς συμμαχῆσαι.
\VS{5}Οὐ γὰρ ἐκριζωτὴς τῶν παθῶν ὁ λογισμός ἐστιν, ἀλλʼ ἀνταγωνιστής.
\VS{6}Ἔστιν γοῦν τοῦτο διὰ τῆς Δαυεὶδ τοῦ βασιλέως δίψης σαφέστερον ἐπιλογίσασθαι.
\VS{7}Ἐπεὶ γὰρ διʼ ὅλης ἡμέρας προσβαλὼν τοῖς ἀλλοφύλοις ὁ Δαυὶδ, πολλοὺς αὐτῶν ἀπέκτεινεν μετὰ τῶν τοῦ ἔθνους στρατιωτῶν·
\VS{8}τότε δὲ γενομένης ἑσπέρας, ὑδρῶν καὶ σφόδρα κεκμηκὼς, ἐπὶ τὴν βασίλειον σκηνὴν ἦλθεν, περὶ ἣν ὁ πᾶς τῶν προγόνων στρατὸς ἐστρατοπέδευκεν.
\par }{\PP \VS{9}Οἱ μὲν οὖν ἄλλοι πάντες ἐπὶ τὸ δεῖπνον ἦσαν.
\VS{10}Ὁ δὲ βασιλεὺς ὡς μάλιστα διψῶν, καίπερ ἀφθόνους ἔχων πηγὰς, οὐκ ἠδύνατο διʼ αὐτῶν ἰάσασθαι τὴν δίψαν·
\VS{11}ἀλλά τις αὐτὸν ἀλόγιστος ἐπιθυμία τοῦ παρὰ τοῖς πολεμίοις ὕδατος ἐπιτείνουσα συνέφρυγεν, καὶ λύουσα κατέφλεγεν.
\par }{\PP \VS{12}Ὅθεν τῶν ὑπερασπιστῶν ἐπὶ τῇ τοῦ βασιλέως ἐπιθυμία σχετλιαζόντων, δύο νεανίσκοι στρατιῶται καρτεροὶ καταιδεσθέντες τὴν τοῦ βασιλέως ἐπιθυμίαν, τὰς πανοπλίας καθωπλίσαντο, καὶ κάλπην λαβόντες ὑπερέβησαν τοὺς τῶν πολεμίων χάρακας·
\VS{13}καὶ λαθόντες τοὺς τῶν πυλῶν ἀκροφύλακας, διεξῄεσαν εὑράμενοι κατὰ πᾶν τὸ τῶν πολεμίων στρατόπεδον.
\VS{14}Καὶ ἀνευράμενοι θαῤῥαλέως τὴν πηγὴν, ἐξ αὐτῆς ἐγέμισαν τῷ βασιλεῖ τὸ ποτόν.
\par }{\PP \VS{15}Ὁ δὲ καὶ περὶ τὴν δίψαν διαπυρούμενος, ἐλογίσατο πάνδεινον εἶναι κίνδυνον τῇ ψυχῇ λογισθὲν ἰσοδύναμον τὸ ποτὸν αἵματι.
\VS{16}Ὅθεν ἀντιθεὶς τῇ ἐπιθυμίᾳ τὸν λογισμὸν, ἔσπεισεν τὸ πόμα τῷ Θεῷ.
\VS{17}Δυνατὸς γὰρ ὁ σώφρων νοῦς νικῆσαι τὰς τῶν παθῶν ἀνάγκας, καὶ σβέσαι τὰς τῶν οἴστρων φλεγμονὰς, καὶ τὰς τῶν σωμάτων ἀλγηδόνας καθʼ ὑπερβολὴν οὔσας καταπαλαῖσαι,
\VS{18}καὶ τῆς καλοκᾳγαθίας τοῦ λογισμοῦ ἀποπτῦσαι πάσας τὰς τῶν παθῶν ἐπικρατείας.
\par }{\PP \VS{19}Ἤδη δὲ καὶ ὁ καιρὸς ἡμᾶς καλεῖ ἐπὶ τὴν ἀπόδειξιν τῆς ἱστορίας τοῦ σώφρονος λογισμοῦ.
\VS{20}Ἐπειδὴ γὰρ βαθεῖαν εἰρήνην διὰ τὴν εὐνομίαν οἱ πατέρες ἡμῶν εἶχον, καὶ ἔπραττον καλῶς, ὥστε καὶ τὸν τῆς Ἀσίας βασιλέα Σέλευκον τὸν Νικάνορα καὶ χρήματα εἰς τὴν ἱερουργίαν αὐτοῖς ἀποφορίσαι, καὶ τὴν πολιτείαν αὐτῶν ἀποδέχεσθαι·
\VS{21}τότε δή τινες πρὸς τὴν κοινὴν νεωτερίσαντες ὁμόνοιαν, πυλυτρόπως ἐχρήσαντο συμφοραῖς.

\par }\Chap{4}{\PP \VerseOne{1}Σίμων γάρ τις πρὸς Ὀνίαν ἀντιπολιτεύομενος τόν ποτε τὴν ἀρχιερωσύνην ἔχοντα διὰ βίου, καλὸν καὶ ἀγαθὸν ἄνδρα, ἐπειδὴ πάντα τρόπον διαβάλλων ὑπὲρ τοῦ ἔθνους οὐκ ἰσχυσεν κακῶσαι, φυγὰς ᾤχετο, τὴν πατρίδα προδώσων.
\par }{\PP \VS{2}Ὅθεν ἥκων πρὸς Ἀπολλώνιος, τὸν Συρίας τε καὶ Φοινίκης καὶ Κιλικίας στρατηγὸν, ἔλεγεν,
\VS{3}εὔνους ὢν τοῖς τοῦ βασιλέως πράγμασιν ἥκω, μηνύων πολλὰς ἰδιωτικῶν χρημάτων μυριάδας ἐν τοῖς Ἱεροσολύμων γαζοφυλακίοις τεθησαύρισται, τῷ ἱερῷ μὴ ἐπικοινωνούσας, ἀλλὰ προσήκειν ταῦτα Σελεύκῳ τῷ βασιλεῖ.
\par }{\PP \VS{4}Τούτων ἕκαστα γνοὺς ὁ Ἀπολλώνιος, τὸν μὲν Σίμωνα τῆς εἰς τὸν βασιλέα κηδεμονίας ἐπαινεῖ, πρὸς δὲ τὸν Σέλευκον ἀναβὰς κατεμήνυε τὸν τῶν χρημάτων θησαυρόν·
\VS{5}καὶ λαβὼν τὴν περὶ αὐτῶν ἐξουσίαν, ταχὺ εἰς τὴν πατρίδα ἡμῶν μετὰ τοῦ καταράτου Σίμωνος καὶ βαρυτάτου στρατοῦ προσελθὼν,
\VS{6}ταῖς τοῦ βασιλέως ἐντολαῖς ἥκειν ἔλεγεν, ὅπως τὰ ἰδιωτικὰ τοῦ γαζοφυλακίου λάβοι χρήματα.
\VS{7}Καὶ τοῦ ἔθνους πρὸς τὸν λόγον σχετλιάζοντος, ἀντιλέγοντός τε, πάνδεινον εἶναι νομίσαντες, εἰ οἱ τὰς παρακαταθήκας πιστεύσαντας τῷ ἱερῷ θησαυρῷ στερηθήσονται, ὡς οἷόν τε ἦν ἐκώλυον.
\VS{8}Μετὰ ἀπειλῆς δὲ ὁ Ἀπολλώνιος ἀπῄει εἰς τὸ ἱερόν.
\par }{\PP \VS{9}Τῶν δὲ ἱερέων μετὰ γυναικῶν καὶ παιδίων ἐν τῷ ἱερῷ ἱκετευσάντων τὸν Θεὸν ὑπερασπίσαι τοῦ ἱεροῦ καταφρονουμένου τόπου.
\VS{10}Ἀνιόντος τε μετὰ καθωπλισμένης τῆς στρατιᾶς τοῦ Ἀπολλωνίου πρὸς τὴν τῶν χρημάτων ἀρπαγὴν οὐρανόθεν ἔφιπποι προϋφάνησαν ἄγγελοι περιαστράπτοντες τοῖς ὅπλοις, καὶ πολὺν αὐτοῖς φόβον τε καὶ τρόμον ἐνιόντες.
\VS{11}Καταπεσὼν γέ τοι ἡμιθανὴς ὁ Ἀπολλώνιος ἐπὶ τὸν πάμφυλον τοῦ ἱεροῦ περίβολον, τὰς χεῖρας ἐξέτεινεν εἰς τὸν οὐρανὸν, μετὰ διακρύων τοὺς Ἑβραίους παρεκάλει, ὅπως περὶ αὐτοῦ εὐξόμενοι, τὸν ἐπουράνιον ἐξευμενίσωνται στρατόν.
\VS{12}Ἔλεγεν γὰρ ἡμαρτηκὼς, ὥστε καὶ ἀποθανεῖν ἄξιος ὑπάρχειν, πᾶσίν τε ἀνθρώποις ὑμνήσειν σωθεὶς τὴν τοῦ ἱεροῦ τόπου μακαριότητα.
\par }{\PP \VS{13}Τούτοις ἐπαχθεὶς τοῖς λόγοις Ὀνίας ὁ ἀρχιερεὺς, καίπερ ἄλλως εὐλαβηθεὶς, μή ποτε νομίσειεν ὁ βασιλεὺς Σέλευκος ἐξ ἀνθρωπίνης ἐπιβουλῆς καὶ μὴ θείας δίκης ἀνῃρήσασθαι τὸν Ἀπωλλώνιον, ηὔξατο περὶ αὐτοῦ.
\VS{14}Καὶ ὁ μὲν παραδὸξως διασωθεὶς ᾤχετο, δηλώσων τῷ βασιλεῖ τὰ συμβάντα αὐτῷ.
\par }{\PP \VS{15}Τελευτήσαντος δὲ Σελεύκου τοῦ βασιλέως διαδέχεται τὴν ἀρχὴν ὁ υἱὸς αὐτοῦ Ἀντίοχος Ἐπιφανὴς, ἀνὴρ ὑπερήφανος καὶ δεινὸς.
\VS{16}Ὃς καταλύσας τὸν Ὀνίαν τῆς ἀρχιερωσύνης,
\VS{17}Ἰάσονα τὸν ἀδελφὸν αὐτοῦ κατέστησεν ἀρχιερέα, συνθέμενον δώσειν, εἰ ἐπιτρέψειεν αὐτῷ τὴν ἀρχὴν, κατʼ ἐνιαυτὸν τρισχίλια ἐξακόσια ἑξήκοντα τάλαντα.
\par }{\PP \VS{18}Ὁ δὲ ἐπέτρεψεν αὐτῷ ἀρχιερᾶσθαι καὶ τοῦ ἔθνους ἀφηγεῖσθαι.
\VS{19}Ὃς καὶ ἐξεζήτησεν τὸ ἔθνος, καὶ ἐξεπολίτευσεν ἐπὶ πᾶσαν παρανομίαν.
\VS{20}Ὥστε μὴ μόνον ἐπʼ αὐτῇ τῆ ἄκρᾳ τῆς πατρίδος ἡμῶν γυμνάσιον κατασκευάσαι, τὴν τοῦ ἱεροῦ κηδεμονίαν.
\VS{21}Ἐφʼ οἷς ἀγανακτήσασα ἡ θεία δίκη αὐτόν τοι τὸν Ἀντίοχον ἐπολέμησεν.
\VS{22}Ἐπειδὴ γὰρ πολεμῶν ἦν κατʼ Αἴγυπ τον Πτολεμαίῳ, ἤκουσέν τε, ὅτι φήμης διαδοθείσης περὶ τιῦ τεθνάναι αὐτὸν, ὡς ἔνι μάλιστα χαίροιεν οἱ Ἱεροσολυμῖται, ταχέως ἐπʼ αὐτοὺς ἀνέζευξεν.
\VS{23}Καὶ ὡς ἐπόρθσεν αὐτοὺς, δόγμα ἔθετο, ὅπως εἴ τινες αὐτῶν φάνοιεν τῷ πατρίῳ πολιτευόμενοι νόμῳ θάνοιεν.
\par }{\PP \VS{24}Καὶ ἐπεὶ κατὰ μηδένα τρόπον ἴσχυεν καταλῦσαι διὰ τῶν δογμάτων τὴν τοῦ ἔθνους εὔνοιαν, ἀλλὰ πάσας τὰς ἑαυτοῦ ἀπειλὰς καὶ τιμωρίας ἑώρα καταλυομένας,
\VS{25}ὥστε καὶ γυναῖκας, ὅτι περιέτεμον τὰ παιδία, μετὰ τῶν βρεφῶν κατακρημνισθῆναι, προειδυίας ὅτι τοῦτο πείσονται·
\VS{26}ἐπεὶ οὖν τὰ δόγματα αὑτοῦ κατεφρονεῖτο ὑπὸ τοῦ λαοῦ, αὐτὸς διὰ βασάνων ἕνα ἕκαστον τούτου ἔθνους ἠνάγκαζεν μικρῶν ἀπογευομένους τροφῶν, ἐξόμνυσθαι τὸν Ἰουδαϊσμόν.

\par }\Chap{5}{\PP \VerseOne{1}Προκαθίσας γέ τοι μετὰ τῶν συνέδρων ὁ τύραννος Ἀντίοχος ἐπί τινος ὑψηλοῦ τόπου,
\VS{2}καὶ τῶν στρατευμάτων αὐτῶν ἐνόπλων κυκλόθεν παρεστηκότων παρεκέλευεν τοῖς δορυφόροις ἕνα ἕκαστον τῶν Ἑβραίων περισπᾶσθαι καὶ κρεῶν ὑείων καὶ εἰδωλοθύτων ἀναγκάζειν ἀπογεύεσθαι.
\VS{3}Εἰ δὲ τινες μὴ θέλοιεν μιαροφαγῆσαι, τούτους τροχισθέντας ἀναιρεθῆναι.
\par }{\PP \VS{4}Πολλῶν δὲ συναρπασθέντων, εἷς πρῶτος ἐκ τῆς ἀγέλης Ἑβραῖος ὀνόματι Ἐλεάζαρος, τὸ γένος ἱερεὺς, τὴν ἐπιστήμην νομικὸς, καὶ τὴν ἡλικίαν προήκων, καὶ πολλοῖς τῶν περὶ τὸν τύραννον διὰ τὴν ἡλικίαν γνώριμος, παρήχθη πλησίον αὐτοῦ.
\par }{\PP \VS{5}Καὶ αὐτὸν ἰδὼν ὁ Ἀντίοχος, ἔφη,
\VS{6}ἐγὼ πρὶν ἂρξασθαι τῶν κατὰ σοῦ βασάνων, ὦ πρεσβύτα, συμβουλεύσαιμʼ ἄν σοι ταῦτα ὅπως ἀπογευσάμενος τῶν ὑείων σώζοιο· αἰδοῦμαι γάρ σου τὴν ἡλικίαν καὶ τὴν πολιὰν, ἥν μετὰ τοσοῦτον ἔχων χρόνον, οὔ μοι δοκεῖς φιλοσοφεῖν, τῇ Ἰουδαιων χρώμενος θρησκείᾳ.
\VS{7}Διατί γὰρ τῆς φύσεως κεχαρισμένης καλλίστην τὴν τοῦδε τοῦ ζώου σαρκοφαγίαν βδελύττῃ;
\VS{8}Καὶ γὰρ ἀνόητον τοῦτο τὸ μὴ ἀπολαύειν τῶν χωρὶς ὀνείδους ἡδέων, καὶ διʼ ἄδικον ἀποστρέφεσθαι τὰς τῆς φύσεως χάριτας.
\par }{\PP \VS{9}Σὺ δέ μοι καὶ ἀνοητότερον ποιήσειν δοκεῖς, εἰ κενοδοξῶν περὶ τὸ ἀληθὲς,
\VS{10}ἔτι κᾀμοῦ καταφρονήσεις ἐπὶ τῇ ἰδίᾳ τιμωρίᾳ· οὐκ ἐξυπνώσεις ἀπὸ τῆς φλυάρου φιλοσοφίας ὑμῶν;
\VS{11}Καὶ ἀποσκεδάσεις ψῶν λογισμῶν σου τὸν λῆρον, καὶ ἄξιον τῆς ἡλικίας ἀναλαβῶν νοῦν φιλοσοφήσεις τήν τοῦ συμφέροντος ἀλήθειαν;
\VS{12}καὶ προσκυνήσας μου τὴν φιλάνθρωπον παρηγορίαν οἰκτειρήσεις τὸ σεαυτοῦ γῆρας;
\VS{13}καὶ γὰρ ἐνθυμήθητι, ὡς εἰ καί τις ἐστιν τῆσδε τῆς θρησκείας ἐποπτικὴ δύναμις, συγνωμονήσειν σοι ἐπὶ πᾶσιν διʼ ἀνάγκην παρανομίᾳ γεινομένῃ.
\par }{\PP \VS{14}Τοῦτον τὸν τρόπον ἐπὶ τὴν ἔκθεσμον σαρκοφαγίαν ἐποτρύνοντος τοῦ τυράννου, λόγον ᾔτησεν ὁ Ἐλεάζαρος.
\VS{15}Καὶ λαβὼν τοῦ λέγειν ἐξουσίαν, ἤρξατο δημηγορεῖν οὕτως·
\VS{16}ἡμεῖς, Αντίοχε, θείῳ πεπεισμένοι νόμῳ πολιτευεσθαι, οὐδεμίαν ἀνάγκην βιαιοτέραν εἶναι νομίζομεν τῆς πρὸς τὸν νόμον ἡμῶν εὐπειθείας.
\VS{17}Διὸ δὲ κατʼ οὐδένα τρόπον παρανομεῖν ἀξιοῦμεν.
\VS{18}Καί τοι εἰ καὶ κατὰ ἀλήθειαν μὴ ἦν ὁ νόμος ἡμῶν, ὡς σὺ ὑπολαμβάνεις, θεῖος, (ἄλλως δὲ νομίζομεν αὐτὸν εἶναι θεῖον) οὐδὲ οὕτως ἐξὸν ἡμῖν ἦν τὴν ἐπὶ τῇ εὐσεβείᾳ δόκαν ἀκυρῶσαι.
\VS{19}Μὴ μικρὰν οὖν εἶναι νομίσῃς ταύτην, εἰ μιαροφαγήσεμεν, ἁμαρτίαν.
\VS{20}Τὸ γὰρ ἐν μικροῖς καὶ ἐν μεγάλοις παρανομεῖν ἰσοδύναμόν ἐστιν·
\VS{21}διʼ ἑκατέρου γὰρ ὡς ὁμοίως ὁ νόμος ὑπερηφανεῖται.
\par }{\PP \VS{22}Χλευάζεις δὲ ἡμῶν τὴν φιλοσοφίαν, ὥσπερ οὐ μετὰ εὐλογιστίας ἐν αὐτῇ βιούντων.
\VS{23}Σωφροσύνην τε γὰρ ἡμᾶς ἐκδιδάσκει, ὥστε πασῶν τῶν ἡδονῶν καὶ ἐπιθυμιῶν κρατεῖν, καὶ ἀνδρείαν ἐξασκεῖν, ὥστε πάντα πόνον ἑκουσίως ὑπομένειν·
\VS{24}καὶ δικαιοσύνην παιδεύει, ὥστε διὰ πάντων τῶν ἠθῶν ἰσονομεῖν καὶ εὐσέβειαν διδάσκειν, ὥστε μόνον τὸν ὄντα Θεὸν σέβειν μεγαλοπρεπῶς.
\VS{25}Διὸ οὐ μιαροφαγοῦμεν· πιστεύοντες γὰρ Θεοῦ καθεστᾶναι τὸν νόμον, οἴδαμεν ὅτι καὶ κατὰ φύσιν ἡμῖν συμπαθεῖ νομοθετῶν ὁ τοῦ κόσμου κτίστης·
\VS{26}τὰ μὲν οἰκειωθωσόμενα ἡμῶν ταῖς ψυχαῖς ἐπέτρεψεν ἐσθίειν, τὰ δὲ ἐναντιωθησόμενα ἐκώλυσεν σαρκοφαγεῖν.
\par }{\PP \VS{27}Τυραννικὸν δὲ, οὐ μόνον ἀναγκάζεις ἡμᾶς παρανομεῖν, ἀλλὰ καὶ ἐσθίειν, ὅπως τῇ ἐχθίστῃ ἡμῶν μιαροφαγίᾳ ταύτῃ ἔτι ἐγγελάσῃς.
\VS{28}Ἀλλʼ οὐ γελάσεις κατʼ ἐμοῦ τοῦτον τὸν γέλωτα·
\VS{29}οὔτε τοὺς ἱεροὺς τῶν προγόνων περὶ τοῦ φυλάξαι τὸν νόμον ὅρκους οὐ παρήσω.
\VS{30}Οὐδʼ ἂν ἐκκόψεις μου τὰ ὄμματα, καὶ τὰ σπλάγχνα μου τήξεις.
\VS{31}Οὐχ οὕτως εἰμὶ γέρων ἐγὼ καὶ ἄνανδρος, ὥστε μοι διὰ τὴν εὐσέβειαν μὴ νεάζειν τὸν λογισμόν.
\par }{\PP \VS{32}Πρὸς ταῦτα τροχοὺς εὐτρέπιζε, καὶ τὸ πῦρ ἐκφύσα σφοδρότερον.
\VS{33}Οὐχ οὕτως οἰκτειρήσω τὸ ἐμαυτοῦ γῆρας, ὥστε με διʼ ἐμαυτοῦ τὸν πάτριον καταλῦσαι νόμον.
\VS{34}Οὐ ψεύσομαί σε, παιδευτὰ νόμε, οὐδὲ φεύξομαί σε, φίλη ἐγκράτεια.
\VS{35}Οὐδὲ καταισχυνῶ σε, φιλόσοφε λόγε, οὐδὲ ἐξαρνήσεμαί σε, ἱερωσύνη τιμία, καὶ νομοθεσίας ἐπιστήμη·
\VS{36}οὐδὲ μιανεῖς μου τὸ σεμνὸν γήρηως στόμα, οὐδὲ νομίμου βίου ἡλικίαν.
\par }{\PP \VS{37}Ἁγνόν με οἱ πατέρες προσδέξονται, μὴ φοβηθέντα σου τὰς μέχρι θανάτου ἀνάγκας.
\VS{38}Ἀσεβῶν μὲν γὰρ τυραννήσεις· τῶν δὲ ἐμῶν περὶ τῆς εὐσεβείας λογισμῶν οὔτε λόγοις δεσπόσεις, οὔτε διʼ ἔργων.

\par }\Chap{6}{\PP \VerseOne{1}Τοῦτον τὸν τρόπον ἀντιρητορεύσαντα ταῖς τοῦ τυράννου παρηγορίαις, παραστάντες οἱ δορυφόροι πικρῶς ἔσυραν ἐπὶ τὰ βασανιστήρια τὸν Ἐλεάζαρον.
\VS{2}Καὶ πρῶτον μὲν περιέδυσαν τὸν γηραιὸν ἐκκεκοσμημένον περὶ τὴν εὐσέβειαν εὐσχημοσύνην.
\VS{3}Ἔπειτα περιαγκωνίσαντες ἑκατέρωθεν, μάστιξιν κατῇκιζον·
\VS{4}πείσθητι ταῖς τοῦ βασιλέως ἐντολαῖς, ἑτέρωθεν κήρυκος ἐπιβοῶντος.
\par }{\PP \VS{5}Ὁ δὲ μεγαλόφρων καὶ εὐγενὴς ὡς ἀληθῶς Ἐλεάζαρος, ὥσπερ ἐν ὀνείρω βασανιζόμενος κατʼ οὐδένα τρόπον μετετρέπετο.
\VS{6}Ἀλλὰ ὑψηλοὺς ἀνατείνας εἰς τὸν οὐρανὸν τοὺς ὀφθαλμοὺς, ἀπεξαίνετο ταῖς μάστιξιν τὰς σάρκας ὁ γέρων, καὶ κατεῤῥεῖτο τῷ αἵματι,
\VS{7}καὶ τὰ πλευρὰ κατετιτρώσκετο, καὶ πίπτων εἰς τὸ ἔδαφος, ἀπὸ τοῦ μὴ φέρειν τὸ σῶμα τὰς ἀλγηδόνας, ὀρθὸν εἶχεν καὶ ἀκλινῆ τὸν λογισμόν.
\VS{8}Λὰξ γέ τοι τῶν πικρῶν τις δορυφόρων, εἰς τοὺς κενεῶνας ἐναλλόμενος ἔτυπτεν, ὅπως ἐξανίσταιτο πίπτων.
\par }{\PP \VS{9}Ὁ δὲ ὑπέμενεν τοὺς πόνους, καὶ περιεφρόνει τῆς ἀνάγκης, καὶ διεκαρτέρει τοὺς αἰκισμοὺς,
\VS{10}καὶ καθάπερ γενναῖος ἀθλητὴς τυπτόμενος ἐνίκα τοὺς βασανίζοντας ὁ γέρων.
\VS{11}Ἱδρῶν γέ τοι τὸ πρόσωπον, καὶ ἐπασθμαίνων σφοδρῶς, καὶ ὑπʼ αὐτῶν τῶν βασανιζόντων ἐθαυμάζετο ἐπὶ τῇ εὐτυχίᾳ.
\par }{\PP \VS{12}Ὅθεν τὰ μὲν ἐλεοῦντες τὰ τοῦ γήρως αὐτοῦ, τὰ δὲ ἐν συμπαθείᾳ τῆς συνηθείας ὄντες,
\VS{13}τὰ δὲ ἐν θαυμαστῷ τῆς καρτερίας προσιόντες αὐτῷ τινὲς τῶν τοῦ βασιλέως ἔλεγον,
\VS{14}τί τοῖς κακοῖς τούτοις σεαντὸν ἀλογίστως ἀπολλεῖς,
\VS{15}Ἐλεάζαρ; ἡμεῖς μὲν τῶν ἡψημένων βρωμάτων παραθήσομεν· σὺ δὲ ὑποκρινόμενος τῶν ὑείων ἀπογεύσασθαι, σώθητι.
\par }{\PP \VS{16}Καὶ ὁ Ἐλεάζαρος, ὥσπερ πικρότερον διὰ τῆς συμβουλίας αἰκισθεὶς, ἀνεβόησεν,
\VS{17}μὴ οὕτως κακῶς φρονήσαιμεν οἱ Ἁβραὰμ παῖδες, ὥστε μαλακοψυχήσαντας ἀπρεπὲς ἡμῖν δρᾶμα ὑποκρίνασθαι.
\VS{18}Καὶ γὰρ ἀλόγιστον, εἰ πρὸς ἀλήθειαν ζήσαντες τὸν μέχρι γήρως βίον, καὶ τὴν ἐπʼ αὐτῶν δόξαν νομίμως φυλάσσοντες, νῦν μεταβαλοίμεθα,
\VS{19}καὶ αὐτοὶ μὲν ἡμεῖς γενοίμεθα τοῖς νέοις ἀσεβείας τύπος, ἵνα παράδειγμα γενώμεθα τῆς μιεροφαγίας.
\VS{20}Αἰσχρὸν γὰρ εἰ ἐπιβιώσωμεν ἀλίγον χρόνον,
\VS{21}καὶ τοῦτον καταγελώμενοι πρὸς ἁπάντων ἐπὶ δειλίᾳ· καὶ ὑπὸ μὲν τοῦ τυράννου καταφρονηθῶμεν ὡς ἄνανδροι, τὸν δὲ θεῖον ἡμῶν νόμον μέχρι θανάτου μὴ προασπίσαιμεν.
\VS{22}Πρὸς ταῦτα ὑμεῖς μὲν, ὦ Ἁβραὰμ παῖδες, εὐγενῶς ὑπὲρ τῆς εὐσεβείας τελευτᾶτε.
\VS{23}Οἱ δὲ τοῦ τυράννου δορυφόροι, τί μέλλετε;
\par }{\PP \VS{24}Πρὸς τὰς ἀνάγκας οὕτως μεγαλοφρονοῦντα αὐτὸν ἰδόντες καὶ μηδὲ πρὸς τὸν οἰκτιρμὸν αὐτῶν μεταβαλλόμενον, ἐπὶ πῦρ αὐτὸν ἤγαγον.
\VS{25}Ἔνθα διὰ κακοτέχνων ὀργάνων καταφλέγοντες αὐτὸν ὑπερέπτοσαν, καὶ δυσώδεις χυλοὺς εἰς τοὺς μυκτῆρας αὐτοῦ κατέχεον.
\par }{\PP \VS{26}Ὁ δὲ μέχρι τῶν ὀστέων ἤδη κατακεκαυμένος καὶ μέλλων λιποθυμεῖν, ἀνέτεινεν τὰ ὄμματα πρὸς τὸν Θεὸν, καὶ εἶπεν, σὺ οἶσθα, Θεὲ, παρόν μοι σώζεσθαι,
\VS{27}βασάνοις καυστικαῖς ἀποθνήσκω διὰ τὸν νόμον.
\VS{28}Ἵλεως γενοῦ τῷ ἔθνει σου, ἀρκεσθεὶς τῇ ἡμετέρᾳ περὶ αὐτῶν δίκῃ.
\VS{29}Καθάρσιον αὐτῶν ποίησον τὸ ἐμὸν αἷμα, καὶ ἀντίψυχον αὐτῶν λαβὲ τὴν ἐμὴν ψυχήν.
\VS{30}Καὶ ταῦτα εἰπὼν ὁ ἱερὸς ἀνὴρ εὐγενῶς ταῖς βασάνοις ἐναπέθανεν, καὶ μέχρι τῶν τοῦ θανάτου βασάνων ἀντέστη τῷ λσγισμῷ διὰ τὸν νόμον.
\par }{\PP \VS{31}Ὁμολογουμένως οἶν δεσπότης ἐστὶν τῶν παθῶν ὁ εὐσεβὴς λογισμός.
\VS{32}Εἰ γὰρ τὰ πάθη τοῦ λογισμοῦ κεκρατήκει, τούτοις ἂν ἀπεδόμην τὴν τῆς ἐπικρατείας μαρτυρίαν.
\VS{33}Νυνὶ δὲ τοῦ λογισμοῦ τὰ πάθη νικήσαντος, αὐτῷ προσηκόντως τὴν τῆς ἡγεμονίας προσνέμομεν ἐξουσίαν.
\par }{\PP \VS{34}Καὶ δίκαιόν ἐστιν ὁμολογεῖν ἡμᾶς, τὸ κράτος εἶναι τοῦ λογισμοῦ, ὅπου γε καὶ τῶν ἔξωθεν ἀλγηδόνων ἐπικρατεῖ.
\VS{35}Ἐπεὶ καὶ γελοῖον· καὶ οὐ μόνον τῶν ἀλγηδόνων ἐπιδείκνυμι κεκρατηκέναι τὸν λογισμὸν, ἀλλὰ καὶ τῶν ἡδονῶν κρατεῖν, μηδὲ αὐταῖς ὑπείκειν.

\par }\Chap{7}{\PP \VerseOne{1}Ὥσπερ καὶ ἄριστος κυβερνήτης ὁ τοῦ πατρὸς ἡμῶν Ἐλεαζάρου λογισμὸς, πηδαλιουεχῶν τὴν τῆς εὐσεβείας ναῦν ἐν τῷ τῶν παθῶν πελάγει,
\VS{2}καὶ καταικιζόμενος ταῖς τοῦ τυράννου ἀπειλαῖς, καὶ καταντλούμενος ταῖς τῶν βασάνων τρικυμίαις,
\VS{3}κατʼ οὐδένα τρόπον μετέτρεψεν τοὺς τῆς εὐσεβείας οἴακας, ἕως οὗ ἔπλευσεν ἐπὶ τὸν τῆς θανάτου νίκης λιμένα.
\par }{\PP \VS{4}Οὐχ οὕτως πόλις πολλοῖς καὶ ποικίλοις μηχανήμασιν ἀντέσχεν ποτὲ πολιορκουμένη, ὡς ὁ πανάγιος ἐκεῖνος τὴν ἱερὰν ψυχὴν αἰκισμοῖς τε καὶ στρέβλαις πυρπολούμενος, ἐκίνησεν τοὺς πολιορκοῦντας, διὰ τὸν ὑπερασπίζοντα τῆς εὐσεβείας λογισμόν.
\VS{5}Ὥσπερ γὰρ πρόκρημνον ἄκραν, τὴν ἑαυτοῦ διὰνοιαν ὁ πατὴρ Ἐλεάζαρος ἐκτείνας, περιέκλασεν τοὺς μαινομένους τῶν παθῶν κλύδωνας.
\par }{\PP \VS{6}Ὦ ἄξιε τῆς ἱερωσύνης ἱερεῦ, οὐκ ἐμίανας τοὺς ἱεροὺς ὀδόντας, οὐδὲ τὴν θεοσέβειαν καὶ καθαρισμὸν χωρήσασαν γαστέρα ἐκοινώνησας μιεροφαγίᾳ·
\VS{7}Ὦ σύμφωνε νόμου, καὶ φιλόσοφε θείου βίου.
\VS{8}Τοίουτους δεῖ εἶναι τοὺς δημιουργοῦντας τὸν νόμον ἰδίῳ αἵματι, καὶ γενναίῳ ἱδρῶτι τοῖς μέχρι θανάτου πάθεσιν ὑπερασπίζοντας.
\par }{\PP \VS{9}Σὺ πάτερ, τὴν εὐνομίαν ἡμῶν διὰ τῶν ὑπομονῶν εἰς δόξαν ἐκύρωσας, καὶ τὴν ἁγιαστίαν σεμνολογήσας οὐ κατέλυσας, καὶ διὰ τῶν ἔργων ἐπιστοποίησας τοὺς τῆς φιλοσοφίας λὸγους.
\VS{10}Ὦ βασάνων βιότερε γέρων, πυρὸς εὐτονώτερε πρεσβύτα, καὶ παθῶν μέγιστε βασιλεῦ Ἐλεάζαρ.
\par }{\PP \VS{11}Ὥσπερ γὰρ ὁ πατὴρ Ἀαρὸν τῷ θυμιατηρίῳ κατθωπλισμένος, διὰ τοῦ ἐθνοπλήθου ἐπιτρέχων τὸν ἐμπυριστὴν ἐνίκησεν ἄγγελον.
\VS{12}Οὕτως ὁ Ἀαρωνίδης Ἐλεάζαρος διὰ τοῦ πυρὸς ὑπερτηκόμενος οὐ μετετράπη τὸν λογισμόν.
\VS{13}Καίτοι τὸ θαυμασιώτατον, γέρων ὢν, λελυμένων μὲν ἤδη τῶν τοῦ σώματος πόνων, καὶ περιεχαλασμένων δὲ τῶν σαρκῶν, κεκμηκότων δὲ καὶ τῶν νεύρων, ἀνενέασεν.
\VS{14}Τῷ πνεύματι τοῦ λογισμοῦ, καὶ τῷ Ἰσακείῳ λογισμῷ τὴν πολυκέφαλον στρέβλαν ἠκύρωσεν.
\VS{15}Ὦ μακαρίου γήρως, καὶ σεμνῆς πολιᾶς, καὶ βίου νομίμου, ὃν πιστὴ θανάτου σφραγὶς ἐτελείωσεν.
\VS{16}Εἰ δὲ τοίνυν γέρων τῶν μέχρι θανάτου βασάνων περιεφρόνησεν διʼ εὐσέβειαν, ὁμολογουμένως ἡγεμών ἐστιν τῶν παθῶν ὁ εὐσεβὴς λογισμός.
\par }{\PP \VS{17}Ἴσως δʼ ἂν εἴποιέν τινες, τῶν παθῶν οὐ πάντες περικρατοῦσιν, ὅτι οὐδὲ πάντες φρόνιμον ἔχουσιν τὸν λογισμόν.
\VS{18}Ἀλλʼ ὅσοι εὑσεβείας προνοοῦσιν ἐξ ὅλης καρδίας, οὗτοι μόνοι δύνανται κρατεῖν τῶν τῆς σαρκὸς παθῶν·
\VS{19}οἱ πιστεύοντες, ὅτι Θεῷ οὐκ ἀποθνήσκουσιν, ὥσπερ γὰρ οἱ πατριάρχαι ἡμῶν Ἁβραὰμ, Ἰσαὰκ, Ἰακὼβ, ζῶσι τῷ Θεῷ.
\par }{\PP \VS{20}Οὐδὲν οὖν ἐναντιοῦται τὸ φαίνεσθαί τινας παθοκρατεῖσθαι διὰ τὸν ἀσθενῆ λογισμόν.
\VS{21}Ἐπεὶ τίς πρὸς ὅλον τὸν τῆς φιλοσοφίας κανόνα εὐσεβῶς φιλοσοφῶν, καὶ πεπιστευκὼς Θεῷ,
\VS{22}καὶ εἰδὼς ὅτι διὰ τὴν ἀρετὴν πάντα πόνον ὑπομένειν μακάριόν ἐστιν, οὐκ ἂν περικρατήσειεν τῶν παθῶν διὰ τὴν εὐσέβειαν;
\VS{23}μόνος γὰρ ὁ σοφὸς καὶ σώφρων ἀνδρεῖός ἐστιν τῶν παθῶν κύριος.
\VS{24}Διὰ τοῦτο γέ τοι καὶ μειρακίσκοι τῷ τῆς εὐσεβείας λογισμῷ φιλοσοφουντες χαλεπωτερων βασανιστηρίων ἐπεκράτησαν.
\VS{25}Ἐπειδὴ γὰρ κατὰ τὴν πρώτην πεῖραν ἐνικήθη περιφανὴς ὁ τύραννος, μὴ δυνηθεὶς ἀναγκάσαι γέροντα μιαιροφαγῆσαι.

\par }\Chap{8}{\PP \VerseOne{1}Τὸ δὲ δὴ σφόδρα περιπαθῶς ἐκέλευσεν ἄλλους ἐκ τῆς ἠλικίας τῶν Ἑβραίων ἀγαγεῖν· καὶ εἰ μὲν μιεροφαγήσαιεν, ἀπολύειν φάγοντας· εἰ δὲ ἀντιλέγοιεν, πικρότερον βασανίζειν.
\par }{\PP \VS{2}Ταῦτα διαδεξαμένου τοῦ τυράννου, παρῆσαν ἀγόμενοι μετὰ γηραιᾶς μητρὸς ἑπτὰ ἀδελφοὶ, καλοί τε καὶ αἰδήμονες καὶ γενναῖοι καὶ ἐν παντὶ χαριέντες.
\VS{3}Οὓς ἰδὼν ὁ τύραννος καθάπερ ἐν χορῷ περιέχοντας μέσην τὴν μητέρα, ἤσθετο ἐπʼ αὐτοῖς, καὶ τῆς εὐπρεπείας ἐκπλαγεὶς καὶ τῆς εὐγενείας προσεμειδίασεν αὐτοῖς, καὶ πλησίον καλέσας, ἔφη,
\par }{\PP \VS{4}Ὦ νεανίαι φιλοφρόνως ἐγὼ καθʼ ἐνὸς ἑκάστου ὑμῶν θαυμάζω τὸ κάλλος· καὶ τὸ πλῆθος τοσούτων ἀδελφῶν ὑπερτιμῶν, οὐ μόνον συμβουλεύω μὴ μανῆναι τὴν αὐτὴν τῷ προβασανισθέντι γέροντι μανίαν·
\VS{5}ἀλλὰ καὶ παρακαλῶ συνείξαντας τῆς ἐμῆς ἀπολαῦσαι φιλίας· δυναίμην γὰρ ὥσπερ κολάζειν τοὺς ἐπιτάγμασιν, σὕτως καὶ εὐεργετεῖν τοὺς εὐπειθοῦντάς μοι.
\par }{\PP \VS{6}Πιστεύσατε οὖν, καὶ ἀρχὰς ἐπὶ τῶν ἐμῶν πραγμάτων ἡγεμονικὰς λήψεσθε, ἀρνησάμενοι τὸν πάτριον ἡμῶν τῆς πολιτείας θεσμόν·
\VS{7}καὶ μεταλαβόντες Ἑλληνικοῦ βίου, καὶ μεταδιαιτηθέντες ἐντρυφήσατε ταῖς νεότησιν ὑμῶν.
\VS{8}Ἐπεὶ ἐὰν ὀργίλως με διάθησθε διὰ τῆς ἀπειθείας ὑμῶν, ἀναγκάσετέ με ἐπὶ δειναῖς κολάσεσιν ἕνα ἕκαστον ὑμῶν διὰ τῶν βασάνων ἀπολέσαι.
\VS{9}Κατελεήσατε οὖν ἑαυτοὺς, οὕς καὶ ὁ πολέμιος ἔγωγε καὶ τῆς ἡλικίας καὶ τῆς εὐμορφίας οἰκτείρομαι.
\VS{10}Οὐ διαλογιεῖσθε τοῦτο, ὅτι οὐδὲν ὑμῖν ἀπειθήσασιν πλὴν τοῦ μετὰ στρεβλῶν ἀποθανεῖν ἀπόκειται;
\par }{\PP \VS{11}Ταῦτα δὲ λέγων, ἐκέλευσεν εἰς τὸ ἔμπροσθεν προτεθῆναι τὰ βασανιστήρια, ὅπως καὶ διὰ τοῦ φόβου πείσειεν αὐτοὺς μιεροφαγῆσαι.
\VS{12}Ὡς δὲ τροχούς τε καὶ ἀρθενβόλους στρεβλωτήρια, καὶ τροχαντῆρας καὶ καταπέλτας καὶ λέβητας, τήγανά τε καὶ δακτυλήθρας, καὶ χεῖρας σιδηρᾶς καὶ σφῆνας, καὶ τὰ ζώπυρα τοῦ πυρὸς οἱ δορυφόροι προέθησαν, ὑπολαβὼν δὲ ὁ τύραννος, ἔφη, μειράκια φοβήθητε,
\VS{13}καὶ ἥν σέβεσθε δίκην, ἵλεως ὑμῖν ἔσται διʼ ἀνάγκην παρανομήσασιν.
\VS{14}Οἱ δὲ ἀκούσαντες ἐπαγωγὰ, καὶ ὁρῶντες δεινὰ, οὐ μόνον οὐκ ἐφοβήθησαν, ἀλλὰ καὶ ἀντεφιλοσόφῃσαν τῷ τυράννῳ, καὶ διὰ τῆς εὐλογιστίας τὴν τυραννίδα αὐτοῦ κατέλυσαν.
\par }{\PP \VS{15}Καί τοι λογισώμεθα· εἰ δειλόψυχοί τινες ἦσαν, καὶ ἄνανδροι ἐν αὐτοῖς, ποίοις ἂν ἐχρήσαντο λόγοις; οὐχὶ τούτοις;
\VS{16}Ὦ τάλανες ἡμεῖς, καὶ λίαν ἀνόητοι· βασιλέως ἡμᾶς παρακαλοῦντος, καὶ ἐπὶ εὐεργεσίᾳ φωνοῦντος, μὴ πεισθείημεν αὐτῷ;
\VS{17}Τί βουλήμασιν κενοῖς ἐαυτοὺς εὐφραίνομεν, καὶ θανατηφόρον ἀπείθειαν τολμῶμεν;
\VS{18}Οὐ φοβησόμεθα, ἄνδρες ἀδελφοί, τὰ βασανιστήρια, καὶ λογιούμεθα τὰς τῶν βασάνων ἀπειλὰς, καὶ φευξόμεθα τὴν κενοδοξίαν ταύτην καὶ ὀλεθροφόρον ἀλαζονείαν;
\VS{19}Ἐλεήσωμεν τὰς ἑαυτῶν ἡλικίας, καὶ κατοικτειρήσωμεν τὸ τῆς μητρὸς γῆρας·
\VS{20}καὶ ἐνθυμηθῶμεν, ὅτι ἀπειθοῦντες τεθνηξόμεθα.
\VS{21}Συγγνώσεται δὲ ἡμῖν καὶ ἡ θεία δίκη διʼ ἀνάγκην τὸν βασιλὲα φοβηθεῖσιν.
\VS{22}Τί ἐξάγομεν ἑαυτοὺς τοῦ ἡδίστου βίου, καὶ ἐπιστεροῦμεν ἐαυτοὺς τοῦ γλυκέος κόσμου;
\VS{23}Μὴ βιαζώμεθα τὴν ἀνάγκην, μηδὲ κενοδοξήσωμεν ἐπʼ τῇ ἑαυτῶν στρέβλῃ.
\VS{24}Οὐδὲ αὐτὸς ὁ ναὸς ἑκουσίως ἡμᾶς θανατοῖ φοβηθέντας τὰ βασανιστήρια.
\VS{25}Πόθεν ἡμῖν ἡ τοσαύτη ἐντέηκεν φιλονεικία, καὶ ἡ θανατεφόρος ἀρέσκει καρτερία, παρὸν μετὰ ἀταραξίας χρὴ τῷ βασιλεῖ πεισθέντας;
\par }{\PP \VS{26}Ἀλλὰ τούτων οὐδὲν εἶπον οἱ νεανίαι βασανίζεσθαι μέλλοντες, οὐδὲ ἐνεθυμήθησαν.
\VS{27}Ἦσαν γὰρ περιφρονες τῶν παθῶν, καὶ αὐτηκράτορες τῶν ἀλγηδόνων. Ὥστε ἅμα τῷ παύσασθαι τὸν τύραννον συμβουλεύοντα αὐτοῖς μιεροφαγῆσαι, πάντες διὰ μιᾶς φωνῆς ὁμοῦ, ὥσπερ ἀπὸ τῆς αὐτῆς ψυχῆς, εἶπον,

\par }\Chap{9}{\PP \VerseOne{1}Τί μέλλεις, ὦ τύραννε; ἕτοιμοι γάρ ἐσμεν ἀποθνήσκειν, ἢ παραβαίνειν τὰς πατρίους ἡμῶν ἐντολάς.
\VS{2}Καὶ αἰσχυνόμεθα γὰρ τοὺς προγόνους εἰκότως, εἰ μὴ τῇ τοῦ νόμου εὐπειθείᾳ καὶ συμβούλῳ γνώσει χρησαίμεθα.
\par }{\PP \VS{3}Σύμβουλε τύραννε παρανομίας, μὴ ἡμᾶς μισῶν ὑπὲρ αὐτοὺς ἡμᾶς ἐλέα.
\VS{4}Χαλεπώτερον γὰρ αὐτοὺς τοῦ θανάτου νομίζομεν εἶναί σου τὸν ἐπὶ τῇ παρανόμῳ σωτηρίᾳ ἡμῶν ἔλεον.
\VS{5}Ἐκφοβεῖς δὲ ἡμᾶς, τὸν διὰ τῶν βασάνων ἡμῖν θάνατον ἀπειλῶν, ὥσπερ οὐχὶ πρὸ βραχέως παρὰ Ἐλεαζάρου μαθών.
\VS{6}Εἰ δʼ οἱ γέροντες τῶν Ἑβραίων διὰ τὴν εὐσέβειαν καὶ βασανισμὸυς ὑπομείναντες ἀπέθανον, ἀποθάνοιμεν ἂν δικαιότερον ἡμεῖς οἱ νέοι, τὰς βασάνους τῶν σῶν ἀναγκῶν ὑπεριδόντες, ἃς καὶ ὁ παιδευτὴς γέρων ἐνίκησεν.
\par }{\PP \VS{7}Πείραζε γαροῦν τύραννε· καὶ τὰς ἡμῶν ψυχὰς εἰ θανατώσεις διὰ τὴν εὐσέβειαν, μὴ νομίσῃς ἡμᾶς βλάπτειν βασανίζων.
\VS{8}Ἡμεῖς μὲν γὰρ διὰ τῆσδε τῆς κακοπαθείας καὶ ὑπομονῆς, τὰ τῆς ἀρετῆς ἆθλα οἴσομεν.
\VS{9}Σὺ δὲ διὰ τὴν ἡμῶν μιαροφονίαν αὐτάρχη καρτερήσεις περὶ τῆς θείας δίκης αἰώνιον βάσανον διὰ πυρός.
\par }{\PP \VS{10}Ταῦτα αὐτῶν εἰπόντων, οὐ μόνον ὡς κατὰ ἀπειθούντων ἐχαλέπαινεν ὁ τύραννος, ἀλλʼ ὡς καὶ κατὰ ἀχαρίστων ὠργίσθη.
\VS{11}Ὅθεν τὸν πρεσβύτατον αὐτῶν κελευθέντες παρήγαγον οἱ μαστισταὶ, καὶ διαῤῥήξαντες τὸν χιτῶνα διέδησαν τὰς χεῖρας αὐτοῦ καὶ τοὺς βραχίονας ἱμᾶσιν ἑκατέρωθεν.
\VS{12}Ὡς δὲ τύπτοντες ταῖς μάστιξιν ἐκοπίασαν, μηδὲν ἀνύοντες, ἀνέβαλον αὐτὸν ἐπὶ τὸν τροχόν.
\VS{13}Περὶ ὃν κατατεινόμενος ὁ εὐγενὴς νεανίας, ἔξαρθρος ἐγίνετο.
\VS{14}Καὶ κατὰ πᾶν μέλος κλώμενος κατηγόρει, λέγων,
\par }{\PP \VS{15}Τύραννε μιαιρώτατε, καὶ τῆς οὐρανίου δίκης ἐχθρὲ, καὶ ὠμόφρον, οὐκ ἀνδροφονήσαντά με τοῦτον καταικίζεις τὸν τρόπον, οὐδὲ ἀσεβήσαντα, ἀλλα θείου νόμου προασπίζοντα.
\VS{16}Καὶ τῶν δορυφόρων λεγόντων, ὁμολόγησον φαγεῖν, οὕπως ἀπαλλαγῇς τῶν βασάνων,
\VS{17}ὁ δὲ εἶπεν, οὐχ οὕτως ἰσχυρὸς ὑμῶν ἐστιν ὁ τρόπος, ὦ μιαιροὶ διὰκονοι, ὥστε μου τὸν λογισμὸν ἄξαι· τέμνετέ μου μέλη, καὶ πυροῦτε τὰς σάρκας, καὶ στρεβλοῦτε τὰ ἄρθρα.
\VS{18}Διὰ πασῶν γὰρ ὑμᾶς πείσω τῶν βασάνων· ὅτι μόνοι παῖδες Ἑβραίων ὑπὲρ ἀρετῆς εἰσιν ἀνίκητοι.
\par }{\PP \VS{19}Ταῦτα λέγοντες εἰς πῦρ ἐπέτρωσαν, καὶ διερεθίζοντες, τὸν τροχὸν προσεπικατέτεινον.
\VS{20}Ἐμολύνετο δὲ πάντοθεν αἵματι ὁ τρόχος, καὶ ὁ σωρὸς τῆς ἀνθρακιᾶς τοῖς τῶν ἰχώρων ἐσβέννυτο σταλαγμοῖς, καὶ περὶ τοὺς αὔξονας τοῦ ὀργάνου περιέῤῥεον αἱ σάρκες.
\par }{\PP \VS{21}Καὶ περιτετηκμένον ἤδη ἔχων τὸ τῶν ὀστέων πῆγμα ὁ μεγαλόφρων καὶ Ἀβραμιαῖος νεανίας οὐκ ἐστέναξεν.
\VS{22}Ἀλλʼ ὥσπερ ἐν πυρὶ μετασχηματιζόμενος εἰς ἀφθαρσίαν, ὑπέμεινεν εὐγενῶς τὰς στρέβλας.
\VS{23}Μιμήσασθέ με, ἀδελφοὶ, λέγων· μή μου τὸν αἰῶνα λειποτακτήσητε, μηδʼ ἐξομόσησθέ μου τὴν τῆς εὐψυχίας ἀδελφότητα· ἱερὰν καὶ εὐγενῆ στρατείαν στρατεύσασθε περὶ τῆς εὐσεβείας.
\VS{24}Διʼ ἧς ἷλεως ἡ δικαία καὶ πάτριος ἡμῶν πρόνοια τῷ ἔθνει γενηθεῖσα τιμωρήσειεν τὸν ἀλάστορα τύραννον.
\VS{25}Καὶ ταῦτα εἰπὼν ὁ ἱεροπρεπὴς νεανίας, ἀπέῤῥηξεν τὴν ψυχήν.
\par }{\PP \VS{26}Θαυμασάντων δὲ πάντων τὴν καρτεροψυχίαν αὐτοῦ, ἦγον οἱ δορυφόροι τὸν καθʼ ἡλικίαν τοῶ προτέρου δεύτερον, καὶ σιδηρᾶς ἐναρμοσάμενοί χεῖρας, ὀξέσιν τοῖς ὄνυξιν, τοῖς ὀργάνοις καταπέλτῃ προσέδησαν αὐτόν.
\VS{27}Ὡς δὲ, εἰ φαγεῖν βούλοιτο πρὶν βασανίζεσθαι πυνθανόμενοι, τὴν εὐγενῆ γνώμην ἤκουσαν·
\VS{28}ἀπὸ τῶν τενόντων ταῖς σιδηραῖς χερσὶν ἐπισπασάμενοι, μέχρι γε τῶν γενείων τὴν σάρκα πᾶσαν καὶ τὴν τῆς κεφαλῆς δορὰν οἱ παρδάλειοι θῆρες ἀπέσυραν· ὁ δὲ ταύτην βαρέως τὴν ἀλγηδόνα καρτερῶν, ἔλεγεν,
\VS{29}Ὡς ἡδὺς πᾶς τρόπος θανάτου, διὰ τὴν πάτριον ἡμῶν εὐσέβειαν· ἔφη τε πρὸς τὸν τύραννον,
\par }{\PP \VS{30}Οὐ δοκεῖς, πάντων ὠμότατε τύραννε, πλεῖων ἐμοῦ σε νὺν βασανίζεσθαι, ὁρῶν σου νικώμενον τὸν τῆς τυραννίδος ὑπερήφανον λογισμὸν ὑπὸ τῆς διὰ τὴν εὐσέβειαν ἡμῶν ὑπομονῆς.
\VS{31}Ἐγὼ μὲν γὰρ ταῖς διὰ τὴν ἀρετὴν ἡδοναῖς τὸν πόνον ἐπικουφίζομαι.
\VS{32}Σὺ δὲ ἐν ταῖς τῆς ἀσεβείας ἀπειλαῖς βασανίζῃ· οὐκ ἐκφεύξῃ δὲ, μιαιρότατε τύραννε, τὰς τῆς θείας ὀργῆς δίκας.

\par }\Chap{10}{\PP \VerseOne{1}Καὶ τούτου τὸν ἀοίδιμον θάνατον καρτερήσαντος, ὁ τρίτος ἤγετο, παρακαλούμενος πολλὰ ὑπὸ πολλῶν ὅπως ἀπογευσάμενος σώζοιτο.
\VS{2}Ὁ δὲ ἀναβοήσας, ἔφη, ἤ ἀγνοεῖτε, ὅτι αὐτός με τοῖς ἀποθανοῦσιν ἔσπειρεν πατὴρ, καὶ ἡ αὐτὴ μήτηρ ἐγέννεσιν, καὶ ἐπὶ τοῖς αὐτοῖς ἀνετράφην δόγμασιν;
\VS{3}Οὐκ ἐξόμνυμαι τὴν εὐγενῆ τῆς ἀδελφότητος συγγένειαν.
\VS{3a}Πρὸς ταῦτα εἴ τι ἔχετε κολαστήριον προσαγάγετε τῷ σώματί μου· τῆς γὰρ ψυχῆς μου, οὐδʼ ἂν θέλητε, ἅψασθαι δύνασθε.
\par }{\PP \VS{5}Οἱ δὲ πίκρῶς ἐνέγκαντες τὴν παῤῥησίαν τοῦ ἀνδρὸς, ἀρθρεμβόλοις ὀργάνοις τὰς χεῖρας αὐτοῦ καὶ τοὺς πόδας ἐξήρθρουν, καὶ ἐξ ἁρμῶν ἀναμοχλεύοντες ἐξεμέλιζον·
\VS{6}καὶ τοὺς δακτύλους, καὶ τοὺς βραχίονας, καὶ τὰ σκέλη, καὶ τοὺς ἀγκῶνας περιέλκων.
\VS{7}Καὶ κατὰ μηδένα τρόπον ἰσχύοντες αὐτὸν ἄγξαι, περισύραντες τὸ δέρμα σὺν ἄκραις ταῖς τῶν δακτύλων κορυφαῖς ἀπεσκύθιζον, καὶ εὐθέως ἦγον ἐπὶ τὸν τροχόν.
\VS{8}Περὶ ὃν ἐκ σφονδύλων ἐκμελιζόμενος ἑώρα τὰς ἑαυτοῦ σάρκας περιλακιζομένας καὶ κατὰ σπλάγχνων σταγόνας αἵματος ἀποῤῥεούσας.
\VS{9}Μέλλων δὲ ἀποθνήσκειν,
\VS{10}ἔφη, ἡμεῖς μὲν ὦ μιαιρώτατε τύραννε, διὰ παιδείαν καί ἀρετὴν Θεοῦ ταῦτα πάσχομεν.
\VS{11}Σὺ δὲ διὰ τὴν ἀσέβειαν καὶ μιαιφονίαν, ἀκαταλύτους καρτερήσεις βασάνους.
\par }{\PP \VS{12}Καὶ τούτου θανόντος ἀδελφοπρεπῶς, τὸν τέταρτον ἐπεσπῶντο, λέγοντες,
\VS{13}Μὴ μανῄς καὶ σὺ τοῖς ἀδελφοῖς σου τὴν αὐτὴν μανίαν· ἀλλὰ πεισθεὶς τῷ βασιλεῖ, σῶζε σεαυτόν.
\VS{14}Ὁ δὲ αὐτοῖς ἔφη, οὐχ οὕτως καυστικώτερον ἔχετε κατʼ ἐμοῦ τὸ πῦρ, ὥστε με δειλανδρῆσαι.
\VS{15}Μὰ τὸν μακάριον τῶν ἀδελφῶν μου θάνατον, καὶ τὸν αἰώνιον τοῦ τυράννου ὄλεθρον, καὶ τὸν ἀοίδιμον τῶν εὐσεβῶν βίον, οὐκ ἀρνήσομαι τὴν εὐγενῆ ἀδελφότητα.
\VS{16}Ἐπινόει, τύραννε, βασάνους· ἵνα καὶ διὰ τούτων μάθῃς, ὅτι ἀδελφός εἰμι τῶν προβεβανασισθέντων.
\par }{\PP \VS{17}Ταῦτα ἀκούσας ὁ αἱμοβόρος καὶ φονώδης καὶ πανμιαιρώτατος Ἀντίοχος, ἐκέλευσεν τὴν γλῶτταν αὐτοῦ ἐκτεμεῖν.
\VS{18}Ὁ δὲ ἔφη, κᾂν ἀφέλῃς τὸ τῆς φωνῆς ὄργανον, καὶ σιωπώντων ἀκούει ὁ Θεός.
\VS{19}Ἰδοὺ κεχάλασται ἡ γλῶσσα· τέμνε· οὐ γὰρ παρὰ τοῦτο τὸν λογισμὸν ἡμῶν γλωσσοτομήσεις.
\VS{20}Ἡδέως ὑπὲρ τοῦ Θεοῦ τὰ τοῦ σώματος μέλη ἀκρωτηριαζόμενα.
\VS{21}Σὲ δὲ ταχέως μετελεύσεται ὁ Θεός· τὴν γὰρ τῶν θείων ὕμνων μελῳδὸν γλῶτταν ἐκτέμνεις.

\par }\Chap{11}{\PP \VerseOne{1}Ὡς δὲ καὶ οὗτος ταῖς βασάνοις καταικισθεὶς ἐναπέθανεν, ὁ πέμπτος παρεπήδησεν, λέγων,
\par }{\PP \VS{2}Οὐ μέλλω, τύραννε, πρὸς τὸν ὑπὲρ τῆς ἀρετῆς βασανισμὸν παραιτεῖσθαι.
\VS{3}Αὐτὸς δʼ ἀπʼ ἐμαυτοῦ παρῆλθον, ὅπως κᾀμὲ κατακτείνας, περὶ πλειόνων ἀδικημάτων ὀφειλήσῃς τῇ οὐρανίῳ δίκῃ τιμωρίαν.
\VS{4}Ὦ μισάρετε καὶ μισάνθρωπε, τὶ δράσαντας ἡμᾶς τοῦτον πορθεῖς τὸν τρόπον;
\VS{5}Ἢ κακόν σοι δοκεῖ, ὅτι τὸν πάντων κτιστὴν εὐσεβοῦμεν, καὶ κατὰ τὸν ἐνάρετον αὐτοῦ ζῶμεν νόμον;
\VS{6}Ἀλλὰ ταῦτα τιμῶν, οὐ βασάνων ἐστὶν ἄξια.
\VS{6a}Εἴπερ ᾐσθάνου ἀνθρώπου πόθων, καὶ ἐλπίδα εἶχες παρὰ Θεῷ σωτηρίου·
\VS{6b}νῦν ἰδὲ ἀλλότριος ὢν Θεοῦ, πολεμεῖς τοὺς εὐσεβοῦντας εἰς τὸν Θεόν.
\par }{\PP \VS{9}Τοιαῦτα λέγοντα οἱ δορυφόροι δήσαντες, αὐτὸν εἷλκον ἐπὶ τὸν καταπέλτην·
\VS{10}ἐφʼ ὃ δήσαντες αὐτὸν ἐπὶ τὰ γόνατα, καὶ ταῦτα ποδάγραις σιδηραῖς ἐφορμάσαντες τὴν ὀσφὺν αὐτοῦ ἐπὶ τὸν τροχιαῖον σφῆνα κατέκαμψαν· περὶ ὃν ὅλος ἐπὶ τὸν τρονὸν σκορπίου τρόπον ἀνακλώμενος ἐξεμελίζετο.
\VS{11}Κατὰ τοῦτον τὸν τρόπον καὶ τὸ πνεῦμα στενοχωρούμενος, καὶ τὸ σῶμα ἀγχόμενος, καλὰς, ἔλεγεν, ἄκων,
\VS{12}ὦ τύραννε, χάριτας ἡμῖν χαρίζῃ διὰ γενναιοτέρων πόνων ἐπιδείξασθαι παρέχων τὴν εἰς τὸν νόμον ἡμῶν καρτερίαν.
\par }{\PP \VS{13}Τελευτήσαντος δὲ καὶ τούτου, ὁ ἕκτος ἤγετο μειρακίσκος· ὃς πυνθανομένου τοῦ τύραννου εἰ βούλοιτο φαγὼν ἀπολύεσθαι, ὁ δὲ ἔφη,
\par }{\PP \VS{14}Ἐγὼ τῇ μὲν ἡλικίᾳ τῶν ἀδελφῶν μου εἰμὶ νεώτερος, τῇ δὲ διανοίᾳ ἡλικιώτης·
\VS{15}Εἰς τὰ αὐτὰ γὰρ καὶ γεννηθέντες καὶ τραφέντες, ὑπὲρ τῶν αὐτῶν καὶ ἀποθνήσκειν ὀφείλομεν ὁμοίως.
\VS{16}Ὥστε εἰ σοὶ δοκεῖ βασανίξειν, μὴ μιαιροφαγοῦτας βασάνιζε.
\par }{\PP \VS{17}Ταῦτα αὐτὸν εἰπόντα παρῆγον ἐπὶ τὸν τροχόν.
\VS{18}Ἐφʼ οὗ κατατεινόμενος εὐμελῶς καὶ ἐκσφονοδυλιζόμενος ὑπεκαίετο.
\VS{19}Καὶ ὀβελίσκους ὀξεῖς πυρώσαντες, τοῖς νότοις προσέφερον· καὶ τὰ πλευπὰ διαπείραντες, ἀπʼ αὐτοῦ σπλάγχνα διέκαιον.
\par }{\PP \VS{20}Ὁ δὲ βασανιζόμενος, ὦ ἱεροπρεποῦς αἰῶνος, ἔλεγεν, ἐφʼ ὃν διὰ τὴν εὐσέβειαν εἰς γυμνασίαν πόνων ἀδελφοὶ τοσοῦτοι κληθέντες οὐκ ἐνικήθημεν.
\VS{21}Ἀνίκητος γάρ ἐστιν, ὦ τύραννε, ἡ εὐσεβὴς ἐπιστήμη.
\VS{22}Καλοκᾳγαθίᾳ καθωπλισμένος τεθνήξομαι κἀγὼ μετὰ τῶν ἀδελφῶν μοῦ.
\VS{23}Μέγαν σοὶ προσβάλλων καὶ αὐτὸς ἀλάστορα, καινουργὲ τῶν βασάνων, καὶ πολέμιε τῶν ἀληθῶς εὐσεβούντων.
\par }{\PP \VS{24}Ἓξ μειράκια κατελύσαμέν σου τὴν τυραννίδα.
\VS{25}Τὸ γὰρ μὴ δυνηθῆναί σε μεταπεῖσαι τὸν λογισμὸν ἡμῶν, μήτε βιάσασθαι πρὸς τὴν μιαιροφαγίαν, οὐ κατάλυσίς ἐστιν σοῦ;
\VS{26}Τὸ πῦρ σου ψυχρὸν ἡμῖν, καὶ ἄπονοι οἱ καταπέλται, καὶ ἀδύνατος ἡ βία σου.
\VS{27}Οὐ γὰρ τυράννου, ἀλλὰ θείου νόμου προεστήκασιν ἡμῶν οἱ δορυφίροι· διὰ τοῦτο ἀνίκητον ἕχομεν τὸν λογι σμόν.

\par }\Chap{12}{\PP \VerseOne{1}Ὡς δὲ καὶ οὗτος μακαρίως ἐναπέθανεν καταβληθεὶς εἰς λέβητα, ὁ ἕβδομος παρεγίνετο, πάντων νεώτερος.
\VS{2}Ὅν κατοικτειρήσας ὁ τύραννος, καίπερ δεινῶς ὑπὸ τῶν ἀδελφῶν αὐτοῦ κακισθεὶς, ὁρῶν ἤση τὰ δεσμὰ περικείμενον,
\VS{3}πλησιέστερον αὐτὸν μετεπέμψατο, καὶ παρηγορεῖν ἐπειρᾶτο, λέγων,
\par }{\PP \VS{4}Τῆς μὲν τῶν ἀδελφῶν σου ἀπονοίας τὸ τὲλος ὁρᾷς· διὰ γὰρ ἀπείθειαν στρεβλωθέντες τεθνήκασιν, σὺ, εἰ μὲν μὴ πεισθείης, τὰλας βασανισθεὶς καὶ πασανισθεὶς καὶ αὐτὸς τεθνήξῃ πρὸ ὥρας.
\VS{5}Πεισθείης δὲ φίλος ἔσῃ, καὶ τῶν ἐπὶ τῆς βασιλείας ἀφηγήσῃ πραγμάτων.
\par }{\PP \VS{6}Καὶ ταῦτα παρακαλῶν, τὴν μητέρα τοῦ παιδὸς μετεπέμψατο, ὄπως αὐτὴν ἐμεήσαν υἱῶν στερηθῖσαν παρορμήσειεν ἐπὶ τὴν μσωτηρίαν, εὐπειθῆ τὸν περιλειπόμενον.
\VS{7}Ὁ δὲ τῆν μητρὸς τῇ Ἐβραΐδι φωνῇ προτρεψαμένης· αὐτὸν (ὡς ἐροῦμεν μετὰ μικρὸν ὕστερον.) ἀπολύσατε με, φησίν·
\VS{8}εἴπω τῷ βασιλεῖ καὶ τοῖς σὺν αὐτῷ φίλοις πᾶσιν.
\VS{9}Καὶ ἐπιχαρέντες μάλιστα ἐπὶ τῇ ἐπαγγελίᾳ τοῦ παιδὸς ταχέως ἔλυσαν αὐτόν.
\par }{\PP \VS{10}Καὶ σραμὼν ἐπὶ πλησίον τῶν τηγάνων, ἔφη, ἀνόσιε,
\VS{11}φησὶν, καὶ πάντων τῶν πονηρῶν ἀσεβέστατε τύραννε, οὐκ ᾐδέσθης παρὰ τοῦ Θεοῦ λαβὼν τὰ ἀγαθὰ καὶ τὴν βασιλείαν, τούς θεράποντας αὐτοῦ κατακτεῖναι, καὶ τοὺς τῆς εὐσεβείας στρεβλῶσαι;
\VS{12}Ἀνθ ὧ ταμιεύεταί σε ἡ θεια δίκη πυκνοτέρῳ καὶ αἰωνίῳ πυεὶ καὶ βασάνοις, αἳ εἰς ὅλον τὸν αιῶνα οὐκ ἀνήσουσίν σε.
\par }{\PP \VS{13}Οὐκ ᾐδέσθης ἄνθρωπος ὢν, θηριωδέστατε, τοὺς ὁμοιοπαθεῖς καὶ ἐκ τῶν αὐτῶν γεγονότας στοιξείων γλωττοτομῆσαι, καὶ τοῦτον καταικίσας τὸν τρόπον βασανίσαι;
\VS{14}Ἀλλʼ οἱ μὴν εὐγενῶς ἀποθανόντες ἐπλήρωσαν τὴν εἰν τὸν Θεὸν εὐσέβειαν.
\VS{15}Σὺ δὲ κακὸς κακῶς οἰμώξεις, τοὺς τῆς ἀρετῆν ἀγωνιστὰν ἀναιτίως ἀποκτεῖναι.
\par }{\PP \VS{16}Ὅθεν καὶ αὐτὸς ἀποθνήσκειν μέλλων, ἔφη,
\VS{17}οὐκ ἀπαυτομολῶ τῆς τῶν ἀδελφῶν μου μαρτυρίας.
\VS{18}Ἐπικαλοῦμαι δὲ τὸν πατρῷον Θεὸν, ὅπως ἵλεως γένει μου.
\VS{19}Σὲ δὲ καὶ ἐν τῷ νῦν βίῳ καὶ θανόντα τιμωρήσεται.
\par }{\PP \VS{20}Καὶ ταῦτα κατευξάμενος, ἐαυτὸν ἔοιψεν κατὰ τῶν τηγάνων· καὶ οὕτως ἀπέδεκεν.

\par }\Chap{13}{\PP \VerseOne{1}Εἰ δὲ τοίνυν τῶν μέχρι θανάτου πόνων ὑπερεφρόνησαν οἱ ἑπτὰ ἀδελφοὶ, συνομολογεῖται πανταχόθεν, ὅτι αὐτοδέσποτός ἐστιν τῶν παθῶν ὁ εὐσεβὴς λογισμός.
\VS{2}Ὤσπερ γὰρ εἰ τοῖς πάθεσιν σουλωθέντες ἐμιεροφάγησαν, ἐλέγομεν γὰρ αὐτοὺς τούτοις νενικῆσθαι.
\VS{3}Νυνὶ δὲ οὐχ οὕτως· ἀλλὰ τῷ ἐπαινουμένῳ λογισμῷ παρὰ περιεγένοντο τῶν παθῶν.
\par }{\PP \VS{4}Καὶ οὐκ ἐστὶν παριδεῖν τὴν ἡγεμονίαν· ἐπεκρὰτησεν γὰρ καὶ πὰθους καὶ πὸνων.
\VS{5}Πῶς οὖν οὐκ ἐστὶν τούτοις τὴν εὐλογιστίας παθοκρὰτειαν ὁμολογεῖν, οἱ τῶν μὲν διὰ πυρὸς ἀλγησόνων οὐκ ἐπεστράφησαν;
\VS{6}Καθάπερ γὰρ προπλήταις λιμένων πύργοις τὰς κυμάτων ἀπειλὰς ἀνακόπτοντες, γαληνὸν παρὲχουσιν τοῖς εἰσπλέουσιν τὸν ὅρμον.
\VS{7}Οὕτος ἡ ἑπτάπυργος τῶν νεανίσκων εὐλογιστία τὸν τῆς εὐσεβείας ὀχυρώσασα λιμένα τὴν τῶν παθῶν ἐνίκησεν ἀκολασίαν.
\par }{\PP \VS{8}Ἱερὸν γὰρ εὐσεβείαν στήσαντες χορὸν παρεθάρσυνον ἀλλήλους, λέγοντες,
\VS{9}ἀδελφικῶς ἀποθάνοιμεν, ἀδελφοὶ, περὶ τοῦ νόμον· μιμησώμεθα τούς τρεῖς τοὺς ἐπὶ Ἀσσυρίας νεανίσκους, οἵ τῆς ἰσεπόλιδος καμίνου κατεφρόνησαν.
\VS{10}Μὴ δειλανδρήσωμεν πρὸς τὴν τῆς εὐσεβείας ἀπόδειξιν.
\VS{11}Καὶ ὁ μὲν, θάῤῥει ἀδελφὲ, ἔλεγεν, ὁ δὲ, εὐγενῶς καρτέρησον.
\VS{12}Ὁ δὲ, ἔλεγεν, μνήσθητε πόθεν ἐστὲ, ἢ τίνος πατρὸς χειρὶ σφαγιασθῆναι διὰ τὴν εὐσέβειαν ὑπέμεινεν ὁ Ἰσαάκ.
\par }{\PP \VS{13}Εἶς δὲ ἕκαστος καὶ ἀλλήλους ὁμοῦ πάντες ἐφόρων φαιδροὶ καὶ μάλα θαῤῥαλέοι, ἐαυτοὺς, ἔλεγον, τῷ Θεῷ ἀφιερώσωμεν ἐξ ὅλης τῆς καρδίας τῷ δόντι τὰς ψυχὰς, καὶ χρήσωμεν τῇ περὶ τὸν νόμον φυλακῇ τὰ σώματα.
\VS{14}Μὴ φοβηθῶμεν τὸν δοκοῦντα ἀποκτενεῖν
\VS{15}Μέγας γὰρ ψυξῆς ἀγὼν καὶ κίνδυνος ἐν αἰωνίῳ βασάνῳ κείμενος τοῖς παραβᾶσιν τῆν ἐντολὴν τοῦ Θεοῦ.
\VS{16}Καθοπλισώμεθα τοιγαροῦν τῇ τοῦ θείου λογισμοῦ παθοκρατείᾳ.
\VS{17}Οὕτως παθόντας ἡμᾶς Αβραὰμ καὶ Ἰακὼβ ὑποδέξονται, καὶ πάντες οἱ πατέρες ἐπαινέσουσιν.
\VS{18}Καὶ ἑνὶ ἑκάστῳ τῶν ἀποστωμένων αὐτῶν ἀδελφῶν ἔλεγον οἱ περιλειπόμενοι, μὴ καταισχύνῃς ἡμᾶς ἀδελφὲ, μηδὲ ἀδελφὲ, μηδὲ ψεύσῃ τούς προαποθανόντας.
\par }{\PP \VS{19}Οὐκ ἀγνοεῖτε δὲ τὰ τῆς ἀνθρωπότητος φίλτρα, ἅπερ ἡ θεία καὶ πάνσοφος πρόνοια διὰ τῆς μητρῴας φυτεύσασα γαστρός·
\VS{20}ἐν ᾗ τὸν ἶσον ἀδελφοὶ κατοικήσαντες χρόνον, καὶ ἐν τῷ αὐτῳ χρόνῳ πλασθέντες, καὶ ἀπὸ τοῦ αὐτῷ αἵματος ἀξηθέντες, καὶ δια τῆς αὐτῆς ψυχῆς τελεσφορηθέντες,
\VS{21}καὶ διὰ τῶν ἴσων ἀποτεχθέντες χρόνον, καὶ ἀπὸ τῶν αὐτῶν γαλακτοποτοῦντες πηγῶν, ἀφʼ οὗ συντέφονται ἐν ἐναγκαλισμάτων φιλάδελφοι ψυχαί·
\VS{22}καὶ αὔξοντες σφοδρότερον διὰ συντροφίας, καὶ τῆς καθʼ ἠμέραν συνηθείας, καὶ τῆς ἄλλης παιδείας, καὶ τῆν ἡμετέρας ἐν νόμῳ Θεοῦ ἀσκήσεως.
\par }{\PP \VS{23}Οὕτως δὲ τοίνυν καθεστηκυίας τῆς φιλαδελφίας συμπαθούσης, οἱ ἑπτὰ ἀδελφοὶ συμπαθέστερον ἔσχον τὴν πρὸς ἀλλήλους ὁμόνοιαν.
\VS{24}Νόμῳ γὰρ τῷ αὐτῷ παιδευθέντες, καὶ τὰς αὐτὰς ἐξασκήσαντες ἀρετὰς, καὶ τῷ δικαίῳ συντραφέντες βίῳ, μᾶλλον ἐπʼ αὐτοὺς ἥγαγον.
\VS{25}Ἡ γὰρ ὁμοζηλία τῆς καλοκᾳγαθίας ἐπέτεινεν αὐτῶν τὴν πρὸς ἀλλήλους ὁμόνοιαν.
\VS{26}Σὺν γάρ τῇ εὐσεβείᾳ ποθεινοτέραν αὐτοῖς κατεσκεύαζεν τὴν φιλαδελφίαν.
\par }{\PP \VS{27}Ἀλλʼ ὁμοίως καίπερ τῆς φύσεως καὶ τῆς συνηθείας καὶ τῶν τῆς ἀρετῆς ἠθῶν τὰ τῆς ἀδελφότητος αὐτοῖν φίλτρα συναυξόντων, ἀνέσχοντο διὰ τὴν εὐσέβειαν τοὺν ἀδελφοὺς οἱ ὑπολελειμμένοι τοὺς καταικιζομένους, ὁρῶντες μέχρι θανάτου βασανιζομένους.

\par }\Chap{14}{\PP \VerseOne{1}Προσέτι καὶ ἐπὶ τὸν αἰκισμὸν ἐποτρύνοντες, ὡς μὴ μόνον τῶν ἀλγηδόνων περιφρονῆσαι αὐτοὺς, ἀλλὰ καὶ τῆς τῶν ἀδελφῶν φιλαδελφίας παθῶν κρατῆσαι.
\par }{\PP \VS{2}Ὦ βασιλέως λογισμοὶ βασιλικώτεροι καὶ ἐλευθέρων ἐλευθερώτεροι.
\VS{3}Ἱερὰς καὶ ἐναρμόστους περὶ τῆς εὐσεβείας τῶν ἑπτὰ ἀδελφῶν συμφωνίας.
\VS{4}Οὐδεὶς ἐκ τῶν ἑπτὰ μειρακίων ἐδειλίασεν, οὐδὲ πρὸς τὸν θάνατον ὤκνησεν.
\VS{5}Ἀλλὰ πάντες, ὥσπερ ἐπʼ ἀθανασίας ὁδὸν τρέχοντες, ἐπὶ τὸν διὰ τῶν βασάνων θάνατον ἔσπευδον.
\VS{6}Καθάπερ γὰρ χεῖρες καὶ πόδες συμφώνως τοῖς τῆς ψυχῆς ἀφηγήμασιν κινοῦνται· οὕτως οἱ ἱεροὶ μείρακες ἐκεῖνοι ὡς ὑπὸ ψυχῆς ἀθανάτου τῆς εὐσεβείας, πρὸς τὸν ὑπὲρ αὐτῆς συνεφώνησαν θάνατον.
\par }{\PP \VS{7}Ὦ παναγία ἡ συμφώνον ἀδελφῶν ἐβδομάς· καθάπερ γὰρ ἑπτὰ τῆς κοσμοποιΐας ἡμέραι περὶ τὴν εὐσέβειαν,
\VS{8}οὕτος περὶ τὴν ἑβδομάδα χορεύοντες οἱ μείρακες ἐκύκλουν τὸν τῶν βασάνων φόβον καταλύοντες.
\VS{9}Νῦν ἡμεῖς ἀκούοντες τῆν θλίψιν τῶν νεανίων ἐκείνων, φρίττομεν· οἱ δὲ οὐ μόνον ὁρῶντες, ἀλλʼ οὐδὲ μόνον ἀκούοντες τὸν παραχρῆμα ἀπειλῆς λόγον, ἀλλὰ καὶ πάσχοντες, ἐκαρτέρουν καὶ τοῦτο ταῖς διὰ πυρὸς ὀδύναις.
\VS{10}Ὧν τί γένοιτο ἐπαλγέστερον; ὀξεῖα γὰρ καὶ σύντομος ἡ τοῦ πυρὸς οὖσα δύναμις, ταχέως διέλυσε τὰ σώματα.
\par }{\PP \VS{11}Καὶ μὴ θαυμαστὸν ἡγεῖσθε, εἰ ὁ λογισμὸς περιεκράτησεν τῶν ἀνδρῶν ἐκείνων ἐν ταῖς βασάνοις, ὅπου γε καὶ γυναικὸς νοῦς πολυτροπωτέρον ὑπερεφρόνησεν ἀλγηδόνων.
\VS{12}Ἡ μήτηρ γὰρ τῶν ἑπτὰ νεανίσκων ὑπήνεγκεν τὰς ἐφʼ ἑνὶ ἐκάστῳ τῶν τέκνων στρέβλας.
\par }{\PP \VS{13}Θεωρεῖτε δὲ πῶς πολύπλοκός ἐστιν ἡ τῆς φιλοτεκνίας στοργὴ, ἕλκουσα πάντα πρὸς τὴν τῶν σπλάγχνων συμπάθειαν.
\VS{14}Ὅπου γε καὶ τὰ ἄλογα ζῶα ὁμοίαν τὴν πρὸς τὰ ἐξ αὐτῶν γεννώμενα συμπάθειαν καὶ στοργὴν ἔχει τοῖς ἀνθρώποις.
\VS{15}Καὶ γὰρ τῶν πετεινῶν, τὰ μὲν ἥμερα κατὰ τὰς οἰκίας ὀροφοιτοῦντα προασπίζει τῶν νεοττῶν.
\VS{16}Τὰ δὲ κατὰ τὰς κορυφὰς ὀρέων καὶ φαράγγων ἀποῤῥῶγας καὶ δένδρων ὀπὰς καὶ τὰς τούτων ἄκρας νοσσοποιησάμενα ἀποτίκτει, καὶ τὸν προσιόντα κωλύει.
\VS{17}Εἰ δὲ καὶ μὴ δύναιντο κωλύειν, περιπτάμενα κυκλόθεν αὐτῶν ἀλγοῦντα τῇ στοργῇ, ἀνακαλούμενα τῇ ἰδίᾳ φωνῇ, καθʼ ὃν δύναται τρόπον βοηθεῖ τοῖς τέκνοις.
\par }{\PP \VS{18}Καὶ τί δεῖ τὴν διὰ τῶν ἀλόγων ζώων ἐπιδεικνύναι τὴν πρὸς τὰ τέκνα συμπάθειαν.
\VS{19}Ὅπου γε καὶ μέλισσαι περὶ τὸν τῆς κηρογονίας καιρὸν ἐπαμύνονται τοὺς προσιόντας, καὶ καθάπερ σιδήρῳ τῷ κέντρῳ πλήσσουσι τοὺς προσιόντας τῇ νοσσιᾷ αὐτῶν, καὶ ἐπαμύνονται ἕως θανάτου.
\par }{\PP \VS{20}Ἀλλʼ οὐχὶ τὴν Ἁβραὰμ ὁμόψυχον τῶν νεανίων μητέρα μετεκίνησεν συμπάθεια τῆς συμπαθείας τέκνων.

\par }\Chap{15}{\PP \VerseOne{1}Ὦ λογίσμε τέκνων, παθῶν τύραννε, καὶ εὐσέβεια μητρὶ τέκνων ποθεινοτέρα.
\VS{2}Μήτηρ δυοῖν προκειμένων εὐσεβείας, καὶ τῆς ἑπτὰ υἱῶν σωτηρίας προκαίρους κατὰ τὴν τοῦ τυράννου ὑπόσχεσιν·
\VS{3}τὴν εὐσέβειαν μᾶλλον ἠγάπησεν τὴν σώζουσαν εἰς αἰώνιον ζωὴν κατὰ Θεόν.
\par }{\PP \VS{4}Ὦ τίνα τρόπον ἠθολογήσαιμι φιλότεκνα γονέων πάθη, ψυχῆς τε καὶ μορφῆς ὁμοιότητα εἰς μικρὸν παιδὸς χαρακτῆρα θαυμάσιον ἐναπεσφράγιζον, μάλιστα διὰ τὸν τῶν παθῶν τοῖς γεννηθεῖσιν τὰς μητέρας καθεστάναι συμπαθευτέρας.
\VS{5}Ὅσῳ γὰρ καὶ ἀσθενόψυχοι καὶ πολυγονώτεραι ὑπάρχουσιν μητέρες, τοσούτῳ μᾶλλόν εἰσιν φιλοτεκνότεραι.
\VS{6}Πασῶν δὲ τῶν μητέρων ἐγένετο ἡ τῶν ἑπτὰ μήτηρ φιλοτεκνοτέρα, ἥ τις ἑπτὰ κυοφορίαις τὴν πρὸς αὐτοὺς ἐπιφυτευομένη φιλοστοργία,
\VS{7}καὶ διὰ πολλὰς τὰς καθʼ ἔκαστον αὐτῶν ὠδῖνας ἠναγκασμένην τὴν εἰς αὐτοὺς ἔχειν συμπάθειαν,
\VS{8}διὰ τὸν πρὸς τὸν Θεὸν φόβον ὑπερεῖδεν τὴν τῶν τέκνων πρόσκαιρον σωτηρίαν.
\par }{\PP \VS{9}Οὐ μὴν δὲ, ἀλλὰ καὶ διὰ τὴν καλοκᾳγαθίαν τῶν υἱῶν, καὶ τὴν πρὸς τὸν νόμον αὐτῶν εὐπείθειαν, μείζων τὴν ἐν αὐτοῖς ἔσχεν φιλοστοργίαν.
\VS{10}Δίκαιοί τε γὰρ ἦσαν, καὶ σώφρονες, καὶ σώφρονες, καὶ ἀνδρεῖοι, καὶ μεγαλόψυχοι, καὶ φιλάδελφοι, καὶ μεγαλόψυχοι, καὶ μεψαλόψυχοι, καὶ φιλάδελφοι, καὶ φιλομήτορες οὕτως, ὥστε καὶ μέχρι θανάτου τὰ νόμιμα φυλάσσοντες πείθεσθαι αὐτῇ.
\par }{\PP \VS{11}Ἀλλʼ ὅμως, καὶ ὑπὲρ τοσούτων ὄντων τῶν περὶ φιλοτεκνίαν εἰς συμπάθειαν ἑλκόντων τὴν μητέρα, ἐπʼ οὐδενὸς αὐτῶν τὸν λογισμὸν αὐτῆς αἱ παμποίκιλοι ἴσχυσαν μετατρέψαι.
\VS{12}Ἀλλὰ καἰ καθʼ ἔνα παῖδα καὶ ὁμοῦ πάντας ἡ μήτηρ ἐπὶ τὸν τῆς εὐσεβείας προετρέπετο θάνατον.
\VS{13}Ὦ φύσις ἱερὰ, καὶ φίλτρα γονέων καὶ γονεῦσιν φιλόστοργε, καὶ τροφεῖα, καὶ μητέρων ἀδάμαστα πάθη.
\par }{\PP \VS{14}Καθʼ ἕνα στρεβλούμενον καὶ φλεγόμενον ὁρῶσα υήτηρ, οὐ μετεβάλετο διὰ τὴν εὐσέβεβειαν.
\VS{15}Τὰς σάρκας τῶν τέκνων ἑώρα περὶ τὸ πῦρ τηκομένας, καὶ τοὺς τὼν ποδῶν καὶ χειρῶν δακτύλους ἐπὶ γῆς σπαίροντας, καὶ τὰς τῶν κεφαλῶν μέχρι τῶν περὶ τὰ γένεια σάρκας, ὥσπερ προσωπεῖα προκειμένας.
\par }{\PP \VS{16}Ὦ πικροτέρων μὲν νῦν μήτηρ πόνων πειρασθεῖσα, ἤπερ τῶν ἐπʼ αὐτοῖς ὠδίνων.
\VS{17}Ὦ μόνη γυνὴ τὴν εὐσέβειαν ὁλόκληρον ἀποκυήσασα.
\VS{18}Οὐ μετέρεψέν σε πρωτότοκος ἀποπνέων· οὐδὲ δεύτερον εἰς οἶκτρον βλέπων ἐν βασάνοις· οὐδὲ τρίτος ἀποψύχων.
\VS{19}Οὐδὲ τοὺς ὀφθαλμοὺς ἑνὸς ἑκάστου θεωροῦσα ταυρηδὸν ἐπὶ τῶν βασάνων ὁρῶντας τὸν αὐτὸν αἰκισμὸν, καὶ τοὺς μυκτῆρας προσημειουμένους αὐτῶν τὸν θάνατον, οὐκ ἔκλαυσας.
\VS{20}Ἐπὶ σαρξὶν τέκνων ὁρῶσα σάρκας τέκνων ἀποκεκομμένας, καὶ ἐπὶ κεφαλαῖς κεφαλὰς ἀποδειροτομουμένας, καὶ ἐπὶ νεκροῖς νεκροὺς πίπτοντας, καὶ πολυάνδριον ὁρῶσα τῶν τέκνων χορεῖον διὰ τῶν βασάνων, οὐκ ἐδάκρυσας.
\par }{\PP \VS{21}Οὐχ οὕτως σειρήνιοι μελῳδίαι, οὐδὲ κύκνειοι πρὸς φιληκοΐαν φωναὶ τοὺς ἀκούοντας ἐφέλκονται, ὦ τέκνων φωναὶ μετὰ βασάνων μητέρα φωνούντων.
\VS{22}Πηλίκαις καὶ πόσαις τότε ἡ μήτηρ, τῶν υἱῶν βασανιζομένων τροχοῖς τε καὶ καυτερίοις ἐβασανίζετο βασάνοις;
\par }{\PP \VS{23}Ἀλλὰ τὰ σπλάγχνα αὐτῆς ὁ εὐσεβὴς λογισμὸς ἐν αὐτοῖς τοῖς πάθεσιν ἀνδρειώσας ἐπέτεινεν τὴν πρόσκαιρον φιλοτεκνίαν παριδεῖν.
\VS{24}Καίπερ ἑπτὰ τέκνων ὁρῶσα ἀπώλειαν· ἀσπάσασα ἡ γενναῖα μήτηρ ἐξέδσεν διὰ τὴν πρὸς Θεὸν πίστιν.
\VS{25}Καθάπερ γὰρ ἐν βουλευτηρίῳ τῇ ἑαυτῆς ψυχῇ δεινοὺς ὁρῶσα συμβούλους, φύσιν καὶ γένεσιν καὶ φιλοτεκνίαν καὶ τέκνων στρέβλαν.
\VS{26}Δύο ψήφους κρατοῦσα μήτηρ, θανατηφόρον τε καὶ σωτήριον ὑπὲρ τένων·
\VS{27}Οὐκ ἐπέγνω τὴν σώζουσαν ἑπτὰ υἱοὺς πρὸς ὀλίγον χρόνον σωτηρίαν.
\VS{28}Ἀλλὰ τῆς θεοσεβοῦς Ἁβραὰμ καρτερίας ἡ θυγάτηρ ἐμνήσθη.
\par }{\PP \VS{29}Ὦ μήτηρ ἔθνους, ἔκδικε τοῦ νόμου, καὶ ὑπερασπίστεια τῆς εὐσεβείας, καὶ τοῦ διὰ σπλάγχνων ἀγῶνος ἀθλοφόρε.
\VS{30}Ὦ ἀῤῥένων πρὸς καρτερίαν γενναιοτέρα, καὶ ἀνδρῶν πρὸς ὑπομονὴν ἀνδρειοτέρα.
\VS{31}Καθάπερ γὰρ ἡ Νῶε κιβωτὸς ἐν τῷ κοσμοπληθεῖ κατακλυσμῷ κοσμοφοροῦσα καρτεροὺς ὑπήνεγκεν τοὺς κλύδωνας·
\VS{32}οὕτως σὺ, ἡ νομοφύλαξ, πανταχόθεν ἐν τῷ τῶν παθῶν περιαντλουμένη κατακλυσμῷ, καὶ καρτεροῖς ἂν λοιμοῖς ταῖς τῶν υἱῶν βασάνοις συνεχομένη, γενναίως ὑπέμεινας τοὺς τῆς εὐσεβείας χειμῶνας.

\par }\Chap{16}{\PP \VerseOne{1}Εἰ δὲ τοίνυν καὶ γυνὴ, καὶ γηραιὰ, καὶ ἑπτὰ παὶδων μήτηρ ὑπέμεινε τὰς μέχρι θανάτου βασάνους ὁρῶσα τῶν τέκνων· ὁμολογουμένως αὐτοκράτωρ ἐστὶν τῶν παθῶν ὁ εὐσεβὴς λογισμός.
\par }{\PP \VS{2}Ἀπέδειξα οὖν ὅτι οὐ μόνον τῶν παθῶν ἄνδρες ἐπεκράτησαν, ἀλλὰ καὶ γυνὴ τῶν μεγίστων βασάνων ὑπερεφρόνησεν.
\VS{3}Καὶ οὐχ οὕτως οἱ περὶ Δανιὴλ λέοντες ἦσαν ἄγριοι, οὐδὲ Μισαὴλ ἐκφλεγομένη κάμινος λαβροτάτῳ πυρὶ, ὡς τῆς φιλοτεκνίας περιέκαιεν ἐκείνη φύσις, ὁρῶσα αὑτῆς τοὺς ἑπτὰ υἱοὺς βασανιζομένους.
\VS{4}Ἀλλὰ τῷ λογισμῷ τῆς εὐσεβείας κατέσβεσε τοσαῦτα καὶ τηλικαῦτα πάθη ἡ μήτηρ.
\par }{\PP \VS{5}Καὶ γὰρ τοῦτο ἐπιλογίσασθαι, ὅτι εἰ δειλόψυχος ἦν ἡ γυνὴ, καίπερ μήτηρ οὖσα, ὠλοφύρετο ἂν ἐπʼ αὐτοῖς· καὶ ἴσως ἄν ταῦτα οὕτως εἶπεν,
\par }{\PP \VS{6}Ὦ μελέα ἔγωγε, καὶ πολλάκις τρισαθλία, ἥτις ἑπτὰ παῖδας τεκοῦσα, οὐδενὸς μήτηρ γεγένημαι.
\VS{7}Ὦ μάταιοι ἐπτὰ κυοφορίαι, καὶ ἀνόνητοι ἐπτὰ δεκάηνοι, καὶ ἄκαρποι τιθηνίαι, καὶ ταλαίπωροι γαλακτοτροφίοι.
\VS{8}Μάτην ἐφʼ ὑμῖν, ὦ παῖδες, πολλὰς ὑπέμεινα ὠδῖνας καὶ χαλεπωτέρας φροντίδας ἀνατροφῆς.
\VS{9}Ὦ τῶν ἐμῶν παίδων, οἱ μὲν ἄγαμοι, οἱ δὲ γαμήσαντες ἀνόνητοι, οὐκ ὄψομαι ὑμῶν τέκνα, οὐδὲ μάμμη κληθεῖσα μακαρισθήσομαι.
\VS{10}Ὦ ἡ πολύπαις καὶ καλλίπαις ἐγὼ γυνὴ χήρα καὶ μόνη πολύθρηνος.
\VS{11}Οὐδʼ ἂν ἀποθάνω, θάπτοντα τῶν υἱῶν ἕξω τινά.
\par }{\PP \VS{12}Ἀλλὰ τούτῳ τῷ θρήνῳ οὐδένα ὠλοφύρετο ἡ ἱερὰ καὶ θεοσεβὴς μήτηρ. Οὐδʼ ἵνα μὴ ἀποθάνωσιν ἀπέτρεπεν αὐτῶν τινα, οὐδʼ ὡς ἀποθνησκόντων ἐλυπήθη.
\VS{13}Ἀλλʼ ὥσπερ ἀδαμάντινον ἔχουσα τὸν νοῦν, καὶ εἰς ἀθανασίαν ἀνατίκτουσα τὸν τῶν υἱῶν ἀριθμὸν, μᾶλλον ὑπὲρ τῆς εὐσεβείας ἐπὶ τὸν θάνατον αὐτοὺς προετρέπετο ἱκετεύουσα.
\par }{\PP \VS{14}Ὦ διʼ εὐσέβειαν Θεοῦ στρατιῶτι, πρεσβύτι καὶ γυνὴ διὰ καρτερίαν καὶ τύραννον ἐνίκησας, καὶ ἔργοις δυνατωτέρα καὶ λόγοις εὑρέθης ἄνανδρος.
\VS{15}Καὶ γὰν ὃτε συνελήφθης μετὰ τῶν παίδων, εἱστήκεις τὸν Ἐλεάζαρον ὁρῶσα βασανιζόμενον, καὶ ἔλεγες τοῖς παισὶν ἐν τῇ Ἑβοαΐδι φωνῇ,
\par }{\PP \VS{16}Ὦ παῖδες, γενναῖος ὁ ἀγών· εφ ὃν κληθέντες ὑπὲρ τῆς διαμαρτυρίας τοῦ ἔθνους, ἐναγωνίσασθε προθύμως ὑπὲρ τοῦ πατρίουνόμου.
\VS{17}Καὶ γὰρ αἰσχρὸν τὸν μὴν γέροντα τοῦτον ὑπομένειν τὰς διὰ τὴν εὐσέβειαν ἀληδόνας, ὑμᾶς δὲ τοὺς νεωτέρους καταπλαγῆναι τὰς βασάνους.
\par }{\PP \VS{18}Ἀναμνήσθητε, ὅτι διὰ τὸν Θεὸν τοῦ κόσμου μετελάβετε, καὶ τοῦ βίου ἀπελαύσατέ·
\VS{19}καὶ διὰ τιῦτο ὀφείλετε πάντα πόνον ὑπομένειν διὰ τὸν Θεόν.
\VS{20}Διʼ ὃν καὶ ὁ πατὴρ ἡμῶν Ἁβραὰμ ἔσπευδεν τὸν ἐθνοπάτορα υἱὸν σφαγιάσαι Ἰσαὰκ, καὶ τὴν πατρῷαν χεῖρα ξιφηφόρον καταφερομένην ἐπʼ αὐτὸν ὁρῶν οὐκ ἔπτηξεν.
\VS{21}Καὶ Δανιὴλ ὁ δίκαιος εἰς λέοντας ἐβλήθη· καὶ Ἀνανίας, καὶ Ἀζαρίας, καὶ Μισαὴλ εἰς κάμινον πυρὸς ἀπεσφενδονήθησαν, καὶ ὑπέμειναν, διὰ τὸν Θεόν.
\VS{22}Καὶ ὑμεῖς οὖν τὴν αὐτὴν πίστιν πρὸς τὸν Θεὸν ἔχοντες, μὴ χαλεπαίνητε.
\VS{23}Ἀλόγιστον γὰρ εἰδότας εὐσέβειαν μὴ ἀντιστασθαι τοῖς πόνοις.
\par }{\PP \VS{24}Διὰ τούτων τῶν λόγων ἡ ἑπταμήτωρ ἕνα ἕκαστον τῶν υἱῶν παρακαλοῦσα, ἔπεισε μᾶλλον, ἢ παραβῆναι τὴν ἐντολὴν τοῦ Θεοῦ.
\VS{25}Ἔτι δὲ καὶ ταῦτα ἰδόντες, ὅτι διὰ τὸν Θεὸν ἀποθανόντες ζῶσιν τῷ Θεῷ, ὥσπερ Ἁβραὰμ καὶ Ἰσαὰκ καὶ Ἰακὼβ, καὶ πάντες οἱ πατριάρχαι.

\par }\Chap{17}{\PP \VerseOne{1}Ἔλεγον δὲ καὶ τῶν δορυφόρων τινὲς, ὡς ὅτε ἔμελλεν καὶ αὐτὴ συλλαμβάνεσθαι πρὸς θάνατον, ἵνα μὴ ψαύσειέν τι τοῦ σώματος ἑαυτῆς, ἑαυτὴν ἔῤῥιψεν κατὰ τῆς πυρᾶς.
\par }{\PP \VS{2}Ὦ μήτηρ σὺν ἑπτὰ παισὶν καταλύσασα τὴν τοῦ τυράννου βίαν, καὶ ἀκυρώσασα τὰς κακὰς ἐπινοίας αὐτοῦ, καὶ ἐπιδείξασα τὴν τῆς πίστεως γενναιότητα.
\VS{3}Καθάπερ γὰρ σὺ στέγη ἐπὶ τοῦ στύλου τῶν παίδων γενναίως ἱδρυμένη, ἀκλινῶς ὑπήνεγκας τὸν διὰ τῶν βασάνων σεισμόν.
\par }{\PP \VS{4}Θάῤῥει τοιγαροῦν, ὦ μήτηρ ἱερόψυχε, τὴν ἐλπίδα τῆς ὑπομονῆς γενναίως ἔχουσα πρὸς Θεόν.
\VS{5}Οὐχ οὕτω σελήνη κατʼ οὐρανὸν σὺν ἄστροις σεμνὴ καθέστηκεν, ὡς σὺ τοὺς εἰς ἀστέρας ἑπτὰ παῖδας φωταγωγήσασα πρὸς τὴν εὐσέβειαν ἔντιμος καθέστηκας Θεῷ, καὶ ἐστήρισαι ἐν οὐρανῷ σὺν αὐτοῖς.
\VS{6}Ἦν γὰρ ἡ παιδοποιΐα σου ἀπὸ Ἁβραὰμ τοῦ παιδός.
\par }{\PP \VS{7}Εἰ δὲ ἐξὸν ἡμῖν ἦν, ὥσπερ τινὸς ζωγραφῆσαι τὴν τῆς ἱστορίας σου εὐσέβειαν, οὐκ ἂν ἔφριττον οἱ θεωροῦντες μητέρα ἑπτὰ τέκνων διʼ εὐσέβειαν ποικίλας βασάνους μέχρι θανάτου ὑπομείνασαν.
\VS{8}Καὶ γὰρ ἄξιον ἦν καὶ ἐπὶ αὐτοῦ τοῦ ἐπιταφίου ἀναγράψαι καὶ ταῦτα τοῖς ἀπὸ τοῦ ἔθνους εἰς μνείαν λεγόμενα.
\VS{9}Ἐνταῦθα γέρων ἱερεὺς, καὶ γυμὴ γεραιὰ καὶ ἑπτὰ παῖδες ἐγκεκήδευνται διὰ τυράννου βίαν, τὴν Ἑβραίων πολιτείαν καταλῦσαι θέλοντος.
\VS{10}Οἳ καὶ ἐξεδίκησαν τὸ ἔθνος εἰς Θεὸν ἀφορῶντες, καὶ μέχρι θανάτου τὰς βασάνους ὑπομείναντες.
\par }{\PP \VS{11}Ἀληθῶς γὰρ ἦν ἀγῶν θεῖος ὁ διʼ αὐτῶν γεγενημένος.
\VS{12}Ἠθλότει γὰρ τότε ἀρετὴ διʼ ὑπομονῆς δοκιμάζουσα τὸ νῖκος ἐν ἀφθαρσίᾳ ἐν ζωῇ πολυχρονίῳ.
\VS{13}Ἐλεάζαρ δὲ προηγωνίζετο· ἡ δὲ μήτηρ τῶν ἑπτὰ παίδων ἐνήθλει·
\VS{14}οἱ δὲ ἀδελφοὶ ἠγωνίζοντο· ὁ τύραννος ἀντηγωνίζετο· ὁ δὲ κόσμος καὶ ὁ τῶν ἀνθρώπων βίος ἐθεώρει.
\VS{15}Θεοσέβεια δὲ ἐνίκα, τοὺς ἑαυτῆς ἀθλητὰς στεφανοῦσα.
\par }{\PP \VS{16}Τίνες οὐκ ἐθαύμασαν τοὺς τῆς ἀληθείας νομοθεσίας ἀθλητὰς; τίνες οὐκ ἐξεπλάγησαν;
\VS{17}Αὐτός γέ τοι ὁ τύραννος καὶ ὅλον τὸν συνέδριον αὐτῶν ἐξεθαύμασαν αὐτῶν τὴν ὑπομονήν.
\VS{18}Διʼ ἣν καὶ τῷ θείῳ νῦν παρεστήκασιν θρόνῳ, καὶ τὸν μακάριον βιοῦσιν αἰῶνα.
\VS{19}Καὶ γάρ φησιν ὁ Μωσῆς, καὶ πάντες οἱ ἡγιασμένοι ὑπὸ τὰς χεῖράς σου.
\par }{\PP \VS{20}Καὶ οὗτοι οὖν ἁγιασθέντες διὰ Θεὸν τετίμηνται οὐ μόνον οὖν ταύτῃ τῇ τιμῇ, ἀλλὰ καὶ τῷ διʼ αὐτοὺς τὸ ἔθνος ἡμῶν τοὺς πολεμίους μὴ ἐπικρατήσας,
\VS{21}καὶ τὸν τύραννον τιμωρηθῆναι, καὶ τὴν πατρίδα καθαρισθῆναι,
\VS{22}ὥσπωρ ἀντίψυχον γεγονότας τῆς τοῦ ἔθνους ἁμαρτίας, καὶ διὰ τοῦ αἵματος τῶν εὐσεβῶν ἐκείνων, καὶ τοῦ ἱλαστηρίου θανάτου αὐτῶν, ἡ θεία πρόνοια τὸν Ἰσραὴλ προκακωθέντα διέσωσεν.
\par }{\PP \VS{23}Πρὸς γὰρ τὴν ἀνδρείαν αὐτῶν τῆς ἀρετῆς, καὶ τὴν ἐπὶ ταῖς βασάνοις αὐτῶν ὑπομονὴν ὁ τύραννος ἀφιδὼν Ἀντίοχος ἀνεκήρυξεν τοῖς στρατιώταις αὐτοῦ εἰς ὑπόδειγμα τὴν ἐκείνων ὑπομονήν.
\VS{24}Ἔσχεν τε αὐτοὺς γενναίους καὶ ἀνδρείους εἰς πεζομαχίαν καὶ πολιορκίαν· καὶ ἐκπορθήσας ἐνίκησεν πάντας τοὺς πολεμίους.

\par }\Chap{18}{\PP \VerseOne{1}Ὦ τῶν Ἁβραμιαίων σπερμάτων ἀπόγονοι παῖδες Ἰσραηλῖται, πείθεσθε τῷ νόμῳ τούτῳ, καὶ πάντα τρόπον εὐσεβεῖτε·
\VS{2}γινώσκοντες, ὅτι τῶν παθῶν δεσπότης ἐστὶν ὁ εὐσεβὴς λογισμός· καὶ οὐ μόνον τῶν ἔνδοθεν, ἀλλὰ καὶ τῶν ἔξωθεν πόνων·
\par }{\PP \VS{3}Ἀνθʼ ὦν διὰ τὴν εὐσέβειαν προϊέμενοι τὰ σώματα τοῖς πόνοις ἐκεῖνοι, οὐ μόνον ὑπὸ τῶν ἀνθρώπων ἐθαυμάσθησαν, ἀλλὰ καὶ θείας μερίδος κατηξιώθησαν.
\VS{4}Καὶ διʼ αὐτοὺς εἰρήνευσεν τὸ ἔθνος, καὶ τὴν εὐνομίαν τὴν ἐπὶ τῆς πατρίδος ἀνανεωσάμενος, ἐκπεπολιόρκηκε τοὺς πολεμίους.
\VS{5}Καὶ ὁ τύραννος Ἀντίοχος καὶ ἐπὶ γῆς τετιμώρηται, καὶ ἀποθανὼν κολάζεται· ὡς γὰρ οὐδὲν οὐδαμῶς ἴσχυσεν ἀναγκάσαι τοὺς Ἱεροσολυμίτας ἀλλοφυλῆσαι, καὶ τῶν πατριῶν ἐθνῶν ἐκδιαιτηθῆναι·
\VS{6}τότε δὴ ἀπάρας ἀπὸ τῶν Ἱεροσολύμων ἐστρατοπέδευσεν ἐπὶ Πέρσας.
\par }{\PP \VS{7}Ἔλεγεν δὲ ἡ μήτηρ τῶν ἑπτὰ παίδων καὶ ταῦτα ἡ δικαία τοῖς τέκνοις, ὅτι ἐγὼ ἐγενήθην παρθένος ἁγνὴ, καὶ οὐχ ὑπερέβην πατρικὸν οἶκον· ἐφύλασσον δὲ τὴν ᾠκοδομουμένην πλευράν.
\VS{8}Οὐ διέφθειρέν με λυμεὼν τῆς ἐρημίας φθορεὺς ἐν πεδίῳ· οὐδὲ ἐλυμῄνατό μου τὰ ἁγνὰ τῆς παρθενίας λυμεὼν ἀπατηλὸς ὄφις· ἔμεινα δὲ χρόνον ἀκμῆς σὺν ἀνδρί.
\par }{\PP \VS{9}Τούτων δὲ ἐνελίκων γενομένων ἐτελεύτησεν ὁ πατήρ· μακάριος μὲν ἐκεῖνος· τὸν γὰρ τῆς εὐτεκνίας βίον ἐπιζητήσας, τὸν τῆς ἀτεκνίας οὐκ ὠδυνήθη καιρόν.
\VS{10}Ὃς ἐδιδασκεν ὑμᾶς, ἔτι ὢν σὺν ὑμῖν, τὸν νόμον καὶ τοὺς προφήτας.
\par }{\PP \VS{11}Τὸν ἀναιρεθέντα Ἀβὲλ ὑπὸ Κάϊν ἀνεγίνωσκεν δὲ ἡμῖν, καὶ τὸν ὁλοκαπούμενον Ἰσαὰκ, καὶ τὸν ἐν φυλακῇ Ἰωσήφ.
\VS{12}Ἔλεγεν δὲ ἡμῖν τὸν ζηλωτὴν Φινεές· ἐδίδασκεν δὲ ὑμᾶς τοὺς ἐν πυρὶ Ανανίαν, καὶ Ἀζαρίαν, καὶ Μισαήλ.
\VS{13}Ἐδόξαζεν δὲ καὶ τὸν ἐν λάκκῳ λεόντων Δανιὴλ, ὃν καὶ ἐμακάριζεν.
\par }{\PP \VS{14}Ὑπεμίμνησκεν δὲ ὑμᾶς τῆν Ἠσαΐου γραφὴν τὴν λέγουσαν, κᾂν διὰ πυρὸς διέλθῃς, φλὸξ οὐ κατακαύσει σε.
\VS{15}Τὸν ὑμνογράφον ἐμελῴδει ὑμῖν Δαρίδ τὸν λέγοντα, πολλαὶ αἱ θλίψεις τῶν δικαίων.
\VS{16}Τὸν Σαλομῶντα ἐπαροιμίαζεν ἡμῖν τὸν λέγοντα, ξύλον ζωῆς ἐστιν πᾶσιν τοῖς ποιοῦσιν αὐτοῦ τὸ θέλημα.
\VS{17}Τὸν Ἰεζεκιὴλ ἐπιστοποιεῖτο τὸν λέγοντα, εἰ ζήσεται τὰ ὀστᾶ τὰ ξηρὰ ταῦτα;
\VS{18}Ὧδην μὲν γὰρ ἣν ἑδίδαξεν Μωϋσῆς οὐκ ἐπελάθετο τὴν διδάσκουσαν, ἐγὼ ἀποκτενῶ καὶ ζῇν ποιήσω.
\VS{19}Αὗτη ἡ ζωὴ ἡμῶν καὶ ἡ μακαριότης τῶν ἡμερῶν.
\par }{\PP \VS{20}Ὧ πικρᾶς τῆς τότε ἡμέρας, καὶ οὐ πικρᾶς, ὅτε ὁ πικρὸς Ἑλλήμων τύραννος πῦρ φλέξας λέβησιν ὠμοῖς, καὶ ζεουσι θυμοῖς ἀγαγὼν ἐπὶ τὸν καταπέλτην καὶ πάλιν τὰς βασάνους αὐτοῦ τοὺς ἑπτὰ παῖδας τῆς Ἁβρααμίτιδος.
\VS{21}Τὰς τῶν ὀμμάτων κόρας ἐπήρωσεν, καὶ γλώσσας ἐξέτεμεν, καὶ βασάνοις ποικίλαις ἀπέκτεινεν.
\VS{22}Ὑπὲρ ὧν ἡ θεία δίκη μετῆλθεν καὶ μετελεύσεται τὸν ἀλάστορα.
\par }{\PP \VS{23}Οἱ δὲ Ἁβραμιαῖοι παῖδες σὺν τῇ ἀθλοφορῳ μητρὶ, εἰς πατέρων χορὸν συναγελάζονται, φυχὰς καὶ ἀθανάτους ἀπειληφότες παρὰ τοῦ Θεοῦ.
\VS{24}Ὦ ἡ δόξα εἰς τοὺς αἰῶνας τῶν αἰώνων. Ἀμήν.
\par }