\NormalFont\ShortTitle{ΜΑΚΚΑΒΑΙΩΝ Βʹ}
{\MT ΜΑΚΚΑΒΑΙΩΝ Βʹ

\par }\ChapOne{1}{\PP \VerseOne{1}ΤΟΙΣ ἀδελφοῖς τοῖς κατʼ Αἴγυπτον Ἰουδαίοις χαίρειν· οἱ ἀδελφοὶ οἱ ἐν Ἱεροσολύμοις Ἰουδαῖοι, καὶ οἱ ἐν τῇ χώρᾳ τῆς Ἰουδαίας, εἰρήνην ἀγαθήν.
\par }{\PP \VS{2}Καὶ ἀγαθοποιήσαι ὑμῖν ὁ Θεὸς, καὶ μνησθείη τῆς διαθήκης αὐτοῦ τῆς πρὸς Ἁβραὰμ, καὶ Ἰσαὰκ, καὶ Ἰακὼβ τῶν δούλων αὐτοῦ τῶν πιστῶν.
\VS{3}Καὶ δῴη ὑμῖν καρδίαν πᾶσιν εἰς τὸ σέβεσθαι αὐτὸν, καὶ ποιεῖν αὐτοῦ τὰ θελήματα καρδίᾳ μεγάλῃ, καὶ ψυχῇ βουλομένῃ·
\VS{4}Καὶ διανοίξαι τὴν καρδίαν ὑμῶν ἐν τῷ νόμῳ αὐτοῦ, καὶ ἐν τοῖς προστάγμασι, καὶ εἰρήνην ποιήσαι,
\VS{5}καὶ ἐπακούσαι ὑμῶν τῶν δεήσεων, καὶ καταλλαγείη ὑμῖν, καὶ μὴ ὑμᾶς ἐγκαταλίποι ἐν καιρῷ πονηρῷ.
\VS{6}Καὶ νῦν ὧδέ ἐσμεν προσευχόμενοι περὶ ὑμῶν.
\par }{\PP \VS{7}Βασιλεύοντος Δημητρίου ἔτους ἑκατοστοῦ ἑξηκοστοῦ ἐννατου, ἡμεῖς ὁ Ιουδαῖοι γεγραφήκαμεν ὑμῖν ἐν τῇ θλίψει, καὶ ἐν τῇ ἀκμῇ τῇ ἐπελθούσῃ ἡμῖν ἐν τοῖς ἔτεσι τούτοις, ἀφʼ οὗ ἀπέστη Ἰάσων καὶ οἱ μετʼ αὐτοῦ ἀπὸ τῆς ἁγίας γῆς, καὶ τῆς βασιλείας·
\VS{8}καὶ ἐνεπύρισαν τὸν πυλῶνα, καὶ ἐξέχεαν αἷμα ἀθῶον· καὶ ἐδεήθημεν τοῦ Κυρίου, καὶ εἰσηκούσθημεν, καὶ προσηνέγκαμεν θυσίαν, καὶ σεμίδαλιν, καὶ ἐξήψαμεν τοὺς λύχνους, καὶ προεθήκαμεν τοὺς ἄρτους.
\VS{9}Καὶ νῦν ἵνα ἄγητε τὰς ἡμέρας τῆς σκηνοπηγίας τοῦ Χασελεὺ μηνός.
\par }{\PP \VS{10}Ἔτους ἑκατοστοῦ ὀγδοηκοστοῦ καὶ ὀγδόου οἱ ἐν Ἱεροσολύμοις, καὶ οἱ ἐν τῇ Ἰουδαίᾳ, καὶ ἡ γερουσία, καὶ Ἰούδας Ἀριστοβούλῳ διδασκάλῳ Πτολεμαίου τοῦ βασιλέως, ὄντι δὲ ἀπὸ τοῦ τῶν χριστῶν ἱερέων γένους, καὶ τοῖς ἐν Αἰγύπτῳ Ἰουδαίοις, χαίρειν καὶ ὑγιαίνειν.
\par }{\PP \VS{11}Ἐκ μεγάλων κινδύνων ὑπὸ τοῦ Θεοῦ σεσωσμένοι, μεγάλως εὐχαριστοῦμεν αὐτῷ, ὡς ἂν πρὸς βασιλέα παρατασσόμενοι.
\VS{12}Αὐτὸς γὰρ ἐξέβρασε τοὺς παραταξαμένους ἐν τῇ ἁγίᾳ πόλει.
\par }{\PP \VS{13}Εἰς γὰρ τὴν Περσίδα γενόμενος ὁ ἡγεμὼν, καὶ ἡ περὶ αὐτὸν ἀνυπόστατος δοκοῦσα εἶναι δύναμις, κατεκόπησαν ἐν τῷ τῆς Ναναίας ἱερῷ, παραλογισμῷ χρησαμένων τῶν περὶ τὴν Ναναίαν ἱερέων.
\VS{14}Ὡς γὰρ συνοικήσων αὐτῇ παρεγένετο εἰς τὸν τόπον ὅ, τε Ἀντίοχος, καὶ οἱ σὺν αὐτῷ φίλοι, χάριν τοῦ λαβεῖν τὰ χρήματα εἰς φερνῆς λόγον·
\VS{15}Καὶ προθέντων αὐτὰ τῶν ἱερέων τῆς Ναναίος, κᾀκείνου προσελθόντος μετʼ ὀλίγων εἰς τὸν περίβολον τοῦ τεμένους, συγκλείσαντες τὸ ἱερὸν, ὡς εἰσῆλθεν Ἀντίοχος,
\VS{16}ἀνοίξαντες τὴν τοῦ φατνώματος κρυπτὴν θύραν, βάλλοντες πέτρους συνεκεραύνωσαν τὸν ἡγεμόνα, καὶ μέλη ποιήσαντες, καὶ τὰς κεφαλὰς ἀφελόντες, τοῖς ἔξω παρέῤῥιψαν.
\par }{\PP \VS{17}Κατὰ πάντα εὐλογητὸς ἡμῶν ὁ Θεὸς, ὃς παρέδωκε τοὺς ἀσεβήσαντας.
\par }{\PP \VS{18}Μέλλοντες οὖν ἄγειν ἐν τῷ Χασελεὺ πέμπτῃ καὶ εἰκάδι τὸν καθαρισμὸν τοῦ ἱεροῦ, δεόν ἡγησάμεθα διασαφῆσαι ὑμῖν, ἵνα καὶ αὐτοὶ ἄγητε τῆς σκηνοπηγίας καὶ τοῦ πυρός, ὅτε Νεεμίας οἰκοδομήσας τό, τε ἱερὸν καὶ τὸ θυσιαστήριον, ἀνήνεγκε θυσίαν.
\VS{19}Καὶ γὰρ ὅτε εἰς τὴν Περσικὴν ἤγοντο οἱ πατέρες ἡμῶν, οἵ τότε εὐσεβεῖς ἱερεῖς λαβόντες ἀπὸ τοῦ πυρὸς τοῦ θυσιαστηρίου λαθραίως, κατέκρυψαν ἐν κοιλώματι φρέατος τάξιν ἔχοντος ἀνύδρον, ἐν ᾧ κατησφαλίσαντο, ὥστε πᾶσιν ἄγνωστον εἶναι τὸν τόπον.
\par }{\PP \VS{20}Διελθόντων δὲ ἐτῶν ἱκανῶν, ὅτε ἔδοξεν τῷ Θεῷ, ἀποσταλεὶς Νεεμίας ὑπὸ τοῦ βασιλέως τῆς Περσίδος, τοὺς ἐκγόνους τῶν ἱερέων τῶν ἀποκρυψάντων ἔπεμψεν ἐπὶ τὸ πῦρ·
\VS{21}ὡς δὲ διεσάφησαν ἡμῖν μὴ εὑρηκέναι πῦρ, ἀλλὰ ὕδωρ παχύ, ἐκέλευσεν αὐτοὺς ἀποβάψαντας φέρειν· ὡς δὲ ἀνηνέχθη τὰ τῶν θυσιῶν, ἐκέλευσε τοὺς ἱερεῖς Νεεμίας ἐπιῤῥᾶναι τῷ ὕδατι τά τε ξύλα, καὶ τὰ ἐπικείμενα.
\VS{22}Ὡς δὲ ἐγένετο τοῦτο, καὶ χρόνος διῆλθεν ὅτε ἥλιος ἀνέλαμψε πρότερον ἐπινεφὴς ὤν, ἀνήφθη πυρὰ μεγάλη, ὥστε θαυμάσαι πάντας.
\par }{\PP \VS{23}Προσευχὴν δὲ ἐποιήσαντο οἱ ἱερεῖς δαπανωμένης τῆς θυσίας, οἵ τε ἱερεῖς, καὶ πάντες, καταρχομένου Ἰωνάθου, τῶν δὲ λοιπῶν ἐπιφωνούντων, ὡς Νεεμίου.
\par }{\PP \VS{24}Ἦν δὲ ἡ προσευχὴ τὸν τρόπον ἔχουσα τοῦτον· Κύριε Κύριε ὁ Θεὸς ὁ πάντων κτίστης, ὁ φοβερὸς, καὶ ἰσχυρὸς, καὶ δίκαιος, καὶ ἐλεήμων, ὁ μόνος βασιλεὺς καὶ χρηστὸς,
\VS{25}ὁ μόνος χορηγός, ὁ μόνος δίκαιος, καὶ παντοκράτωρ, καὶ αἰώνιος, ὁ διασώζων τὸν Ἰσραὴλ ἐκ παντος κακοῦ, ὁ ποιήσας τοὺς πατέρας ἐκλεκτούς, καὶ ἁγιάσας αὐτούς,
\VS{26}πρόσδεξαι τὴν θυσίαν ὑπὲρ παντὸς τοῦ λαοῦ σου Ἰσραὴλ, καὶ διαφύλαξον τὴν μερίδα σου καὶ καθαγίασον.
\VS{27}Ἐπισυνάγαγε τὴν διασπορὰν ἡμῶν, ἐλευθέρωσον τοὺς δουλεύοντας ἐν τοῖς ἔθνεσι, τοὺς ἐξουθενημένους καὶ βδελυκτοὺς ἔπιδε, καὶ γνώτωσαν τὰ ἔθνη ὅτι σὺ εἶ ὁ Θεὸς ἡμῶν.
\VS{28}Βασάνισον τοὺς καταδυναστεύοντας, καὶ ἐξυβρίζοντας ἐν ὑπερηφανίᾳ·
\VS{29}Καταφύτευσον τὸν λαόν σου εἰς τὸν τόπον τὸν ἅγιόν σου, καθὼς εἶπε Μωυσῆς.
\VS{30}Οἱ δὲ ἱερεῖς ἐπέψαλλον τοὺς ὕμνους.
\par }{\PP \VS{31}Καθὼς δὲ ἀνηλώθη τὰ τῆς θυσίας, καὶ τὸ περιλειπόμενον ὕδωρ, ὁ Νεεμίας ἐκέλευσε λίθους μείζονας καταχεῖν.
\VS{32}Ὡς δὲ τοῦτο ἐγενήθη, φλὸξ ἀνήφθη· τοῦ δὲ ἀπὸ τοῦ θυσιαστηρίου ἀντιλάμψαντος φωτὸς ἐδαπανήθη.
\par }{\PP \VS{33}Ὡς δὲ φανερὸν ἐγενήθη τὸ πρᾶγμα, καὶ διηγγέλη τῷ βασιλεῖ τῶν Περσῶν, ὅτι εἰς τὸν τόπον οὗ τὸ πῦρ ἄπέκρυψαν οἱ μεταχθέντες ἱερεῖς, τὸ ὕδωρ ἐφάνη, ἀφʼ οὗ καὶ οἱ περὶ τὸν Νεεμίαν ἥγνισαν τὰ τῆς θυσίας.
\VS{34}Περιφράξας δὲ ὁ βασιλεὺς ἱερὸν ἐποίησε, δοκιμάσας τὸ πρᾶγμα.
\par }{\PP \VS{35}Καὶ οἷς ἐχαρίζετο ὁ βασιλεὺς πολλὰ διάφορα ἐλάμβανε καὶ μετεδίδου.
\VS{36}Προσηγόρευσαν δὲ οἱ περὶ τὸν Νεεμίαν τοῦτο Νέφθαρ, ὃ διερμηνεύεται Καθαρισμός· καλεῖται δὲ παρὰ τοῖς πολλοῖς Νεφθαεί.

\par }\Chap{2}{\PP \VerseOne{1}Εὑρίσκεται δὲ ἐν ταῖς ἀπογραφαῖς Ἱερεμίας ὁ προφήτης, ὅτι ἐκέλευσε τοῦ πυρὸς λαβεῖν τοὺς μεταγινομένους, ὡς σεσήμανται,
\VS{2}καὶ ὡς ἐνετείλατο τοῖς μεταγενομένοις ὁ προφήτης, δοὺς αὐτοῖς τὸν νόμον, ἵνα μὴ ἐπιλάθωνται τῶν προσταγμάτων τοῦ Κυρίου, καὶ ἵνα μὴ ἀποπλανηθῶσι ταῖς διανοίαις, βλέποντες ἀγάλματα χρυσᾶ καὶ ἀργυρᾶ, καὶ τὸν περὶ αὐτὰ κόσμον.
\VS{3}Καὶ ἕτερα τοιαῦτα λέγων, παρεκάλει μὴ ἀποστῆναι τὸν νόμον ἀπὸ τῆς καρδίας αὐτῶν.
\par }{\PP \VS{4}Ἦν δὲ ἐν τῇ γραφῇ, ὡς τὴν σκηνὴν καὶ τὴν κιβωτὸν ἐκέλευσεν ὁ προφήτης, χρηματισμοῦ γενηθέντος, αὐτῷ συνακολουθεῖν, ὡς δὲ ἐξῆλθεν εἰς τὸ ὄρος οὗ ὁ Μωυσῆς ἀναβὰς ἐθεάσατο τὴν τοῦ Θεοῦ κληρονομίαν.
\VS{5}Καὶ ἐλθὼν ὁ Ἱερεμίας εὗρεν οἶκον ἀντρώδη, καὶ τὴν σκηνὴν, καὶ τὴν κιβωτὸν, καὶ τὸ θυσιαστήριον τοῦ θυμιάματος εἰσήνεγκεν ἐκεῖ, καὶ τὴν θύραν ἐνέφραξε.
\par }{\PP \VS{6}Καὶ προσελθόντες τινὲς τῶν συνακολουθούντων ὥστε ἐπισημῄνασθαι τὴν ὁδὸν, καὶ οὐκ ἠδυνήθησαν εὑρεῖν.
\VS{7}Ὡς δὲ ὁ Ἱερεμίας ἔγνω, μεμψάμενος αὐτοῖς εἶπεν, ὅτι καὶ ἄγνωστος ὁ τόπος ἔσται ἕως ἂν συναγάγῃ ὁ Θεὸς ἐπισυναγωγὴν τοῦ λαοῦ, καὶ ἵλεως γένηται.
\VS{8}Καὶ τότε ὁ Κύριος ἀναδείξει ταῦτα, καὶ ὀφθήσεται ἡ δόξα τοῦ Κυρίου καὶ ἡ νεφέλη, ὡς καὶ ἐπὶ Μωυσῇ ἐδηλοῦτο, ὡς καὶ ὁ Σαλωμὼν ἠξίωσεν ἵνα ὁ τόπος καθαγιασθῇ μεγάλως.
\par }{\PP \VS{9}Διεσαφεῖτο δὲ καὶ ὡς σοφίαν ἔχων ἀνήνεγκε θυσίαν ἐγκαινισμοῦ, καὶ τῆς τελειώσεως τοῦ ἱεροῦ.
\VS{10}Καθὼς καὶ Μωυσῆς προσηύξατο πρὸς Κύριον, καὶ κατέβη πῦρ ἐκ τοῦ οὐρανοῦ, καὶ τὰ τῆς θυσίας ἐδαπάνησεν· οὕτως καὶ Σαλωμὼν προσηύξατο, καὶ καταβὰν τὸ πῦρ ἀνήλωσε τὰ ὁλοκαυτώματα.
\VS{11}Καὶ εἶπε Μωυσῆς, διὰ τὸ μὴ βεβρῶσθαι τὸ περὶ τῆς ἁμαρτίας, ἀνηλώθη.
\VS{12}Ὡσαύτως καὶ ὁ Σαλωμὼν τὰς ὀκτὼ ἡμέρας ἤγαγεν.
\par }{\PP \VS{13}Ἐξηγοῦντο δὲ καὶ ἐν ταῖς ἀναγραφαῖς, καὶ ἐν τοῖς ὑπομνηματισμοῖς τοῖς κατὰ τὸν Νεεμίαν τὰ αὐτά, καὶ ὡς καταβαλλόμενος βιβλιοθήκην, ἐπισυνήγαγε τὰ περὶ τῶν βασιλέων καὶ προφητῶν, καὶ τὰ τοῦ Δαυεὶδ, καὶ ἐπιστολὰς βασιλέων περὶ ἀναθημάτων.
\VS{14}Ὡσαύτως δὲ καὶ Ἰούδας, τὰ διαπεπτωκότα διὰ τὸν πόλεμον τὸν γεγονότα ἡμῖν ἐπισυνήγαγε πάντα, καὶ ἔστι παρʼ ἡμῖν.
\VS{15}Ὧν οὖν ἐὰν χρείαν ἔχητε, τοὺς ἀποκομιοῦντας ὑμῖν ἀποστέλλετε.
\par }{\PP \VS{16}Μέλλοντες οὖν ἄγειν τὸν καθαρισμὸν, ἐγράψαμεν ὑμῖν· καλῶς οὖν ποιήσετε ἄγοντες τὰς ἡμέρας.
\VS{17}Ὁ δὲ Θεὸς ὁ σώσας τὸν πάντα λαὸν αὐτοῦ, καὶ ἀποδοὺς τὴν κληρονομίαν πᾶσι, καὶ τὸ βασίλειον, καὶ τὸ ἱεράτευμα, καὶ τὸν ἁγιασμὸν.
\VS{18}Καθὼς ἐπηγγείλατο διὰ τοῦ νόμου ἐλπίζομεν γὰρ ἐπὶ τῷ Θεῷ, ὅτι ταχέως ἡμᾶς ἐλεήσει, καὶ ἐπισυνάξει ἐκ τῆς ὑπὸ τὸν οὐρανὸν εἰς τὸν ἅγιον τόπον· ἐξείλετο γὰρ ἡμᾶς ἐκ μεγάλων κακῶν, καὶ τὸν τόπον ἐκαθάρισε.
\par }{\PP \VS{19}Τὰ δὲ κατὰ τὸν Ἰούδαν τὸν Μακκαβαῖον, καὶ τοὺς τούτου ἀδελφοὺς, καὶ τὸν τοῦ ἱεροῦ τοῦ μεγάλου καθαρισμὸν, καὶ τὸν τοῦ βωμοῦ ἐγκαινισμὸν,
\VS{20}ἔτι τε τοὺς πρὸς Ἀντιοχον τὸν Ἐπιφανῆ, καὶ τὸν τούτου υἱὸν Εὐπάτορα πολέμους,
\VS{21}καὶ τὰς ἐξ οὐρανοῦ γενομένας ἐπιφανείας τοῖς ὑπὲρ τοῦ Ἰουδαϊσμοῦ φιλοτίμως ἀνδραγαθήσασιν, ὡστε τὴν ὅλην χώραν ὀλίγους ὄντας λεηλατεῖν, καὶ τὰ βάρβαρα πλήθη διώκειν.
\VS{22}Καὶ τὸ περιβόητον καθʼ ὅλην τὴν οἰκουμένην ἱερὸν ἀνακομίσασθαι, καὶ τὴν πόλιν ἐλευθερῶσαι, καὶ τοὺς μέλλοντας καταλύεσθαι νόμους ἐπανορθῶσαι, τοῦ Κυρίου μετὰ πάσης ἐπιεικείας ἵλεω γενομένου αὐτοῖς,
\VS{23}τὰ ὑπὸ Ἰάσωνος τοῦ Κυρηναίου δεδηλωμένα διὰ πέντε βιβλίων, πειρασόμεθα διʼ ἑνὸς συντάγματος ἐπιτεμεῖν.
\par }{\PP \VS{24}Συνορῶντες γὰρ τὸ χύμα τῶν ἀριθμῶν, καὶ τὴν οὖσαν δυσχέρειαν τοῖς θέλουσιν εἰσκυκλεῖσθαι τοῖς τῆς ἱστορίας διηγήμασι διὰ τὸ πλῆθος τῆς ὕλης,
\VS{25}ἐφροντίσαμεν τοῖς μὲν βουλομένοις ἀναγινώσκειν ψυχαγωγίαν, τοῖς δέ φιλοφρονοῦσιν εἰς τὸ διὰ μνήμης ἀναλαβεῖν εὐκοπίαν, πᾶσι δὲ τοῖς ἐντυγχάνουσιν ὠφέλειαν.
\par }{\PP \VS{26}Καὶ ἡμῖν μὲν τοῖς τὴν κακοπάθειαν ἐπιδεδεγμένοις τῆς ἐπιτομῆς οὐ ῥᾴδιον, ἱδρῶτος δὲ καὶ ἀγρυπνίας τὸ πρᾶγμα·
\VS{27}καθάπερ τῷ παρασκευάζοντι συμπόσιον, καὶ ζητοῦντι τὴν ἑτέρων λυσιτέλειαν οὐκ εὐχερές μὲν, ὅμως διὰ τὴν τῶν πολλῶν εὐχαριστίαν, ἡδέως τὴν κακοπάθειαν ὑποίσομεν,
\VS{28}τὸ μὲν διακριβοῦν περὶ ἑκάστων τῷ συγγραφεῖ παραχωρήσαντες, τὸ δὲ ἐπιπορεύεσθαι τοῖς ὑπογραμμοῖς τῆς ἐπιτομῆς διαπονοῦντες.
\VS{29}Καθάπερ γὰρ τῆς καινῆς οἰκίας ἀρχιτέκτονι τῆς ὅλης καταβολῆς φροντιστέον, τῷ δὲ ἐγκαίειν καὶ ζωγραφεῖν ἐπιχειροῦντι, τὰ ἐπιτήδεια πρὸς διακόσμησιν ἐξεταστέον· οὕτω δοκῶ καὶ ἐπὶ ἡμῖν.
\VS{30}Τὸ μὲν ἐμβατεύειν, καὶ περί πάντων ποιεῖσθαι λόγον καὶ πολυπραγμονεῖν ἐν τοῖς καταμέρος, τῷ τῆς ἱστορίας ἀρχηγέτῃ καθήκει·
\VS{31}Τὸ δὲ σύντομον τῆς λέξεως μεταδιώκειν, καὶ τὸ ἐξεργαστικὸν τῆς πραγματείας παραιτεῖσθαι, τῷ τὴν μετάφρασιν ποιουμένῳ συγχωρητέον.
\VS{32}Ἐντεῦθεν οὖν ἀρξώμεθα τῆς διηγήσεως, τοῖς προειρημένοις τοσοῦτον ἐπιζεύξαντες· εὔηθες γὰρ τὸ μὲν πρὸ τῆς ἱστορίας πλεονάζειν, τὴν δὲ ἱστορίαν ἐπιτεμεῖν.

\par }\Chap{3}{\PP \VerseOne{1}Τῆς ἁγίας τοίνυν πόλεως κατοικουμένης μετὰ πάσης εἰρήνης, καὶ τῶν νόμων ἕτι κάλλιστα συντηρουμένων διὰ τὴν Ὀνίου τοῦ ἀρχιερέως εὐσέβειάν τε καὶ μισοπονηρίαν,
\VS{2}συνέβαινε καὶ αὐτοὺς τοὺς βασιλεῖς τιμᾷν τὸν τόπον, καὶ τὸ ἱερὸν ἀποστολαῖς ταῖς κρατίσταις δοξάζειν,
\VS{3}ὥστε καὶ Σέλευκον τὸν τῆς Ἀσίας βασιλέα χορηγεῖν ἐκ τῶν ἰδίων προσόδων πάντα τὰ πρὸς τὰς λειτουργίας τῶν θυσιῶν ἐπιβάλλοντα δαπανήματα.
\par }{\PP \VS{4}Σίμων δέ τις ἐκ τῆς Βενιαμὶν φυλῆς προστάτης τοῦ ἱεροῦ καθεσταμένος, διηνέχθη τῷ ἀρχιερεῖ περὶ τῆς κατὰ τὴν πόλιν παρανομίας·
\VS{5}καὶ νικῆσαι τὸν Ὀνίαν μὴ δυνάμενος, ἦλθε πρὸς Ἀπολλώνιον Θρασαίου, τὸν κατʼ ἐκεῖνον τὸν καιρὸν κοιλῆς Συρίας καὶ Φοινίκης στρατηγόν.
\VS{6}Καὶ προσήγγειλε περὶ τοῦ χρημάτων ἀμυθήτων γέμειν τὸ ἐν Ἱεροσολύμοις γαζοφυλάκιον, ὥστε τὸ πλῆθος τῶν διαφόρων ἐναρίθμητον εἶναι, καὶ μὴ προσήκειν αὐτὰ πρὸς τὸν τῶν θυσιῶν λόγον, εἶναι δὲ δυνατὸν ὑπὸ τὴν τοῦ βασιλέως ἐξουσίαν πεσεῖν ἅπαντα ταῦτα.
\par }{\PP \VS{7}Συμμίξας δὲ ὁ Ἀπολλώνιος τῷ βασιλεῖ, περὶ τῶν μηνυθέντων αὐτῷ χρημάτων ἐνεφάνισεν· ὁ δὲ προχειρισάμενος Ἡλιόδωρον τὸν ἐπὶ τῶν πραγμάτων, ἀπέστειλε δοὺς ἐντολὰς, τὴν τῶν προειρημένων χρημάτων ἐκκομιδὴν ποιήσασθαι.
\VS{8}Εὐθέως δὲ ὁ Ἡλιόδωρος ἐποιεῖτο τὴν παρείαν, τῇ μὲν ἐμφάσει ὡς τὰς κατὰ κοίλην Συρίαν καὶ Φοινίκην πόλεις ἐφοδεύσων, τῷ πράγματι δὲ τὴν τοῦ βασιλέως πρόθεσιν ἐπιτελέσων.
\par }{\PP \VS{9}Παραγενηθεὶς δὲ εἰς Ἱεροσόλυμα, καὶ φιλοφρόνως ὑπὸ τοῦ ἀρχιερέως τῆς πόλεως ἀποδεχθείς, ἀνέθετο περὶ τοῦ γεγονότος ἐμφανισμοῦ, καὶ τίνος ἕνεκεν πάρεστι διεσάφήσεν· ἐπυνθάνετο δὲ εἰ ταῖς ἀληθείαις ταῦτα οὕτως ἔχοντα τυγχάνει.
\par }{\PP \VS{10}Τοῦ δὲ ἀρχιερέως ὑποδείξαντος παραθήκας εἶναι χηρῶν τε καὶ ὀρφανῶν,
\VS{11}τινὰ δὲ καὶ Ὑρκανοῦ τοῦ Τωβίου σφόδρα ἀνδρὸς ἐν ὑπεροχῇ κειμένου, οὐχ ὥσπερ ἦν διαβάλλων ὁ δυσσεβὴς Σίμων, τὰ δὲ πάντα ἀργυρίου τετρακόσια τάλαντα, χρυσίου δὲ διακόσια·
\VS{12}ἀδικηθῆναι δὲ τοὺς πεπιστευκότας τῇ τοῦ τόπου ἁγιωσύνῃ, καὶ τῇ τοῦ τετιμημένου κατὰ τὸν σύμπαντα κόσμον ἱεροῦ σεμνότητι καὶ ἀσυλίᾳ, παντελῶς ἀμήχανον εἶναι.
\par }{\PP \VS{13}Ὁ δὲ Ἡλιόδωρος διʼ ἃς εἶχε βασιλικὰς ἐντολὰς, πάντως ἔλεγεν εἰς τὸ βασιλικὸν ἀναληπτέα ταῦτα εἶναι.
\VS{14}Ταξάμενος δὲ ἡμέραν εἰσῄει τὴν περὶ τούτων ἐπίσκεψιν οἰκονομήσων· ἦν δὲ οὐ μικρὰ καθʼ ὅλην τὴν πόλιν ἀγωνία.
\VS{15}Οἱ δὲ ἱερεῖς πρὸ τοῦ θυσιαστηρίου ἐν ταῖς ἱερατικαῖς στολαῖς ῥίψαντες ἑαυτοὺς, ἐπεκαλοῦντο εἰς οὐρανὸν τὸν περὶ παραθήκης νομοθετήσαντα τοῖς παρακαταθεμένοις ταῦτα σῶα διαφυλάξαι.
\par }{\PP \VS{16}Ἦν δὲ ὁρῶντα τὴν τοῦ ἀρχιερέως ἰδέαν, τιτρώσκεσθαι τὴν διάνοιαν· ἡ γὰρ ὄψις καὶ τὸ τῆς χρόας παρηλλαγμένον ἐνέφαινε τὴν κατὰ ψυχὴν ἀγωνίαν.
\VS{17}Περιεκέχυτο γὰρ περὶ τὸν ἄνδρα δέος τι καὶ φρικασμὸς σώματος, διʼ ὧν πρόδηλον ἐγένετο τοῖς θεωροῦσι τὸ κατὰ καρδίαν ἐνεστὸς ἄλγος.
\par }{\PP \VS{18}Οἱ δὲ ἐκ τῶν οἰκιῶν ἀγεληδὸν ἐξεπήδων ἐπὶ πάνδημον ἱκετείαν, διὰ τὸ μέλλειν εἰς καταφρόνησιν ἔρχεσθαι τὸν τόπον.
\VS{19}Ὑπεζωσμέναι δὲ ὑπὸ τοὺς μαστοὺς αἱ γυναῖκες σάκκους κατὰ τὰς ὁδοὺς ἐπλήθυον· αἱ δὲ κατάκλειστοι τῶν παρθένων, αἱ μὲν συνέτρεχον ἐπὶ τοὺς πυλῶνας, αἱ δὲ ἐπὶ τὰ τείχη, τινὲς δὲ διὰ τῶν θυρίδων διεξέκυπτον.
\VS{20}Πᾶσαι δὲ προτείνουσαι τὰς χεῖρας εἰς τὸν οὐρανον, ἐποιοῦντο τὴν λιτανείαν·
\par }{\PP \VS{21}Ἐλεεῖν δʼ ἦν τὴν τοῦ πλήθους παμμιγῆ πρόπτωσιν, τήν τε τοῦ μεγάλως διαγωνιῶντος ἀρχιερέως προσδοκίαν.
\VS{22}Οἱ μὲν οὖν ἐπεκαλοῦντο τὸν παντοκράτορα Θεὸν τὰ πεπιστευμένα τοῖς πεπιστευκόσι σῶα διαφυλάγγειν μετὰ πάσης ἀσφαλείας.
\par }{\PP \VS{23}Ὁ δὲ Ἡλιόδωρος τὸ διεγνωσμένον ἐπετέλει.
\par }{\PP \VS{24}Αὐτόθι δὲ αὐτοῦ σὺν τοῖς δορυφόροις κατὰ τὸ γαζοφυλάκιον ἤδη παρόντος, ὁ τῶν πατέρων Κύριος καὶ πάσης ἐξουσίας δυνάστης ἐπιφάνειαν μεγάλην ἐποίησεν, ὥστε πάντας τοὺς κατατολμήσαντας συνελθεῖν, καταπλαγέντας τὴν τοῦ Θεοῦ δύναμιν, εἰς ἔκλυσιν καὶ δειλίαν τραπῆναι.
\VS{25}Ὤφθη γάρ τις ἵππος αὐτοῖς φοβερὸν ἔχων τὸν ἐπιβάτην, καὶ καλλίστῃ σαγῇ διακεκοσμημένος, φερόμενος δὲ ῥύδην ἐνέσεισε τῷ Ἡλιοδώρῳ τὰς ἐμπροσθίους ὁπλάς· ὁ δὲ ἐπικαθήμενος ἐφαίνετο χρυσῆν πανοπλίαν ἔχων.
\par }{\PP \VS{26}Ἕτεροι δὲ δύο προεφάνησαν αὐτῷ νεανίαι, τῇ ῥώμῃ μὲν ἐκπρεπεῖς, κάλλιστοι δὲ τῇ δόξῃ, διαπρεπεῖς δὲ τὴν περιβολήν· οἳ καὶ παρασταντες ἐξ ἑκατέρου μέρους, ἐμαστίγουν αὐτὸν ἀδιαλείπτως, πολλὰς ἐπιῤῥιπτοῦντες αὐτῷ πληγάς.
\par }{\PP \VS{27}Ἄφνω δὲ πεσόντα πρὸς τὴν γῆν, καὶ πολλῷ σκότει περιχυθέντα, συναρπάσαντες, καὶ εἰς φορεῖον ἐνθέντες,
\VS{28}τὸν ἄρτι μετὰ πολλῆς παραδρομῆς καὶ πάσης δορυφορίας εἰς τὸ προειρημένον εἰσελθόντα γαζοφυλάκιον, ἔφερον ἀβοήθητον ἑαυτῷ καθεστῶτα, φανερῶς τὴν τοῦ Θεοῦ δυναστείαν ἐπεγνωκότες.
\VS{29}Καὶ ὁ μὲν διὰ τὴν θείαν ἐνέργειαν ἄφωνος καὶ πάσης ἐστερημένος ἐλπίδος καὶ σωτηρίας ἔῤῥιπτο.
\VS{30}Οἱ δὲ τὸν κύριον εὐλόγουν τὸν παραδοξάζοντα τὸν ἑαυτοῦ τόπον· καὶ τὸ μικρῷ πρότερον δέους καὶ ταραχῆς γέμον ἱερὸν, τοῦ παντοκράτορος ἐπιφανέντος Κυρίου, χαρᾶς καὶ εὐφροσύνης ἐπεπλήρωτο.
\par }{\PP \VS{31}Ταχὺ δέ τινες τῶν τοῦ Ἡλιοδώρου συνήθων ἠξίουν τὸν Ὀνίαν ἐπικαλέσασθαι τὸν ὕψιστον, καὶ τὸ ζῇν χαρίσασθι τῷ παντελῶς ἐν ἐσχάτῃ πνοῇ κειμένῳ.
\VS{32}Ὕποπτος δὲ γενόμενος ὁ ἀρχιερεῦς, μήποτε διάληψιν ὁ βασιλεὺς σχῇ, κακουργίαν τινὰ περὶ τὸν Ἡλιόδωρον ὑπὸ τῶν Ἰουδαίων συντετελέσθαι, προσήγαγε θυσίαν ὑπὲρ τῆς τοῦ ἀνδρὸς σωτηρίας.
\par }{\PP \VS{33}Ποιουμένου δὲ τοῦ ἀρχιερέως τὸν ἱλασμὸν, οἱ αὐτοὶ νεανίαι πάλιν ἐφάνησαν τῷ Ἡλιοδώρῳ ἐν ταῖς αὐταῖς ἐσθήσεσιν ἐστολισμένοι, καὶ στάντες εἶπον, πολλὰς τῷ Ὀνία τῷ ἀρχιερεῖ χάριτας ἔχε, διὰ γὰρ αὐτὸν σοι κεχάρισται τὸ ζῇν ὁ Κύριος.
\VS{34}Σὺ δὲ ὑπʼ αὐτοῦ μεμαστιγωμένος διάγγελε πᾶσι τὸ μεγαλεῖον τοῦ Θεοῦ κράτος· ταῦτα δὲ εἰπόντες ἀφανεῖς ἐγένοντο.
\par }{\PP \VS{35}Ὁ δὲ Ἡλιόδωρος θυσίαν ἀνενέγκας τῷ Κυρίῳ, καὶ εὐχὰς μεγίστας εὐξάμενος τῷ τὸ ζῇν περιποιήσαντι, καὶ τὸν Ὀνίαν ἀποδεξάμενος, ἀνεστρατοπέδευσε πρὸς τὸν βασιλέα.
\VS{36}Ἐξεμαρτύρει δὲ πᾶσιν ἅπερ ἦν ὑπʼ ὄψιν τεθεαμένος ἔργα τοῦ μεγίστου Θεοῦ.
\par }{\PP \VS{37}Τοῦ δὲ βασιλέως ἐπερωτήσαντος τὸν Ἡλιόδωρον, ποῖός τις εἴη ἐπιτήδειος ἔτι ἅπαξ διαπεμφθῆναι εἰς Ἱεροσόλυμα, ἔφησεν,
\VS{38}εἴ τινα ἔχεις πολέμιον ἢ πραγμάτων ἐπίβουλον, πέμψον αὐτὸν ἐκεῖ, καὶ μεμαστιγωμένον αὐτὸν προσδέξῃ, ἐάνπερ καὶ διασωθείη, διὰ τὸ περὶ τὸν τόπον ἀληθῶς εἶναί τινα Θεοῦ δύναμιν.
\VS{39}Αὐτὸς γὰρ ὁ τὴν κατοικίαν ἐπουράνιον ἔχων, ἐπόπτης ἐστὶ καὶ βοηθὸς ἐκείνου τοῦ τόπου, καὶ τοὺς παραγινομένους ἐπὶ κακώσει, τύπτων ἀπόλλυσι.
\par }{\PP \VS{40}Καὶ τὰ μὲν κατὰ Ἡλιόδωρον, καὶ τὴν τοῦ γαζοφυλακίου τήρησιν οὕτως ἐχώρησεν.

\par }\Chap{4}{\PP \VerseOne{1}Ὁ δὲ προειρημένος Σίμων ὁ τῶν χρημάτων καὶ τῆς πατρίδος ἐνδείκτης γεγονὼς, ἐκακολόγει τὸν Ὀνίαν, ὡς αὐτός τε εἴη τὸν Ἡλιόδωρον ἐπισεσεικὼς, καὶ τῶν κακῶν δημιουργὸς καθεστηκώς.
\VS{2}Καὶ τὸν εὐεργέτην τῆς πόλεως, καὶ τὸν κηδεμόνα τῶν ὁμοεθνῶν, καὶ ζηλωτὴν τῶν νόμων, ἐπίβουλον τῶν πραγμάτων ἐτόλμα λέγειν.
\par }{\PP \VS{3}Τῆς δὲ ἔχθρας ἐπὶ τοσοῦτον προβαινούσης, ὥστε καὶ διά τινος τῶν ὑπὸ τοῦ Σίμωνος δεδοκιμασμένων φόνους συντελεῖσθαι,
\VS{4}συνορῶν ὁ Ὀνίας τὸ χαλεπὸν τῆς φιλονεικίας, καὶ Ἀπολλώνιον μαίνεσθαι, ὡς τὸν κοίλης Συρίας καὶ Φοινίκης στρατηγὸν, συναύξοντα τὴν κακίαν τοῦ Σίμωνος,
\VS{5}ὡς τὸν βασιλέα διεκομίσθη, οὐ γινόμενος τῶν πολιτῶν κατήγορος, τὸ δὲ συμφέρον κοινῇ κατʼ ἰδίαν παντὶ τῷ πλήθει σκοπῶν.
\VS{6}Ἑώρα γὰρ ἄνευ βασιλικῆς προνοίας ἀδύνατον εἶναι τυχεῖν εἰρήνης ἔτι τὰ πράγματα, καὶ τὸν Σίμωνα παῦλαν οὐ ληψόμενον τῆς ἀνοίας.
\par }{\PP \VS{7}Μεταλλάξαντος δὲ τὸν βίον Σελεύκου, καὶ παραλαβόντος τὴν βασιλείαν Ἀντιόχου τοῦ προσαγορευθέντος Ἐπιφανοῦς, ὑπενόθευσεν Ἰάσων ὁ ἀδελφὸς Ὀνίου τὴν ἀρχιερωσύνην,
\VS{8}ἐπαγγειλάμενος τῷ βασιλεῖ διʼ ἐντεύξεως ἀργυοίου τάλαντα ἑξήκοντα πρὸς τοῖς τριακοσίοις, καὶ προσόδου τινὸς ἄλλης τάλαντα ὀγδοήκοντα,
\VS{9}πρὸς δὲ τούτοις ὑπισχνεῖτο καὶ ἕτερα διαγράψαι πεντήκοντα πρὸς τοῖς ἑκατόν, ἐὰν συγχωρηθῇ διὰ τῆς ἐξουσίας αὐτοῦ, γυμνάσιον καὶ ἐφηβίαν αὐτῷ συστήσασθαι, καὶ τοὺς ἐν Ἰεροσολύμοις Ἀντιοχεῖς ἀναγράψαι.
\VS{10}Ἐπινεύσαντος δὲ τοῦ βασιλέως, καὶ τῆς ἀρχῆς κρατήσας, εὐθέως ἐπὶ τὸν Ἑλληνικὸν χαρακτῆρα τοὺς ὁμοφύλους μετῆγε.
\par }{\PP \VS{11}Καὶ τὰ κείμενα τοῖς Ἰουδαίοις φιλάνθρωπα βασιλικὰ διὰ Ἰωάννου τοῦ πατρὸς Εὐπολέμου, τοῦ ποιησαμένου τὴν πρεσβείαν ὑπὲρ φιλίας καὶ συμμαχίας πρὸς τοὺς Ῥωμαίους, παρώσατο· καὶ τὰς μὲν νομίμους καταλύων πολιτείας, παρανόμους ἐθισμοὺς ἐκαίνιζεν.
\VS{12}Ἀσμένως γὰρ ὑπʼ αὐτὴν τὴν ἀκρόπολιν γυμνάσιον καθίδρυσε, καὶ τοὺς κρατίστους τῶν ἐφήβων ὑποτάσσων, ὑπὸ πέτασον ἦγεν.
\par }{\PP \VS{13}Ἦν δʼ οὕτως ἀκμή τις Ἑλληνισμοῦ, καὶ πρόσβασις ἀλλοφυλισμοῦ διὰ τὴν τοῦ ἀσεβοῦς καὶ οὐκ ἀρχιερέως Ἰάσωνος ὑπερβάλλουσαν ἀναγνείαν,
\VS{14}ὥστε μηκέτι περὶ τὰς τοῦ θυσιαστηρίου λειτουργίας προθύμους εἶναι τοὺς ἱερεῖς, ἀλλὰ τοῦ μὲν ναοῦ καταφρονοῦντες, καὶ τῶν θυσιῶν ἀμελοῦντες ἔσπευδον μετέχειν τῆς ἐν παλαίστρᾳ παρανόμου χορηγίας, μετὰ τὴν τοῦ δίσκου πρόκλησιν.
\VS{15}Καὶ τὰς μὲν πατρῴους τιμὰς ἐν οὐδενὶ τιθέμενοι, τὰς δὲ Ἑλληνικὰς δόξας καλλίστας ἡγούμενοι.
\par }{\PP \VS{16}Ὧν χάριν περιέσχεν αὐτοὺς χαλεπὴ περίστασις, καὶ ὧν ἐζήλουν τὰς ἀγωγὰς, καὶ καθάπαν ἤθελον ἐξομοιοῦσθαι, τούτους πολεμίους καὶ τιμωρητὰς ἔσχον.
\VS{17}Ἀσεβεῖν γὰρ εἰς τοὺς θείους νόμους οὐ ῥᾴδιον, ἀλλὰ ταῦτα ὁ ἀκόλουθος καιρὸς δηλώσει.
\par }{\PP \VS{18}Ἀγομένου δὲ πενταετηρικοῦ ἀγῶνος ἐν Τύρῳ, καὶ τοῦ βασιλέως παρόντος,
\VS{19}ἀπέστειλεν Ἰάσων ὁ μιαρὸς θεωροὺς ἀπὸ Ἱεροσολύμων Ἀντιοχεῖς ὄντας, παρακομίζοντας ἀργυρίου δραχμὰς τριακοσίας εἰς τὴν τοῦ Ἡρακλέους θυσίαν· ἃς καὶ ἠξίωσαν οἱ παρακομίσαντες μὴ χρῆσθαι πρὸς θυσίαν διὰ τὸ μὴ καθήκειν, εἰς ἑτέραν δὲ καταθέσθαι δαπάνην.
\VS{20}Ἔπεμψεν οὖν ταῦτα, διὰ μὲν τὸν ἀποστείλαντα εἰς τὴν τοῦ Ἡρακλέους θυσίαν, ἕνεκεν δὲ τῶν παρακομιζόντων, εἰς τὰς τῶν τριήρων κατασκευάς.
\par }{\PP \VS{21}Ἀποσταλέντος δὲ εἰς Αἴγυπτον Ἀπολλωνίου τοῦ Μενεσθέως διὰ τὰ πρωτοκλίσια Πτολεμαίου τοῦ Φιλομήτορος βασιλέως, μεταλαβὼν Ἀντίοχος ἀλλότριον αὐτὸν τῶυ αὐτῶν γεγονέναι πραγμάτων, τῆς κατʼ αὑτὸν ἀσφαλείας ἐφρόντιζεν· ὅθεν εἰς Ἰόππην παραγενόμενος, κατήντησεν εἰς Ἰεροσόλυμα.
\VS{22}Μεγαλοπρεπῶς δὲ ὑπὸ τοῦ Ἰάσωνος καὶ τῆς πόλεως παραδεχθεὶς, μετὰ δᾳδουχίας καὶ βοῶν εἰσπεπόρευται εἶθʼ οὕτως εἰς τὴν Φοινίκην κατεστρατοπέδευσε.
\par }{\PP \VS{23}Μετὰ δὲ τριετῆ χρόνον ἀπέστειλεν Ἰάσων Μενέλαον τὸν τοῦ προσημαινομένου Σίμωνος ἀδελφόν, παρακομίζοντα τὰ χρήματα τῷ βασιλεῖ, καὶ περὶ πραγμάτων ἀναγκαίων ὑπομνηματισμοὺς τελέσοντα.
\VS{24}Ὁ δὲ συσταθεὶς τῷ βασιλεῖ, καὶ δοξάσας αὐτὸν τῷ προσώπῳ τῆς ἐξουσίας, εἰς ἑαυτὸν κατήντησε τὴν ἀρχιερωσύνην, ὑπερβαλὼν τὸν Ἰάσωνα τάλαντα ἀργυρίου τριακόσια.
\VS{25}Λαβὼν δὲ τὰς βασιλικὰς ἐντολὰς παρεγένετο, τῆς μὲν ἀρχιερωσύνης οὐδὲν ἄξιον φέρων, θυμοὺς δὲ ὠμοῦ τυράννου, καὶ θηρὸς βαρβάρου ὀργὰς ἔχων.
\par }{\PP \VS{26}Καὶ ὁ μὲν Ἰάσων ὁ τὸν ἴδιον ἀδελφὸν ὑπονοθεύσας, ὑπονοθευθεὶς ὑφʼ ἑτέρου φυγὰς εἰς τὴν Ἀμμανῖτιν χώραν συνήλαστο.
\VS{27}Ὁ δὲ Μενέλαος τῆς μὲν ἀρχῆς ἐκράτει, τῶν δὲ ἐπηγγελμένων τῷ βασιλεῖ χρημάτων οὐδὲν εὐτάκτει,
\VS{28}ποιουμένου δὲ τὴν ἀπαίτησιν Σωστράτου τοῦ τῆς ἀκροπόλεως ἐπάρχου· πρὸς τοῦτον γὰρ ἦν ἡ τῶν φόρων πρᾶξις· διʼ ἣν αἰτίαν οἱ δύο ὑπὸ τοῦ βασιλέως προσεκλήθησαν.
\par }{\PP \VS{29}Καὶ ὁ μὲν Μενέλαος ἀπέλιπε τῆς ἀρχιερωσύνης διάδοχον Λυσίμαχον τὸν ἑαυτοῦ ἀδελφόν, Σώστρατος δὲ, Κράτητα τὸν ἐπὶ τῶν Κυπρίων.
\par }{\PP \VS{30}Τοιούτων δὲ συνεστηκότων, συνέβη Ταρσεῖς, καὶ Μαλλώτας στασιάζειν, διὰ τὸ Ἀντιοχίδι τῇ παλλακῇ τοῦ βασιλέως ἐν δωρεᾷ δεδόσθαι.
\VS{31}Θᾶττον οὖν ὁ βασιλεὺς ἧκε καταστεῖλαι τὰ πράγματα, καταλιπὼν τὸν διαδεχόμενον Ἀνδρόνικον, τῶν ἐν ἀξιώματι κειμένων.
\par }{\PP \VS{32}Νομίσας δὲ ὁ Μενέλαος εἰληφέναι καιρὸν εὐφυῆ, χρυσώματά τινα τῶν τοῦ ἱεροῦ νοσφισάμενος ἐχαρίσατο τῷ Ἀνδρονίκῳ, καὶ ἕτερα ἐτύγχανε πεπρακὼς εἴς τε Τύρον καὶ τὰς κύκλῳ πόλεις.
\VS{33}Ἃ καὶ σαφῶς ἐπεγνωκὼς ὁ Ὀνίας, παρήλεγχεν ἀποκεχωρηκὼς εἰς ἄσυλον τόπον, ἐπὶ Δάφνης τῆς πρὸς Ἀντιόχειαν κειμένης.
\par }{\PP \VS{34}Ὅθεν ὁ Μενέλαος λαβὼν ἰδίᾳ τὸν Ἀνδρόνικον, παρεκάλει χειρώσασθαι τὸν Ὀνίαν· ὁ δὲ παραγενόμενος ἐπὶ τὸν Ὀνίαν, καὶ πεισθεὶς ἐπὶ δόλῳ, καὶ δεξιὰς μεθʼ ὅρκων δοὺς, καίπερ ἐν ὑποψίᾳ κείμενος ἔπεισεν ἐκ τοῦ ἀσύλου προελθεῖν, ὃν καὶ παραχρῆμα παρέκλεισεν, οὐκ αἰδεσθεὶς τὸ δίκαιον.
\VS{35}Διʼ ἣν αἰτίαν οὐ μόνον Ἰουδαῖοι, πολλοὶ δὲ καὶ τῶν ἄλλων ἐθνῶν ἐδείναζον, καὶ ἐδυσφόρουν ἐπὶ τῷ τοῦ ἀνδρὸς ἀδίκῳ φόνῳ.
\par }{\PP \VS{36}Τοῦ δὲ βασιλέως ἐπανελθόντος ἀπὸ τῶν κατὰ Κιλικίαν τόπων, ἐνετύγχανον οἱ κατὰ πόλιν Ἰουδαῖοι συμισοπονηρούντων καὶ τῶν Ἑλλήνων, ὑπὲρ τοῦ παρὰ λόγον τὸν Ὀνίαν ἀπεκτάνθαι.
\VS{37}Ψυχικῶς οὖν ὁ Ἀντίοχος ἐπιλυπηθεὶς, καὶ τραπεὶς εἰς ἔλεον, καὶ δακρύσας διὰ τὴν τοῦ μετηλλαχότος σωφροσύνην, καὶ πολλὴν εὐταξίαν,
\VS{38}καὶ πυρωθεὶς τοῖς θυμοῖς, παραχρῆμα τὴν τοῦ Ἀνδρονίκου πορφύραν περιελόμενος, καὶ τοὺς χιτῶνας περιῤῥήξας, περιαγαγὼν καθʼ ὅλην τὴν πόλιν, ἐπʼ αὐτὸν τὸν τόπον οὗπερ εἰς τὸν Ὀνίαν ἠσέβησεν, ἐκεῖ τὸν μιαιφόνον ἀπεκόσμησε, τοῦ Κυρίου τὴν ἀξίαν αὐτῷ κόλασιν ἀποδόντος.
\par }{\PP \VS{39}Γενομένων δὲ πολλῶν ἱεροσυλημάτων κατὰ τὴν πόλιν ὑπὸ τοῦ Λυσιμάχου μετὰ τῆς Μενελάου γνώμης, καὶ διαδοθείσης ἔξω τῆς φήμης, ἐπισυνήχθη τὸ πλῆθος ἐπὶ τὸν Λυσίμαχον, χρυσωμάτων ἤδη πολλῶν διενηνεγμένων.
\VS{40}Ἐπεγειρομένων δὲ τῶν ὄχλων, καὶ ταῖς ὀργαῖς διεμπιπλαμένων, καθοπλίσας ὁ Λυσίμαχος πρὸς τρισχιλίους, κατήρξατο χειρῶν ἀδίκων, προηγησαμένου τινὸς Τυράννου προβεβηκότος τὴν ἡλικίαν, οὐδὲν δὲ ἧττον καὶ τὴν ἄνοιαν.
\par }{\PP \VS{41}Συνιδότες δὲ καὶ τὴν ἐπίθεσιν τοῦ Λυσιμάχου, συναρπάσαντες οἱ μὲν πέτρους, οἱ δὲ ξύλων πάχη, τινὲς δὲ ἐκ τῆς παρακειμένης σποδοῦ δρασσόμενοι, φύρδην ἐνετίνασσον εἰς τοὺς περὶ τὸν Λυσίμαχον.
\VS{42}Διʼ ἣν αἰτίαν πολλοὺς μὲν αὐτῶν τραυματίας ἐποίησαν, τινὰς δὲ καὶ κατέβαλον, πάντας δὲ εἰς φυγὴν συνήλασαν, αὐτὸν δὲ τὸν ἱερόσυλον παρὰ τὸ γαζοφυλάκιον ἐχειρώσαντο.
\par }{\PP \VS{43}Περὶ δὲ τούτων ἐνέστη κρίσις πρὸς τὸν Μενέλαον.
\VS{44}Καταντήσαντος δὲ τοῦ βασιλέως εἰς Τύρον, ἐπʼ αὐτοῦ τὴν δικαιολογίαν ἐποιήσαντο οἱ πεμφθέντες ἄνδρες τρεῖς ὑπὸ τῆς γερουσίας.
\VS{45}Ἤδη δὲ λελειμμένος ὁ Μενέλαος ἐπηγγείλατο χρήματα ἱκανὰ τῷ Πτολεμαίῳ τῷ Δορυμένους πρὸς τὸ πεῖσαι τὸν βασιλέα.
\par }{\PP \VS{46}Ὅθεν ἀπολαβὼν ὁ Πτολεμαῖος εἴς τι περίστυλον ὡς ἀναψύξοντα τὸν βασιλέα, μετέθηκε.
\VS{47}Καὶ τὸν μὲν τῆς ὅλης κακίας αἴτιον Μενέλαον ἀπέλυσε τῶν κατηγορημάτων, τοῖς δὲ ταλαιπώροις, οἵτινες εἰ καὶ ἐπὶ Σκυθῶν ἔλεγον, ἀπελύθησαν ἄν ἀκατάγνωστοι, τούτοις θάνατον ἐπέκρινε.
\par }{\PP \VS{48}Ταχέως οὖν τὴν ἄδικον ζημίαν ὑπέσχον οἱ ὑπὲρ πόλεως καὶ δήμων καὶ τῶν ἱερῶν σκευῶν προαγορεύσαντες.
\VS{49}Διʼ ἣν αἰτίαν καὶ Τύριοι μισοπονηρήσαντες τὰ πρὸς τὴν κηδείαν αὐτῶν μεγαλοπρεπῶς ἐχορήγησαν.
\VS{50}Ὁ δὲ Μενέλαος διὰ τὰς τῶν κρατούντων πλεονεξίας, ἔμενεν ἐπὶ τῆς ἀρχῆς, ἐπιφυόμενος τῇ κακίᾳ, μέγας τῶν πολιτῶν ἐπίβουλος καθεστώς.

\par }\Chap{5}{\PP \VerseOne{1}Περὶ δὲ τὸν καιρὸν τοῦτον τὴν δευτέραν ἔφοδον ὁ Ἀντίοχος εἰς Αἴγυπτον ἐστείλατο.
\VS{2}Συνέβη δὲ καθʼ ὅλην τὴν πόλιν σχεδὸν ἐφʼ ἡμέρας τεσσαράκοντα φαίνεσθαι διὰ τοῦ ἀέρος τρέχοντας ἱππεῖς διαχρύσους στολὰς ἔχοντας, καὶ λόγχας σπειρηδὸν ἐξωπλισμένους,
\VS{3}καὶ ἴλας ἵππων διατεταγμένας, καὶ προσβολὰς γινομένας, καὶ καταδρομὰς ἑκατέρων, καὶ ἀσπίδων κινήσεις, καὶ καμάκων πλήθη, καὶ μαχαιρῶν σπασμούς, καὶ βελῶν βολὰς, καὶ χρυσῶν κόσμων ἐκλάμψεις, καὶ παντοίους θωρακισμούς.
\VS{4}Διὸ πάντες ἠξίουν ἐπʼ ἀγαθῷ τὴν ἐπιφάνειαν γενέσθαι.
\par }{\PP \VS{5}Γενομένης δὲ λαλιᾶς ψευδοῦς, ὡς μετηλλαχότος τὸν βίον Ἀντιόχου, παραλαβὼν ὁ Ἰάσων οὐκ ἐλάττους τῶν χιλίων, αἰφνιδίως ἐπὶ τὴν πόλιν συνετελέσατο ἐπίθεσιν· τῶν δὲ ἐπὶ τῷ τείχει συνελασθέντων, καὶ τέλος ἤδη καταλαμβανομένης τῆς πόλεως, ὁ Μενέλαος εἰς τὴν ἀκρόπολιν ἐφυγάδευσεν.
\VS{6}Ὁ δὲ Ἰάσων ἐποιεῖτο σφαγὰς τῶν πολιτῶν τῶν ἰδίων ἀφειδῶς, οὐ συννοῶν τὴν εἰς τοὺς συγγενεῖς εὐημερίαν, δυσημερίαν εἶναι τὴν μεγίστην· δοκῶν δὲ πολεμίων καὶ οὐχ ὁμοεθνῶν τρόπαια καταβάλλεσθαι,
\VS{7}τῆς μὲν ἀρχῆς οὐκ ἐκράτησε, τὸ δὲ τέλος τῆς ἐπιβουλῆς αἰσχύνην λαβών, φυγὰς πάλιν εἰς τὴν Ἀμμανίτιν ἀπῆλθε.
\par }{\PP \VS{8}Πέρας οὖν κακῆς ἀναστροφῆς ἔτυχεν ἐγκλεισθεὶς πρὸς Ἀρέταν τὸν τῶν Ἀράβων τύραννον, πόλιν ἐκ πόλεως φεύγων, διωκόμενος ὑπὸ πάντων, καὶ στυγούμενος ὡς τῶν νόμων ἀποστάτης, καὶ βδελυσσόμενος ὡς πατρίδος καὶ πολιτῶν δήμιος, εἰς Αἴγυπτον συνεβράσθη.
\VS{9}Καὶ ὁ συχνοὺς τῆς πατρίδος ἀποξενώσας, ἐπὶ ξένης ἀπώλετο πρὸς Λακεδαιμονίους ἀναχθείς, ὡς διὰ τὴν συγγένειαν τευξόμενος σκέπης.
\VS{10}Καὶ ὁ πλῆθος ἀτάφων ἐκρίψας ἀπένθητος ἐγενήθη, καὶ κηδείας οὐδʼ ἡστινοσοῦν οὔτε πατρῴου τάφου μετέσχε.
\par }{\PP \VS{11}Προσπεσόντων δὲ τῷ βασιλεῖ περὶ τῶν γεγονότων, διέλαβεν ἀποστατεῖν τὴν Ἰουδαίαν· ὅθεν ἀναζεύξας ἐξ Αἰγύπτου τεθηριωμένος τῇ ψυχῇ, ἔλαβε τὴν μὲν πόλιν δορυάλωτον.
\VS{12}Καὶ ἐκέλευσε τοῖς στρατιώταις κόπτειν ἀφειδῶς τοὺς ἐμπίπτοντας, καὶ τοὺς εἰς τὰς οἰκίας ἀναβαίνοντας κατασφάζειν.
\VS{13}Ἐγίνοντο δὲ νέων καὶ πρεσβυτέρων ἀναιρέσιες, ἀνδρῶν τε καὶ γυναικῶν καὶ τέκνων ἀφανισμὸς, παρθένων τε καὶ νηπίων σφαγαί.
\VS{14}Ὀκτὼ δὲ μυριάδες ἐν ταῖς πάσαις ἡμέραις τρισὶ κατεφθάρησαν, τέσσαρες μὲν ἐν χειρῶν νομαῖς, οὐκ ἧττον δὲ τῶν ἐσφαγμένων ἐπράθησαν.
\par }{\PP \VS{15}Καὶ οὐκ ἀρκεσθεὶς δὲ τούτοις, κατετόλμησεν εἰς τὸ πάσης τῆς γῆς ἁγιώτατον ἱερὸν εἰσελθεῖν, ὁδηγὸν ἔχων τὸν Μενέλαον, τὸν καὶ τῶν νόμων καὶ τῆς πατρίδος προδότην γεγονότα.
\VS{16}Καὶ ταῖς μιαραῖς χερσὶ τὰ ἱερὰ σκεύη λαμβάνων, καὶ τὰ ὑπʼ ἄλλων βασιλέων ἀνατεθέντα πρὸς αὔξησιν καὶ δόξαν τοῦ τόπου καὶ τιμήν, ταῖς βεβήλοις χερσὶ συσσύρων ἐπεδίδου.
\par }{\PP \VS{17}Καὶ ἐμετεωρίζετο τὴν διάνοιαν ὁ Ἀντίοχος, οὐ συνορῶν ὅτι διὰ τὰς ἁμαρτίας τῶν τὴν πόλιν οἰκούντων ἀπώργισται βραχέως ὁ Δεσπότης, διὸ γέγονε περὶ τὸν τόπον παρόρασις.
\VS{18}Εἰ δὲ μὴ συνέβαινε προενέχεσθαι πολλοῖς ἁμαρτήμασι, καθάπερ ὁ Ἡλιόδωρος ὁ πεμφθεὶς ὑπὸ Σελεύκου τοῦ βασιλέως ἐπὶ τὴν ἐπίσκεψιν τοῦ γαζοφυλακίου, οὗτος προαχθεὶς παραχρῆμα μαστιγωθεὶς ἀνετράπη τοῦ θράσους.
\par }{\PP \VS{19}Ἀλλʼ οὐ διὰ τὸν τόπον τὸ ἔθνος, ἀλλὰ διὰ τὸ ἔθνος τὸν τόπον ὁ Κύριος ἐξελέξατο.
\VS{20}Διόπερ καὶ αὐτὸς ὁ τόπος συμμετασχὼν τῶν τοῦ ἔθνους δυσπετημάτων γενομένων, ὕστερον εὐεργετημάτων ὑπὸ τοῦ Κυρίου ἐκοινώνησε· καὶ ὁ καταλειφθεὶς ἐν τῇ τοῦ παντοκράτορος ὀργῇ, πάλιν ἐν τῇ τοῦ μεγάλου Δεσπότου καταλλαγῇ μετὰ πάσης δόξης ἐπανωρθώθη.
\par }{\PP \VS{21}Ὁ γοῦν Ἀντίοχος ὀκτακόσια πρὸς τοῖς χιλίοις ἀπενεγκάμενος ἐκ τοῦ ἱεροῦ τάλαντα, θᾶττον εἰς Ἀντιόχειαν ἐχωρίσθη, οἰόμενος ἀπὸ τῆς ὑπερηφανίας τὴν μὲν γῆν πλωτὴν, καὶ τὸ πέλαγος πορευτὸν θέσθαι διὰ τὸν μετεωρισμὸν τῆς καρδίας.
\par }{\PP \VS{22}Κατέλιπε δὲ καὶ ἐπιστάτας τοῦ κακοῦν τὸ γένος, ἐν μὲν Ἱεροσολύμοις Φίλιππον, τὸ μὲν γένος Φρύγα, τὸν δὲ τρόπον βαρβαρώτερον ἔχοντα τοῦ καταστήσαντος·
\VS{23}ἐν δὲ Γαριζὶν Ἀνδρόνικον, πρὸς δὲ τούτοις Μενέλαον, ὃς χείριστα τῶν ἄλλων ὑπερῄρετο τοῖς πολίταις, ἀπεχθῆ δὲ πρὸς τοὺς πολίτας Ἰουδαίους ἔχων διάθεσιν.
\par }{\PP \VS{24}Ἔπεμψε δὲ τὸν μυσάρχην Ἀπολλώνιον μετὰ στρατεύματος δισμυρίων πρὸς τοῖς δισχιλίοις, προστάξας τοὺς ἐν ἡλικίᾳ πάντας κατασφάξαι, τὰς δὲ γυναῖκας καὶ νεωτέρους πωλεῖν.
\VS{25}Οὗτος δὲ παραγενόμενος εἰς Ἱεροσόλυμα, καὶ τὸν εἰρηνικὸν ὑποκριθείς, ἐπέσχεν ἕως τῆς ἁγίας ἡμέρας τοῦ σαββάτου· καὶ λαβὼν ἀργοῦντας τοὺς Ἰουδαίους, τοῖς ὑφʼ ἑαυτὸν ἐξοπλησίαν παρήγγειλε.
\VS{26}Καὶ τοὺς ἐξελθόντας πάντας ἐπὶ τὴν θεωρίαν συνεξεκέντησε, καὶ εἰς τὴν πόλιν σὺν τοῖς ὅπλοις εἰσδραμὼν ἱκανὰ κατέστρωσε πλήθη.
\par }{\PP \VS{27}Ἰούδας δὲ ὁ Μακκαβαῖος δέκατός που γενηθεὶς, καὶ ἀναχωρήσας εἰς τὴν ἔρημον, θηρίων τρόπον ἐν τοῖς ὄρεσι διέζη σὺν τοῖς μετʼ αὐτοῦ, καὶ τὴν χορτώδη τροφὴν σιτούμενοι διατέλουν, πρὸς τὸ μὴ μετασχεῖν τοῦ μολυσμοῦ

\par }\Chap{6}{\PP \VerseOne{1}Μετʼ οὐ πολὺν δὲ χρόνον ἐξαπέστειλεν ὁ βασιλεὺς γέροντα Ἀθηναῖον, ἀναγκάζειν τοὺς Ἰουδαίους μεταβαίνειν ἐκ τῶν πατρῴων νόμων, καὶ τοῖς τοῦ Θεοῦ νόμοις μὴ πολιτεύεσθαι,
\VS{2}μολῦναι δὲ καὶ τὸν ἐν Ἱεροσολύμοις νεὼν, καὶ προσονομάσαι Διὸς Ὀλυμπίου, καὶ τὸν ἐν Γαριζὶν, καθὼς ἐτύγχανον οἱ τὸν τόπον οἰκοῦντες, Διὸς Ξενίου.
\par }{\PP \VS{3}Χαλεπὴ δὲ καὶ τοῖς ὄχλοις ἦν καὶ δυσχερὴς ἡ ἐπίστασις τῆς κακίας.
\VS{4}Τὸ μὲν γὰρ ἱερὸν ἀσωτίας καὶ κώμων ἐπεπλήρωτο ὑπὸ τῶν ἐθνῶν ῥαθυμούντων μεθʼ ἑταιρῶν, καὶ ἐν τοῖς ἱεροῖς περιβόλοις γυναιξὶ πλησιαζόντων, ἔτι δὲ τὰ μὴ καθήκοντα ἔνδον φερόντων.
\VS{5}Τὸ δὲ θυσιαστήριον τοῖς ἀποδιεσταλμένοις ἀπὸ τῶν νόμων ἀθεμίτοις ἐπεπλήρωτο.
\VS{6}Ἦν δʼ οὔτε σαββατίζειν, οὔτε πατρῴους ἑορτὰς διαφυλάττειν, οὔτε ἁπλῶς Ἰουδαῖον ὁμολογεῖν εἶναι.
\par }{\PP \VS{7}Ἤγοντο δὲ μετὰ πικρᾶς ἀνάγκης εἰς τὴν κατὰ μῆνα τοῦ βασιλέως γενέθλιον ἡμέραν ἐπὶ σπλαγχνισμόν· γενομένης δὲ Διονυσίων ἑορτῆς, ἠναγκάζοντο οἱ Ἰουδαῖοι κισσοὺς ἔχοντες πομπεύειν τῷ Διονύσῳ.
\par }{\PP \VS{8}Ψήφισμα δὲ ἐξέπεσεν εἰς τὰς ἀστυγείτονας πόλεις Ἑλληνίδας, Πτολεμαίων ὑποτιθεμένων τὴν αὐτὴν ἀγωγὴν κατὰ τῶν Ἰουδαίων, ἄγειν καὶ σπλαγχνίζειν·
\VS{9}τοὺς δὲ μὴ προαιρουμένους μεταβαίνειν ἐπὶ τὰ Ἑλληνικὰ, κατασφάζειν· παρῆν οὖν ὁρᾷν τὴν ἐνεστῶσαν ταλαιπωρίαν.
\par }{\PP \VS{10}Δύο γὰρ γυναῖκες ἀνηνέγθησαν περιτετμηκυῖαι τὰ τέκνα αὐτῶν· τούτων δὲ ἐκ τῶν μαστῶν κρεμάσαντες τὰ βρέφη, καὶ δημοσίᾳ περιαγαγόντες αὐτὰς τὴν πόλιν, κατὰ τοῦ τείχους ἐκρήμνισαν.
\VS{11}Ἕτεροι δὲ πλησίον συνδραμόντες εἰς τὰ σπήλαια, λεληθότως ἄγειν τὴν ἑβδομάδα, μηνυθέντες τῷ Φιλίππῳ συνεφλογίσθησαν, διὰ τὸ εὐλαβῶς ἔχειν βοηθῆσαι ἑαυτοῖς κατὰ τὴν δόξαν τῆς σεμνοτάτης ἡμέρας.
\par }{\PP \VS{12}Παρακαλῶ οὖν τοὺς ἐντυγχάνοντας τῇδε τῇ βίβλῳ, μὴ συστέλλεσθαι διὰ τὰς συμφορὰς, λογίζεσθαι δὲ τὰς τιμωρίας μὴ πρὸς ὄλεθρον, ἀλλὰ πρὸς παιδείαν τοῦ γένους ἡμῶν εὖναι.
\VS{13}Καὶ τὸ μὴ πολὺν χρόνον ἐᾶσθαι τοὺς δυσσεβοῦντας, ἀλλʼ εὐθέως περιπίπτειν ἐπιτιμίοις, μεγάλης εὐεργεσίας σημεῖόν ἐστιν.
\par }{\PP \VS{14}Οὐ γὰρ, καθάπερ καὶ ἐπὶ τῶν ἄλλων ἐθνῶν ἀναμένει μακροθυμῶν ὁ Δεσπότης, μέχρι τοῦ καταντήσαντας αὐτοὺς πρὸς ἐκπλήρωσιν ἁμαρτιῶν, κολάσαι, οὕτω καὶ ἐφʼ ἡμῶν ἔκρινεν
\VS{15}εἶναι, ἵνα μὴ πρὸς τέλος ἀφικομένων ἡμῶν τῶν ἁμαρτιῶν, ὕστερον ἡμᾶς ἐκδικᾷ.
\VS{16}Διόπερ οὐδέ ποτε μὲν τὸν ἔλεον αὐτοῦ ἀφʼ ἡμῶν ἀφίστησι· παιδεύων δὲ μετὰ συμφορᾶς, οὐκ ἐγκαταλείπει τὸν ἑαυτοῦ λαόν.
\VS{17}Πλὴν ἕως ὑπομνήσεως ταῦθʼ ἡμῖν εἰρήσθω· διʼ ὀλίγων δʼ ἐλευστέον ἐπὶ τὴν διήγησιν.
\par }{\PP \VS{18}Ἐλεάζαρός τις τῶν πρωτευόντων γραμματέων, ἀνὴρ ἤδη προβεβηκὼς τὴν ἡλικίαν, καὶ τὴν πρόσοψιν τοῦ προσώπου κάλλιστος τυγχάνων, ἀναχανὼν ἠναγκάζετο φαγεῖν ὕειον κρέας.
\VS{19}Ὁ δὲ τὸν μετʼ εὐκλείας θάνατον μᾶλλον ἢ τὸν μετὰ μύσους βίον ἀναδεξάμενος, αὐθαιρέτως ἐπὶ τὸ τύμπανον προσῆγε·
\VS{20}προπτύσας δέ, καθʼ ὃν ἔδει τρόπον προσέρχεσθαι τοὺς ὑπομένοντας ἀμύνεσθαι, ὧν οὐ θέμις γεύσασθαι διὰ τὴν πρὸς τὸ ζῇν φιλοστοργίαν.
\par }{\PP \VS{21}Οἱ δὲ πρὸς τῷ παρανόμῳ σπλαγχνισμῷ τεταγμένοι, διὰ τὴν ἐκ τῶν παλαιῶν χρόνων πρὸς τὸν ἄνδρα γνῶσιν, ἀπολαβόντες αὐτὸν κατιδίαν παρεκάλουν, ἐνέγκαντα κρέα οἷς καθῆκον αὐτῷ χρήσασθαι διʼ αὐτοῦ παρασκευασθέντα, ὑποκριθῆναι δὲ ὡς ἐσθίοντα τὰ ὑπὸ τοῦ βασιλέως προστεταγμένα τῶν ἀπὸ τῆς θυσίας κρεῶν,
\VS{22}ἵνα τοῦτο πράξας ἀπολυθῇ τοῦ θανάτου, καὶ διὰ τὴν ἀρχαίαν πρὸς αὐτοὺς φιλίαν τύχῃ φιλανθρωπίας.
\par }{\PP \VS{23}Ὁ δὲ λογισμὸν ἀστεῖον ἀναλαβὼν καὶ ἄξιον τῆς ἡλικίας, καὶ τῆς τοῦ γήρως ὑπεροχῆς, καὶ τῆς ἐπικτήτου καὶ ἐπιφανοῦς πολιᾶς, καὶ τῆς ἐκ παιδὸς καλλίστης ἀνατροφῆς, μᾶλλον δὲ τῆς ἁγίας καὶ θεοκτίστου νομοθεσίας, ἀκολούθως ἀπεφῄνατο, ταχέως λέγων προπέμπειν εἰς τὸν ᾅδην.
\par }{\PP \VS{24}Οὐ γὰρ τῆς ἡμετέρας ἡλικίας ἄξιόν ἐστιν ὑποκριθῆναι, ἵνα πολλοὶ τῶν νέων ὑπολαβόντες Ἐλεάζαρον τὸν ἐννενηκονταετῆ μεταβεβηκέναι εἰς ἀλλοφυλισμόν,
\VS{25}καὶ αὐτοὶ διὰ τὴν ἐμὴν ὑπόκρισιν, καὶ διὰ τὸ μικρὸν καὶ ἀκαριαῖον ζῇν πλανηθῶσι διʼ ἐμέ, καὶ μῦσος καὶ κηλίδα τοῦ γήρως κατακτήσομαι.
\VS{26}Εἰ γὰρ καὶ ἐπὶ τοῦ παρόντος ἐξελοῦμαι τὴν ἐξ ἀνθρώπων τιμωρίαν, ἀλλὰ τὰς τοῦ παντοκράτορος χεῖρας οὔτε ζῶν οὔτε ἀποθανὼν ἐκφεύξομαι.
\par }{\PP \VS{27}Διόπερ ἀνδρείως μὲν νῦν διαλλάξας τὸν βίον, τοῦ μὲν γήρως ἄξιος φανήσομαι,
\VS{28}τοῖς δὲ νέοις ὑπόδειγμα γενναῖον καταλελοιπὼς, εἰς τὸ προθύμως καὶ γενναίως ὑπὲρ τῶν σεμνῶν καὶ ἁγίων νόμων ἀπευθανατίζειν· τοσαῦτα δὲ εἰπὼν. ἐπὶ τὸ τύμπανον εὐθέως ἦλθε.
\VS{29}Τῶν δὲ ἀγόντων τὴν μικρῷ πρότερον εὐμένειαν πρὸς αὐτὸν εἰς δυσμένειαν μεταβαλόντων διὰ τὸ τοὺς προειρημένους λόγους, ὡς αὐτοὶ διελάμβανον, ἀπόνοιαν εἶναι·
\par }{\PP \VS{30}Μέλλων δὲ ταῖς πληγαῖς τελευτᾷν, ἀναστενάξας εἶπε, τῷ Κυρίῳ τῷ τὴν ἁγίαν γνῶσιν ἔχοντι φανερόν ἐστιν, ὅτι δυνάμενος ἀπολυθῆναι τοῦ θανάτου, σκληρὰς ὑποφέρω κατὰ τὸ σῶμα ἀλγηδόνας μαστιγούμενος, κατὰ ψυχὴν δὲ ἡδέως διὰ τὸν αὐτοῦ φόβον ταῦτα πάσχω.
\VS{31}Καὶ οὗτος οὖν τοῦτον τὸν τρόπον μετήλλαξεν, οὐ μόνον τοῖς νέοις, ἀλλὰ καὶ τοῖς πλείστοις τοῦ ἔθνους τὸν ἑαυτοῦ θάνατον ὑπόδειγμα γενναιότητος καὶ μνημόσυνον ἀρετῆς καταλιπών.

\par }\Chap{7}{\PP \VerseOne{1}Συνέβη δὲ καὶ ἑπτὰ ἀδελφοὺς μετὰ τῆς μητρὸν συλληφθέντας ἀναγκάζεσθαι ὑπὸ τοῦ βασιλέως ἀπὸ τῶν ἀθεμίτων ὑείων κρεῶν ἐφάπτεσθαι, μάστιξι καὶ νευραῖς αἰκιζομένους.
\par }{\PP \VS{2}Εἷς δὲ αὐτῶν γενόμενος προήγορος, οὕτως ἔφη, τί μέλλεις ἐρωτᾷν, καὶ μανθάνειν παρʼ ἡμῶν; ἕτοιμοι γὰρ ἀποθνήσκειν ἐσμὲν ἢ παραβαίνειν τοὺς πατρίους νόμους.
\par }{\PP \VS{3}Ἔκθυμος δὲ γενόμενος ὁ βασιλεὺς, προσέταξε τήγανα, καὶ λέβητας ἐκπυροῦν.
\VS{4}Τῶν δὲ ἐκπυρωθέντων, παραχρῆμα τὸν γενόμενον αὐτῶν προήγορον προσέταξε γλωσσοτομεῖν, καὶ περισκυθίσαντας ἀκρωτηριάζειν, τῶν λοιπῶν ἀδελφῶν, καὶ τῆς μετρὸς, συνορώντων.
\par }{\PP \VS{5}Ἄχρηστον δὲ αὐτὸν τοῖς ὅλοις γενόμενον, ἐκέλευσε τῇ πυρᾷ προσάγειν ἔμπνουν, καὶ τηγανίζειν· τῆς δὲ ἀτμίδος ἐφʼ ἱκανὸν διαδιδούσης τοῦ τηγάνου, ἀλλήλους παρεκάλουν σὺν τῇ μητρὶ γενναίως τελευτᾷν, λέγοντες οὕτως.
\VS{6}Ὁ Κύριος ὁ Θεὸς ἐφορᾷ, καὶ ταῖς ἀληθείαις ἐφʼ ἡμῖν παρακαλεῖται, καθάπερ διὰ τῆς κατὰ πρόσωπον ἀντιμαρτυρούσης ᾠδῆς διεσάφησε Μωυσῆς, λέγων, καὶ ἐπὶ τοῖς δούλοις αὐτοῦ παρακληθήσεται.
\par }{\PP \VS{7}Μεταλλάξαντος δὲ τοῦ πρώτου τὸν τρόπον τοῦτον, τὸν δεύτερον ἦγον ἐπὶ τὸν ἐμπαιγμόν· καὶ τὸ τῆς κεφαλῆς δέρμα σὺν ταῖς θριξὶ περισύραντες, ἐπηρώτων, εἰ φάγεσαι πρὸ τοῦ τιμωρηθῆναι τὸ σῶμα κατὰ μέλος;
\par }{\PP \VS{8}Ὁ δὲ ἀποκριθεὶς τῇ πατρίῳ φωνῇ εἶπεν, οὐχί· διόπερ καὶ
\VS{9}οὗτος τὴν ἑξῆς ἔλαβε βάσανον, ὡς ὁ πρῶτος. Ἐν ἐσχάτῇ δὲ πνοῇ γενόμενος, εἶπε, σὺ μὲν ἀλάστωρ ἐκ τοῦ παρόντος ἡμᾶς ζῇν ἀπολύεις, ὁ δὲ τοῦ κόσμου βασιλεὺς ἀποθανόντας ἡμᾶς ὑπὲρ τῶν αὐτοῦ νόμων εἰς αἰώνιον ἀναβίωσιν ζωῆς ἡμᾶς ἀναστήσει.
\par }{\PP \VS{10}Μετὰ δὲ τοῦτον ὁ τρίτος ἐνεπαίζετο, καὶ τὴν γλῶσσαν αἰτηθεὶς ταχέως προέβαλε, καὶ τὰς χεῖρας εὐθαρσῶς προέτεινε,
\VS{11}καὶ γενναίως εἶπεν, ἐξ οὐρανοῦ ταῦτα κέκτημαι, καὶ διὰ τοὺς αὐτοῦ νόμους ὑπερορῶ ταῦτα, καὶ παρʼ αὐτοῦ ταῦτα πάλιν ἐλπίζω κομίσασθαι.
\VS{12}Ὥστε αὐτὸν τὸν βασιλέα καὶ τοὺς σὺν αὐτῷ ἐκπλήσσεσθαι τὴν τοῦ νεανίσκου ψυχὴν, ὡς ἐν οὐδενὶ τὰς ἀλγηδόνας ἐτίθετο.
\par }{\PP \VS{13}Καὶ τούτου δὲ μεταλλάξαντος, τὸν τέταρτον ὡσαύτως ἐβασάνιζον αἰκιζόμενοι.
\VS{14}Καὶ γεννόμενος πρὸς τὸ τελευτᾷν, οὕτως ἔφη, αἱρετὸν μεταλλάσσοντας ὑπʼ ἀνθρώπων τὰς ὑπὸ τοῦ Θεοῦ προσδοκᾷν ἐλπίδας, πάλιν ἀναστήσεσθαι ὑπʼ αὐτοῦ· σοὶ μὲν γὰρ ἀνάστασις εἰς ζωὴν οὐκ ἔσται.
\par }{\PP \VS{15}Ἐχομένως δὲ τὸν πέμπτον προσάγοντες ᾐκίζοντο.
\VS{16}Ὁ δὲ πρὸς αὐτὸν ἰδὼν, εἶπεν, ἐξουσίαν ἐν ἀνθρώποις ἔχων φθαρτὸς ὤν, ὅ θελεις ποιεῖς· μὴ δόκει δὲ τὸ γένος ἡμῶν ὑπὸ τοῦ Θεοῦ καταλελεῖφθαι.
\VS{17}Σὺ δὲ καρτέρει, καὶ θεώρει τὸ μεγαλεῖον αὐτοῦ κράτος, ὡς σὲ καὶ τὸ σπέρμα σου βασανίσει.
\par }{\PP \VS{18}Μετὰ δὲ τοῦτον ἦγον τὸν ἕκτον, καὶ μέλλων ἀποθνήσκειν, ἔφη, μὴ πλανῶ μάτην, ἡμεῖς γὰρ διʼ ἑαυτοὺς ταῦτα πάσχομεν ἁμαρτάνοντες εἰς τὸν ἑαυτῶν Θεὸν, διὸ ἄξια θαυμασμοῦ γέγονε.
\VS{19}Σὺ δὲ μὴ νομίσῃς ἀθῶος ἔσεσθαι, θεομαχεῖν ἐπιχειρήσας.
\par }{\PP \VS{20}Ὑπεραγόντως δὲ ἡ μήτηρ θαυμαστὴ καὶ μνήμης ἀγαθῆς ἀξία, ἥτις ἀπολλυμένους υἱοὺς ἑπτὰ συνορῶσα μιᾶς ὑπὸ καιρὸν ἡμέρας, εὐψύχως ἔφερε διὰ τὰς ἐπὶ Κύριον ἐλπίδας.
\VS{21}Ἕκαστον δὲ αὐτῶν παρεκάλει τῇ πατρίῳ φωνῇ, γενναίῳ πεπληρωμένη φρονήματι, καὶ τὸν θῆλυν λογισμὸν ἄρσενι θυμῷ διεγείρασα, λέγουσα πρὸς αὐτοὺς,
\VS{22}οὐδʼ οἶδʼ ὅπως εἰς τὴν ἐμὴν ἐφάνητε κοιλίαν, οὐδὲ ἐγὼ τὸ πνεῦμα καὶ τὴν ζωὴν ὑμῖν ἐχαρισάμην, καὶ τὴν ἑκάστου στοιχείωσιν οὐκ ἐγὼ διερύθμισα.
\VS{23}Τοιγαροῦν ὁ τοῦ κόσμου κτίστης ὁ πλάσας ἀνθρώπου γένεσιν, καὶ πάντων ἐξευρὼν γένεσιν, καὶ τὸ πνεῦμα καὶ τὴν ζωὴν ὑμῖν πάλιν ἀποδώσει μετʼ ἐλέους, ὡς νῦν ὑπερορᾶτε ἑαυτοὺς διὰ τοὺς αὐτοῦνόμους.
\par }{\PP \VS{24}Ὁ δὲ Ἀντίοχος οἰόμενος καταφρονεῖσθαι, καὶ τὴν ὀνειδίζουσαν ὑφορώμενος φωνὴν, ἔτι τοῦ νεωτέρου περιόντος, οὐ μόνον διὰ λόγων ἐποιεῖτο τὴν παράκλησιν, ἀλλὰ καὶ διʼ ὅρκων ἐπίστου, ἅμα πλουτιεῖν καὶ μακαριστὸν ποιήσειν μεταθέμενον ἀπὸ τῶν πατρίων νόμων, καὶ φίλον ἕξειν, καὶ χρείας ἐμπιστεύσειν.
\par }{\PP \VS{25}Τοῦ δὲ νεανίου μηδαμῶς προσέχοντος, προσκαλεσάμενος ὁ βασιλεὺς τὴν μητέρα, παρήνει τοῦ μειρακίου γενέσθαι σύμβουλον ἐπὶ σωτηρία.
\VS{26}Πολλὰ δὲ αὐτοῦ παραινέσαντος, ἐπεδέξατο πείσειν τὸν υἱόν.
\par }{\PP \VS{27}Προσκύψασα δὲ αὐτῷ, χλευάσασα τὸν ὠμὸν τύραννον, οὕτως ἔφησε τῇ πατρῴᾳ φωνῇ, υἱὲ, ἐλέησόν με τὴν ἐν γαστρὶ περιενέγκασάν σε μῆνας ἐννέα, καὶ θηλάσασάν σε ἔτη τρία, καὶ ἐκθρέψασάν σε καὶ ἀγαγοῦσαν εἰς τὴν ἡλικίαν ταύτην, καὶ τροφοφορήσασαν.
\VS{28}Ἀξιῶ σε, τέκνον, ἀναβλέψαντα εἰς τὸν οὐρανὸν καὶ τὴν γῆν, καὶ τὰ ἐν αὐτοῖς πάντα ἰδόντα, γνῶναι ὅτι ἐξ οὐκ ὄντων ἐποίησεν αὐτὰ ὁ Θεὸς, καὶ τὸ τῶν ἀνθρώπων γένος οὕτως γεγένηται,
\VS{29}μὴ φοβηθῇς τὸν δήμιον τοῦτον, ἀλλὰ τῶν ἀδελφῶν ἄξιος γενόμενος, ἐπίδεξαι τὸν θάνατον, ἵνα ἐν τῷ ἐλέει σὺν τοῖς ἀδελφοῖς σου κομίσωμαί σε.
\par }{\PP \VS{30}Ἔτι δὲ ταύτης καταλεγούσης ὁ νεανίας εἶπε, τίνα μένετε; οὐχ ὑπακούω τοῦ προστάγματος τοῦ βασιλέως· τοῦ δὲ προστάγματος ἀκούω τοῦ νόμου τοῦ δοθέντος τοῖς πατράσιν ἡμῶν διὰ Μωυσέως.
\VS{31}Σὺ δὲ πάσης κακίας εὑρετὴς γενόμενος εἰς τοὺς Ἑβραίους, οὐ μὴ διαφύγῃς τὰς χεῖρας τοῦ Θεοῦ.
\par }{\PP \VS{32}Ἡμεῖς γὰρ διὰ τὰς ἑαυτῶν ἁμαρτίας πάσχομεν.
\VS{33}Εἰ δὲ χάριν ἐπιπλήξεως καὶ παιδείας ὁ ζῶν Κύριος ἡμῶν βραχέως ἐπώργισται, καὶ πάλιν καταλλαγήσεται τοῖς ἑαυτοῦ δούλοις.
\VS{34}Σὺ δὲ ὦ ἀνόσιε, καὶ πάντων ἀνθρώπων μιαρώτατε, μὴ μάτην μετεωρίζου φρυαττόμενος ἀδήλοις ἐλπίσιν, ἐπὶ τοὺς δούλους αὐτοῦ ἐπαιρόμενος χεῖρα.
\VS{35}Οὔπω γὰρ τὴν τοῦ παντοκράτορος ἐπόπτου Θεοῦ κρίσιν ἐκπέφευγας.
\par }{\PP \VS{36}Οἱ μὲν γὰρ νῦν ἡμέτεροι ἀδελφοὶ βραχὺν ὑπενέγκαντες πόνον, ἀεννάου ζωῆς ὑπὸ διαθήκην Θεοῦ πεπτώκασι· σὺ δὲ τῇ τοῦ Θεοῦ κρίσει δίκαια τὰ πρόστιμα τῆς ὑπετηφανίας ἀποίσῃ.
\VS{37}Ἐγὼ δὲ καθάπερ οἱ ἀδελφοί μου, καὶ σῶμα καὶ ψυχὴν προδίδωμι περὶ τῶν πατρίων νόμων, ἐπικαλούμενος τὸν Θεὸν ἵλεων ταχὺ τῷ ἔθνει γενέσθαι, καὶ σὲ μετὰ ἐτασμῶν καὶ μαστίγων ἐξομολογήσασθαι, διότι μόνος αὐτὸς Θεός ἐστιν,
\VS{38}ἐν ἐμοὶ δὲ καὶ τοῖς ἀδελφοῖς μου στῆναι τὴν τοῦ παντοκράτορος ὀργὴν τὴν ἐπὶ τὸ σύμπαν ἡμῶν γένος δικαίως ἐπηγμένην.
\par }{\PP \VS{39}Ἔκθυμος δὲ γενόμενος ὁ βασιλεὺς, τούτῳ παρὰ τοὺς ἄλλους χειρίστως ἀπήντησε, πικρῶς φέρων ἐπὶ τῷ μυκτηρισμῷ.
\VS{40}Καὶ οὗτος οὖν καθαρὸς τὸν βίον μετήλλαξε, παντελῶς ἐπὶ τῷ Κυρίῳ πεποιθώς.
\par }{\PP \VS{41}Ἐσχάτη δὲ τῶν υἱῶν ἡ μήτηρ ἐτελεύτησε.
\par }{\PP \VS{42}Τὰ μὲν οὖν περὶ σπλαγχνισμοὺς, καὶ τὰς ὑπερβαλλούσας αἰκίας ἐπὶ τοσοῦτον δεδηλώσθω.

\par }\Chap{8}{\PP \VerseOne{1}Ἰούδας δὲ ὁ Μακκαβαῖος καὶ οἱ σὺν αὐτῷ, παρεισπορευόμενοι λεληθότως εἰς τὰς κώμας, προσεκαλοῦντο τοὺς συγγενεῖς, καὶ τοὺς μεμενηκότας ἐν τῷ Ἰουδαϊσμῷ προσλαβόμενοι, συνήγαγον εἰς ἑξακισχιλίους.
\par }{\PP \VS{2}Καὶ ἐπεκαλοῦντο τὸν Κύριον ἐπιδεῖν ἐπὶ τὸν ὑπὸ πάντων καταπατούμενον λαόν, οἰκτεῖραι δὲ καὶ τὸν ναὸν, τὸν ὑπὸ τῶν ἀσεβῶν ἀνθρώπων βεβηλωθέντα,
\VS{3}ἐλεῆσαι δὲ καὶ τὴν καταφθειρομένην πόλιν καὶ μέλλουσαν ἰσόπεδον γίνεσθαι, καὶ τῶν καταβοώντων πρὸς αὐτὸν αἱμάτων εἰσακοῦσαι,
\VS{4}μνησθῆναι δὲ καὶ τῆς τῶν ἀναμαρτήτων νηπίων παρανόμου ἀπωλείας, καὶ περὶ τῶν γενομένων εἰς τὸ ὄνομα αὐτοῦ βλασφημιῶν, καὶ μισοπονηρῆσαι.
\par }{\PP \VS{5}Γενόμενος δὲ ἐν συστήματι ὁ Μακκαβαῖος, ἀνυπόστατος ἤδη τοῖς ἔθνεσιν ἐγίνετο, τῆς ὀργῆς τοῦ Κυρίου εἰς ἔλεον τραπείσης.
\VS{6}Πόλεις δὲ καὶ κώμας ἀπροσδοκήτως ἐρχόμενος ἐνεπίμπρα, καὶ τοὺς ἐπικαίρους τόπους ἀπολαμβάνων, οὐκ ὀλίγους τῶν πολεμίων ἐνίκα τροπούμενος.
\VS{7}Μάλιστα τὰς νύκτας πρὸς τὰς τοιαύτας ἐπιβουλὰς συνεργοὺς ἐλάμβανε· καὶ λαλιά τις τῆς εὐανδρίας αὐτοῦ διεχεῖτο πανταχῆ.
\par }{\PP \VS{8}Συνορῶν δὲ ὁ Φίλιππος κατὰ μικρὸν εἰς προκοπὴν ἐρχόμενον τὸν ἄνδρα πυκνότερον δὲ ἐν ταῖς εὐημερίαις προβαίνοντα, πρὸς Πτολεμαῖον τὸν κοίλης Συρίας καὶ Φοινίκης στρατηγὸν ἔγραψεν ἐπιβοηθεῖν τοῖς τοῦ βασιλέως πράγμασιν.
\par }{\PP \VS{9}Ὁ δὲ ταχέως προχειρισάμενος, Νικάνορα τὸν τοῦ Πατρόκλου, τῶν πρώτων φίλως, ἀπέστειλεν, ὑποτάξας παμφύλων ἔθνη οὐκ ἐλάττους τῶν δισμυρίων, τὸ σύμπαν τῶν Ἰουδαίως ἐξᾶραι γένος· συνέστησε δὲ αὐτῷ καὶ Γοργίαν ἄνδρα στρατηγὸν, καὶ ἐν πολεμικαῖς χρείαις πείραν ἔχοντα.
\par }{\PP \VS{10}Διεστήσατο δὲ ὁ Νικάνωρ τὸν φόρον τῷ βασιλεῖ τοῖς Ῥωμαίοις ὄντα ταλάντων δισχιλίων ἐκ τῆς τῶν Ἰουδαίων αἰχμαλωσίας ἐκπληρώσειν.
\VS{11}Εὐθέως δὲ εἰς τὰς παραθαλασσίους πόλεις ἀπέστειλε προσκαλούμενος ἐπʼ ἀγορασμὸν Ἰουδαϊκῶν σωμάτων, ὑπισχνούμενον ἐννενήκοντα σώματα ταλάντου παραχωρήσειν· οὐ προσδεχόμενος τὴν παρὰ τοῦ παντοκράτορος μέλλουσαν παρακολουθήσειν ἐπʼ αὐτῷ δίκην.
\par }{\PP \VS{12}Τῷ δὲ ʼΙούδᾳ προσέπεσε περὶ τῆς τοῦ Νικάνορος ἐφόδου· καὶ μεταδόντος αὐτοῦ σὺν αὐτῷ τὴν παρουσίαν τοῦ στρατοπέδου,
\VS{13}οἱ δει λανδροῦντες καὶ ἀπιστοῦντες τὴν τοῦ Θεοῦ δίκην, διεδίδρασκον, καὶ ἐξετόπιζον ἑαυτούς.
\par }{\PP \VS{14}Οἱ δὲ τὰ περιλελειμμένα πάντα ἐπώλουν, ὁμοῦ δὲ τὸν Κύριον ἠξίουν ῥύσασθαι τοὺς ὑπὸ τοῦ δυσσεβοῦς Νικάνορος πρὶν συντυχεῖν πεπραμένους.
\VS{15}Καὶ εἰ μὴ διʼ αὐτούς, ἀλλὰ διὰ τὰς πρὸς τοὺς πατέρας αὐτῶν διαθήκας, καὶ ἕνεκεν τῆς ἐπʼ αὐτοὺς ἐπικλήσεως τοῦ σεμνοῦ καὶ μεγαλοπρεποῦς ὀνόματος αὐτοῦ.
\par }{\PP \VS{16}Συναγαγὼν δὲ ὁ Μακκαβαῖος τοὺς περὶ αὐτὸν ὄντας τὸν ἀριθμὸν ἑξακισχιλίους, παρεκάλει μὴ καταπλαγῆναι τοὺς πολεμίους, μηδὲ εὐλαβεῖσθαι τὴν τῶν ἀδίκως παραγινομένων ἐπʼ αὐτοὺς ἐθνῶν πολυπληθίαν, ἀγωνίσασθαι δὲ γενναίως,
\VS{17}πρὸ ὀφθαλμῶν λαβόντας τὴν ἀνόμως εἰς τὸν ἅγιον τόπον συντετελεσμένην ὑπʼ αὐτῶν ὕβριν, καὶ τὸν τῆς ἐμπεπαιγμένης πόλεως αἰκισμόν, ἔτι δὲ τὴν τῆς προγονικῆς πολιτείας κατάλυσιν.
\VS{18}Οἱ μὲν γὰρ ὅπλοις πεποίθασιν ἅμα καὶ τόλμαις, ἔφησεν, ἡμεῖς δὲ ἐπὶ τῷ παντοκράτορι Θεῷ δυναμένῳ καὶ τοὺς ἐρχομένους ἐφʼ ἡμᾶς, καὶ τὸν ὅλον κόσμον ἐν ἑνὶ νεύματι καταβαλεῖν, πεποίθαμεν.
\par }{\PP \VS{19}Προσαναλεξάμενος δὲ αὐτοῖς καὶ τὰς ἐπὶ τῶν προγόνων γενομένας ἀντιλήψεις, καὶ τὴν ἐπὶ Σενναχηρεὶμ τῶν ἑκατὸν ὀγδοήκοντα πέντε χιλιάδων ὡς ἀπώλοντο.
\VS{20}Καὶ τὴν ἐν τῇ Βαβυλωνίᾳ τὴν πρὸς αὐτοὺς Γαλάτας παράταξιν γενομένην, ὡς οἱ πάντες ἐπὶ τὴν χρείαν ἦλθον ὀκτακισχιλιοι σὺν Μακεδόσι τετρακισχιλίοις, τῶν Μακεδόνων ἀπορουμένων, οἱ ὀκτακισχίλιοι τὰς δώδεκα μυρίαδας ἀπώλεσαν διὰ τὴν γενομένην αὐτοῖς ἀπʼ οὐρανοῦ βοήθειαν, καὶ ὠφέλειαν πολλὴν ἔλαβον.
\par }{\PP \VS{21}Ἐφʼ οἷς εὐθαρσεῖς αὐτοὺς παραστήσας, καὶ ἑτοίμους ὑπὲρ τῶν νόμων καὶ τῆς πατρίδος ἀποθνήσκειν, τετραμερές τι τὸ στράτευμα ἐποίησε·
\VS{22}τάξας καὶ τοὺς ἀδελφοὺς αὐτοῦ προηγουμένους ἑκατέρας τάξεως, Σίμωνα καὶ Ἰώσηφον καὶ Ἰωνάθαν, ὑποτάξας ἑκάστῳ χιλίους πρὸς τοῖς πεντακοσίοις,
\VS{23}ἔτι δὲ καὶ Ἐλεάζαρον, παραγνοὺς τὴν ἱερὰν βίβλον, καὶ δοὺς σύνθημα Θεοῦ βοηθείας, τῆς πρώτης σπείρας αὐτὸς προηγούμενος, συνέβαλε τῷ Νικάνορι.
\par }{\PP \VS{24}Γενομένου δὲ αὐτοῖς τοῦ παντοκράτορος συμμάχου, κατέσφαξαν τῶν πολεμίων ὑπὲρ τοὺς ἐννακισχιλίους, τραυματίας δὲ καὶ τοῖς μέλεσιν ἀναπήρους τὸ πλεῖστον μέρος τῆς τοῦ Νικάνορος στρατιᾶς ἐποίησαν, πάντας δὲ φυγεῖν ἠνάγκασαν.
\VS{25}Τὰ δὲ χρήματα τῶν παραγεγονότων ἐπὶ τὸν ἀγορασμὸν αὐτῶν ἔλαβον· συνδιώξαντες δὲ αὐτοὺς ἐφʼ ἱκανὸν, ἀνέλυσαν ὑπὸ τῆς ὥρας συγκλειόμενοι.
\VS{26}Ἦν γὰρ ἡ πρὸ τοῦ σαββάτου, διʼ ἥν αἰτίαν οὐκ ἐμακροθύμησαν κατατρέχοντες αὐτούς.
\par }{\PP \VS{27}Ὁπλολογήσαντες δὲ αὐτοὺς, καὶ τὰ σκῦλα ἐκδύσαντες τῶν πολεμίων, περὶ τὸ σάββατον ἐγίνοντο, περισσῶς εὐλογοῦντες, καὶ ἐξομολογούμενοι τῷ Κυρίῳ τῷ διασώσαντι αὐτοὺς εἰς τὴν ἡμέραν ταύτην, ἀρχὴν ἐλέους τάξαντος αὐτοῖς.
\par }{\PP \VS{28}Μετὰ δὲ τὸ σάββατον τοῖς ᾐκισμένοις, καὶ ταῖς χήραις, καὶ ὀρφανοῖς, μερίσαντες ἀπὸ τῶν σκύλων, τὰ λοιπὰ αὐτοὶ καὶ τὰ παιδία ἐμερίσαντο.
\VS{29}Ταῦτα δὲ διαπραξάμενοι, καὶ κοινὴν ἱκετείαν ποιησάμενοι, τὸν ἐλεήμονα Κύριον ἠξίουν εἰς τέλος, καταλλαγῆναι τοῖς αὐτοῦ δούλοις.
\par }{\PP \VS{30}Καὶ τῶν περὶ Τιμόθεον καὶ Βακχίδην συνεριζοντων, ὑπὲρ τοὺς δισμυρίους αὐτῶν ἀνεῖλον, καὶ ὀχυρωμάτων ὑψηλῶν εὖ μάλα ἐγκρατεῖς ἐγένοντο· καὶ λάφυρα πλεῖστα ἐμερίσαντο, ἰσομοίρους ἑαυτοὺς καὶ τοῖς ᾐκισμένοις, καὶ ὀρφανοῖς, καὶ χήραις, ἔτι δὲ καὶ πρεσβυτέροις ποιήσαντες.
\VS{31}Ὁπλολογήσαντες δὲ αὐτοὺς, ἐπιμελῶς πάντα συνέθηκαν εἰς τοὺς ἐπικαίρους τόπους, τὰ δὲ λοιπὰ τῶν σκύλων ἤνεγκαν εἰς Ἱεροσόλυμα.
\par }{\PP \VS{32}Τὸν δὲ φυλάρχην τῶν περὶ Τιμόθεον ἀνεῖλον, ἀνοσιώτατον ἄνδρα καὶ πολλὰ τοὺς Ἰουδαίους ἐπιλελυπηκότα.
\VS{33}Ἐπινίκια δὲ ἄγοντες ἐν τῇ πατρίδι, τοὺς ἐμπρήσαντας τοὺς ἱεροὺς πυλῶνας, Καλλισθένην, καί τινας ἄλλους ὑφῆψαν εἰς ἓν οἰκίδιον πεφευγότας, οἵ τινες ἄξιον τῆς δυσσεβείας ἐκομίσαντομισθόν.
\par }{\PP \VS{34}Ὁ δὲ τρισαλιτήριος Νικάνωρ, ὁ τοὺς χιλίους ἐμπόρους ἐπὶ τὴν πράσιν τῶν Ἰουδαίων ἀγαγὼν,
\VS{35}ταπεινωθεὶς ὑπὸ τῶν κατʼ αὑτὸν νομιζομένων ἐλαχίστων εἶναι, τῇ τοῦ Κυρίου βοηθείᾳ, τὴν δοξικὴν ἀποθέμενος ἐσθῆτα, διὰ τῆς μεσογείου, δραπέτου τρόπον ἔρημον ἑαυτὸν ποιήσας, ἧκεν εἰς Ἀντιόχειαν, ὑπεράγαν δυσημερήσας ἐπὶ τῇ τοῦ στρατοῦ διαφθορᾷ.
\VS{36}Καὶ ὁ τοῖς Ῥωμαίοις ἀναδεξάμενος φόρον ἀπὸ τῆς τῶν ἐν Ἱεροσολύμοις αἰχμαλωσίας κατορθώσασθαι, κατήγγελλεν ὑπέρμαχον ἔχειν τὸν Θεὸν τοὺς Ἰουδαίους, καὶ διὰ τὸν τρόπον τοῦτον ἀτρώτους εἶναι τοὺς Ἰουδαίους, διὰ τὸ ἀκολουθεῖν τοῖς ὑπʼ αὐτοῦ προτεταγμένοις νόμοις.

\par }\Chap{9}{\PP \VerseOne{1}Περὶ δὲ τὸν καιρὸν ἐκεῖνον ἐτύγχανεν Ἀντίοχος ἀναλελυκὼς ἀκόσμως ἐκ τῶν κατὰ τὴν Περσίδα τόπων.
\VS{2}Εἰσεληλύθει γὰρ εἰς τὴν λεγομένην Περσέπολιν, καὶ ἐπεχείρησεν ἱεροσυλεῖν, καὶ τὴν πόλιν συνέχειν· διὸ δὴ τῶν πληθῶν ὁρμησάντων, ἐπὶ τὴν τῶν ὅπλων βοήθειαν ἐτράπησαν· καὶ συνέβη τροπωθέντα τὸν Ἀντίοχον ὑπὸ τῶν ἐγχωρίων, ἀσχήμονα τὴν ἀναζυγὴν ποιήσασθαι.
\par }{\PP \VS{3}Ὄντι δὲ αὐτῷ κατʼ Ἐκβάτανα, προσέπεσε τὰ κατὰ Νικάνορα, καὶ τοὺς περὶ Τιμόθεον, γεγονότα.
\VS{4}Ἐπαρθεὶς δὲ τῷ θυμῷ, ᾤετο καὶ τὴν τῶν πεφυγαδευκότων αὐτὸν κακίαν εἰς τοὺς Ἰουδαίους ἐναπερείσασθαι· διὸ συνέταξε τὸν ἁρματηλάτην ἀδιαλείπτως ἐλαύνοντα κατανύειν τὴν πορείαν, τῆς ἐξ οὐρανοῦ δὴ κρίσεως συνούσης αὐτῷ· οὕτως γὰρ ὑπερηφάνως εἶπε, πολυάνδριον Ἰουδαίων Ἱεροσόλυμα ποιήσω παραγενόμενος ἐκεῖ.
\par }{\PP \VS{5}Ὁ δὲ πανεπόπτης Κύριος ὁ Θεὸς τοῦ Ἰσραὴλ ἐπάταξεν αὐτὸν ἀνιάτῳ καὶ ἀοράτῳ πληγῇ· ἄρτι δὲ αὐτοῦ καταλήξαντος τὸν λόγον, ἔλαβεν αὐτὸν ἀνήκεστος τῶν σπλάγχνων ἀλγηδὼν, καὶ πικραὶ τῶν ἔνδον βάσανοι,
\VS{6}πάνυ δικαίως, τὸν πολλαῖς καὶ ξενιζούσαις συμφοραῖς ἑτέρων σπλάγχνα βασανίσαντα.
\par }{\PP \VS{7}Ὁ δʼ οὐδαμῶς τῆς ἀγερωχίας ἔληγεν· ἔτι δὲ καὶ τῆς ὑπερηφανίας ἐπεπλήρωτο, πῦρ πνέων τοῖς θυμοῖς ἐπὶ τοὺς Ἰουδαίους, καὶ κελεύων ἐποξύνειν τὴν πορείαν· συνέβη δὲ καὶ πεσεῖν αὐτὸν ἀπὸ τοῦ ἅρματος φερομένου ῥοίζῳ, καὶ δυσχερεῖ πτώματι περιπεσόντα, πάντα τὰ μέλη τοῦ σώματος ἀποστρεβλοῦσθαι.
\par }{\PP \VS{8}Ὁ δʼ ἄρτι δοκῶν τοῖς τῆς θαλάσσης κύμασιν ἐπιτάσσειν, διὰ τὴν ὑπὲρ ἄνθρωπον ἀλαζονείαν, καὶ πλάστιγγι τὰ τῶν ὀρέων οἰόμενος ὕψη στήσειν, κατὰ γῆν γενόμενος, ἐν φορείῳ παρεκομίζετο, φανερὰν τοῦ Θεοῦ πᾶσι τὴν δύναμιν ἐνδεικνύμενος·
\VS{9}ὥστε καὶ ἐκ τοῦ σώματος τοῦ δυσσεβοῦς σκώληκας ἀναζεῖν, καὶ ζῶντος ἐν ὀδύναις καὶ ἀλγηδόσι τὰς σάρκας αὐτοῦ διαπίπτειν, ὑπὸ δὲ τῆς ὀσμῆς αὐτοῦ πᾶν τὸ στρατόπεδον βαρύνεσθαι τῇν σαπρίᾳ.
\VS{10}Καὶ τὸν μικρῷ πρότερον τῶν οὐρανίων ἄστρων ἅπτεσθαι δοκοῦντα, παρακομίζειν οὐδεὶς ἐδύνατο, διὰ τὸ τῆς ὀσμῆς ἀφόρητον βάρος.
\par }{\PP \VS{11}Ἐνταῦθα οὖν ἤρξατο τὸ πολὺ τῆς ὑπερηφανίας λήγειν ὑποτεθραυσμένος, καὶ εἰς ἐπίγνωσιν ἔρχεσθαι θείᾳ μάστιγι κατὰ στιγμὴν ἐπιτεινόμενος ταῖς ἀλγηδόσι.
\VS{12}Καὶ μηδὲ τῆς ὀσμῆς αὐτοῦ δυνάμενος ἀνέχεσθαι, ταῦτʼ ἔφη, δίκαιον ὑποτάσσεσθαι τῷ Θεῷ, καὶ μὴ θνητὸν ὄντα ἰσόθεα φρονεῖν ὑπερηφανῶς.
\par }{\PP \VS{13}Ηὔχετο δὲ ὁ μιαρὸς πρὸς τὸν οὐκέτι αὐτὸν ἐλεήσοντα δεσπότην, οὕτω λέγων,
\VS{14}τὴν μὲν ἁγίαν πόλιν ἣν σπεύδων παρεγίνετο ἰσόπεδον ποιῆσαι, καὶ πολυάνδριον οἰκοδομῆσαι, ἐλευθέραν ἀναδεῖξαι·
\VS{15}τοὺς δὲ Ἰουδαίους, οὕς διεγνώκει μηδὲ ταφῆς ἀξιῶσαι, οἰωνοβρώτους δὲ σὺν τοῖς νηπίοις ἐκρίψειν θηρίοις, πάντας αὐτοὺς ἴσους Ἀθηναίοις ποιήσειν·
\VS{16}ὃν δὲ πρότερον ἐσκύλευσεν ἅγιον νεὼν, καλλίστοις ἀναθήμασι κοσμήσειν, καὶ τὰ ἱερὰ σκεύη πολυπλάσια πάντα ἀποδώσειν, τὰς δὲ ἐπιβαλλούσας πρὸς τὰς θυσίας συντάξεις ἐκ τῶν ἰδίων προσόδων χορηγήσειν·
\VS{17}πρὸς δὲ τούτοις, καὶ Ἰουδαῖο ἔσεσθαι, καὶ πάντα τόπον οἰκητὸν ἐπελεύσεσθαι καταγγέλλοντα τὸ τοῦ Θεοῦ κράτος.
\par }{\PP \VS{18}Οὐδαμῶς δὲ ληγόντων τῶν πόνων, ἐπεληλύθει γὰρ ἐπʼ αὐτὸν δικαία ἡ τοῦ Θεοῦ κρίσις, τὰ κατʼ αὐτὸν ἀπελπίσας, ἔγραψε πρὸς τοὺς Ἰουδαίους τὴν ὑπογεγραμμένην ἐπιστολήν, ἱκετηρίας τάξιν ἔχουσαν, περιέχουσαν δὲ οὕτως·
\par }{\PP \VS{19}Τοῖς χρηστοῖς Ἰουδαίοις τοῖς πολίταις πολλὰ χαίρειν, καὶ ὑγιαίνειν, καὶ εὖ πράττειν, βασιλεὺς καὶ στρατηγὸς Ἀντίοχος.
\VS{20}Εἰ ἔῤῥωσθε, καὶ τὰ τέκνα καὶ τὰ ἴδια κατὰ γνώμην ἔστιν ὑμῖν, εὔχομαι μὲν τῷ Θεῷ τὴν μεγίστην χάριν, εἰς οὐρανὸν τὴν ἐλπίδα ἔχων.
\par }{\PP \VS{21}Κᾀγὼ δὲ ἀσθενῶς διεκείμην, ὑμῶν τὴν τιμὴν καὶ τὴν εὔνοιαν ἄν ἐμνημόνευον φιλοστόργως· ἐπανάγων ἐκ τῶν περὶ τὴν Περσίδα τόπων, καὶ περιπεσὼν ἀσθενείᾳ δυσχέρειαν ἐχούσῃ, ἀναγκαῖον ἡγησάμην φροντίσαι τῆς κοινῆς πάντων ἀσφαλείας·
\VS{22}Οὐκ ἀπογινώσκων τὰ κατʼ ἐμαυτόν, ἀλλὰ ἔχων πολλὴν ἐλπίδα ἐκφεύξεσθαι τὴν ἀσθένειαν,
\VS{23}θεωρῶν δὲ ὅτι καὶ ὁ πατήρ καθʼ οὓς καιροὺς εἰς τοὺς ἄνω τόπους ἐστρατοπέδευσεν, ἀνέδειξε τὸν διαδεξόμενον,
\VS{24}ὅπως ἐάν τι παράδοξον ἀποβαίνῃ, ἤ καὶ προσαγγελθῇ τι δυοχερὲς, εἰδότες οἱ κατὰ τὴν χώραν ᾧ καταλέλειπται τὰ πράγματα, μὴ ἐπιταράσσωνται·
\par }{\PP \VS{25}Πρὸς δὲ τούτοις κατανοῶν τοὺς παρακειμένους δυνάστας, καὶ γειτνιῶντας τῇ βασιλείᾳ τοῖς καιροῖς ἐπέχοντας, προσδεχομένους τὸ ἀποβησόμενον, ἀναδέδειχα τὸν υἱὸν μου Ἀντίοχον βασιλέα, ὃν πολλάκις ἀνατρέχων εἰς τὰς ἐπάνω σατραπείας τοῖς πλείστοις ὑμῶν παρακατετιθέμην καὶ συνίστων· γέγραφα δὲ πρὸς αὐτὸν τὰ ὑπογεγραμμένα.
\par }{\PP \VS{26}Παρακαλῶ οὖν ὑμᾶς καὶ ἀξιῶ, μεμνημένους τῶν εὐεργεσιῶν κοινῇ καὶ κατιδίαν, ἕκαστον συντηρεῖν τὴν οὖσαν εὔνοιαν εἰς ἐμὲ καὶ τὸν υἱόν μου.
\VS{27}Πέπεισμαι γὰρ αὐτὸν ἐπιεικῶς καὶ φιλανθρώπως παρακολουθοῦντα τῇ ἐμῇ προαιρέσει, συμπεριενεχθήσεσθαι ὑμῖν.
\par }{\PP \VS{28}Ὁ μὲν οὖν ἀνδροφόνος καὶ βλάσφημος τὰ χείριστα παθών, ὡς ἑτέρους διέθηκεν, ἐπὶ ξένης ἐν τοῖς ὄρεσιν οἰκτίστῳ μόρῳ κατέσπρεψε τὸν βίον.
\VS{29}Παρεκομίζετο δὲ τὸ σῶμα Φίλιππος ὁ σύντροφος αὐτοῦ· ὃς καὶ διευλαβηθεὶς τὸν υἱὸν Ἀντιόχου, πρὸς Πτολεμαῖον τὸν Φιλομήτορα εἰς Αἴγυπτον διεκομίσθη.

\par }\Chap{10}{\PP \VerseOne{1}Μακκαβαῖος δὲ καὶ οἱ σὺν αὐτῷ, τοῦ Κυρίου προάγοντος αὐτοὺς, τὸ μὲν ἱερὸν ἐκομίσαντο καὶ τὴν πόλιν,
\VS{2}τοὺς δὲ κατὰ τὴν ἀγορὰν βωμοὺς ὑπὸ τῶν ἀλλοφύλων δεδημιουργημένους, ἔτι δὲ τεμένη καθεῖλον.
\par }{\PP \VS{3}Καὶ τὸν νεὼν καθαρίσαντες, ἕτερον θυσιαστήριον ἐποίησαν, καὶ πυρώσαντες λίθους, καὶ πῦρ ἐκ τούτων λαβόντες, ἀνήνεγκαν θυσίαν μετὰ διετῆ χρόνον, καὶ θυμίαμα καὶ λύχνους, καὶ τῶν ἄρτων τὴν πρόθεσιν ἐποιήσαντο.
\VS{4}Ταῦτα δὲ ποιήσαντες ἠξίωσαν τὸν Κύριον πεσόντες ἐπὶ κοιλίαν, μηκέτι περιπεσεῖν τοιούτοις κακοῖς, ἀλλʼ ἐάν ποτε καὶ ἁμάρτωσιν, ὑπʼ αὐτοῦ μετʼ ἐπιεικείας παιδεύεσθαι, καὶ μὴ βλασφήμοις καὶ βαρβάροις ἔθνεσι παραδίδοσθαι.
\par }{\PP \VS{5}Ἐν ᾗ δὲ ἡμέρᾳ ὁ νεὼς ὑπὸ ἀλλοφύλων ἐβεβηλώθη, συνέβη κατὰ τὴν αὐτὴν ἡμέραν τὸν καθαρισμὸν γενέσθαι τοῦ ναοῦ, τῇ πέμπτῃ καὶ εἰκάδι τοῦ αὐτοῦ μηνὸς, ὅς ἐστι Χασελεῦ.
\par }{\PP \VS{6}Καὶ μετʼ εὐφροσύνης ἦγον ἡμέρας ὀκτὼ σκηνωμάτων τρόπον, μνημονεύοντες ὡς πρὸ μικροῦ χρόνου τὴν τῶν σκηνῶν ἑορτὴν ἐν τοῖς ὄρεσι καὶ ἐν τοῖς σπηλαίοις θηρίων τρόπον ἦσαν νεμόμενοι.
\VS{7}Διὸ θύρσους καὶ κλάδους ὡραίους, ἔτι δὲ φοίνικας ἔχοντες, ὕμνους ἀνέφερον τῷ εὐοδώσαντι καθαρισθῆναι τὸν ἑαυτοῦ τόπον.
\VS{8}Ἐδογμάτισαν δὲ μετὰ κοινοῦ προστάγματος καὶ ψηφίσματος παντὶ τῷ τῶν Ἰουδαίων ἔθνει κατʼ ἐνιαυτὸν ἄγειν τάσδε τὰς ἡμέρας.
\par }{\PP \VS{9}Καὶ τὰ μὲν τῆς Ἀντιόχου τοῦ προσαγορευθέντος Ἐπιφανοῦς τελευτῆς οὕτως εἶχε.
\par }{\PP \VS{10}Νυνὶ δὲ τὰ κατὰ τὸν Εὐπάτορα Ἀντίοχον, υἱὸν δὲ τοῦ ἀσεβοῦς γενόμενον, δηλώσομεν, αὐτά συντέμνοντες τὰ τῶν πολέμων κακά.
\VS{11}Αὐτὸς γὰρ παραλαβὼν βασιλείαν, ἀνέδειξεν ἐπὶ τῶν πραγμάτων Λυσίαν τινά, κοίλης δὲ Συρίας καὶ Φοινίκης στρατηγὸν πρώταρχον.
\par }{\PP \VS{12}Πτολεμαῖος γὰρ ὁ καλούμενος Μάκρων τὸ δίκαιον συντηρεῖν προηγούμενος εἰς τοὺς Ἰουδαίους διὰ τὴν γεγονυῖαν εἰς αὐτοὺς ἀδικίαν, καί ἐπειρᾶτο τὰ πρὸς αὐτοὺς εἰρηνικῶς διεξάγειν.
\VS{13}Ὅθεν κατηγορούμενος ὑπὸ τῶν φίλων πρὸς τὸν Εὐπάτορα, καὶ προδότης παρέκαστα ἀκούων, διὰ τὸ τὴν Κύπρον ἐμπιστευθέντα ὑπὸ τοῦ Φιλομήτορος ἐκλιπεῖν, καὶ πρὸς Ἀντίοχον τὸν Ἐπιφανῆ ἀναχωρῆσαι, μήτʼ εὐγενῆ τὴν ἐξουσίαν ἔχων, ὑπʼ ἀθυμίας φαρμακεύσας ἑαυτὸν ἐξέλιπε τὸν βίον.
\par }{\PP \VS{14}Γοργίας δὲ γενόμενος στρατηγὸς τῶν τόπων, ἐξενοτρόφει, καὶ παρέκαστα πρὸς τοὺς Ἰουδαίους ἐπολεμοτρόφει,
\VS{15}Ὁμοῦ δὲ τούτῳ καὶ οἱ Ἰδουμαῖοι ἐγκρατεῖς ἐπικαίρων ὀχυρωμάτων ὄντες, ἐγύμναζον τοὺς Ἰουδαίους, καὶ τοὺς φυγαδευθέντας ἀπὸ Ἰεροσολύμων προσλαβόμενοι πολεμοτροφεῖν ἐπεχείρουν.
\par }{\PP \VS{16}Οἱ δὲ περὶ τὸν Μακκαβαῖον, ποιησάμενοι λιτανείαν, καὶ ἀξιώσαντες τὸν Θεὸν σύμμαχον αὐτοῖς γενέσθαι, ἐπὶ τὰ τῶν Ἰδουμαίων ὀχυρώματα ὥρμησαν,
\VS{17}οἷς καὶ προσβαλόντες εὐρώστως, ἐγκρατεῖς ἐγένοντο τῶν τόπων, πάντας τε τοὺς ἐπὶ τῷ τείχει μαχομένους ἠμύναντο· κατέσφαζόν δε τοὺς ἐμπίπτοντας, ἀνεῖλον δὲ οὐχ ἧττον τῶν δισμυρίων.
\par }{\PP \VS{18}Συμφυγόντων δὲ οὐκ ἔλαττον τῶν ἐννακισχιλίων εἰς δύο πύργους ὀχυροὺς εὖ μάλα, καὶ πάντα τὰ πρὸς πολιορκίαν ἔχοντας,
\VS{19}ὁ Μακκαβαῖος εἰς ἐπείγοντας τόπους ἀπολιπὼν Σίμωνα καὶ Ἰώσηφον, ἔτι δὲ καὶ Ζακχαῖον καὶ τοὺς σὺν αὐτῷ ἱκανοὺς πρὸς τὴν τούτων πολιορκίαν, αὐτὸς ἐχωρίσθη.
\par }{\PP \VS{20}Οἱ δὲ περὶ τὸν Σίμωνα φιλαργυρήσαντες ὑπό τινων τῶν ἐν τοῖς πύργοις ἐπείσθησαν ἀργυρίῳ· ἑπτάκις δὲ μυριάδας δραχμὰς λαβόντες, εἴασάν τινας διαῤῥυῆναι.
\VS{21}Προσαγγελθέντος δὲ τῷ Μακκαβαίῳ περὶ τοῦ γεγονότος, συναγαγὼν τοὺς ἡγουμένους τοῦ λαοῦ, κατηγόρησεν ὡς ἀργυρίου πεπράκασι τοὺς ἀδελφοὺς, τοὺς πολεμίους κατʼ αὐτῶν ἀπολύσαντες.
\VS{22}Τούτους μὲν οὖν προδότας γενομένους ἀπέκτεινε, καὶ παραχρῆμα τοὺς δύο πύργους κατελάβετο.
\VS{23}Τοῖς δὲ ὅπλοις τὰ πάντα ἐν ταῖς χερσὶν εὐοδούμενος, ἀπώλεσεν ἐν τοῖς δυσὶν ὀχυρώμασι πλείους τῶν δισμυρίων.
\par }{\PP \VS{24}Τιμόθεος δὲ ὁ πρότερον ἡττηθεὶς ὑπὸ τῶν Ἰουδαίων, συναγαγὼν ξένας δυνάμεις παμπληθεῖς, καὶ τοὺς τῆς Ἀσίας γενομένους ἵππους συναθροίσας οὐκ ὀλίγους, παρῆν ὡς δοριάλωτον ληψόμενος τὴν Ἰουδαίαν.
\VS{25}Οἱ δὲ περὶ τὸν Μακκαβαῖον, συνεγγίζοντος αὐτοῦ, πρὸς ἱκετείαν τοῦ Θεοῦ ἐτράπησαν, γῇ τὰς κεφαλὰς καταπάσαντες, καὶ τὰς ὀσφύας σάκκοις ζώσαντες,
\VS{26}ἐπὶ τὴν ἀπέναντι τοῦ θυσιαστηρίου κρηπίδα προσπεσόντες, ἠξίουν ἵλεων αὐτοῖς γενόμενον ἐχθρεῦσαι τοῖς ἐχθροῖς αὐτῶν, καὶ ἀντικεῖσθαι τοῖς ἀντικειμένοις, καθὼς ὁ νόμος διασαφεῖ.
\VS{27}Γενόμενοι δὲ ἀπὸ τῆς δεήσεως, ἀναλαβόντες τὰ ὅπλα, προῆγον ἀπὸ τῆς πόλεως ἐπὶ πλεῖον· συνεγγίσαντες δὲ τοῖς πολεμίοις, ἐφʼ ἑαυτῶν ἦσαν.
\par }{\PP \VS{28}Ἄρτι δὲ τῆς ἀνατολῆς διαδεχομένης, προσέβαλον ἑκάτεροι· οἱ μὲν ἔγγυον ἔχοντες εὐημερίας καὶ νίκης μετʼ ἀρετῆς τὴν ἐπὶ τὸν Κύριον καταφυγὴν, οἱ δὲ καθηγεμόνα τῶν ἀγώνων ταττόμενοι τὸν θυμόν.
\par }{\PP \VS{29}Γενομένης δὲ καρτερᾶς μάχης, ἐφάνησαν τοῖς ὑπεναντίοις ἐξ οὐρανοῦ ἐφʼ ἵππων χρυσοχαλίνων ἄνδρες πέντε διαπρεπεῖς, καὶ ἀφηγούμενοι τῶν Ἰουδαίων οἱ δύο,
\VS{30}καὶ τὸν Μακκαβαῖον μέσον λαβόντες, καὶ σκεπάζοντες ταῖς ἑαυτῶν πανοπλίαις, ἄτρωτον διεφύλαττον· εἰς δὲ τοὺς ὑπεναντίους τοξεύματα καὶ κεραυνοὺς ἐξεῤῥίπτουν· διὸ συγχυθέντες ἀορασίᾳ, κατεκόπτοντο ταραχῆς πεπληρωμένοι.
\VS{31}Κατεσφάγησαν δὲ δισμύριοι πρὸς τοῖς πεντακοσίοις, ἱππεῖς δὲ ἑξακόσιοι.
\par }{\PP \VS{32}Αὐτὸς δὲ ὁ Τιμόθεος συνέφυγεν εἰς Γάζαρα λεγόμενον ὀχύρωμα, εὖ μάλα φρούριον, στρατηγοῦντος ἐκεῖ Χαιρέου.
\par }{\PP \VS{33}Οἱ δὲ περὶ τὸν Μακκαβαῖον ἄσμενοι περιεκάθισαν τὸ φρούριον ἡμέρας τέσσαρας.
\VS{34}Οἱ δὲ ἔνδον τῇ ἐρυμνότητι τοῦ τόπου πεποιθότες, ὑπεράγαν ἐβλασφήμουν, καὶ λόγους ἀθεμίτους προΐοντο.
\par }{\PP \VS{35}Ὑποφαινούσης δὲ τῆς πέμπτης ἡμέρας, εἴκοσι νεανίαι τῶν περὶ τὸν Μακκαβαῖον πυρωθέντες τοῖς θυμοῖς διὰ τὰς βλασφημίας, προσβαλόντες τῷ τείχει, ἀῤῥενωδῶς καὶ θηριώδει θυμῷ τὸν ἐμπίπτοντα ἔκοπτον,
\VS{36}ἕτεροι δὲ ὁμοίως προσαναβάντες ἐν τῷ περισπασμῷ πρὸς τοὺς ἔνδον, ἐνεπίμπρων τοὺς πύργους, καὶ πυρὰς ἀνάψαντες ζῶντας τοὺς βλασφήμους κατέκαιον· οἱ δὲ τὰς πύλας διέκοπτον, εἰσδεξάμενοι δὲ τὴν λοιπὴν τάξιν, προκατελάβοντο τὴν πόλιν,
\VS{37}καὶ τὸν Τιμόθεον ἀποκεκρυμμένον ἔν τινι λάκκῳ κατέσφαξαν, καὶ τὸν τούτου ἀδελφὸν Χαιρέαν, καὶ τὸν Ἀπολλοφάνη.
\par }{\PP \VS{38}Ταῦτα δὲ διαπραξάμενοι, μεθʼ ὕμνων καὶ ἐξομολογήσεων εὐλόγουν τῷ Κυρίῳ τῷ μεγάλως εὐεργετοῦντι τὸν Ἰσραὴλ, καὶ τὸ νῖκος αὐτοῖς διδόντι.

\par }\Chap{11}{\PP \VerseOne{1}Μετʼ ὀλίγον δὲ παντελῶς χρόνον Λυσίας ἐπίτροπος τοῦ βασιλέως καὶ συγγενὴς, καὶ ἐπὶ τῶν πραγμάτων, λίαν βαρέως φέρων ἐπὶ τοῖς γεγονόσι,
\VS{2}συναθροίσας περὶ τὰς ὀκτὼ μυριάδας καὶ τὴν ἵππον πᾶσαν, παρεγένετο ἐπὶ τοὺς Ἰουδαίους, λογιζόμενος τὴν μὲν πόλιν Ἕλλησιν οἰκητήριον ποιήσειν,
\VS{3}τὸ δὲ ἱερὸν ἀργυρολόγητον καθὼς τὰ λοιπὰ τῶν ἐθνῶν τεμένη, πρατὴν δὲ τὴν ἀρχιερωσύνην κατʼ ἔτος ποιήσειν,
\VS{4}οὐδαμῶς ἐπιλογιζόμενος τὸ τοῦ Θεοῦ κράτος, πεφρενωμένος δὲ ταῖς μυριάσι τῶν πεζῶν καὶ ταῖς χιλιάσι τῶν ἱππέων καὶ τοῖς ἐλέφασι τοῖς ὀγδοήκοντα.
\par }{\PP \VS{5}Εἰσελθὼν δὲ εἰς τὴν Ἰουδαίαν, καὶ συνεγγίσας τῷ Βαιθσούρᾳ, ὄντι μὲν ἐρυμνῷ χωρίῳ, ἀπὸ δὲ Ἱεροσολύμων ἀπέχοντι ὡσεὶ σταδίους πέντε, τοῦτο ἔθλιβεν.
\par }{\PP \VS{6}Ὡς δὲ μετέλαβον οἱ περὶ τὸν Μακκαβαῖον πολιορκοῦντα αὐτὸν τὰ ὀχυρώματα, μετʼ ὀδύρμῶν καὶ δακρύων ἱκέτευον σὺν τοῖς ὄχλοις τὸν Κύριον, ἀγαθὸν ἄγγελον ἀποστεῖλαι πρὸς σωτηρίαν τῷ Ἰσραήλ.
\VS{7}Αὐτὸς δὲ πρῶτος ὁ Μακκαβαῖος ἀναλαβὼν τὰ ὅπλα προετρέψατο τοὺς ἄλλους, ἅμα αὐτῷ διακινδυνεύοντας, ἐπιβοηθεῖν τοῖς ἀδελφοῖς αὐτῶν· ὁμοῦ δὲ καὶ προθύμως ἐξώρμησαν.
\par }{\PP \VS{8}Αὐτόθι δὲ καὶ πρὸς τοῖς Ἱεροσολύμοις ὄντων, ἐφάνη προηγούμενος αὐτῶν ἔφιππος ἐν λευκῇ ἐσθῆτι, πανοπλίαν χρυσῆν κραδαίνων.
\VS{9}Ὁμοῦ δὲ πάντες εὐλόγησαν τὸν ἐλεήμονα Θεὸν, καὶ ἐπεῤῥώσθησαν ταῖς ψυχαῖς, οὐ μόνον ἀνθρώπους ἀλλὰ καὶ θῆρας τοὺς ἀγριωτάτους, καὶ σιδηρᾶ τείχη τιτρώσκειν ὄντες ἕτοιμοι.
\VS{10}Προσῆγον ἐν διασκευῇ τὸν ἀπʼ οὐρανοῦ σύμμαχον ἔχοντες, ἐλεήσαντος αὐτοὺς τοῦ Κυρίου.
\VS{11}Λεοντηδὸν δὲ ἐντινάξαντες εἰς τοὺς πολεμίους, κατέστρωσαν αὐτῶν χιλίους πρὸς τοῖς μυρίοις, ἱππεῖς δὲ ἑξακοσίους πρὸς τοῖς χιλίοις· τοὺς δὲ πάντας ἠνάγκασαν φυγεῖν.
\VS{12}Οἱ πλείονες δὲ αὐτῶν τραυματίαι γυμνοὶ διεσώθησαν· καὶ αὐτὸς δὲ ὁ Λυσίας αἰσχρῶς φεύγων διεσώθη.
\par }{\PP \VS{13}Οὐκ ἄνους δὲ ὑπάρχων, πρὸς ἑαυτὸν ἀντιβάλλων τὸ γεγονὸς περὶ ἐαυτὸν ἐλάσσωμα, καὶ συννοήσας ἀνικήτους εἶναι τοὺς Ἑβραίους, τοῦ πάντα δυναμένου Θεοῦ συμμαχοῦντος αὐτοῖς,
\VS{14}προσαποστείλας ἔπεισε συλλύσεσθαι ἐπὶ πᾶσι τοῖς δικαίοις· καὶ διότι καὶ τὸν βασιλέα πείσειν φίλον αὐτοῖς ἀναγκάζειν γενέσθαι.
\VS{15}Ἐπένευσε δὲ ὁ Μακκαβαῖος ἐπὶ πᾶσιν οἷς ὁ Λυσίας παρεκάλει τοῦ συμφέροντος φροντίζων· ὅσα γὰρ ὁ Μακκαβαῖος ἐπέδωκε τῷ Λυσίᾳ διὰ γραπτῶν περὶ τῶν Ἰουδαίων, συνεχώρησεν ὁ βασιλεύς.
\par }{\PP \VS{16}Ἦσαν γὰρ αἱ γεγραμμέναι τοῖς Ἰουδαίοις ἐπιοτολαὶ παρὰ μὲν Λυσίου περιέχουσαι τὸν τρόπον τοῦτον· Λυσίας τῷ πλήθει τῶν Ἰουδαίων χαίρειν.
\VS{17}Ἰωάννης καὶ Ἀβεσσαλὼμ οἱ πεμφθέντες παρʼ ὑμῶν, ἐπιδόντες τὸν ὑπογεγραμμένον χρηματισμὸν, ἠξίουν περὶ τῶν διʼ αὐτοῦ σημαινομένων.
\VS{18}Ὅσα μὲν οὖν ἔδει καὶ τῷ βασιλεῖ προσενεχθῆναι διεσάφησα, ἃ δὲ ἦν ἐνδεχόμενα, συνεχώρησεν.
\VS{19}Ἐὰν μὲν οὖν συντηρήσητε τὴν εἰς τὰ πράγματα εὔνοιαν, καὶ εἰς τὸ λοιπὸν πειράσομαι παραίτιος ὑμῖν ἀγαθῶν γενέσθαι.
\VS{20}Ὑπὲρ δὲ τῶν κατὰ μέρος ἐντέταλμαι τούτοις τε καὶ τοῖς παρʼ ἐμοῦ διαλεχθῆναι ὑμῖν.
\VS{21}Ἔῤῥωσθε· ἔτους ἑκατοστοῦ τεσσαρακοστοῦ ὀγδόου, Διοσκορινθίου εἰκοστῇ τετάρτῃ.
\par }{\PP \VS{22}Ἡ δὲ τοῦ βασιλέως ἐπιστολὴ περιεῖχεν οὕτως· βασιλεὺς Ἀντίοχος τῷ ἀδελφῷ Λυσίᾳ χαίρειν.
\VS{23}Τοῦ πατρὸς ἡμῶν εἰς θεοὺς μεταστάντος, βουλόμενοι τοὺς ἐκ τῆς βασιλείας ἀταράχους ὄντας γενέσθαι πρὸς τὴν τῶν ἰδίων ἐπιμέλειαν,
\VS{24}ἀκηκοότες τοὺς Ἰουδαίους μὴ συνευδοκοῦντας τῇ τοῦ πατρὸς ἐπὶ τὰ Ἑλληνικὰ μεταθέσει, ἀλλὰ τὴν ἑαυτῶν ἀγωγὴν αἱρετίζοντας, καὶ διὰ τοῦτο ἀξιοῦντας συγχωρηθῆναι αὐτοῖς τὰ νόμιμα αὐτῶν·
\VS{25}Αἱρούμενοι οὖν καὶ τοῦτο τὸ ἔθνος ἐκτὸς ταραχῆς εἶναι, κρίνομεν τό, τε ἱερὸν αὐτοῖς ἀποκατασταθῆναι, καὶ πολιτεύεσθαι κατὰ τὰ ἐπὶ τῶν προγόνων αὐτῶν ἔθη.
\VS{26}Εὖ οὖν ποιήσεις διαπεμψάμενος πρὸς αὐτοὺς καὶ δοὺς δεξιὰς, ὅπως εἰδότες τὴν ἡμετέραν προαίρεσιν, εὔθυμοί τε ὦσι, καὶ ἡδέως διαγίνωνται πρὸς τὴν τῶν ἰδίων ἀντίληψιν.
\par }{\PP \VS{27}Πρὸς δὲ τὸ ἔθνος ἡ τοῦ βασιλέως ἐπιστολὴ τοιαύτη ἦν· βασιλεὺς Ἀντίοχος τῇ γερουσίᾳ τῶν Ἰουδαίων καὶ τοῖς ἄλλοις Ἰουδαίοις χαίρειν.
\VS{28}Εἰ ἔῤῥωσθε, εἴη ἂν ὡς βουλόμεθα· καὶ αὐτοὶ δὲ ὑγιαίνομεν.
\VS{29}Ἐνεφάνισεν ἡμῖν ὁ Μενέλαος βούλεσθαι κατελθόντας ὑμᾶς γίνεσθαι πρὸς τοῖς ἰδίοις.
\VS{30}Τοῖς οὖν καταπορευομένοις μέχρι τριακάδος Ξανθικοῦ ὑπάρξει δεξιὰ μετὰ τῆς ἀδείας,
\VS{31}χρῆσθαι τοὺς Ἰουδαίους τοῖς ἑαυτῶν δαπανήμασι καὶ νόμοις καθὰ καὶ τὸ πρότερον, καὶ οὐδεὶς αὐτῶν κατʼ οὐδένα τρόπον παρενοχληθήσεται περὶ τῶν ἠγνοημένων.
\VS{32}Πέπομφα δὲ καὶ τὸν Μενέλαον παρακαλέσοντα ὑμᾶς.
\VS{33}Ἔῤῥωσθε· ἔτους ἑκατοστοῦ τεσσαρακοστοῦ ὀγδόου, Ξανθικοῦ πέμπτῃ καὶ δεκάτῃ.
\par }{\PP \VS{34}Ἔπεμψαν δὲ καὶ οἱ Ῥωμαιοῖ πρὸς αὐτοὺς ἐπιστολὴν ἔχουσαν οὕτως· Κόϊντος Μέμμιος, Τίτος Μάνλιος, πρεσβύται Ῥωμαίων, τῷ δήμῳ τῶν Ἰουδαίων χαίρειν.
\VS{35}Ὑπὲρ ὧν Λυσίας ὁ συγγενὴς τοῦ βασιλέως συνεχώρησεν ὑμῖν, καὶ ἡμεῖς συνευδοκοῦμεν.
\VS{36}Ἃ δὲ ἔκρινε προσανενεχθῆναι τῷ βασιλεῖ, πέμψατέ τινα παραχρῆμα ἐπισκεψάμενοι περὶ τούτων, ἵνα ἐκθῶμεν ὡς καθήκει ὑμῖν· ἡμεῖς γὰρ προσάγομεν πρὸς Ἀντιόχειαν.
\VS{37}Διὸ σπεύσατε, καὶ πέμψατέ τινας, ὅπως καὶ ἡμεῖς ἐπιγνῶμεν ὁποίας ἐστὲ γνώμης.
\VS{38}Ὑγιαίνετε· ἔτους ἑκατοστοῦ τεσσαρακοστοῦ ὀγδόου, Ξανθικοῦ πέμπτῃ καὶ δεκάτῃ.

\par }\Chap{12}{\PP \VerseOne{1}Γενομένων τῶν συνθηκῶν τούτων, ὁ μὲν Λυσίας ἀπῄει πρὸς τὸν βασιλέα, οἱ δὲ Ἰουδαῖοι περὶ τὴν γεωργίαν ἐγίνοντο.
\VS{2}Τῶν δὲ κατὰ τόπον στρατηγῶν Τιμόθεος καὶ Ἀπολλώνιος ὁ τοῦ Γενναίου, ἔτι δὲ Ἱερώνυμος καὶ Δημοφὼν, πρὸς δὲ τούτοις Νικάνωρ ὁ Κυπριάρχης, οὐκ εἴων αὐτοὺς εὐσταθεῖς, καὶ τὰ τῆς ἡσυχίας ἄγειν.
\par }{\PP \VS{3}Ἰοππίται δὲ τηλικοῦτο συνετέλεσαν τὸ δυσσέβημα· παρακαλέσαντες τοὺς σὺν αὐτοῖς οἰκοῦντας Ἰουδαίους ἐμβῆναι εἰς τὰ παρασταθέντα ὑπʼ αὐτῶν σκάφη σὺν γυναιξὶ καὶ τέκνοις, ὡς μηδεμιᾶς ἐνεστώσης πρὸς αὐτοὺς δυσμενείας,
\VS{4}κατὰ δὲ τὸ κοινὸν τῆς πόλεως ψήφισμα, καὶ τούτων ἐπιδεξαμένων ὡς ἂν εἰρηνεύειν θελόντων, καὶ μηδὲν ὕποπτον ἐχόντων, ἐπαναχθέντας αὐτοὺς ἐβύθισαν, ὄντας οὐκ ἔλαττον τῶν διακοσίων.
\par }{\PP \VS{5}Μεταλαβὼν δὲ Ἰούδας τὴν γεγονυῖαν εἰς τοὺς ὁμοεθνεῖς ὠμότητα, παραγγείλας τοῖς περὶ αὐτὸν ἀνδράσι,
\VS{6}καὶ ἐπικαλεσάμενος τὸν δίκαιον κριτὴν Θεὸν, παρεγένετο ἐπὶ τοὺς μιαιφόνους τῶν ἀδελφῶν· καὶ τὸν μὲν λιμένα νύκτωρ ἐνέπρησε, καὶ τὰ σκάφη κατέφλεξε, τοὺς δὲ ἐκεῖ συμφυγόντας ἐξεκέντησε.
\VS{7}Τοῦ δὲ χωρίου συγκλεισθέντος, ἀνέλυσεν, ὡς πάλιν ἥξων καὶ τὸ σύμπαν τῶν Ἰοππιτῶν ἐκριζῶσαι πολίτευμα.
\par }{\PP \VS{8}Μεταλαβὼν δὲ καὶ τοὺς ἐν Ἰαμνείᾳ τὸν αὐτὸν ἐπιτελεῖν βουλομένους τρόπον τοῖς παροικοῦσιν Ἰουδαίοις,
\VS{9}καὶ τοῖς Ἰαμνίταις νυκτὸς ἐπιβαλὼν, ὑφῆψε τὸν λιμένα σὺν τῷ στόλῳ, ὥστε φαίνεσθαι τὰς αὐγὰς τοῦ φέγγους εἰς τὰ Ἱεροσόλυμα, σταδίων ὄντων διακοσίων τεσσαράκοντα.
\par }{\PP \VS{10}Ἐκεῖθεν δὲ ἀποσπασθέντων σταδίους ἐννέα, ποιουμένων τὴν πορείαν ἐπὶ τὸν Τιμόθεον, προσέβαλον Ἄραβες αὐτῷ οὐκ ἐλάττους τῶν πεντακισχιλίων, ἱππεῖς δὲ πεντακόσιοι.
\VS{11}Γενομένης δὲ καρτερᾶς μάχης, καὶ τῶν περὶ τὸν Ἰούδαν διὰ τὴν παρὰ τοῦ Θεοῦ βοήθειαν εὐημερησάντων, ἐλαττωθέντες οἱ Νομάδες Ἄραβες ἠξίουν δοῦναι τὸν Ἰούδαν δεξιὰν αὐτοῖς, ὑπισχνούμενοι καὶ βοσκήματα δώσειν, καὶ ἐν τοῖς λοιποῖς ὠφελήσειν αὐτούς.
\par }{\PP \VS{12}Ἰούδας δὲ ὑπολαβὼν ὡς ἀληθῶς ἐν πολλοῖς αὐτοὺς χρησίμους, ἐπεχώρησεν εἰρήνην ἄξειν πρὸς αὐτούς· καὶ λαβόντες δεξιὰς, εἰς τὰς σκηνὰς αὐτῶν ἐχωρίσθησαν.
\par }{\PP \VS{13}Ἐπέβαλε δὲ καὶ ἐπί τινα πόλιν γεφυροῦν ὀχυρὰν καὶ τείχεσι περιπεφραγμένην, καὶ παμμιγέσιν ἔθνεσι κατοικουμένην, ὄνομα δὲ Κάσπιν.
\VS{14}Οἱ δʼ ἔνδον πεποιθότες τῇ τῶν τειχέων ἐρυμνότητι, τῇ τε τῶν βρωμάτων παραθέσει, ἀναγωγότερον ἐχρῶντο, τοῖς περὶ τὸν Ἰούδαν λοιδοροῦντες, καὶ προσέτι βλασφημοῦντες, καὶ λαλοῦντες ἃ μὴ θέμις.
\VS{15}Οἱ δὲ περὶ τὸν Ἰούδαν ἐπικαλεσάμενοι τὸν μέγαν τοῦ κόσμου δυνάστην, τὸν ἄτερ κριῶν καὶ μηχανῶν ὀργανικῶν κατακρημνίσαντα τὴν Ἱεριχὼ κατὰ τοὺς Ἰησοῦ χρόνους, ἐνέσεισαν θηριωδῶς τῷ τείχει.
\VS{16}Καταλαβόμενοί τε τὴν πόλιν τῇ τοῦ Θεοῦ θελήσει, ἀμυθήτους ἐποιήσαντο σφαγὰς, ὥστε τὴν παρακειμένην λίμνην τὸ πλάτος ἔχουσαν σταδίων δύο, κατάῤῥυτον αἵματι πεπληρωμένην φαίνεσθαι.
\par }{\PP \VS{17}Ἐκεῖθεν δὲ ἀποσπάσαντες σταδίους ἑπτακοσίους πεντήκοντα διήνυσαν εἰς τὸν Χάρακα, πρὸς τοὺς λεγομένους Τουβιήνους Ἰουδαίους.
\VS{18}Καὶ Τιμόθεον μὲν ἐπὶ τῶν τόπων οὐ κατέλαβον, ἄπρακτόν τε ἀπὸ τῶν τόπων ἐκλελυκότα, καταλελοιπότα δὲ φρουρὰν ἔν τινι τόπῳ, καὶ μάλα ὀχυράν.
\VS{19}Δωσίθεος δὲ καὶ Σωσίπατρος τῶν περὶ τὸν Μακκαβαῖον ἡγεμόνων, ἐξοδεύσαντες ἀπώλεσαν τοὺς ὑπὸ Τιμοθέου καταλειφθέντας ἐν τῷ ὀχυρώματι πλείους τῶν μυρίων ἀνδρῶν.
\par }{\PP \VS{20}Ὁ δὲ Μακκαβαῖος διατάξας τὴν ἑαυτοῦ στρατιὰν σπειρηδὸν, κατέστησεν αὐτοὺς ἐπὶ τῶν σπείρων, καὶ ἐπὶ τὸν Τιμόθεον ὥρμησεν ἔχοντα περὶ αὐτὸν μυριάδας δώδεκα πεζῶν, ἱππεῖς δὲ χιλίους πρὸς τοῖς πεντακοσίοις.
\par }{\PP \VS{21}Τὴν δὲ ἔφοδον μεταλαβὼν Ἰούδα, ὁ Τιμόθεος προεξαπέστειλε τὰς γυναῖκας, καὶ τὰ τέκνα, καὶ τὴν ἄλλην ἀποσκευὴν εἰς τὸ λεγόμενον Καρνίον· ἦν γὰρ δυσπολιόρκητον καὶ δυσπρόσιτον τὸ χωρίον διὰ τὴν τῶν πάντων τῶν τόπων στενότητα.
\par }{\PP \VS{22}Ἐπιφανείσης δὲ τῆς Ἰούδα σπείρας πρώτης, καὶ γενομένου δέους ἐπὶ τοὺς πολεμίους, φόβου τε ἐκ τῆς τοῦ πάντα ἐφορῶντος ἐπιφανείας γενομένου ἐπʼ αὐτοὺς, εἰς φυγὴν ὥρμησαν ἄλλος ἀλλαχῃ φερόμενος, ὥστε πολλάκις ὑπὸ τῶν ἰδίων βλάπτεσθαι, καὶ ταῖς τῶν ξιφῶν ἀκμαῖς ἀναπείρεσθαι.
\VS{23}Ἐποιεῖτο δὲ τὸν διωγμὸν εὐτονώτερον Ἰούδας, συγκεντῶν τοὺς ἀλιτηρίους, διέφθειρέ τε εἰς μυριάδας τρεῖς ἀνδρῶν.
\par }{\PP \VS{24}Αὐτὸς δὲ ὁ Τιμόθεος ἐμπεσὼν τοῖς περὶ τὸν Δωσίθεον καὶ Σωσίπατρον, ἠξίου μετὰ πολλῆς γοητείας ἐξαφεῖναι σῶον αὐτόν· διὰ τὸ πλειόνων μὲν γονεῖς, ὧν δὲ ἀδελφοὺς ἔχειν, καὶ τούτους ἀλογηθῆναι συμβήσεται, εἰ ἀποθάνοι.
\VS{25}Πιστώσαντος δὲ αὐτοῦ διὰ πλειόνων τὸν ὁρισμὸν ἀποκαταστήσειν τούτους ἀπημάντους, ἀπέλυσαν αὐτὸν ἕνεκα τῆς τῶν ἀδελφῶν σωτηρίας.
\par }{\PP \VS{26}Ἐξελθὼν δὲ ἐπὶ τὸ Καρνίον καὶ τὸ Ἀταργατεῖον, κατέσφαξε μυριάδας σωμάτων δύο καὶ πεντακισχιλίους.
\par }{\PP \VS{27}Καὶ μετὰ τὴν τούτων τροπὴν καὶ ἀπώλειαν ἐπεστράτευσεν Ἰούδας καὶ ἐπὶ Ἐφρὼν, πόλιν ὀχυρὰν, ἐν ᾗ κατῴκει Λυσίας, καὶ πάμφυλα πλήθη· νεανίαι δὲ πρὸ τῶν τειχῶν καθεστῶτες ῥωμαλέοι ἀπεμάχοντο εὐρώστως, ἐνθάδε ὀργάνων καὶ βελῶν πολλαὶ παραθέσεις ὑπῆρχον.
\VS{28}Ἐπικαλεσάμενοι δὲ τὸν Δυνάστην τὸν μετὰ κράτους συντρίβοντα τὰς τῶν πολεμίων ἀλκὰς, ἔλαβον τὴν πόλιν ὑποχείριον, καὶ κατέστρωσαν τῶν ἔνδον εἰς μυριάδας δύο καὶ πεντακισχιλίους.
\par }{\PP \VS{29}Ἀναζεύξαντες δὲ ἐκεῖθεν, ὥρμησαν ἐπὶ Σκυθῶν πόλιν, ἀπέχουσαν ἀπὸ Ἱεροσολύμων σταδίους ἑξακοσίους.
\VS{30}Ἀπομαρτυρησάντων δὲ τῶν ἐκεῖ κατοικούντων Ἰουδαίων, ἣν οἱ Σκυθοπολίται ἔσχον πρὸς αὐτοὺς εὔνοιαν, καὶ ἐν τοῖς τῆς ἀτυχίας καιροῖς ἥμερον ἀπάντησιν ἐποιοῦντο,
\VS{31}εὐχαριστήσαντες αὐτοῖς, καὶ προσπαρακαλέσαντες καὶ εἰς τὰ λοιπὰ πρὸς τὸ γένος εὐμενεῖς εἶναι, παρεγένοντο εἰς Ἱεροσόλυμα, τῆς τῶν ἑβδομάδων ἑορτῆς οὕσης ὑπογύου.
\par }{\PP \VS{32}Μετὰ δὲ τὴν λεγομένην Πεντηκοστὴν, ὥρμησαν ἐπὶ Γοργίαν τὸν τῆς Ἰδουμαίας στρατηγόν.
\VS{33}Ἐξῆλθε δὲ μετὰ πεζῶν τρισχιλίων, ἱππέων δὲ τετρακοσίων.
\VS{34}Καὶ παραταξαμένων συνέβη πεσεῖν ὀλίγους τῶν Ἰουδαίων.
\VS{35}Δωσίθεος δέ τις τῶν τοῦ Βακήνορος, ἔφιππος ἀνὴρ καὶ καρτερὸς, εἴχετο τοῦ Γοργίου, καὶ λαβόμενος τῆς χλαμύδος, ἦγεν αὐτὸν εὐρώστως, καὶ βουλόμενος τὸν κατάρατον λαβεῖν ζωγρίαν, τῶν ἱππέων Θρακῶν τινὸς ἐπενεχθέντος αὐτῷ καὶ τὸν ὦμον καθελόντος, διέφυγεν ὁ Γοργίας εἰς Μαρισά.
\par }{\PP \VS{36}Τῶν δὲ περὶ τὸν Ἔσδριν ἐπιπλεῖον μαχομένων, καὶ κατακόπων ὄντων, ἐπικαλεσάμενος ὁ Ἰούδας τὸν Κύριον σύμμαχον φανῆναι καὶ προοδηγὸν τοῦ πολέμου,
\VS{37}καταρξάμενος τῇ πατρίῳ φωνῇ τὴν μεθʼ ὕμνων κραυγὴν, ἀναβοήσας, καὶ ἐνσείσας ἀπροσδοκήτως τοῖς περὶ τὸν Γοργίαν, τροπὴν αὐτῶν ἐποιήσατο.
\VS{38}Ἰούδας δὲ ἀναλαβὼν τὸ στράτευμα, ἦγεν εἰς Ὀδολλὰμ πόλιν· τῆς δὲ ἑβδομάδος ἐπιβαλλούσης, κατὰ τὸν ἐθισμὸν ἁγνισθέντες αὐτόθι τὸ σάββατον διήγαγον.
\par }{\PP \VS{39}Τῇ δὲ ἐχομένῃ ἦλθον οἱ περὶ τὸν Ἰούδαν καθʼ ὃν τρόπον τὸ τῆς χρείας ἐγεγόνει, τὰ τῶν προπεπτωκότων σώματα ἀνακομίσασθαι, καὶ μετὰ τῶν συγγενῶν ἀποκαταστῆσαι εἰς τοὺς πατρῴους τάφους.
\VS{40}Εὗρον δὲ ἑκάστου τῶν τεθνηκότων ὑπὸ τοὺς χιτῶνας ἱερώματα τῶν ἀπὸ Ἰαμνείας εἰδώλων, ἀφʼ ὧν ὁ νόμος ἀπείργει τοὺς Ἰουδαίους· τοῖς δὲ πᾶσι σαφὲς ἐγένετο διὰ τήνδε τὴν αἰτίαν τούσδε πεπτωκέναι.
\VS{41}Πάντες οὖν εὐλογήσαντες τοῦ δικαιοκρίτου Κυρίου τοῦ τὰ κεκρυμμένα φανερὰ ποιοῦντος,
\VS{42}εἰς ἱκετείαν ἐτράπησαν, ἀξιώσαντες τὸ γεγονὸς ἁμάρτημα τελείως ἐξαλειφθῆναι· ὁ δὲ γενναῖος Ἰούδας παρεκάλεσε τὸ πλῆθος συντηρεῖν ἑαυτοὺς ἀναμαρτήτους εἶναι, ὑπʼ ὄψιν ἑωρακότας τὰ γεγονότα, διὰ τὴν τῶν προπεπτωκότων ἁμαρτίαν.
\par }{\PP \VS{43}Ποιησάμενός τε κατʼ ἀνδραλογίαν κατασκευάσματα εἰς ἀργυρίου δραχμὰς δισχιλίας, ἀπέστειλεν εἰς Ἱεροσόλυμα προσαγαγεῖν περὶ ἁμαρτίας θυσίαν, πάνυ καλῶς καὶ ἀστείως πράττων, ὑπὲρ ἀναστάσεως διαλογιζόμενος·
\VS{44}εἰ γὰρ μὴ τοὺς προπεπτωκότας ἀναστῆναι προσεδόκα, περισσὸν ἂν ἦν καὶ ληρῶδες ὑπὲρ νεκρῶν προσεύχεσθαι·
\VS{45}εἶτʼ ἐμβλέπων τοῖς μετʼ εὐσεβείας κοιμωμένοις κάλλιστον ἀποκείμενον χαριστήριον· ὁσία καὶ εὐσεβὴς ἡ ἐπίνοια· ὅθεν περὶ τῶν τεθνηκότων τὸν ἐξιλασμὸν ἐποιήσατο, τῆς ἁμαρτίας ἀπολυθῆναι.

\par }\Chap{13}{\PP \VerseOne{1}Τῷ δὲ ἐννάτῳ καὶ τεσσαρακοστῷ καὶ ἑκατοστῷ ἔτει προσέπεσε τοῖς περὶ τὸν Ἰούδαν, Ἀντίοχον τὸν Εὐπάτορα παραγενέσθαι σὺν πλήθεσιν ἐπὶ τὴν Ἰουδαίαν,
\VS{2}καὶ σὺν αὐτῷ Λυσίαν τὸν ἐπίτροπον καὶ ἐπὶ τῶν πραγμάτων, ἕκαστον ἔχοντα δύναμιν Ἑλληνικὴν πεζῶν μυριάδας ἕνδεκα, καὶ ἱππεῖς πεντακισχιλίους τριακοσίους, καὶ ἐλέφαντας εἰκοσιδύο, ἅρματα δὲ δρεπανηφόρα τριακόσια.
\par }{\PP \VS{3}Καὶ Μενέλαος δὲ συνέμιξεν αὐτοῖς, καὶ παρεκάλει μετὰ πολλῆς εἰρωνείας τὸν Ἀντίοχον, οὐκ ἐπὶ σωτηρίᾳ τῆς πατρίδος, οἰόμενος δὲ ἐπὶ τῆς ἀρχῆς κατασταθήσεσθαι.
\VS{4}Ὁ δὲ βασιλεὺς τῶν βασιλέων ἐξήγειρε τὸν θυμὸν τοῦ Ἀντιόχου ἐπὶ τὸν ἀλιτήριον, καὶ Λυσίου ὑποδείξαντος τοῦτον αἴτιον εἶναι πάντων τῶν κακῶν, προσέταξεν, ὡς ἔθος ἐστὶν ἐν τῷ τόπῳ, προσαπολέσαι ἀγαγόντας αὐτὸν εἰς Βέροιαν.
\par }{\PP \VS{5}Ἔστι δὲ ἐν τῷ τόπῳ πύργος πεντήκοντα πηχῶν πλήρης σποδοῦ οὗτος δὲ ὄργανον εἶχε περιφερὲς πάντοθεν ἀπόκρημνον εἰς τὴν σποδόν.
\VS{6}Ἐνταῦθα τὸν ἱεροσυλίας ἔνοχον ὄντα, ἢ καί τινων ἄλλων κακῶν ὑπεροχὴν πεποιημένον, ἅπαντες προσωθοῦσιν εἰς ὄλεθρον.
\VS{7}Τοιούτῳ μόρῳ τὸν παράνομον συνέβη θανεῖν, μηδὲ τῆς γῆς τυχόντα Μενέλαον·
\VS{8}πάνυ δικαίως. Ἐπεὶ γὰρ συνετελέσατο πολλὰ περὶ τὸν βωμὸν ἁμαρτήματα, οὗ τὸ πῦρ ἁγνὸν ἦν καὶ ἡ σποδὸς, ἐν σποδῷ τὸν θάνατον ἐκομισατο.
\par }{\PP \VS{9}Τοῖς δὲ φρονήμασιν ὁ βασιλεὺς βεβαρβαρωμένος ἤρχετο, τὰ χείριστα τῶν ἐπὶ τοῦ πατρὸς αὐτοῦ γεγονότων ἐνδειξόμενος τοῖς Ἰουδαίοις.
\VS{10}Μεταλαβὼν δὲ Ἰούδας ταῦτα, παρήγγειλε τῷ πλήθει διʼ ἡμέρας καὶ νυκτὸς ἐπικαλεῖσθαι τὸν Κύριον, εἴποτε ἄλλοτε, καὶ νῦν ἐπιβοηθεῖν τοῖς τοῦ νόμου καὶ πατρίδος καὶ ἱεροῦ ἁγίου στερεῖσθαι μέλλουσι,
\VS{11}καὶ τὸν ἄρτι βραχέως ἀνεψυχότα λαὸν μὴ ἐᾶσαι τοῖς δυσφήμοις ἔθνεσιν ὑποχειρίους γενέσθαι.
\par }{\PP \VS{12}Πάντων δὲ τὸ αὐτὸ ποιησάντων ὁμοῦ καὶ καταξιωσάντων τὸν ἐλεήμονα Κύριον μετὰ κλαυθμοῦ καὶ νηστειῶν καὶ προπτώσεως ἐφʼ ἡμέρας τρεῖς ἀδιαλείπτως, παρακαλέσας αὐτοὺς ὁ Ἰούδας ἐκέλευσε παραγίνεσθαι.
\par }{\PP \VS{13}Καθʼ ἑαυτὸν δὲ σὺν τοῖς πρεσβυτέροις γενόμενος, ἐβουλεύσατο πρὶν εἰσβαλεῖν τοῦ βασιλέως τὸ στράτευμα εἰς τὴν Ἰουδαίαν, καὶ γενέσθαι τῆς πόλεως ἐγκρατεῖς, ἐξελθόντας κρῖναι τὰ πράγματα τῇ τοῦ Κυρίου βοηθείᾳ.
\par }{\PP \VS{14}Δοὺς δὲ τὴν ἐπιτροπὴν τῷ κτίστῃ τοῦ κόσμου, παρακαλέσας τοὺς σὺν αὐτῷ γενναίως ἀγωνίσασθαι μέχρι θανάτου περὶ νόμων, περὶ ἱεροῦ, πόλεως, πατρίδος, πολιτείας, ἐποιήσατο περὶ Μωδεῒν τὴν στρατοπεδείαν.
\VS{15}Δοὺς δὲ τοῖς περὶ αὐτὸν σύνθημα Θεοῦ νίκης, μετὰ νεανίσκων ἀρίστων κεκριμένων ἐπιβαλὼν νύκτωρ ἐπὶ τὴν βασιλικὴν αὐλὴν, ἐν τῇ παρεμβολῇ ἀνεῖλεν εἰς ἄνδρας τετρακισχιλίους, καὶ τὸν πρωτεύοντα τῶν ἐλεφάντων σὺν τῷ κατʼ οἰκίαν ὄχλῳ συνέθηκε,
\VS{16}καὶ τὸ τέλος τὴν παρεμβολὴν δέους καὶ ταραχῆς ἐπλήρωσαν, καὶ ἐξέλυσαν εὑημεροῦντες.
\VS{17}Ὑποφαινούσης δὲ ἤδη τῆς ἡμέρας τοῦτʼ ἐγεγόνει, διὰ τὴν ἐπαρήγουσαν αὐτῷ τοῦ Κυρίου σκέπην.
\par }{\PP \VS{18}Ὁ δὲ βασιλεὺς εἰληφὼς γεῦσιν τῆς τῶν Ἰουδαίων εὐτολμίας, κατεπείρασε διὰ μεθόδων τοὺς τόπους.
\VS{19}Καὶ ἐπὶ Βαιθσούρᾳ φρούριον ὀχυρὸν τῶν Ἰουδαίων προσῆγεν· καὶ ἐτροποῦτο, προσέκρουεν, ἠλαττονοῦτο.
\VS{20}Τοῖς δὲ ἔνδον Ἰούδας τὰ δέοντα εἰσέπεμψε.
\par }{\PP \VS{21}Προσήγγειλε δὲ τὰ μυστήρια τοῖς πολεμίοις Ῥόδοκος ἐκ τῆς Ἰουδαϊκῆς τάξεως· ἀνεζητήθη δὲ, καὶ κατελήφθη, καὶ κατεκλείσθη.
\par }{\PP \VS{22}Ἐδευτερολόγησεν ὁ βασιλεὺς τοῖς ἐν Βαιθσούρᾳ δεξιὰν ἔδωκεν, ἔλαβεν, ἀπῄει, προσέβαλε τοῖς περὶ τὸν Ἰούδαν, ἥττων ἐγένετο,
\VS{23}μετέλαβεν ἀπονενοῆσθαι τὸν Φίλιππον ἐν Ἀντιοχείᾳ τὸν ἀπολελειμμένον ἐπὶ τῶν πραγμάτων, συνεχύθη· τοὺς Ἰουδαίους παρεκάλεσεν, ὑπετάγη, καὶ ὤμοσεν ἐπὶ πᾶσι τοῖς δικαίοις· συνελύθη καὶ θυσίαν προσήγαγεν, ἐτίμησε τὸν νεὼν, καὶ τὸν τόπον ἐφιλανθρώπησε,
\VS{24}καὶ τὸν Μακκαβαῖον ἀπεδέξατο· κατέλιπε στρατηγὸν ἀπὸ Πτολεμαΐδος ἕως τῶν Γεῤῥηνῶν ἡγεμονίδην,
\VS{25}ἦλθεν εἰς Πτολεμαΐδα· ἐδυσφόρουν περὶ τῶν συνθηκῶν οἱ Πτολεμαεῖς, ἐδείναζον γὰρ ὑπὲρ ὧν ἠθέλησαν ἀθετεῖν τὰς διαστάλσεις.
\par }{\PP \VS{26}Προσῆλθεν ἐπὶ τὸ βῆμα Λυσίας, ἀπελογήσατο ἐνδεχομένως, συνέπεισε, κατεπρᾴυνεν, εὐμενεῖς ἐποίησεν, ἀνέζευξεν εἰς Ἀντιόχειαν· οὕτω τὰ τοῦ βασιλέως τῆς ἐφόδου καὶ τῆς ἀναζυγῆς ἐχώρησε.

\par }\Chap{14}{\PP \VerseOne{1}Μετὰ δὲ τριετῆ χρόνον προσέπεσε τοῖς περὶ τὸν Ἰούδαν, Δημήτριον τὸν τοῦ Σελεύκου διὰ τοῦ κατὰ Τρίπολιν λιμένος εἰσπλεύσαντα μετὰ πλήθους ἰσχυροῦ καὶ στόλου,
\VS{2}κεκρατηκέναι τῆς χώρας, ἐπανελόμενον Ἀντίοχον καὶ τὸν τούτου ἐπίτροπον Λυσίαν.
\par }{\PP \VS{3}Ἄλκιμος δέ τις προγενόμενος ἀρχιερεὺς, ἑκουσίως δὲ μεμολυμμένος ἐν τοῖς τῆς ἐπιμιξίας χρόνοις, συννοήσας ὅτι καθʼ ὁντιναοῦν τρόπον οὐκ ἔστιν αὐτῷ σωτηρία, οὐδὲ πρὸς ἅγιον θυσιαστήριον ἔτι πρόσοδος,
\VS{4}ἧκε πρὸς τὸν βασιλέα Δημήτριον πρώτῳ καὶ πεντηκοστῷ καὶ ἑκατοστῷ ἔτει, προσάγων αὐτῷ στέφανον χρυσοῦν καὶ φοίνικα, πρὸς δὲ τούτοις τῶν νομιζομένων θαλλῶν τοῦ ἱεροῦ· καὶ τὴν ἡμέραν ἐκείνην ἡσυχίαν ἔσχε.
\par }{\PP \VS{5}Καιρὸν δὲ λαβὼν τῆς ἰδίας ἀνοίας συνεργὸν, προσκληθεὶς εἰς συνέδριον ὑπὸ τοῦ Δημητρίου, καὶ ἐπερωτηθεὶς ἐν τίνι διαθέσει καὶ βουλῇ καθεστήκασιν οἱ Ἰουδαῖοι, πρὸς ταῦτα ἔφη,
\VS{6}οἱ λεγόμενοι τῶν Ἰουδαίων Ἀσιδαῖοι, ὧν ἀφηγεῖται Ἰούδας ὁ Μακκαβαῖος, πολεμοτροφοῦσι καὶ στασιάζουσιν, οὐκ ἐῶντες τὴν βασιλείαν εὐσταθείας τυχεῖν.
\par }{\PP \VS{7}Ὅθεν ἀφελόμενος τὴν προγονικὴν δόξαν, λέγω δὴ τὴν ἀρχιερωσύνην, δεῦρο νῦν ἐλήλυθα.
\VS{8}Πρῶτον μὲν ὑπὲρ τῶν ἀνηκόντων τῷ βασιλεῖ γνησίως φρονῶν, δεύτερον δὲ καὶ τῶν ἰδίων πολιτῶν στοχαζόμενος· τῇ μὲν γὰρ τῶν προειρημένων ἀλογιστίᾳ τὸ σύμπαν ἡμῶν γένος οὐ μικρῶς ἀκληρεῖ.
\par }{\PP \VS{9}Ἕκαστα δὲ τούτων ἐπεγνωκὼς σὺ βασιλεῦ, καὶ τῆς χώρας καὶ τοῦ περιϊσταμένου γένους ἡμῶν προνοήθητι, καθʼ ἣν ἔχεις πρὸς ἅπαντας εὐαπάντητον φιλανθρωπίαν.
\VS{10}Ἄχρι γὰρ Ἰούδας περίεστιν, ἀδύνατον εἰρήνης τυχεῖν τὰ πράγματα.
\VS{11}Τοιούτων δὲ ῥηθέντων ὑπὸ τούτου, θᾶττον οἱ λοιποὶ φίλοι δυσμενῶς ἔχοντες τὰ πρὸς τὸν Ἰούδαν προσεπύρωσαν τὸν Δημήτριον.
\par }{\PP \VS{12}Προσκαλεσάμενος δὲ εὐθέως Νικάνορα τὸν γενόμενον ἐλεφαντάρχην, καὶ στρατηγὸν ἀναδείξας τῆς Ἰουδαίας, ἐξαπέστειλε,
\VS{13}δοὺς ἐντολὰς, αὐτὸν μὲν τὸν Ἰούδαν ἐπανελέσθαι, τοὺς δὲ σὺν αὐτῷ σκορπίσαι, καταστῆσαι δὲ Ἄλκιμον ἀρχιερέα τοῦ μεγίστου ἱεροῦ.
\VS{14}Τὰ δὲ ἐκ τῆς Ἰουδαίας πεφυγαδευκότα τὸν Ἰούδαν ἔθνη συνέμισγον ἀγεληδὸν τῷ Νικάνορι, τὰς τῶν Ἰουδαίων ἀτυχίας καὶ συμφορὰς, ἰδίας εὐημερίας δοκοῦντες ἔσεσθαι.
\par }{\PP \VS{15}Ἀκούσαντες δὲ τὴν τοῦ Νικάνορος ἔφοδον καὶ τὴν ἐπίθεσιν τῶν ἐθνῶν, καταπασάμενοι γῆν ἐλιτάνευον τὸν ἄχρι αἰῶνος συστήσαντα τὸν ἑαυτοῦ λαὸν, ἀεὶ δὲ μετʼ ἐπιφανείας ἀντιλαμβανόμενον τῆς ἑαυτοῦ μερίδος.
\VS{16}Προστάξαντος δὲ τοῦ ἡγουμένου, ἐκεῖθεν εὐθέως ἀνέζευξαν, καὶ συμμίσγουσιν αὐτοῖς ἐπὶ κώμην Δεσσαού.
\par }{\PP \VS{17}Σίμων δὲ ὁ ἀδελφὸς Ἰούδα συμβεβληκῶς ἦν τῷ Νικάνορι, βραχέως δὲ διὰ τὴν αἰφνίδιον τῶν ἀντιπάλων ἀφασίαν ἐπταικώς.
\VS{18}Ὅμως δὲ ἀκούων ὁ Νικάνωρ ἣν εἶχον οἱ περὶ τὸν Ἰούδαν ἀνδραγαθίαν, καὶ ἐν τοῖς ὑπὲρ τῆς πατρίδος ἀγῶσιν εὐψυχίαν, ἐπευλαβεῖτο τὴν κρίσιν διʼ αἱμάτων ποιήσασθαι·
\VS{19}Διόπερ ἔπεμψε Ποσιδώνιον καὶ Θεόδοτον καὶ Ματταθίαν, δοῦναι καὶ λαβεῖν δεξιάς.
\par }{\PP \VS{20}Πλείονος δὲ γενομένης περὶ τούτων ἐπισκέψεως, καὶ τοῦ ἡγεμόνος τοῖς πλήθεσιν ἀνακοινωσαμένου, καὶ φανείσης ὁμοψήφου γνώμης, ἐπένευσαν ταῖς συνθήκαις.
\VS{21}Ἐτάξαντο δὲ ἡμέραν ἐν ᾗ κατʼ ἰδίαν ἥξουσιν εἰς τὸ αὐτό· καὶ προῆλθε, καὶ παρʼ ἑκάστου διαφόρους ἔθεσαν δίφρους.
\VS{22}Διέταξεν Ἰούδας ἐνόπλους ἑτοίμους ἐν τοῖς ἐπικαίροις τόποις, μήποτε ἐκ τῶν πολεμίων αἰφνιδίως κακουργία γένηται· τὴν ἁρμόζουσαν ἐποιήσαντο κοινολογίαν.
\par }{\PP \VS{23}Διέτριβεν δὲ ὁ Νικάνωρ ἐν Ἱεροσολύμοις, καὶ ἔπραττεν οὐθὲν ἄτοπον· τοὺς δὲ συναχθέντας ἀγελαίους ὄχλους ἀπέλυσε.
\VS{24}Καὶ εἶχε τὸν Ἰούδαν διαπαντὸς ἐν προσώπῳ, ψυχικῶς τῷ ἀνδρὶ προσεκέκλιτο.
\VS{25}Παρεκάλεσεν αὐτὸν γῆμαι καὶ παιδοποιήσασθαι· ἐγάμησεν, εὐστάθησεν, ἐκοινώνησε βίου.
\par }{\PP \VS{26}Ὁ δὲ Ἄλκιμος συνιδὼν τὴν πρὸς ἀλλήλους εὔνοιαν καὶ τὰς γενομένας συνθήκας, ἀναλαβὼν, ἧκε πρὸς τὸν Δημήτριον, καὶ ἔλεγε τὸν Νικάνορα ἀλλότρια φρονεῖν τῶν πραγμάτων· τὸν γὰρ ἐπίβουλον τῆς βασιλείας Ἰούδαν διάδοχον ἀναδέδειχεν ἑαυτοῦ.
\VS{27}Ὁ δὲ βασιλεὺς ἔκθυμος γενόμενος, καὶ ταῖς τοῦ παμπονήρου ἐρεθισθεὶς διαβολαῖς, ἔγραψε Νικάνορι φάσκων, ὑπὲρ μὲν τῶν συνθηκῶν βαρέως φέρειν, κελεύων δὲ τὸν Μακκαβαῖον δέσμιον ἐξαποστέλλειν ταχέως εἰς Ἀντιόχειαν.
\par }{\PP \VS{28}Προσπεσόντων δὲ τούτων τῷ Νικάνορι, συνεκέχυτο καὶ δυσφόρως ἔφερεν, εἰ τὰ διεσταλμένα ἀθετήσει μηδὲν τʼ ἀνδρὸς ἠδικηκότος.
\VS{29}Ἐπεὶ δὲ τῷ βασιλεῖ ἀντιπράττειν οὐκ ἦν, εὔκαιρον ἐτήρει στρατηγήματι τοῦτʼ ἐπιτελέσαι.
\par }{\PP \VS{30}Ὁ δὲ Μακκαβαῖος αὐστηρότερον διεξάγοντα συνιδὼν τὸν Νικάνορα πρὸς αὐτὸν, καὶ τὴν εἰθισμένην ἀπάντησιν ἀγριωτέραν ἐσχηκότα, νοήσας οὐκ ἀπὸ τοῦ βελτίστου τὴν αὐστηρίαν εἶναι, συστρέψας οὐκ ὀλίγους τῶν περὶ ἑαυτὸν, συνεκρύπτετο τὸν Νικάνορα.
\VS{31}Συγγνοὺς δὲ ὁ ἕτερος ὅτι γενναίως ὑπὸ τοῦ ἀνδρὸς ἐστρατήγηται, παραγενόμενος ἐπὶ τὸ μέγιστον καὶ ἅγιον ἱερὸν, τῶν ἱερέων τὰς καθηκούσας θυσίας προσαγόντων, ἐκέλευσε παραδιδόναι τὸν ἄνδρα.
\VS{32}Τῶν δὲ μεθʼ ὅρκων φασκόντων μὴ γινώσκειν ποῦ ποτʼ ἐστὶν ὁ ζητούμενος,
\VS{33}προτείνας τὴν δεξιὰν εἰς τὸν νεὼν, ταῦτα ὤμοσεν, ἐὰν μὴ δέσμιόν μοι τὸν Ἰούδαν παραδῶτε, τόνδε τοῦ Θεοῦ σηκὸν εἰς πεδίον ποιήσω, καὶ τὸ θυσιαστήριον κατασκάφω, καὶ ἱερὸν ἐνταῦθα τῷ Διονύσῳ ἐπιφανὲς ἀναστήσω.
\par }{\PP \VS{34}Τοσαῦτα δὲ εἰπὼν ἀπῆλθεν· οἱ δὲ ἱερεῖς προτείναντες τὰς χεῖρας εἰς τὸν οὐρανὸν, ἐπεκαλοῦντο τὸν διαπαντὸς ὑπέρμαχον τοῦ ἔθνους ἡμῶν, ταῦτα λέγοντες,
\VS{35}σὺ, Κύριε, τῶν ὅλων ἀπροσδεὴς ὑπάρχων, εὐδόκησας ναὸν τῆς σῆς κατασκηνώσεως ἐν ἡμῖν γενέσθαι.
\VS{36}Καὶ νῦν, ἅγιε παντὸς ἁγιασμοῦ Κύριε διατήρησον εἰς αἰῶνα ἀμίαντον τόνδε τὸν προσφάτως κεκαθαρισμένον οἶκον.
\par }{\PP \VS{37}Ῥαζὶς δέ τις τῶν ἀπὸ Ἱεροσολύμων πρεσβυτέρων, ἐμηνύθη τῷ Νικάνορι, ἀνὴρ φιλοπολίτης καὶ σφόδρα καλῶς ἀκούων, καὶ κατὰ τὴν εὔνοιαν πατὴρ τῶν Ἰουδαίων προσαγορευόμενος.
\VS{38}Ἦν γὰρ ἐν τοῖς ἔμπροσθεν χρόνοις τῆς ἀμιξίας κρίσιν εἰσενηνεγμένος Ἰουδαϊσμοῦ, καὶ σῶμα καὶ ψυχὴν ὑπὲρ τοῦ Ἰουδαϊσμοῦ παραβεβλημένος μετὰ πάσης ἐκτενίας.
\par }{\PP \VS{39}Βουλόμενος δὲ Νικάνωρ πρόδηλον ποιῆσαι ἣν εἶχε πρὸς τοὺς Ἰουδαίους δυσμένειαν, ἀπέστειλε στρατιώτας ὑπὲρ τοὺς πεντακοσίους συλλαβεῖν αὐτόν.
\VS{40}Ἔδοξε γὰρ, ἐκεῖνον συλλαβὼν, τούτοις ἐργάσασθαι συμφοράν,
\VS{41}Τῶν δὲ πληθῶν μελλόντων τὸν πύργον καταλαβέσθαι, καὶ τὴν αὐλαίαν θύραν βιαζομένων, καὶ κελευόντων πῦρ προσάγειν καὶ τὰς θύρας ὑφάπτειν, περικατάληπτος γενόμενος ὑπέθηκεν ἑαυτῷ ξίφος,
\VS{42}εὐγενῶς θέλων ἀποθανεῖν, ἤπερ τοῖς ἀλιτηρίοις ὑποχείριος γενέσθαι, καὶ τῆς ἰδίας εὐγενείας ἀναξίως ὑβρισθῆναι.
\VS{43}Τῇ δὲ πληγῇ μὴ κατευθικτήσας διὰ τὴν τοῦ ἀγῶνος σπουδὴν, καὶ τῶν ὄχλων εἴσω τῶν θυρωμάτων εἰσβαλόντων, ἀναδραμὼν γενναίως ἐπὶ τὸ τεῖχος, κατεκρήμνισεν ἑαυτὸν ἀνδρείως εἰς τοὺς ὄχλους.
\VS{44}Τῶν δὲ ταχέως ἀναποδισάντων, γενομένου διαστήματος ἦλθε κατὰ μέσον τὸν κενεῶνα.
\par }{\PP \VS{45}Ἔτι δὲ ἔμπνους ὑπάρχων καὶ πεπυρωμένος τοῖς θυμοῖς, ἐξαναστὰς φερομένων κρουνηδὸν τῶν αἱμάτων, καὶ δυσχερῶν ὄντων τῶν τραυμάτων, δρόμῳ τοὺς ὄχλους διελθὼν, καὶ στὰς ἐπί τινος πέτρας ἀποῤῥωγάδος,
\VS{46}παντελῶς ἔξαιμος ἤδη γενόμενος, προβαλὼν τὰ ἔντερα, καὶ λαβὼν ἑκατέραις ταῖς χερσὶν, ἐνέσεισε τοῖς ὄχλοις· καὶ ἐπικαλεσάμενος τὸν δεσπόζοντα τῆς ζωῆς καὶ τοῦ πνεύματος, ταῦτα αὐτῷ πάλιν ἀποδοῦναι, τόνδε τὸν τρόπον μετήλλαξεν.

\par }\Chap{15}{\PP \VerseOne{1}Ὁ δὲ Νικάνωρ μεταλαβὼν τοὺς περὶ τὸν Ἰούδαν ὄντας ἐν τοῖς κατὰ Σαμάρειαν τόποις, ἐβουλεύσατο τῇ τῆς καταπαύσεως ἡμέρᾳ μετὰ πάσης ἀσφαλείας αὐτοῖς ἐπιβαλεῖν.
\par }{\PP \VS{2}Τῶν δὲ κατʼ ἀνάγκην συνεπομένων αὐτῷ Ἰουδαίων, λεγόντων, μηδαμῶς οὕτως ἀγρίως καὶ βαρβάρως ἀπολέσῃς, δόξαν δὲ ἀπομέρισον τῇ προτετιμημένῃ ὑπὸ τοῦ πάντα ἐθορῶντος μεθʼ ἁγιότητος ἡμέρᾳ.
\par }{\PP \VS{3}Ὁ δὲ τρισαλιτήριος ἐπηρώτησεν, εἰ ἔστιν ἐν οὐρανῷ δυνάστης ὁ προστεταχὼς ἄγειν τὴν τῶν σαββάτων ἡμέραν;
\VS{4}Τῶν δὲ ἀποφῃναμένων, ἔστιν ὁ Κύριος ζῶν αὐτὸς ἐν οὐρανῷ δυνάστης, ὁ κελεύσας ἀσκεῖν τὴν ἑβδομάδα.
\VS{5}Ὁ δὲ ἕτερος, κᾀγώ φησι, δυνάστης ἐπὶ τῆς γῆς ὁ προστάσσων αἴρειν ὅπλα, καὶ τὰς βασιλικὰς χρείας ἐπιτελεῖν· ὅμως οὐ κατέσχεν ἐπιτελέσαι τό σχέτλιον αὐτοῦ βούλημα.
\VS{6}Καὶ ὁ μὲν Νικάνωρ μετὰ πάσης ἀλαζονείας ὑψαυχενῶν, διεγνώκει κοινὸν τῶν περὶ τὸν Ἰούδαν συστήσασθαι τρόπαιον.
\par }{\PP \VS{7}Ὁ δὲ Μακκαβαῖος ἦν ἀδιαλείπτως πεποιθὼς μετὰ πάσης ἐλπίδος ἀντιλήψεως τεύξασθαι παρὰ τοῦ Κυρίου.
\VS{8}Καὶ παρεκάλει τοὺς σὺν αὐτῷ μὴ δειλιᾷν τὴν τῶν ἔθνῶν ἔφοδον, ἔχοντας δὲ κατὰ νοῦν τὰ προγεγονότα αὐτοῖς ἁπʼ οὐρανοῦ βοηθήματα, καὶ τανῦν προσδοκᾷν τὴν παρὰ τοῦ παντοκράτορος ἐσομένην αὐτοῖς νίκην καὶ βοήθειαν.
\VS{9}Καὶ παραμυθούμενος αὐτοὺς ἐκ τοῦ νόμου καὶ τῶν προφητῶν, προσυπομνήσας δὲ αὐτοὺς καὶ τοὺς ἀγῶνας οὓς ἦσαν ἐκτετελεκότες, προθυμοτέρους αὐτοὺς κατέστησε.
\par }{\PP \VS{10}Καὶ τοῖς θυμοῖς διεγείρας αὐτοὺς, παρήγγειλεν, ἅμα παρεπιδεικνὺς τὴν τῶν ἐθνῶν ἀθεσίαν καὶ τὴν τῶν ὅρκων παράβασιν.
\VS{11}Ἕκαστον δὲ αὐτῶν καθοπλίσας, οὐ τὴν ἀσπίδων καὶ λογχῶν ἀσφάλειαν, ὡς τὴν ἐν τοῖς ἀγαθοῖς λόγοις παράκλησιν, καὶ προσεξηγησάμενος ὄνειρον ἀξιόπιστον ὕπαρ τι πάντας εὔφρανεν.
\par }{\PP \VS{12}Ἦν δὲ ἡ τούτου θεωρία τοιάδε· Ὀνίαν τὸν γενόμενον ἀρχιερέα, ἄνδρα καλὸν καὶ ἀγαθὸν, αἰδήμονα μὲν τὴν ἀπάντησιν, πρᾷον δὲ τὸν τρόπον, καὶ λαλιὰν προϊέμενον πρεπόντως, καὶ ἐκ παιδὸς ἐκμεμελετηκότα πάντα τὰ τῆς ἀρετῆς οἰκεῖα, τοῦτον τὰς χεῖρας προτείναντα κατεύχεσθαι τῷ παντὶ τῶν Ἰουδαίων συστήματι.
\VS{13}Εἶθʼ οὕτως ἐπιφανῆναι ἄνδρα πολιᾷ καὶ δόξῃ διαφέροντα, θαυμαστὴν δέ τινα καὶ μεγαλοπρεπεστάτην εἶναι τὴν περὶ αὐτὸν ὑπεροχήν.
\VS{14}Ἀποκριθέντα δὲ τὸν Ὀνιαν εἰπεῖν, ὁ φιλάδελφος οὗτός ἐστιν ὁ πολλὰ προσευχόμενος περὶ τοῦ λαοῦ καὶ τῆς ἁγίας πόλεως, Ἱερεμίας ὁ τοῦ Θεοῦ προφήτης.
\VS{15}Προτείναντα δὲ τὸν Ἱερεμίαν τὴν δεξιὰν παραδοῦναι τῷ Ἰούδᾳ ῥομφαίαν χρυσῆν, διδόντα δὲ προσφωνῆσαι τάδε,
\VS{16}λάβε τὴν ἁγίαν ῥομφαίαν δῶρον παρὰ τοῦ Θεοῦ, διʼ ἧς θραύσεις τοὺς ὑπεναντίους.
\par }{\PP \VS{17}Παρακληθέντες δὲ τοῖς Ἰούδα λόγοις πάνυ καλοῖς καὶ δυναμένοις ἐπʼ ἀρετὴν παρορμῆσαι, καὶ ψυχὰς νέων ἐπανορθῶσαι, διέγνωσαν μὴ στρατοπεδεύεσθαι, γενναίως δὲ ἐμφέρεσθαι, καὶ μετὰ πάσης εὐανδρίας ἐμπλακέντες κρῖναι τὰ πράγματα, διὰ τὸ καὶ τὴν πόλιν, καὶ τὰ ἅγια, καὶ τὸ ἱερὸν κινδυνεύειν.
\VS{18}Ἦν γὰρ ὁ περὶ γυναικῶν καὶ τέκνων, ἔτι δὲ ἀδελφῶν καὶ συγγενῶν ἐν ἥττονι μέρει κείμενος αὐτοῖς ἀγὼν, μέγιστος δὲ καὶ πρῶτος ὁ περὶ τοῦ καθηγιασμένου ναοῦ φόβος.
\VS{19}Ἦν δὲ καὶ τοῖς ἐν τῇ πόλει κατειλημμένοις οὐ πάρεργος ἀγωνία ταρασσομένοις τῆς ἐν ὑπαίθρῳ προσβολῆς.
\par }{\PP \VS{20}Καὶ πάντων ἤδη προσδοκώντων τὴν ἐσομένην κρίσιν, καὶ ἤδη συμμιξάντων τῶν πολεμίων, καὶ τῆς στρατιᾶς ἐκταγείσης, καὶ τῶν θηρίων ἐπὶ μέρος εὔκαιρον ἀποκατασταθέντων, τῆς τε ἵππου κατὰ κέρας τεταγμένης,
\par }{\PP \VS{21}Συνιδὼν ὁ Μακκαβαῖος τὴν τῶν πληθῶν παρουσίαν, καὶ τῶν ὅπλων τὴν ποικίλην παρασκευὴν, τήν τε τῶν θηρίων ἀγριότητα, προτείνας τὰς χεῖρας εἰς τὸν οὐρανὸν, ἐπεκαλέσατο τὸν τερατοποιὸν Κύριον τὸν κατόπτην, γινώσκων ὅτι οὐκ ἔστι διʼ ὅπλων ἡ νίκη, καθὼς δὲ ἂν αὐτῷ κριθείη, τοῖς ἀξίοις περιποιεῖται τὴν νίκην.
\VS{22}Ἔλεγε δὲ ἐπικαλούμενος τόνδε τὸν τρόπον, σὺ, Δέσποτα, ἀπέστειλας τὸν ἄγγελόν σου ἐπὶ Ἑζεκίου τοῦ βασιλέως τῆς Ἰουδαίας, καὶ ἀνεῖλες ἐκ τῆς παρεμβολῆς Σενναχηρεὶμ εἰς ἑκατὸν ὀγδοηκονταπέντε χιλιάδας.
\VS{23}Καὶ νῦν, Δυνάστα τῶν οὐρανῶν, ἀπόστειλον ἄγγελον ἀγαθὸν ἔμπροσθεν ἡμῶν εἰς δέος καὶ τρόμον.
\VS{24}Μεγέθει βραχίονός σου καταπλαγείησαν οἱ μετὰ βλασφημίας παραγενόμενοι ἐπὶ τὸν ἅγιόν σου λαόν· καὶ οὗτος μὲν ἐν τούτοις ἔληξεν.
\par }{\PP \VS{25}Οἱ δὲ περὶ τὸν Νικάνορα μετὰ σαλπίγγων καὶ παιάνων προσῆγον,
\VS{26}οἱ δὲ περὶ τὸν Ἰούδαν μετʼ ἐπικλήσεως καὶ εὐχῶν συνέμιξαν τοῖς πολεμίοις.
\VS{27}Καὶ ταῖς μὲν χερσὶν ἀγωνιζόμενοι, ταῖς δὲ καρδίαις πρὸς τὸν Θεὸν εὐχόμενοι, κατέστρωσαν οὐδὲν ἧττον μυριάδων τριῶν καὶ πεντακισχιλίων, τῇ τοῦ Θεοῦ μεγάλως εὐφρανθέντες ἐπιφανείᾳ.
\par }{\PP \VS{28}Γενόμενοι δὲ ἀπὸ τῆς χρείας, καὶ μετὰ χαρᾶς ἀναλύοντες, ἐπέγνωσαν προπεπτωκότα Νικάνορα σὺν τῇ πανοπλίᾳ.
\VS{29}Γενομένης δὲ κραυγῆς καὶ ταραχῆς, εὐλόγουν τὸν Δυνάστην τῇ πατρίῳ φωνῇ.
\par }{\PP \VS{30}Καὶ προσέταξεν ὁ καθʼ ἅπαν σώματι καὶ ψυχῇ πρωταγωνιστὴς ὑπὲρ τῶν πολιτῶν, ὁ τὴν τῆς ἡλικίας εὔνοιαν εἰς ὁμοεθνεῖς διαφυλάξας, τὴν τοῦ Νικάνορος κεφαλὴν ἀποτεμόντας, καὶ τὴν χεῖρα σὺν τῷ ὤμῳ φέρειν εἰς Ἱεροσόλυμα.
\par }{\PP \VS{31}Παραγενόμενος δὲ ἐκεῖ, καὶ συγκαλέσας τοὺς ὁμοεθνεῖς, καὶ τοὺς ἱερεῖς πρὸ τοῦ θυσιαστηρίου στήσας, μετεπέμψατο τοὺς ἐκ τῆς ἄκρας.
\VS{32}Καὶ ἐπιδειξάμενος τὴν τοῦ μιαροῦ Νικάνορος κεφαλὴν, καὶ τὴν χεῖρα τοῦ δυσφήμου, ἣν ἐκτείνας ἐπὶ τὸν ἅγιον τοῦ παντοκράτορος οἶκον ἐμεγαλαύχησε.
\par }{\PP \VS{33}Καὶ τὴν γλῶσσαν τοῦ δυσσεβοῦς Νικάνορος ἐκτεμὼν, ἔφη κατὰ μέρος δώσειν τοῖς ὀρνέοις, τὰ δὲ ἐπίχειρα τῆς ἀνοίας κατέναντι τοῦ ναοῦ κρεμᾶσαι.
\VS{34}Οἱ δὲ πάντες εἰς τὸν οὐρανὸν εὐλόγησαν τὸν ἐπιφανῆ Κύριον, λέγοντες, εὐλογητὸς ὁ διατηρήσας τὸν ἑαυτοῦ τόπον ἀμίαντον.
\VS{35}Ἐξέδησε δὲ τὴν τοῦ Νικάνορος κεφαλὴν ἐκ τῆς ἄκρας, ἐπίδηλον πᾶσι καὶ φανερὸν τῆς τοῦ Κυρίου βοηθείας σημεῖον.
\par }{\PP \VS{36}Καὶ ἐδογμάτισαν πάντες μετὰ κοινοῦ ψηφίσματος μηδαμῶς ἐᾶσαι ἀπαρασήμαντον τὴνδε τὴν ἡμέραν· ἔχειν δὲ ἐπίσημον τὴν τρισκαιδεκάτην τοῦ δωδεκάτου μηνὸς, Ἄδαρ λέγεται τῇ Συριακῇ φωνῇ, πρὸ μιᾶς ἡμέρας τῆς Μαρδοχαϊκῆς ἡμέρας.
\VS{37}Τῶν οὖν κατὰ Νικάνορα χωρησάντων οὕτω, καὶ ἀπʼ ἐκείνων τῶν καιρῶν κρατηθείσης τῆς πόλεως ὑπὸ τῶν Ἑβραίων. Καὶ αὐτὸς αὐτόθι καταπαύσω τὸν λόγον.
\par }{\PP \VS{38}Καὶ εἰ μὲν καλῶς καὶ εὐθίκτως τῇ συντάξει, τοῦτο καὶ αὐτὸς ἤθελον· εἰ δὲ εὐτελῶς καὶ μετρίως, τοῦτο ἐφικτὸν ἦν μοι.
\VS{39}Καθάπερ γὰρ οἶνον καταμόνας πίνειν, ὡσαύτως δὲ καὶ ὕδωρ πάλιν, πολέμιον· ὃν δὲ τρόπον οἶνος ὕδατι συγκερασθεὶς ἡδὺς, καὶ ἐπιτερπῆ τὴν χάριν ἀποτελεῖ, οὕτω καὶ τὸ τῆς κατασκευῆς τοῦ λόγου τέρπει τὰς ἀκοὰς τῶν ἐντυγχανόντων τῇ συντάξει· ἐνταῦθα δὲ ἔσται ἡ τελευτή.
\par }