\NormalFont\ShortTitle{ΛΕΥΙΤΙΚΟΝ}
{\MT ΛΕΥΙΤΙΚΟΝ

\par }\ChapOne{1}{\PP \VerseOne{1}ΚΑΙ ἀνεκάλεσε Μωυσῆν, καὶ ἐλάλησε Κύριος αὐτῷ ἐκ τῆς σκηνῆς τοῦ μαρτυρίου, λέγων,
\VS{2}λάλησον τοῖς υἱοῖς Ἰσραὴλ, καὶ ἐρεῖς πρὸς αὐτοὺς, ἄνθρωπος ἐξ ὑμῶν ἐὰν προσαγάγῃ δῶρα τῷ Κυρίῳ, ἀπὸ τῶν κτηνῶν καὶ ἀπὸ τῶν βοῶν καὶ ἀπὸ τῶν προβάτων προσοίσετε τὰ δῶρα ὑμῶν.
\VS{3}Ἐὰν ὁλοκαύτωμα τὸ δῶρον αὐτοῦ, ἐκ τῶν βοῶν ἄρσεν ἄμωμον προσάξει πρὸς τὴν θύραν τῆς σκηνῆς τοῦ μαρτυρίου, προσοίσει αὐτὸ δεκτὸν ἐναντίον Κυρίου.
\VS{4}Καὶ ἐπιθήσει τὴν χεῖρα ἐπὶ τὴν κεφαλὴν τοῦ καρπώματος δεκτὸν αὐτῷ, ἐξιλάσασθαι περὶ αὐτοῦ.
\VS{5}Καὶ σφάξουσι τὸν μόσχον ἔναντι Κυρίου· καὶ προσοίσουσιν οἱ υἱοὶ Ἀαρὼν οἱ ἱερεῖς τὸ αἷμα, καὶ προσχεοῦσι τὸ αἷμα ἐπὶ τὸ θυσιαστήριον κύκλῳ τὸ ἐπὶ τῶν θυρῶν τῆς σκηνῆς τοῦ μαρτυρίου·
\VS{6}καὶ ἐκδείραντες τὸ ὁλοκαύτωμα, μελιοῦσιν αὐτὸ κατὰ μέλη.
\VS{7}Καὶ ἐπιθήσουσιν οἱ υἱοὶ Ἀαρὼν οἱ ἱερεῖς πῦρ ἐπὶ τὸ θυσιαστήριον, καὶ ἐπιστοιβάσουσι ξύλα ἐπὶ τὸ πῦρ.
\VS{8}Καὶ ἐπιστοιβάσουσιν οἱ υἱοὶ Ἀαρὼν οἱ ἱερεῖς τὰ διχοτομήματα, καὶ τὴν κεφαλὴν, καὶ τὸ στέαρ ἐπὶ τὰ ξύλα τὰ ἐπὶ τοῦ πυρὸς τὰ ὄντα ἐπὶ τοῦ θυσιαστηρίου.
\VS{9}Τὰ δὲ ἐγκοίλια καὶ τοὺς πόδας πλυνοῦσιν ὕδατι· καὶ ἐπιθήσουσιν οἱ ἱερεῖς τὰ πάντα ἐπὶ τὸ θυσιαστήριον· κάρπωμά ἐστι θυσία ὀσμὴ εὐωδίας τῷ Κυρίῳ.
\VS{10}Ἐὰν δὲ ἀπὸ τῶν προβάτων τὸ δῶρον αὐτοῦ τῷ Κυρίῳ, ἀπό τε τῶν ἀρνῶν, καὶ τῶν ἐρίφων εἰς ὁλοκαυτώματα, ἄρσεν ἄμωμον προσάξει αὐτό.
\VS{11}Καὶ ἐπιθήσει τὴν χεῖρα ἐπὶ τὴν κεφαλὴν αὐτοῦ· καὶ σφάξουσιν αὐτὸ ἐκ πλαγίων τοῦ θυσιαστηρίου πρὸς βοῤῥᾶν ἔναντι Κυρίου· καὶ προσχεοῦσιν οἱ υἱοὶ Ἀαρὼν οἱ ἱερεῖς τὸ αἷμα αὐτοῦ ἐπὶ τὸ θυσιαστήριον κύκλῳ.
\VS{12}Καὶ διελουσιν αὐτὸ κατὰ μέλη, καὶ τὴν κεφαλὴν, καὶ τὸ στέαρ· καὶ ἐπιστοιβάσουσιν οἱ ἱερεῖς αὐτὰ ἐπὶ τὰ ξύλα τὰ ἐπὶ τοῦ πυρὸς τὰ ἐπὶ τοῦ θυσιαστηρίου.
\VS{13}Καὶ τὰ ἐγκοίλια, καὶ τοὺς πόδας πλυνοῦσιν ὕδατι· καὶ προσοίσει ὁ ἱερεὺς τὰ πάντα, καὶ ἐπιθήσει ἐπὶ τὸ θυσιαστήριον· κάρπωμά ἐστι θυσία ὀσμὴ εὐωδίας τῷ Κυρίῳ.
\VS{14}Ἐὰν δὲ ἀπὸ τῶν πετεινῶν κάρπωμα προσφέρει δῶρον αὐτοῦ τῷ κυρίῳ, καὶ προσοίσει ἀπὸ τῶν τρυγόνων, ἢ ἀπὸ τῶν περιστερῶν τὸ δῶρον αὐτοῦ.
\VS{15}Καὶ προσοίσει αὐτὸ ὁ ἱερεὺς πρὸς τὸ θυσιαστήριον, καὶ ἀποκνίσει τὴν κεφαλήν, καὶ ἐπιθήσει ὁ ἱερεὺς ἐπὶ τὸ θυσιαστήριον, καὶ στραγγιεῖ τὸ αἷμα πρὸς τὴν βάσιν τοῦ θυσιαστηρίου.
\VS{16}Καὶ ἀφελεῖ τὸν πρόλοβον σὺν τοῖς πτεροῖς, καὶ ἐκβαλεῖ αὐτὸ παρὰ τὸ θυσιαστήριον κατʼ ἀνατολὰς εἰς τὸν τόπον τῆς σποδοῦ·
\VS{17}Καὶ ἐκκλάσει αὐτὸ ἐκ τῶν πτερύγων, καὶ οὐ διελεῖ, καὶ ἐπιθήσει αὐτὸ ὁ ἱερεὺς ἐπὶ τὸ θυσιαστήριον ἐπὶ τὰ ξύλα τὰ ἐπὶ τοῦ πυρός· κάρπωμά ἐστι θυσία ὀσμὴ εὐωδίας τῷ Κυρίῳ.

\par }\Chap{2}{\PP \VerseOne{1}Ἐὰν δὲ ψυχὴ προσφέρῃ δῶρον θυσίαν τῷ Κυρίῳ, σεμίδαλις ἔσται τὸ δῶρον αὐτοῦ, καὶ ἐπιχεεῖ ἐπʼ αὐτὸ ἔλαιον, καὶ ἐπιθήσει ἐπʼ αὐτὸ λίβανον· θυσία ἐστί.
\VS{2}Καὶ οἴσει πρὸς τοὺς υἱοὺς Ἀαρὼν τοὺς ἱερεῖς· καὶ δραξάμενος ἀπʼ αὐτῆς πλήρη τὴν δράκα ἀπὸ τῆς σεμιδάλεως σὺν τῷ ἐλαίῳ, καὶ πάντα τὸν λίβανον αὐτῆς, καὶ ἐπιθήσει ὁ ἱερεὺς τὸ μνημόσυνον αὐτῆς ἐπὶ τὸ θυσιαστήριον· θυσία ὀσμὴ εὐωδίας τῷ Κυρίῳ.
\VS{3}Καὶ τὸ λοιπὸν ἀπὸ τῆς θυσίας Ἀαρὼν καὶ τοῖς υἱοῖς αὐτοῦ, ἅγιον τῶν ἁγίων ἀπὸ τῶν θυσιῶν Κυρίου.
\VS{4}Ἐὰν δὲ προσφέρῃ δῶρον θυσίαν πεπεμμένην ἐκ κλιβάνου δῶρον Κυρίῳ ἐκ σεμιδάλεως, ἄρτους ἀζύμους πεφυραμένους ἐν ἐλαίῳ, καὶ λάγανα ἄζυμα διακεχρισμένα ἐν ἐλαίῳ.
\VS{5}Ἐὰν δὲ θυσία ἀπὸ τηγάνου τὸ δῶρόν σου, σεμίδαλις πεφυραμένη ἐν ἐλαίῳ ἄζυμά ἐστι.
\VS{6}Καὶ διαθρύψεις αὐτὰ κλάσματα, καὶ ἐπιχεεῖς ἐπʼ αὐτὰ ἔλαιον· θυσία ἐστὶ Κυρίῳ.
\VS{7}Ἐὰν δὲ θυσία ἀπὸ ἐσχάρας τὸ δῶρόν σου, σεμίδαλις ἐν ἐλαίῳ ποιηθήσεται.
\VS{8}Καὶ προσοίσει τὴν θυσίαν ἣν ἂν ποιήσῃ ἐκ τούτων τῷ Κυρίῳ, καὶ προσοίσει πρὸς τὸν ἱερέα.
\VS{9}Καὶ προσεγγίσας πρὸς τὸ θυσιαστήριον, ἀφελεῖ ὁ ἱερεὺς ἀπὸ τῆς θυσίας τὸ μνημόσυνον αὐτῆς, καὶ ἐπιθήσει ὁ ἱερεὺς ἐπὶ τὸ θυσιαστήριον, κάρπωμα· ὀσμὴ εὐωδίας Κυρίῳ.
\VS{10}Τὸ δὲ καταλειφθὲν ἀπὸ τῆς θυσίας, Ἀαρὼν καὶ τοῖς υἱοῖς αὐτοῦ, ἅγια τῶν ἁγίων ἀπὸ τῶν καρπωμάτων Κυρίου.
\VS{11}Πᾶσαν θυσίαν, ἣν ἂν προσφέρητε Κυρίῳ, οὐ ποιήσετε ζυμωτόν· πᾶσαν γὰρ ζύμην, καὶ πᾶν μέλι οὐ προσοίσετε ἀπʼ αὐτοῦ, καρπῶσαι Κυρίῳ δῶρον.
\VS{12}Ἀπαρχῆς προσοίσετε αὐτὰ Κυρίῳ, ἐπὶ δὲ τὸ θυσιαστήριον οὐκ ἀναβιβασθήσεται εἰς ὀσμὴν εὐωδίας Κυρίῳ.
\VS{13}Καὶ πᾶν δῶρον θυσίας ὑμῶν ἁλὶ ἁλισθήσεται· οὐ διαπαύσατε ἅλας διαθήκης Κυρίου ἀπὸ θυσιασμάτων ὑμῶν· ἐπὶ παντὸς δώρου ὑμῶν προσοίσετε Κυρίῳ τῷ Θεῷ ὑμῶν ἅλας.
\VS{14}Ἐὰν δὲ προσφέρῃς θυσίαν πρωτογεννημάτων τῷ Κυρίῳ, νέα πεφρυγμένα χίδρα ἐρικτὰ τῷ Κυρίῳ· καὶ προσοίσεις τὴν θυσίαν τῶν πρωτογεννημάτων.
\VS{15}Καὶ ἐπιχεεῖς ἐπʼ αὐτὴν ἔλαιον, καὶ ἐπιθήσεις ἐπʼ αὐτὴν λίβανον· θυσία ἐστί.
\VS{16}Καὶ ἀνοίσει ὁ ἱερεὺς τὸ μνημόσυνον αὐτῆς ἀπὸ τῶν χίδρων σὺν τῷ ἐλαίῳ, καὶ πάντα τὸν λίβανον αὐτῆς· κάρπωμά ἐστι Κυρίῳ.

\par }\Chap{3}{\PP \VerseOne{1}Ἐὰν δὲ θυσία σωτηρίου τὸ δῶρον αὐτοῦ τῷ Κυρίῳ, ἐὰν μὲν ἐκ τῶν βοῶν αὐτὸ προσαγάγῃ, ἐάν τω ἄρσεν, ἐάν τε θῆλυ, ἄμωμον προσάξει αὐτὸ ἔναντι Κυρίου.
\VS{2}Καὶ ἐπιθήσει τὰς χεῖρας ἐπὶ τὴν κεφαλὴν τοῦ δώρου, καὶ σφάξει αὐτὸ ἐναντίον Κυρίου παρὰ τὰς θύρας τῆς σκηνῆς τοῦ μαρτυρίου· καὶ προσχεοῦσιν οἱ υἱοὶ Ἀαρὼν οἱ ἱερεῖς τὸ αἷμα ἐπὶ τὸ θυσιαστήριον τῶν ὁλοκαυτωμάτων κύκλῳ.
\VS{3}Καὶ προσάξουσιν ἀπὸ τῆς θυσίας τοῦ σωτηρίου κάρπωμα Κυρίῳ, τὸ στέαρ τὸ κατακαλύπτον τὴν κοιλίαν, καὶ πᾶν τὸ στέαρ τὸ ἐπὶ τῆς κοιλίας.
\VS{4}Καὶ τοὺς δύο νεφροὺς, καὶ τὸ στέαρ τὸ ἐπʼ αὐτῶν, τὸ ἐπὶ τῶν μηρίων, καὶ τὸν λοβὸν τὸν ἐπὶ τοῦ ἥπατος σὺν τοῖς νεφροῖς περιελεῖ.
\VS{5}Καὶ ἀνοίσουσιν αὐτὰ οἱ υἱοὶ Ἀαρὼν οἱ ἱερεῖς ἐπὶ τὸ θυσιαστήριον ἐπὶ τὰ ὁλοκαυτώματα ἐπὶ τὰ ξύλα, τὰ ἐπὶ τοῦ πυρὸς ἐπὶ τοῦ θυσιαστηρίου· κάρπωμα ὀσμὴ εὐωδίας Κυρίῳ.
\VS{6}Ἐὰν δὲ ἀπὸ τῶν προβάτων τὸ δῶρον αὐτοῦ θυσία σωτηρίου τῷ Κυρίῳ, ἄρσεν ἢ θῆλυ, ἄμωμον προσοίσει αὐτό.
\VS{7}Ἐὰν ἄρνα προσαγάγῃ τὸ δῶρον αὐτοῦ, προσάξει αὐτὸ ἔναντι Κυρίου.
\VS{8}Καὶ ἐπιθήσει τὰς χεῖρας ἐπὶ τὴν κεφαλὴν τοῦ δώρου αὐτοῦ, καὶ σφάξει αὐτὸ παρὰ τὰς θύρας τῆς σκηνῆς τοῦ μαρτυρίου· καὶ προσχεοῦσιν οἱ υἱοὶ Ἀαρὼν οἱ ἱερεῖς τὸ αἷμα ἐπὶ τὸ θυσιαστήριον κύκλῳ.
\VS{9}Καὶ προσοίσει ἀπὸ τῆς θυσίας τοῦ σωτηρίου κάρπωμα τῷ Κυρίῷ· τὸ στέαρ καὶ τὴν ὀσφὺν ἄμωμον σὺν ταῖς ψόαις περιελεῖ αὐτό· καὶ πᾶν τὸ στέαρ τὸ κατακαλύπτον τὴν κοιλίαν, καὶ πᾶν τὸ στέαρ τὸ ἐπὶ τῆς κοιλίας.
\VS{10}Καὶ ἀμφοτέρους τοὺς νεφροὺς, καὶ τὸ στέαρ τὸ ἐπʼ αὐτῶν, τὸ ἐπὶ τῶν μηρίων, καὶ τὸν λοβὸν τὸν ἐπὶ τοῦ ἥπατος σὺν τοῖς νεφροῖς περιελὼν,
\VS{11}ἀνοίσει ὁ ἱερεὺς ἐπὶ τὸ θυσιαστήριον· ὀσμὴ εὐωδίας κάρπωμα Κυρίῳ.
\par }{\PP \VS{12}Ἐὰν δὲ ἀπὸ τῶν αἰγῶν τὸ δῶρον αὐτοῦ, καὶ προσάξει ἔναντι Κυρίου.
\VS{13}Καὶ ἐπιθήσει τὰς χεῖρας ἐπὶ τὴν κεφαλὴν αὐτοῦ, καὶ σφάξουσιν αὐτὸ ἔναντι Κυρίου παρὰ τὰς θύρας τῆς σκηνῆς τοῦ μαρτυρίου· καὶ προσχεοῦσιν οἱ υἱοὶ Ἀαρὼν οἱ ἱερεῖς τὸ αἷμα ἐπὶ τὸ θυσιαστήριον κύκλῳ.
\VS{14}Καὶ ἀνοίσει ἀπʼ αὐτοῦ κάρπωμα Κυρίῳ τὸ στέαρ τὸ κατακαλύπτον τὴν κοιλίαν, καὶ πᾶν τὸ στέαρ τὸ ἐπὶ τῆς κοιλίας.
\VS{15}Καὶ ἀμφοτέρους τοὺς νεφροὺς, καὶ πᾶν τὸ στέαρ τὸ ἐπʼ αὐτῶν, τὸ ἐπὶ τῶν μηρίων, καὶ τὸν λοβὸν τοῦ ἥπατος σὺν τοῖς νεφροῖς περιελεῖ,
\VS{16}καὶ ἀνοίσει ὁ ἱερεὺς ἐπὶ τὸ θυσιαστήριον· κάρπωμα ὀσμὴ εὐωδίας τῷ Κυρίῳ· πᾶν τὸ στέαρ τῷ Κυρίῳ.
\VS{17}Νόμιμον εἰς τὸν αἰῶνα εἰς τὰς γενεὰς ὑμῶν, ἐν πάσῃ κατοικίᾳ ὑμῶν· πᾶν στέαρ καὶ πᾶν αἷμα οὐκ ἔδεσθε.

\par }\Chap{4}{\PP \VerseOne{1}Καὶ ἐλάλησε Κύριος πρὸς Μωυσῆν, λέγων,
\VS{2}Λάλησον πρὸς τοὺς υἱοὺς Ἰσραὴλ, λέγων, ψυχὴ ἐὰν ἁμάρτῃ ἔναντι Κυρίου ἀκουσίως ἀπὸ πάντων τῶν προσταγμάτων Κυρίου, ὧν οὐ δεῖ ποιεῖν, καὶ ποιήσῃ ἕν τι ἀπʼ αὐτῶν·
\VS{3}Ἐὰν μὲν ὁ ἀρχιερεὺς ὁ κεχρισμένος ἁμάρτῃ τοῦ τὸν λαὸν ἁμαρτεῖν, καὶ προσάξει περὶ τῆς ἁμαρτίας αὐτοῦ, ἧς ἥμαρτε, μόσχον ἐκ βοῶν ἄμωμον τῷ Κυρίῳ περὶ τῆς ἁμαρτίας.
\VS{4}Καὶ προσάξει τὸν μόσχον παρὰ τὴν θύραν τῆς σκηνῆς τοῦ μαρτυρίου ἔναντι Κυρίου, καὶ ἐπιθήσει τὴν χεῖρα αὐτοῦ ἐπὶ τὴν κεφαλὴν τοῦ μόσχου ἔναντι Κυρίου, καὶ σφάξει τὸν μόσχον ἐνώπιον Κυρίου.
\VS{5}Καὶ λαβὼν ὁ ἱερεὺς ὁ χριστὸς ὁ τετελειωμένος τὰς χεῖρας ἀπὸ τοῦ αἵματος τοῦ μόσχου, καὶ εἰσοίσει αὐτὸ εἰς τὴν σκηνὴν τοῦ μαρτυρίου.
\VS{6}Καὶ βάψει ὁ ἱερεὺς τὸν δάκτυλον εἰς τὸ αἷμα, καὶ προσρανεῖ ἀπὸ τοῦ αἵματος ἑπτάκις ἔναντι Κυρίου, κατὰ τὸ καταπέτασμα τὸ ἅγιον.
\VS{7}Καὶ ἐπιθήσει ὁ ἱερεὺς ἀπὸ τοῦ αἵματος τοῦ μόσχου ἐπὶ τὰ κέρατα τοῦ θυσιαστηρίου τοῦ θυμιάματος τῆς συνθέσεως τοῦ ἐναντίον Κυρίου, ὅ ἐστιν ἐν τῇ σκηνῇ τοῦ μαρτυρίου· καὶ πᾶν τὸ αἷμα τοῦ μόσχου ἐκχεεῖ παρὰ τὴν βάσιν τοῦ θυσιαστηρίου τῶν ὁλοκαυτωμάτων, ὅ ἐστι παρὰ τὰς θύρας τῆς σκηνῆς τοῦ μαρτυρίου.
\VS{8}Καὶ πὰν τὸ στέαρ τοῦ μόσχου τοῦ τῆς ἁμαρτίας περιελεῖ ἀπʼ αὐτοῦ, τὸ στέαρ τὸ κατακαλύπτον τὰ ἐνδόσθια, καὶ πᾶν τὸ στέαρ τὸ ἐπὶ τῶν ἐνδοσθίων,
\VS{9}καὶ τοὺς δύο νεφροὺς, καὶ τὸ στέαρ τὸ ἐπʼ αὐτῶν, ὅ ἐστιν ἐπὶ τῶν μηρίων, καὶ τὸν λοβὸν τὸν ἐπὶ τοῦ ἥπατος σὺν τοῖς νεφροῖς περιελεῖ αὐτό,
\VS{10}ὃν τρόπον ἀφαιρεῖται αὐτὸ ἀπὸ τοῦ μόσχου τοῦ τῆς θυσίας τοῦ σωτηρίου, καὶ ἀνοίσει ὁ ἱερεὺς ἐπὶ τὸ θυσιαστήριον τῆς καρπώσεως.
\VS{11}Καὶ τὸ δέρμα τοῦ μόσχου, καὶ πᾶσαν αὐτοῦ τὴν σάρκα σὺν τῇ κεφαλῇ καὶ τοῖς ἀκρωτηρίοις καὶ τῇ κοιλίᾳ καὶ τῇ κόπρῳ·
\VS{12}καὶ ἐξοίσουσιν ὅλον τὸν μόσχον ἔξω τῆς παρεμβολῆς εἰς τόπον καθαρὸν, οὗ ἐκχεοῦσι τὴν σποδιὰν, καὶ κατακαύσουσιν αὐτὸν ἐπὶ ξύλων ἐν πυρί· ἐπὶ τῆς ἐκχύσεως τῆς σποδιᾶς καυθήσεται.
\par }{\PP \VS{13}Ἐὰν δὲ πᾶσα συναγωγὴ Ἰσραὴλ ἀγνοήσῃ ἀκουσίως, καὶ λάθῃ ῥῆμα ἐξ ὀφθαλμῶν τῆς συναγωγῆς, καὶ ποιήσωσι μίαν ἀπὸ πασῶν τῶν ἐντολῶν Κυρίου, ἣ οὐ ποιηθήσεται, καὶ πλήμμελήσωσι,
\VS{14}καὶ γνωσθῇ αὐτοῖς ἡ ἁμαρτία, ἣν ἥμαρτον ἐν αὐτῇ, καὶ προσάξει ἡ συναγωγὴ μόσχον ἐκ βοῶν ἄμωμον περὶ τῆς ἁμαρτίας, καὶ προσάξει αὐτὸν παρὰ τὰς θύρας τῆς σκηνῆς τοῦ μαρτυρίου.
\VS{15}Καὶ ἐπιθήσουσιν οἱ πρεσβύτεροι τῆς συναγωγῆς τὰς χείρας αὐτῶν ἐπὶ τὴν κεφαλὴν τοῦ μόσχου ἔναντι Κυρίου, καὶ σφάξουσιν τὸν μόσχον ἔναντι Κυρίου.
\VS{16}Καὶ εἰσοίσει ὁ ἱερεὺς ὁ χριστὸς ἀπὸ τοῦ αἵματος τοὺ μόσχου εἰς τὴν σκηνὴν τοῦ μαρτυρίου.
\VS{17}Καὶ βάψει ὁ ἱερεὺς τὸν δάκτυλον ἀπὸ τοῦ αἵματος τοῦ μόσχου, καὶ ῥανεῖ ἑπτάκις ἔναντι Κυρίου, κατενώπιον τοῦ καταπετάσματος τοῦ ἁγίου.
\VS{18}Καὶ ἀπὸ τοῦ αἵματος ἐπιθήσει ὁ ἱερεὺς ἐπὶ τὰ κέρατα τοῦ θυσιαστηρίου τῶν θυμιαμάτων τῆς συνθέσεως, ὅ ἐστιν ἐνώπιον Κυρίου, ὅ ἐστιν ἐν τῇ σκηνῇ τοῦ μαρτυρίου· καὶ τὸ πᾶν αἷμα ἐκχεεῖ πρὸς τὴν βάσιν τοῦ θυσιαστηρίου τῶν καρπώσεων, τοῦ πρὸς τῇ θύρᾳ τῆς σκηνῆς τοῦ μαρτυρίου.
\VS{19}Καὶ τὸ πᾶν στέαρ περιελεῖ ἀπʼ αὐτοῦ, καὶ ἀνοίσει ἐπὶ τὸ θυσιαστήριον.
\VS{20}Καὶ ποιήσει τὸν μόσχον, ὃν τρόπον ἐποίησε τὸν μόσχον τὸν τῆς ἁμαρτίας, οὕτω ποιηθήσεται· καὶ ἐξιλάσεται περὶ αὐτῶν ὁ ἱερεὺς, καὶ ἀφεθήσεται αὐτοῖς ἡ ἁμαρτία.
\VS{21}Καὶ ἐξοίσουσι τὸν μόσχον ὅλον ἔξω τῆς παρεμβολῆς, καὶ κατακαύσουσι τὸν μόσχον, ὃν τρόπον κατέκαυσαν τὸν μόσχον τὸν πρότερον· ἁμαρτία συναγωγῆς ἐστιν.
\par }{\PP \VS{22}Ἐὰν δὲ ὁ ἄρχων ἁμάρτῃ, καὶ ποιήσῃ μίαν ἀπὸ πασῶν τῶν ἐντολῶν Κυρίου τοῦ Θεοῦ αὐτοῦ, ἣ οὐ ποιηθήσεται, ἀκουσίως, καὶ ἁμάρτῃ καὶ πλημμελήσῃ,
\VS{23}καὶ γνωσθῇ αὐτῷ ἡ ἁμαρτία, ἣν ἥμαρτεν ἐν αὐτῇ, καὶ προσοίσει τὸ δῶρον αὐτοῦ χίμαρον ἐξ αἰγῶν, ἄρσεν ἄμωμον.
\VS{24}Καὶ ἐπιθήσει τὴν χεῖρα ἐπὶ τὴν κεφαλὴν τοῦ χιμάρου· καὶ σφάξουσιν αὐτὸν ἐν τόπῳ οὗ σφάζουσι τὰ ὁλοκαυτώματα ἐνώπιον Κυρίου· ἁμαρτία ἐστί.
\VS{25}Καὶ ἐπιθήσει ὁ ἱερεὺς ἀπὸ τοῦ αἵματος τοῦ τῆς ἁμαρτίας τῷ δακτύλῳ ἐπὶ τὰ κέρατα τοῦ θυσιαστηρίου τῶν ὁλοκαυτωμάτων· καὶ τὸ πᾶν αἷμα αὐτοῦ ἐκχεεῖ παρὰ τὴν βάσιν τοῦ θυσιαστηρίου τῶν ὁλοκαυτωμάτων.
\VS{26}Καὶ τὸ πᾶν στέαρ αὐτοῦ ἀνοίσει ἐπὶ τὸ θυσιαστήριον, ὥσπερ τὸ στέαρ θυσίας σωτηρίου· καὶ ἐξιλάσεται περὶ αὐτοῦ ὁ ἱερεὺς ἀπὸ τῆς ἁμαρτίας αὐτοῦ, καὶ ἀφεθήσεται αὐτῷ.
\par }{\PP \VS{27}Ἐὰν δὲ ψυχὴ μία ἁμάρτῃ ἀκουσίως ἐκ τοῦ λαοῦ τῆς γῆς, ἐν τῷ ποιῆσαι μίαν ἀπὸ πασῶν τῶν ἐντολῶν Κυρίου, ἣ οὐ ποιηθήσεται, καὶ πλημμελήσῃ·
\VS{28}καὶ γνωσθῇ αὐτῷ ἡ ἁμαρτία, ἣν ἥμαρτεν ἐν αὐτῇ, καὶ οἴσει χίμαιραν ἐξ αἰγῶν, θήλειαν ἄμωμον οἴσει περὶ τῆς ἁμαρτίας, ἧς ἥμαρτε.
\VS{29}Καὶ ἐπιθήσει τὴν χεῖρα ἐπὶ τὴν κεφαλὴν τοῦ ἁμαρτήματος αὐτοῦ· καὶ σφάξουσιν τὴν χίμαιραν τὴν τῆς ἁμαρτίας ἐν τῷ τόπῳ, οὗ σφάζουσι τὰ ὁλοκαυτώματα.
\VS{30}Καὶ λήψεται ὁ ἱερεὺς ἀπὸ τοῦ αἵματος αὐτῆς τῷ δακτύλῳ, καὶ ἐπιθήσει ἐπὶ τὰ κέρατα τοῦ θυσιαστηρίου τῶν ὁλοκαυτωμάτων· καὶ πᾶν τὸ αἷμα αὐτῆς ἐκχεεῖ παρὰ τὴν βάσιν τοῦ θυσιαστηρίου.
\VS{31}Καὶ πᾶν τὸ στέαρ περιελεῖ, ὃν τρόπον περιαιρεῖται στέαρ ἀπὸ θυσίας σωτηρίου· καὶ ἀνοίσει ὁ ἱερεὺς ἐπὶ τὸ θυσιαστήριον εἰς ὀσμὴν εὐωδίας Κυρίῳ· καὶ ἐξιλάσεται περὶ αὐτοῦ ὁ ἱερεὺς, καὶ ἀφεθήσεται αὐτῷ.
\par }{\PP \VS{32}Εὰν δὲ πρόβατον προσενέγκῃ τὸ δῶρον αὐτοῦ περὶ τῆς ἁμαρτίας, θῆλυ ἄμωμον προσοίσει αὐτό.
\VS{33}Καὶ ἐπιθήσει τὴν χεῖρα ἐπὶ τῆν κεφαλὴν τοῦ τῆς ἁμαρτίας· καὶ σφάξουσιν αὐτὸ ἐν τόπῳ, οὗ σφάζουσι τὰ ὁλοκαυτώματα.
\VS{34}Καὶ λαβὼν ὁ ἱερεὺς ἀπὸ τοῦ αἵματος τοῦ τῆς ἁμαρτίας τῷ δακτύλῳ, ἐπιθήσει ἐπὶ τὰ κέρατα τοῦ θυσιαστηρίου τῆς ὁλοκαρπώσεως· καὶ πᾶν αὐτοῦ τὸ αἷμα ἐκχεεῖ παρὰ τὴν βάσιν τοῦ θυσιαστηρίου τῆς ὁλοκαυτώσεως.
\VS{35}Καὶ πᾶν αὐτοῦ τὸ στέαρ περιελεῖ, ὃν τρόπον περιαιρεῖται στέαρ προβάτου ἐκ τῆς θυσίας τοῦ σωτηρίου· καὶ ἐπιθήσει αὐτὸ ὁ ἱερεὺς ἐπὶ τὸ θυσιαστήριον ἐπὶ τὸ ὁλοκαύτωμα Κυρίου· καὶ ἐξιλάσεται περὶ αὐτοῦ ὁ ἱερεὺς περὶ τῆς ἁμαρτίας ἧς ἥμαρτε, καὶ ἀφεθήσεται αὐτῷ.

\par }\Chap{5}{\PP \VerseOne{1}Ἐὰν δὲ ψυχὴ ἁμάρτῃ, καὶ ἀκούσῃ φωνὴν ὁρκισμοῦ, καὶ οὗτος μάρτυς ἢ ἑώρακεν ἢ σύνοιδεν, ἐὰν μὴ ἀπαγγείλῃ, λήψεται τὴν ἁμαρτίαν.
\VS{2}Ἡ ψυχὴ ἐκείνη ἥτις ἐὰν ἅψηται παντὸς πράγματος ἀκαθάρτου, ἢ θνησιμαίου, ἢ θηριαλώτου ἀκαθάρτου, ἢ τῶν θνησιμαίων βδελυγμάτων τῶν ἀκαθάρτων, ἢ τῶν θνησιμαίων κτηνῶν τῶν ἀκαθάρτων,
\VS{3}ἢ ἅψηται ἀπὸ ἀκαθαρσίας ἀνθρώπου, ἀπὸ πάσης ἀκαθαρσίας αὐτοῦ, ἧς ἂν ἁψάμενος μιανθῇ καὶ ἔλαθεν αὐτόν, μετὰ τοῦτο δὲ γνῷ, καὶ πλημμελήσῃ.
\VS{4}Ἡ ψυχὴ ἡ ἄνομος, ἡ διαστέλλουσα τοῖς χείλεσι κακοποιῆσαι ἢ καλῶς ποιῆσαι κατὰ πάντα ὅσα ἐὰν διαστείλῃ ὁ ἄνθρωπος μεθʼ ὅρκου, καὶ λάθῃ αὐτὸν πρὸ ὀφθαλμῶν, καὶ οὗτος γνῷ, καὶ ἁμάρτῃ ἕν τι τούτων.
\VS{5}Καὶ ἐξαγορεύσει τὴν ἁμαρτίαν περὶ ὧν ἡμάρτηκε κατʼ αὐτῆς.
\VS{6}Καὶ οἴσει περὶ ὧν ἐπλημμέλησε Κυρίῳ, περὶ τῆς ἁμαρτίας ἧς ἥμαρτε, θῆλυ ἀπὸ τῶν προβάτων ἀμνάδα, ἢ χίμαιραν ἐξ αἰγῶν, περὶ ἁμαρτίας· καὶ ἐξιλάσεται περὶ αὐτοῦ ὁ ἱερεὺς περὶ τῆς ἁμαρτίας αὐτοῦ, ἧς ἥμαρτε, καὶ ἀφεθήσεται αὐτῷ ἡ ἁμαρτία.
\VS{7}Ἐὰν δὲ μὴ ἰσχύῃ ἡ χεὶρ αὐτοῦ τὸ ἱκανὸν εἰς τὸ πρόβατον, οἴσει περὶ τῆς ἁμαρτίας αὐτοῦ, ἧς ἥμαρτε, δύο τρυγόνας, ἢ δύο νοσσοὺς περιστερῶν Κυρίῳ, ἕνα περὶ ἁμαρτίας, καὶ ἕνα εἰς ὁλοκαύτωμα.
\VS{8}Καὶ οἴσει αὐτὰ πρὸς τὸν ἱερέα· καὶ προσάξει ὁ ἱερεὺς τὸ περὶ τῆς ἁμαρτίας πρότερον· καὶ ἀποκνίσει ὁ ἱερεὺς τὴν κεφαλὴν αὐτοῦ ἀπὸ τοῦ σφονδύλου, καὶ οὐ διελεῖ.
\VS{9}Καὶ ῥανεῖ ἀπὸ τοῦ αἵματος τοῦ περὶ τῆς ἁμαρτίας ἐπὶ τὸν τοῖχον τοῦ θυσιαστηρίου· τὸ δὲ κατάλοιπον τοῦ αἵματος καταστραγγιεῖ ἐπὶ τὴν βάσιν τοῦ θυσιαστηρίου· ἁμαρτία γάρ ἐστι·
\VS{10}Καὶ τὸ δεύτερον ποιήσει ὁλοκάρπωμα, ὡς καθήκει· καὶ ἐξιλάσεται ὁ ἱερεὺς περὶ τῆς ἁμαρτίας αὐτοῦ, ἧς ἥμαρτε, καὶ ἀφεθήσεται αὐτῷ.
\par }{\PP \VS{11}Ἐὰν δὲ μὴ εὑρίσκῃ ἡ χεὶρ αὐτοῦ ζεῦγος τρυγόνων, ἢ δύο νοσσούς περιστερῶν, καὶ οἴσει τὸ δῶρον αὐτοῦ, περὶ οὗ ἥμαρτε, τὸ δέκατον τοῦ οἰφὶ σεμιδάλεως περὶ ἁμαρτίας· οὐκ ἐπιχεεῖ ἐπʼ αὐτὸ ἔλαιον, οὐδὲ ἐπιθήσει ἐπʼ αὐτῷ λίβανον, ὅτι περὶ ἁμαρτίας ἐστί.
\VS{12}Καὶ οἴσει αὐτὸ πρὸς τὸν ἱερέα· καὶ δραξάμενος ὁ ἱερεὺς ἀπʼ αὐτῆς πλήρη τὴν δράκα, τὸ μνημόσυνον αὐτῆς ἐπιθήσει ἐπὶ τὸ θυσιαστήριον τῶν ὁλοκαυτωμάτων Κυρίῳ· ἁμαρτία ἐστί.
\VS{13}Καὶ ἐξιλάσεται περὶ αὐτοῦ ὁ ἱερεὺς περὶ τῆς ἁμαρτίας αὐτοῦ, ἧς ἥμαρτεν ἀφʼ ἑνὸς τούτων, καὶ ἀφεθήσεται αὐτῷ. τὸ δὲ καταλειφθὲν ἔσται τῷ ἱερεῖ, ὡς θυσία τῆς σεμιδάλεως.
\par }{\PP \VS{14}Καὶ ἐλάλησε Κύριος πρὸς Μωυσῆν, λέγων,
\VS{15}ψυχὴ ἣ ἂν λάθῃ αὐτὸν λήθῃ, καὶ ἁμάρτῃ ἀκουσίως ἀπὸ τῶν ἁγίων Κυρίου, καὶ οἴσει τῆς πλημμελείας αὐτοῦ τῷ Κυρίῳ κριὸν ἄμωμον ἐκ τῶν προβάτων, τιμῆς ἀργυρίου σίκλων, τῷ σίκλῳ τῶν ἁγίων, περὶ οὗ ἐπλημμέλησε.
\VS{16}Καὶ ὃ ἥμαρτεν ἀπὸ τῶν ἁγίων ἀποτίσει αὐτὸ, καὶ τὸ ἐπίπεμπτον προσθήσει ἐπʼ αὐτὸ, καὶ δώσει αὐτὸ τῷ ἱερεῖ· καὶ ὁ ἱερεὺς ἐξιλάσεται περὶ αὐτοῦ ἐν τῷ κριῷ τῆς πλημμελείας, καὶ ἀφεθήσεται αὐτῷ.
\VS{17}Καὶ ἡ ψυχὴ ἣ ἂν ἁμάρτῃ, καὶ ποιήσει μίαν ἀπὸ πασῶν τῶν ἐντολῶν Κυρίου, ὧν οὐ δεῖ ποιεῖν, καὶ οὐκ ἔγνω, καὶ πλημμελήσῃ, καὶ λάβῃ τὴν ἁμαρτίαν,
\VS{18}καὶ οἴσει κριὸν ἄμωμον ἐκ τῶν προβάτων, τιμῆς ἀργυρίου εἰς πλημμέλειαν πρὸς τὸν ἱερέα· καὶ ἐξιλάσεται περὶ αὐτοῦ ὁ ἱερεὺς περὶ τῆς ἀγνοίας αὐτοῦ, ἧς ἠγνόησε, καὶ αὐτὸς οὐκ ᾔδει, καὶ ἀφεθήσεται αὐτῷ.
\VS{19}Ἐπλημμέλησε γὰρ πλημμελείᾳ ἔναντι Κυρίου.
\par }{\PP \VS{20}Καὶ ἐλάλησε Κύριος πρὸς Μωυσῆν, λέγων,
\VS{21}ψυχὴ ἣ ἂν ἁμάρτῃ, καὶ παριδὼν παρίδῃ τὰς ἐντολὰς Κυρίου, καὶ ψεύσηται τὰ πρὸς τὸν πλησίον ἐν παραθήκῃ, ἢ περὶ κοινωνίας, ἢ περὶ ἁρπαγῆς, ἢ ἠδίκησέ τι τὸν πλησίον,
\VS{22}ἢ εὗρεν ἀπωλίαν, καὶ ψεύσηται περὶ αὐτῆς, καὶ ὀμόσῃ ἀδίκως περὶ ἑνὸς ἀπὸ πάντων, ὧν ἐὰν ποιήσῃ ὁ ἄνθρωπος, ὥστε ἁμαρτεῖν ἐν τούτοις·
\VS{23}Καὶ ἔσται ἡνίκα ἐὰν ἁμάρτῃ, καὶ πλημμελήσῃ, καὶ ἀποδῷ τὸ ἅρπαγμα, ὃ ἥρπασεν, ἢ τὸ ἀδίκημα, ὃ ἠδίκησεν, ἢ τὴν παραθήκην, ἥτις παρετέθη αὐτῷ, ἢ τὴν ἀπώλειαν, ἣν εὗρεν
\VS{24}ἀπὸ παντὸς πράγματος, οὗ ὤμοσε περὶ αὐτοῦ ἀδίκως, καὶ ἀποτίσει αὐτὸ τὸ κεφάλαιον, καὶ τὸ ἐπίπεμπτον προσθήσει ἐπʼ αὐτὸ, τίνος ἐστίν, αὐτῷ ἀποδώσει ᾗ ἡμέρᾳ ἐλεγχθῇ.
\VS{25}Καὶ τῆς πλημμελείας αὐτου οἴσει τῷ Κυρίῳ κριὸν ἀπὸ τῶν προβάτων ἄμωμον, τιμῆς, εἰς ὃ ἐπλημμέλησε.
\VS{26}Καὶ ἐξιλάσεται περὶ αὐτοῦ ὁ ἱερεὺς ἔναντι Κυρίου, καὶ ἀφεθήσεται αὐτῷ περὶ ἑνὸς ἀπὸ πάντων ὧν ἐποίησε καὶ ἐπλημμέλησεν αὐτῷ.

\par }\Chap{6}{\PP \VerseOne{1}Καὶ ἐλάλησε Κύριος πρὸς Μωυσῆν, λέγων,
\VS{2}ἔντειλαι τῷ Ἀαρὼν καὶ τοῖς υἱοῖς αὐτοῦ, λέγων, οὗτος ὁ νόμος τῆς ὁλοκαυτώσεως· αὕτη ἡ ὁλοκαύτωσις ἐπὶ τῆς καύσεως αὐτῆς ἐπὶ τοῦ θυσιαστηρίου ὅλην τὴν νύκτα ἕως τοπρωῒ, καὶ τὸ πῦρ τοῦ θυσιαστηρίου καυθήσεται ἐπʼ αὐτοῦ, οὐ σβεσθήσεται.
\VS{3}Καὶ ἐνδύσεται ὁ ἱερεὺς χιτῶνα λινοῦν, καὶ περισκελὲς λινοῦν ἐνδύσεται περὶ τὸ σῶμα αὐτοῦ, καὶ ἀφελεῖ τὴν κατακάρπωσιν, ἣν ἂν καταναλώσῃ τὸ πῦρ, τὴν ὁλοκαύτωσιν ἀπὸ τοῦ θυσιαστηρίου· καὶ παραθήσει αὐτὸ ἐχόμενον τοῦ θυσιαστηρίου.
\VS{4}Καὶ ἐκδύσεται τὴν στολὴν αὐτοῦ, καὶ ἐνδύσεται στολὴν ἄλλην· καὶ ἐξοίσει τὴν κατακάρπωσιν ἔξω τῆς παρεμβολῆς εἰς τόπον καθαρόν.
\VS{5}Καὶ πῦρ ἐπὶ τὸ θυσιαστήριον καυθήσεται ἀπʼ αὐτοῦ, καὶ οὐ σβεσθήσεται· καὶ καύσει ἐπʼ αὐτοῦ ὁ ἱερεὺς ξύλα τοπρωῒ πρωῒ, καὶ στοιβάσει ἐπʼ αὐτοῦ τὴν ὁλοκαύτωσιν, καὶ ἐπιθήσει ἐπʼ αὐτὸ τὸ στέαρ τοῦ σωτηρίου.
\VS{6}Καὶ πῦρ διαπαντὸς καυθήσεται ἐπὶ τὸ θυσιαστήριον, οὐ σβεσθήσεται.
\VS{7}Οὗτος ὁ νόμος τῆς θυσίας, ἣν προσάξουσιν αὐτὴν οἱ υἱοὶ Ἀαρὼν ἔναντι Κυρίου, ἀπέναντι τοῦ θυσιαστηρίου.
\VS{8}Καὶ ἀφελεῖ ἀπʼ αὐτοῦ τῇ δρακὶ ἀπὸ τῆς σεμιδάλεως τῆς θυσίας σὺν τῷ ἐλαίῳ αὐτῆς, καὶ σὺν παντὶ τῷ λιβάνῳ αὐτῆς, τὰ ὄντα ἐπὶ τῆς θυσίας· καὶ ἀνοίσει ἐπὶ τὸ θυσιαστήριον κάρπωμα ὀσμὴν εὐωδίας, τὸ μνημόσυνον αὐτῆς τῷ Κυρίῳ.
\VS{9}Τὸ δὲ καταλειφθὲν ἀπʼ αὐτῆς ἔδεται Ἀαρὼν καὶ οἱ υἱοὶ αὐτοῦ· ἄζυμα βρωθήσεται ἐν τόπῳ ἁγίῳ· ἐν αὐλῇ τῆς σκηνῆς τοῦ μαρτυρίου ἔδονται αὐτήν.
\VS{10}Οὐ πεφθήσεται ἐζυμωμένη· μερίδα αὐτὴν ἔδωκα αὐτοῖς ἀπὸ τῶν καρπωμάτων Κυρίου· ἅγια ἁγίων ἐστὶν, ὥσπερ τὸ τῆς ἁμαρτίας, καὶ ὥσπερ τὸ τῆς πλημμελείας.
\VS{11}Πᾶν ἀρσενικὸν τῶν ἱερέων ἔδονται αὐτήν· νόμιμον αἰώνιον εἰς τὰς γενεὰς ὑμῶν ἀπὸ τῶν καρπωμάτων Κυρίου· πᾶς ὃς ἐὰν ἅψηται αὐτῶν, ἁγιασθήσεται.
\par }{\PP \VS{12}Καὶ ἐλάλησε Κύριος πρὸς Μωυσῆν, λέγων,
\VS{13}τοῦτο τὸ δῶρον Ἀαρὼν καὶ τῶν υἱῶν αὐτοῦ, ὃ προσοίσουσι Κυρίῳ ἐν τῇ ἡμέρᾳ, ᾗ ἂν χρίσῃς αὐτόν· τὸ δέκατον τοῦ οἰφὶ σεμιδάλεως εἰς θυσίαν διαπαντὸς, τὸ ἥμισυ αὐτῆς τὸπρωῒ, καὶ τὸ ἥμισυ αὐτῆς τοδειλινόν.
\VS{14}Ἐπὶ τηγάνου ἐν ἐλαίῳ ποιηθήσεται, πεφυραμένην οἴσει αὐτήν ἑλικτά, θυσίαν ἐκ κλασμάτων, θυσίαν εἰς ὀσμὴν εὐωδίας Κυρίῳ.
\VS{15}Ὁ ἱερεὺς ὁ χριστὸς ὁ ἀντʼ αὐτοῦ ἐκ τῶν υἱῶν αὐτοῦ ποιήσει αὐτήν· νόμος αἰώνιος· ἅπαν ἐπιτελεσθήσεται.
\VS{16}Καὶ πᾶσα θυσία ἱερέως ὁλόκαυτος ἔσται, καὶ οὐ βρωθήσεται.
\VS{17}Καὶ ἐλάλησε Κύριος πρὸς Μωυσῆν, λέγων,
\VS{18}λάλησον τῷ Ἀαρὼν καὶ τοῖς υἱοῖς αὐτοῦ, λέγων, οὗτος ὁ νόμος τῆς ἁμαρτίας· ἐν τόπῳ οὗ σφάζουσι τὸ ὁλοκαύτωμα, σφάξουσι τὰ περὶ τῆς ἁμαρτίας ἔναντι Κυρίου· ἅγια ἁγίων ἐστίν.
\VS{19}Ὁ ἱερεὺς ὁ ἀναφέρων αὐτὴν, ἔδεται αὐτήν· ἐν τόπῳ ἁγίῳ βρωθήσεται, ἐν αὐλῇ τῆς σκηνῆς τοῦ μαρτυρίου.
\VS{20}Πᾶς ὁ ἁπτόμενος τῶν κρεῶν αὐτῆς, ἁγιασθήσεται· καὶ ᾧ ἐὰν ἐπιῤῥαντισθῇ ἀπὸ τοῦ αἵματος αὐτῆς ἐπὶ τὸ ἱμάτιον, ὃς ἐὰν ῥαντισθῇ ἐπʼ αὐτὸ, πλυθήσεται ἐν τόπῳ ἁγίῳ.
\VS{21}Καὶ σκεῦος ὀστράκινον, οὗ ἐὰν ἑψεθῇ ἐν αὐτῷ, συντριβήσεται· ἐὰν δὲ ἐν σκεύει χαλκῷ ἑψηθῇ, ἐκτρίψει αὐτὸ, καὶ ἐκκλύσει ὕδατι.
\VS{22}Πᾶς ἄρσην ἐν τοῖς ἱερεῦσιν φάγεται αὐτά ἅγια ἁγίων ἐστὶ Κυρίῳ.
\VS{23}Καὶ πάντα τὰ περὶ τῆς ἁμαρτίας, ὧν ἐὰν εἰσενεχθῇ ἀπὸ τοῦ αἵματος αὐτῶν εἰς τὴν σκηνὴν τοῦ μαρτυρίου ἐξιλάσασθαι ἐν τῷ ἁγίῳ, οὐ βρωθήσεται· ἐν πυρὶ κατακαυθήσεται.

\par }\Chap{7}{\PP \VerseOne{1}Καὶ οὗτος ὁ νόμος τοῦ κριοῦ τοῦ περὶ τῆς πλημμελείας· ἅγια ἁγίων ἐστίν.
\VS{2}Ἐν τόπῳ οὗ σφάζουσι τὸ ὁλοκαύτωμα, σφάξουσι τὸν κριὸν τῆς πλημμελείας ἔναντι Κυρίου· καὶ τὸ αἷμα προσχεεῖ ἐπὶ τὴν βάσιν τοῦ θυσιαστηρίου κύκλῳ·
\VS{3}Καὶ πᾶν τὸ στέαρ αὐτοῦ προσοίσει ἀπʼ αὐτοῦ, καὶ τὴν ὀσφὺν, καὶ πᾶν τὸ στέαρ τὸ κατακαλύπτον τὰ ἐνδόσθια, καὶ πᾶν τὸ στέαρ τὸ ἐπὶ τῶν ἐνδοσθίων,
\VS{4}καὶ τοὺς δύο νεφροὺς, καὶ τὸ στέαρ τὸ ἐπʼ αὐτῶν, τὸ ἐπὶ τῶν μηρίων, καὶ τὸν λοβὸν τὸν ἐπὶ τοῦ ἥπατος σὺν τοῖς νεφροῖς, περιελεῖ αὐτά.
\VS{5}Καὶ ἀνοίσει αὐτὰ ὁ ἱερεὺς ἐπὶ τὸ θυσιαστήριον κάρπωμα τῷ Κυρίῳ· περὶ πλημμελείας ἐστί.
\VS{6}Πᾶς ἄρσην ἐκ τῶν ἱερέων ἔδεται αὐτά· ἐν τόπῳ ἁγίῳ ἔδονται αὐτά ἅγια ἁγίων ἐστίν.
\VS{7}Ὥσπερ τὸ περὶ τῆς ἁμαρτίας, οὕτω καὶ τὸ τῆς πλημμελείας· νόμος εἷς αὐτῶν· ὁ ἱερεὺς ὅστις ἐξιλάσεται ἐν αὐτῷ, αὐτῷ ἔσται.
\VS{8}Καὶ ὁ ἱερεὺς ὁ προσάγων ὁλοκαύτωμα ἀνθρώπου, τὸ δέρμα τῆς ὁλοκαυτώσεως, ἧς προσφέρει αὐτὸς, αὐτῷ ἔσται.
\VS{9}Καὶ πᾶσα θυσία ἥτις ποιηθήσεται ἐν τῷ κλιβάνῳ, καὶ πᾶσα ἥτις ποιηθήσεται ἐπʼ ἐσχάρας, ἢ ἐπὶ τηγάνου, τοῦ ἱερέως τοῦ προσφέροντος αὐτὴν, αὐτῷ ἔσται.
\VS{10}Καὶ πᾶσα θυσία ἀναπεποιημένη ἐν ἐλαίῳ, καὶ μὴ ἀναπεποιημένη, πᾶσι τοῖς υἱοῖς Ἀαρὼν ἔσται, ἑκάστῳ τὸ ἶσον.
\par }{\PP \VS{11}Οὗτος ὁ νόμος θυσίας σωτηρίου, ἣν προσοίσουσι Κυρίῳ.
\VS{12}Ἐὰν μὲν περὶ αἰνέσεως προσφέρῃ αὐτήν, καὶ προσοίσει ἐπὶ τῆς θυσίας τῆς αἰνέσεως ἄρτους ἐκ σεμιδάλεως ἀναπεποιημένους ἐν ἐλαίῳ, καὶ λάγανα ἄζυμα διακεχρισμένα ἐν ἐλαίῳ, καὶ σεμίδαλιν πεφυραμένην ἐν ἐλαίῳ.
\VS{13}Ἐπʼ ἄρτοις ζυμίταις προσοίσει τὰ δῶρα αὐτοῦ ἐπὶ θυσίᾳ αἰνέσεως σωτηρίου.
\VS{14}Καὶ προσάξει ἓν ἀπὸ πάντων τῶν δώρων αὐτοῦ, ἀφαίρεμα Κυρίῳ· τῷ ἱερεῖ τῷ προσχέοντι τὸ αἷμα τοῦ σωτηρίου, αὐτῷ ἔσται.
\VS{15}Καὶ τὰ κρέα θυαίας αἰνέσεως σωτηρίου αὐτῷ ἔσται· καὶ ἐν ᾗ ἡμέρᾳ δωρεῖται, βρωθήσεται· οὐ καταλείψουσιν ἀπʼ αὐτοῦ εἰς τὸ πρωΐ.
\VS{16}Καὶ ἐὰν εὐχὴ ᾖ, ἢ ἑκούσιον θυσιάζῃ τὸ δῶρον αὐτοῦ, ᾗ ἂν ἡμέρᾳ προσαγάγῃ τὴν θυσίαν αὐτοῦ, βρωθήσεται, καὶ τῇ αὔριον.
\VS{17}Καὶ τὸ καταλειφθὲν ἀπὸ τῶν κρεῶν τῆς θυσίας ἕως ἡμέρας τρίτης, ἐν πυρὶ κατακαυθήσεται.
\VS{18}Ἐὰν δὲ φαγὼν φάγῃ ἀπὸ τῶν κρεῶν τῇ ἡμέρᾳ τῇ τρίτῃ, οὐ δεχθήσεται αὐτῷ τῷ προσφέροντι αὐτό· οὐ λογισθήσεται αὐτῷ, μίασμά ἐστιν· ἡ δὲ ψυχὴ ἥτις ἐὰν φάγῃ ἀπʼ αὐτοῦ, τὴν ἁμαρτίαν λήψεται.
\VS{19}Καὶ κρέα ὅσα ἐὰν ἅψηται παντὸς ἀκαθάρτου, οὐ βρωθήσεται, ἐν πυρὶ κατακαυθήσεται· πᾶς καθαρὸς φάγεται κρέα.
\VS{20}Ἡ δὲ ψυχὴ ἥτις ἐὰν φάγῃ ἀπὸ τῶν κρεῶν τῆς θυσίας τοῦ σωτηρίου, ὅ ἐστι Κυρίου, καὶ ἡ ἀκαθαρσία αὐτοῦ ἐπʼ αὐτῷ, ἀπολεῖται ἡ ψυχὴ ἐκείνη ἐκ τοῦ λαοῦ αὐτῆς.
\VS{21}Καὶ ἡ ψυχὴ ἣ ἂν ἅψηται παντὸς πράγματος ἀκαθάρτου, ἢ ἀπὸ ἀκαθαρσίας ἀνθρώπου, ἢ τῶν τετραπόδων τῶν ἀκαθάρτω ἢ παντὸς βδελύγματος ἀκαθάρτου, καὶ φάγῃ ἀπὸ τῶν κρεῶν τῆς θυσίας τοῦ σωτηρίου, ὅ ἐστι Κυρίου, ἀπολεῖται ἡ ψυχὴ ἐκείνη ἐκ τοῦ λαοῦ αὐτῆς.
\par }{\PP \VS{22}Καὶ ἐλάλησε Κύριος πρὸς Μωυσῆν, λέγων,
\VS{23}λάλησον τοῖς υἱοῖς Ἰσραὴλ, λέγων, πᾶν στέαρ βοῶν, καὶ προβάτων, καὶ αἰγῶν οὐκ ἔδεσθε.
\VS{24}Καὶ στέαρ θνησιμαίων καὶ θηριαλώτων ποιηθήσεται εἰς πᾶν ἔργον, καὶ εἰς βρῶσιν οὐ βρωθήσεται.
\VS{25}Πᾶς ὁ ἔσθων στέαρ ἀπὸ τῶν κτηνῶν, ὧν προσάξει ἀπʼ αὐτῶν κάρπωμα Κυρίῳ, ἀπολεῖται ἡ ψυχὴ ἐκείνη ἀπὸ τοῦ λαοῦ αὐτῆς.
\VS{26}Πᾶν αἷμα οὐκ ἔδεσθε ἐν πάσῃ τῇ κατοικίᾳ ὑμῶν, ἀπό τε τῶν κτηνῶν καὶ ἀπὸ τῶν πετεινῶν.
\VS{27}Πᾶσα ψυχὴ ἣ ἂν φάγῃ αἷμα, ἀπολεῖται ἡ ψυχὴ ἐκείνη ἀπὸ τοῦ λαοῦ αὐτῆς.
\par }{\PP \VS{28}Καὶ ἐλάλησε Κύριος πρὸς Μωυσῆν, λέγων,
\VS{29}καὶ τοῖς υἱοῖς Ἰσραὴλ λαλήσεις, λέγων, ὁ προσφέρων θυσίαν σωτηρίου, οἴσει τὸ δῶρον αὐτοῦ Κυρίῳ καὶ ἀπὸ τῆς θυσίας τοῦ σωτηρίου.
\VS{30}Αἱ χεῖρες αὐτοῦ προσοίσουσι τὰ καρπώματα Κυρίῳ· τὸ στέαρ τὸ ἐπὶ τοῦ στηθυνίου, καὶ τὸν λοβὸν τοῦ ἥπατος προσοίσει αὐτὰ, ὥστε ἐπιτιθέναι δόμα ἔναντι Κυρίου.
\VS{31}Καὶ ἀνοίσει ὁ ἱερεὺς τὸ στέαρ ἐπὶ τοῦ θυσιαστηρίου· καὶ ἔσται τὸ στηθύνιον Ἀαρὼν καὶ τοῖς υἱοῖς αὐτοῦ.
\VS{32}Καὶ τὸν βραχίονα τὸν δεξιὸν δώσετε ἀφαίρεμα τῷ ἱερεῖ ἀπὸ τῶν θυσιῶν τοῦ σωτηρίου ὑμῶν.
\VS{33}Ὁ προσφέρων τὸ αἷμα τοῦ σωτηρίου, καὶ τὸ στέαρ τὸ ἀπὸ τῶν υἱῶν Ἀαρὼν, αὐτῷ ἔσται ὁ βραχίων ὁ δεξιὸς ἐν μερίδι.
\VS{34}Τὸ γὰρ στηθύνιον τοῦ ἐπιθέματος καὶ τὸν βραχίονα τοῦ ἀφαιρέματος εἴληφα παρὰ τῶν υἱῶν Ἰσραὴλ ἀπὸ τῶν θυσιῶν τοῦ σωτηρίου ὑμῶν, καὶ ἔδωκα αὐτὰ Ἀαρὼν τῷ ἱερεῖ καὶ τοῖς υἱοῖς αὐτοῦ, νόμιμον αἰώνιον παρὰ τῶν υἱῶν Ἰσραήλ.
\VS{35}Αὕτη ἡ χρίσις Ἀαρὼν, καὶ ἡ χρίσις τῶν υἱῶν αὐτοῦ ἀπὸ τῶν καρπωμάτων Κυρίου, ἐν ᾗ ἡμέρᾳ προσηγάγετο αὐτοὺς τοῦ ἱερατεύειν τῷ Κυρίῳ,
\VS{36}καθὰ ἐνετείλατο Κύριος δοῦναι αὐτοῖς ᾗ ἡμέρᾳ ἔχρισεν αὐτοὺς παρὰ τῶν υἱῶν Ἰσραὴλ, νόμιμον αἰώνιον εἰς τὰς γενεὰς αὐτῶν.
\VS{37}Οὗτος ὁ νόμος τῶν ὁλοκαυτωμάτων, καὶ θυσίας, καὶ περὶ ἁμαρτίας, καὶ τῆς πλημμελείας καὶ τῆς τελειώσεως, καὶ τῆς θυσίας τοῦ σωτηρίου,
\VS{38}ὃν τρόπον ἐνετείλατο Κύριος τῷ Μωυσῇ ἐν τῷ ὄρει Σινᾷ, ᾗ ἡμέρᾳ ἐνετείλατο τοῖς υἱοῖς Ἰσραὴλ προσφέρειν τὰ δῶρα αὐτῶν ἔναντι Κυρίου ἐν τῇ ἐρήμῳ Σινᾷ.

\par }\Chap{8}{\PP \VerseOne{1}Καὶ ἐλάλησε Κύριος πρὸς Μωυσῆν, λέγων,
\VS{2}λάβε Ἀαρὼν καὶ τοὺς υἱοὺς αὐτοῦ, καὶ τὰς στολὰς αὐτοῦ, καὶ τὸ ἔλαιον τῆς χρίσεως, καὶ τὸν μόσχον τὸν περὶ τῆς ἁμαρτίας, καὶ τοὺς δύο κριοὺς, καὶ τὸ κανοῦν τῶν ἀζύμων,
\VS{3}καὶ πᾶσαν τὴν συναγωγὴν ἐκκλησίασον ἐπὶ τὴν θύραν τῆς σκηνῆς τοῦ μαρτυρίου.
\VS{4}Καὶ ἐποίησε Μωυσῆς ὃν τρόπον συνέταξεν αὐτῷ Κύριος· καὶ ἐξεκκλησίασε τὴν συναγωγὴν ἐπὶ τὴν θύραν τῆς σκηνῆς τοῦ μαρτυρίου.
\VS{5}Καὶ εἶπε Μωυσῆς τῇ συναγωγῇ, τοῦτό ἐστι τὸ ῥῆμα, ὃ ἐνετείλατο Κύριος ποιῆσαι.
\VS{6}Καὶ προσήνεγκε Μωυσῆς τὸν Ἀαρὼν, καὶ τοὺς υἱοὺς αὐτοῦ, καὶ ἔλουσεν αὐτοὺς ὕδατι.
\VS{7}Καὶ ἐνέδυσεν αὐτὸν τὸν χιτῶνα, καὶ ἔζωσεν αὐτὸν τὴν ζώνην, καὶ ἐνέδυσεν αὐτὸν τὸν ὑποδύτην, καὶ ἐπέθηκεν ἐπʼ αὐτὸν τὴν ἐπωμίδα.
\VS{8}Καὶ συνέζωσεν αὐτὸν κατὰ τὴν ποίησιν τῆς ἐπωμίδος, καὶ συνέσφιγξεν αὐτὸν ἐν αὐτῇ· καὶ ἐπέθηκεν ἐπʼ αὐτὴν τὸ λογεῖον, καὶ ἐπέθηκεν ἐπὶ τὸ λογεῖον τὴν δήλωσιν καὶ τὴν ἀλήθειαν.
\VS{9}Καὶ ἐπέθηκε τὴν μίτραν ἐπὶ τὴν κεφαλὴν αὐτοῦ, καὶ ἐπέθηκεν ἐπὶ τὴν μίτραν κατὰ πρόσωπον αὐτοῦ τὸ πέταλον τὸ χρυσοῦν τὸ καθηγιασμένον ἅγιον, ὃν τρόπον συνέταξε Κύριος τῷ Μωυσῇ.
\par }{\PP \VS{10}Καὶ ἔλαβε Μωυσῆς ἀπὸ τοῦ ἐλαίου τῆς χρίσεως,
\VS{11}καὶ ἔῤῥανεν ἀπʼ αὐτοῦ ἐπὶ τὸ θυσιαστήριον ἑπτάκις· καὶ ἔχρισε τὸ θυσιαστήριον, καὶ ἡγίασεν αὐτὸ, καὶ πάντα τὰ ἐν αὐτῷ, καὶ τὸν λουτῆρα, καὶ τὴν βάσιν αὐτοῦ, καὶ ἡγίασεν αὐτά· καὶ ἔχρισε τὴν σκηνὴν, καὶ πάντα τὰ σκεύη αὐτῆς, καὶ ἡγίασεν αὐτήν.
\VS{12}Καὶ ἐπέχεε Μωυσῆς ἀπὸ τοῦ ἐλαίου τῆς χρίσεως ἐπὶ τὴν κεφαλὴν Ἀαρών· καὶ ἔχρισεν αὐτὸν, καὶ ἡγίασεν αὐτόν.
\VS{13}Καὶ προσήγαγε Μωυσῆς τοὺς υἱοὺς Ἀαρὼν, καὶ ἐνέδυσεν αὐτοὺς χιτῶνας, καὶ ἔζωσεν αὐτοὺς ζώνας, καὶ περιέθηκεν αὐτοῖς κιδάρεις, καθάπερ συνέταξε Κύριος τῷ Μωυσῇ.
\par }{\PP \VS{14}Καὶ προσήγαγε Μωυσῆς τὸν μόσχον τὸν περὶ τῆς ἁμαρτίας· καὶ ἐπέθηκεν Ἀαρὼν καὶ οἱ υἱοὶ αὐτοῦ τὰς χεῖρας ἐπὶ τὴν κεφαλὴν τοῦ μόσχου τοῦ τῆς ἁμαρτίας.
\VS{15}Καὶ ἔσφαξεν αὐτόν· καὶ ἔλαβε Μωυσῆς ἀπὸ τοῦ αἵματος, καὶ ἐπέθηκεν ἐπὶ τὰ κέρατα τοῦ θυσιαστηρίου κύκλῳ τῷ δακτύλῳ, καὶ ἐκαθάρισε τὸ θυσιαστήριον· καὶ τὸ αἷμα ἐξέχεεν ἐπὶ τὴν βάσιν τοῦ θυσιαστηρίου, καὶ ἡγίασεν αὐτὸ, τοῦ ἐξιλάσασθαι ἐπʼ αὐτοῦ.
\VS{16}Καὶ ἔλαβε Μωυσῆς πᾶν τὸ στέαρ τὸ ἐπὶ τῶν ἐνδοσθίων, καὶ τὸν λοβὸν τὸν ἐπὶ τοῦ ἥπατος, καὶ ἀμφοτέρους τοὺς νεφροὺς, καὶ τὸ στέαρ τὸ ἐπʼ αὐτῶν, καὶ ἀνήνεγκε Μωυσῆς ἐπὶ τὸ θυσιαστήριον.
\VS{17}Καὶ τὸν μόσχον, καὶ τὴν βύρσαν αὐτοῦ, καὶ τὰ κρέα αὐτοῦ, καὶ τὴν κόπρον αὐτοῦ, κατέκαυσεν αὐτὰ πυρὶ ἔξω τῆς παρεμβολῆς, ὃν τρόπον συνέταξε Κύριος τῷ Μωυσῇ.
\VS{18}Καὶ προσήγαγε Μωυσῆς τὸν κριὸν τὸν εἰς ὁλοκαύτωμα· καὶ ἐπέθηκεν Ἀαρὼν καὶ υἱοὶ αὐτοῦ τὰς χεῖρας αὐτῶν ἐπὶ τὴν κεφαλὴν τοῦ κριοῦ. Καὶ ἔσφαξε Μωυσῆς τὸν κριόν· καὶ προσέχεε Μωυσῆς τὸ αἷμα ἐπὶ τὸ θυσιαστήριον κύκλῳ.
\VS{19}Καὶ τὸν κριὸν ἐκρεανόμησε κατὰ μέλη· καὶ ἀνήνεγκε Μωυσῆς τὴν κεφαλὴν, καὶ τὰ μέλη, καὶ τὸ στέαρ· καὶ τὴν κοιλίαν, καὶ τοὺς πόδας ἔπλυνεν ὕδατι.
\VS{20}Καὶ ἀνήνεγκε Μωυσῆς ὅλον τὸν κριὸν ἐπὶ τὸ θυσιαστήριον· ὁλοκαύτωμά ἐστιν εἰς ὀσμὴν εὐωδίας· κάρπωμά ἐστι τῷ Κυρίῳ, καθάπερ ἐνετείλατο Κύριος τῷ Μωυσῇ.
\par }{\PP \VS{21}Καὶ προσήγαγε Μωυσῆς τὸν κριὸν τὸν δεύτερον, κριὸν τελειώσεως· καὶ ἐπέθηκεν Ἀαρὼν καὶ οἱ υἱοὶ αὐτοῦ τὰς χεῖρας αὐτῶν ἐπὶ τὴν κεφαλὴν τοῦ κριοῦ.
\VS{22}Καὶ ἔσφαξεν αὐτόν· καὶ ἔλαβε Μωυσῆς ἀπὸ τοῦ αἵματος αὐτοῦ, καὶ ἐπέθηκεν ἐπὶ τὸν λοβὸν τοῦ ὠτὸς Ἀαρὼν τοῦ δεξιοῦ, καὶ ἐπὶ τὸ ἄκρον τῆς χειρὸς τῆς δεξιᾶς, καὶ ἐπὶ τὸ ἄκρον τοῦ ποδὸς τοῦ δεξιοῦ.
\VS{23}Καὶ προσήγαγε Μωυσῆς τοὺς υἱοὺς Ἀαρών· καὶ ἐπέθηκε Μωυσῆς ἀπὸ τοῦ αἵματος ἐπὶ τοὺς λοβοὺς τῶν ὤτων τῶν δεξιῶν, καὶ ἐπὶ τὰ ἄκρα τῶν χειρῶν αὐτῶν τῶν δεξιῶν· καὶ ἐπὶ τὰ ἄκρα τῶν ποδῶν αὐτῶν τῶν δεξιῶν· καὶ προσέχεε Μωυσῆς τὸ αἷμα ἐπὶ τὸ θυσιαστήριον κύκλῳ.
\VS{24}Καὶ ἔλαβε τὸ στέαρ, καὶ τὴν ὀσφὺν, καὶ τὸ στέαρ τὸ ἐπὶ τῆς κοιλίας, καὶ τὸν λοβὸν τοῦ ἥπατος, καὶ τοὺς δύο νεφροὺς, καὶ τὸ στέαρ τὸ ἐπʼ αὐτῶν, καὶ τὸν βραχίονα τὸν δεξιόν.
\VS{25}Καὶ ἀπὸ τοῦ κανοῦ τῆς τελειώσεως, τοῦ ὄντος ἔναντι Κυρίου, καὶ ἔλαβεν ἄρτον ἕνα ἄζυμον, καὶ ἄρτον ἐξ ἐλαίου ἕνα, καὶ λάγανον ἓν, καὶ ἐπέθηκεν ἐπὶ τὸ στέαρ, καὶ τὸν βραχίονα τὸν δεξιόν.
\VS{26}Καὶ ἐπέθηκεν ἅπαντα ἐπὶ τὰς χεῖρας Ἀαρὼν, καὶ ἐπὶ τὰς χεῖρας τῶν υἱῶν αὐτοῦ, καὶ ἀνήνεγκεν αὐτὰ ἀφαίρεμα ἔναντι Κυρίου.
\VS{27}Καὶ ἔλαβε Μωυσῆς ἀπὸ τῶν χειρῶν αὐτῶν, καὶ ἀνήνεγκεν αὐτὰ Μωυσῆς ἐπὶ τὸ θυσιαστήριον, ἐπὶ τὸ ὁλοκαύτωμα τῆς τελειώσεως, ὅ ἐστι ὀσμὴ εὐωδίας· κάρπωμά ἐστιν τῷ Κυρίῳ.
\VS{28}Καὶ λαβὼν Μωυσῆς τὸ στηθύνιον, ἀφεῖλεν αὐτὸ ἐπίθεμα ἔναντι Κυρίου, ἀπὸ τοῦ κριοῦ τῆς τελειώσεως· καὶ ἐγένετο Μωυσῇ ἐν μεριδι, καθὰ ἐνετείλατο Κύριος τῷ Μωυσῇ.
\par }{\PP \VS{29}Καὶ ἔλαβε Μωυσῆς ἀπὸ τοῦ ἐλαίου τῆς χρίσεως, καὶ ἀπὸ τοῦ αἵματος τοῦ ἐπὶ τοῦ θυσιαστηρίου, καὶ προσέῤῥνεν ἐπὶ Ἀαρὼν, καὶ τὰς στολὰς αὐτοῦ, καὶ τοὺς υἱοὺς αὐτοῦ, καὶ τὰς στολὰς τῶν υἱῶν αὐτοῦ μετʼ αὐτοῦ.
\VS{30}Καὶ ἡγίασεν Ἀαρὼν, καὶ τὰς στολάς αὐτοῦ, καὶ τοὺς υἱοὺς αὐτοῦ, καὶ τὰς στολὰς τῶν υἱῶν αὐτοῦ μετʼ αὐτοῦ.
\VS{31}Καὶ εἶπε Μωυσῆς πρὸς Ἀαρὼν, καὶ τοὺς υἱοὺς αὐτοῦ, ἑψήσατε τὰ κρέα ἐν τῇ αὐλῇ τῆς σκηνῆς τοῦ μαρτυρίου ἐν τόπῳ ἁγίῳ· καὶ ἐκεῖ φάγεσθε αὐτὰ, καὶ τοὺς ἄρτους τοὺς ἐν τῷ κανῷ τῆς τελειώσεως, ὃν τρόπον συντέτακταί μοι, λέγων, Ἀαρὼν καὶ οἱ υἱοὶ αὐτοῦ φάγονται αὐτά.
\VS{32}Καὶ τὸ καταλειφθὲν τῶν κρεῶν καὶ τῶν ἄρτων ἐν πυρὶ κατακαύσατε.
\VS{33}Καὶ ἀπὸ τῆς θύρας τῆς σκηνῆς τοῦ μαρτυρίου οὐκ ἐξελεύσεσθε ἑπτὰ ἡμέρας, ἕως ἡμέρα πληρωθῇ, ἡμέρα τελειώσεως ὑμῶν· ἑπτὰ γὰρ ἡμέρας τελειώσει τὰς χεῖρας ὑμῶν.
\VS{34}Καθάπερ ἐποίησεν ἐν τῇ ἡμέρᾳ ταύτῃ, ᾗ ἐνετείλατο Κύριος τοῦ ποιῆσαι, ὥστε ἐξιλάσασθαι περὶ ὑμῶν.
\VS{35}Καὶ ἐπὶ τὴν θύραν τῆς σκηνῆς τοῦ μαρτυρίου καθήσεσθε ἑπτὰ ἡμέρας, ἡμέραν καὶ νύκτα· φυλάξεσθε τὰ φυλάγματα Κυρίου, ἵνα μὴ ἀποθάνητε· οὕτω γὰρ ἐνετείλατό μοι Κύριος ὁ Θεός.
\VS{36}Καὶ ἐποίησεν Ἀαρὼν καὶ οἱ υἱοὶ αὐτοῦ πάντας τοὺς λόγους, οὓς συνέταξε Κύριος τῷ Μωυσῇ.

\par }\Chap{9}{\PP \VerseOne{1}Καὶ ἐγενήθη τῇ ἡμέρᾳ τῇ ὀγδόῃ, ἐκάλεσε Μωυσῆς Ἀαρὼν, καὶ τοὺς υἱοὺς αὐτοῦ, καὶ τὴν γερουσίαν Ἰσραὴλ,
\VS{2}καὶ εἶπε Μωυσῆς πρὸς Ἀαρών, λάβε σεαυτῷ μοσχάριον ἐκ βοῶν περὶ ἁμαρτίας, καὶ κριὸν εἰς ὁλοκαύτωμα, ἄμωμα, καὶ προσένεγκε αὐτὰ ἔναντι Κυρίου.
\VS{3}Καὶ τῇ γερουσίᾳ Ἰσραὴλ λάλησον, λέγων, λάβετε χίμαρον ἐξ αἰγῶν ἕνα περὶ ἁμαρτίας, καὶ μοσχάριον, καὶ ἀμνὸν ἐνιαύσιον εἰς ὁλοκάρπωσιν, ἄμωμα,
\VS{4}καὶ μόσχον, καὶ κριὸν εἰς θυσίαν σωτηρίου ἔναντι Κυρίου, καὶ σεμίδαλιν πεφυραμένην ἐν ἐλαίῳ· ὅτι σήμερον Κύριος ὀφθήσεται ἐν ὑμῖν.
\VS{5}Καὶ ἔλαβον καθὸ ἐνετείλατο Μωυσῆς ἀπέναντι τῆς σκηνῆς τοῦ μαρτυρίου· καὶ προσῆλθε πᾶσα συναγωγὴ, καὶ ἔστησαν ἔναντι Κυρίου.
\VS{6}Καὶ εἶπε Μωυσῆς, τοῦτο τὸ ῥῆμα, ὃ εἶπε Κύριος, ποιήσατε, καὶ ὀφθήσεται ἐν ὑμῖν ἡ δόξα Κυρίου.
\VS{7}Καὶ εἶπε Μωυσῆς τῷ Ἀαρὼν, πρόσελθε πρὸς τὸ θυσιαστήριον, καὶ ποίησον τὸ περὶ τῆς ἁμαρτίας σου, καὶ τὸ ὁλοκαύτωμά σου, καὶ ἐξίλασαι περὶ σεαυτοῦ, καὶ τοῦ οἴκου σου· καὶ ποίησον τὰ δῶρα τοῦ λαοῦ, καὶ ἐξίλασαι περὶ αὐτῶν, καθάπερ ἐνετείλατο Κύριος τῷ Μωυσῇ.
\VS{8}Καὶ προσῆλθεν Ἀαρὼν πρὸς τὸ θυσιαστήριον, καὶ ἔσφαξε τὸ μοσχάριον τὸ περὶ τῆς ἁμαρτίας αὐτοῦ.
\VS{9}Καὶ προσήνεγκαν οἱ υἱοὶ Ἀαρὼν τὸ αἷμα πρὸς αὐτόν· καὶ ἔβαψε τὸν δάκτυλον εἰς τὸ αἷμα, καὶ ἐπέθηκεν ἐπὶ τὰ κέρατα τοῦ θυσιαστηρίου· καὶ τὸ αἷμα ἐξέχεεν ἐπὶ τὴν βάσιν τοῦ θυσιαστηρίου.
\VS{10}Καὶ τὸ στέαρ καὶ τοὺς νεφροὺς καὶ τὸν λοβὸν τοῦ ἥπατος τοῦ περὶ τῆς ἁμαρτίας ἀνήνεγκεν ἐπὶ τὸ θυσιαστήριον, ὃν τρόπον ἐνετείλατο Κύριος τῷ Μωυσῇ.
\VS{11}Καὶ τὰ κρέα καὶ τὴν βύρσαν κατέκαυσεν αὐτὰ πυρὶ, ἔξω τῆς παρεμβολῆς.
\VS{12}Καὶ ἔσφαξε τὸ ὁλοκαύτωμα· καὶ προσήνεγκαν οἱ υἱοὶ Ἀαρὼν τὸ αἷμα πρὸς αὐτόν· καὶ προσέχεεν ἐπὶ τὸ θυσιαστήριον κύκλῳ.
\VS{13}Καὶ τὸ ὁλοκαύτωμα προσήνεγκαν αὐτὸ κατὰ μέλη· αὐτὰ καὶ τὴν κεφαλὴν ἐπέθηκεν ἐπὶ τὸ θυσιαστήριον.
\VS{14}Καὶ ἔπλυνε τὴν κοιλίαν καὶ τοὺς πόδας ὕδατι· καὶ ἐπέθηκεν ἐπὶ τὸ ὁλοκαύτωμα ἐπὶ τὸ θυσιαστήριον.
\par }{\PP \VS{15}Καὶ προσήνεγκε τὸ δῶρον τοῦ λαοῦ, καὶ ἔλαβε τὸν χίμαρον τὸν περὶ τῆς ἁμαρτίας τοῦ λαοῦ, καὶ ἔσφαξεν αὐτὸν, καὶ ἐκαθάρισεν αὐτὸν, καθὰ καὶ τὸν πρῶτον.
\VS{16}Καὶ προσήνεγκε τὸ ὁλοκαύτωμα, καὶ ἐποίησεν αὐτὸ ὡς καθήκει.
\VS{17}Καὶ προσήνεγκε τὴν θυσίαν, καὶ ἔπλησε τὰς χεῖρας ἀπʼ αὐτῆς, καὶ ἐπέθηκεν ἐπὶ τὸ θυσιαστήριον χωρὶς τοῦ ὁλοκαυτώματος τοῦ πρωϊνοῦ.
\VS{18}Καὶ ἔσφαξε τὸν μόσχον, καὶ τὸν κριὸν τῆς θυσίας τοῦ σωτηρίου τῆς τοῦ λαοῦ· καὶ προσήνεγκαν οἱ υἱοὶ Ἀαρὼν τὸ αἷμα πρὸς αὐτόν, καὶ προσέχεε πρὸς τὸ θυσιαστήριον κύκλῳ,
\VS{19}καὶ τὸ στέαρ τὸ ἀπὸ τοῦ μόσχου, καὶ τοῦ κριοῦ τὴν ὀσφὺν, καὶ τὸ στέαρ τὸ κατακαλύπτον ἐπὶ τῆς κοιλίας, καὶ τοὺς δύο νεφροὺς, καὶ τὸ στέαρ τὸ ἐπʼ αὐτῶν, καὶ τὸν λοβὸν τὸν ἐπὶ τοῦ ἥπατος.
\VS{20}Καὶ ἐπέθηκε τὰ στέατα ἐπὶ τὰ στηθύνια καὶ ἀνήνεγκε τὰ στέατα ἐπὶ τὸ θυσιαστήριον.
\VS{21}Καὶ τὸ στηθύνιον, καὶ τὸν βραχίονα τὸν δεξιὸν ἀφεῖλεν Ἀαρὼν ἀφαίρεμα ἔναντι Κυρίου, ὃν τρόπον συνέταξε Κύριος τῷ Μωυσῇ.
\VS{22}Καὶ ἐξάρας Ἀαρὼν τὰς χεῖρας ἐπὶ τὸν λαὸν, εὐλόγησεν αὐτούς· καὶ κατέβη ποιήσας τὸ περὶ τῆς ἁμαρτίας, καὶ τὰ ὁλοκαυτώματα, καὶ τὰ τοῦ σωτηρίου.
\VS{23}Καὶ εἰσῆλθε Μωυσῆς καὶ Ἀαρὼν εἰς τὴν σκηνὴν τοῦ μαρτυρίου· καὶ ἐξελθόντες εὐλόγησαν πάντα τὸν λαὸν· καὶ ὤφθη δόξα Κυρίου παντὶ τῷ λαῷ.
\VS{24}Καὶ ἐξῆλθε πῦρ παρὰ Κυρίου, καὶ κατέφαγε τὰ ἐπὶ τοῦ θυσιαστηρίου, τά τε ὁλοκαυτώματα, καὶ τὰ στέατα· καὶ εἶδε πᾶς ὁ λαὸς, καὶ ἐξέστη, καὶ ἔπεσαν ἐπὶ πρόσωπον.

\par }\Chap{10}{\PP \VerseOne{1}Καὶ λαβόντες οἱ δύο υἱοὶ Ἀαρὼν Ναδὰβ καὶ Ἀβιοὺδ, ἕκαστος τὸ πυρεῖον αὐτοῦ, ἐπέθηκαν ἐπʼ αὐτὸ πῦρ, καὶ ἐπέβαλον ἐπʼ αὐτὸ θυμίαμα, καὶ προσήνεγκαν ἔναντι Κυρίου πῦρ ἀλλότριον, ὃ οὐ προσέταξε Κύριος αὐτοῖς.
\VS{2}Καὶ ἐξῆλθε πῦρ παρὰ Κυρίου, καὶ κατέφαγεν αὐτοὺς, καὶ ἀπέθανον ἔναντι Κυρίου.
\VS{3}Καὶ εἶπε Μωυσῆς πρὸς Ἀαρὼν, τοῦτό ἐστιν, ὃ εἶπε Κύριος, λέγων, ἐν τοῖς ἐγγίζουσί μοι ἁγιασθήσομαι, καὶ ἐν πάσῃ τῇ συναγωγῇ δοξασθήοσμαι· καὶ κατενύχθη Ἀαρών.
\VS{4}Καὶ ἐκάλεσε Μωυσῆς τὸν Μισαδάη, καὶ τὸν Ἐλισαφὰν, υἱοὺς Ὀζιὴλ, υἱοὺς τοῦ ἀδελφοῦ τοῦ πατρὸς Ἀαρὼν, καὶ εἶπεν αὐτοῖς, προσέλθατε καὶ ἄρατε τοὺς ἀδελφοὺς ὑμῶν ἐκ προσώπου τῶν ἁγίων ἔξω τῆς παρεμβολῆς.
\VS{5}Καὶ προσῆλθον, καὶ ᾖραν αὐτοὺς ἐν τοῖς χιτῶσιν αὐτῶν ἔξω τῆς παρεμβολῆς, ὃν τρόπον εἶπε Μωυσῆς.
\VS{6}Καὶ εἶπε Μωυσῆς πρὸς Ἀαρὼν καὶ Ἐλεάζαρ καὶ Ἰθάμαρ τοὺς υἱοὺς αὐτοῦ τοὺς καταλελειμμένους, τὴν κεφαλὴν ὑμῶν οὐκ ἀποκιδαρώσετε, καὶ τὰ ἱμάτια ὑμῶν οὐ διαῤῥήξετε, ἵνα μὴ ἀποθάνητε, καὶ ἐπὶ πᾶσαν τὴν συναγωγὴν ἔσται θυμός· οἱ δὲ ἀδελφοὶ ὑμῶν, πᾶς ὁ οἶκος Ἰσραὴλ, κλαύσονται τὸν ἐμπυρισμὸν, ὃν ἐνεπυρίσθησαν ὑπὸ Κυρίου.
\VS{7}Καὶ ἀπὸ τὴς θύρας τῆς σκηνῆς τοῦ μαρτυρίου οὐκ ἐξελεύσεσθε, ἵνα μὴ ἀποθάνητε· τὸ ἔλαιον γὰρ τῆς χρίσεως, τὸ παρὰ Κυρίου, ἐφʼ ὑμῖν, καὶ ἐποίησαν κατὰ τὸ ῥῆμα Μωυσῆ.
\par }{\PP \VS{8}Καὶ ἐλάλησε Κύριος τῷ Ἀαρὼν, λέγων,
\VS{9}οἶνον καὶ σίκερα οὐ πίεσθε σὺ καὶ οἱ υἱοί σου μετὰ σοῦ, ἡνίκα ἐὰν εἰσπορεύησθε εἰς τὴν σκηνὴν τοῦ μαρτυρίου, ἢ προσπορευομένων ὑμῶν πρὸς τὸ θυσιαστήριον, καὶ οὐ μὴ ἀποθάνητε· νόμιμον αἰώνιον εἰς τὰς γενεὰς ὑμῶν,
\VS{10}διαστεῖλαι ἀναμέσον τῶν ἁγίων καὶ τῶν βεβήλων, καὶ ἀναμέσον τῶν ἀκαθάρτων καὶ τῶν καθαρῶν,
\VS{11}καὶ συμβιβάξειν τοὺς υἱοὺς Ἰσραὴλ ἅπαντα τὰ νόμιμα, ἃ ἐλάλησε Κύριος πρὸς αὐτοὺς διὰ χειρὸς Μωυσῆ.
\VS{12}Καὶ εἶπε Μωυσῆς πρὸς Ἀαρὼν καὶ πρὸς Ἐλεάζαρ καὶ Ἰθάμαρ τοὺς υἱοὺς Ἀαρὼν τοὺς καταλειφθέντας, λάβετε τὴν θυσίαν τὴν καταλειφθεῖσαν ἀπὸ τῶν καρπωμάτων Κυρίου, καὶ φάγεσθε ἄζυμα παρὰ τὸ θυσιαστήριον· ἅγια ἁγίων ἐστί.
\VS{13}Καὶ φάγεσθε αὐτὴν ἐν τόπῳ ἁγίῳ· νόμιμον γάρ σοι ἐστὶ, καὶ νόμιμον τοῖς υἱοῖς σου τοῦτο ἀπὸ τῶν καρπωμάτων Κυρίου· οὕτω γὰρ ἐντέταλταί μοι.
\VS{14}Καὶ τὸ στηθύνιον τοῦ ἀφορίσματος, καὶ τὸν βραχίονα τοῦ ἀφαιρέματος φάγεσθε ἐν τόπῳ ἁγίῳ, σὺ καὶ οἱ υἱοί σου καὶ ὁ οἶκός σου μετὰ σοῦ· νόμιμον γὰρ σοι, καὶ νόμιμον τοῖς υἱοῖς σου ἐδόθη ἀπὸ τῶν θυσιῶν τοῦ σωτηρίου τῶν υἱῶν Ἰσραήλ.
\VS{15}Τὸν βραχίονα τοῦ ἀφαιρέματος, καὶ τὸ στηθύνιον τοῦ ἀφορίσματος ἐπὶ τῶν καρπωμάτων τῶν στεάτωι· προσοίσουσιν ἀφόρισμα ἀφορίσαι ἔναντι Κυρίου· καὶ ἔσται σοι καὶ τοῖς υἱοῖς σου καὶ ταῖς θυγατράσι σου μετὰ σοῦ νόμιμον αἰώνιον, ὃν τρόπον συνέταξε Κύριος τῷ Μωυσῇ.
\par }{\PP \VS{16}Καὶ τὸν χίμαρον τὸν περὶ τῆς ἁμαρτίας ζητῶν ἐξεζήτησε Μωυσῆς· καὶ ὁ δὲ ἐνπεπύριστο· καὶ ἐθυμώθη Μωυσῆς ἐπὶ Ἐλεάζαρ καὶ Ἰθάμαρ τοὺς υἱοὺς Ἀαρὼν τοὺς καταλελειμμένους, λέγων,
\VS{17}διατί οὐκ ἐφάγετε τὸ περὶ τῆς ἁμαρτίας ἐν τόπῳ ἁγίῳ; ὅτι γὰρ ἅγια ἁγίων ἐστι, τοῦτο ἔδωκεν ὑμῖν φαγεῖν, ἵνα ἀφέλητε τὴν ἁμαρτίαν τῆς συναγωγῆς, καὶ ἐξιλάσησθε περὶ αὐτῶν ἔναντι Κυρίου.
\VS{18}Οὐ γὰρ εἰσήχθη τοῦ αἵματος αὐτοῦ εἰς τὸ ἅγιον· κατὰ πρόσωπον ἔσω φάγεσθε αὐτὸ ἐν τόπῳ ἁγίῳ, ὃν τρόπον μοι συνέταξε Κύριος.
\VS{19}Καὶ ἐλάλησεν Ἀαρὼν πρὸς Μωυσῆν, λέγων, εἰ σήμερον προσαγιόχασι τὰ περὶ τῆς ἁμαρτίας αὐτῶν, καὶ τὰ ὁλοκαυτώματα αὐτῶν ἔναντι Κυρίου, καὶ συμβέβηκέ μοι τοιαῦτα, καὶ φάγομαι τὰ περὶ τῆς ἁμαρτίας σήμερον, μὴ ἀρεστὸν ἔται Κυρίῳ;
\VS{20}Καὶ ἤκουσε Μωυσῆς, καὶ ἤρεσεν αὐτῷ.

\par }\Chap{11}{\PP \VerseOne{1}Καὶ ἐλάλησε Κύριος πρὸς Μωυσῆν καὶ Ἀαρὼν, λέγων,
\VS{2}λαλήσατε τοῖς υἱοῖς Ἰσραὴλ, λέγοντες, ταῦτα τὰ κτήνη, ἃ φάγεσθε ἀπὸ πάντων τῶν κτηνῶν τῶν ἐπὶ τῆς γῆς.
\VS{3}Πᾶν κτῆνος διχηλοῦν ὁπλὴν καὶ ὀνυχιστῆρας ὀνυχίζον δύο χηλῶν, καὶ ἀνάγον μηρυκισμὸν ἐν τοῖς κτήνεσι, ταῦτα φάγεσθε.
\VS{4}Πλὴν ἀπὸ τούτων οὐ φάγεσθε, ἀπὸ τῶν ἀναγόντων μηρυκισμὸν, καὶ ἀπὸ τῶν διχηλούντων τὰς ὁπλὰς, καὶ ὀνυχιζόντων ὀνυχιστῆρας· τὸν κάμηλον, ὅτι ἀνάγει μηρυκισμὸν τοῦτο, ὁπλὴν δὲ οὐ διχηλεῖ, ἀκάθαρτον τοῦτο ὑμῖν.
\VS{5}Καὶ τὸν δασύποδα, ὅτι ἀνάγει μηρυκισμὸν τοῦτο, καὶ ὁπλὴν οὐ διχηλεῖ, ἀκάθαρτον τοῦτο ὑμῖν.
\VS{6}Καὶ τὸν χοιρογρύλλιον, ὅτι οὐκ ἀνάγει μηρυκισμὸν τοῦτο, καὶ ὁπλὴν οὐ διχηλεῖ, ἀκάθαρτον τοῦτο ὑμῖν.
\VS{7}Καὶ τὸν ὗν, ὅτι διχηλεῖ ὁπλὴν τοῦτο, καὶ ὀνυχίζει ὄνυχας ὁπλῆς, καὶ τοῦτο οὐκ ἀνάγει μηρυκισμὸν, ἀκάθαρτον τοῦτο ὑμῖν.
\VS{8}Ἀπὸ τῶν κρεῶν αὐτῶν οὐ φάγεσθε, καὶ τῶν θνησιμαίων αὐτῶν οὐχ ἅψεσθε· ἀκάθαρτα ταῦτα ὑμῖν.
\par }{\PP \VS{9}Καὶ ταῦτα, ἃ φάγεσθε ἀπὸ πάντων τῶν ἐν τοῖς ὕδασι· πάντα ὅσα ἐστὶν αὐτοῖς πτερύγια καὶ λεπίδες ἐν τοῖς ὕδασι, καὶ ἐν ταῖς θαλάσσαις, καὶ ἐν τοῖς χειμάῤῥοις, ταῦτα φάγεσθε.
\VS{10}Καὶ πάντα ὅσα οὐκ ἔστιν αὐτοῖς πτερύγια, οὐδὲ λεπίδες ἐν τῷ ὕδατι, ἢ ἐν ταῖς θαλάσσαις, καὶ ἐν τοῖς χειμάῤῥοις, ἀπὸ πάντων ὧν ἐρεύγεται τὰ ὕδατα, καὶ ἀπὸ πάσης ψυχῆς τῆς ζώσης ἐν τῷ ὕδατι, βδέλυγμά ἐστι, καὶ βδελύγματα ἔσονται ὑμῖν.
\VS{11}Ἀπὸ τῶν κρεῶν αὐτῶν οὐκ ἔδεσθε, καὶ τὰ θνησιμαῖα αὐτῶν βδελύξεσθε.
\VS{12}Καὶ πάντα ὅσα οὐκ ἔστιν αὐτοῖς πτερύγια, οὐδὲ λεπίδες τῶν ἐν τοῖς ὕδασι, βδέλυγμα τοῦτό ἐστιν ὑμῖν.
\VS{13}Καὶ ταῦτα, ἃ βδελύξεσθε ἀπὸ τῶν πετεινῶν, καὶ οὐ βρωθήσεται, βδέλυγμά ἐστι· τὸν ἀετὸν, καὶ τὸν γρύπα, καὶ τὸν ἁλιαίετον,
\VS{14}καὶ τὸν γύπα, καὶ τὸν ἴκτινον καὶ τὰ ὅμοια αὐτῷ.
\VS{15}Καὶ στρουθὸν, καὶ γλαῦκα, καὶ λάρον, καὶ τὰ ὅμοια αὐτῷ·
\VS{16}Καὶ πάντα κόρακα, καὶ τὰ ὅμοια αὐτῷ· καὶ ἱέρακα, καὶ τὰ ὅμοια αὐτῷ·
\VS{17}καὶ νυκτικόρακα, καὶ καταράκτην, καὶ ἴβιν,
\VS{18}καὶ πορφυρίωνα, καὶ πελεκᾶνα, καὶ κύκνον,
\VS{19}καὶ ἐρωδιὸν, καὶ χαράδριον, καὶ τὰ ὅμοια αὐτῷ· καὶ ἔποπα, καὶ νυκτερίδα.
\VS{20}Καὶ πάντα τὰ ἑρπετὰ τῶν πετεινῶν, ἃ πορεύεται ἐπὶ τέσσαρα, βδελύγματά ἐστιν ὑμῖν.
\VS{21}Ἀλλὰ ταῦτα φάγεσθε ἀπὸ τῶν ἑρπετῶν τῶν πετεινῶν, ἃ πορεύεται ἐπὶ τέσσαρα, ἃ ἔχει σκέλη ἀνώτερον τῶν ποδῶν αὐτοῦ, πηδᾷν ἐν αὐτοῖς ἐπὶ τῆς γῆς.
\VS{22}Καὶ ταῦτα φάγεσθε ἀπʼ αὐτῶν· τὸν βροῦχον, καὶ τὰ ὅμοια αὐτῷ· καὶ τὸν ἀττάκην, καὶ τὰ ὅμοια αὐτῷ· καὶ ὀφιομάχην, καὶ τὰ ὅμοια αὐτῷ· καὶ τὴν ἀκρίδα, καὶ τὰ ὅμοια αὐτῇ·
\VS{23}Πᾶν ἑρπετὸν ἀπὸ τῶν πετεινῶν, οἷς εἰσι τέσσαρες πόδες, βδελύγματά ἐστιν ὑμῖν,
\VS{24}καὶ ἐν τούτοις μιανθήσεσθε· πᾶς ὁ ἁπτόμενος τῶν θνησιμαίων αὐτῶν, ἀκάθαρτος ἔσται ἕως ἑσπέρας.
\VS{25}Καὶ πᾶς ὁ αἴρων τῶν θνησιμαίων αὐτῶν, πλυνεῖ τὰ ἱμάτια αὐτοῦ, καὶ ἀκάθαρτος ἔσται ἕως ἑσπέρας.
\VS{26}Καὶ ἐν πᾶσι τοῖς κτήνεσιν ὅ ἐστι διχηλοῦν ὁπλὴν, καὶ ὀνυχιστῆρας ὀνυχίζει, καὶ μηρυκισμὸν οὐ μηρυκᾶται, ἀκάθαρτα ἔσονται ὑμῖν· πᾶς ὁ ἁπτόμενος τῶν θνησιμαίων αὐτῶν, ἀκάθαρτος ἔσται ἕως ἑσπέρας.
\VS{27}Καὶ πᾶς ὃς πορεύεται ἐπὶ χειρῶν ἐν πᾶσι τοῖς θηρίοις, ἃ πορεύεται ἐπὶ τέσσαρα, ἀκάθαρτά ἐστιν ὑμῖν· πᾶς ὁ ἁπτόμενος τῶν θνησιμαίων αὐτῶν, ἀκάθαρτος ἔσται ἕως ἑσπέρας.
\VS{28}Καὶ ὁ αἴρων τῶν θνησιμαίων αὐτῶν, πλυνεῖ τὰ ἱμάτια αὐτοῦ, καὶ ἀκάθαρτος ἔσται ἕως ἑσπέρας· ἀκάθαρτα ταῦτά ἔστιν ὑμῖν.
\par }{\PP \VS{29}Καὶ ταῦτα ὑμῖν ἀκάθαρτα ἀπὸ τῶν ἑρπετῶν τῶν ἐπὶ τῆς γῆς· ἡ γαλὴ, καὶ ὁ μῦς, καὶ ὁ κροκόδειλος ὁ χερσαῖος,
\VS{30}μυγάλη, καὶ χαμαιλέων, καὶ χαλαβώτης, καὶ σαῦρα, καὶ ἀσπάλαξ.
\VS{31}Ταῦτα ἀκάθαρτα ὑμῖν ἀπὸ πάντων τῶν ἑρπετῶν τῶν ἐπὶ τῆς γῆς· πᾶς ὁ ἁπτόμενος αὐτῶν τεθνηκότων, ἀκάθαρτος ἔσται ἕως ἑσπέρας.
\VS{32}Καὶ πᾶν ἐφʼ ὃ ἂν ἐπιπέσῃ ἀπʼ αὐτῶν ἐπʼ αὐτὸ τεθνηκότων αὐτῶν, ἀκάθαρτον ἔσται ἀπὸ παντὸς σκεύους ξυλίνου ἢ ἱματίου ἢ δέρματος ἢ σάκκου· πᾶν σκεῦος ὃ ἂν ποιηθῇ ἔργον ἐν αὐτῷ, εἰς ὕδωρ βαφήσεται, καὶ ἀκάθαρτον ἔσται ἕως ἑσπέρας· καὶ καθαρὸν ἔσται.
\VS{33}Καὶ πᾶν σκεῦος ὀστράκινον εἰς ὃ ἐὰν πέσῃ ἀπὸ τούτων ἔνδον, ὅσα ἐὰν ἔνδον ᾖ, ἀκάθαρτα ἔσται, καὶ αὐτὸ συντριβήσεται.
\VS{34}Καὶ πᾶν βρῶμα, ὃ ἔσθεται, εἰς ὃ ἂν ἐπέλθῃ ἐπʼ αὐτὸ ὕδωρ, ἀκάθαρτον ἔσται· καὶ πᾶν ποτὸν, ὃ πίνεται ἐν παντὶ ἀγγείῳ, ἀκάθαρτον ἔσται.
\VS{35}Καὶ πᾶν ὃ ἐὰν ἐπιπέσῃ ἀπὸ τῶν θνησιμαίων αὐτῶν ἐπʼ αὐτό, ἀκάθαρτον ἔσται· κλίβανοι καὶ χυτρόποδες καθαιρεθήσονται· ἀκάθαρτα ταῦτά ἐστι, καὶ ἀκάθαρτα ταῦτα ὑμῖν ἔσονται.
\VS{36}Πλὴν πηγῶν ὑδάτων καὶ λάκκου καὶ συναγωγῆς ὕδατος, ἔσται καθαρόν· ὁ δὲ ἁπτόμενος τῶν θνησιμαίων αὐτῶν, ἀκάθαρτος ἔσται.
\VS{37}Ἐὰν δὲ ἐπιπέσῃ ἀπὸ τῶν θνησιμαίων αὐτῶν ἐπὶ πᾶν σπέρμα σπόριμον, ὃ σπαρήσεται, καθαρὸν ἔσται.
\VS{38}Ἐὰν δὲ ἐπιχυθῇ ὕδωρ ἐπὶ πᾶν σπέρμα, καὶ ἐπιπέσῃ τῶν θνησιμαίων αὐτῶν ἐπʼ αὐτό, ἀκάθαρτόν ἐστιν ὑμῖν.
\VS{39}Ἐὰν δὲ ἀποθάνῃ τῶν κτηνῶν, ὅ ἐστιν ὑμῖν φαγεῖν τοῦτο, ὁ ἁπτόμενος τῶν θνησιμαίων αὐτῶν, ἀκάθαρτος ἔσται ἕως ἑσπέρας·
\VS{40}καὶ ὁ ἐσθίων ἀπὸ τῶν θνησιμαίων τούτων, πλυνεῖ τὰ ἱμάτια, καὶ ἀκάθαρτος ἔσται ἕως ἑσπέρας· καὶ ὁ αἴρων ἀπὸ θνησιμαίων αὐτῶν, πλυνεῖ τὰ ἱμάτια, καὶ λούσεται ὕδατι, καὶ ἀκάθαρτος ἔσται ἕως ἑσπέρας.
\VS{41}Καὶ πᾶν ἑρπετὸν, ὃ ἕρπει ἐπὶ τῆς γῆς, βδέλυγμα ἔσται τοῦτο ὑμῖν· οὐ βρωθήσεται.
\VS{42}Καὶ πᾶς ὁ πορευόμενος ἐπὶ κοιλίας, καὶ πᾶς ὁ πορευόμενος ἐπὶ τέσσαρα διαπαντός, ὃ πολυπληθεῖ ποσὶν ἐν πᾶσι τοῖς ἑρπετοῖς τοῖς ἕρπουσιν ἐπὶ τῆς γῆς, οὐ φάγεσθε αὐτὸ, ὅτι βδέλυγμα ὑμῖν ἐστι.
\VS{43}Καὶ οὐ μὴ βδελύξητε τὰς ψυχὰς ὑμῶν ἐν πᾶσι τοῖς ἑρπετοῖς τοῖς ἕρπουσιν ἐπὶ τῆς γῆς, καὶ οὐ μιανθήσεσθε ἐν τούτοις, καὶ οὐκ ἀκάθαρτοι ἔσεσθε ἐν αὐτοῖς,
\VS{44}ὅτι ἐγώ εἰμι Κύριος ὁ Θεὸς ὑμῶν· καὶ ἁγιασθήσεσθε, καὶ ἅγιοι ἔσεσθε, ὅτι ἅγιός εἰμι ἐγὼ Κύριος ὁ Θεὸς ὑμῶν· καὶ οὐ μιανεῖτε τὰς ψυχὰς ὑμῶν ἐν πᾶσι τοῖς ἑρπετοῖς τοῖς κινουμένοις ἐπὶ τῆς γῆς,
\VS{45}ὅτι ἐγώ εἰμι Κύριος ὁ ἀναγαγὼν ὑμᾶς ἐκ γῆς Αἰγύπτου εἶναι ὑμῶν Θεός· καὶ ἔσεσθε ἅγιοι, ὅτι ἅγιός εἰμι ἐγὼ Κύριος.
\VS{46}Οὗτος ὁ νόμος περὶ τῶν κτηνῶν καὶ τῶν πετεινῶν καὶ πάσης ψυχῆς τῆς κινουμένης ἐν τῷ ὕδατι, καὶ πάσης ψυχῆς ἑρπούσης ἐπὶ τῆς γῆς,
\VS{47}διαστεῖλαι ἀναμέσον τῶν ἀκαθάρτων καὶ ἀναμέσον τῶν καθαρῶν, καὶ ἀναμέσον τῶν ζωογονούντων τὰ ἐσθιόμενα καὶ ἀναμέσον τῶν ζωογονούντων τὰ μὴ ἐσθιόμενα.

\par }\Chap{12}{\PP \VerseOne{1}Καὶ ἐλάλησε Κύριος πρὸς Μωυσῆν, λέγων,
\VS{2}λάλησον τοῖς υἱοῖς Ἰσραὴλ, καὶ ἐρεῖς πρὸς αὐτοὺς, γυνὴ ἥτις ἐὰν σπερματισθῇ, καὶ τέκῃ ἄρσεν, καὶ ἀκάθαρτος ἔσται ἑπτὰ ἡμέρας· κατὰ τὰς ἡμέρας τοῦ χωρισμοῦ τῆς ἀφέδρου αὐτῆς, ἀκάθαρτος ἔσται.
\VS{3}Καὶ τῇ ἡμέρᾳ τῇ ὀγδόῃ περιτεμεῖ τὴν σάρκα τῆς ἀκροβυστίας αὐτοῦ.
\VS{4}Καὶ τριάκοντα καὶ τρεῖς ἡμέρας καθήσεται ἐν αἵματι ἀκαθάρτῳ αὐτῆς· παντὸς ἁγίου οὐξ ἅψεται, καὶ εἰς τὸ ἁγιαστήριον οὐκ εἰσελεύσεται, ἕως ἂν πληρωθῶσιν αἱ ἡμέραι καθάρσεως αὐτῆς.
\VS{5}Ἐὰν δὲ θῆλυ τέκῃ, καὶ ἀκάθαρτος ἔσται δὶς ἑπτὰ ἡμέρας, κατὰ τὴν ἄφεδρον αὐτῆς· καὶ ἑξήκοντα ἡμέρας καὶ ἓξ καθεσθήσεται ἐν αἵματι ἀκαθάρτῳ αὐτῆς.
\par }{\PP \VS{6}Καὶ ὅταν ἀναπληρωθῶσιν αἱ ἡμέραι καθάρσεως αὐτῆς ἐφʼ υἱῷ ἢ ἐπὶ θυγατρι, προσοίσει ἀμνὸν ἐνιαύσιον ἄμωμον εἰς ὁλοκαύτωμα, καὶ νοσσὸν περιστερᾶς ἢ τρυγόνα περὶ ἁμαρτίας ἐπὶ τὴν θύραν τῆς σκηνῆς τοῦ μαρτυρίου, πρὸς τὸν ἱερέα.
\VS{7}Καὶ προσοίσει αὐτὸν ἔναντι Κυρίου· καὶ ἐξιλάσεται περὶ αὐτῆς ὁ ἱερεὺς, καὶ καθαριεῖ αὐτὴν ἀπὸ τῆς πηγῆς τοῦ αἵματος αὐτῆς· οὗτος ὁ νόμος τῆς τικτούσης ἄρσεν ἢ θῆλυ.
\VS{8}Ἐὰν δὲ μὴ εὑρίσκῃ ἡ χεὶρ αὐτῆς τὸ ἱκανὸν εἰς ἀμνὸν, καὶ λήψεται δύο τρυγόνας ἢ δύο νοσσοὺς περιστερῶν, μίαν εἰς ὁλοκαύτωμα, καὶ μίαν περὶ ἁμαρτίας· καὶ ἐξιλάσεται περὶ αὐτῆς ὁ ἱερεὺς, καὶ καθαρισθήσεται.

\par }\Chap{13}{\PP \VerseOne{1}Καὶ ἐλάλησε Κύριος πρὸς Μωυσῆν καὶ Ἀαρὼν, λέγων,
\VS{2}ἀνθρώπῳ ἐάν τινι γένηται ἐν δέρματι χρωτὸς αὐτοῦ οὐλὴ σημασίας τηλαυγὴς, καὶ γένηται ἐν δέρματι χρωτὸς αὐτοῦ ἁφὴ λέπρας, ἀχθήσεται πρὸς Ἀαρὼν τὸν ἱερέα, ἢ ἕνα τῶν υἱῶν αὐτοῦ τῶν ἱερέων.
\VS{3}Καὶ ὄψεται ὁ ἱερεὺς τὴν ἁφὴν ἐν δέρματι τοῦ χρωτὸς αὐτοῦ, καὶ ἡ θρὶξ ἐν τῇ ἁφῇ μεταβάλῃ λευκὴ, καὶ ἡ ὄψις τῆς ἁφῆς ταπεινὴ ἀπὸ τοῦ δέρματος τοῦ χρωτὸς, ἁφὴ λέπρας ἐστί· καὶ ὄψεται ὁ ἱερεὺς, καὶ μιανεῖ αὐτόν.
\VS{4}Ἐὰν δὲ καὶ τηλαυγὴς λευκὴ ἦ ἐν τῷ δέρματι τοῦ χρωτὸς αὐτοῦ, καὶ ταπεινὴ μὴ ἦ ἡ ὄψις αὐτῆς ἀπὸ τοῦ δέρματος, καὶ ἡ θρὶξ αὐτοῦ οὐ μετέβαλε τρίχα λευκὴν, αὐτὴ δέ ἐστιν ἀμαυρὰ, καὶ ἀφοριεῖ ὁ ἱερεὺς τὴν ἁφὴν ἑπτὰ ἡμέρας.
\VS{5}Καὶ ὄψεται ὁ ἱερεὺς τὴν ἁφὴν τῇ ἡμέρᾳ τῇ ἑβδόμῃ· καὶ ἰδοὺ ἡ ἁφὴ μένει ἐναντίον αὐτοῦ, οὐ μετέπεσεν ἡ ἁφὴ ἐν τῷ δέρματι, καὶ ἀφοριεῖ αὐτὸν ὁ ἱερεὺς ἑπτὰ ἡμέρας τοδεύτερον.
\VS{6}Καὶ ὄψεται ὁ ἱερεὺς αὐτὸν τῇ ἡμέρᾳ τῇ ἑβδόμῃ τοδεύτερον· καὶ ἰδοὺ ἀμαυρὰ ἡ ἁφή, οὐ μετέπεσεν ἡ ἁφὴ ἐν τῷ δέρματι· καὶ καθαριεῖ αὐτὸν ὁ ἱερεύς, σημασία γάρ ἐστι· καὶ πλυνάμενος τὰ ἱμάτια αὐτοῦ, καθαρὸς ἔσται.
\VS{7}Ἐὰν δὲ μεταβαλοῦσα μεταπέσῃ ἡ σημασία ἐν τῷ δέρματι, μετὰ τὸ ἰδεῖν αὐτὸν τὸν ἱερέα τοῦ καθαρίσαι αὐτόν, καὶ ὀφθήσεται τοδεύτερον τῷ ἱερεῖ.
\VS{8}Καὶ ὄψεται αὐτὸν ὁ ἱερεύς, καὶ ἰδοὺ μετέπεσεν ἡ σημασία ἐν τῷ δέρματι, καὶ μιανεῖ αὐτὸν ὁ ἱερεύς· λέπρα ἐστί.
\par }{\PP \VS{9}Καὶ ἁφὴ λέπρας ἐὰν γένηται ἐν ἀνθρώπῳ, και ἥξει πρὸς τὸν ἱερέα·
\VS{10}Καὶ ὄψεται ὁ ἱερεὺς, καὶ ἰδοὺ οὐλὴ λευκὴ ἐν τῷ δέρματι, καὶ αὕτη μετέβαλε τρίχα λευκὴν, καὶ ἀπὸ τοῦ ὑγιοῦς τῆς σαρκὸς τῆς ζώσης ἐν τῇ οὐλῇ.
\VS{11}Λέπρα παλαιουμένη ἐστὶν ἐν τῷ δέρματι τοῦ χρωτός, καὶ μιανεῖ αὐτὸν ὁ ἱερεὺς, καὶ ἀφοριεῖ αὐτὸν, ὅτι ἀκάθαρτός ἐστιν.
\par }{\PP \VS{12}Ἐὰν δὲ ἀνθοῦσα ἐξανθήσῃ λέπρα ἐν τῷ δέρματι, καὶ καλύψῃ ἡ λέπρα πᾶν τὸ δέρμα τῆς ἁφῆς ἀπὸ κεφαλῆς ἕως ποδῶν, καθʼ ὅλην τὴν ὅρασιν τοῦ ἱερέως·
\VS{13}Καὶ ὄψεται ὁ ἱερεὺς, καὶ ἰδοὺ ἐκάλυψεν ἡ λέπρα πᾶν τὸ δέρμα τοῦ χρωτός· καὶ καθαριεῖ αὐτὸν ὁ ἱερεὺς τὴν ἁφήν, ὅτι πᾶν μετέβαλε λευκὸν, καθαρόν ἐστι.
\VS{14}Καὶ ᾗ ἂν ἡμέρᾳ ὀφθῇ ἐν αὐτῷ χρὼς ζῶν, μιανθήσεται.
\VS{15}Καὶ ὅψεται ὁ ἱερεὺς τὸν χρῶτα τὸν ὑγιῆ, καὶ μιανεῖ αὐτὸν ὁ χρὼς ὁ ὑγιὴς, ὅτι ἀκάθαρτός ἐστι· λέπρα ἐστίν.
\VS{16}Ἐὰν δὲ ἀποκαταστῇ ὁ χρὼς ὁ ὑγιὴς, καὶ μεταβάλῃ λευκὴ, καὶ ἐλεύσεται πρὸς τὸν ἱερέα·
\VS{17}Καὶ ὄψεται ὁ ἱερεὺς, καὶ ἰδοὺ μετέβαλεν ἡ ἁφὴ εἰς τὸ λευκόν, καὶ καθαριεῖ ὁ ἱερεὺς τὴν ἁφήν· καθαρός ἐστι.
\par }{\PP \VS{18}Καὶ σὰρξ ἐὰν γένηται ἐν τῷ δέρματι αὐτοῦ ἕλκος, καὶ ὑγιασθῇ,
\VS{19}καὶ γένηται ἐν τῷ τόπῳ τοῦ ἕλκους οὐλὴ λευκὴ, ἢ τηλαυγὴς λευκαίνουσα, ἢ πυῤῥίζουσα, καὶ ὀφθήσεται τῷ ἱερεῖ·
\VS{20}Καὶ ὄψεται ὁ ἱερεὺς, καὶ ἰδοὺ ἡ ὄψις ταπεινοτέρα τοῦ δέρματος, καὶ ἡ θρὶξ αὐτῆς μετέβαλεν εἰς λευκὴν, καὶ μιανεῖ αὐτὸν ὁ ἱερεὺς, ὅτι λέπρα ἐστίν· ἐν τῷ ἕλκει ἐξήνθησεν.
\VS{21}Ἐὰν δὲ ἴδῃ ὁ ἱερεὺς, καὶ ἰδοὺ οὐκ ἔστιν ἐν αὐτῷ θρὶξ λευκὴ, καὶ ταπεινὸν μὴ ᾖ ἀπὸ τοῦ δέρματος τοῦ χρωτὸς, καὶ αὐτὴ ᾖ ἀμαυρά, καὶ ἀφοριεῖ αὐτὸν ὁ ἱερεὺς ἑπτὰ ἡμέρας.
\VS{22}Ἐὰν δὲ διαχύσει διαχέηται ἐν τῷ δέρματι, καὶ μιανεῖ αὐτὸν ὁ ἱερεὺς, ἁφὴ λέπρας ἐστίν· ἐν τῷ ἕλκει ἐξήνθησεν·
\VS{23}Ἐὰν δὲ κατὰ χώραν μείνῃ τὸ τηλαύγημα καὶ μὴ διαχέηται, οὐλὴ τοῦ ἕλκους ἐστὶ, καὶ καθαριεῖ αὐτὸν ὁ ἱερεύς.
\par }{\PP \VS{24}Καὶ σὰρξ ἐὰν γένηται ἐν τῷ δέρματι αὐτοῦ κατάκαυμα πυρὸς, καὶ γένηται ἐν τῷ δέρματι αὐτοῦ τὸ ὑγιασθὲν τοῦ κατακαύματος αὐγάζον τηλαυγὲς λευκὸν, ὑποπυῤῥίζον, ἢ ἔκλευκον·
\VS{25}Καὶ ὄψεται αὐτὸν ὁ ἱερεὺς, καὶ ἰδοὺ μετέβαλε θρὶξ λευκὴ εἰς τὸ αὐγάζον, καὶ ἡ ὄψις αὐτοῦ ταπεινὴ ἀπὸ τοῦ δέρματος, λέπρα ἐστίν· ἐν τῷ κατακαύματι ἐξήνθησε· καὶ μιανεῖ αὐτὸν ὁ ἱερεὺς, ἁφὴ λέπρας ἐστίν.
\VS{26}Ἐὰν δὲ ἴδῃ ὁ ἱερεὺς, καὶ ἰδοὺ οὐκ ἔστιν ἐν τῷ· αὐγάζοντι θρὶξ λευκή, καὶ ταπεινὸν μὴ ᾖ ἀπὸ τοῦ δέρματος, αὐτὸ δὲ ἀμαυρὸν, καὶ ἀφοριεῖ αὐτὸν ὁ ἱερεὺς ἑπτὰ ἡμέρας.
\VS{27}Καὶ ὄψεται αὐτὸν ὁ ἱερεὺς τῇ ἡμέρᾳ τῇ ἑβδόμῃ· ἐὰν δὲ διαχύσει διαχέηται ἐν τῷ δέρματι, καὶ μιανεῖ αὐτὸν ὁ ἱερεὺς, ἁφὴ λέπρας ἐστίν· ἐν τῷ ἕλκει ἐξήνθησεν.
\VS{28}Ἐὰν δὲ κατὰ χώραν μείνῃ τὸ αὐγάζον, καὶ μὴ διαχυθῇ ἐν τῷ δέρματι, αὐτὴ δὲ ἀμαυρὰ ᾖ, οὐλὴ τοῦ κατακαύματός ἐστι, καὶ καθαριεῖ αὐτὸν ὁ ἱερεύς· ὁ γὰρ χαρακτὴρ τοῦ κατακαύματός ἐστι.
\par }{\PP \VS{29}Καὶ ἀνδρὶ ἢ γυναικὶ ἐὰν γένηται ἐν αὐτοῖς ἁφὴ λέπρας ἐν τῇ κεφαλῇ ἢ ἐν τῷ πώγωνι·
\VS{30}Καὶ ὄψεται ὁ ἱερεὺς τὴν ἁφὴν, καὶ ἰδοὺ ἡ ὄψις αὐτῆς ἐγκοιλοτέρα τοῦ δέρματος, ἐν αὐτῇ δὲ θρὶξ ξανθίζουσα λεπτὴ, καὶ μιανεῖ αὐτὸν ὁ ἱερεύς· θραῦσμά ἐστι, λέπρα τῆς κεφαλῆς ἢ λέπρα τοῦ πώγωνός ἐστι.
\VS{31}Καὶ ἐὰν ἴδῃ ὁ ἱερεὺς τὴν ἁφὴν τοῦ θραύσματος, καὶ ἰδοὺ οὐχ ἡ ὄψις ἐγκοιλοτέρα τοῦ δέρματος, καὶ θρὶξ ξανθίζουσα οὐκ ἔστιν ἐν αὐτῇ, καὶ ἀφοριεῖ ὁ ἱερεὺς τὴν ἁφὴν τοῦ θραύσματος ἑπτὰ ἡμέρας.
\VS{32}Καὶ ὄψεται ὁ ἱερεὺς τὴν ἁφὴν τῇ ἡμέρᾳ τῇ ἑβδόμῃ, καὶ ἰδοὺ οὐ διεχύθη τὸ θραῦσμα, καὶ θρὶξ ξανθίζουσα οὐκ ἔστιν ἐν αὐτῇ, καὶ ἡ ὄψις τοῦ θραύσματος οὐκ ἔστι κοίλη ἀπὸ τοῦ δέρματος·
\VS{33}Καὶ ξυρηθήσεται τὸ δέρμα, τὸ δὲ θραῦσμα οὐ ξυρηθήσεται, καὶ ἀφοριεῖ ὁ ἱερεὺς τὸ θραῦσμα ἑπτὰ ἡμέρας τὸ δεύτερον.
\VS{34}Καὶ ὄψεται ὁ ἱερεὺς τὸ θραῦσμα τῇ ἡμέρᾳ τῇ ἑβδόμῃ, καὶ ἰδοὺ οὐ διεχύθη τὸ θραῦσμα ἐν τῷ δέρματι μετὰ τὸ ξυρηθῆναι αὐτόν, καὶ ἡ ὄψις τοῦ θραύσματος οὐκ ἔστιν κοίλη ἀπὸ τοῦ δέρματος, καὶ καθαριεῖ αὐτὸν ὁ ἱερεὺς, καὶ πλυνάμενος τὰ ἱμάτια, καθαρὸς ἔσται.
\VS{35}Ἐὰν δὲ διαχύσει διαχέηται τὸ θραῦσμα ἐν τῷ δέρματι μετὰ τὸ καθαρισθῆναι αὐτόν·
\VS{36}Καὶ ὄψεται ὁ ἱερεὺς, καὶ ἰδοὺ διακέχυται τὸ θραῦσμα ἐν τῷ δέρματι, οὐκ ἐπισκέψεται ὁ ἱερεὺς περὶ τῆς τριχὸς τῆς ξανθῆς, ὅτι ἀκάθαρτός ἐστιν.
\VS{37}Ἐὰν δὲ ἐνώπιον μείνῃ ἐπὶ χώρας τὸ θραῦσμα, καὶ θρὶξ μέλαινα ἀνατείλῃ ἐν αὐτῷ, ὑγίακε τὸ θραῦσμα, καθαρός ἐστι, καὶ καθαριεῖ αὐτὸν ὁ ἱερεύς.
\VS{38}Καὶ ἀνδρὶ ἢ γυναικὶ ἐὰν γένηται ἐν δέρματι τῆς σαρκὸς αὐτοῦ αὐγάματα αὐγάζοντα λευκανθίζοντα·
\VS{39}καὶ ὄψεται ὁ ἱερεὺς, καὶ ἰδοὺ ἐν δέρματι τῆς σαρκὸς αὐτοῦ αὐγάσματα αὐγάζοντα λευκαθίζοντα, ἀλφός ἐστιν· ἐξανθεῖ ἐν τῷ δέρματι τῆς σαρκὸς αὐτοῦ, καθαρός ἐστιν.
\VS{40}Ἐὰν δέ τινι μαδήσῃ ἡ κεφαλὴ αὐτοῦ, φαλακρός ἐστι, καθαρός ἐστιν.
\VS{41}Ἐὰν δὲ κατὰ πρόσωπον μαδήσῃ ἡ κεφαλὴ αὐτοῦ, ἀναφάλαντός ἐστι, καθαρός ἐστιν.
\VS{42}Ἐὰν δὲ γένηται ἐν τῷ φαλακρώματι αὐτοῦ ἢ ἐν τῷ ἀναφαλαντώματι αὐτοῦ ἁφὴ λευκὴ ἢ πυῤῥίζουσα, λέπρα ἐστὶν ἐν τῷ φαλακρώματι αὐτοὺ, ἢ ἐν τῷ ἀναφαλαντώματι αὐτοῦ·
\VS{43}καὶ ὄψεται αὐτὸν ὁ ἱερεύς, καὶ ἰδοὺ ἡ ὄψις τῆς ἁφῆς λευκὴ ἢ πυῤῥίζουσα ἐν τῷ φαλακρώματι αὐτοῦ ἢ ἐν τῷ ἀναφαλαντώματι αὐτοῦ, ὡς ωἶδος λέπρας ἐν δέρματι τῆς σαρκὸς αὐτοῦ·
\VS{44}Ἄνθρωπος λεπρός ἐστι· μιάνσει μιανεῖ αὐτὸν ὁ ἱερεὺς, ἐν τῇ κεφαλῇ αὐτοῦ ἡ ἁφὴ αὐτοῦ.
\VS{45}Καὶ ὁ λεπρὸς ἐν ᾧ ἐστιν ἡ ἁφὴ, τὰ ἱμάτια αὐτοῦ ἔστω παραλελυμένα, καὶ ἡ κεφαλὴ αὐτοῦ ἀκάλυπτος, καὶ περὶ τὸ στόμα αὐτοῦ περιβαλέσθω, καὶ ἀκάθαρτος κεκλήσεται.
\VS{46}Πάσας τὰς ἡμέρας, ὅσας ἐὰν ᾖ ἐπʼ αὐτὸν ἡ ἁφὴ, ἀκάθαρτος ὢν ἀκάθαρτος ἔσται· κεχωρισμένος καθήσεται, ἔξω τῆς παρεμβολῆς αὐτοῦ ἔσται ἡ διατριβή.
\par }{\PP \VS{47}Καὶ ἱματίῳ ἐὰν γένηται ἁφὴ ἐν αὐτῷ λέπρας, ἐν ἱματίῳ ἐρέῳ, ἢ ἐν ἱματίῳ στυππυίνῳ,
\VS{48}ἢ ἐν στήμονι, ἢ ἐν κρόκῃ, ἢ ἐν τοῖς λινοῖς, ἢ ἐν τοῖς ἐρέοις, ἢ ἐν δέρματι, ἢ ἐν παντὶ ἐργασίμῳ δέρματι,
\VS{49}καὶ γένηται ἡ ἁφὴ χλωρίζουσα ἢ πυῤῥίζουσα ἐν τῷ δέρματι, ἢ ἐν τῷ ἱματίῳ, ἢ ἐν τῷ στήμονι, ἢ ἐν τῇ κρόκῃ, ἢ ἐν παντὶ σκεύει ἐργασίμῳ δέρματος, ἁφὴ λέπρας ἐστί· καὶ δείξει τῷ ἱερεῖ·
\VS{50}Καὶ ὄψεται ὁ ἱερεὺς τὴν ἁφήν, καὶ ἀφοριεῖ ὁ ἱερεὺς τὴν ἁφὴν ἑπτὰ ἡμέρας.
\VS{51}Καὶ ὄψεται ὁ ἱερεὺς τὴν ἁφὴν τῇ ἡμέρᾳ τῇ ἑβδόμῃ· ἐὰν δὲ διαχέηται ἡ ἁφὴ ἐν τῷ ἱματίῳ, ἢ ἐν τῷ στήμονι, ἢ ἐν τῇ κρόκῃ, ἢ ἐν τῷ δέρματι, κατὰ πάντα ὅσα ἐὰν ποιηθῇ δέρματα ἐν τῇ ἐργασίᾳ, λέπρα ἔμμονός ἐστιν ἡ ἁφὴ, ἀκάθαρτός ἐστι.
\VS{52}Κατακαύσει τὸ ἱμάτιον, ἢ τὸν στήμονα, ἢ τὴν κρόκην ἐν τοῖς ἐρέοις, ἢ ἐν τοῖς λινοῖς, ἢ ἐν παντὶ σκεύει δερματίνῳ, ἐν ᾧ ἂν ᾖ ἐν αὐτῷ ἡ ἁφὴ, ὅτι λέπρα ἔμμονός ἐστιν, ἐν πυρὶ κατακαυθήσεται.
\par }{\PP \VS{53}Ἐὰν δὲ ἴδῃ ὁ ἱερεὺς, καὶ μὴ διαχέηται ἡ ἁφὴ ἐν τῷ ἱματίῳ, ἢ ἐν τῷ στήμονι, ἢ ἐν τῇ κρόκῃ, ἢ ἐν παντὶ σκεύει δερματίνῳ·
\VS{54}Καὶ συντάξει ὁ ἱερεύς, καὶ πλυνεῖ ἐφʼ οὗ ἐὰν ᾖ ἐπʼ αὐτοῦ ἡ ἁφὴ, καὶ ἀφοριεῖ ὁ ἱερεὺς τὴν ἁφὴν ἑπτὰ ἡμέρας τοδέυτερον.
\VS{55}Καὶ ὄψεται ὁ ἱερεὺς μετὰ τὸ πλυθῆναι αὐτὸ τὴν ἁφὴν, καὶ ἥδε οὐ μὴ μετέβαλεν ἡ ἁφὴ τὴν ὄψιν, καὶ ἡ ἁφὴ οὐ διαχεῖται, ἀκάθαρτόν ἐστιν, ἐν πυρὶ κατακαυθήσεται· ἐστήρικται ἐν τῷ ἱματίῳ, ἢ ἐν τῷ στήμονι, ἢ ἐν τῇ κρόκη.
\VS{56}Καὶ ἐὰν ἴδῃ ὁ ἱερεὺς, καὶ ᾖ ἀμαυρὰ ἡ ἁφὴ μετὰ τὸ πλυθήναι αὐτὸ, ἀποῤῥήξει αὐτὸ ἀπὸ τοῦ ἱματίου, ἢ ἀπὸ τοῦ στήμονος, ἢ ἀπὸ τῆς κρόκης, ἢ ἀπὸ τοῦ δέρματος.
\VS{57}Ἐὰν δὲ ὀφθῇ ἔτι ἐν τῷ ἱματίῳ, ἢ ἐν τῷ στήμονι, ἢ ἐν τῇ κρόκῃ, ἢ ἐν παντὶ σκεύει δερματίνῳ, λέπρα ἐξανθοῦσά ἐστιν ἐν πυρὶ κατακαυθήσεται ἐν ᾧ ἐστιν ἡ ἁφή.
\VS{58}Καὶ τὸ ἱμάτιον, ἢ ὁ στήμων, ἢ ἡ κρόκη, ἢ πᾶν σκεῦος δερμάτινον, ὃ πλυθήσεται, καὶ ἀποστήσεται ἀπʼ αὐτοῦ ἡ ἁφὴ, καὶ πλυθήσεται τὸ δέυτερον, καὶ καθαρὸν ἔσται.
\VS{59}Οὗτος ὁ νόμος ἁφῆς λέπρας ἱματίου ἐρέου, ἢ στυππυίνου, ἢ στήμονος, ἢ κρόκης, ἢ παντὸς σκεύους δερματίνου, εἰς τὸ καθαρίσαι αὐτὸ, ἢ μιᾶναι αὐτό.

\par }\Chap{14}{\PP \VerseOne{1}Καὶ ἐλάλησε Κύριος πρὸς Μωυσῆν, λέγων,
\VS{2}οὗτος ὁ νὁμος τοῦ λεπροῦ· ᾗ ἂν ἡμέρᾳ καθαρισθῇ, καὶ προσαχθήσεται πρὸς τὸν ἱερέα.
\VS{3}Καὶ ἐξελεύσεται ὁ ἱερεὺς ἔξω τῆς παρεμβολῆς, καὶ ὄψεται ὁ ἱερεὺς, καὶ ἰδοὺ ἰᾶται ἡ ἁφὴ τῆς λέπρας ἀπὸ τοῦ λεπροῦ.
\VS{4}Καὶ προστάξει ὁ ἱερεὺς, καὶ λήψονται τῷ κεκαθαρισμένῳ δύο ὀρνίθια ζῶντα καθαρὰ, καὶ ξύλον κέδρινον, καὶ κεκλωσμένον κόκκινον, καὶ ὕσσωπον.
\VS{5}Καὶ προστάξει ὁ ἱερεὺς, καὶ σφάξουσι τὸ ὀρνίθιον τὸ ἓν εἰς ἀγγεῖον ὀστράκινον ἐφʼ ὕδατι ζῶντι.
\VS{6}Καὶ τὸ ὀρνίθιον τὸ ζῶν λήμψεται αὐτὸ, καὶ τὸ ξύλον τὸ κέδρινον, καὶ τὸ κλωστὸν κόκκινον, καὶ τὸν ὕσσωπον, καὶ βάψει αὐτὰ καὶ τὸ ὀρνίθιον τὸ ζῶν εἰς τὸ αἷμα τοῦ ὀρνιθίου τοῦ σφαγέντος ἐφʼ ὕδατι ζῶντι.
\VS{7}Καὶ περιῤῥανεῖ ἐπὶ τὸν καθαρισθέντα ἀπὸ τῆς λέπρας ἑπτάκις, καὶ καθαρὸς ἔσται· καὶ ἐξαποστελεῖ τὸ ὀρνίθιον τὸ ζῶν εἰς τὸ πεδίον.
\VS{8}Καὶ πλυνεῖ ὁ καθαρισθεὶς τὰ ἱμάτια αὐτοῦ, καὶ ξυρηθήσεται αὐτοῦ πᾶσαν τὴν τρίχα, καὶ λούσεται ἐν ὕδατι, καὶ καθαρὸς ἔσται· καὶ μετὰ ταῦτα εἰσελεύσεται εἰς τὴν παρεμβολὴν, καὶ διατρίψει ἔξω τοῦ οἴκου αὐτοῦ ἑπτὰ ἡμέρας.
\VS{9}Καὶ ἔσται τῇ ἡμέρᾳ τῇ ἑβδόμῃ, ξυρηθήσεται πᾶσαν τὴν τρίχα αὐτοῦ, τὴν κεφαλὴν αὐτοῦ, καὶ τὸν πώγωνα, καὶ τὰς ὀφρῦς, καὶ πᾶσαν τὴς τρίχα αὐτοῦ ξυρηθήσεται· καὶ πλυνεῖ τὰ ἱμάτια, καὶ λούσεται τὸ σῶμα αὐτοῦ ὕδατι, καὶ καθαρὸς ἔσται.
\VS{10}Καὶ τῇ ἡμέρᾳ τῇ ὀγδόῃ λήψεται δύο ἀμνοὺς ἀμώμους ἑνιαυσίους, καὶ πρόβατον ἄμωμον ἑνιαύσιον, καὶ τρία δέκατα σεμιδάλεως εἰς θυσίαν πεφυραμένης ἐν ἐλαίῳ, καὶ κοτύλην ἐλαίου μίαν.
\VS{11}Καὶ στήσει ὁ ἱερεὺς ὁ καθαρίζων, τὸν ἄνθρωπον τὸν καθαριζόμενον, καὶ ταῦτα ἔναντι Κυρίου, ἐπὶ τὴν θύραν τῆς σκηνῆς τοῦ μαρτυρίου.
\VS{12}Καὶ λήψεται ὁ ἱερεὺς τὸν ἀμνὸν τὸν ἕνα, καὶ προσάξει αὐτὸν τῆς πλημμελείας, καὶ τὴν κοτύλην τοῦ ἐλαίου, καὶ ἀφοριεῖ αὐτὰ ἀφόρισμα ἔναντι Κυρίου.
\VS{13}Καὶ σφάξουσι τὸν ἀμνὸν ἐν τόπῳ, οὗ σφάζουσι τὰ ὁλοκαυτώματα, καὶ τὰ περὶ ἁμαρτίας, ἐν τόπῳ ἁγίῳ· ἔστι γὰρ τὸ περὶ ἁμαρτίας, ὥσπερ τὸ τῆς πλημμελείας ἐστὶ τῷ ἱερει· ἅγια ἁγίων ἐστί.
\VS{14}Καὶ λήψεται ὁ ἱερεὺς ἀπὸ τοῦ αἵματος τοῦ τῆς πλημμελείας, καὶ ἐπιθήσει ὁ ἱερεὺς ἐπὶ τὸν λοβὸν τοῦ ὠτὸς τοῦ καθαριζομένου τοῦ δεξιοῦ, καὶ ἐπὶ τὸ ἄκρον τῆς χειρὸς τῆς δεξιᾶς, καὶ ἐπὶ τὸ ἄκρον τοῦ ποδὸς τοῦ δεξιοῦ.
\VS{15}Καὶ λαβὼν ὁ ἱερεὺς ἀπὸ τῆς κοτύλης τοῦ ἐλαίου, ἐπιχεεῖ ἐπὶ τὴν χεῖρα τοῦ ἱερέως τὴν ἀριστεράν.
\VS{16}Καὶ βάψει τὸν δάκτυλον τὸν δεξιὸν ἀπὸ τοὺ ἐλαίου τοῦ ὄντος ἐπὶ τῆς χειρὸς αὐτοῦ τῆς ἀριστερᾶς· καὶ ῥανεῖ τῷ δακτύλῳ ἑπτάκις ἔναντι Κυρίου.
\VS{17}Τὸ δὲ καταλειφθὲν ἔλαιον τὸ ὂν ἐν τῇ χειρὶ, ἐπιθήσει ὁ ἱερεὺς ἐπὶ τὸν λοβὸν τοῦ ὠτὸς τοῦ καθαριζομένου τοῦ δεξιοῦ, καὶ ἐπὶ τὸ ἄκρον τῆς χειρὸς τῆς δεξιᾶς, καὶ ἐπὶ τὸ ἄκρον τοῦ ποδὸς τοῦ δεξιοῦ, ἐπὶ τὸν τόπον τοῦ αἵματος τοῦ τῆς πλημμελείας.
\VS{18}Τὸ δὲ καταλειφθὲν ἔλαιον τὸ ἐπὶ τῆς χειρὸς τοῦ ἱερέως, ἐπιθήσει ὁ ἱερεὺς ἐπὶ τὴν κεφαλὴν τοῦ καθαρισθέντος· καὶ ἐξιλάσεται περὶ αὐτοῦ ὁ ἱερεὺς ἔναντι Κυρίου.
\VS{19}Καὶ ποιήσει ὁ ἱερεὺς τὸ περὶ τῆς ἁμαρτίας, καὶ ἐξιλάσεται ὁ ἱερεὺς περὶ τοῦ καθαριζομένου ἀπὸ τῆς ἁμαρτίας αὐτοῦ· καὶ μετὰ τοῦτο σφάξει ὁ ἱερεὺς τὸ ὁλοκαύτωμα.
\VS{20}Καὶ ἀνοίσει ὁ ἱερεὺς τὸ ὁλοκαύτωμα, καὶ τὴν θυσίαν ἐπὶ τὸ θυσιαστήριον ἔναντι κυρίου· καὶ ἐξιλάσεται περὶ αὐτοῦ ὁ ἱερεὺς, καὶ καθαρισθήσεται.
\VS{21}Ἐὰν δὲ πένηται, καὶ ἡ χεὶρ αὐτοῦ μὴ εὑρίσκῃ, λήψεται ἀμνὸν ἕνα εἰς ὃ ἐπλημμέλησεν εἰς ἀφαίρεμα, ὥστε ἐξιλάσασθαι περὶ αὐτοῦ, καὶ δέκατον σεμιδάλεως πεφυραμένης ἐν ἐλαίῳ εἰς θυσίαν, καὶ κοτύλην ἐλαίου μίαν,
\VS{22}καὶ δύο τρυγόνας, ἢ δύο νοσσοὺς περιστερῶν, ὅσα εὗρεν ἡ χεὶρ αὐτοῦ, καὶ ἔσται ἡ μία περὶ ἁμαρτίας, καὶ ἡ μία εἰς ὁλοκαύτωμα.
\VS{23}Καὶ προσοίσει αὐτὰ τῇ ἡμέρᾳ τῇ ὀγδόῃ, εἰς τὸ καθαρίσαι αὐτὸν, πρὸς τὸν ἱερέα, ἐπὶ τὴν θύραν τῆς σκηνῆς τοῦ μαρτυρίου ἔναντι Κυρίου.
\VS{24}Καὶ λαβὼν ὁ ἱερεὺς τὸν ἀμνὸν τῆς πλημμελείας, καὶ τὴν κοτύλην τοῦ ἐλαίου, ἐπιθήσει αὐτὰ ἐπίθεμα ἔναντι Κυρίου.
\VS{25}Καὶ σφάξει τὸν ἀμνὸν τὸν τῆς πλημμελείας, καὶ λήψεται ὁ ἱερεὺς ἀπὸ τοῦ αἵματος τοῦ τῆς πλημμελείας, καὶ ἐπιθήσει ἐπὶ τὸν λοβὸν τοῦ ὠτὸς τοῦ καθαριζομένου τοῦ δεξιοῦ, καὶ ἐπὶ τὸ ἄκρον τῆς χειρὸς τῆς δεξιᾶς, καὶ ἐπὶ τὸ ἄκρον τοῦ ποδὸς τοῦ δεξιοῦ.
\VS{26}Καὶ ἀπὸ τοῦ ἐλαίου ἐπιχεεῖ ὁ ἱερεὺς ἐπὶ τὴν χεῖρα τοῦ ἱερέως τὴν ἀριστεράν.
\VS{27}Καὶ ῥανεῖ ὁ ἱερεὺς τῷ δακτύλῳ τῷ δεξιῷ ἀπὸ τοῦ ἐλαίου τοῦ ἐν τῇ χειρὶ αὐτοῦ τῇ ἀριστερᾷ ἑπτάκις ἔναντι Κυρίου.
\VS{28}Καὶ ἐπιθήσει ὁ ἱερεὺς ἀπὸ τοῦ ἐλαίου τοῦ ἐπὶ τῆς χειρὸς αὐτοῦ ἐπὶ τὸν λοβὸν τοῦ ὠτὸς τοῦ καθαριζομένου τοῦ δεξιοῦ, καὶ ἐπὶ τὸ ἄκρον τῆς χειρὸς αὐτοῦ τῆς δεξιᾶς, καὶ ἐπὶ τὸ ἄκρον τοῦ ποδὸς αὐτοῦ τοῦ δεξιοῦ, ἐπὶ τὸν τόπον τοῦ αἵματος τοῦ τῆς πλημμελείας.
\VS{29}Τὸ δὲ καταλειφθὲν ἀπὸ τοῦ ἐλαίου τὸ ὂν ἐπὶ τῆς χειρὸς τοῦ ἱερέως, ἐπιθήσει ἐπὶ τὴν κεφαλὴν τοῦ καθαρισθέντος· καὶ ἐξιλάσεται περὶ αὐτοῦ ὁ ἱερεὺς ἔναντι Κυρίου.
\par }{\PP \VS{30}Καὶ ποιήσει μίαν ἀπὸ τῶν τρυγόνων ἢ ἀπὸ τῶν νοσσῶν τῶν περιστερῶν, καθότι εὗρεν αὐτοῦ ἡ χεὶρ,
\VS{31}τὴν μίαν περὶ ἁμαρτίας, καὶ τὴν μίαν εἰς ὁλοκαύτωμα σὺν τῇ θυσίᾳ· καὶ ἐξιλάσεται ὁ ἱερεὺς περὶ τοῦ καθαριζομένου ἔναντι Κυρίου.
\VS{32}Οὗτος ὁ νόμος ἐν ᾧ ἐστιν ἡ ἁφὴ τῆς λέπρας, καὶ τοῦ μὴ εὑρίσκοντος τῇ χειρὶ εἰς τὸν καθαρισμὸν αὐτοῦ.
\par }{\PP \VS{33}Καὶ ἐλάλησε Κύριος πρὸς Μωυσῆν καὶ Ἀαρὼν, λέγων,
\VS{34}ὡς ἂν εἰσέλθητε εἰς τὴν γῆν τῶν Χαναναίων, ἣν ἐγὼ δίδωμι ὑμῖν ἐν κτήσει, καὶ δώσω ἁφὴν λέπρας ἐν ταῖς οἰκίαις τῆς γῆς τῆς ἐγκτήτου ὑμῖν·
\VS{35}καὶ ἥξει τίνος αὐτοῦ ἡ οἰκία, καὶ ἀναγγελεῖ τῷ ἱερεῖ, λέγων, ὥσπερ ἁφὴ ἑώραταί μοι ἐν τῇ οἰκίᾳ.
\VS{36}Καὶ προστάξει ὁ ἱερεὺς ἀποσκευάσαι τὴν οἰκίαν, πρὸ τοῦ εἰσελθόντα τὸν ἱερέα ἰδεῖν τὴν ἁφὴν, καὶ οὐ μὴ ἀκάθαρτα γένηται ὅσα ἂν ᾖ ἐν τῇ οἰκίᾳ· καὶ μετὰ ταῦτα εἰσελεύσεται ὁ ἱερεὺς καταμαθεῖν τὴν οἰκίαν.
\VS{37}Καὶ ὄψεται τὴν ἁφὴν, καὶ ἰδοὺ ἡ ἁφὴ ἐν τοῖς τοίχοις τῆς οἰκίας, κοιλάδας χλωριζούσας, ἢ πυῤῥιζούσας, καὶ ἡ ὄψις αὐτῶν ταπεινοτέρα τῶν τοίχων.
\VS{38}Καὶ ἐξελθὼν ὁ ἱερεὺς ἐκ τῆς οἰκίας ἐπὶ τὴν θύραν τῆς οἰκίας, καὶ ἀφοριεῖ ὁ ἱερεὺς τὴν οἰκίαν ἑπτὰ ἡμέρας.
\VS{39}Καὶ ἐπανήξει ὁ ἱερεὺς τῇ ἡμέρᾳ τῇ ἑβδόμῃ, καὶ ὄψεται τὴν οἰκίαν, καὶ ἰδοὺ διεχύθη ἡ ἁφὴ ἐν τοῖς τοίχοις τῆς οἰκίας.
\VS{40}Καὶ προστάξει ὁ ἱερεὺς, καὶ ἐξελοῦσι τοὺς λίθους ἐν οἷς ἐστιν ἡ ἁφὴ, καὶ ἐκβαλοῦσιν αὐτοὺς ἔξω τῆς πόλεως εἰς τόπον ἀκάθαρτον.
\VS{41}Καὶ τὴν οἰκίαν ἀποξύσουσιν ἔσωθεν κύκλῳ, καὶ ἐκχεοῦσι τὸν χοῦν τὸν ἀπεξυσμένον ἔξω τῆς πόλεως εἰς τόπον ἀκάθαρτον.
\VS{42}Καὶ λήψονται λίθους ἀπεξυσμένους ἑτέρους, καὶ ἀντιθήσουσιν ἀντὶ τῶν λίθων· καὶ χοῦν ἕτερον λήψονται, καὶ ἐξαλείψουσι τὴν οἰκίαν.
\VS{43}Ἐὰν δὲ ἐπέλθῃ πάλιν ἡ ἁφὴ, καὶ ἀνατείλῃ ἐν τῇ οἰκίᾳ μετὰ τὸ ἐξελεῖν τοὺς λίθους, καὶ μετὰ τὸ ἀποξυσθῆναι τὴν οἰκίαν, καὶ μετὰ τὸ ἐξαλειφθῆναι,
\VS{44}καὶ εἰσελεύσεται ὁ ἱερεὺς, καὶ ὄψεται εἰ διακέχυται ἡ ἁφὴ ἐν τῇ οἰκίᾳ, λέπρα ἔμμονός ἐστιν ἐν τῇ οἰκίᾳ, ἀκάθαρτός ἐστι.
\VS{45}Καὶ καθελοῦσι τὴν οἰκίαν, καὶ τὰ ξύλα αὐτῆς, καὶ τοὺς λίθους αὐτῆς, καὶ πάντα τὸν χοῦν ἐξοίσουσιν ἔξω τῆς πόλεως εἰς τόπον ἀκάθαρτον.
\VS{46}Καὶ ὁ εἰσπορευόμενος εἰς τὴν οἰκίαν πάσας τὰς ἡμέρας, ἃς ἀφωρισμένη ἐστὶν, ἀκάθαρτος ἔσται ἕως ἑσπέρας·
\VS{47}Καὶ ὁ κοιμώμενος ἐν τῇ οἰκίᾳ, πλυνεῖ τὰ ἱμάτια αὐτοῦ, καὶ ἀκάθαρτος ἔσται ἕως ἑσπέρας· καὶ ὁ ἔσθων ἐν τῇ οἰκίᾳ, πλυνεῖ τὰ ἱμάτια αὐτοῦ, καὶ ἀκάθαρτος ἔσται ἕως ἑσπέρας.
\par }{\PP \VS{48}Ἐὰν δὲ παραγενόμενος εἰσέλθῃ ὁ ἱερεὺς καὶ ἴδῃ, καὶ ἰδοὺ οὐ διαχύσει οὐ διαχεῖται ἡ ἁφὴ ἐν τῇ οἰκίᾳ μετὰ τὸ ἐξαλειφθῆναι τὴν οἰκίαν, καὶ καθαριεῖ ὁ ἱερεὺς τὴν οἰκίαν, ὅτι ἰάθη ἡ ἁφή.
\VS{49}Καὶ λήψεται ἀφαγνίσαι τὴν οἰκίαν, δύο ὀρνίθια ζῶντα καθαρὰ, καὶ ξύλον κέδρινον, καὶ κεκλωσμένον κόκκινον, καὶ ὕσσωπον.
\VS{50}Καὶ σφάξει τὸ ὀρνίθιον τὸ ἓν εἰς σκεῦος ὀστράκινον ἐφʼ ὕδατι ζῶντι·
\VS{51}Καὶ λήψεται τὸ ξύλον τὸ κέδρινον, καὶ τὸ κεκλωσμένον κόκκινον, καὶ τὸν ὕσσωπον, καὶ τὸ ὀρνίθιον τὸ ζῶν· καὶ βάψει αὐτὸ εἰς τὸ αἷμα τοῦ ὀρνιθίου τοῦ ἐσφαγμενου ἐφʼ ὕδατι ζῶντι· καὶ περιῤῥανεῖ ἐν αὐτοῖς ἐπὶ τὴν οἰκίαν ἑπτάκις.
\VS{52}Καὶ ἀφαγνιεῖ τὴν οἰκίαν ἐν τῷ αἵματι τοῦ ὀρνιθίου, καὶ ἐν τῷ ὕδατι τῷ ζῶντι, καὶ ἐν τῷ ὀρνιθίῳ τῷ ζῶντι, καὶ ἐν τῷ ξύλῳ τῷ κεδρίνῳ, καὶ ἐν τῷ ὑσσώπῳ, καὶ ἐν τῷ κεκλωσμένῳ κοκκίνῳ.
\VS{53}Καὶ ἐξαποστελεῖ τὸ ὀρνίθιον τὸ ζῶν ἔξω τῆς πόλεως εἰς τὸ πεδίον, καὶ ἐξιλάσεται περὶ τῆς οἰκίας, καὶ καθαρὰ ἔσται.
\VS{54}Οὗτος ὁ νόμος κατὰ πᾶσαν ἁφὴν λέπρας, καὶ θραύσματος,
\VS{55}καὶ τῆς λέπρας ἱματίου, καὶ οἰκίας,
\VS{56}καὶ οὐλῆς, καὶ σημασίας, καὶ τοῦ αὐγάζοντος,
\VS{57}καὶ τοῦ ἐξηγήσασθαι ᾗ ἡμέρᾳ ἀκάθαρτον, καὶ ᾗ ἡμέρᾳ καθαρισθήσεται· οὗτος ὁ νόμος τῆς λέπρας.

\par }\Chap{15}{\PP \VerseOne{1}Καὶ ἐλάλησε Κύριος πρὸς Μωυσῆν καὶ Ἀαρὼν, λέγων,
\VS{2}λάλησον τοῖς υἱοῖς Ἰσραὴλ, καὶ ἐρεῖς αὐτοῖς, ἀνδρὶ ἀνδρὶ ᾧ ἐὰν γένηται ῥύσις ἐκ τοῦ σώματος αὐτοῦ, ἡ ῥύσις αὐτοῦ ἀκάθαρτός ἐστι.
\VS{3}Καὶ οὗτος ὁ νόμος τῆς ἀκαθαρσίας αὐτοῦ· ῥέων γόνον ἐκ σώματος αὐτοῦ, ἐκ τῆς ῥύσεως, ἧς συνέστηκε τὸ σῶμα αὐτοῦ διὰ τῆς ῥύσεως, αὕτη ἡ ἀκαθαρσία αὐτοῦ ἐν αὐτῷ· πᾶσαι αἱ ἡμέραι ῥύσεως σώματος αὐτοῦ, ᾗ συνέστηκε τὸ σῶμα αὐτοῦ διὰ τῆς ῥύσεως, ἀκαθαρσία αὐτοῦ ἐστι.
\VS{4}Πᾶσα κοίτη ἐφʼ ἧς ἂν κοιμηθῇ ἐπʼ αὐτῆς ὁ γονοῤῥυὴς, ἀκάθαρτός ἐστι, καὶ πᾶν σκεῦος ἐφʼ ὃ ἂν καθίσῃ ἐπʼ αὐτὸ ὁ γονοῤῤυὴς, ἀκάθαρτον ἔσται.
\VS{5}Καὶ ἄνθρωπος, ὃς ἐὰν ἅψηται τῆς κοίτης αὐτοῦ, πλυνεῖ τὰ ἱμάτια αὐτοῦ, καὶ λούσεται ὕδατι, καὶ ἀκάθαρτος ἔσται ἕως ἑσπέρας.
\VS{6}Καὶ ὁ καθήμενος ἐπὶ τοῦ σκεύους ἐφʼ ὃ ἂν καθίσῃ ὁ γονοῤῥυὴς, πλυνεῖ τὰ ἱμάτια αὐτοῦ, καὶ λούσεται ὕδατι, καὶ ἀκάθαρτος ἔσται ἕως ἑσπέρας.
\VS{7}Καὶ ὁ ἁπτόμενος τοῦ χρωτὸς τοῦ γονοῤῥυοῦς, πλυνεῖ τὰ ἱμάτια, καὶ λούσεται ὕδατι, καὶ ἀκάθαρτος ἔσται ἕως ἑσπέρας.
\VS{8}Ἐὰν δὲ προσσιελίσῃ ὁ γονοῤῥυὴς ἐπὶ τὸν καθαρὸν, πλυνεῖ τὰ ἱμάτια αὐτοῦ, καὶ λούσεται ὕδατι, καὶ ἀκάθαρτος ἔσται ἕως ἑσπέρας.
\VS{9}Καὶ πᾶν ἐπίσαγμα ὄνου, ἐφʼ ὃ ἂν ἐπιβῇ ἐπʼ αὐτὸ ὁ γονοῤῥυὴς, ἀκάθαρτον ἔσται ἕως ἑσπέρας.
\VS{10}Καὶ πᾶς ὁ ἁπτόμενος ὅσα ἂν ᾖ ὑποκάτω αὐτοῦ, ἀκάθαρτος ἔσται ἕως ἑσπέρας· καὶ ὁ αἴρων αὐτὰ, πλυνεῖ τὰ ἱμάτια αὐτοῦ, καὶ λούσεται ὕδατι, καὶ ἀκάθαρτος ἔσται ἕως ἑσπέρας.
\VS{11}Καὶ ὅσων ἐὰν ἅψηται ὁ γονοῤῥυὴς, καὶ τὰς χεῖρας οὐ νένιπται ὕδατι, πλυνεῖ τὰ ἱμάτια, καὶ λούσεται τὸ σῶμα ὕδατι, καὶ ἀκάθαρτος ἔσται ἕως ἑσπέρας.
\VS{12}Καὶ σκεῦος ὀστράκινον οὗ ἂν ἅψηται ὁ γονοῤῥυὴς, συντριβήσεται· καὶ σκεῦος ξύλινον νιφήσεται ὕδατι, καὶ καθαρὸν ἔσται.
\VS{13}Ἐὰν δὲ καθαρισθῇ ὁ γονοῤῥυὴς ἐκ τῆς ῥύσεως αὐτοῦ, καὶ ἐξαριθμηθήσεται αὐτῷ ἑπτὰ ἡμέρας εἰς τὸν καθαρισμὸν αὐτοῦ, καὶ πλυνεῖ τὰ ἱμάτια αὐτοῦ, καὶ λούσεται τὸ σῶμα ὕδατι, καὶ καθαρὸς ἔσται.
\VS{14}Καὶ τῇ ἡμέρᾳ τῇ ὀγδόῃ λήψεται ἑαυτῷ δύο τρυγόνας, ἢ δύο νοσσοὺς περιστερῶν, καὶ οἴσει αὐτὰ ἔναντι Κυρίου ἐπὶ τὰς θύρας τῆς σκηνῆς τοῦ μαρτυρίου, καὶ δώσει αὐτὰ τῷ ἱερεῖ.
\VS{15}Καὶ ποιήσει αὐτὰ ὁ ἱερεὺς μίαν περὶ ἁμαρτίας, καὶ μίαν εἰς ὁλοκαύτωμα· καὶ ἐξιλάσεται περὶ αὐτοῦ ὁ ἱερεὺς ἔναντι Κυρίου ἀπὸ τῆς ῥύσεως αὐτοῦ.
\par }{\PP \VS{16}Καὶ ἄνθρωπος ᾧ ἂν ἐξέλθῃ ἐξ αὐτοῦ κοίτη σπέρματος, καὶ λούσεται ὕδατι πᾶν τὸ σῶμα αὐτοῦ, καὶ ἀκάθαρτος ἔσται ἕως ἑσπέρας.
\VS{17}Καὶ πᾶν ἱμάτιον, καὶ πᾶν δέρμα ἐφʼ ὃ ἂν ᾖ ἐπʼ αὐτὸ κοίτη σπέρματος, καὶ πλυθήσεται ὕδατι, καὶ ἀκάθαρτον ἔσται ἕως ἑσπέρας.
\VS{18}Καὶ γυνὴ ἐὰν κοιμηθῇ ἀνὴρ μετʼ αὐτῆς κοίτην σπέρματος, καὶ λούσονται ὕδατι, καὶ ἀκάθαρτοι ἔσονται ἕως ἑσπέρας.
\VS{19}Καὶ γυνὴ ἥτις ἂν ᾖ ῥέουσα αἵματι, καὶ ἔσται ἡ ῥύσις αὐτῆς ἐν τῷ σώματι αὐτῆς, ἑπτὰ ἡμέρας ἔσται ἐν τῇ ἀφέδρῳ αὐτῆς· πᾶς ὁ ἁπτόμενος αὐτῆς, ἀκάθαρτος ἔσται ἕως ἑσπέρας.
\VS{20}Καὶ πᾶν ἐφʼ ὃ ἂν κοιτάζηται ἐπʼ αὐτὸ ἐν τῇ ἀφέδρῳ αὐτῆς, ἀκάθαρτον ἔσται· καὶ πᾶν ἐφʼ ὃ ἂν ἐπικαθίσῃ ἐπʼ αὐτὸ, ἀκάθαρτον ἔσται.
\VS{21}Καὶ πᾶς ὃς ἂν ἅψηται τῆς κοίτης αὐτῆς, πλυνεῖ τὰ ἱμάτια αὐτοῦ, καὶ λούσεται τὸ σῶμα αὐτοῦ ὕδατι, καὶ ἀκάθαρτος ἔσται ἕως ἑσπέρας.
\VS{22}Καὶ πᾶς ὁ ἁπτόμενος παντὸς σκεύους οὗ ἐὰν καθίσῃ ἐπʼ αὐτὸ, πλυνεῖ τὰ ἱμάτια αὐτοῦ, καὶ λούσεται ὕδατι, καὶ ἀκάθαρτος ἔσται ἕως ἑσπέρας.
\VS{23}Ἐὰν δὲ ἐν τῇ κοίτῃ αὐτῆς οὔσης, ἢ ἐπὶ τοῦ σκεύους οὗ ἐὰν καθίσῃ ἐπʼ αὐτῷ ἐν τῷ ἅπτεσθαι αὐτὸν αὐτῆς, ἀκάθαρτος ἔσται ἕως ἑσπέρας.
\par }{\PP \VS{24}Ἐὰν δὲ κοίτῃ κοιμηθῇ τις μετʼ αὐτῆς, καὶ γένηται ἡ ἀκαθαρσία αὐτῆς ἐπʼ αὐτῷ, ἀκάθαρτος ἔσται ἑπτὰ ἡμέρας· καὶ πᾶσα κοίτη ἐφʼ ᾗ ἂν κοιμηθῇ ἐπʼ αὐτῇ, ἀκάθαρτος ἔσται.
\VS{25}Καὶ γυνὴ ἐὰν ῥέῃ ῥύσει αἵματος ἡμέρας πλείους, οὐκ ἐν καιρῷ τῆς ἀφέδρου αὐτῆς, ἐὰν καὶ ῥέῃ μετὰ τὴν ἄφεδρον αὐτῆς, πᾶσαι αἱ ἡμέραι ῥύσεως ἀκαθαρσίας αὐτῆς, καθάπερ αἱ ἡμέραι τῆς ἀφέδρου αὐτῆς, ἔσται ἀκάθαρτος.
\VS{26}Καὶ πᾶσα κοίτη ἐφʼ ἧς ἂν κοιμηθῇ ἐπʼ αὐτῆς πάσας τὰς ἡμέρας τῆς ῥύσεως, κατὰ τὴν κοίτην τῆς ἀφέδρου, ἔσται αὐτῇ· καὶ πᾶν σκεῦος ἐφʼ ὃ ἂν καθίσῃ ἐπʼ αὐτὸ, ἀκάθαρτον ἔσται κατὰ τὴν ἀκαθαρσίαν τῆς ἀφέδρου.
\VS{27}Πᾶς ὁ ἁπτόμενος αὐτῆς ἀκάθαρτος ἔσται, καὶ πλυνεῖ τὰ ἱμάτια καὶ λούσεται τὸ σῶμα ὕδατι, καὶ ἀκάθαρτος ἔσται ἕως ἑσπέρας.
\VS{28}Ἐὰν δὲ καθαρισθῇ ἀπὸ τῆς ῥύσεως, καὶ ἐξαριθμήσεται αὐτῇ ἑπτὰ ἡμέρας, καὶ μετὰ ταῦτα καθαρισθήσεται.
\VS{29}Καὶ τῇ ἡμέρᾳ τῇ ὀγδόῃ λήψεται αὑτῇ δύο τρυγόνας, ἢ δύο νοσσοὺς περιστερῶν, καὶ οἴσει αὐτὰ πρὸς τὸν ἱερέα ἐπὶ τὴν θύραν τῆς σκηνῆς τοῦ μαρτυρίου.
\VS{30}Καὶ ποιήσει ὁ ἱερεὺς τὴν μίαν περὶ ἁμαρτίας, καὶ τὴν μίαν εἰς ὁλοκαύτωμα· καὶ ἐξιλάσεται περὶ αὐτῆς ὁ ἱερεὺς ἔναντι Κυρίου ἀπὸ ῥύσεως ἀκαθαρσίας αὐτῆς.
\par }{\PP \VS{31}Καὶ εὐλαβεῖς ποιήσετε τοὺς υἱοὺς Ἰσραὴλ ἀπὸ τῶν ἀκαθαρσιῶν αὐτῶν· καὶ οὐκ ἀποθανοῦνται διὰ τὴν ἀκαθαρσίαν αὐτῶν, ἐν τῷ μιαίνειν αὐτοὺς τὴν σκηνήν μου τὴν ἐν αὐτοῖς.
\VS{32}Οὗτος ὁ νόμος τοῦ γονοῤῥυοῦς· καὶ ἐάν τινι ἐξέλθῃ ἐξ αὐτοῦ κοίτη σπέρματος, ὥστε μιανθῆναι ἐν αὐτῇ,
\VS{33}καὶ τῇ αἱμοῤῥοούσῃ ἐν τῇ ἀφέδρῳ αὐτῆς, καὶ ὁ γονοῤῥυὴς ἐν τῇ ῥύσει αὐτοῦ τῷ ἄρσενι ἢ τῇ θηλείᾳ, καὶ τῷ ἀνδρὶ, ὃς ἂν κοιμηθῇ μετὰ ἀποκαθημένης.

\par }\Chap{16}{\PP \VerseOne{1}Καὶ ἐλάλησε Κύριος πρὸς Μωυσῆν, μετὰ τὸ τελευτῆσαι τοὺς δύο υἱοὺς Ἀαρὼν ἐν τῷ προσάγειν αὐτοὺς πῦρ ἀλλότριον ἔναντι Κυρίου, καὶ ἐτελεύτησαν.
\VS{2}Καὶ εἶπε Κύριος πρὸς Μωυσῆν, λάλησον πρὸς Ἀαρὼν τὸν ἀδελφόν σου, καὶ μὴ εἰσπορευέσθω πᾶσαν ὥραν εἰς τὸ ἅγιον ἐσώτερον τοῦ καταπετάσματος εἰς πρόσωπον τοῦ ἱλαστηρίου, ὅ ἐστιν ἐπὶ τῆς κιβωτοῦ τοῦ μαρτυρίου, καὶ οὐκ ἀποθανεῖται· ἐν γὰρ νεφέλῃ ὀφθήσομαι ἐπὶ τοῦ ἱλαστηρίου.
\VS{3}Οὕτως εἰσελεύσεται Ἀαρὼν εἰς τὸ ἅγιον· ἐν μόσχῳ ἐκ βοῶν περὶ ἁμαρτίας, καὶ κριὸν εἰς ὁλοκαύτωμα.
\VS{4}Καὶ χιτῶνα λινοῦν ἡγιασμένον ἐνδύσεται, καὶ περισκελὲς λινοῦν ἔσται ἐπὶ τοῦ χρωτὸς αὐτοῦ, καὶ ζώνῃ λινῇ ζώσεται, καὶ κίδαριν λινῆν περιθήσεται, ἱμάτια ἅγιά ἐστι· καὶ λούσεται ὕδατι πᾶν τὸ σῶμα αὐτοῦ καὶ ἐνδύσεται αὐτά.
\VS{5}Καὶ παρὰ τῆς συναγωγῆς τῶν υἱῶν Ἰσραὴλ λήψεται δύο χιμάρους ἐξ αἰγῶν περὶ ἁμαρτίας, καὶ κριὸν ἕνα εἰς ὁλοκαύτωμα.
\VS{6}Καὶ προσάξει Ἀαρὼν τὸν μόσχον τὸν περὶ τῆς ἁμαρτίας αὐτοῦ, καὶ ἐξιλάσεται περὶ αὐτοῦ, καὶ τοῦ οἴκου αὐτοῦ.
\VS{7}Καὶ λήψεται τοὺς δύο χιμάρους, καὶ στήσει αὐτοὺς ἔναντι Κυρίου παρὰ τὴν θύραν τῆς σκηνῆς τοῦ μαρτυρίου.
\VS{8}Καὶ ἐπιθήσει Ἀαρὼν ἐπὶ τοὺς δύο χιμάρους κλῆρους· κλῆρον ἕνα τῷ Κυρίῳ, καὶ κλῆρον ἕνα τῷ ἀποπομπαίῳ.
\VS{9}Καὶ προσάξει Ἀαρὼν τὸν χίμαρον ἐφʼ ὃν ἐπῆλθεν ἐπʼ αὐτὸν ὁ κλῆρος τῷ Κυρίῳ, καὶ προσοίσει περὶ ἁμαρτίας.
\VS{10}Καὶ τὸν χίμαρον, ἐφʼ ὃν ἐπῆλθεν ἐπʼ αὐτὸν ὁ κλῆρος τοῦ ἀποπομπαίου, στήσει αὐτὸν ζῶντα ἔναντι Κυρίου, τοῦ ἐξιλάσασθαι ἐπʼ αὐτοῦ, ὥστε ἀποστεῖλαι αὐτὸν εἰς τὴν ἀποπομπήν, καὶ ἀφήσει αὐτὸν εἰς τὴν ἔρημον.
\VS{11}Καὶ προσάξει Ἀαρὼν τὸν μόσχον τὸν περὶ τῆς ἁμαρτίας αὑτοῦ, καὶ ἐξιλάσεται περὶ ἑαυτοῦ, καὶ τοῦ οἴκου· καὶ σφάξει τὸν μόσχον περὶ τῆς ἁμαρτίας αὐτοῦ.
\VS{12}Καὶ λήψεται τὸ πυρεῖον πλῆρες ἀνθράκων πυρὸς ἀπὸ τοῦ θυσιαστηρίου, τοῦ ἀπέναντι Κυρίου· καὶ πλήσει τὰς χεῖρας θυμιάματος συνθέσεως λεπτῆς, καὶ εἰσοίσει ἐσώτερον τοῦ καταπετάσματος.
\VS{13}Καὶ ἐπιθήσει τὸ θυμίαμα ἐπὶ τὸ πῦρ ἔναντι Κυρίου· καὶ καλύψει ἡ ἀτμὶς τοῦ θυμιάματος τὸ ἱλαστήριον τὸ ἐπὶ τῶν μαρτυρίων, καὶ οὐκ ἀποθανεῖται.
\VS{14}Καὶ λήψεται ἀπὸ τοῦ αἵματος τοῦ μόσχου, καὶ ῥανεῖ τῷ δακτύλῳ ἐπὶ τὸ ἱλαστήριον κατὰ ἀνατολάς· κατὰ πρόσωπον τοῦ ἱλαστηρίου ῥανεῖ ἑπτάκις ἀπὸ τοῦ αἵματος τῷ δακτύλῳ.
\par }{\PP \VS{15}Καὶ σφάξει τὸν χίμαρον τὸν περὶ ἁμαρτίας, τὸν περὶ τοῦ λαοῦ, ἔναντι Κυρίου· καὶ εἰσοίσει τοῦ αἵματος αὐτοῦ ἐσώτερον τοῦ καταπετάσματος, καὶ ποιήσει τὸ αἷμα αὐτοῦ, ὃν τρόπον ἐποίησε τὸ αἷμα τοῦ μόσχου· καὶ ῥανεῖ τὸ αἷμα αὐτοῦ ἐπὶ τὸ ἱλαστήριον, κατὰ πρόσωπον τοῦ ἱλαστηρίου.
\VS{16}Καὶ ἐξιλάσεται τὸ ἅγιον ἀπὸ τῶν ἀκαθαρσιῶν τῶν υἱῶν Ἰσραὴλ, καὶ ἀπὸ τῶν ἀδικημάτων αὐτῶν περὶ πασῶν τῶν ἁμαρτιῶς αὐτῶν· καὶ οὕτω ποιήσει τῇ σκηνῇ τοῦ μαρτυρίου τῇ ἐκτισμένῃ ἐν αὐτοῖς ἐν μέσῳ τῆς ἀκαθαρσίας αὐτῶν.
\VS{17}Καὶ πᾶς ἄνθρώπος οὐκ ἔσται ἐν τῇ σκηνῇ τοῦ μαρτυρίου, εἰσπορευομένου αὐτοῦ ἐξιλάσασθαι ἐν τῷ ἁγίῳ, ἕως ἂν ἐξέλθῃ· καὶ ἐξιλάσεται περὶ ἑαυτοῦ, καὶ τοῦ οἴκου αὐτοῦ, καὶ περὶ πάσης συναγωγῆς υἱῶν Ἰσραήλ.
\VS{18}Καὶ ἐξελεύσεται ἐπὶ τὸ θυσιαστήριον τὸ ὂν ἀπέναντι Κυρίου, καὶ ἐξιλάσεται ἐπʼ αὐτοῦ· καὶ λήψεται ἀπὸ τοῦ αἵματος τοῦ μόσχου, καὶ ἀπὸ τοῦ αἵματος τοῦ χιμάρου, καὶ ἐπιθήσει ἐπὶ τὰ κέρατα τοῦ θυσιαστηρίου κύκλῳ.
\VS{19}Καὶ ῥανεῖ ἐπʼ αὐτὸ ἀπὸ τοῦ αἵματος τῷ δακτύλῳ ἑπτάκις, καὶ καθαριεῖ αὐτό, καὶ ἁγιάσει αὐτὸ ἀπὸ τῶν ἀκαθαρσιῶν τῶν υἱῶν Ἰσραήλ.
\VS{20}Καὶ συντελέσει ἐξιλασκόμενος τὸ ἅγιον, καὶ τὴν σκηνὴν τοῦ μαρτυρίου, καὶ τὸ θυσιαστήριον, καὶ περὶ τῶν ἱερέων καθαριεῖ· καὶ προσάξει τὸν χίμαρον τὸν ζῶντα.
\VS{21}Καὶ ἐπιθήσει Ἀαρὼν τὰς χεῖρας αὐτοῦ ἐπὶ τὴν κεφαλὴν τοῦ χιμάρου τοῦ ζῶντος, καὶ ἐξαγορεύσει ἐπʼ αὐτοῦ πάσας τὰς ἀνομίας τῶν υἱῶν Ἰσραὴλ, καὶ πάσας τὰς ἀδικίας αὐτῶν, καὶ πάσας τὰς ἁμαρτίας αὐτῶν· καὶ ἐπιθήσει αὐτὰς ἐπὶ τὴν κεφαλὴν τοῦ χιμάρου τοῦ ζῶντος· καὶ ἐξαποστελεῖ ἐν χειρὶ ἀνθρώπου ἑτοιμου εἰς τὴν ἔρημον.
\VS{22}Καὶ λήψεται ὁ χίμαρος ἐφʼ ἑαυτῷ τὰς ἀδικίας αὐτῶν εἰς γῆν ἄβατον· καὶ ἐξαποστελεῖ τὸν χίμαρον εἰς τὴν ἔρημον.
\VS{23}Καὶ εἰσελεύσεται Ἀαρὼν εἰς τὴν σκηνὴν τοῦ μαρτυρίου, καὶ ἐκδύσεται τὴν στολὴν τὴν λινῆν, ἣν ἐνδεδύκει, εἰσπορευομένου αὐτοῦ εἰς τὸ ἅγιον, καὶ ἀποθήσει αὐτὴν ἐκεῖ.
\VS{24}Καὶ λούσεται τὸ σῶμα αὐτοῦ ὕδατι ἐν τόπῳ ἁγίῳ, καὶ ἐνδύσεται τὴν στολὴν αὐτοῦ, καὶ ἐξελθὼν ποιήσει τὸ ὁλοκαύτωμα αὐτοῦ καὶ τὸ ὁλοκάρπωμα τοῦ λαοῦ, καὶ ἐξιλάσεται περὶ αὐτοῦ, καὶ περὶ τοῦ οἴκου αὐτοῦ, καὶ περὶ τοῦ λαοῦ, ὡς περὶ τῶν ἱερέων.
\VS{25}Καὶ τὸ στέαρ τὸ περὶ τῶν ἁμαρτιῶν ἀνοίσει ἐπὶ τὸ θυσιαστήριον.
\par }{\PP \VS{26}Καὶ ὁ ἐξαποστέλλων τὸν χίμαρον τὸν διεσταλμένον εἰς ἄφεσιν, πλυνεῖ τὰ ἱμάτια, καὶ λούσεται τὸ σῶμα αὐτοῦ ὕδατι, καὶ μετὰ ταῦτα εἰσελεύσεται εἰς τὴν παρεμβολήν.
\VS{27}Καὶ τὸν μόσχον τὸν περὶ τῆς ἁμαρτίας, καὶ τὸν χίμαρον τὸν περὶ τῆς ἁμαρτίας, ὧν τὸ αἷμα εἰσηνέχθη ἐξιλάσασθαι ἐν τῷ ἁγίῳ, ἐξοίσουσιν αὐτὰ ἔξω τῆς παρεμβολῆς, ταὶ κατακαύσουσιν αὐτὰ ἐν πυρὶ, καὶ τὰ δέρματα αὐτῶν καὶ τὰ κρέα αὐτῶν καὶ τὴν κόπρον αὐτῶν.
\VS{28}Ὁ δὲ κατακαίων αὐτὰ, πλυνεῖ τὰ ἱμάτια, καὶ λούσεται τὸ σῶμα αὐτοῦ ὕδατι, καὶ μετὰ ταῦτα εἰσελεύσεται εἰς τὴν παρεμβολήν.
\par }{\PP \VS{29}Καὶ ἔσται τοῦτο ὑμῖν νόμιμον αἰώνιον· ἐν τῷ μηνὶ τῷ ἑβδόμῳ, δεκάτῃ τοῦ μηνὸς, ταπεινώσετε τὰς ψυχὰς ὑμῶν, καὶ πᾶν ἔργον οὐ ποιήσετε ὁ αὐτόχθων, καὶ ὁ προσήλυτος ὁ προσκείμενος ἐν ὑμῖν.
\VS{30}Ἐν γὰρ τῇ ἡμέρᾳ ταύτῃ ἐξιλάσεται περὶ ὑμῶν, καθαρίσαι ὑμᾶς ἀπὸ πασῶν τῶν ἁμαρτιῶν ὑμῶν ἔναντι Κυρίου, καὶ καθαρισθήσεσθε.
\VS{31}Σάββατα σαββάτων ἀνάπαυσις αὕτη ἔσται ὑμῖν· καὶ ταπεινώσετε τὰς ψυχὰς ὑμῶν, νόμιμον αἰώνιον.
\VS{32}Ἐξιλάσεται ὁ ἱερεὺς, ὃν ἂν χρίσωσιν αὐτὸν, καὶ ὃν ἂν τελειώσωσι τὰς χεῖρας αὐτοῦ ἱερατεύειν μετὰ τὸν πατέρα αὐτοῦ· καὶ ἐνδύσεται τὴν στολὴν τὴν λινῆν, στολὴν ἁγίαν.
\VS{33}Καὶ ἐξιλάσεται τὸ ἅγιον τοῦ ἁγίου, καὶ τὴν σκηνὴν τοῦ μαρτυρίου, καὶ τὸ θυσιαστήριον ἐξιλάσεται, καὶ περὶ τῶν ἱερέων, καὶ περὶ πάσης συναγωγῆς ἐξιλάσεται.
\VS{34}Καὶ ἔσται τοῦτο ὑμῖν νόμιμον αἰώνιον ἐξιλάσκεσθαι περὶ τῶν νἱῶν Ἰσραὴλ ἀπὸ πασῶν τῶν ἁμαρτιῶν αὐτῶν· ἅπαξ τοῦ ἐνιαυτοῦ ποιηθήσεται, καθὰ συνέταξε Κύριος τῷ Μωυσῇ.

\par }\Chap{17}{\PP \VerseOne{1}Καὶ ἐλάλησε Κύριος πρὸς Μωυσῆν, λέγων,
\VS{2}λάλησον πρὸς Ἀαρὼν καὶ πρὸς τοὺς υἱοὺς αὐτοῦ, καὶ πρὸς πάντας υἱοὺς Ἰσραὴλ, καὶ ἐρεῖς πρὸς αὐτοὺς, τοῦτο τὸ ῥῆμα ὃ ἐνετείλατο Κύριος, λέγων,
\VS{3}ἄνθρωπος ἄνθρωπος τῶν υἱῶν Ἰσραὴλ, ἢ τῶν προσηλύτων τῶν προσκειμένων ἐν ὑμῖν, ὃς ἐὰν σφάξῃ μόσχον, ἢ πρόβατον, ἢ αἶγα ἐν τῇ παρεμβολῇ, καὶ ὃς ἂν σφάξῃ ἔξω τῆς παρεμβολῆς,
\VS{4}καὶ ἐπὶ τὴν θύραν τῆς σκηνῆς τοῦ μαρτυρίου μὴ ἐνέγκῃ, ὥστε ποιῆσαι αὐτὸ εἰς ὁλοκαύτωμα ἢ σωτήριον Κυρίῳ δεκτὸν εἰς ὀσμὴν εὐωδίας· καὶ ὃς ἂν σφάξῃ ἔξω, καὶ ἐπὶ τὴν θύραν τῆς σκηνῆς τοῦ μαρτυρίου μὴ ἐνέγκῃ αὐτὸ, ὥστε προσενέγκαι δῶρον τῷ Κυρίῳ ἀπέναντι τῆς σκηνῆς Κυρίου· καὶ λογισθήσεται τῷ ἀνθρώπῳ ἐκείνῳ αἷμα· αὶμα ἐξέχεεν· ἐξολοθρευθήσεται ἡ ψυχὴ ἐκείνη ἐκ τοῦ λαοῦ αὐτῆς.
\VS{5}Ὅπως ἀναφέρωσιν οἱ υἱοὶ Ἰσραὴλ τὰς θυσίας αὐτῶν, ὅσας ἂν αὐτοὶ σφάξουσιν ἐν τοῖς πεδίοις, καὶ οἴσουσι τῷ Κυρίῳ ἐπὶ τὰς θύρας τῆς σκηνῆς τοῦ μαρτυρίου πρὸς τὸν ἱερέα· καὶ θύσουσι θυσίαν σωτηρίου τῷ Κυρίῳ αὐτά.
\VS{6}Καὶ προσχεεῖ ὁ ἱερεὺς τὸ αἷμα ἐπὶ τὸ θυσιαστήριον κύκλῳ ἀπέναντι Κυρίου παρὰ τὰς θύρας τῆς σκηνῆς τοῦ μαρτυρίου· καὶ ἀνοίσει τὸ στέαρ εἰς ὀσμὴν εὐωδίας Κυρίῳ.
\par }{\PP \VS{7}Καὶ οὐ θύσουσιν ἔτι τὰς θυσίας αὐτῶν τοῖς ματαίοις, οἷς αὐτοὶ ἐκπορνεύουσιν ὀπίσω αὐτῶν· νόμιμον αἰώνιον ἔσται ὑμῖν εἰς τὰς γενεὰς ὑμῶν.
\VS{8}Καὶ ἐρεῖς πρὸς αὐτοὺς, ἄνθρωπος ἄνθρωπος τῶν υἱῶν Ἰσραὴλ, ἢ ἀπὸ τῶν υἱῶν τῶν προσηλύτων τῶν προσκειμένων ἐν ὑμῖν, ὃς ἂν ποιήσῃ ὁλοκαύτωμα ἢ θυσίαν,
\VS{9}καὶ ἐπὶ τὴν θύραν τῆς σκηνῆς τοῦ μαρτυρίου μὴ ἐνέγκῃ ποιῆσαι αὐτὸ τῷ Κυρίῳ, ἐξολοθρευθήσεται ὁ ἄνθρωπος ἐκεῖνος ἐκ τοῦ λαοῦ αὐτοῦ.
\VS{10}Καὶ ἄνθρωπος ἄνθρωπος τῶν υἱῶν Ἰσραὴλ, ἢ τῶν προσηλύτων τῶν προσκειμένων ἐν ὑμῖν, ὃς ἂν φάγῃ πᾶν αἷμα· καὶ ἐπιστήσω τὸ πρόσωπόν μου ἐπὶ τὴν ψυχὴν τὴν ἔσθουσαν τὸ αἷμα, καὶ ἀπολῶ αὐτὴν ἐκ τοῦ λαοῦ αὐτῆς.
\VS{11}Ἡ γὰρ ψυχὴ πάσης σαρκὸς αἷμα αὐτοῦ ἐστι· καὶ ἐγὼ δέδωκα αὐτὸ ὑμῖν ἐπὶ τοῦ θυσιαστηρίου ἐξιλάσκεσθαι περὶ τῶν ψυχῶν ὑμῶν· τὸ γὰρ αἷμα αὐτοῦ ἀντὶ ψυχῆς ἐξιλάσεται.
\VS{12}Διὰ τοῦτο εἴρηκα τοῖς υἱοῖς Ἰσραὴλ, πᾶσα ψυχὴ ἐξ ὑμῶν οὐ φάγεται αἷμα· καὶ ὁ προσήλυτος ὁ προσκείμενος ἐν ὑμῖν οὐ φάγεται αἷμα.
\VS{13}Καὶ ἄνθρωπος ἄνθρωπος τῶν υἱῶν Ἰσραὴλ, ἢ τῶν προσηλύτων τῶν προσκειμένων ἐν ὑμῖν, ὃς ἂν θηρεύσῃ θήρευμα θηρίον ἢ πετεινὸν, ὃ ἔσθεται, καὶ ἐκχεεῖ τὸ αἷμα, καὶ καλύψει αὐτὸ τῇ γῇ.
\VS{14}Ἡ γὰρ ψυχὴ πάσης σαρκὸς αἷμα αὐτοῦ ἐστι· καὶ εἶπα τοῖς υἱοῖς Ἰσραὴλ, αἷμα πάσης σαρκὸς οὐ φάγεσθε, ὅτι ἡ ψυχὴ πάσης σαρκὸς αἷμα αὐτοῦ ἐστί· πᾶς ὁ ἔσθων αὐτὸ, ἐξολοθρευθήσεται.
\VS{15}Καὶ πᾶσα ψυχὴ, ἥτις φάγεται θνησιμαῖον, ἢ θηριάλωτον ἐν τοῖς αὐτόχθοσιν, ἢ ἐν τοῖς προσηλύτοις, πλυνεῖ τὰ ἱμάτια αὐτοῦ, καὶ λούσεται ὕδατι, καὶ ἀκάθαρτος ἔσται ἕως ἑσπέρας, καὶ καθαρὸς ἔσται.
\VS{16}Ἐὰν δὲ μὴ πλύνῃ τὰ ἱμάτια, καὶ τὸ σῶμα μὴ λούσηται ὕδατι, καὶ λήψεται ἀνόμημα αὐτοῦ.

\par }\Chap{18}{\PP \VerseOne{1}Καὶ εἶπε Κύριος πρὸς Μωυσῆν, λέγων,
\VS{2}λάλησον τοῖς υἱοῖς Ἰσραὴλ, καὶ ἐρεῖς πρὸς αὐτοὺς, ἐγὼ Κύριος ὁ Θεὸς ὑμῶν.
\VS{3}Κατὰ τὰ ἐπιτηδεύματα Αἰγύπτου, ἐν ᾗ κατῳκήσατε ἐπʼ αὐτῇ, οὐ ποιήσετε· καὶ κατὰ τὰ ἐπιτηδεύματα γῆς Χαναὰν, εἰς ἣν ἐγὼ εἰσάγω ὑμᾶς ἐκεῖ, οὐ ποιήσετε, καὶ τοῖς νομίμοις αὐτῶν οὐ πορεύσεσθε.
\VS{4}Τὰ κρίματά μου ποιήσετε, καὶ τὰ προστάγματά μου φυλάξεσθε, καὶ πορεύεσθε ἐν αὐτοῖς· ἐγὼ Κύριος ὁ Θεὸς ὑμῶν.
\VS{5}Καὶ φυλάξεσθε πάντα τὰ προστάγματά μου, καὶ πάντα τὰ κρίματά μου, καὶ ποιήσετε αὐτά· ἃ ποιήσας αὐτὰ ἄνθρωπος, ζήσεται ἐν αὐτοῖς· ἐγὼ Κύριος ὁ Θεὸς ὑμῶν.
\VS{6}Ἄνθρωπος ἄνθρωπος πρὸς πάντα οἰκεῖα σαρκὸς αὐτοῦ οὐ προσελεύσεται ἀποκαλύψαι ἀσχημοσύνην· ἐγὼ Κύριος.
\VS{7}Ἀσχημοσύνην πατρός σου καὶ ἀσχημοσύνην μητρός σου οὐκ ἀποκαλύψεις, μήτηρ γάρ σου ἐστὶν, οὐκ ἀποκαλύψεις τὴν ἀσχημοσύνην αὐτῆς.
\VS{8}Ἀσχημοσύνην γυναικὸς πατρός σου οὐκ ἀποκαλύψεις, ἀσχημοσύνη πατρός σου ἐστίν.
\VS{9}Ἀσχημοσύνην τῆς ἀδελφῆς σου ἐκ πατρός σου ἢ ἐκ μητρός σου, ἐνδογενοῦς ἢ γεγεννημένης ἔξω, οὐκ ἀποκαλύψεις ἀσχημοσύνην αὐτῶν.
\VS{10}Ἀσχημοσύνην θυγατρὸς υἱοῦ σου, ἢ θυγατρὸς θυγατρός σου, οὐκ ἀποκαλύψεις τὴν ἀσχημοσύνην αὐτῶν, ὅτι σὴ ἀσχημοσύνη ἐστίν.
\VS{11}Ἀσχημοσύνην θυγατρὸς γυναικὸς πατρός σου οὐκ ἀποκαλύψεις, ὁμοπατρία ἀδελφή σου ἐστὶν, οὐκ ἀποκαλύψεις τὴν ἀσχημοσύνην αὐτῆς.
\VS{12}Ἀσχημοσύνην ἀδελφῆς πατρός σου οὐκ ἀποκαλύψεις, οἰκεία γὰρ πατρός σου ἐστιν.
\VS{13}Ἀσχημοσύνην ἀδελφῆς μητρός σου οὐκ ἀποκαλύψεις, οἰκεία γὰρ μητρός σου ἐστίν.
\VS{14}Ἀσχημοσύνην ἀδελφοῦ τοῦ πατρός σου οὐκ ἀποκαλύψεις, καὶ πρὸς τὴν γυναῖκα αὐτοῦ οὐκ εἰσελεύσῃ, συγγενὴς γάρ σου ἐστίν.
\VS{15}Ἀσχημοσύνην νύμφης σου οὐκ ἀποκαλύψεις, γυνὴ γὰρ υἱοῦ σου ἐστὶν, οὐκ ἀποκαλύψεις τὴν ἀσχημοσύνην αὐτῆς.
\VS{16}Ἀσχημοσύνην γυναικὸς ἀδελφοῦ σου οὐκ ἀποκαλύψεις, ἀσχημοσύνη ἀδελφοῦ σου ἐστίν.
\VS{17}Ἀσχημοσύνην γυναικὸς καὶ θυγατρὸς αὐτῆς οὐκ ἀποκαλύψεις· τὴν θυγατέρα τοῦ υἱοῦ αὐτῆς, καὶ τὴν θυγατέρα τῆς θυγατρὸς αὐτῆς οὐ λήψῃ ἀποκαλύψαι τὴν ἀσχημοσύνην αὐτῶν, οἰκεῖαι γάρ σου εἰσίν· ἀσέβημα ἐστι.
\VS{18}Γυναῖκα ἐπʼ ἀδελφῇ αὐτῆς οὐ λήψῃ ἀντίζηλον ἀποκαλύψαι τὴν ἀσχημοσύνην αὐτῆς ἐπʼ αὐτῇ, ἔτι ζώσης αὐτῆς.
\par }{\PP \VS{19}Καὶ πρὸς γυναῖκα ἐν χωρισμῷ ἀκαθαρσίας αὐτῆς οὐκ εἰσελεύσῃ ἀποκαλύψαι τὴν ἀσχημοσύνην αὐτῆς.
\VS{20}Καὶ πρὸς τὴν γυναῖκα τοῦ πλησίον σου οὐ δώσεις κοίτην σπέρματός σου, ἐκμιανθῆναι πρὸς αὐτήν.
\VS{21}Καὶ ἀπὸ τοῦ σπέρματός σου οὐ δώσεις λατρεύειν ἄρχοντι· καὶ οὐ βεβηλώσεις τὸ ὄνομα τὸ ἅγιον· ἐγὼ Κύριος.
\VS{22}Καὶ μετὰ ἄρσενος οὐ κοιμηθήσῃ κοίτην γυναικείαν, βδέλυγμα γάρ ἐστι.
\VS{23}Καὶ πρὸς πᾶν τετράπουν οὐ δώσεις τὴν κοίτην σου εἰς σπερματισμὸν, ἐκμιανθῆναι πρὸς αὐτό· καὶ γυνὴ οὐ στήσεται πρὸς πᾶν τετράπουν βιβασθῆναι· μυσαρὸν γάρ ἐστι.
\VS{24}Μὴ μιαίνεσθε ἐν πᾶσι τούτοις· ἐν πᾶσι γὰρ τούτοις ἐμίανθησαν τὰ ἔθνη, ἃ ἐγὼ ἐξαποστέλλω πρὸ προσώπου ὑμῶν,
\VS{25}καὶ ἐξεμιάνθη ἡ γῆ καὶ ἀνταπέδωκα ἀδικίαν αὐτοῖς διʼ αὐτὴν, καὶ προσώχθισεν ἡ γῆ τοῖς ἐγκαθημένοις ἐπʼ αὐτῆς.
\VS{26}Καὶ φυλάξεσθε πάντα τὰ νόμιμά μου, καὶ πάντα τὰ προστάγματά μου, καὶ οὐ ποιήσετε ἀπὸ πάντων τῶν βδελυγμάτων τούτων ὁ ἐγχώριος, καὶ ὁ προσγενόμενος προσήλυτος ἐν ὑμῖν·
\VS{27}(Πάντα γὰρ τὰ βδελύγματα ταῦτα ἐποίησαν οἱ ἄνθρωποι τῆς γῆς, οἱ ὄντες πρότερον ὑμῶν, καὶ ἐμιάνθη ἡ γῆ·)
\VS{28}καὶ ἵνα μὴ προσοχθίσῃ ὑμῖν ἡ γῆ ἐν τῷ μιαίνειν ὑμᾶς αὐτὴν, ὃν τρόπον προσώχθισε τοῖς ἔθνεσι τοῖς πρὸ ὑμῶν.
\VS{29}Ὅτι πᾶς ὃς ἐὰν ποιήσῃ ἀπὸ πάντων τῶν βδελυγμάτων τούτων, ἐξολοθρευθήσονται αἱ ψυχαὶ αἱ ποιοῦσαι ἐκ τοῦ λαοῦ αὐτῶν.
\VS{30}Καὶ φυλάξετε τὰ προστάγματά μου, ὅπως μὴ ποιήσητε ἀπὸ πάντων τῶν νομίμων τῶν ἐβδελυγμένων, ἃ γέγονε πρὸ τοῦ ὑμᾶς· καὶ οὐ μιανθήσεσθε ἐν αὐτοῖς, ὅτι ἐγὼ Κύριος ὁ Θεὸς ὑμῶν.

\par }\Chap{19}{\PP \VerseOne{1}Καὶ ἐλάλησε Κύριος πρὸς Μωυσῆν, λέγων,
\VS{2}λάλησον τῇ συναγωγῇ τῶν υἱῶν Ἰσραὴλ, καὶ ἐρεῖς πρὸς αὐτοῦς, ἅγιοι ἔσεσθε, ὅτι ἅγιος ἐγὼ Κύριος ὁ Θεὸς ὑμῶν.
\VS{3}Ἕκαστος πατέρα αὐτοῦ καὶ μητέρα αὐτοῦ φοβείσθω, καὶ τὰ σάββατά μου φυλάξεσθε· ἐγὼ Κύριος ὁ Θεὸς ὑμῶν.
\VS{4}Οὐκ ἐπακολουθήσετε εἰδώλοις, καὶ θεοὺς χωνευτοὺς οὐ ποιήσετε ὑμῖν· ἐγὼ Κύριος ὁ Θεὸς ὑμῶν.
\VS{5}Καὶ ἐὰν θύσητε θυσίαν σωτηρίου τῷ Κυρίῳ, δεκτὴν ὑμῶν θύσετε.
\VS{6}ᾟ ἂν ἡμέρᾳ θύσετε, βρωθήσεται, καὶ τῇ αὔριον· καὶ ἐὰν καταλειφθῇ ἕως ἡμέρας τρίτης, ἐν πυρὶ κατακαυθήσεται.
\VS{7}Ἐὰν δὲ βρώσει βρωθῇ τῇ ἡμέρᾳ τῇ τρίτῃ, ἄθυτόν ἐστιν, οὐ δεχθήσεται.
\VS{8}Ὁ δὲ ἔσθων αὐτὸ, ἁμαρτίαν λήψεται, ὅτι τὰ ἅγια Κυρίου ἐβεβήλωσε· καὶ ἐξολοθρευθήσονται αἱ ψυχαὶ αἱ ἔσθουσαι ἐκ τοῦ λαοῦ αὐτῶν.
\par }{\PP \VS{9}Καὶ ἐκθεριζόντων ὑμῶν τὸν θερισμὸν τῆς γῆς ὑμῶν, οὐ συντελέσετε τὸν θερισμὸν ὑμῶν τοῦ ἀγροῦ σου ἐκθερίσαι· καὶ τὰ ἀποπίπτοντα τοῦ θερισμοῦ σου οὐ συλλέξεις,
\VS{10}καὶ τὸν ἀμπελῶνά σου οὐκ ἐπανατρυγήσεις, οὐδὲ τὰς ῥῶγας τοῦ ἀμπελῶνός σου συλλέξεις· τῷ πτωχῷ καὶ τῷ προσηλύτῳ καταλείψεις αὐτά. ἐγώ εἰμι Κύριος ὁ Θεὸς ὑμῶν.
\VS{11}Οὐ κλέψετε, οὐ ψεύσεσθε, οὐδὲ συκοφαντήσει ἕκαστος τὸν πλησίον.
\VS{12}Καὶ οὐκ ὀμεῖσθε τῷ ὀνόματί μου ἐπʼ ἀδίκῳ, καὶ οὐ βεβηλώσετε τὸ ὄνομα τὸ ἅγιον τοῦ Θεοῦ ὑμῶν· ἐγώ εἰμι Κύριος ὁ Θεὸς ὑμῶν.
\VS{13}Οὐκ ἀδικήσεις τὸν πλησίον, καὶ οὐχ ἁρπᾷ· καὶ οὐ μὴ κοιμηθήσεται ὁ μισθὸς τοῦ μισθωτοῦ σου παρὰ σοὶ ἕως πρωΐ.
\par }{\PP \VS{14}Οὐ κακῶς ἐρεῖς κωφόν, καὶ ἀπέναντι τυφλοῦ οὐ προσθήσεις σκάνδαλον· καὶ φοβηθήσῃ Κύριον τὸν Θεόν σου· ἐγώ εἰμι Κύριος ὁ Θεὸς ὑμῶν.
\VS{15}Οὐ ποιήσετε ἄδικον ἐν κρίσει· οὐ λήψῃ πρόσωπον πτωχοῦ, οὐδὲ μὴ θαυμάσῃς πρόσωπον δυνάστου· ἐν δικαιοσύνῃ κρίνεις τὸν πλησίον σου.
\VS{16}Οὐ πορεύσῃ δόλῳ ἐν τῷ ἔθνει σου· οὐκ ἐπιστήσῃ ἐφʼ αἷμα τοῦ πλησίον σου· ἐγώ εἰμι Κύριος ὁ Θεὸς ὑμῶν.
\VS{17}Οὐ μισήσεις τὸν ἀδελφόν σου τῇ διανοίᾳ σου· ἐλεγμῷ ἐλέγξεις τὸν πλησίον σου, καὶ οὐ λήψῃ διʼ αὐτὸν ἁμαρτίαν.
\VS{18}Καὶ οὐκ ἐκδικᾶταί σου ἡ χείρ· καὶ οὐ μηνιεῖς τοῖς υἱοῖς τοῦ λαοῦ σου· καὶ ἀγαπήσεις τὸν πλησίον σου ὡς σεαυτόν· ἐγώ εἰμι Κύριος.
\par }{\PP \VS{19}Τὸν νόμον μου φυλάξεσθε· τὰ κτήνη σου οὐ κατοχεύσεις ἑτεροζύγῳ· καὶ τὸν ἀμπελῶνά σου οὐ κατασπερεῖς διάφορον· καὶ ἱμάτιον ἐκ δύο ὑφασμένον κίβδηλον οὐκ ἐπιβαλεῖς σεαυτῷ.
\VS{20}Καὶ ἐάν τις κοιμηθῇ μετὰ γυναικὸς κοίτην σπέρματος, καὶ αὕτη ᾖ οἰκέτις διαπεφυλαγμένη ἀνθρώπῳ, καὶ αὕτη λύτροις οὐ λελύτρωται, ἢ ἐλευθερία οὐκ ἐδόθη αὐτῇ, ἐπισκοπὴ ἔσται αὐτοῖς· οὐκ ἀποθανοῦνται, ὅτι οὐκ ἀπηλευθερώθη.
\VS{21}Καὶ προσάξει τῆς πλημμελείας αὐτοῦ τῷ Κυρίῳ παρὰ τὴν θύραν τῆς σκηνῆς τοῦ μαρτυρίου κριὸν πλημμελείας.
\VS{22}Καὶ ἐξιλάσεται περὶ αὐτοῦ ὁ ἱερεὺς ἐν τῷ κριῷ τῆς πλημμελείας ἔναντι Κυρίου περὶ τῆς ἁμαρτίας ἧς ἥμαρτε, καὶ ἀφεθήσεται αὐτῷ ἡ ἁμαρτία ἣν ἥμαρτεν.
\VS{23}Ὅταν δὲ εἰσέλθητε εἰς τὴν γῆν, ἣν Κύριος ὁ Θεὸς ὑμῶν δίδωσιν ὑμῖν, καὶ καταφυτεύσετε πᾶν ξύλον βρώσιμον, καὶ περικαθαριεῖτε τὴν ἀκαθαρσίαν αὐτοῦ· ὁ καρπὸς αὐτοῦ τρία ἔτη ἔσται ὑμῖν ἀπερικάθαρτος, οὐ βρωθήσεται.
\VS{24}Καὶ τῷ ἔτει τῷ τετάρτῳ ἔσται πᾶς ὁ καρπὸς αὐτοῦ ἅγιος αἰνετὸς τῷ Κυρίῳ.
\VS{25}Ἐν δὲ τῷ ἔτει τῷ πέμπτῳ φάγεσθε τὸν καρπόν, πρόσθεμα ὑμῖν τὰ γεννήματα αὐτοῦ· ἐγώ εἰμι Κύριος ὁ Θεὸς ὑμῶν.
\par }{\PP \VS{26}Μὴ ἔσθετε ἐπὶ τῶν ὀρέων, καὶ οὐκ οἰωνιεῖσθε, οὐδὲ ὀρνιθοσκοπήσεσθε.
\VS{27}Οὐ ποιήσετε σισόην ἐκ τῆς κόμης τῆς κεφαλῆς ὑμῶν, οὐδὲ φθερεῖτε τὴν ὄψιν τοῦ πώγωνος ὑμῶν.
\VS{28}Καὶ ἐντομίδας οὐ ποιήσετε ἐπὶ ψυχῇ ἐν τῷ σώματι ὑμῶν· καὶ γράμματα στικτὰ οὐ ποιήσετε ἐν ὑμῖν· ἐγώ εἰμι Κύριος ὁ Θεὸς ὑμῶν.
\VS{29}Οὐ βεβηλώσεις τὴν θυγατέρα σου ἐκπορνεῦσαι αὐτήν· καὶ οὐκ ἐκπορνεύσει ἡ γῆ, καὶ ἡ γῆ πλησθήσεται ἀνομίας.
\VS{30}Τὰ σάββατά μον φυλάξεσθε, καὶ ἀπὸ τῶν ἁγίων μου φοβηθήσεσθε· ἐγώ εἰμι Κύριος.
\VS{31}Οὐκ ἐπακολουθήσετε ἐγγαστριμύθοις, καὶ τοῖς ἐπαοιδοῖς οὐ προσκολληθήσεσθε, ἐκμιανθῆναι ἐν αὐτοῖς· ἐγώ εἰμι Κύριος ὁ Θεὸς ὑμῶν.
\VS{32}Ἀπὸ προσώπου πολιοῦ ἐξαναστήσῃ, καὶ τιμήσεις πρόσωπον πρεσβυτέρου, καὶ φοβηθήσῃ τὸν Θεόν σου· ἐγώ εἰμι Κύριος ὁ Θεὸς ὑμῶν.
\VS{33}Ἐὰν δέ τις προσέλθῃ ὑμῖν προσήλυτος ἐν τῇ γῇ ὑμῶν, οὐ θλίψετε αὐτόν.
\VS{34}Ὡς ὁ αὐτόχθων ἐν ὑμῖν ἔσται ὁ προσήλυτος ὁ προσπορευόμενος πρὸς ὑμᾶς, καὶ ἀγαπήσεις αὐτὸν ὡς σεαυτόν· ὅτι προσήλυτοι ἐγενήθητε ἐν γῇ Αἰγύπτῳ· ἐγώ εἰμι Κύριος ὁ Θεὸς ὑμῶν.
\VS{35}Οὐ ποιήσετε ἄδικον ἐν κρίσει, ἐν μέτροις καὶ ἐν σταθμίοις καὶ ἐν ζυγοῖς.
\VS{36}Ζυγὰ δίκαια καὶ σταθμία δίκαια καὶ χοῦς δίκαιος ἔσται ἐν ὑμιν· ἐγώ εἰμι Κύριος ὁ Θεὸς ὑμῶν, ὁ ἐξαγαγὼν ὑμᾶς ἐκ γῆς Αἰγύπτου.
\VS{37}Καὶ φυλάξεσθε πάντα τὸν νόμον μου, καὶ πάντα τὰ προστάγματά μου, καὶ ποιήσετε αὐτά· ἐγώ εἰμι Κύριος ὁ Θεὸς ὑμῶν.

\par }\Chap{20}{\PP \VerseOne{1}Καὶ ἐλάλησε Κύριος πρὸς Μωυσῆν, λέγων, καὶ τοῖς υἱοῖς Ἰσραὴλ λαλήσεις,
\VS{2}ἐάν τις ἀπὸ τῶν υἱῶν Ἰσραὴλ, ἢ ἀπὸ τῶν γεγενημένων προσηλύτων ἐν Ἰσραὴλ, ὃς ἂν δῷ τοῦ σπέρματος αὐτοῦ ἄρχοντι, θανάτῳ θανατούσθω· τὸ ἔθνος τὸ ἐπὶ τῆς γῆς λιθοβολήσουσιν αὐτὸν ἐν λίθοις.
\VS{3}Καὶ ἐγὼ ἐπιστήσω τὸ πρόσωπόν μου ἐπὶ τὸν ἄνθρωπον ἐκεῖνον, καὶ ἀπολῶ αὐτὸν ἐκ τοῦ λαοῦ αὐτοῦ, ὅτι τοῦ σπέρματος αὐτοῦ ἔδωκεν ἄρχοντι, ἵνα μιάνῃ τὰ ἅγιά μου, καὶ βεβηλώσῃ τὸ ὄνομα τῶν ἡγιασμένων μοι.
\VS{4}Ἐὰν δὲ ὑπερόψει ὑπερίδωσιν οἱ αὐτόχθονες τῆς γῆς τοῖς ὀφθαλμοῖς αὐτῶν ἀπὸ τοῦ ἀνθρώπου ἐκείνου, ἐν τῷ δοῦναι αὐτὸν τοῦ σπέρματος αὐτοῦ ἄρχοντι, τοῦ μὴ ἀποκτεῖναι αὐτόν·
\VS{5}καὶ ἐπιστήσω τὸ πρόσωπόν μου ἐπὶ τὸν ἄνθρωπον ἐκεῖνον, καὶ τὴν συγγένειαν αὐτοῦ, καὶ ἀπολῶ αὐτὸν, καὶ πάντας τοὺς ὁμονοοῦντας αὐτῷ, ὥστε ἐκπορνεύειν αὐτὸν εἰς τοὺς ἄρχοντας, ἐκ τοῦ λαοῦ αὐτῶν.
\par }{\PP \VS{6}Καὶ ψυχὴ ἣ ἂν ἐπακολουθήσῃ ἐγγαστριμύθοις ἢ ἐπαοιδοῖς, ὥστε ἐκπορνεῦσαι ὀπίσω αὐτῶν, ἐπιστήσω τὸ πρόσωπόν μου ἐπὶ τὴν ψυχὴν ἐκείνην, καὶ ἀπολῶ αὐτῆν ἐκ τοῦ λαοῦ αὐτῆς.
\VS{7}Καὶ ἔσεσθε ἅγιοι, ὅτι ἅγιος ἐγὼ Κύριος ὁ Θεὸς ὑμῶν.
\VS{8}Καὶ φυλάξεσθε τὰ προστάγματά μου, καὶ ποιήσετε αὐτά· ἐγὼ Κύριος ὁ ἁγιάζεν ὑμᾶς.
\VS{9}Ἄνθρωπος ἄνθρωπος, ὃς ἂν κακῶς εἴπῃ τὸν πατέρα αὐτοῦ ἢ τὴν μητέρα αὐτοῦ, θανάτῳ θανατούσθω· πατέρα αὐτοῦ ἢ μητέρα αὐτοῦ κακῶς εἶπεν; ἔνοχος ἔσται.
\par }{\PP \VS{10}Ἄνθρωπος ὃς ἂν μοιχεύσηται γυναῖκα ἀνδρός, ἢ ὃς ἂν μοιχεύσηται γυναῖκα τοῦ πλησίον, θανάτῳ θανατούσθωσαν, ὁ μοιχεύων καὶ ἡ μοιχευομένη.
\VS{11}Καὶ ἐάν τις κοιμηθῇ μετὰ γυναικὸς τοῦ πατρὸς αὐτοῦ, ἀσχημοσύνην τοῦ πατρὸς αὐτοῦ ἀπεκάλυσε· θανάτῳ θανατούσθωσαν ἀμφότεροι, ἔνοχοί εἰσι.
\VS{12}Καὶ ἐάν τις κοιμηθῇ μετὰ νύμφης αὐτοῦ, θανάτῳ θανατούσθωσαν ἀμφότεροι· ἠσεβήκασι γάρ, ἔνοχοί εἰσι.
\VS{13}Καὶ ὃς ἂν κοιμηθῇ μετὰ ἄρσενος κοίτην γυναικὸς, βδέλυγμα ἐποίησαν ἀμφότεροι· θανάτῳ θανατούσθωσαν, ἔνοχοι εἰσιν.
\VS{14}Ὃς ἂν λάβῃ γυναῖκα καὶ τὴν μητέρα αὐτῆς, ἀνόμημά ἐστιν· ἐν πυρὶ κατακαύσουσιν αὐτὸν καὶ αὐτὰς, καὶ οὐκ ἔσται ἀνομία ἐν ὑμῖν.
\VS{15}Καὶ ὃς ἂν δῷ κοιτασίαν αὐτοῦ ἐν τετράποδι, θανάτῳ θανάτούσθω, καὶ τὸ τετράπουν ἀποκτενεῖτε.
\VS{16}Καὶ γυνὴ ἥτις προσελεύσεται πρὸς πᾶν κτῆνος βιβασθῆναι αὐτὴν ὑπʼ αὐτοὺ, ἀποκτενεῖτε τὴν γυναῖκα καὶ τὸ κτῆνος· θανάτῳ θανατούσθωσαν, ἔνοχοί εἰσιν.
\VS{17}Ὃς ἂν λάβῃ τὴν ἀδελφὴν αὐτοῦ ἐκ πατρὸς αὐτοῦ ἢ ἐκ μητρὸς αὐτοῦ, καὶ ἴδῃ τὴν ἀσχημοσύνην αὐτῆς, καὶ αὕτη ἴδῃ τὴν ἀσχημοσύνην αὐτοῦ, ὄνειδός ἐστιν, ἐξολοθρευθήσονται ἐνωπιον υἱῶν γένους αὐτῶν· ἀσχημοσύνην ἀδελφῆς αὐτοῦ ἀπεκάλυψεν, ἁμαρτίαν κομιοῦνται.
\VS{18}Καὶ ἀνὴρ ὃς ἂν κοιμηθῇ μετὰ γυναικὸς ἀποκαθημένης, καὶ ἀποκαλύψῃ τὴν ἀσχημοσύνην αὐτῆς, τὴν πηγὴν αὐτῆς ἀπεκάλυψε, καὶ αὕτη ἀπεκάλυψε τὴν ῥύσιν τοῦ αἵματος αὐτῆς· ἐξολοθρευθήσονται ἀμφότεροι ἐκ τῆς γενεᾶς αὐτῶν.
\VS{19}Καὶ ἀσχημοσύνην ἀδελφῆς πατρὸς σου, καὶ ἀδελφῆς μητρός σου οὐκ ἀποκαλύψεις· τὴν γὰρ οἰκειότητα ἀπεκάλυψεν, ἁμαρτίαν ἀποίσονται.
\VS{20}Ὃς ἂν κοιμηθῇ μετὰ τῆς συγγενοῦς αὐτοῦ, ἀσχημοσύνην τῆς συγγενείας αὐτοῦ ἀπεκάλυψεν, ἄτεκνοι ἀποθανοῦνται.
\VS{21}Ὃς ἐὰς λάβῃ γυναῖκα τοῦ ἀδελφοῦ αὐτοῦ, ἀκαθαρσία ἐστίν· ἀσχημοσύνην τοῦ ἀδελφοῦ αὐτοῦ ἀπεκάλυψεν, ἄτεκνοι ἀποθανοῦνται.
\par }{\PP \VS{22}Καὶ φυλάξασθε πάντα τὰ προστάγματά μου, καὶ τὰ κρίματά μου, καὶ ποιήσετε αὐτὰ, καὶ οὐ μὴ προσοχθίσῃ ὑμῖν ἡ γῆ, εἰς ἣν ἐγὼ εἰσάγω ὑμᾶς ἐκεῖ κατοικεῖν ἐπʼ αὐτῆς.
\VS{23}Καὶ οὐχὶ πορεύεσθε τοῖς νομίμοις τῶν ἐθνῶν, οὓς ἐξαποστέλλω ἀφʼ ὑμῶν· ὅτι ταῦτα πάντα ἐποίησαν, καὶ ἐβδελυξάμην αὐτούς.
\VS{24}Καὶ εἶπα ὑμῖν, ὑμεῖς κληρονομήσετε τὴν γῆν αὐτῶν, καὶ ἐγὼ δώσω ὑμῖν αὐτὴν ἐν κτήσει, γῆν ῥέουσαν γάλα καὶ μέλι· ἐγὼ Κύριος ὁ Θεὸς ὑμῶν, ὃς διώρισα ὑμᾶς ἀπὸ πάντων τῶν ἐθνῶν.
\VS{25}Καὶ ἀφοριεῖτε αὐτοὺς ἀναμέσον τῶν κτηνῶν τῶν καθαρῶν καὶ ἀναμέσον τῶν κτηνῶν τῶν ἀκαθάρτων, καὶ ἀναμέσον τῶν πετεινῶν τῶν καθαρῶν καὶ τῶν ἀκαθάρτων· καὶ οὐ βδελύξετε τὰς ψυχὰς ὑμῶν ἐν τοῖς κτήνεσι, καὶ ἐν τοῖς πετεινοῖς, καὶ ἐν πᾶσι τοῖς ἑρπετοῖς τῆς γῆς ἃ ἐγὼ ἀφώρισα ὑμῖν ἐν ἀκαθαρσίᾳ.
\VS{26}Καὶ ἔσεσθέ μοι ἅγιοι, ὅτι ἐγὼ ἅγιός εἶμι Κύριος ὁ Θεὸς ὑμῶν, ὁ ἀφορίσας ὑμᾶς ἀπὸ πάντων τῶν ἐθνῶν, εἶναί μοι.
\par }{\PP \VS{27}Καὶ ἀνὴρ ἢ γυνὴ ὃς ἂν γένηται αὐτῶν ἐγγαστρίμυθος ἢ ἐπαοιδός, θανάτῳ θανατούσθωσαν ἀμφότεροι· λίθοις λιθοβολήσετε αὐτοὺς, ἔνοχοί εἰσι.

\par }\Chap{21}{\PP \VerseOne{1}Καὶ εἶπε Κύριος πρὸς Μωυσῆν, λέγων, εἶπον τοῖς ἱερεύσι τοῖς υἱοῖς Ἀαρὼν, καὶ ἐρεῖς πρὸς αὐτούς, ἐν ταῖς ψυχαῖς οὐ μιανθήσονται ἐν τῷ ἔθνει αὐτῶν,
\VS{2}ἀλλʼ ἢ ἐν τῷ οἰκείῳ τῷ ἔγγιστα αὐτῶν, ἐπὶ πατρὶ καὶ μητρὶ, καὶ υἱοῖς, καὶ θυγατράσιν, ἐπʼ ἀδελφῷ,
\VS{3}καὶ ἐπʼ ἀδελφῇ παρθένῳ τῇ ἐγγιζούσῃ αὐτῷ, τῇ μὴ ἐκδεδομένῃ ἀνδρί, ἐπὶ τούτοις μιανθήσεται.
\VS{4}Οὐ μιανθήσεται ἐξάπινα ἐν τῷ λαῷ αὐτοῦ εἰς βεβήλωσιν αὐτοῦ.
\VS{5}Καὶ φαλάκρωμα οὐ ξυρηθήσεσθε τὴν κεφαλὴν ἐπὶ νεκρῷ· καὶ τὴν ὄψιν τοῦ πώγωνος οὐ ξυρήσονται· καὶ ἐπὶ τὰς σάρκας αὐτῶν οὐ κατατεμοῦσιν ἐντομίδας.
\VS{6}Ἅγιοι ἔσονται τῷ Θεῷ αὐτῶν, καὶ οὐ βεβηλώσουσι τὸ ὄνομα τοῦ Θεοῦ αὐτῶν· τὰς γὰρ θυσίας Κυρίου δῶρα τοῦ Θεοῦ αὐτῶν αὐτοὶ προσφέρουσι, καὶ ἔσονται ἅγιοι.
\VS{7}Γυναῖκα πόρνην καὶ βεβηλωμένην οὐ λήψονται, καὶ γυναῖκα ἐκβεβλημένην ἀπὸ ἀνδρὸς αὐτῆς, ὅτι ἅγιός ἐστι Κυρίῳ τῷ Θεῷ αὐτοῦ.
\VS{8}Καὶ ἁγιάσεις αὐτόν· τὰ δῶρα Κυρίου τοῦ Θεοῦ ὑμῶν οὗτος προσφέρει, ἅγιος ἔσται· ὅτι ἅγιος ἐγὼ Κύριος ὁ ἁγιάζων αὐτούς.
\VS{9}Καὶ θυγάτηρ ἀνθρώπου ἱερέως ἐὰν βεβηλωθῇ τοῦ ἐκπορνεύσαι, τὸ ὄνομα τοῦ πατρὸς αὐτῆς αὐτὴ βεβηλοῖ· ἐπὶ πυρὸς κατακαυθήσεται.
\par }{\PP \VS{10}Καὶ ὁ ἱερεὺς ὁ μέγας ἀπὸ τῶν ἀδελφῶν αὐτοῦ, τοῦ ἐπικεχυμένου ἐπὶ τὴν κεφαλὴν τοῦ ἐλαίου τοῦ χριστοῦ, καὶ τετελειωμένου ἐνδύσασθαι τὰ ἱμάτια, τὴν κεφαλὴν οὐκ ἀποκιδαρώσει, καὶ τὰ ἱμάτια οὐ διαῤῥήξει,
\VS{11}καὶ ἐπὶ πάσῃ ψυχῇ τετελευτηκυίᾳ οὐκ εἰσελεύσεται, ἐπὶ πατρὶ αὐτοῦ οὐδὲ ἐπὶ μητρὶ αὐτοῦ οὐ μιανθήσεται.
\VS{12}Καὶ ἐκ τῶν ἁγίων οὐκ ἐξελεύσεται, καὶ οὐ βεβηλώσει τὸ ἡγιασμένον τοῦ Θεοῦ αὐτοῦ, ὅτι τὸ ἅγιον ἔλαιον τὸ χριστὸν τοῦ Θεοῦ ἐπʼ αὐτῷ· ἐγὼ Κύριος.
\VS{13}Οὗτος γυναῖκα παρθένον ἐκ τοῦ γένους αὐτοῦ λήψεται.
\VS{14}Χήραν δὲ καὶ ἐκβεβλημένην καὶ βεβηλωμένην καὶ πόρνην, ταύτας οὐ λήψεται, ἀλλʼ ἢ παρθένον ἐκ τοῦ λαοῦ αὐτοῦ λήψεται γυναῖκα.
\VS{15}Καὶ οὐ βεβηλώσει τὸ σπέρμα αὐτοῦ ἐν τῷ λαῷ αὐτοῦ· ἐγὼ Κύριος ὁ ἁγιάζων αὐτόν.
\VS{16}Καὶ ἐλάλησε Κύριος πρὸς Μωυσῆν, λέγων,
\VS{17}εἶπον Ἀαρὼν, ἄνθρωπος ἐκ τοῦ γένους σου εἰς τὰς γενεὰς ὑμῶν, τινὶ ἐὰν ᾖ ἐν αὐτῷ μῶμος, οὐ προσελεύσεται προσφέρειν τὰ δῶρα τοῦ Θεοῦ αὐτοῦ.
\VS{18}πᾶς ἄνθρωπος ᾧ ἐστιν ἐν αὐτῷ μῶμος, οὐ προσελεύσεται· ἄνθρωπος τυφλὸς.
\VS{19}ἢ χωλὸς, ἢ κολοβόριν, ἢ ὠτότμητος, ἢ ἄνθρωπος ᾧ ἂν ᾖ ἐν αὐτῳ σύντριμμα χειρὸς, ἢ σύντριμμα ποδὸς,
\VS{20}ἢ κυρτὸς, ἢ ἔφηλος, ἢ πτίλλος τοὺς ὀφθαλμούς, ἢ ἄνθρωπος ᾧ ἂν ᾖ ἐν αὐτῷ ψώρα ἀγρία, ἢ λειχὴν, ἢ μονόρχις.
\VS{21}Πᾶς ᾧ ἐστιν ἐν αὐτῷ μῶμος, ἐκ τοῦ σπέρματος Ἀαρὼν τοῦ ἱερέως, οὐκ ἐγγιεῖ τοῦ προσενεγκεῖν τὰς θυσίας τῷ Θεῷ σου, ὅτι μῶμος ἐν αὐτῷ· τὰ δῶρα τοῦ Θεοῦ οὐ προσελεύσεται προσενεγκεῖν.
\VS{22}Τὰ δῶρα τοῦ Θεοῦ τὰ ἅγια τῶν ἁγίων, καὶ ἀπὸ τῶν ἁγίων φάγεται.
\VS{23}Πλὴν πρὸς τὸ καταπέτασμα οὐ προσελεύσεται, καὶ πρὸς τὸ θυσιαστήριον οὐκ ἐγγιεῖ, ὅτι μῶμον ἔχει· καὶ οὐ βεβηλώσει τὸ ἅγιον τοῦ Θεοῦ αὐτοῦ, ὅτι ἐγώ εἰμι Κύριος ὁ ἁγιάζων αὐτούς.
\VS{24}Καὶ ἐλάλησε Μωυσῆς πρὸς Ἀαρὼν καὶ τοὺς υἱοὺς αὐτοῦ, καὶ πρὸς πάντας υἱοὺς Ἱσραήλ.

\par }\Chap{22}{\PP \VerseOne{1}Καὶ ἐλάλησε Κύριος πρὸς Μωυσῆν, λέγων,
\VS{2}εἶπον Ἀαρὼν καὶ τοῖς υἱοῖς αὐτοῦ· καὶ προσεχέτωσαν ἀπὸ τῶν ἁγίων τῶν υἱῶν Ἰσραὴλ, καὶ οὐ βεβηλώσουσι τὸ ὄνομα τὸ ἅγιόν μου, ὅσα αὐτοὶ ἁγιάζουσί μοι· ἐγὼ Κύριος.
\VS{3}Εἶπον αὐτοῖς, εἰς τὰς γενεὰς ὑμῶν πᾶς ἄνθρωπος, ὃς ἂν προσέλθῃ ἀπὸ παντὸς τοῦ σπέρματος ὑμῶν πρὸς τὰ ἅγια, ὅσα ἂν ἁγιάζωσιν οἱ υἱοὶ Ἰσραὴλ τῷ Κυρίῳ, καὶ ἡ ἀκαθαρσία αὐτοῦ ἐπʼ αὐτῷ ᾖ, ἐξολοθρευθήσεται ἡ ψυχὴ ἐκείνη ἀπʼ ἐμοῦ· ἐγὼ Κύριος ὁ Θεὸς ὑμῶν.
\VS{4}Καὶ ἄνθρωπος ἐκ τοῦ σπέρματος Ἀαρὼν τοῦ ἱερέως, καὶ οὗτος λεπρᾷ ἢ γονοῤῥυεῖ, τῶν ἁγίων οὐκ ἔδεται, ἕως ἂν καθαρισθῇ· καὶ ὁ ἁπτόμενος πάσης ἀκαθαρσίας ψυχῆς, ἢ ἄνθρωπος ᾧ ἂν ἐξέλθῃ ἐξ αὐτοῦ κοίτη σπέρματος,
\VS{5}ἢ ὅστις ἂν ἅψηται παντὸς ἑρπετοῦ ἀκαθάρτου, ὃ μιανεῖ αὐτὸν, ἢ ἐπʼ ἀνθρώπῳ, ἐν ᾧ μιανεῖ αὐτὸν κατὰ πᾶσαν ἀκαθαρσίαν αὐτοῦ·
\VS{6}Ψυχὴ ἥτις ἐὰν ἅψηται αὐτῶν, ἀκάθαρτος ἔσται ἕως ἑσπέρας· οὐκ ἔδεται ἀπὸ τῶν ἁγίων, ἐὰν μὴ λούσηται τὸ σῶμα αὐτοῦ ὕδατι.
\VS{7}Καὶ δύῃ ὁ ἥλιος, καὶ καθαρὸς ἔσται· καὶ τότε φάγεται τῶν ἁγίων, ὅτι ἄρτος αὐτοῦ ἐστι.
\VS{8}Θνησιμαῖον καὶ θηριάλωτον οὐ φάγεται, μιανθῆναι αὐτὸν ἐν αὐτοῖς· ἐγὼ Κύριος.
\VS{9}Καὶ φυλάξονται τὰ φυλάγματά μου, ἵνα μὴ λάβωσι διʼ αὐτὰ ἁμαρτίαν, καὶ ἀποθάνωσι διʼ αὐτὰ, ἐὰν βεβηλώσουσιν αὐτά· ἐγὼ Κύριος ὁ Θεὸς ὁ ἁγιάζων αὐτούς.
\VS{10}Καὶ πᾶς ἀλλογενὴς οὐ φάγεται ἅγια· πάροικος ἱερέως, ἢ μισθωτὸς, οὐ φάγεται ἅγια.
\VS{11}Ἐὰν δὲ ἱερεὺς κτήσηται ψυχὴν ἔγκτητον ἀργυρίου, οὗτος φάγεται ἐκ τῶν ἄρτων αὐτοῦ· καὶ οἱ οἰκογενεῖς αὐτοῦ, καὶ οὗτοι φάγονται τῶν ἄρτων αὐτοῦ.
\VS{12}Καὶ θυγάτηρ ἀνθρώπου ἱερέως ἐὰν γένηται ἀνδρὶ ἀλλογενεῖ, αὐτὴ τῶν ἀπαρχῶν ἁγίου οὐ φάγεται.
\VS{13}Καὶ θυγάτηρ ἱερέως ἐὰν γένηται χήρα ἢ ἐκβεβλημένη, σπέρμα δὲ μὴ ᾖ αὐτῇ, ἐπαναστρέψει ἐπὶ τὸν οἶκον τὸν πατρικὸν κατὰ τὴν νεότητα αὐτῆς· ἀπὸ τῶν ἄρτων τοῦ πατρὸς αὐτῆς φάγεται· καὶ πᾶς ἀλλογενὴς οὐ φάγεται ἀπʼ αὐτῶν.
\VS{14}Καὶ ἄνθρωπος ὃς ἂν φάγῃ ἅγια κατʼ ἄγνοιαν, καὶ προσθήσει τὸ ἐπίπεμπτον αὐτοῦ ἐπʼ αὐτὸ, καὶ δώσει τῷ ἱερεῖ τὸ ἅγιον.
\VS{15}Καὶ οὐ βεβηλώσουσι τὰ ἅγια τῶν υἱῶν Ἰσραὴλ, ἃ αὐτοὶ ἀφαιροῦσι τῷ Κυρίῳ,
\VS{16}καὶ ἐπάξουσιν ἐφʼ ἑαυτοὺς ἀνομίαν πλημμελείας ἐν τῷ ἐσθίειν αὐτοὺς τὰ ἅγια αὐτῶν, ὅτι ἐγὼ Κύριος ὁ ἁγιάζων αὐτούς.
\par }{\PP \VS{17}Καὶ ἐλάλησε Κύριος πρὸς Μωυσῆν, λέγων,
\VS{18}λάλησον Ἀαρὼν καὶ τοῖς υἱοῖς αὐτοῦ, καὶ πάσῃ συναγωγῇ Ἰσραὴλ, καὶ ἐρεῖς πρὸς αὐτοὺς, ἄνθρωπος ἄνθρωπος ἀπὸ τῶν υἱῶν Ἰσραὴλ, ἢ τῶν προσηλύτων τῶν προσκειμένων πρὸς αὐτοὺς ἐν Ἰσραὴλ, ὃς ἂν προσενέγκῃ τὰ δῶρα αὐτοῦ κατὰ πᾶσαν ὁμολογίαν αὐτῶν, ἢ κατὰ πᾶσαν αἵρεσιν αὐτῶν, ὅσα ἂν προσενέγκωσι τῷ Θεῷ εἰς ὁλοκαύτωμα.
\VS{19}Δεκτὰ ὑμῖν ἄμωμα ἄρσενα ἐκ τῶν βουκολίων, ἢ ἐκ τῶν προβάτων, καὶ ἐκ τῶν αἰγῶν.
\VS{20}Πάντα ὅσα ἂν ἔχῃ μῶμον ἐν αὐτῷ οὐ προσάξουσι Κυρίῳ, διότι οὐ δεκτὸν ἔσται ὑμῖν.
\VS{21}Καὶ ἄνθρωπος ὃς ἂν προσενέγκῃ θυσίαν σωτηρίου τῷ Κυρίῳ, διαστείλας εὐχὴν ἢ κατὰ αἵρεσιν ἢ ἐν ταῖς ἑορταῖς ὑμῶν, ἐκ τῶν βουκολίων ἢ ἐκ τῶν προβάτων, ἄμωμον ἔσται εἰσδεκτὸν, πᾶς μῶμος οὐκ ἔσται ἐν αὐτῷ.
\VS{22}Τυφλὸν ἢ συντετριμμένον ἢ γλωσσότμητον ἢ μυρμηκιῶντα ἢ ψωραγριῶντα ἢ λειχῆνας ἔχοντα, οὐ προσάξουσι ταῦτα τῷ Κυρίῳ, καὶ εἰς κάρπωσιν οὐ δώσετε ἀπʼ αὐτῶν ἐπὶ τὸ θυσιαστήριον τῷ Κυρίῳ.
\VS{23}Καὶ μόσχον ἢ πρόβατον ὠτότμητον ἢ κολοβόκερκον, σφάγια ποιήσεις αὐτὰ σεαυτῷ, εἰς δὲ εὐχήν σου οὐ δεχθήσεται.
\VS{24}Θλαδίαν καὶ ἐκτεθλιμμένον καὶ ἐκτομίαν καὶ ἀπεσπασμένον, οὐ προσάξεις αὐτὰ τῷ Κυρίῳ, καὶ ἐπὶ τῆς γῆς ὑμῶν οὐ ποιήσετε.
\VS{25}Καὶ ἐκ χειρὸς ἀλλογενοῦς οὐ προσοίσετε τὰ δῶρα τοῦ Θεοῦ ὑμῶν ἀπὸ πάντων τούτων· ὅτι φθάρματά ἐστιν ἐν αὐτοῖς, μῶμος ἐν αὐτοῖς οὐ δεχθήσεται ταῦτα ὑμῖν.
\VS{26}Καὶ ἐλάλησε Κύριος πρὸς Μωυσῆν, λέγων,
\VS{27}μόσχον ἢ πρόβατον ἢ αἶγα, ὡς ἂν τεχθῇ, καὶ ἔσται ἑπτὰ ἡμέρας ὑπὸ τὴν μητέρα, τῇ δὲ ἡμέρᾳ τῇ ὀγδόῃ καὶ ἐπέκεινα δεχθήσεται εἰς δῶρα, κάρπωμα Κυρίῳ.
\VS{28}Καὶ μόσχον καὶ πρόβατον, αὐτὴν καὶ τὰ παιδία αὐτῆς, οὐ σφάξεις ἐν ἡμέρᾳ μιᾷ.
\par }{\PP \VS{29}Ἐὰν δὲ θύσῃς θυσίαν εὐχὴν χαρμοσύνης Κυρίῳ, εἰσδεκτὸν ὑμῖν θύσετε αὐτό.
\VS{30}Αὐτῇ τῇ ἡμέρᾳ ἐκείνῃ βρωθήσεται· οὐκ ἀπολείψετε ἀπὸ τῶν κρεῶν εἰς τοπρωΐ· ἐγώ εἰμι Κύριος.
\VS{31}Καὶ φυλάξετε τὰς ἐντολάς μου, καὶ ποιήσετε αὐτάς.
\VS{32}Καὶ οὐ βεβηλώσετε τὸ ὄνομα τοῦ ἁγίου, καὶ ἁγιασθήσομαι ἐν μέσῳ τῶν υἱῶν Ἰσραήλ· ἐγὼ Κύριος ὁ ἁγιάζων ὑμᾶς,
\VS{33}ὁ ἐξαγαγὼν ὑμᾶς ἐκ γῆς Αἰγύπτου, ὥστε εἶναι ὑμῶν Θεός· ἐγὼ Κύριος.

\par }\Chap{23}{\PP \VerseOne{1}Καὶ εἶπε Κύριος πρὸς Μωυσῆν, λέγων,
\VS{2}λάλησον τοῖς υἱοῖς Ἰσραὴλ, καὶ ἐρεῖς πρὸς αὐτούς, αἱ ἑορταὶ Κυρίου ἃς καλέσετε αὐτὰς κλητὰς ἁγίας, αὗταί εἰσιν αἱ ἑορταί μου.
\VS{3}Ἓξ ἡμέρας ποιήσεις ἔργα, τῇ δὲ ἡμέρᾳ τῇ ἑβδόμῃ σάββατα, ἀνάπαυσις, κλητὴ ἁγία τῷ Κυρίῳ· πᾶν ἔργον οὐ ποιήσεις· σάββατά ἐστι τῷ Κυρίῳ ἐν πάσῃ κατοικίᾳ ὑμῶν.
\par }{\PP \VS{4}Αὗται αἱ ἑορταὶ τῷ Κυρίῳ κληταὶ ἅγιαι, ἃς καλέσετε αὐτὰς ἐν τοῖς καιροῖς αὐτῶν.
\VS{5}Ἐν τῷ πρώτῳ μηνὶ, ἐν τῇ τεσσαρεσκαιδεκάτῃ ἡμέρᾳ τοῦ μηνὸς, ἀναμέσον τῶν ἑσπερινῶν πάσχα τῷ Κυρίῳ.
\VS{6}Καὶ ἐν τῇ πεντεκαιδεκάτῃ ἡμέρᾳ τοῦ μηνὸς τούτου ἑορτὴ τῶν ἀζύμων τῷ Κυρίῳ· ἑπτὰ ἡμέρας ἄζυμα ἔδεσθε.
\VS{7}Καὶ ἡμέρα ἡ πρώτη κλητὴ ἁγία ἔσται ὑμῖν· πᾶν ἔργον λατρευτὸν οὐ ποιήσετε.
\VS{8}Καὶ προσάξετε ὁλοκαυτώματα τῷ Κυρίῳ ἑπτὰ ἡμέρας· καὶ ἡ ἡμέρα ἡ ἑβδόμη κλητὴ ἁγία ἔσται ὑμῖν· πᾶν ἔργον λατρευτὸν οὐ ποιήσετε.
\VS{9}Καὶ ἐλάλησε Κύριος πρὸς Μωυσῆν, λέγων,
\VS{10}εἶπον τοῖς υἱοῖς Ἰσραὴλ, καὶ ἐρεῖς πρὸς αὐτοὺς, ὅταν εἰσέλθητε εἰς τὴν γῆν, ἣν ἐγὼ δίδωμι ὑμῖν, καὶ θερίζητε τὸν θερισμὸν αὐτῆς, καὶ οἴσετε τὸ δράγμα ἀπαρχὴν τοῦ θερισμοῦ ὑμῶν πρὸς τὸν ἱερέα·
\VS{11}Καὶ ἀνοίσει τὸ δράγμα ἔναντι Κυρίου δεκτὸν ὑμῖν· τῇ ἐπαύριον τῆς πρώτης ἀνοίσει αὐτό ὁ ἱερεύς.
\VS{12}Καὶ ποιήσετε ἐν τῇ ἡμέρᾳ ἐν ᾗ ἂν φέρητε τὸ δράγμα, πρόβατον ἄμωμον ἐνιαύσιον εἰς ὁλοκαύτωμα τῷ Κυρίῳ.
\VS{13}Καὶ τὴν θυσίαν αὐτοῦ δύο δέκατα σεμιδάλεως ἀναπεποιημένης ἐν ἐλαίῳ· θυσία τῷ Κυρίῳ, ὀσμὴ εὐωδίας Κυρίῳ· καὶ σπονδὴν αὐτοῦ τὸ τέταρτον τοῦ ἳν οἴνου.
\VS{14}Καὶ ἄρτον, καὶ πεφρυγμένα χίδρα νέα οὐ φάγεσθε ἕως εἰς αὐτὴν τὴν ἡμέραν ταύτην, ἕως ἂν προσενέγκητε ὑμεῖς τὰ δῶρα τῷ Θεῷ ὑμῶν· νόμιμον αἰώνιον εἰς τὰς γενεὰς ὑμῶν ἐν πάσῃ κατοικίᾳ ὑμῶν.
\par }{\PP \VS{15}Καὶ ἀριθμήσετε ὑμῖν ἀπὸ τῆς ἐπαύριον τῶν σαββάτων, ἀπὸ τῆς ἡμέρας ἧς ἂν προσενέγκητε τὸ δράγμα τοῦ ἐπιθέματος, ἑπτὰ ἑβδομάδας ὁλοκλήρους,
\VS{16}ἕως τῆς ἐπαύριον τῆς ἐσχάτης ἑβδομάδος ἀριθμήσετε πεντήκοντα ἡμέρας, καὶ προσοίσετε θυσίαν νέαν τῷ Κυρίῳ.
\VS{17}Ἀπὸ τῆς κατοικίας ὑμῶν προσοίσετε ἄρτους ἐπίθεμα, δύο ἄρτους· ἐκ δύο δεκάτων σεμιδάλεως ἔσονται, ἐζυμωμένοι πεφθήσονται πρωτογεννημάτων τῷ Κυρίῳ.
\VS{18}Καὶ προσάξετε μετὰ τῶν ἄρτων ἑπτὰ ἀμνοὺς ἀμώμους ἐνιαυσίους, καὶ μόσχον ἕνα ἐκ βουκολίου, καὶ κριοὺς δύο ἀμώμους, καὶ ἔσονται ὁλοκαύτωμα τῷ Κυρίῳ· καὶ αἱ θυσίαι αὐτῶν καὶ αἱ σπονδαὶ αὐτῶν θυσία ὀσμὴ εὐωδίας τῷ Κυρίῳ.
\VS{19}Καὶ ποιήσουσι χίμαρον ἐξ αἰγῶν ἕνα περὶ ἁμαρτίας, καὶ δύο ἀμνοὺς ἐνιαυσίους εἰς θυσίαν σωτηρίου μετὰ τῶν ἄρτων τοῦ πρωτογεννήματος.
\VS{20}Καὶ ἐπιθήσει αὐτὰ ὁ ἱερεὺς μετὰ τῶν ἄρτων τοῦ πρωτογεννήματος ἐπίθεμα ἐναντίον Κυρίου μετὰ τῶν δύο ἀμνῶν, ἅγια ἔσονται τῷ Κυρίῳ· τῷ ἱερεῖ τῷ προσφέροντι αὐτὰ αὐτῷ ἔσται.
\VS{21}Καὶ καλέσετε ταύτην τὴν ἡμέραν κλητήν· ἁγία ἔσται ὑμῖν· πᾶν ἔργον λατρευτὸν οὐ ποιήσετε ἐν αὐτῇ· νόμιμον αἰώνιον εἰς τὰς γενεὰς ὑμῶν ἐν πάσῃ τῇ κατοικίᾳ ὑμῶν.
\VS{22}Καὶ ὅταν θερίζητε τὸν θερισμὸν τῆς γῆς ὑμῶν, οὐ συντελέσετε τὸ λοιπὸν τοῦ θερισμοῦ τοῦ ἀγροῦ σου ἐν τῷ θερίζειν σε, καὶ τὰ ἀποπίπτοντα τοῦ θερισμοῦ σου οὐ συλλέξεις· τῷ πτωχῷ καὶ τῷ προσηλύτῳ ὑπολείψεις αὐτά· ἐγὼ Κύριος ὁ Θεὸς ὑμῶν.
\par }{\PP \VS{23}Καὶ ἐλάλησε Κύριος πρὸς Μωυσῆν, λέγων,
\VS{24}λάλησον τοῖς υἱοῖς Ἰσραὴλ, λέγων, τοῦ μηνὸς τοῦ ἐβδόμου μιᾷ τοῦ μηνὸς ἔσται ὑμῖν ἀνάπαυσις, μνημόσυνον σαλπίγγων· κλητὴ ἁγία ἔσται ὑμῖν·
\VS{25}Πᾶν ἔργον λατρευτὸν οὐ ποιήσετε· καὶ προσάξετε ὁλοκαύτωμα Κυρίῳ.
\par }{\PP \VS{26}Καὶ ἐλάλησε Κύριος πρὸς Μωυσῆν, λέγων,
\VS{27}καὶ τῇ δεκάτῃ τοῦ μηνὸς τοῦ ἐβδόμου τούτου, ἡμέρα ἐξιλασμοῦ, κλητὴ ἁγία ἔσται ὑμῖν· καὶ ταπεινώσετε τὰς ψυχὰς ὑμῶν, καὶ προσάξετε ὁλοκαύτωμα τῷ Κυρίῳ.
\VS{28}Πᾶν ἔργον οὐ ποιήσετε ἐν αὐτῇ τῇ ἡμέρᾳ ταύτῃ· ἔστι γὰρ ἡμέρα ἐξιλασμοῦ αὕτη ὑμῖν, ἐξιλάσασθαι περὶ ὑμῶν ἔναντι Κυρίου τοῦ Θεοῦ ὑμῶν.
\VS{29}Πᾶσα ψυχὴ, ἥτις μὴ ταπεινωθήσεται ἐν αὐτῇ τῇ ἡμέρᾳ ταύτῃ, ἐξολοθρευθήσεται ἐκ τοῦ λαοῦ αὐτῆς.
\VS{30}Καὶ πᾶσα ψυχὴ, ἥτις ποιήσει ἔργον ἐν αὐτῇ τῇ ἡμέρᾳ ταύτῃ, ἀπολεῖται ἡ ψυχὴ ἐκείνη ἐκ τοῦ λαοῦ αὐτῆς.
\VS{31}Πᾶν ἔργον οὐ ποιήσετε· νόμιμον αἰώνιον εἰς τὰς γενεὰς ὑμῶν ἐν πάσαις κατοικίαις ὑμῶν.
\VS{32}Σάββατα σαββάτων ἔσται ὑμῖν· καὶ ταπεινώσετε τὰς ψυχὰς ὑμῶν· ἀπὸ ἐνάτης τοῦ μηνὸς, ἀπὸ ἑσπέρας ἕως ἑσπέρας σαββατιεῖτε τὰ σάββατα ὑμῶν.
\par }{\PP \VS{33}Καὶ ἐλάλησε Κύριος πρὸς Μωυσῆν, λέγων,
\VS{34}λάλησον τοῖς υἱοῖς Ἰσραὴλ, λέγων, τῇ πεντεκαιδεκάτῃ τοῦ μηνὸς τοῦ ἐβδόμου τούτου, ἑορτὴ σκηνῶν ἑπτὰ ἡμέρας τῷ Κυρίῳ.
\VS{35}Καὶ ἡ ἡμέρα ἡ πρώτη κλητὴ ἁγία· πᾶν ἔργον λατρευτὸν οὐ ποιήσετε.
\VS{36}Ἑπτὰ ἡμέρας προσάξετε ὁλοκαυτώματα τῷ Κυρίῳ, καὶ ἡ ἡμέρα ἡ ὀγδόη κλητὴ ἁγία ἔσται ὑμῖν· καὶ προσάξετε ὁλοκαυτώματα Κυρίῳ· ἐξόδιόν ἐστι· πᾶν ἔργον λατρευτὸν οὐ ποιήσετε.
\VS{37}Αὗται ἑορταὶ Κυρὶῳ, ἃς καλέσετε κλητὰς ἁγίας, ὥστε προσενέγκαι καρπώματα τῷ Κυρίῳ, ὁλοκαυτώματα καὶ θυσίας αὐτῶν, καὶ σπονδὰς αὐτῶν τὸ καθʼ ἡμέραν εἰς ἡμέραν·
\VS{38}πλὴν τῶν σαββάτων Κυρίου, καὶ πλὴν τῶν δομάτων ὑμῶν, καὶ πλὴν πασῶν τῶν εὐχῶν ὑμῶν, καὶ πλὴν τῶν ἐκουσίων ὑμῶν, ἃ ἂν δώτε τῷ Κυρίῳ.
\VS{39}Καὶ ἐν τῇ πεντεκαιδεκάτῃ ἡμέρᾳ τοῦ μηνὸς τοῦ ἑβδόμου τούτου, ὅταν συντελέσητε τὰ γεννήματα τῆς γῆς, ἑορτάσατε τῷ Κυρίῳ ἑπτὰ ἡμέρας· τῇ ἡμέρᾳ τῇ πρώτῃ ἀνάπαυσις, καὶ τῇ ἡμέρᾳ τῇ ὀγδόῃ ἀνάπαυσις.
\VS{40}Καὶ λήψεσθε τῇ ἡμέρᾳ τῇ πρώτῃ καρπὸν ξύλου ὡραῖου, καὶ κάλλυνθρα φοινίκων, καὶ κλάδους ξύλου δασεῖς, καὶ ἰτέας, καὶ ἄγνου κλάδους ἐκ χειμάῤῥου, εὐφρανθῆναι ἔναντι Κυρίου τοῦ Θεοῦ ὑμῶν ἑπτὰ ἡμέρας τοῦ ἐνιαυτοῦ.
\VS{41}Νόμιμον αἰώνιον εἰς τὰς γενεὰς ὑμῶν· ἐν τῷ μηνὶ τῷ ἑβδόμῳ ἑορτάσετε αὐτήν.
\VS{42}Ἐν σκηναῖς κατοικήσετε ἑπτὰ ἡμέρας· πᾶς ὁ αὐτόχθων ἐν Ἰσραὴλ κατοικήσει ἐν σκηναῖς,
\VS{43}ὅπως ἴδωσιν αἱ γενεαὶ ὑμῶν, ὅτι ἐν σκηναῖς κατῴκισα τοὺς υἱοὺς Ἰσραὴλ, ἐν τῷ ἐξαγαγεῖν με αὐτοὺς ἐκ γῆς Αἰγύπτου· ἐγὼ Κύριος ὁ Θεὸς ὑμῶν.
\VS{44}Καὶ ἐλάλησε Μωυσῆς τὰς ἑορτὰς Κυρίου τοῖς υἱοῖς Ἰσραήλ.

\par }\Chap{24}{\PP \VerseOne{1}Καὶ ἐλάλησε Κύριος πρὸς Μωυσῆν, λέγων,
\VS{2}ἔντειλαι τοῖς υἱοῖς Ἰσραὴλ, καὶ λαβέτωσάν σοι ἔλαιον ἐλάϊνον καθαρὸν κεκομμένον εἰς φῶς, καῦσαι λύχνον διαπαντὸς,
\VS{3}ἔξωθεν τοῦ καταπετάσματος ἐν τῇ σκηνῇ τοῦ μαρτυρίου· καὶ καύσουσιν αὐτὸ Ἀαρὼν καὶ οἱ υἱοὶ αὐτοῦ ἀπὸ ἑσπέρας ἕως πρωῒ ἐνώπιον Κυρίου ἐνδελεχῶς, νόμιμον αἰώνιον εἰς τὰς γενεὰς ὑμῶν.
\VS{4}Ἐπὶ τῆς λυχνίας τῆς καθαρᾶς καύσετε τοὺς λύχνους ἐναντίον Κυρίου ἕως εἰς τοπρωΐ.
\VS{5}Καὶ λήμψεσθε σεμίδαλιν, καὶ ποιήσετε αὐτὴν δώδεκα ἄρτους· δύο δεκάτων ἔσται ὁ ἄρτος ὁ εἷς.
\VS{6}Καὶ ἐπιθήσετε αὐτοὺς δύο θέματα, ἓξ ἄρτους τὸ ἓν θέμα ἐπὶ τὴν τράπεζαν τὴν καθαρὰν ἔναντι Κυρίου.
\VS{7}Καὶ ἐπιθήσετε ἐπὶ τὸ θέμα λίβανον καθαρὸν καὶ ἅλα, καὶ ἔσονται εἰς ἄρτους εἰς ἀνάμνησιν προκείμενα τῷ Κυρίῳ.
\VS{8}Τῇ ἡμέρᾳ τῶν σαββάτων προσθήσεται ἔναντι Κυρίου διαπαντὸς ἐνώπιον τῶν υἱῶν Ἰσραὴλ, διαθήκην αἰώνιον.
\VS{9}Καὶ ἔσται Ἀαρὼν καὶ τοῖς υἱοῖς αὐτοῦ· καὶ φάγονται αὐτὰ ἐν τόπῳ ἁγίῳ· ἔστι γὰρ ἅγια τῶν ἁγίῳν τοῦτο αὐτῶν ἀπὸ τῶν θυσιαζομένων τῷ Κυρίῳ, νόμιμον αἰώνιον.
\par }{\PP \VS{10}Καὶ ἐξῆλθεν υἱὸς γυναικὸς Ἰσραηλίτιδος, καὶ οὗτος ἦν υἱὸς Αἰγυπτίου ἐν τοῖς υἱοῖς Ἰσραήλ· καὶ ἐμαχέσαντο ἐν τῇ παρεμβολῇ ὁ ἐκ τῆς Ἰσραηλίτιδος, καὶ ὁ ἄνθρωπος ὁ Ἰσραηλίτης.
\VS{11}Καὶ ἐπονομάσας ὁ υἱὸς τῆς γυναικὸς τῆς Ἰσραηλίτιδος τὸ ὄνομα κατηράσατο· καὶ ἤγαγον αὐτὸν πρὸς Μωυσῆν· καὶ τὸ ὄνομα τῆς μητρὸς αὐτοῦ Σαλωμεὶθ θυγάτηρ Δαβρεὶ ἐκ τῆς φυλῆς Δάν.
\VS{12}Καὶ ἀπέθεντο αὐτὸν εἰς φυλακὴν διακρῖναι αὐτὸν διὰ προστάγματος Κυρίου.
\VS{13}Καὶ ἐλάλησε Κύριος πρὸς Μωυσῆν, λέγων,
\VS{14}ἐξάγαγε τὸν καταρασάμενον ἔξω τῆς παρεμβολῆς, καὶ ἐπιθήσουσι πάντες οἱ ἀκούσαντες τὰς χεῖρας αὐτῶν ἐπὶ τὴν κεφαλὴν αὐτοῦ, καὶ λιθοβολήσουσιν αὐτὸν πᾶσα ἡ συναγωγή.
\VS{15}Καὶ τοῖς υἱοῖς Ἰσραὴλ λάλησον, καὶ ἐρεῖς πρὸς αὐτοὺς, ἄνθρωπος ὃς ἐὰν καταράσηται Θεὸν, ἁμαρτίαν λήψεται.
\VS{16}Ὀνομάζων δὲ τὸ ὄνομα Κυρίου, θανάτῳ θανατούσθω· λίθοις λιθοβολείτω αὐτὸν πᾶσα ἡ συναγωγὴ Ἰσραήλ· ἐάν τε προσήλυτος ἐάν τε αὐτόχθων, ἐν τῷ ὀνομάσαι αὐτὸν τὸ ὄνομα Κυρίου, τελευτάτω.
\VS{17}Καὶ ἄνθρωπος ὃς ἂν πατάξῃ ψυχὴν ἀνθρώπου, καὶ ἀποθάνῃ, θανάτῳ θανατούσθω.
\VS{18}Καὶ ὃς ἂν πατάξῃ κτῆνος, καὶ ἀποθάνῃ, ἀποτισάτω ψυχὴν ἀντὶ ψυχῆς.
\VS{19}Καὶ ἐάν τις δῷ μῶμον τῷ πλησίον, ὡς ἐποίησεν αὐτῷ, ὡσαύτως ἀντιποιηθήσεται αὐτῷ·
\VS{20}Σύντριμμα ἀντὶ συντρίμματος, ὀφθαλμὸν ἀντὶ ὀφθαλμοῦ, ὀδόντα ἀντὶ ὀδόντος, καθότι ἂν δῷ μῶμον τῷ ἀνθρώπῳ, οὕτω δοθήσεται αὐτῷ.
\VS{21}Ὃς ἂν πατάξῃ ἄνθρωπον, καὶ ἀποθάνῃ, θανάτῳ θανατούσθω.
\VS{22}Δικαίωσις μία ἔσται τῷ προσηλύτῳ καὶ τῷ ἐγχωρίῳ, ὅτι ἐγώ εἰμι Κύριος ὁ Θεὸς ὑμῶν.
\VS{23}Καὶ ἐλάλησε Μωυσῆς τοῖς υἱοῖς Ἰσραήλ· καὶ ἐξήγαγον τὸν καταρασάμενον ἔξω τῆς παρεμβολῆς, καὶ ἐλιθοβόλησαν αὐτὸν ἐν λίθοις· καὶ οἱ υἱοὶ Ἰσραὴλ ἐποίησαν καθάπερ συνέταξε Κύριος τῷ Μωυσῇ.

\par }\Chap{25}{\PP \VerseOne{1}Καὶ ἐλάλησε Κύριος πρὸς Μωυσῆν ἐν τῷ ὄρει Σινᾷ, λέγων,
\VS{2}λάλησον τοῖς υἱοῖς Ἰσραὴλ, καὶ ἐρεῖς πρὸς αὐτοὺς, ὅταν εἰσέλθητε εἰς τὴν γῆν, ἣν ἐγὼ δίδωμι ὑμῖν, καὶ ἀναπαύσεται ἡ γῆ, ἣν ἐγὼ δίδωμι ὑμῖν, σάββατα τῷ Κυρίῳ.
\VS{3}Ἓξ ἔτη σπερεῖς τὸν ἀγρόν σου, καὶ ἓξ ἔτη τεμεῖς τὴν ἄμπελόν σου, καὶ συνάξεις τὸν καρπὸν αὐτῆς.
\VS{4}Τῷ δὲ ἔτει τῷ ἑβδόμῳ σάββατα· ἀνάπαυσις ἔσται τῇ γῇ, σάββατα τῷ Κυρίῳ· τὸν ἀγρόν σου οὐ σπερεῖς, καὶ τὴν ἄμπελόν σου οὐ τεμεῖς,
\VS{5}καὶ τὰ αὐτόματα ἀναβαίνοντα τοῦ ἀγροῦ σου οὐκ ἐκθερίσεις, καἰ τὴν σταφυλὴν τοῦ ἁγιάσματός σου οὐκ ἐκτρυγήσεις· ἐνιαυτὸς ἀναπαύσεως ἔσται τῇ γῇ.
\VS{6}Καὶ ἔσται τὰ σάββατα τῆς γῆς βρώματά σοι, καὶ τῷ παιδί σου, καὶ τῇ παιδίσκῃ σου, καὶ τῷ μισθωτῷ σου, καὶ τῷ παροίκῳ τῷ προσκειμένῳ πρὸς σέ.
\VS{7}Καὶ τοῖς κτήνεσί σου, καὶ τοῖς θηρίοις τοῖς ἐν τῇ γῇ σου ἔσται πᾶν τὸ γέννημα αὐτοῦ εἰς βρῶσιν.
\par }{\PP \VS{8}Καὶ ἐξαριθμήσεις σεαυτῷ ἑπτὰ ἀναπαύσεις ἐτῶν, ἑπτὰ ἔτη ἑπτάκις· καὶ ἔσονταί σοι ἑπτὰ ἑβδομάδες ἐτῶν ἐννέα καὶ τεσσαράκοντα ἔτη.
\VS{9}Διαγγελεῖτε σάλπιγγος φωνῇ ἐν πάσῃ τῇ γῇ ὑμῶν ἐν τῷ μηνὶ τῷ ἑβδόμῳ τῇ δεκάτῃ τοῦ μηνός· τῇ ἡμέρᾳ τοῦ ἱλασμοῦ διαγγελεῖτε σάλπιγγι ἐν πάσῃ τῇ γῇ ὑμῶν.
\VS{10}Καὶ ἁγιάσετε τὸ ἔτος τὸν πεντηκοστὸν ἐνιαυτὸν, καὶ διαβοήσετε ἄφεσιν ἐπὶ τῆς γῆς πᾶσι τοῖς κατοικοῦσιν αὐτήν· ἐνιαυτὸς ἀφέσεως σημασία αὕτη ἔσται ὑμῖν· καὶ ἀπελεύσεται εἷς ἕκαστος εἰς τὴν κτῆσιν αὐτοῦ, καὶ ἕκαστος εἰς τὴν πατριὰν αὐτοῦ ἀπελεύσεσθε.
\VS{11}Ἀφέσεως σημασία αὕτη, τὸ ἔτος τὸ πεντηκοστὸν ἐνιαυτὸς ἔσται ὑμῖν· οὐ σπερεῖτε, οὐδὲ ἀμήσετε τὰ αὐτόματα ἀναβαίνοντα αὐτῆς, καὶ οὐ τρυγήσετε τὰ ἡγιασμένα αὐτῆς,
\VS{12}ὅτι ἀφέσεως σημασία ἐστίν· ἅγιον ἔσται ὑμῖν· ἀπὸ τῶν πεδίων φάγεσθε τὰ γεννήματα αὐτῆς.
\VS{13}Ἐν τῷ ἔτει τῆς ἀφέσεως σημασίας αὐτῆς ἐπανελεύσεται εἰς τὴν ἔγκτησιν αὐτοῦ.
\VS{14}Ἐὰν δὲ ἀποδῷ πράσιν τῷ πλησίον σου, ἐὰν δὲ καὶ κτήσῃ παρὰ τοῦ πλησίον σου, μὴ θλιβέτω ἄνθρωπος τὸν πλησίον.
\VS{15}Κατὰ ἀριθμὸν ἐτῶν μετὰ τὴν σημασίαν κτήσῃ παρὰ τοῦ πλησίον, κατὰ ἀριθμὸν ἐνιαυτῶν γεννημάτων ἀποδώσεταί σοι.
\VS{16}Καθότι ἂν πλεῖον τῶν ἐτῶν πληθυνεῖ τὴν ἔγκτησιν αὐτοῦ, καὶ καθότι ἂν ἔλαττον τῶν ἐτῶν ἐλαττονώσει τὴν ἔγκτησιν αὐτοῦ· ὅτι ἀριθμὸν γεννημάτων αὐτοῦ, οὕτως ἀποδώσεταί σοι.
\VS{17}Μὴ θλιβέτω ἄνθρωπος τὸν πλησίον· καὶ φοβηθήσῃ Κύριον τὸν Θεόν σου· ἐγώ εἰμι Κύριος ὁ Θεὸς ὑμῶν.
\par }{\PP \VS{18}Καὶ ποιήσετε πάντα τὰ δικαιώματά μου, καὶ πάσας τὰς κρίσεις μου, καὶ φυλάξασθε, καὶ ποιήσετε αὐτὰ, καὶ κατοικήσετε ἐπὶ τῆς γῆς πεποιθότες.
\VS{19}Καὶ δώσει ἡ γῆ τὰ ἐκφόρια αὐτῆς, καὶ φάγεσθε εἰς πλησμονὴν, καὶ κατοικήσετε πεποιθότες ἐπʼ αὐτῆς.
\VS{20}Ἐὰν δὲ λέγητε, τί φαγόμεθα ἐν τῷ ἔτει τῷ ἑβδόμῳ τούτῳ, ἐὰν μὴ σπείρωμεν μηδὲ συναγάγωμεν τὰ γεννήματα ἡμῶν;
\VS{21}Καὶ ἀποστέλλω τὴν εὐλογίαν μου ὑμῖν ἐν τῷ ἔτει τῷ ἕκτῳ, καὶ ποιήσει τὰ γεννήματα αὐτῆς εἰς τὰ τρία ἔτη.
\VS{22}Καὶ σπερεῖτε τὸ ἔτος τὸ ὄγδοον, καὶ φάγεσθε ἀπὸ τῶν γεννημάτων παλαιὰ ἕως τοῦ ἔτους τοῦ ἐνάτου· ἕως ἂν ἔλθῃ τὸ γέννημα αὐτῆς, φάγεσθε παλαιὰ παλαιῶν.
\VS{23}Καὶ ἡ γῆ οὐ πραθήσεται εἰς βεβαίωσιν· ἐμὴ γάρ ἐστιν ἡ γῆ, διότι προσήλυτοι καὶ πάροικοι ὑμεῖς ἐστε ἐναντίον μου.
\VS{24}Καὶ κατὰ πᾶσαν γῆν κατασχέσεως ὑμῶν, λύτρα δώσετε τῆς γῆς.
\VS{25}Ἐὰν δὲ πένηται ὁ ἀδελφός σου ὁ μετὰ σοῦ, καὶ ἀποδῶται ἀπὸ τῆς κατασχέσεως αὐτοῦ, καὶ ἔλθῃ ὁ ἀγχιστεύων ὁ ἐγγίζων αὐτῷ, καὶ λυτρώσεται τὴν πρᾶσιν τοῦ ἀδελφοῦ αὐτοῦ.
\VS{26}Ἐὰν δὲ μὴ ᾖ τινι ὁ ἀγχιστεύων, καὶ εὐπορηθῇ τῇ χειρὶ, καὶ εὑρεθῇ αὐτῷ τὸ ἱκανὸν, λύτρα αὐτοῦ·
\VS{27}καὶ συλλογιεῖται τὰ ἔτη τῆς πράσεως αὐτοῦ, καὶ ἀποδώσει ὅ ὑπερέχει τῷ ἀνθρώπῳ, ᾧ ἀπέδοτο αὐτὸ αὐτῷ, καὶ ἀπελεύσεται εἰς τὴν κατάσχεσιν αὐτοῦ.
\VS{28}Ἐὰν δὲ μὴ εὑπορηθῇ αὐτοῦ ἡ χεὶρ τὸ ἱκανὸν, ὥστε ἀποδοῦναι αὐτῷ, καὶ ἔσται ἡ πράσις τῷ κτησαμένῳ αὐτὰ ἕως τοῦ ἕκτου ἔτους τῆς ἀφέσεως, καὶ ἐξελεύσεται ἐν τῇ ἀφέσει, καὶ ἀπελεύσεται εἰς τὴν κατάσχεσιν αὐτοῦ.
\par }{\PP \VS{29}Ἐὰν δέ τις ἀποδῶται οἰκίαν οἰκητὴν ἐν πόλει τετειχισμένῃ, καὶ ἔσται ἡ λύτρωσις αὐτῆς, ἕως πληρωθῇ· ἐνιαυτὸς ἡμερῶν ἔσται ἡ λύτρωσις αὐτῆς.
\VS{30}Ἐὰν δὲ μὴ λυτρωθῇ ἕως ἂν πληρωθῇ αὐτῆς ἐνιαυτὸς ὅλος, κυρωθήσεται ἡ οἰκία ἡ οὖσα ἐν πόλει τῇ ἐχούσῃ τεῖχος, βεβαίως τῷ κτησαμένῳ αὐτὴν εἰς τὰς γενεὰς αὐτοῦ, καὶ οὐκ ἐξελεύσεται ἐν τῇ ἀφέσει.
\VS{31}Αἱ δὲ οἰκίαι αἱ ἐν ἐπαύλεσιν, αἷς οὐκ ἔστιν ἐν αὐταῖς τεῖχος κύκλῳ, πρὸς τὸν ἀγρὸν τῆς γῆς λογισθήσονται· λυτρωταὶ διαπαντὸς ἔσονται, καὶ ἐν τῇ ἀφέσει ἐξελεύσονται.
\VS{32}Καὶ αἱ πόλεις τῶν Λευιτῶν, οἰκίαι τῶν πόλεων κατασχέσεως αὐτῶν, λυτρωταὶ διαπαντὸς ἔσονται τοῖς Λευίταις.
\VS{33}Καὶ ὃς ἂν λυτρώσηται παρὰ τῶν Λευιτῶν, καὶ ἐξελεύσεται ἡ διάπρασις αὐτῶν οἰκιῶν πόλεως κατασχέσεως αὐτῶν ἐν τῇ ἀφέσει, ὅτι οἰκίαι τῶν πόλεων τῶν Λευιτῶν κατάσχεσις αὐτῶν ἐν μέσῳ υἱῶν Ἰσραήλ.
\VS{34}Καὶ οἱ ἀγροὶ ἀφωρισμένοι ταῖς πόλεσιν αὐτῶν οὐ πραθήσονται, ὅτι κατάσχεσις αἰωνία τοῦτο αὐτῶν ἐστον.
\par }{\PP \VS{35}Ἐὰν δὲ πένηται ὁ ἀδελφός σοῦ ὁ μετὰ σοῦ, καὶ ἀδυνατήσῃ ταῖς χερσὶ παρὰ σοὺ, ἀντιλήψῃ αὐτοῦ ὡς προσηλύτου καὶ παροίκου, καὶ ζήσεται ὁ ἀδελφός σου μετὰ σοῦ.
\VS{36}Οὐ λήψῃ παρʼ αὐτοῦ τόκον, οὐδὲ ἐπὶ πλήθει· καὶ φοβηθήσῃ τὸν Θεόν σου· ἐγὼ Κύριος· καὶ ζήσεται ὁ ἀδελφός σου μετὰ σοῦ.
\VS{37}Τὸ ἀργύριόν σου οὐ δώσεις αὐτῷ ἐπὶ τὸκῳ, καὶ ἐπὶ πλεονασμῷ οὐ δώσεις αὐτῷ τὰ βρώματά σου.
\VS{38}Ἐγὼ Κύριος ὁ Θεὸς ὑμῶν, ὁ ἐξαγαγὼν ὑμᾶς ἐκ γῆς Αἰγύπτου, δοῦναι ὑμῖν τὴν γῆν Χαναὰν, ὥστε εἶναι ὑμῶν Θεός.
\par }{\PP \VS{39}Ἐὰν δὲ ταπεινωθῇ ὁ ἀδελφός σου παρὰ σοὶ, καὶ πραθῇ σοι, οὐ δουλεύσει σοι δουλείαν οἰκέτου.
\VS{40}Ὡς μισθωτὸς ἢ πάροικος ἔσται σοι· ἕως τοῦ ἔτους τῆς ἀφέσεως ἐργᾶται παρὰ σοί,
\VS{41}καὶ ἐξελεύσεται τῇ ἀφέσει, καὶ τὰ τέκνα αὐτοῦ μετʼ αὐτοῦ, καὶ ἀπελεύσεται εἰς τὴν γενεὰν αὐτοῦ, εἰς τὴν κατάσχεσιν τὴν πατρικὴν ἀποδραμεῖται.
\VS{42}Διότι οἰκέται μου εἰσὶν οὗτοι, οὓς ἐξήγαγον ἐκ γῆς Αἰγύπτου· οὐ πραθήσεται ἐν πράσει οἰκέτου.
\VS{43}Οὐ κατατενεῖς αὐτὸν ἐν τῷ μόχθῳ, καὶ φοβηθήσῃ Κύριον τὸν Θεόν σου,
\VS{44}καὶ παῖς καὶ παιδίσκη ὅσοι ἂν γένωνταί σοι, ἀπὸ τῶν ἐθνῶν ὅσοι κύκλῳ σου εἰσὶν, ἀπʼ αὐτῶν κτήσεσθε δοῦλον καὶ δούλην,
\VS{45}καὶ ἀπὸ τῶν υἱῶν τῶν παροίκων τῶν ὄντων ἐν ὑμῖν, ἀπὸ τούτων κτήσεσθε καὶ ἀπὸ τῶν συγγενῶν αὐτῶν, ὅσοι ἂν γένωνται ἐν τῇ γῇ ὑμῶν, ἔστωσαν ὑμῖν εἰς κατάσχεσιν.
\VS{46}Καὶ καταμεριεῖτε αὐτοὺς τοῖς τέκνοις ὑμῶν μεθʼ ὑμᾶς· καὶ ἔσονται ὑμῖν κατόχιμοι εἰς τὸν αἰῶνα· τῶν δὲ ἀδελφῶν ὑμῶν τῶν υἱῶν Ἰσραὴλ, ἕκαστος τὸν ἀδελφὸν αὐτοῦ οὐ κατατενεῖ αὐτὸν ἐν τοῖς μόχθοις.
\par }{\PP \VS{47}Ἐὰν δὲ εὕρῃ ἡ χεὶρ τοὺ προσηλύτου ἢ τοῦ παροίκου τοῦ παρὰ σοὶ, καὶ ἀπορηθεὶς ὁ ἀδελφός σου πραθῇ τῷ προσηλύτῳ ἢ τῷ παροίκῳ τῷ παρὰ σοὶ, ἢ ἐκ γενετῆς προσηλύτῳ,
\VS{48}μετὰ τὸ πραθῆναι αὐτῷ, λύτρωσις ἔσται αὐτοῦ· εἷς τῶν ἀδελφῶν αὐτοῦ λυτρώσεται αὐτόν.
\VS{49}Ἀδελφὸς πατρὸς αὐτοῦ, ἢ υἱὸς ἀδελφοῦ πατρὸς λυτρώσεται αὐτόν, ἢ ἀπὸ τῶν οἰκείων τῶν σαρκῶν αὐτοῦ ἐκ τῆς φυλῆς αὐτοῦ λυτρῶται αὐτόν· ἐὰν δὲ εὐπορηθεὶς ταῖς χερσὶ λυτρῶται ἑαυτὸν,
\VS{50}καὶ συλλογιεῖται πρὸς τὸν κεκτημένον αὐτὸν ἀπὸ τοῦ ἔτους οὗ ἀπέδοτο ἑαυτὸν αὐτῷ ἕως τοῦ ἐνιαυτοῦ τῆς ἀφέσεως· καὶ ἔσται τὸ ἀργύριον τῆς πράσεως αὐτοῦ ὡς μισθίου· ἔτος ἐξ ἔτους ἔσται μετʼ αὐτοῦ.
\VS{51}Ἐὰν δέ τινι πλεῖον τῶν ἐτῶν ᾖ, πρὸς ταῦτα ἀποδώσει τὰ λύτρα αὐτοῦ ἀπὸ τοῦ ἀργυρίου τῆς πράσεως αὐτοῦ.
\VS{52}Ἐὰν δὲ ὀλίγον καταλειφθῇ ἀπὸ τῶν ἐτῶν εἰς τὸν ἐνιαυτὸν τῆς ἀφέσεως, καὶ συλλογιεῖται αὐτῷ κατὰ τὰ ἔτη αὐτοῦ, καὶ ἀποδώσει τὰ λύτρα αὐτοῦ ὡς μισθωτός·
\VS{53}ἐνιαυτὸς ἐξ ἐνιαυτοῦ ἔσται μετʼ αὐτοῦ· οὐ κατατενεῖς αὐτὸν ἐν τῷ μόχθῳ ἐνώπιόν σου.
\VS{54}Ἐὰν δὲ μὴ λυτρῶται κατὰ ταῦτα, ἐξελεύσεται ἐν τῷ ἔτει τῆς ἀφέσεως αὐτὸς καὶ τὰ παιδία αὐτοῦ μετʼ αὐτοῦ.
\VS{55}Ὅτι ἐμοὶ οἱ υἱοὶ Ἰσραὴλ οἰκέται εἰσὶ, παῖδές μου οὗτοί εἰσιν, οὓς ἐξήγαγον ἐκ γῆς Αἰγύπτου.

\par }\Chap{26}{\PP \VerseOne{1}Ἐγὼ Κύριος ὁ Θεὸς ὑμῶν· οὐ ποιήσετε ὑμῖν αὐτοῖς χειροποίητα, οὐδὲ γλυπτὰ, οὐδὲ στήλην ἀναστήσετε ὑμῖν, οὐδὲ λίθον σκοπὸν θήσετε ἐν τῇ γῇ ὑμῶν προσκυνῆσαι αὐτῷ· ἐγώ εἰμι Κύριος ὁ Θεὸς ὑμῶν.
\VS{2}Τὰ σάββατά μου φυλάξεσθε, καὶ ἀπὸ τῶν ἁγίων μου φοβηθήσεσθε· ἐγώ εἰμι Κύριος.
\VS{3}Ἐὰν τοῖς προστάγμασί μου πορεύησθε, καὶ τὰς ἐντολάς μου φυλάσσησθε, καὶ ποιήσητε αὐτὰς,
\VS{4}καὶ δώσω τὸν ὑετὸν ὑμῖν ἐν καιρῷ αὐτοῦ, καὶ ἡ γῆ δώσει τὰ γεννήματα αὐτῆς, καὶ τὰ ξύλα τῶν πεδίων ἀποδώσει τὸν καρπὸν αὐτῶν·
\VS{5}Καὶ καταλήψεται ὑμῖν ὁ ἁλοητὸς τὸν τρυγητὸν, καὶ ὁ τρυγητὸς καταλήψεται τὸν σπόρον· καὶ φάγεσθε τὸν ἄρτον ὑμῶν εἰς πλησμονήν· καὶ κατοικήσετε μετὰ ἀσφαλείας ἐπὶ τῆς γῆς ὑμῶν, καὶ πόλεμος οὐ διελεύσεται διὰ τῆς γῆς ὑμῶν·
\VS{6}Καὶ δώσω εἰρήνην ἐν τῇ γῇ ὑμῶν· καὶ κοιμηθήσεσθε, καὶ οὐκ ἔσται ὑμᾶς ὁ ἐκφοβῶν· καὶ ἀπολῶ θηρία πονηρὰ ἐκ τῆς γῆς ὑμῶν.
\VS{7}Καὶ διώξεσθε τοὺς ἐχθροὺς ὑμῶν, καὶ πεσοῦνται ἐναντίον ὑμῶν φόνῳ.
\VS{8}Καὶ διώξονται ἐξ ὑμῶν πέντε ἑκατὸν, καὶ ἑκατὸν ὑμῶν διώξονται μυριάδας· καὶ πεσοῦνται οἱ ἐχθροὶ ὑμῶν ἐναντίον ὑμῶν μαχαίρᾳ.
\VS{9}Καὶ ἐπιβλέψω ἐφʼ ὑμᾶς, καὶ αὐξανῶ ὑμᾶς, καὶ πληθυνῶ ὑμᾶς, καὶ στήσω τὴν διαθήκην μου μεθʼ ὑμῶν·
\VS{10}Καὶ φάγεσθε παλαιὰ καὶ παλαιὰ παλαιῶν, καὶ παλαιὰ ἐκ προσώπου νέων ἐξοίσετε.
\VS{11}Καὶ θήσω τὴν σκηνήν μου ἐν ὑμῖν, καὶ οὐ βδελύξεται ἡ ψυχή μου ὑμᾶς,
\VS{12}καὶ ἐμπεριπατήσω ἐν ὑμῖν· καὶ ἔσομαι ὑμῶν Θεὸς, καὶ ὑμεῖς ἔσεσθέ μοι λαός.
\VS{13}Ἐγώ εἰμι Κύριος ὁ Θεὸς ὑμῶν, ὁ ἐξαγαγὼν ὑμᾶς ἐκ γῆς Αἰγύπτου, ὄντων ὑμῶν δούλων· καὶ συνέτριψα τὸν δεσμὸν τοῦ ζυγοῦ ὑμῶν, καὶ ἤγαγον ὑμᾶς μετὰ παῤῥησίας.
\par }{\PP \VS{14}Ἐὰν δὲ μὴ ὑπακούσητέ μου, μηδὲ ποιήσητε τὰ προστάγματά μου ταῦτα,
\VS{15}ἀλλὰ ἀπειθήσητε αὐτοῖς, καὶ τοῖς κρίμασί μου προσοχθίσῃ ἡ ψυχὴ ὑμῶν, ὥστε ὑμᾶς μὴ ποιεῖν πάσας τὰς ἐντολάς μου, ὥστε διασκεδάσαι τὴν διαθήκην μου,
\VS{16}καὶ ἐγὼ ποιήσω οὕτως ὑμῖν· καὶ ἐπιστήσω ἐφʼ ὑμᾶς τὴν ἀπορίαν, τήν τε ψώραν, καὶ τὸν ἴκτερα σφακελίζοντα τοὺς ὀφθαλμοὺς ὑμῶν, καὶ τὴν ψυχὴν ὑμῶν ἐκτήκουσαν· καὶ σπερεῖτε διακενῆς τὰ σπέρματα ὑμῶν, καὶ ἔδονται οἱ ὑπεναντίοι ὑμῶν.
\VS{17}Καὶ ἐπιστήσω τὸ πρόσωπόν μου ἐφʼ ὑμᾶς, καὶ πεσεῖσθε ἐναντίον τῶν ἐχθρῶν ὑμῶν, καὶ διώξονται ὑμᾶς οἱ μισοῦντες ὑμᾶς, καὶ φεύξεσθε οὐδενὸς διώκοντος ὑμᾶς.
\VS{18}Καὶ ἐὰν ἕως τούτου μὴ ὑπακούσητέ μου, καὶ προσθήσω τοῦ παιδεῦσαι ὑμᾶς ἑπτάκις ἐπὶ ταῖς ἁμαρτίαις ὑμῶν.
\VS{19}Καὶ συντρίψω τὴν ὕβριν τῆς ὑπερηφανίας ὑμῶν· καὶ θήσω τὸν οὐρανὸν ὑμῖν σιδηροῦν, καὶ τὴν γῆν ὑμῶν ὡσεὶ χαλκῆν.
\VS{20}Καὶ ἔσται εἰς κενὸν ἡ ἰσχὺς ὑμῶν· καὶ οὐ δώσει ἡ γῆ ὑμῶν τὸν σπόρον αὐτῆς, καὶ τὸ ξύλον τοῦ ἀγρου ὑμῶν οὐ δώσει τὸν καρπὸν αὐτοῦ.
\par }{\PP \VS{21}Καὶ ἐὰν μετὰ ταῦτα πορεύησθε πλάγιοι, καὶ μὴ βούλησθε ὑπακούειν μου, προσθήσω ὑμῖν πληγὰς ἑπτὰ κατὰ τὰς ἁμαρτίας ὑμῶν.
\VS{22}Καὶ ἀποστέλλω ἐφʼ ὑμᾶς τὰ θηρία τὰ ἄγρια τῆς γῆς, καὶ κατέδεται ὑμᾶς, καὶ ἐξαναλώσει τὰ κτήνη ὑμῶν, καὶ ὀλιγοστοὺς ποιήσω ὑμᾶς, καὶ ἐρημωθήσονται αἱ ὁδοὶ ὑμῶν.
\VS{23}Καὶ ἐπὶ τούτοις ἐὰν μὴ παιδευθῆτε, ἀλλὰ πορεύησθε πρός με πλάγιοι,
\VS{24}πορεύσομαι κᾀγὼ μεθʼ ὑμῶν θυμῷ πλαγίῳ, καὶ πατάξω ὑμᾶς κᾀγὼ ἑπτάκις ἀντὶ τῶν ἁμαρτιῶν ὑμῶν.
\VS{25}Καὶ ἐπάξω ἐφʼ ὑμᾶς μάχαιραν ἐκδικοῦσαν δίκην διαθήκης, καὶ καταφεύξεσθε εἰς τὰς πόλεις ὑμῶν· καὶ ἐξαποστελῶ θάνατον εἰς ὑμᾶς, καὶ παραδοθήσεσθε εἰς χεῖρας τῶν ἐχθρῶν.
\VS{26}Ἐν τῷ θλίψαι ὑμᾶς σιτοδείᾳ ἄρτων, καὶ πέψουσι δέκα γυναῖκες τοὺς ἄρτους ὑμῶν ἐν κλιβάνῳ ἑνὶ, καὶ ἀποδώσουσι τοὺς ἄρτους ὑμῶν ἐν σταθμῷ, καὶ φάγεσθε, καὶ οὐ μὴ ἐμπλησθῆτε.
\par }{\PP \VS{27}Ἐὰν δὲ ἐπὶ τούτοις μὴ ὑπακούσητέ μου, καὶ πορεύησθε πρός με πλάγιοι,
\VS{28}καὶ αὐτὸς πορεύσομαι μεθʼ ὑμῶν ἐν θυμῷ πλαγίῳ, καὶ παιδεύσω ὑμᾶς ἐγὼ ἑπτάκις κατὰ τὰς ἁμαρτίας ὑμῶν.
\VS{29}Καὶ φάγεσθε τὰς σάρκας τῶν υἱῶν ὑμῶν, καὶ τὰς σάρκας τῶν θυγατέρων ὑμῶν φάγεσθε.
\VS{30}Καὶ ἐρημώσω τὰς στήλας ὑμῶν, καὶ ἐξολοθρεύσω τὰ ξύλινα χειροποίητα ὑμῶν, καὶ θήσω τὰ κῶλα ὑμῶν ἐπὶ τὰ κῶλα τῶν εἰδώλων ὑμῶν, καὶ προσοχθιεῖ ἡ ψυχή μου ὑμῖν.
\VS{31}Καὶ θήσω τὰς πόλεις ὑμῶν ἐρήμους, καὶ ἐξερημώσω τὰ ἅγια ὑμῶν, καὶ οὐ μὴ ὀσφρανθῶ τῆς ὀσμῆς τῶν θυσιῶν ὑμῶν.
\VS{32}Καὶ ἐξερημώσω ἐγὼ τὴν γῆν ὑμῶν, καὶ θαυμάσονται ἐπʼ αὐτῇ οἱ ἐχθροὶ ὑμῶν, οἱ ἐνοικοῦντες ἐν αὐτῇ.
\VS{33}Καὶ διασπερῶ ὑμᾶς εἰς τὰ ἔθνη, καὶ ἐξαναλώσει ὑμᾶς ἐπιπορευομένη ἡ μάχαιρα, καὶ ἔσται ἡ γῆ ὑμῶν ἔρημος, καὶ αἱ πόλεις ὑμῶν ἔσονται ἔρημοι.
\VS{34}Τότε εὐδοκήσει ἡ γῆ τὰ σάββατα αὐτῆς πάσας τὰς ἡμέρας τῆς ἐρημώσεως αὐτῆς,
\VS{35}καὶ ὑμεῖς ἔσεσθε ἐν τῇ γῇ τῶν ἐχθρῶν ὑμῶν· τότε σαββατιεῖ ἡ γη, καὶ εὐδοκήσει ἡ γῆ τὰ σάββατα αὐτῆς πάσας τὰς ἡμέρας τῆς ἐρημώσεως αὐτῆς· σαββατιεῖ ἃ οὐκ ἐσαββάτισεν ἐν τοῖς σαββάτοις ὑμῶν, ἡνίκα κατῳκεῖτε αὐτήν.
\VS{36}Καὶ τοῖς καταλειφθεῖσιν ἐξ ὑμῶν ἐπάξω δουλείαν εἰς τὴν καρδίαν αὐτῶν ἐν τῇ γῇ τῶν ἐχθρῶν αὐτῶν· καὶ διώξεται αὐτοὺς φωνὴ φύλλου φερομένου, καὶ φεύξονται ὡς φεύγοντες ἀπὸ πολέμου, καὶ πεσοῦνται οὐθενὸς διώκοντος.
\VS{37}Καὶ ὑπερόψεται ὁ ἀδελφὸς τὸν ἀδελφὸν ὡσεὶ ἐν πολέμῳ, οὐθενὸς κατατρέχοντος· καὶ οὐ δυνήσεσθε ἀντιστῆναι τοῖς ἐχθροῖς ὑμῶν.
\VS{38}Καὶ ἀπολεῖσθε ἐν τοῖς ἔθνεσι, καὶ κατέδεται ὑμᾶς ἡ γῆ τῶν ἐχθρῶν ὑμῶν.
\VS{39}Καὶ οἱ καταλειφθέντες ἀφʼ ὑμῶν, καταφθαρήσονται διὰ τὰς ἁμαρτίας αὐτῶν, καὶ διὰ τὰς ἁμαρτίας τῶν πατέρων αὐτῶν· ἐν τῇ γῇ τῶν ἐχθρῶν αὐτῶν τακήσονται.
\par }{\PP \VS{40}Καὶ ἐξαγορεύσουσι τὰς ἁμαρτίας αὐτῶν, καὶ τὰς ἁμαρτίας τῶν πατέρων αὐτῶν, ὅτι παρέβησαν καὶ ὑπερεῖδόν με, καὶ ὅτι ἐπορεύθησαν ἐναντίον μου πλάγιοι,
\VS{41}καὶ ἐγὼ ἐπορεύθην μετʼ αὐτῶν ἐν θυμῷ πλαγίῳ· καὶ ἀπολῶ αὐτοὺς ἐν τῇ γῇ τῶν ἐχθρῶν αὐτῶν· τότε ἐντραπήσεται ἡ καρδία αὐτῶν ἡ ἀπερίτμητος, καὶ τότε εὐδοκήσουσι τὰς ἁμαρτίας αὐτῶν.
\VS{42}Καὶ μνησθήσομαι τῆς διαθήκης Ἰακὼβ, καὶ τῆς διαθήκης Ἰσαὰκ, καὶ τῆς διαθήκης Ἁβραὰμ μνησθήσομαι.
\par }{\PP \VS{43}Καὶ τῆς γῆς μνησθήσομαι, καὶ ἡ γῆ ἐγκαταλειφθήσεται ἀπʼ αὐτῶν· τότε προσδέξεται ἡ γῆ τὰ σάββατα αὐτῆς, ἐν τῷ ἐρημωθῆναι αὐτὴν διʼ αὐτούς· καὶ αὐτοὶ προσδέξονται τὰς αὐτῶν ἀνομίας, ἀνθʼ ὧν τὰ κρίματά μου ὑπερεῖδον, καὶ τοῖς προστάγμασί μου προσώχθισαν τῇ ψυχῇ αὐτῶν.
\VS{44}Καὶ οὐδʼ ὡς ὄντων αὐτῶν ἐν τῇ γῇ τῶν ἐχθρῶν αὐτῶν, οὐχ ὑπερεῖδον αὐτούς, οὐδὲ προσώχθισα αὐτοῖς ὥστε ἐξαναλῶσαι αὐτούς τοῦ διασκεδάσαι τὴν διαθήκην μου τὴν πρὸς αὐτούς· ἐγὼ γάρ εἰμι Κύριος ὁ Θεὸς αὐτῶν.
\VS{45}Καὶ μνησθήσομαι διαθήκης αὐτῶν τῆς προτέρας, ὅτε ἐξήγαγον αὐτοὺς ἐκ γῆς Αἰγύπτου, ἐξ οἴκου δουλείας ἔναντι τῶν ἐθνῶν, τοῦ εἶναι αὐτῶν Θεός· ἐγώ εἰμι Κύριος.
\VS{46}Ταῦτα τὰ κρίματά μου, καὶ τὰ προστάγματά μου, καὶ ὁ νόμος ὃν ἔδωκε Κύριος ἀναμέσον αὐτοῦ καὶ ἀναμέσον τῶν υἱῶν Ἰσραὴλ, ἐν τῷ ὄρει Σινᾷ ἐν χειρὶ Μωυσῆ.

\par }\Chap{27}{\PP \VerseOne{1}Καὶ ἐλάλησε Κύριος πρὸς Μωυσῆν, λέγων,
\VS{2}λάλησον τοῖς υἱοῖς Ἰσραὴλ, καὶ ἐρεῖς αὐτοῖς, ὃς ἂν εὔξηται εὐχὴν ὥστε τιμὴν τῆς ψυχῆς αὐτοῦ τῷ Κυρίῳ,
\VS{3}ἔσται ἡ τιμὴ τοῦ ἄρσενος ἀπὸ εἰκοσαετοῦς, ἕως ἑξηκονταετοῦς, ἔσται αὐτοῦ ἡ τιμὴ πεντήκοντα δίδραχμα ἀργυρίου τῷ σταθμῷ τῷ ἁγίῳ·
\VS{4}Τῆς δὲ θηλείας ἔσται ἡ συντίμησις τριάκοντα δίδραχμα.
\VS{5}Ἐὰν δὲ ἀπὸ πενταετοῦς ἕως εἴκοσι ἐτῶν, ἔσται ἡ τιμὴ τοῦ ἄρσενος εἴκοσι δίδραχμα· τῆς δὲ θηλείας, δέκα δίδραχμα.
\VS{6}Ἀπὸ δὲ μηνιαίου ἕως πενταετοῦς, ἔσται ἡ τιμὴ τοῦ ἄρσενος πέντε δίδραχμα· τῆς δὲ θηλείας, τρία δίδραχμα ἀργυρίου.
\VS{7}Ἐὰν δὲ ἀπὸ ἑξήκοντα ἐτῶν καὶ ἐπάνω, ἐὰν μὲν ἄρσεν ᾖ, ἔσται ἡ τιμὴ αὐτοῦ πεντεκαίδεκα δίδραχμα ἀργυρίου· ἐὰν δὲ θήλεια, δέκα δίδραχμα.
\VS{8}Ἐὰν δὲ ταπεινὸς ᾖ τῇ τιμῇ, στήσεται ἐναντίον τοῦ ἱερέως· καὶ τιμήσεται αὐτὸν ὁ ἱερεύς· καθάπερ ἰσχύει ἡ χεὶρ τοῦ εὐξαμένου, τιμήσεται αὐτὸν ὁ ἱερεύς.
\par }{\PP \VS{9}Ἐὰν δὲ ἀπὸ τῶν κτηνῶν τῶν προσφερομένων ἀπʼ αὐτῶν δῶρον τῷ Κυρίῳ, ὃς ἂν δῷ ἀπὸ τούτων τῷ Κυρίῳ, ἔσται ἅγιον.
\VS{10}Οὐκ ἀλλάξει αὐτὸ καλὸν πονηρῷ, οὐδὲ πονηρὸν καλῷ· ἐὰν δὲ ἀλλάσσων ἀλλάξῃ αὐτὸ κτῆνος κτήνει, ἔσται αὐτὸ καὶ τὸ ἄλλαγμα ἅγια.
\VS{11}Ἐὰν δὲ πᾶν κτῆνος ἀκάθαρτον, ἀφʼ ὧν οὐ προσφέρεται ἀπʼ αὐτῶν δῶρον τῷ Κυρίῳ, στήσει τὸ κτῆνος ἔναντι τοῦ ἱερέως,
\VS{12}καὶ τιμήσεται αὐτὸ ὁ ἱερεὺς ἀναμέσον καλοῦ καὶ ἀναμέσον πονηροῦ· καὶ καθότι ἂν τιμήσηται αὐτὸ ὁ ἱερεὺς, οὕτω στήσεται.
\VS{13}Ἐὰν δὲ λυτρούμενος λυτρώσηται αὐτὸ, προσθήσει τὸ ἐπίπεμπτον πρὸς τὴν τιμὴν αὐτοῦ.
\VS{14}Καὶ ἄνθρωπος ὃς ἂν ἁγιάσῃ τὴν οἰκίαν αὐτοῦ ἁγίαν τῷ Κυρίῳ, καὶ τιμήσεται αὐτὴν ὁ ἱερεὺς ἀναμέσον καλῆς καὶ ἀναμέσον πονηρᾶς· ὡς ἂν τιμήσηται αὐτὴν ὁ ἱερεὺς, οὕτω σταθήσεται.
\VS{15}Ἐὰν δὲ ὁ ἁγιάσας αὐτὴν λυτρῶται τὴν οἰκίαν αὐτοῦ, προσθήσει ἐπʼ αὐτὸ τὸ ἐπίπεμπτον τοῦ ἀργυρίου τῆς τιμῆς, καὶ ἔσται αὐτῷ.
\par }{\PP \VS{16}Ἐὰν δὲ ἀπὸ τοῦ ἀγροῦ τῆς κατασχέσεως αὐτοῦ ἁγιάσῃ ἄνθρωπος τῷ Κυρίῳ, καὶ ἔσται ἡ τιμὴ κατὰ τὸν σπόρον αὐτοῦ, κόρου κριθῶν πεντήκοντα δίδραχμα ἀργυρίου.
\VS{17}Ἐὰν δὲ ἀπὸ τοῦ ἐνιαυτοῦ τῆς ἀφέσεως ἁγιάσῃ τὸν ἀγρὸν αὐτοῦ, κατὰ τὴν τιμὴν αὐτοῦ στήσεται.
\VS{18}Ἐὰν δὲ ἔσχατον μετὰ τὴν ἄφεσιν ἁγιάσῃ τὸν ἀγρὸν αὐτοῦ, προσλογιεῖται αὐτῷ ὁ ἱερεὺς τὸ ἀργύριον ἐπὶ τὰ ἔτη τὰ ἐπίλοιπα, ἕως εἰς τὸν ἐνιαυτὸν τῆς ἀφέσεως, καὶ ἀνθυφαιρεθήσεται ἀπὸ τῆς συντιμήσεως αὐτοῦ.
\VS{19}Ἐὰν δὲ λυτρῶται τὸν ἀγρὸν ὁ ἁγιάσας αὐτὸν, προσθήσει τὸ ἐπίπεμπτον τοῦ ἀργυρίου πρὸς τὴν τιμὴν αὐτοῦ, καὶ ἔσται αὐτῷ.
\VS{20}Ἐὰν δὲ μὴ λυτρῶται τὸν ἀγρὸν, καὶ ἀποδῶται τὸν ἀγρὸν ἀνθρώπῳ ἑτέρῳ, οὐκέτι μὴ λυτρώσηται αὐτόν.
\VS{21}Ἀλλʼ ἔσται ὁ ἀγρὸς ἐξεληλυθυίας τῆς ἀφέσεως ἅγιος τῷ Κυρίῳ, ὥσπερ ἡ γῆ ἡ ἀφωρισμένη τῷ ἱερεῖ ἔσται κατάσχεσις αὐτοῦ.
\VS{22}Ἐὰν δὲ ἀπὸ τοῦ ἀγροῦ οὗ κέκτηται, ὃς οὐκ ἔστιν ἀπὸ τοῦ ἀγροῦ τῆς κατασχέσεως αὐτοῦ, ἁγιάσῃ τῷ Κυρίῳ,
\VS{23}λογιεῖται πρὸς αὐτὸν ὁ ἱερεὺς τὸ τέλος τῆς τιμῆς ἐκ τοῦ ἐνιαυτοῦ τῆς ἀφέσεως, καὶ ἀποδώσει τὴν τιμὴν ἐν τῇ ἡμέρᾳ ἐκείνῃ ἁγίαν τῷ Κυρίῳ·
\VS{24}Καὶ ἐν τῷ ἐνιαυτῷ τῆς ἀφέσεως ἀποδοθήσεται ὁ ἀγρὸς τῷ ἀνθρώπῳ παρʼ οὗ κέκτηται αὐτὸν, οὗ ἦν ἡ κατάσχεσις τῆς γῆς.
\VS{25}Καὶ πᾶσα τιμὴ ἔσται σταθμίοις ἁγίοις· εἴκοσι ὀβολοὶ ἔσται τὸ δίδραχμον.
\VS{26}Καὶ πᾶν πρωτότοκον ὃ ἐὰν γένηται ἐν τοῖς κτήνεσί σου, ἔσται τῷ Κυρίῳ, καὶ οὐ καθαγιάσει αὐτὸ οὐδείς· ἐάν τε μόσχον, ἐάν τε πρόβατον, τῷ Κυρίῳ ἐστίν.
\VS{27}Ἐὰν δὲ τῶν τετραπόδων τῶν ἀκαθάρτων ἀλλάξῃ κατὰ τὴν τιμὴν αὐτοῦ, καὶ προσθήσει τὸ ἐπίπεμπτον πρὸς αὐτὸ, καὶ ἔσται αὐτῷ· ἐὰν δὲ μὴ λυτρῶται, πραθήσεται κατὰ τὸ τίμημα αὐτοῦ.
\par }{\PP \VS{28}Πᾶν δὲ ἀνάθεμα, ὃ ἂν ἀναθῇ ἄνθρωπος τῷ Κυρίῳ ἀπὸ πάντων, ὅσα αὐτῷ ἐστιν, ἀπὸ ἀνθρώπου ἕως κτήνους, καὶ ἀπὸ ἀγροῦ κατασχέσεως αὐτοῦ, οὐκ ἀποδώσεται οὐδὲ λυτρώσεται· πᾶν ἀνάθεμα ἅγιον ἁγίων ἔσται τῷ Κυρίῳ.
\VS{29}Καὶ πᾶν ὃ ἐὰν ἀνατεθῇ ἀπὸ τῶν ἀνθρώπων, οὐ λυτρωθήσεται, ἀλλὰ θανάτῳ θανατωθήσεται.
\VS{30}Πᾶσα δεκάτη τῆς γῆς, ἀπὸ τοῦ σπέρματος τῆς γῆς, καὶ τοῦ καρποῦ τοῦ ξυλίνου, τῷ. Κυρίῳ ἐστίν, ἅγιον τῷ Κυρίῳ.
\VS{31}Ἐὰν δὲ λυτρῶται λύτρῳ ἄνθρωπος τὴν δεκάτην αὐτοῦ, τὸ ἐπίπεμπτον προσθήσει πρὸς αὐτὸν, καὶ ἔσται αὐτῷ.
\VS{32}Καὶ πᾶσα δεκάτη βοῶν, καὶ προβάτων, καὶ πᾶν ὃ ἂν ἔλθῃ ἐν τῷ ἀριθμῷ ὑπὸ τὴν ῥάβδον, τὸ δέκατον ἔσται ἅγιον τῷ Κυρίῳ.
\VS{33}Οὐκ ἀλλάξεις καλὸν πονηρῷ, οὐδὲ πονηρὸν καλῷ· ἐὰν δὲ ἀλλάσσων ἀλλάξῃς αὐτό, καὶ τὸ ἄλλαγμα αὐτοῦ ἔσται ἅγιον, οὐ λυτρωθήσεται.
\par }{\PP \VS{34}Αὗταί εἰσιν αἱ ἐντολαὶ ἃς ἐνετείλατο Κύριος τῷ Μωυσῇ πρὸς τοὺς υἱοὺς Ἰσραὴλ ἐν τῷ ὄρει Σινᾷ.
\par }