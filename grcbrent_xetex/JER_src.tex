\NormalFont\ShortTitle{ΙΕΡΕΜΙΑΣ}
{\MT ΙΕΡΕΜΙΑΣ

\par }\ChapOne{1}{\PP \VerseOne{1}ΤΟ ῥῆμα τοῦ Θεοῦ ὃ ἐγένετο ἐπὶ Ἱερεμίαν τὸν του Χελκίου, ἐκ τῶν ἱερέων, ὃς κατῴκει ἐν Ἀναθὼθ ἐν γῇ Βενιαμὶν,
\VS{2}ὡς ἐγενήθη λόγος τοῦ Θεοῦ πρὸς αὐτὸν, ἐν ταῖς ἡμέραις Ἰωσία υἱοῦ Ἀμὼς βασιλέως Ἰούδα, ἔτους τρισκαιδεκάτου ἐν τῇ βασιλείᾳ αὐτοῦ.
\VS{3}Καὶ ἐγένετο ἐν ταῖς ἡμέραις Ἰωακεὶμ υἱοῦ Ἰωσία βασιλέως Ἰούδα, ἕως ἑνδεκάτου ἔτους τοῦ Σεδεκία υἱοῦ Ἰωσία βασιλέως Ἰούδα, ἕως τῆς αἰχμαλωσίας Ἱερουσαλὴμ ἐν τῷ πέμπτῳ μηνί.
\par }{\PP \VS{4}Καὶ ἐγένετο λόγος Κυρίου πρὸς αὐτόν·
\VS{5}Πρὸ τοῦ με πλάσαι σε ἐν κοιλίᾳ, ἐπίσταμαί σε, καὶ πρὸ τοῦ σε ἐξελθεῖν ἐκ μήτρας, ἡγίακά σε, προφήτην εἰς ἔθνη τέθεικά σε.
\par }{\PP \VS{6}Καὶ εἶπα, ὁ ὢν δέσποτα, Κύριε, ἰδοὺ οὐκ ἐπίσταμαι λαλεῖν, ὅτι νεώτερος ἐγώ εἰμι.
\VS{7}Καὶ εἶπε Κύριος πρὸς μὲ, μὴ λέγε, ὅτι νεώτερος ἐγώ εἰμι, ὅτι πρὸς πάντας οὓς ἐὰν ἐξαποστείλω σε, πορεύσῃ, καὶ κατὰ πάντα ὅσα ἐὰν ἐντείλωμαί σοι, λαλήσεις.
\VS{8}Μὴ φοβηθῇς ἀπὸ προσώπου αὐτῶν, ὅτι μετὰ σοῦ ἐγώ εἰμι τοῦ ἐξαιρεῖσθαί σε, λέγει Κύριος.
\VS{9}Καὶ ἐξέτεινε Κύριος τὴν χεῖρα αὐτοῦ πρὸς μὲ, καὶ ἥψατο τοῦ στόματός μου, καὶ εἶπε Κύριος πρὸς μὲ, ἰδοὺ δέδωκα τοὺς λόγους μου εἰς τὸ στόμα σου.
\par }{\PP \VS{10}Ἰδοὺ καθέστακά σε σήμερον ἐπὶ ἔθνη καὶ ἐπὶ βασιλείας, ἐκριζοῦν, καὶ κατασκάπτειν, καὶ ἀπολύειν, καὶ ἀνοικοδομεῖν, καὶ καταφυτεύειν.
\par }{\PP \VS{11}Καὶ ἐγένετο λόγος Κυρίου πρὸς μὲ, λέγων, τί σὺ ὁρᾷς; καὶ εἶπα, βακτηρίαν καρυΐνην.
\VS{12}Καὶ εἶπε Κύριος πρὸς μὲ, καλῶς ἑώρακας, διότι ἐγρήγορα ἐγὼ ἐπὶ τοὺς λόγους μου τοῦ ποιῆσαι αὐτούς.
\VS{13}Καὶ ἐγένετο λόγος Κυρίου ἐκ δευτέρου πρὸς μὲ, λέγων, τί σὺ ὁρᾷς; καὶ εἶπα, λέβητα ὑποκαιόμενον, καὶ τὸ πρόσωπον αὐτοῦ ἀπὸ προσώπου Βοῤῥᾶ.
\VS{14}Καὶ εἶπε Κύριος πρὸς μὲ, ἀπὸ προσώπου Βοῤῥᾶ ἐκκαυθήσεται τὰ κακὰ ἐπὶ πάντας τοὺς κατοικοῦντας τὴν γῆν.
\VS{15}Διότι ἰδοὺ ἐγὼ συγκαλῶ πάσας τὰς βασιλείας τῆς γῆς ἀπὸ Βοῤῥᾶ, λέγει Κύριος· καὶ ἥξουσι καὶ θήσουσιν ἕκαστος τὸν θρόνον αὐτοῦ ἐπὶ τὰ πρόθυρα τῶν πυλῶν Ἱερουσαλὴμ, καὶ ἐπὶ πάντα τὰ τείχη τὰ κύκλῳ αὐτῆς, καὶ ἐπὶ πάσας τὰς πόλεις Ἰούδα.
\VS{16}Καὶ λαλήσω πρὸς αὐτοὺς μετὰ κρίσεως, περὶ πάσης τῆς κακίας αὐτῶν, ὡς ἐγκατέλιπόν με, καὶ ἔθυσαν θεοῖς ἀλλοτρίοις, καὶ προσεκύνησαν τοῖς ἔργοις τῶν χειρῶν αὐτῶν.
\par }{\PP \VS{17}Καὶ σὺ περίζωσαι τὴν ὀσφύν σου, καὶ ἀνάστηθι, καὶ εἰπὸ πάντα ὅσα ἄν ἐντείλωμαί σοι· μὴ φοβηθῇς ἀπὸ προσώπου αὐτῶν, μηδὲ πτοηθῇς ἐναντίον αὐτῶν, ὅτι μετὰ σοῦ εἰμι, τοῦ ἐξαιρεῖσθαί σε, λέγει Κύριος.
\VS{18}Ἰδοὺ τέθεικά σε ἐν τῇ σήμερον ἡμέρᾳ ὡς πόλιν ὀχυρὰν καὶ ὡς τεῖχος χαλκοῦν, ὀχυρὸν πᾶσι τοῖς βασιλεύσιν Ἰούδα, καὶ τοῖς ἄρχουσιν αὐτοῦ, καὶ τῷ λαῷ τῆς γῆς.
\VS{19}Καὶ πολεμήσουσί σε, καὶ οὐ μὴ δύνωνται πρὸς σὲ διότι μετὰ σοῦ ἐγώ εἰμι, τοῦ ἐξαιρεῖσθαί σε, εἶπε Κύριος.

\par }\Chap{2}{\PP \VS{2}Καὶ εἶπε, τάδε λέγει Κύριος, ἐμνήσθην ἐλέους νεότητός σου, καὶ ἀγάπης τελειώσεώς σου, τοῦ ἐξακολουθῆσαί σε τῷ ἁγίῳ Ἰσραὴλ, λέγει Κύριος.
\VS{3}Ὁ ἅγιος Ἰσραὴλ τῷ Κυρίῳ, ἀρχὴ γεννημάτων αὐτοῦ· πάντες οἱ ἔσθοντες αὐτὸν πλημμελήσουσι, κακὰ ἥξει ἐπʼ αὐτοὺς, φησὶ Κύριος.
\par }{\PP \VS{4}Ἀκούσατε λόγον Κυρίου οἶκος Ἰακὼβ, καὶ πᾶσα πατριὰ οἴκου Ἰσραήλ.
\VS{5}Τάδε λέγει Κύριος, τί εὕροσαν οἱ πατέρες ὑμῶν ἐν ἐμοὶ πλημμέλημα, ὅτι ἀπέστησαν μακρὰν ἀπʼ ἐμοῦ, καὶ ἐπορεύθησαν ὀπίσω τῶν ματαίων, καὶ ἐματαιώθησαν;
\VS{6}Καὶ οὐκ εἶπαν, ποῦ ἐστι Κύριος, ὁ ἀναγαγὼν ἡμᾶς ἐκ γῆς Αἰγύπτου, ὁ καθοδηγήσας ἡμᾶς ἐν τῇ ἐρήμῳ ἐν γῇ ἀπείρῳ καὶ ἀβάτῳ, ἐν γῇ ἀνύδρῳ καὶ ἀκάρπῳ, ἐν γῇ ᾗ οὐ διώδευσεν ἐν αὐτῇ ἀνὴρ εὐθὲν, καὶ οὐ κατῴκησεν ἄνθρωπος ἐκεῖ;
\VS{7}Καὶ ἤγαγον ὑμᾶς εἰς τὸν Κάρμηλον, τοῦ φαγεῖν ὑμᾶς τοὺς καρποὺς αὐτοῦ, καὶ τὰ ἀγαθὰ αὐτοῦ· καὶ εἰσήλθετε, καὶ ἐμιάνατε τὴν γῆν μου, καὶ τὴν κληρονομίαν μου ἔθεσθε εἰς βδέλυγμα.
\VS{8}Οἱ ἱερεῖς οὐκ εἶπαν, ποῦ ἐστι Κύριος; καὶ οἱ ἀντεχόμενοι τοῦ νόμου, οὐκ ἠπίσταντό με, καὶ οἱ ποιμένες ἠσέβουν εἰς ἐμὲ, καὶ οἱ προφῆται ἐπροφήτευον τῇ Βάαλ, καὶ ὀπίσω ἀνωφελοῦς ἐπορεύθησαν.
\par }{\PP \VS{9}Διατοῦτο ἔτι κριθήσομαι πρὸς ὑμᾶς, καὶ πρὸς τοὺς υἱοὺς τῶν υἱῶν ὑμῶν κριθήσομαι.
\VS{10}Διότι ἔλθετε εἰς νήσους Χεττιεὶμ, καὶ ἴδετε, καὶ εἰς Κηδὰρ ἀποστείλατε, καὶ νοήσατε σφόδρα, καὶ ἴδετε εἰ γέγονε τοιαῦτα·
\VS{11}εἰ ἀλλάξωνται ἔθνη θεοὺς αὐτῶν, καὶ οὗτοι οὐκ εἰσὶ θεοί· ὁ δὲ λαός μου ἠλλάξατο τὴν δόξαν αὐτοῦ, ἐξ ἧς οὐκ ὠφεληθήσονται.
\VS{12}Ἐξέστη ὁ οὐρανὸς ἐπὶ τούτῳ, καὶ ἔφριξεν ἐπὶ πλεῖον σφόδρα, λέγει Κύριος.
\VS{13}Ὅτι δύο καὶ πονηρὰ ἐποίησεν ὁ λαός μου· ἐμὲ ἐγκατέλιπον πηγὴν ὕδατος ζωῆς, καὶ ὤρυξαν ἑαυτοῖς λάκκους συντετριμμένους, οἳ οὐ δυνήσονται ὕδωρ συνέχειν.
\par }{\PP \VS{14}Μὴ δοῦλός ἐστιν Ἰσραὴλ, ἢ οἰκογενής ἐστι; διατί εἰς προνομὴν ἐγένετο;
\VS{15}Ἐπʼ αὐτὸν ὠρύοντο λέοντες, καὶ ἔδωκαν τὴν φωνὴν αὐτῶν, οἳ ἔταξαν τὴν γῆν αὐτοῦ εἰς ἔρημον, καὶ αἱ πόλεις αὐτοῦ κατεσκάφησαν, παρὰ τὸ μὴ κατοικεῖσθαι.
\VS{16}Καὶ υἱοὶ Μέμφεως καὶ Τάφνας ἔγνωσάν σε, καὶ κατέπαιζόν σου.
\VS{17}Οὐχὶ ταῦτα ἐποίησέ σοι τὸ καταλιπεῖν σε ἐμέ; λέγει Κύριος ὁ Θεός σου.
\par }{\PP \VS{18}Καὶ νῦν τί σοι καὶ τῇ ὁδῷ Αἰγύπτου τοῦ πιεῖν ὕδωρ Γηῶν; καὶ τί σοι καὶ τῇ ὁδῷ Ἀσσυρίων τοῦ πιεῖν ὕδωρ ποταμῶν;
\VS{19}Παιδεύσει σε ἡ ἀποστασία σου, καὶ ἡ κακία σου ἐλέγξει σε· καὶ γνῶθι, καὶ ἴδε, ὅτι πικρόν σοι τὸ καταλιπεῖν σε ἐμὲ, λέγει Κύριος ὁ Θεός σου· καὶ οὐκ εὐδόκησα ἐπὶ σοὶ, λέγει Κύριος ὁ Θεός σου.
\par }{\PP \VS{20}Ὅτι ἀπʼ αἰῶνος συνέτριψας τὸν ζυγόν σου, καὶ διέσπασας τοὺς δεσμούς σου, καὶ εἶπας, οὐ δουλεύσω σοι ἀλλὰ πορεύσομαι ἐπὶ πάντα βουνὸν ὑψηλὸν, καὶ ὑποκάτω παντὸς ξύλου κατασκίου, ἐκεῖ διαχυθήσομαι ἐν τῇ πορνείᾳ μου.
\VS{21}Ἐγὼ δὲ ἐφύτευσά σε ἄμπελον καρποφόρον πᾶσαν ἀληθινήν· πῶς ἐστράφης εἰς πικρίαν ἡ ἄμπελος ἡ ἀλλοτρία;
\VS{22}Ἐὰν ἀποπλύνῃ ἐν νίτρῳ, καὶ πληθυνῇς σεαυτῇ ποίαν, κεκηλίδωσαι ἐν ταῖς ἀδικίαις σου ἐναντίον ἐμοῦ, λέγει Κύριος.
\par }{\PP \VS{23}Πῶς ἐρεῖς, οὐκ ἐμιάνθην, καὶ ὀπίσω τῆς Βάαλ οὐκ ἐπορεύθην; ἴδε τὰς ὁδούς σου ἐν τῷ πολυανδρίῳ, καὶ γνῶθι τί ἐποίησας· ὀψὲ φωνὴ αὐτῆς ὠλόλυξε·
\VS{24}Τὰς ὁδοὺς αὐτῆς ἐπλάτυνεν ἐφʼ ὕδατα ἐρήμου, ἐν ἐπιθυμίαις ψυχῆς αὐτῆς ἐπνευματοφορεῖτο, παρεδόθη, τίς ἐπιστρέψει αὐτήν; πάντες οἱ ζητοῦντες αὐτὴν οὐ κοπιάσουσιν, ἐν τῇ ταπεινώσει αὐτῆς εὑρήσουσιν αὐτήν.
\VS{25}Ἀπόστρεψον τὸν πόδα σου ἀπὸ ὁδοῦ τραχείας, καὶ τὸν φάρυγγά σου ἀπὸ δίψους· ἡ δὲ εἶπεν ἀνδριοῦμαι, ὅτι ἠγαπήκει ἀλλοτρίους, καὶ ὀπίσω αὐτῶν ἐπορεύετο.
\par }{\PP \VS{26}Ὡς αἰσχύνη κλέπτου ὅταν ἁλῷ, οὕτως αἰσχυνθήσονται οἱ υἱοὶ Ἰσραὴλ, αὐτοὶ καὶ οἱ βασιλεῖς αὐτῶν, καὶ οἱ ἄρχοντες αὐτῶν, καὶ οἱ ἱερεῖς αὐτῶν, καὶ οἱ προφῆται αὐτῶν.
\VS{27}Τῷ ξύλῳ εἶπαν, ὅτι πατήρ μου εἶ σὺ, καὶ τῷ λίθῳ, σὺ ἐγέννησάς με· καὶ ἔστρεψαν ἐπʼ ἐμὲ νῶτα, καὶ οὐ πρόσωπα αὐτῶν· καὶ ἐν τῷ καιρῷ τῶν κακῶν αὐτῶν ἐροῦσιν, ἀνάστα καὶ σῶσον ἡμᾶς.
\VS{28}Καὶ ποῦ εἰσὶν οἱ θεοί σου, οὓς ἐποίησας σεαυτῷ; εἰ ἀναστήσονται καὶ σώσουσιν ἐν καιρῷ τῆς κακώσεώς σου; ὅτι κατʼ ἀριθμὸν τῶν πόλεών σου ἦσαν θεοί σου Ἰούδα, καὶ κατʼ ἀριθμὸν διόδων τῆς Ἱερουσαλὴμ ἔθυον τῇ Βάαλ.
\VS{29}Ἱνατί λαλεῖτε πρὸς μέ; πάντες ὑμεῖς ἠσεβήσατε, καὶ πάντες ὑμεῖς ἠνομήσατε εἰς ἐμὲ, λέγει Κύριος.
\VS{30}Μάτην ἐπάταξα τὰ τέκνα ὑμῶν, παιδείαν οὐκ ἐδέξασθε, μάχαιρα κατέφαγε τοὺς προφήτας ὑμῶν ὡς λέων ὀλοθρεύων, καὶ οὐκ ἐφοβήθητε.
\par }{\PP \VS{31}Ἀκούσατε λόγον Κυρίου· τάδε λέγει Κύριος, μὴ ἔρημος ἐγενόμην τῷ Ἰσραὴλ ἢ γῆ κεχερσωμένη; διατί εἶπεν ὁ λαός μου, οὐ κυριευθησόμεθα, καὶ οὐχ ἥξομεν πρὸς σὲ ἔτι;
\VS{32}Μὴ ἐπιλήσεται νύμφη τὸν κόσμον αὐτῆς, καὶ παρθένος τὴν στηθοδεσμίδα αὐτῆς; ὁ δὲ λαός μου ἐπελάθετό μου ἡμέρας ὧν οὐκ ἔστιν ἀριθμός.
\VS{33}Τί ἔτι καλὸν ἐπιτηδεύσεις ἐν ταῖς ὁδοῖς σου, τοῦ ζητῆσαι ἀγάπησιν; οὐχ οὕτως· ἀλλὰ καὶ σὺ ἐπονηρεύσω τοῦ μιάναι τὰς ὁδούς σου,
\VS{34}καὶ ἐν ταῖς χερσί σου εὑρέθησαν αἵματα ψυχῶν ἀθώων· οὐκ ἐν διορύγμασιν εὗρον αὐτοὺς, ἀλλʼ ἐπὶ πάσῃ δρυΐ.
\VS{35}Καὶ εἶπας, ἀθῶός εἰμι, ἀλλὰ ἀποστραφήτω ὁ θυμὸς αὐτοῦ ἀπʼ ἐμοῦ.
\par }{\PP Ἰδοὺ ἐγὼ κρίνομαι πρὸς σὲ, ἐν τῷ λέγειν σε, οὐχ ἥμαρτον·
\VS{36}Ὅτι κατεφρόνησας σφόδρα τοῦ δευτερῶσαι τὰς ὁδούς σου· καὶ ἀπὸ Αἰγύπτου καταισχυνθήσῃ, καθὼς κατῃσχύνθης ἀπὸ Ἀσσούρ·
\VS{37}ὅτι καὶ ἐντεῦθεν ἐξελεύσῃ, καὶ αἱ χεῖρές σου ἐπὶ τῆς κεφαλῆς σου· ὅτι ἀπώσατο Κύριος τὴν ἐλπίδα σου, καὶ οὐκ εὐοδωθήσῃ ἐν αὐτῇ.

\par }\Chap{3}{\PP \VerseOne{1}Ἐὰν ἐξαποστείλῃ ἀνὴρ τὴν γυναῖκα αὐτοῦ, καὶ ἀπέλθῃ ἀπʼ αὐτοῦ, καὶ γένηται ἀνδρὶ ἑτέρῳ, μὴ ἀνακάμπτουσα ἀνακάμψει πρὸς αὐτὸν ἔτι; οὐ μιαινομένη μιανθήσεται ἡ γυνὴ ἐκείνη; καὶ σὺ ἐξεπόρνευσας ἐν ποιμέσι πολλοῖς, καὶ ἀνέκαμπτες πρὸς μὲ, λέγει Κύριος.
\VS{2}Ἆρον τοὺς ὀφθαλμούς σου εἰς εὐθεῖαν, καὶ ἴδε, ποῦ οὐχὶ ἐξεφύρθης· ἐπὶ ταῖς ὁδοῖς ἐκάθισας αὐτοῖς ὡσεὶ κορώνη ἐρημουμένη, καὶ ἐμίανας τὴν γῆν ἐν ταῖς πορνείαις σου καὶ ἐν ταῖς κακίαις σου,
\VS{3}καὶ ἔσχες ποιμένας πολλοὺς εἰς πρόσκομμα σεαυτῇ· ὄψις πόρνης ἐγένετό σου, ἀπηναισχύντησας πρὸς πάντας.
\par }{\PP \VS{4}Οὐχ ὡς οἶκόν με ἐκαλέσας, καὶ πατέρα καὶ ἀρχηγὸν τῆς παρθενίας σου;
\VS{5}Μὴ διαμενεῖ εἰς τὸν αἰῶνα, ἢ φυλαχθήσεται εἰς νῖκος; ἰδοὺ ἐλάλησας, καὶ ἐποίησας τὰ πονηρὰ ταῦτα, καὶ ἠδυνάσθης.
\par }{\PP \VS{6}Καὶ εἶπε Κύριος πρὸς μὲ ἐν ταῖς ἡμέραις Ἰωσείου τοῦ βασιλέως, εἶδες ἃ ἐποίησέ μοι ἡ κατοικία τοῦ Ἰσραήλ; ἐπορεύθησαν ἐπὶ πᾶν ὄρος ὑψηλὸν, καὶ ὑποκάτω παντὸς ξύλου ἀλσώδους, καὶ ἐπόρνευσαν ἐκεῖ.
\VS{7}Καὶ εἶπα, μετὰ τὸ πορνεῦσαι αὐτὴν ταῦτα πάντα, πρὸς μὲ ἀνάστρεψον· καὶ οὐκ ἀνέστρεψε· καὶ εἶδε τὴν ἀσυνθεσίαν αὐτῆς ἡ ἀσύνθετος Ἰούδα.
\VS{8}Καὶ εἶδον, ὅτι περὶ πάντων ὧν κατελήφθη ἐν οἷς ἐμοιχᾶτο ἡ κατοικία Ἰσραὴλ, καὶ ἐξαπέστειλα αὐτὴν, καὶ ἔδωκα αὐτῇ βιβλίον ἀποστασίου εἰς τὰς χεῖρας αὐτῆς· καὶ οὐκ ἐφοβήθη ἡ ἀσύνθετος Ἰούδα, καὶ ἐπορεύθη, καὶ ἐπόρνευσε καὶ αὐτὴ,
\VS{9}και ἐγένετο εἰς οὐθὲν ἡ πορνεία αὐτῆς, καὶ ἐμοίχευσε τὸ ξύλον καὶ τὸν λίθον.
\VS{10}Καὶ ἐν πᾶσι τούτοις οὐκ ἐπεστράφη πρὸς μὲ ἡ ἀσύνθετος Ἰούδα ἐξ ὅλης τῆς καρδίας αὐτῆς, ἀλλʼ ἐπὶ ψεύδει.
\par }{\PP \VS{11}Καὶ εἶπε Κύριος πρὸς μὲ, ἐδικαίωσε τὴν ψυχὴν αὐτοῦ Ἰσραὴλ ἀπὸ τῆς ἀσυνθέτου Ἰούδα.
\VS{12}Πορεύου καὶ ἀνάγνωθι τοὺς λόγους τούτους πρὸς Βοῤῥᾶν, καὶ ἐρεῖς, ἐπιστράφηθι πρὸς μὲ ἡ κατοικία τοῦ Ἰσραὴλ, λέγει Κύριος· καὶ μὴ στηριῶ τὸ πρόσωπόν μου ἐφʼ ὑμᾶς, ὅτι ἐλεήμων ἐγώ εἰμι, λέγει Κύριος, καὶ οὐ μηνιῶ ὑμῖν εἰς τὸν αἰῶνα.
\VS{13}Πλὴν, γνῶθι τὴν ἀδικίαν σου, ὅτι εἰς Κύριον τὸν Θεόν σου ἠσέβησας, καὶ διέχεας τὰς ὁδούς σου εἰς ἀλλοτρίους ὑποκάτω παντὸς ξύλου ἀλσώδους, τῆς δὲ φωνῆς μου οὐχ ὑπήκουσας, λέγει Κύριος.
\VS{14}Ἐπιστράφητε υἱοὶ ἀφεστηκότες, λέγει Κύριος, διότι ἐγὼ κατακυριεύσω ὑμῶν, καὶ λήψομαι ὑμᾶς ἕνα ἐκ πόλεως καὶ δύο ἐκ πατριᾶς, καὶ εἰσάξω ὑμᾶς εἰς Σιὼν,
\VS{15}καὶ δώσω ὑμῖν ποιμένας κατὰ τὴν καρδίαν μου, καὶ ποιμανοῦσιν ὑμᾶς ποιμαίνοντες μετʼ ἐπιστήμης.
\par }{\PP \VS{16}Καὶ ἔσται ἐὰν πληθυνθῆτε, καὶ αὐξηθῆτε ἐπὶ τῆς γῆς, λέγει Κύριος, ἐν ταῖς ἡμέραις ἐκείναις οὐκ ἐροῦσιν ἔτι, κιβωτὸς διαθήκης ἁγίου Ἰσραὴλ, οὐκ ἀναβήσεται ἐπὶ καρδίαν, οὐκ ὀνομασθήσεται, οὐδὲ ἐπισκεφθήσεται, καὶ οὐ ποιηθήσεται ἔτι.
\VS{17}Ἐν ταῖς ἡμέραις ἐκείναις καὶ ἐν τῷ καιρῷ ἐκείνῳ καλέσουσι τὴν Ἱερουσαλὴμ, Θρόνον Κυρίου· καὶ συναχθήσονται πάντα τὰ ἔθνη εἰς αὐτὴν, καὶ οὐ πορεύσονται ἔτι ὀπίσω τῶν ἐνθυμημάτων τῆς καρδίας αὐτῶν τῆς πονηρᾶς.
\par }{\PP \VS{18}Ἐν ταῖς ἡμέραις ἐκείναις συνελεύσονται ὁ οἶκος Ἰούδα ἐπὶ τὸν οἶκον τοῦ Ἰσραὴλ, καὶ ἥξουσιν ἐπιτοαυτὸ ἀπὸ γῆς Βοῤῥᾶ, καὶ ἀπὸ πασῶν τῶν χωρῶν ἐπὶ τὴν γῆν, ἣν κατεκληρονόμησα τοὺς πατέρας αὐτῶν.
\VS{19}Καὶ ἐγὼ εἶπα, γένοιτο Κύριε· ὅτι τάξω σε εἰς τέκνα, καὶ δώσω σοι γῆν ἐκλεκτὴν, κληρονομίαν Θεοῦ παντοκράτορος ἐθνῶν· καὶ εἶπα, πατέρα καλέσετέ με, καὶ ἀπʼ ἐμοῦ οὐκ ἀποστραφήσεσθε.
\VS{20}Πλὴν ὡς ἀθετεῖ γυνὴ εἰς τὸν συνόντα αὐτῇ, οὕτως ἠθέτησεν εἰς ἐμὲ ὁ οἶκος Ἰσραὴλ, λέγει Κύριος.
\par }{\PP \VS{21}Φωνὴ ἐκ χειλέων ἠκούσθη κλαυθμοῦ καὶ δεήσεως υἱῶν Ἰσραὴλ, ὅτι ἠδίκησαν ἐν ταῖς ὁδοῖς αὐτῶν, ἐπελάθοντο Θεοῦ ἁγίου αὐτῶν.
\VS{22}Ἐπιστράφητε υἱοὶ ἐπιστρέφοντες, καὶ ἰάσομαι τὰ συντρίμματα ὑμῶν.
\par }{\PP Ἰδοὺ δοῦλοι ἡμεῖς ἐσόμεθά σοι· ὅτι σὺ Κύριος ὁ Θεὸς ἡμῶν εἶ.
\VS{23}Ὄντως εἰς ψεῦδος ἦσαν οἱ βουνοὶ, καὶ ἡ δύναμις τῶν ὀρέων, πλὴν διὰ Κυρίου Θεοῦ ἡμῶν ἡ σωτηρία τοῦ Ἰσραήλ.
\VS{24}Ἡ δὲ αἰσχύνη κατηνάλωσε τοὺς μόχθους τῶν πατέρων ἡμῶν, ἀπὸ νεότητος ἡμῶν, τὰ πρόβατα αὐτῶν καὶ τοὺς μόσχους αὐτῶν καὶ τοὺς υἱοὺς αὐτῶν καὶ τὰς θυγατέρας αὐτῶν.
\VS{25}Ἐκοιμήθημεν ἐν τῇ αἰσχύνῃ ἡμῶν, καὶ ἐπεκάλυψεν ἡμᾶς ἡ ἀτιμία ἡμῶν, διότι ἔναντι τοῦ Θεοῦ ἡμῶν ἡμάρτομεν ἡμεῖς, καὶ οἱ πατέρες ἡμῶν, ἀπὸ νεότητος ἡμῶν ἕως τῆς ἡμέρας ταύτης· καὶ οὐχ ὑπηκούσαμεν τῆς φωνῆς Κυρίου τοῦ Θεοῦ ἡμῶν.

\par }\Chap{4}{\PP \VerseOne{1}Ἐὰν ἐπιστραφῇ Ἰσραὴλ, λέγει Κύριος, πρὸς μὲ, ἐπιστραφήσεται· καὶ ἐὰν περιέλῃ τὰ βδελύγματα αὐτοῦ ἐκ στόματος αὐτοῦ, καὶ ἀπὸ τοῦ προσώπου μου εὐλαβηθῇ,
\VS{2}καὶ ὀμόσῃ, ζῇ Κύριος, μετὰ ἀληθείας ἐν κρίσει καὶ ἐν δικαιοσύνῃ, καὶ εὐλογήσουσιν ἐν αὐτῷ ἔθνη, καὶ ἐν αὐτῷ αἰνέσουσι τῷ Θεῷ ἐν Ἰερουσαλήμ.
\par }{\PP \VS{3}Ὅτι τάδε λέγει Κύριος τοῖς ἀνδράσιν Ἰούδα, καὶ τοῖς κατοικοῦσιν Ἱερουσαλὴμ, νεώσατε ἑαυτοῖς νεώματα, καὶ μὴ σπείρητε ἐπʼ ἀκάνθαις.
\VS{4}Περιτμήθητε τῷ Θεῷ ὑμῶν, καὶ περιτέμεσθε τὴν σκληροκαρδίαν ὑμῶν ἄνδρες Ἰούδα, καὶ οἱ κατοικοῦντες Ἱερουσαλὴμ, μὴ ἐξέλθῃ ὡς πῦρ ὁ θυμός μου, καὶ ἐκκαυθήσεται, καὶ οὐκ ἔσται ὁ σβέσων, ἀπὸ προσώπου πονηρίας ἐπιτηδευμάτων ὑμῶν.
\par }{\PP \VS{5}Ἀναγγείλατε ἐν τῷ Ἰούδᾳ, καὶ ἀκουσθήτω ἐν Ἱερουσαλήμ· εἴπατε, σημάνατε ἐπὶ τῆς γῆς σάλπιγγι, κεκράξατε μέγα· εἴπατε, συνάχθητε, καὶ εἰσέλθωμεν εἰς τὰς πόλεις τὰς τειχήρεις.
\VS{6}Ἀναλαβόντες φεύγετε εἰς Σιών· σπεύσατε, μὴ στῆτε, ὅτι κακὰ ἐγὼ ἐπάγω ἀπὸ Βοῤῥα, καὶ συντριβὴν μεγάλην.
\VS{7}Ἀνέβη λέων ἐκ τῆς μάνδρας αὐτοῦ, ἐξολοθρεύων ἔθνη ἐξῇρε, καὶ ἐξῆλθεν ἐκ τοῦ τόπου αὐτοῦ, τοῦ θεῖναι τὴν γῆν εἰς ἐρήμωσιν· καὶ αἱ πόλεις καθαιρεθήσονται, παρὰ τὸ μὴ κατοικεῖσθαι αὐτάς.
\VS{8}Ἐπὶ τούτοις περιζώσασθε σάκκους, καὶ κόπτεσθε, καὶ ἀλαλάξατε, διότι οὐκ ἀπεστράφη ὁ θυμὸς Κυρίου ἀφʼ ὑμῶν.
\VS{9}Καὶ ἔσται ἐν ἐκείνῃ τῇ ἡμέρᾳ, λέγει Κύριος, ἀπολεῖται ἡ καρδία τοῦ βασιλέως, καὶ ἡ καρδία τῶν ἀρχόντων, καὶ οἱ ἱερεῖς ἐκστήσονται, καὶ οἱ προφῆται θαυμάσονται.
\par }{\PP \VS{10}Καὶ εἶπα, ὦ δέσποτα Κύριε, ἆρά γε ἀπατῶν ἠπάτησας τὸν λαὸν τοῦτον καὶ τὴν Ἱερουσαλὴμ, λέγων, εἰρήνη ἔσται, καὶ ἰδοὺ ἥψατο ἡ μάχαιρα ἕως τῆς ψυχῆς αὐτῶν.
\par }{\PP \VS{11}Ἐν τῷ καιρῷ ἐκείνῳ ἐροῦσι τῷ λαῷ τούτῳ καὶ τῇ Ἱερουσαλὴμ, πνεῦμα πλανήσεως ἐν τῇ ἐρήμῳ, ὁδὸς τῆς θυγατρὸς τοῦ λαοῦ μου, οὐκ εἰς καθαρὸν, οὐδʼ εἰς ἅγιον.
\VS{12}Πνεῦμα πληρώσεως ἥξει μοι· νῦν δὲ ἐγὼ λαλῶ κρίματά μου πρὸς αὐτούς.
\VS{13}Ἰδοὺ ὡς νεφέλη ἀναβήσεται, καὶ ὡς καταιγὶς τὰ ἄρματα αὐτοῦ, κουφότεροι ἀετῶν οἱ ἵπποι αὐτοῦ· οὐαὶ ἡμῖν, ὅτι ταλαιπωροῦμεν.
\par }{\PP \VS{14}Ἀπόπλυνε ἀπὸ κακίας τὴν καρδίαν σου Ἱερουσαλὴμ, ἵνα σωθῇς· ἕως πότε ὑπάρχουσιν ἐν σοὶ διαλογισμοὶ πόνων σου;
\VS{15}Διότι φωνὴ ἀγγέλλοντος ἐκ Δὰν ἥξει, καὶ ἀκουσθήσεται πόνος ἐξ ὄρους Ἐφραίμ.
\VS{16}Ἀναμνήσατε ἔθνη, ἰδοὺ ἥκασιν· ἀναγγείλατε ἐν Ἱερουσαλὴμ, συστροφαὶ ἔρχονται ἐκ γῆς μακρόθεν, καὶ ἔδωκαν ἐπὶ τὰς πόλεις Ἰούδα φωνὴν αὐτῶν.
\VS{17}Ὡς φυλάσσοντες ἀγρὸν, ἐγένοντο ἐπʼ αὐτὴν κύκλῳ, ὅτι ἐμοῦ ἠμέλησας, λέγει Κύριος.
\VS{18}Αἱ ὁδοί σου καὶ τὰ ἐπιτηδεύματά σου ἐποίησαν ταῦτά σοι· αὕτη ἡ κακία σου, ὅτι πικρὰ, ὅτι ἥψατο ἕως τῆς καρδίας σου.
\par }{\PP \VS{19}Τὴν κοιλίαν μου, τὴν κοιλίαν μου ἀλγῶ, καὶ τὰ αἰσθητήρια τῆς καρδίας μου, μαιμάσσει ἡ ψυχή μου, σπαράσσεται ἡ καρδία μου· οὐ σιωπήσομαι, ὅτι φωνὴν σάλπιγγος ἤκουσεν ἡ ψυχή μου, κραυγὴν πολέμου
\VS{20}καὶ ταλαιπωρίας συντριμμὸν ἐπικαλεῖται, ὅτι τεταλαιπώρηκε πᾶσα ἡ γῆ, ἄφνω τεταλαιπώρηκεν ἡ σκηνὴ, διεσπάσθησαν αἱ δέῤῥεις μου.
\VS{21}Ἕως πότε ὄψομαι φεύγοντας ἀκούων φωνὴν σαλπίγγων;
\par }{\PP \VS{22}Διότι οἱ ἡγούμενοι τοῦ λαοῦ μου ἐμὲ οὐκ ᾔδεισαν· υἱοὶ ἄφρονές εἰσι καὶ οὐ συνετοὶ, σοφοί εἰσι τοῦ κακοποιῆσαι, τὸ δὲ καλῶς ποιῆσαι οὐκ ἐπέγνωσαν.
\par }{\PP \VS{23}Ἐπέβλεψα ἐπὶ τὴν γῆν, καὶ ἰδοὺ οὐθὲν, καὶ εἰς τὸν οὐρανὸν, καὶ οὐκ ἦν τὰ φῶτα αὐτοῦ.
\VS{24}Εἶδον τὰ ὄρη, καὶ ἦν τρέμοντα, καὶ πάντας τοὺς βουνοὺς ταρασσομένους.
\VS{25}Ἐπέβλεψα, καὶ ἰδοὺ οὐκ ἦν ἄνθρωπος, καὶ πάντα τὰ πετεινὰ τοῦ οὐρανοῦ ἐπτοεῖτο.
\VS{26}Εἶδον, καὶ ἰδοὺ ὁ Κάρμηλος ἔρημος, καὶ πᾶσαι αἱ πόλεις ἐμπεπυρισμέναι πυρὶ ἀπὸ προσώπου Κυρίου, καὶ ἀπὸ προσώπου ὀργῆς θυμοῦ αὐτοῦ ἠφανίσθησαν.
\par }{\PP \VS{27}Τάδε λέγει Κύριος, ἔρημος ἔσται πᾶσα ἡ γῆ, συντέλειαν δὲ οὐ μὴ ποιήσω.
\VS{28}Ἐπὶ τούτοις πενθείτω ἡ γῆ, καὶ συσκοτασάτω ὁ οὐρανὸς ἄνωθεν· διότι ἐλάλησα, καὶ οὐ μετανοήσω, ὥρμησα, καὶ οὐκ ἀποστρέψω ἀπʼ αὐτῆς.
\VS{29}Ἀπὸ φωνῆς ἱππέως, καὶ ἐντεταμένου τόξου ἀνεχώρησε πᾶσα ἡ χώρα· εἰσέδυσαν εἴς τὰ σπήλαια, καὶ εἰς τὰ ἄλση ἐκρύβησαν, καὶ ἐπὶ τὰς πέτρας ἀνέβησαν· πᾶσα πόλις ἐγκατελείφθη, οὐ κατῴκει ἐν αὐταῖς ἄνθρωπος.
\VS{30}Καὶ σὺ τί ποιήσεις; ἐὰν περιβάλῃ κόκκινον, καὶ κοσμήσῃ κόσμῳ χρυσῷ· ἐὰν ἐγχρίσῃ στίβι τοὺς ὀφθαλμούς σου, εἰς μάταιον ὡραϊσμός σου· ἀπώσαντό σε οἱ ἐρασταί σου, τὴν ψυχήν σου ζητοῦσιν.
\par }{\PP \VS{31}Ὅτι φωνὴν ὡς ὠδινούσης ἤκουσα τοῦ στεναγμοῦ σου, ὡς πρωτοτοκούσης· φωνὴ θυγατρὸς Σιὼν ἐκλυθήσεται, καὶ παρήσει τὰς χεῖρας αὐτῆς· οἴμοι ἐγὼ, ὅτι ἐκλείπει ἡ ψυχή μου ἐπὶ τοῖς ἀνῃρημένοις.

\par }\Chap{5}{\PP \VerseOne{1}Περιδράμετε ἐν ταῖς ὁδοῖς Ἱερουσαλὴμ, καὶ ἴδετε, καὶ γνῶτε, καὶ ζητήσατε ἐν ταῖς πλατείαις αὐτῆς, ἐὰν εὕρητε, εἰ ἔστι ποιῶν κρίμα, καὶ ζητῶν πίστιν· καὶ ἵλεως ἔσομαι αὐτοῖς, λέγει Κύριος.
\VS{2}Ζῇ Κύριος, λέγουσι, διατοῦτο οὐκ ἐν ψεύδεσιν ὀμνύουσι;
\VS{3}Κύριε, οἱ ὀφθαλμοί σου εἰς πίστιν· ἐμαστίγωσας αὐτοὺς, καὶ οὐκ ἐπόνεσαν, συνετέλεσας αὐτοὺς, καὶ οὐκ ἠθέλησαν δέξασθαι παιδείαν· ἐστερέωσαν τὰ πρόσωπα αὐτῶν ὑπὲρ πέτραν, καὶ οὐκ ἠθέλησαν ἐπιστραφῆναι.
\VS{4}Καὶ ἐγὼ εἶπα, ἴσως πτωχοί εἰσι, διότι οὐκ ἐδυνάσθησαν, ὅτι οὐκ ἔγνωσαν ὁδὸν Κυρίου, καὶ κρίσιν Θεοῦ.
\VS{5}Πορεύσομαι πρὸς τοὺς ἁδροὺς, καὶ λαλήσω αὐτοῖς, ὅτι αὐτοὶ ἐπέγνωσαν ὁδὸν Κυρίου, καὶ κρίσιν Θεοῦ· καὶ ἰδοὺ ὁμοθυμαδὸν συνέτριψαν ζυγὸν, διέῤῥηξαν δεσμούς.
\par }{\PP \VS{6}Διατοῦτο ἔπαισεν αὐτοὺς λέων ἐκ τοῦ δρυμοῦ, καὶ λύκος ἕως τῶν οἰκιῶν ὠλόθρευσεν αὐτοὺς, καὶ πάρδαλις ἐγρηγόρησεν ἐπὶ τὰς πόλεις αὐτῶν· πάντες οἱ ἐκπορευόμενοι ἀπʼ αὐτῶν θηρευθήσονται, ὅτι ἐπλήθυναν ἀσεβείας αὐτῶν, ἴσχυσαν ἐν ταῖς ἀποστροφαῖς αὐτῶν.
\VS{7}Ποίᾳ τούτων ἵλεως γένωμαί σοι; οἱ υἱοί σου ἐγκατέλιπόν με, καὶ ὤμνυον ἐν τοῖς οὐκ οὖσι θεοῖς· καὶ ἐχόρτασα αὐτοὺς, καὶ ἐμοιχῶντο, καὶ ἐν οἴκοις πορνῶν κατέλυον.
\VS{8}Ἵπποι θηλυμανεῖς ἐγενήθησαν, ἕκαστος ἐπὶ τὴν γυναῖκα τοῦ πλησίον αὐτοῦ ἐχρεμέτιζον.
\VS{9}Μὴ ἐπὶ τούτοις οὐκ ἐπισκέψομαι; λέγει Κύριος· ἢ ἐν ἔθνει τοιούτῳ οὐκ ἐκδικήσει ἡ ψυχή μου;
\par }{\PP \VS{10}Ἀνάβητε ἐπὶ τοὺς προμαχῶνας αὐτῆς, καὶ κατασκάψατε, συντέλειαν δὲ μὴ ποιήσητε· ὑπολίπεσθε τὰ ὑποστηρίγματα αὐτῆς, ὅτι τοῦ Κυρίου εἰσίν.
\VS{11}Ὅτι ἀθετῶν ἠθέτησεν εἰς ἐμὲ, λέγει Κύριος, οἶκος Ἰσραὴλ, καὶ οἶκος Ἰούδα
\VS{12}ἐψεύσατο τῷ Κυρίῳ αὐτῶν, καὶ εἶπαν, οὐκ ἔστι ταῦτα· οὐχ ἥξει ἐφʼ ἡμᾶς κακὰ, καὶ μάχαιραν καὶ λιμὸν οὐκ ὀψόμεθα.
\VS{13}Οἱ προφῆται ἡμῶν ἦσαν εἰς ἄνεμον, καὶ λόγος Κυρίου οὐχ ὑπῆρχεν ἐν αὐτοῖς.
\par }{\PP \VS{14}Διατοῦτο τάδε λέγει Κύριος παντοκράτωρ, ἀνθʼ ὧν ἐλαλήσατε τὸ ῥῆμα τοῦτο, ἰδοὺ ἐγὼ δέδωκα τοὺς λόγους μου εἰς τὸ στόμα σου πῦρ, καὶ τὸν λαὸν τοῦτον ξύλα, καὶ καταφάγεται αὐτούς.
\VS{15}Ἰδοὺ ἐγὼ ἐπάγω ἐφʼ ὑμᾶς ἔθνος πόῤῥωθεν, οἶκος Ἰσραὴλ, λέγει Κύριος, ἔθνος οὗ οὐκ ἀκούσει τῆς φωνῆς τῆς γλώσσης αὐτοῦ.
\VS{16}Πάντες ἰσχυροὶ, καὶ κατέδονται τὸν θερισμὸν ὑμῶν,
\VS{17}καὶ τοὺς ἄρτους ὑμῶν, καὶ κατέδονται τοὺς υἱοὺς ὑμῶν, καὶ τὰς θυγατέρας ὑμῶν, καὶ κατέδονται τὰ πρόβατα ὑμῶν, καὶ τοὺς μόσχους ὑμῶν, καὶ κατέδονται τοὺς ἀμπελῶνας ὑμῶν, καὶ τοὺς συκῶνας ὑμῶν, καὶ τοὺς ἐλαιῶνας ὑμῶν· καὶ ἀλοήσουσι τὰς πόλεις τὰς ὀχυρὰς ὑμῶν, ἐφʼ αἷς ὑμεῖς πεποίθατε ἐπʼ αὐταῖς, ἐν ῥομφαίᾳ.
\VS{18}Καὶ ἔσται ἐν ταῖς ἡμέραις ἐκείναις, λέγει Κύριος ὁ Θεός σου, οὐ μὴ ποιήσω ὑμᾶς εἰς συντέλειαν.
\par }{\PP \VS{19}Καὶ ἔσται ὅταν εἴπητε, τίνος ἕνεκεν ἐποίησε Κύριος ὁ Θεὸς ἡμῶν ἡμῖν πάντα ταῦτα; καὶ ἐρεῖς αὐτοῖς, ἀνθʼ ὧν ἐδουλεύσατε θεοῖς ἀλλοτρίοις ἐν τῇ γῇ ὑμῶν, οὕτως δουλεύσετε ἀλλοτρίοις ἐν τῇ γῇ οὐχ ὑμῶν.
\par }{\PP \VS{20}Ἀναγγείλατε ταῦτα εἰς τὸν οἶκον Ἰακὼβ, καὶ ἀκουσθήτω ἐν τῷ οἴκῳ Ἰούδα.
\VS{21}Ἀκούσατε δὴ ταῦτα λαὸς μωρὸς καὶ ἀκάρδιος, ὀφθαλμοὶ αὐτοῖς καὶ οὐ βλέπουσιν, ὦτα αὐτοῖς καὶ οὐκ ἀκούουσι.
\VS{22}Μὴ ἐμὲ οὐ φοβηθήσεσθε; λέγει Κύριος· ἢ ἀπὸ προσώπου μου οὐκ εὐλαβηθήσεσθε; τὸν τάξαντα ἄμμον ὅριον τῇ θαλάσσῃ, πρόσταγμα αἰώνιον, καὶ οὐχ ὑπερβήσεται αὐτὸ, καὶ ταραχθήσεται, καὶ οὐ δυνήσεται· καὶ ἠχήσουσι τὰ κύματα αὐτῆς, καὶ οὐχ ὑπερβήσεται αὐτό.
\par }{\PP \VS{23}Τῷ δὲ λαῷ τούτῳ ἐγενήθη καρδία ἀνήκοος καὶ ἀπειθὴς, καὶ ἐξέκλιναν καὶ ἀπήλθοσαν,
\VS{24}καὶ οὐκ εἶπον ἐν τῇ καρδίᾳ αὐτῶν, φοβηθῶμεν δὴ Κύριον τὸν Θεὸν ἡμῶν, τὸν διδόντα ἡμῖν ὑετὸν πρώϊμον καὶ ὄψιμον, κατὰ καιρὸν πληρώσεως προστάγματος θερισμοῦ, καὶ ἐφύλαξεν ἡμῖν.
\VS{25}Αἱ ἀνομίαι ὑμῶν ἐξέκλιναν ταῦτα, καὶ αἱ ἁμαρτίαι ὑμῶν ἀπέστησαν τὰ ἀγαθὰ ἀφʼ ὑμῶν.
\VS{26}Ὅτι εὑρέθησαν ἐν τῷ λαῷ μου ἀσεβεῖς, καὶ παγίδας ἔστησαν τοῦ διαφθεῖραι ἄνδρας, καὶ συνελαμβάνοσαν.
\par }{\PP \VS{27}Ὡς παγὶς ἐφεσταμένη πλήρης πετεινῶν, οὕτως οἱ οἶκοι αὐτῶν πλήρεις δόλου· διατοῦτο ἐμεγαλύνθησαν, καὶ ἐπλούτησαν,
\VS{28}καὶ παρέβησαν κρίσιν, οὐκ ἔκριναν κρίσιν ὀρφανοῦ, καὶ κρίσιν χήρας οὐκ ἐκρίνοσαν.
\VS{29}Μὴ ἐπὶ τούτοις οὐκ ἐπισκέψομαι; λέγει Κύριος· ἢ ἐν ἔθνει τῷ τοιούτῳ οὐκ ἐκδικήσει ἡ ψυχή μου;
\par }{\PP \VS{30}Ἔκστασις καὶ φρικτὰ ἐγενήθη ἐπὶ τῆς γῆς.
\VS{31}Οἱ προφῆται προφητεύουσιν ἄδικα, καὶ οἱ ἱερεῖς ἐπεκρότησαν ταῖς χερσὶν αὐτῶν, καὶ ὁ λαός μου ἠγάπησεν οὕτως· καὶ τί ποιήσετε εἰς τὰ μετὰ ταῦτα;

\par }\Chap{6}{\PP \VerseOne{1}Ἐνισχύσατε υἱοὶ Βενιαμὶν ἐκ μέσου τῆς Ἱερουσαλὴμ, καὶ ἐν Θεκουὲ σημάνατε σάλπιγγι, καὶ ὑπὲρ Βαιθαχαρμὰ ἄρατε σημεῖον, ὅτι κακὰ ἐκκέκυφεν ἀπὸ Βοῤῥᾶ, καὶ συντριβὴ μεγάλη γίνεται·
\VS{2}καὶ ἀφαιρεθήσεται τὸ ὕψος θύγατερ Σιών.
\VS{3}Εἰς αὐτὴν ἥξουσι ποιμένες, καὶ τὰ ποίμνια αὐτῶν· καὶ πήξουσιν ἐπʼ αὐτὴν σκηνὰς κύκλῳ, καὶ ποιμανοῦσιν ἕκαστος τῇ χειρὶ αὐτοῦ.
\par }{\PP \VS{4}Παρασκευάσασθε ἐπʼ αὐτὴν εἰς πόλεμον, ἀνάστητε, καὶ ἀναβῶμεν ἐπʼ αὐτὴν μεσημβρίας· οὐαὶ ἡμῖν, ὅτι κέκλικεν ἡ ἡμέρα, ὅτι ἐκλείπουσιν αἱ σκιαὶ τῆς ἡμέρας.
\VS{5}Ἀνάστητε, καὶ ἀναβῶμεν ἐπʼ αὐτὴν νυκτὶ, καὶ διαφθείρωμεν τὰ θεμέλια αὐτῆς.
\par }{\PP \VS{6}Ὅτι τάδε λέγει Κύριος, ἔκκοψον τὰ ξύλα αὐτῆς, ἔκχεον ἐπὶ Ἱερουσαλὴμ δύναμιν· ὦ πόλις ψευδὴς, ὅλη καταδυναστεία ἐν αὐτῇ.
\VS{7}Ὡς ψύχει λάκκος ὕδωρ, οὕτω ψύχει κακία αὐτῆς· ἀσέβεια καὶ ταλαιπωρία ἀκουσθήσεται ἐν αὐτῇ ἐπὶ πρόσωπον αὐτῆς διαπαντός.
\par }{\PP \VS{8}Πόνῳ καὶ μάστιγι παιδευθήσῃ Ἱερουσαλὴμ, μὴ ἀποστῇ ἡ ψυχή μου ἀπὸ σοῦ, μὴ ποιήσω σε ἄβατον γῆν, ἥτις οὐ κατοικισθῇ.
\par }{\PP \VS{9}Ὅτι τάδε λέγει Κύριος, καλαμᾶσθε, καλαμᾶσθε ὡς ἄμπελον τὰ κατάλοιπα τοῦ Ἰσραὴλ, ἐπιστρέψατε ὡς ὁ τρυγῶν ἐπὶ τὸν κάρταλλον αὐτοῦ.
\VS{10}Πρὸς τίνα λαλήσω, καὶ διαμαρτύρωμαι, καὶ εἰσακούσεται; ἰδοὺ ἀπερίτμητα τὰ ὦτα αὐτῶν, καὶ οὐ δυνήσονται ἀκούειν· ἰδοὺ τὸ ῥῆμα Κυρίου ἐγένετο αὐτοῖς εἰς ὀνειδισμὸν, οὐ μὴ βουληθῶσιν αὐτό.
\par }{\PP \VS{11}Καὶ τὸν θυμόν μου ἔπλησα, καὶ ἐπέσχον, καὶ οὐ συνετέλεσα αὐτούς· ἐκχεῶ ἐπὶ νήπια ἔξωθεν, καὶ ἐπὶ συναγωγὴν νεανίσκων ἅμα, ὅτι ἀνὴρ καὶ γυνὴ συλληφθήσονται, πρεσβύτερος μετὰ πλήρους ἡμερῶν.
\VS{12}Καὶ μεταστραφήσονται αἱ οἰκίαι αὐτῶν εἰς ἑτέρους, ἀγροὶ καὶ αἱ γυναῖκες αὐτῶν ἐπιτοαυτό· ὅτι ἐκτενῶ τὴν χεῖρά μου ἐπὶ τοὺς κατοικοῦντας τὴν γῆν ταύτην, λέγει Κύριος.
\par }{\PP \VS{13}Ὅτι ἀπὸ μικροῦ αὐτῶν καὶ ἕως μεγάλου πάντες συνετελέσαντο ἄνομα, ἀπὸ ἱερέως καὶ ἕως ψευδοπροφήτου πάντες ἐποίησαν ψευδῆ.
\VS{14}Καὶ ἰῶντο σύντριμμα τοῦ λαοῦ μου, ἐξουθενοῦντες καὶ λέγοντες, εἰρήνη εἰρήνη· καὶ ποῦ ἐστιν εἰρήνη;
\VS{15}Κατῃσχύνθησαν, ὅτι ἐξελίποσαν· καὶ οὐδʼ ὡς καταισχυνόμενοι κατῃσχύνθησαν, καὶ τὴν ἀτιμίαν αὐτῶν οὐκ ἔγνωσαν· διατοῦτο πεσοῦνται ἐν τῇ πτώσει αὐτῶν, καὶ ἐν καιρῷ ἐπισκοπῆς ἀπολοῦνται, εἶπε Κύριος.
\par }{\PP \VS{16}Τάδε λέγει Κύριος, στῆτε ἐπὶ ταῖς ὁδοῖς, καὶ ἴδετε, καὶ ἐρωτήσατε τρίβους Κυρίου αἰωνίους· καὶ ἴδετε ποία ἐστὶν ἡ ὁδὸς ἡ ἀγαθὴ, καὶ βαδίζετε ἐν αὐτῇ, καὶ εὑρήσετε ἁγνισμὸν ταῖς ψυχαῖς ὑμῶν· Καὶ εἶπαν, οὐ πορευσόμεθα.
\VS{17}Καθέστακα ἐφʼ ὑμᾶς σκοπούς· ἀκούσατε τῆς φωνῆς τῆς σάλπιγγος· καὶ εἶπαν, οὐκ ἀκουσόμεθα.
\par }{\PP \VS{18}Διὰ τοῦτο ἤκουσαν τὰ ἔθνη, καὶ οἱ ποιμαίνοντες τὰ ποίμνια αὐτῶν.
\VS{19}Ἄκουε γῆ, ἰδοὺ ἐγὼ ἐπάγω ἐπὶ τὸν λαὸν τοῦτον κακὰ, τὸν καρπὸν ἀποστροφῆς αὐτῶν, ὅτι τῶν λόγων μου οὐ προσέσχον, καὶ τὸν νόμον μου ἀπώσαντο.
\VS{20}Ἱνατί μοι λίβανον ἐκ Σαβὰ φέρετε, καὶ κινάμωμον ἐκ γῆς μακρόθεν; τὰ ὁλοκαυτώματα ὑμῶν οὐκ εἰσὶ δεκτὰ, καὶ αἱ θυσίαι ὑμῶν οὐχ ἥδυνάν μοι.
\VS{21}Διατοῦτο τάδε λέγει Κύριος, ἰδοὺ ἐγὼ δίδωμι ἐπὶ τὸν λαὸν τοῦτον ἀσθένειαν, καὶ ἀσθενήσουσι πατέρες καὶ υἱοὶ ἅμα, γείτων καὶ ὁ πλησίον αὐτοῦ ἀπολοῦνται.
\par }{\PP \VS{22}Τάδε λέγει Κύριος, ἰδοὺ λαὸς ἔρχεται ἀπὸ Βοῤῥᾶ, καὶ ἔθνη ἐξεγερθήσονται ἀπʼ ἐσχάτου τῆς γῆς.
\VS{23}Τόξον καὶ ζιβύνην κρατήσουσιν· ἰταμός ἐστι, καὶ οὐκ ἐλεήσει· φωνὴ αὐτοῦ, ὡς θάλασσα κυμαίνουσα· ἐφʼ ἵπποις καὶ ἅρμασι παρατάξεται ὡς πῦρ εἰς πόλεμον πρὸς σὲ, θύγατερ Σιών.
\par }{\PP \VS{24}Ἠκούσαμεν τὴν ἀκοὴν αὐτῶν, παρελύθησαν αἱ χεῖρες ἡμῶν, θλίψις κατέσχεν ἡμᾶς, ὠδῖνες ὡς τικτούσης.
\VS{25}Μὴ ἐκπορεύεσθε εἰς ἀγρὸν, καὶ ἐν ταῖς ὁδοῖς μὴ βαδίζετε, ὅτι ῥομφαία τῶν ἐχθρῶν παροικεῖ κυκλόθεν.
\VS{26}Θύγατερ λαοῦ μου περίζωσαι σάκκον, κατάπασσε ἐν σποδῷ, πένθος ἀγαπητοῦ ποιήσαι σεαυτῇ κοπετὸν οἰκτρὸν, ὅτι ἐξαίφνης ἥξει ταλαιπωρία ἐφʼ ὑμᾶς.
\par }{\PP \VS{27}Δοκιμαστὴν δέδωκά σε ἐν λαοῖς δεδοκιμασμένοις, καὶ γνώσῃ με ἐν τῷ δοκιμάσαι με τὴν ὁδὸν αὐτῶν,
\VS{28}πάντες ἀνήκοοι πορεύομενοι σκολιῶς· χαλκὸς καὶ σίδηρος, πάντες διεφθαρμένοι εἰσίν.
\VS{29}Ἐξέλιπε φυσητὴρ ἀπὸ πυρὸς, ἐξέλιπε μόλιβος· εἰς κενὸν ἀργυροκόπος ἀργυροκοπεῖ, πονηρία αὐτῶν οὐκ ἐτάκη.
\VS{30}Ἀργύριον ἀποδεδοκιμασμένον καλέσατε αὐτοὺς, ὅτι ἀπεδοκίμασεν αὐτοὺς Κύριος.

\par }\Chap{7}{\PP \VS{2}Ἀκούσατε λόγον Κυρίου πᾶσα Ἰουδαία.
\VS{3}Τάδε λέγει Κύριος ὁ Θεὸς Ἰσραὴλ, διορθώσατε τὰς ὁδοὺς ὑμῶν, καὶ τὰ ἐπιτηδεύματα ὑμῶν, καὶ κατοικιῶ ὑμᾶς ἐν τῷ τόπῳ τούτῳ.
\VS{4}Μὴ πεποίθατε ἐφʼ ἑαυτοῖς ἐπὶ λόγοις ψευδέσιν, ὅτι τὸ παράπαν οὐκ ὠφελήσουσιν ὑμᾶς, λέγοντες, ναὸς Κυρίου, ναὸς Κυρίου ἐστίν.
\par }{\PP \VS{5}Ὅτι ἐὰν διορθοῦντες διορθώσητε τὰς ὁδοὺς ὑμῶν καὶ τὰ ἐπιτηδεύματα ὑμῶν, καὶ ποιοῦντες ποιήσητε κρίσιν ἀναμέσον ἀνδρὸς, καὶ ἀναμέσον τοῦ πλησίον αὐτοῦ,
\VS{6}καὶ προσήλυτον καὶ ὀρφανὸν καὶ χήραν μὴ καταδυναστεύσητε, καὶ αἷμα ἀθῶον μὴ ἐκχέητε ἐν τῷ τόπῳ τούτῳ, καὶ ὀπίσω θεῶν ἀλλοτρίων μὴ πορεύησθε εἰς κακὸν ὑμῖν,
\VS{7}καὶ κατοικιῶ ὑμᾶς ἐν τῷ τόπῳ τούτῳ ἐν γῇ ᾗ ἔδωκα τοῖς πατράσιν ὑμῶν ἐξ αἰῶνος καὶ ἕως αἰῶνος.
\par }{\PP \VS{8}Εἰ δὲ ὑμεῖς πεποίθατε ἐπὶ λόγοις ψευδέσιν, ὅθεν οὐκ ὠφεληθήσεσθε,
\VS{9}καὶ φονεύετε, καὶ μοιχᾶσθε, καὶ κλέπτετε, καὶ ὀμνύετε ἐπʼ ἀδίκῳ, καὶ θυμιᾶτε τῇ Βάαλ, καὶ ἐπορεύεσθε ὀπίσω θεῶν ἀλλοτρίων, ὧν οὐκ οἴδατε, τοῦ κακῶς εἶναι ὑμῖν,
\VS{10}καὶ ἤλθετε καὶ ἔστητε ἐνώπιον ἐμοῦ ἐν τῷ οἴκῳ οὗ ἐπικέκληται τὸ ὄνομά μου ἐπʼ αὐτῷ, καὶ εἴπατε, ἀπεσχήμεθα τοῦ μὴ ποιεῖν πάντα τὰ βδελύγματα ταῦτα.
\VS{11}Μὴ σπήλαιον λῃστῶν ὁ οἶκός μου, οὗ ἐπικέκληται τὸ ὄνομά μου ἐπʼ αὐτῷ ἐκεῖ ἐνώπιον ὑμῶν; καὶ ἰδοὺ ἐγὼ ἑώρακα, λέγει Κύριος.
\VS{12}Ὅτι πορεύθητε εἰς τὸν τόπον μου τὸν ἐν Σηλὼ, οὗ κατεσκήνωσα τὸ ὄνομά μου ἐκεῖ ἔμπροσθεν, καὶ ἴδετε ἃ ἐποίησα αὐτῷ ἀπὸ προσώπου κακίας λαοῦ μου Ἰσραήλ.
\par }{\PP \VS{13}Καὶ νῦν ἀνθʼ ὧν ἐποιήσατε πάντα τὰ ἔργα ταῦτα, καὶ ἐλάλησα πρὸς ὑμᾶς καὶ οὐκ ἠκούσατέ μου, καὶ ἐκάλεσα ὑμᾶς καὶ οὐκ ἀπεκρίθητε,
\VS{14}τοίνυν κᾀγὼ ποιήσω τῷ οἴκῳ ᾧ ἐπικέκληται τὸ ὄνομά μου ἐπʼ αὐτῷ, ἐφʼ ᾧ ὑμεῖς πεποίθατε ἐπʼ αὐτῷ, καὶ τῷ τόπῳ ᾧ ἔδωκα ὑμῖν καὶ τοῖς πατράσιν ὑμῶν, καθὼς ἐποίησα τῇ Σηλώ.
\VS{15}Καὶ ἀποῤῥίψω ὑμᾶς ἀπὸ προσώπου μου, καθὼς ἀπέῤῥιψα τοὺς ἀδελφοὺς ὑμῶν, πᾶν τὸ σπέρμα Ἐφραΐμ.
\par }{\PP \VS{16}Καὶ σὺ μὴ προσεύχου περὶ τοῦ λαοῦ τούτου, καὶ μὴ ἀξιοῦ τοῦ ἐλεηθῆναι αὐτοὺς, καὶ μὴ εὔχου, καὶ μὴ προσέλθῃς μοι περὶ αὐτῶν, ὅτι οὐκ εἰσακούσομαι.
\VS{17}Ἢ οὐχ ὁρᾷς, τί αὐτοὶ ποιοῦσιν ἐν ταῖς πόλεσιν Ἰούδα, καὶ ἐν ταῖς ὁδοῖς Ἱερουσαλήμ;
\VS{18}Οἱ υἱοὶ αὐτῶν συλλέγουσι ξύλα, καὶ οἱ πατέρες αὐτῶν καίουσι πῦρ, καὶ αἱ γυναῖκες αὐτῶν τρίβουσι σταῖς, τοῦ ποιῆσαι καυῶνας τῇ στρατιᾷ τοῦ οὐρανοῦ, καὶ ἔσπεισαν σπονδὰς θεοῖς ἀλλοτρίοις, ἵνα παροργίσωσί με.
\VS{19}Μὴ ἐμὲ αὐτοὶ παροργίζουσι; λέγει Κύριος· οὐχὶ ἑαυτοὺς, ὅπως καταισχυνθῇ τὰ πρόσωπα αὐτῶν;
\par }{\PP \VS{20}Διατοῦτο τάδε λέγει Κύριος, ἰδοὺ ὀργὴ καὶ θυμός μου χεῖται ἐπὶ τὸν τόπον τοῦτον, καὶ ἐπὶ τοὺς ἀνθρώπους, καὶ ἐπὶ τὰ κτήνη, καὶ ἐπὶ πᾶν ξύλου τοῦ ἀγροῦ αὐτῶν, καὶ ἐπὶ τὰ γεννήματα τῆς γῆς, καὶ καυθήσεται καὶ οὐ σβεσθήσεται.
\par }{\PP \VS{21}Τάδε λέγει Κύριος, τὰ ὁλοκαυτώματα ὑμῶν συναγάγετε μετὰ τῶν θυσιῶν ὑμῶν, καὶ φάγετε κρέα.
\VS{22}Ὅτι οὐκ ἐλάλησα πρὸς τοὺς πατέρας ὑμῶν, καὶ οὐκ ἐνετειλάμην αὐτοῖς ἐν ἡμέρᾳ ᾗ ἀνήγαγον αὐτοὺς ἐκ γῆς Αἰγύπτου, περὶ ὁλοκαυτωμάτων καὶ θυσίας.
\VS{23}Ἀλλʼ ἢ τὸ ῥῆμα τοῦτο ἐνετειλάμην αὐτοῖς, λέγων, ἀκούσατε τῆς φωνῆς μου, καὶ ἔσομαι ὑμῖν εἰς Θεὸν, καὶ ὑμεῖς ἔσεσθέ μοι εἰς λαὸν, καὶ πορεύεσθε ἐν πάσαις ταῖς ὁδοῖς μου, αἷς ἂν ἐντείλωμαι ὑμῖν, ὅπως ἂν εὖ ᾖ ὑμῖν.
\VS{24}Καὶ οὐκ ἤκουσάν μου, καὶ οὐ προσέσχε τὸ οὖς αὐτῶν, ἀλλʼ ἐπορεύθησαν ἐν τοῖς ἐνθυμήμασι τῆς καρδίας αὐτῶν τῆς κακῆς, καὶ ἐγενήθησαν εἰς τὰ ὄπισθεν· καὶ οὐκ εἰς τὰ ἔμπροσθεν,
\VS{25}ἀφʼ ἧς ἡμέρας ἐξήλθοσαν οἱ πατέρες αὐτῶν ἐκ γῆς Αἰγύπτου, καὶ ἕως τῆς ἡμέρας ταύτης· καὶ ἐξαπέστειλα πρὸς ὑμᾶς πάντας τοὺς δούλους μου, τοὺς προφήτας, ἡμέρας καὶ ὄρθρου, καὶ ἀπέστειλα,
\VS{26}καὶ οὐκ εἰσήκουσάν μου, καὶ οὐ προσέσχε τὸ οὖς αὐτῶν, καὶ ἐσκλήρυναν τὸν τράχηλον αὐτῶν ὑπὲρ τοὺς πατέρας αὐτῶν.
\par }{\PP \VS{27}Καὶ ἐρεῖς αὐτοῖς τοῦτον τὸν λόγον,
\VS{28}τοῦτο τὸ ἔθνος ὃ οὐκ ἤκουσε τῆς φωνῆς Κυρίου, οὐδὲ ἐδέξατο παιδείαν, ἐξέλιπεν ἡ πίστις ἐκ στόματος αὐτῶν.
\par }{\PP \VS{29}Κεῖρε τὴν κεφαλήν σου, καὶ ἀπόῤῥιπτε, καὶ ἀνάλαβε ἐπὶ χειλέων θρῆνον, ὅτι ἀπεδοκίμασε Κύριος, καὶ ἀπώσατο τὴν γενεὰν τὴν ποιοῦσαν ταῦτα.
\VS{30}Ὅτι ἐποίησαν οἱ υἱοὶ Ἰούδα τὸ πονηρὸν ἐναντίον ἐμοῦ, λέγει Κύριος· ἔταξαν τὰ βδελύγματα αὐτῶν ἐν τῷ οἴκῳ, οὗ ἐπικέκληται τὸ ὄνομά μου ἐπʼ αὐτὸν, τοῦ μιάναι αὐτόν.
\VS{31}Καὶ ᾠκοδόμησαν τὸν βωμὸν τοῦ Ταφὲθ, ὅς ἐστιν ἐν φάραγγι υἱοῦ Ἐννὸμ, τοῦ κατακαίειν τοὺς υἱοὺς αὐτῶν καὶ τὰς θυγατέρας αὐτῶν ἐν πυρὶ, ὃ οὐκ ἐνετειλάμην αὐτοῖς, καὶ οὐ διενοήθην ἐν τῇ καρδίᾳ μου.
\par }{\PP \VS{32}Διατοῦτο ἰδοὺ ἡμέραι ἔρχονται, λέγει Κύριος, καὶ οὐκ ἐροῦσιν ἔτι, Βωμὸς τοῦ Ταφὲθ καὶ φάραγξ υἱοῦ Ἐννὸμ, ἀλλʼ ἡ φάραγξ τῶν ἀνῃρημένων· καὶ θάψουσιν ἐν τῷ Ταφὲθ, διὰ τὸ μὴ ὑπάρχειν τόπον.
\VS{33}Καὶ ἔσονται οἱ νεκροὶ τοῦ λαοῦ τούτου εἰς βρῶσιν τοῖς πετεινοῖς τοῦ οὐρανοῦ, καὶ τοῖς θηρίοις τῆς γῆς, καὶ οὐκ ἔσται ὁ ἀποσοβῶν.
\VS{34}Καὶ καταλύσω ἐκ πόλεων Ἰούδα καὶ ἐκ διόδων Ἱερουσαλὴμ φωνὴν εὐφραινομένων, καὶ φωνὴν χαιρόντων, φωνὴν νυμφίου, καὶ φωνὴν νύμφης, ὅτι εἰς ἐρήμωσιν ἔσται πᾶσα ἡ γῆ.

\par }\Chap{8}{\PP \VerseOne{1}Ἐν τῷ καιρῷ ἐκείνῳ, λέγει Κύριος, ἐξοίσουσι τὰ ὀστᾶ τῶν βασιλέων Ἰούδα, καὶ τὰ ὀστᾶ τῶν ἀρχόντων αὐτοῦ, καὶ τὰ ὀστᾶ τῶν ἱερέων, καὶ τὰ ὀστᾶ προφητῶν, καὶ τὰ ὀστᾶ τῶν κατοικούντων ἐν Ἱερουσαλὴμ ἐκ τῶν τάφων αὐτῶν,
\VS{2}καὶ ψύξουσιν αὐτὰ πρὸς τὸν ἥλιον καὶ τὴν σελήνην, καὶ πρὸς πάντας τοὺς ἀστέρας, καὶ πρὸς πᾶσαν τὴν στρατιὰν τοῦ οὐρανοῦ, ἃ ἠγάπησαν, καὶ οἷς ἐδούλευσαν, καὶ ὧν ἐπορεύθησαν ὀπίσω αὐτῶν, καὶ ὧν ἀντείχοντο, καὶ οἷς προσεκύνησαν αὐτοῖς· οὐ κοπήσονται, καὶ οὐ ταφήσονται, καὶ ἔσονται εἰς παράδειγμα ἐπὶ πρόσωπου τῆς γῆς,
\VS{3}ὅτι εἵλοντο τὸν θάνατον ἢ τὴν ζωὴν, καὶ πᾶσι τοῖς καταλοίποις τοῖς καταλειφθεῖσιν ἀπὸ τῆς γενεᾶς ἐκείνης, ἐν παντὶ τόπῳ οὗ ἐὰν ἐξώσω αὐτοὺς ἐκεῖ.
\par }{\PP \VS{4}Ὅτι τάδε λέγει Κύριος, μὴ ὁ πίπτων οὐκ ἀνίσταται; ἢ ὁ ἀποστρέφων οὐκ ἀναστρέφει;
\VS{5}Διατί ἀπέστρεψεν ὁ λαός μου οὗτος ἀποστροφὴν ἀναιδῆ, καὶ κατεκρατήθησαν ἐν τῇ προαιρέσει αὐτῶν, καὶ οὐκ ἠθέλησαν τοῦ ἐπιστρέψαι;
\VS{6}ἐνωτίσασθε δὴ, καὶ ἀκούσατε· οὐχ οὕτω λαλήσουσιν, οὐκ ἔστιν ἄνθρωπος ὁ μετανοῶν ἀπὸ τῆς κακίας αὐτοῦ, λέγων, τί ἐποίησα; διέλιπεν ὁ τρέχων ἀπὸ τοῦ δρόμου αὐτοῦ, ὡς ἵππος κάθιδρος ἐν χρεμετισμῷ αὐτοῦ.
\VS{7}Καὶ ἡ ἀσίδα ἐν τῷ οὐρανῷ ἔγνω τὸν καιρὸν αὐτῆς, τρυγὼν καὶ χελιδὼν ἀγροῦ, στρουθία ἐφύλαξαν καιροὺς εἰσόδων ἑαυτῶν, ὁ δὲ λαός μου οὗτος οὐκ ἔγνω τὰ κρίματα Κυρίου.
\par }{\PP \VS{8}Πῶς ἐρεῖτε, ὅτι σοφοί ἐσμεν ἡμεῖς, καὶ νόμος Κυρίου μεθʼ ἡμῶν ἐστιν; εἰς μάτην ἐγενήθη σχοῖνος ψευδὴς γραμματεῦσιν.
\VS{9}Ἠσχύνθησαν σοφοὶ, καὶ ἐπτοήθησαν καὶ ἑάλωσαν, ὅτι τὸν νόμον Κυρίου ἀπεδοκίμασαν· σοφία τίς ἐστιν ἐν αὐτοῖς;
\VS{10}Διατοῦτο δώσω τὰς γυναῖκας αὐτῶν ἑτέροις, καὶ τοὺς ἀγροὺς αὐτῶν τοῖς κληρονόμοις,
\VS{13}καὶ συνάξουσι τὰ γεννήματα αὐτῶν, λέγει Κύριος. Οὐκ ἔστι σταφυλὴ ἐν ταῖς ἀμπέλοις, καὶ οὐκ ἐστι σῦκα ἐν ταῖς συκαῖς, καὶ τὰ φύλλα κατεῤῥύηκεν.
\par }{\PP \VS{14}Ἐπὶ τί ἡμεῖς καθήμεθα; συνάχθητε, καὶ εἰσέλθωμεν εἰς τὰς πόλεις τὰς ὀχυρὰς, καὶ ἀποῤῥιφῶμεν ἐκεῖ, ὅτι ὁ Θεὸς ἀπέῤῥιψεν ἡμᾶς, καὶ ἐπότισεν ἡμᾶς ὕδωρ χολῆς, ὅτι ἡμάρτομεν ἐναντίον αὐτοῦ.
\VS{15}Συνήχθημεν εἰς εἰρήνην, καὶ οὐκ ἦν ἀγαθὰ, εἰς καιρὸν ἰάσεως, καὶ ἰδοὺ σπουδή.
\par }{\PP \VS{16}Ἐκ Δὰν ἀκουσόμεθα φωνὴν ὀξύτητος ἵππων αὐτοῦ· ἀπὸ φωνῆς χρεμετισμοῦ ἱππασίας ἵππων αὐτοῦ ἐσείσθη πᾶσα ἡ γῆ. καὶ ἥξει καὶ καταφάγεται τὴν γῆν, καὶ τὸ πλήρωμα αὐτῆς, πόλιν καὶ τοὺς κατοικοῦντας ἐν αὐτῇ.
\VS{17}Διότι ἰδοὺ ἐγὼ ἐξαποστέλλω εἰς ὑμᾶς ὄφεις θανατοῦντας, οἷς οὐκ ἔστιν ἐπᾴσαι, καὶ δήξονται ὑμᾶς
\VS{18}ἀνίατα μετʼ ὀδύνης καρδίας ὑμῶν ἀπορουμένης.
\par }{\PP \VS{19}Ἰδοὺ φωνὴ κραυγῆς θυγατρὸς λαοῦ μου ἀπὸ γῆς μακρόθεν· μὴ Κύριος οὐκ ἔστιν ἐν Σιών; ἢ βασιλεὺς οὐκ ἔστιν ἐκεῖ; διότι παρώργισάν με ἐν τοῖς γλυπτοῖς αὐτῶν, καὶ ἐν ματαίοις ἀλλοτρίοις.
\VS{20}Διῆλθε θέρος, παρῆλθεν ἀμητὸς, καὶ ἡμεῖς οὐ διεσώθημεν.
\par }{\PP \VS{21}Ἐπὶ συντρίμματι θυγατρὸς λαοῦ μου ἐσκοτώθην· ἐν ἀπορίᾳ κατίσχυσάν με ὠδῖνες ὡς τικτούσης.
\VS{22}Καὶ μὴ ῥητίνη οὐκ ἔστιν ἐν Γαλαὰδ, ἢ ἰατρὸς οὐκ ἔστιν ἐκεῖ; διατί οὐκ ἀνέβη ἴασις θυγατρὸς λαοῦ μου;
\par }{\PP \VS{23}Τίς δώσει κεφαλῇ μου ὕδωρ, καὶ ὀφθαλμοῖς μου πηγὴν δακρύων; καὶ κλαύσομαι τὸν λαόν μου τοῦτον ἡμέρας καὶ νυκτὸς, τοὺς τετραυματισμένους θυγατρὸς λαοῦ μου.

\par }\Chap{9}{\PP \VerseOne{1}Τίς δῴη μοι ἐν τῇ ἐρήμῳ σταθμὸν ἔσχατον, καὶ καταλείψω τὸν λαόν μου, καὶ ἀπελεύσομαι ἀπʼ αὐτῶν; ὅτι πάντες μοιχῶνται, σύνοδος ἀθετούντων,
\VS{2}καὶ ἐνέτειναν τὴν γλῶσσαν αὐτῶν ὡς τόξον· ψεῦδος, καὶ οὐ πίστις ἐνίσχυσεν ἐπὶ τῆς γῆς, ὅτι ἐκ κακῶν εἰς κακὰ ἐξήλθοσαν, καὶ ἐμὲ οὐκ ἔγνωσαν, φησὶ Κύριος.
\VS{3}Ἕκαστος ἀπὸ τοῦ πλησίον αὐτοῦ φυλάξασθε, καὶ ἐπʼ ἀδελφοῖς αὐτῶν μὴ πεποίθατε, ὅτι πᾶς ἀδελφὸς πτέρνῃ πτερνιεῖ, καὶ πᾶς φίλος δολίως πορεύσεται.
\VS{4}Ἕκαστος κατὰ τοῦ φίλου αὐτοῦ καταπαίξεται, ἀλήθειαν οὐ μὴ λαλήσωσι· μεμάθηκεν ἡ γλῶσσα αὐτῶν λαλεῖν ψευδῆ, ἠδίκησαν, καὶ οὐ διέλιπον τοῦ ἐπιστρέψαι.
\VS{5}Τόκος ἐπὶ τόκῳ, καὶ δόλος ἐπὶ δόλῳ· οὐκ ἤθελον εἰδέναι με, φησὶ Κύριος.
\par }{\PP \VS{6}Διατοῦτο τάδε λέγει Κύριος, ἰδοὺ ἐγὼ πυρώσω αὐτοὺς, καὶ δοκιμῶ αὐτούς· ὅτι ποιήσω ἀπὸ προσώπου πονηρίας θυγατρὸς λαοῦ μου.
\VS{7}Βολὶς τιτρώσκουσα ἡ γλῶσσα αὐτῶν, δόλια τὰ ῥήματα τοῦ στόματος αὐτῶν· τῷ πλησίον αὐτοῦ λαλεῖ εἰρηνικὰ, καὶ ἐν ἑαυτῷ ἔχει τὴν ἔχθραν.
\VS{8}Μὴ ἐπὶ τούτοις οὐκ ἐπισκέψομαι; λέγει Κύριος· ἢ ἐν λαῷ τοιούτῳ οὐκ ἐκδικήσει ἡ ψυχή μου;
\VS{9}Ἐπὶ τὰ ὄρη λάβετε κοπετὸν, καὶ ἐπὶ τὰς τρίβους τῆς ἐρήμου θρῆνον, ὅτι ἐξέλιπον παρὰ τὸ μὴ εἶναι ἀνθρώπους· οὐκ ἤκουσαν φωνὴν ὑπάρξεως ἀπὸ πετεινῶν τοῦ οὐρανοῦ, καὶ ἕως κτηνῶν, ἐξέστησαν, ᾤχοντο.
\VS{10}Καὶ δώσω τὴν Ἱερουσαλὴμ εἰς μετοικίαν, καὶ εἰς κατοικητήριον δρακόντων, καὶ τὰς πόλεις Ἰούδα εἰς ἀφανισμὸν θήσομαι, παρὰ τὸ μὴ κατοικεῖσθαι.
\par }{\PP \VS{11}Τίς ὁ ἄνθρωπος ὁ συνετὸς, καὶ συνέτω τοῦτο; καὶ ᾧ λόγος στόματος Κυρίου πρὸς αὐτὸν, ἀναγγειλάτω ὑμῖν, ἕνεκεν τίνος ἀπώλετο ἡ γῆ, ἀνήφθη, ὡς ἔρημος, παρὰ τὸ μὴ διοδεύεσθαι αὐτήν;
\VS{12}Καὶ εἶπε Κύριος πρὸς μὲ, διὰ τὸ ἐγκαταλιπεῖν αὐτοὺς τὸν νόμον μου, ὃν ἔδωκα πρὸ προσώπου αὐτῶν, καὶ οὐκ ἤκουσαν τῆς φωνῆς μου,
\VS{13}ἀλλʼ ἐπορεύθησαν ὀπίσω τῶν ἀρεστῶν τῆς καρδίας αὐτῶν τῆς κακῆς, καὶ ὀπίσω τῶν εἰδώλων ἃ ἐδίδαξαν αὐτοὺς οἱ πατέρες αὐτῶν.
\VS{14}Διατοῦτο τάδε λέγει Κύριος ὁ Θεὸς Ἰσραὴλ, ἰδοὺ ἐγὼ ψωμιῶ αὐτοὺς ἀνάγκας, καὶ ποτιῶ αὐτοὺς ὕδωρ χολῆς,
\VS{15}καὶ διασκορπιῶ αὐτοὺς ἐν τοῖς ἔθνεσιν, εἰς οὓς οὐκ ἐγίνωσκον αὐτοὶ καὶ οἱ πατέρες αὐτῶν, καὶ ἐπαποστελῶ ἐπʼ αὐτοὺς τὴν μάχαιραν, ἕως τοῦ ἐξαναλῶσαι αὐτοὺς ἐν αὐτῇ.
\par }{\PP \VS{16}Τάδε λέγει Κύριος, καλέσατε τὰς θρηνούσας, καὶ ἐλθέτωσαν, καὶ πρὸς τὰς σοφὰς ἀποστείλατε, καὶ φθεγξάσθωσαν,
\VS{17}καὶ λαβέτωσαν ἐφʼ ὑμᾶς θρῆνον, καὶ καταγαγέτωσαν οἱ ὀφθαλμοὶ ὑμῶν δάκρυα, καὶ τὰ βλέφαρα ὑμῶν ῥείτω ὕδωρ,
\VS{18}ὅτι φωνὴ οἰκτροῦ ἠκούσθη ἐν Σιών· πῶς ἐταλαιπωρήσαμεν, κατῃσχύνθημεν σφόδρα, ὅτι ἐγκατελίπομεν τὴν γῆν, καὶ ἀπεῤῥίψαμεν τὰ σκηνώματα ἡμῶν;
\VS{19}Ἀκούσατε δὴ γυναῖκες λόγον Θεοῦ, καὶ δεξάσθω τὰ ὦτα ὑμῶν λόγους στόματος αὐτοῦ, καὶ διδάξατε τὰς θυγατέρας ὑμῶν οἶκτον, καὶ γυνὴ τὴν πλησίον αὐτῆς θρῆνον.
\VS{20}Ὅτι ἀνέβη θάνατος διὰ τῶν θυρίδων ὑμῶν, εἰσῆλθεν εἰς τὴν γῆν ὑμῶν, τοῦ ἐκτρίψαι νήπια ἔξωθεν, καὶ νεανίσκους ἀπὸ τῶν πλατειῶν.
\VS{21}Καὶ ἔσονται οἱ νεκροὶ τῶν ἀνθρώπων εἰς παράδειγμα ἐπὶ προσώπου τοῦ πεδίου τῆς γῆς ὑμῶν, ὡς χόρτος ὀπίσω θερίζοντος, καὶ οὐκ ἔσται ὁ συνάγων.
\par }{\PP \VS{22}Τάδε λέγει Κύριος, μὴ καυχάσθω ὁ σοφὸς ἐν τῇ σοφίᾳ αὐτοῦ, καὶ μὴ καυχάσθω ὁ ἰσχυρὸς ἐν τῆ ἰσχύϊ αὐτοῦ, καὶ μὴ καυχάσθω ὁ πλούσιος ἐν τῷ πλούτῳ αὐτοῦ,
\VS{23}ἀλλʼ ἢ ἐν τούτῳ καυχάσθω ὁ καυχώμενος, συνιεῖν καὶ γινώσκειν, ὅτι ἐγώ εἰμι Κύριος ὁ ποιῶν ἔλεος καὶ κρίμα καὶ δικαιοσύνην ἐπὶ τῆς γῆς, ὅτι ἐν τούτοις τὸ θέλημά μου, λέγει Κύριος.
\par }{\PP \VS{24}Ἰδοὺ ἡμέραι ἔρχονται, λέγει Κύριος, καὶ ἐπισκέψομαι ἐπὶ πάντας περιτετμημένους ἀκροβυστίας αὐτῶν.
\VS{25}Ἐπʼ Αἴγυπτον, καὶ ἐπὶ Ἰδουμαίαν, καὶ ἐπὶ Ἐδὼμ, καὶ ἐπὶ υἱοὺς Ἀμμὼν, καὶ ἐπὶ υἱοὺς Μωὰβ, καὶ ἐπὶ πάντα περικειρόμενον τὰ κατὰ πρόσωπον αὐτοῦ, τοὺς κατοικοῦντας ἐν τῇ ἐρήμῳ, ὅτι πάντα τὰ ἔθνη ἀπερίτμητα σαρκὶ, καὶ πᾶς οἶκος Ἰσραὴλ ἀπερίτμητοι καρδίας αὐτῶν.

\par }\Chap{10}{\PP \VerseOne{1}Ἀκούσατε τὸν λόγον Κυρίου, ὃν ἐλάλησεν ἐφʼ ὑμᾶς, οἶκος Ἰσραήλ.
\par }{\PP \VS{2}Τάδε λέγει Κύριος, κατὰ τὰς ὁδοὺς τῶν ἐθνῶν μὴ μανθάνετε, καὶ ἀπὸ τῶν σημείων τοῦ οὐρανοῦ μὴ φοβεῖσθε, ὅτι φοβοῦνται αὐτὰ τοῖς προσώποις αὐτῶν.
\VS{3}Ὅτι τὰ νόμιμα τῶν ἐθνῶν μάταια· ξύλον ἐστὶν ἐκ τοῦ δρυμοῦ ἐκκεκομμένον, ἔργον τέκτονος, καὶ χώνευμα,
\VS{4}ἀργυρίῳ καὶ χρυσίῳ κεκαλλωπισμένα, ἐν σφύραις καὶ ἥλοις ἐστερέωσαν αὐτά·
\VS{5}Θήσουσιν αὐτὰ, καὶ οὐ κινηθήσονται· ἀργύριον τορευτόν ἐστιν, οὐ πορεύσονται,
\VS{9}ἀργύριον προσβλητόν ἐστιν. Ἀπὸ Θαρσεὶς ἥξει χρυσίον Μωφὰζ, καὶ χεὶρ χρυσοχόων, ἔργα τεχνιτῶν πάντα, ὑάκινθον καὶ πορφύραν ἐνδύσουσιν αὐτά.
\VS{9a}Αἰρόμενα ἀρθήσονται, ὅτι οὐκ ἐπιβήσονται· μὴ φοβηθῆτε αὐτὰ, ὅτι οὐ μὴ κακοποιήσωσι, καὶ ἀγαθὸν οὐκ ἔστιν ἐν αὐτοῖς.
\par }{\PP \VS{11}Οὕτως ἐρεῖτε αὐτοῖς, θεοὶ οἳ τὸν οὐρανὸν καὶ τὴν γῆν οὐκ ἐποίησαν, ἀπολέσθωσαν ἐκ τῆς γῆς, καὶ ὑποκάτωθεν τοῦ οὐρανοῦ τούτου.
\VS{12}Κύριος ὁ ποιήσας τὴν γῆν ἐν τῇ ἰσχύϊ αὐτοῦ, ὁ ἀνορθώσας τὴν οἰκουμένην ἐν τῇ σοφίᾳ αὐτοῦ, καὶ τῇ φρονήσει αὐτοῦ ἐξέτεινε τὸν οὐρανὸν,
\VS{13}καὶ πλῆθος ὕδατος ἐν οὐρανῷ· καὶ ἀνήγαγε νεφέλας ἐξ ἐσχάτου τῆς γῆς, ἀστραπὰς εἰς ὑετὸν ἐποίησε, καὶ ἐξήγαγε φῶς ἐκ θησαυρῶν αὐτοῦ.
\VS{14}Ἐμωράνθη πᾶς ἄνθρωπος ἀπὸ γνώσεως, κατῃσχύνθη πᾶς χρυσοχόος ἐπὶ τοῖς γλυπτοῖς αὐτοῦ, ὅτι ψευδῆ ἐχώνευσεν, οὐκ ἔστι πνεῦμα ἐν αὐτοῖς.
\VS{15}Μάταιά ἐστιν ἔργα ἐμπεπαιγμένα, ἐν καιρῷ ἐπισκοπῆς αὐτῶν ἀπολοῦνται.
\VS{16}Οὐκ ἔστι τοιαύτη μερὶς τῷ Ἰακὼβ, ὅτι ὁ πλάσας τὰ πάντα, αὐτὸς κληρονομία αὐτοῦ, Κύριος ὄνομα αὐτῷ.
\par }{\PP \VS{17}Συνήγαγεν ἔξωθεν τὴν ὑπόστασίν σου, κατοικοῦσαν ἐν ἐκλεκτοῖς.
\VS{18}Ὅτι τάδε λέγει Κύριος, ἰδοὺ ἐγὼ σκελίζω τοὺς κατοικοῦντας τὴν γῆν ταύτην ἐν θλίψει, ὅπως εὑρεθῇ ἡ πληγή σου.
\par }{\PP \VS{19}Οὐαὶ ἐπὶ συντρίμματί σου, ἀλγηρὰ ἡ πληγή σου· κᾀγὼ εἶπα, ὄντως τοῦτο τὸ τραῦμά σου, καὶ κατέλαβέ σε.
\VS{20}Ἡ σκηνή σου ἐταλαιπώρησεν, ὤλετο· καὶ πᾶσαι αἱ δέῤῥεις σου διεσπάσθησαν· οἱ υἱοί μου καὶ τὰ πρόβατά μου οὐκ εἰσὶν, οὐκ ἔστιν ἔτι τόπος τῆς σκηνῆς μου, τόπος τῶν δέῤῥεών μου.
\VS{21}Ὅτι οἱ ποιμένες ἠφρονεύσαντο, καὶ τὸν Κύριον οὐκ ἐζήτησαν· διατοῦτο οὐκ ἐνόησε πᾶσα ἡ νομὴ, καὶ διεσκορπίσθησαν,
\VS{22}φωνὴ ἀκοῆς ἰδοὺ ἔρχεται καὶ σειαμὸς μέγας ἐκ γῆς Βοῤῥᾶ, τοῦ τάξαι τὰς πόλεις Ἰούδα εἰς ἀφανισμὸν, καὶ κοίτην στρουθῶν.
\par }{\PP \VS{23}Οἶδα, Κύριε, ὅτι οὐχὶ τοῦ ἀνθρώπου ἡ ὁδὸς αὐτοῦ, οὐδὲ ἀνὴρ πορεύσεται καὶ κατορθώσει πορείαν αὐτοῦ.
\VS{24}Παίδευσον ἡμᾶς Κύριε, πλὴν ἐν κρίσει καὶ μὴ ἐν θυμῷ, ἵνα μὴ ὀλίγους ἡμᾶς ποιήσῃς.
\VS{25}Ἔκχεον τὸν θυμόν σου ἐπὶ ἔθνη τὰ μὴ εἰδότα σε, καὶ ἐπὶ γενεὰς αἳ τὸ ὄνομά σου οὐκ ἐπεκαλέσαντο, ὅτι κατέφαγον τὸν Ἰακὼβ καὶ ἐξανήλωσαν αὐτὸν, καὶ τὴν νομὴν αὐτοῦ ἠρήμωσαν.

\par }\Chap{11}{\PP \VerseOne{1}Ὁ ΛΟΓΟΣ Ὁ ΓΕΝΟΜΕΝΟΣ ΠΑΡΑ ΚΥΡΙΟΥ ΠΡΟΣ ἹΕΡΕΜΙΑΝ, ΛΕΓΩΝ,
\par }{\PP \VS{2}Ἀκούσατε τοὺς λόγους τῆς διαθήκης ταύτης, καὶ λαλήσεις πρὸς ἄνδρας Ἰούδα, καὶ πρὸς τοὺς κατοικοῦντας ἐν Ἱερουσαλὴμ,
\VS{3}καὶ ἐρεῖς πρὸς αὐτοὺς, τάδε λέγει Κύριος ὁ Θεὸς Ἰσραὴλ, ἐπικατάρατος ὁ ἄνθρωπος, ὃς οὐκ ἀκούσεται τῶν λόγων τῆς διαθήκης ταύτης,
\VS{4}ἧς ἐνετειλάμην τοῖς πατράσιν ὑμῶν, ἐν ἡμέρᾳ ᾗ ἀνήγαγον αὐτοὺς ἐκ γῆς Αἰγύπτου ἐκ καμίνου τῆς σιδηρᾶς, λέγων, ἀκούσατε τῆς φωνῆς μου, καὶ ποιήσατε πάντα ὅσα ἐὰν ἐντείλωμαι ὑμῖν, καὶ ἔσεσθέ μοι εἰς λαὸν, καὶ ἐγὼ ἔσομαι ὑμῖν εἰς Θεόν·
\VS{5}Ὅπως στήσω τὸν ὅρκον μου, ὃν ὤμοσα τοῖς πατράσιν ὑμῶν, τοῦ δοῦναι αὐτοῖς γῆν ῥέουσαν γάλα καὶ μέλι, καθὼς ἡ ἡμέρα αὕτη· καὶ ἀπεκρίθην καὶ εἶπα, γένοιτο Κύριε.
\VS{6}Καὶ εἶπε Κύριος πρὸς μὲ, ἀνάγνωθι τοὺς λόγους τούτους ἐν πόλεσιν Ἰούδα, καὶ ἔξωθεν Ἱερουσαλὴμ, λέγων, ἀκούσατε τοὺς λόγους τῆς διαθήκης ταύτης, καὶ ποιήσατε αὐτούς.
\VS{8}Καὶ οὐκ ἐποίησαν.
\par }{\PP \VS{9}Καὶ εἶπε Κύριος πρὸς μὲ, εὑρέθη σύνδεσμος ἐν ἀνδράσιν Ἰούδα, καὶ ἐν τοῖς κατοικοῦσιν ἐν Ἱερουσαλήμ·
\VS{10}Ἐπεστράφησαν ἐπὶ τὰς ἀδικίας τῶν πατέρων αὐτῶν τῶν πρότερον, οἳ οὐκ ἠθέλησαν εἰσακοῦσαι τῶν λόγων μου, καὶ ἰδοὺ αὐτοὶ πορεύονται ὀπίσω θεῶν ἀλλοτρίων, τοῦ δουλεύειν αὐτοῖς· καὶ διεσκέδασεν οἶκος Ἰσραὴλ καὶ οἶκος Ἰούδα τὴν διαθήκην μου, ἣν διεθέμην πρὸς τοὺς πατέρας αὐτῶν.
\par }{\PP \VS{11}Διατοῦτο τάδε λέγει Κύριος, ἰδοὺ ἐγὼ ἐπάγω ἐπὶ τὸν λαὸν τοῦτον κακὰ, ἐξ ὧν οὐ δυνήσονται ἐξελθεῖν ἐξ αὐτῶν· καὶ κεκράξονται πρὸς μὲ, καὶ οὐκ εἰσακούσομαι αὐτῶν.
\VS{12}Καὶ πορεύσονται πόλεις Ἰούδα καὶ οἱ κατοικοῦντες Ἱερουσαλὴμ, καὶ κεκράξονται πρὸς τοὺς θεοὺς, οἷς αὐτοὶ θυμιῶσιν αὐτοῖς, οἳ μὴ σώσουσιν αὐτοὺς ἐν τῷ καιρῷ τῶν κακῶν αὐτῶν.
\VS{13}Ὅτι κατʼ ἀριθμὸν τῶν πόλεών σου ἦσαν θεοί σου Ἰούδα, καὶ κατʼ ἀριθμὸν ἐξόδων τῆς Ἱερουσαλὴμ ἐτάξατε βωμοὺς θυμιᾷν τῇ Βάαλ.
\par }{\PP \VS{14}Καὶ σὺ μὴ προσεύχου περὶ τοῦ λαοῦ τούτου, καὶ μὴ ἀξίου περὶ αὐτῶν ἐν δεήσει καὶ προσευχῇ, ὅτι οὐκ εἰσακούσομαι ἐν τῷ καιρῷ ἐν ᾧ ἐπικαλοῦνταί με, ἐν καιρῷ κακώσεως αὐτῶν.
\VS{15}Τί ἡ ἠγαπημένη ἐν τῷ οἴκῳ μου ἐποίησε βδέλυγμα; μὴ εὐχαὶ καὶ κρέα ἅγια ἀφελοῦσιν ἀπὸ σοῦ τὰς κακίας σου, ἢ τούτοις διαφεύξῃ;
\VS{16}Ἐλαίαν ὡραίαν εὔσκιον τῷ εἴδει ἐκάλεσε Κύριος τὸ ὄνομά σου, εἰς φωνὴν περιτομῆς αὐτῆς· ἀνήφθη πῦρ ἐπʼ αὐτὴν, μεγάλη ἡ θλίψις ἐπὶ σὲ, ἠχρειώθησαν οἱ κλάδοι αὐτῆς·
\VS{17}Καὶ Κύριος ὁ καταφυτεύσας σε, ἐλάλησεν ἐπὶ σὲ κακὰ ἀντὶ τῆς κακίας οἴκου Ἰσραὴλ καὶ οἴκου Ἰούδα, ὅ, τι ἐποίησαν ἑαυτοῖς τοῦ παροργίσαι με ἐν τῷ θυμιᾷν αὐτοὺς τῇ Βάαλ.
\par }{\PP \VS{18}Κύριε γνώρισόν μοι, καὶ γνώσομαι· τότε εἶδον τὰ ἐπιτηδεύματα αὐτῶν.
\VS{19}Ἐγὼ δὲ ὡς ἀρνίον ἄκακον ἀγόμενον τοῦ θύεσθαι, οὐκ ἔγνων· ἐπʼ ἐμὲ ἐλογίσαντο λογισμὸν πονηρὸν, λέγοντες, δεῦτε καὶ ἐμβάλωμεν ξύλον εἰς τὸν ἄρτον αὐτοῦ, καὶ ἐκτρίψωμεν αὐτὸν ἀπὸ γῆς ζώντων, καὶ τὸ ὄνομα αὐτοῦ οὐ μὴ μνησθῇ οὐκέτι.
\VS{20}Κύριε κρίνων δίκαια, δοκιμάζων νεφροὺς καὶ καρδίας, ἴδοιμι τῆν παρὰ σοῦ ἐκδίκησιν ἐξ αὐτῶν, ὅτι πρὸς σὲ ἀπεκάλυψα τὸ δικαίωμά μου.
\par }{\PP \VS{21}Διατοῦτο τάδε λέγει Κύριος ἐπὶ τοὺς ἄνδρας Ἀναθὼθ τοὺς ζητοῦντας τὴν ψυχήν μου, τοὺς λέγοντας, οὐ μὴ προφητεύσεις ἐπὶ τῷ ὀνόματι Κυρίου, εἰ δὲ μὴ, ἀποθάνῃ ἐν ταῖς χερσὶν ἡμῶν·
\VS{22}Ἰδοὺ ἐγὼ ἐπισκέψομαι ἐπʼ αὐτούς· οἱ νεανίσκοι αὐτῶν ἐν μαχαίρᾳ ἀποθανοῦνται, καὶ οἱ υἱοὶ αὐτῶν καὶ αἱ θυγατέρες αὐτῶν τελευτήσουσιν ἐν λιμῷ,
\VS{23}καὶ ἐγκατάλειμμα οὐκ ἔσται αὐτῶν, ὅτι ἐπάξω κακὰ ἐπὶ τοὺς κατοικοῦντας ἐν Ἀναθὼθ, ἐν ἐνιαυτῷ ἐπισκέψεως αὐτῶν.

\par }\Chap{12}{\PP \VerseOne{1}Δίκαιος εἶ Κύριε, ὅτι ἀπολογήσομαι πρὸς σέ· πλὴν κρίματα λαλήσω πρὸς σέ· τί ὅτι ὁδὸς ἀσεβῶν εὐοδοῦται; εὐθήνησαν πάντες οἱ ἀθετοῦντες ἀθετήματα;
\VS{2}Ἐφύτευσας αὐτοὺς, καὶ ἐῤῥιζώθησαν· ἐτεκνοποιήσαντο, καὶ ἐποίησαν καρπόν· ἐγγὺς εἶ σὺ τοῦ στόματος αὐτῶν, καὶ πόῤῥω ἀπὸ τῶν νεφρῶν αὐτῶν.
\VS{3}Καὶ σύ Κύριε γινώσκεις με, δεδοκίμακας τὴν καρδίαν μου ἐναντίον σου· ἅγνισον αὐτοὺς εἰς ἡμέραν σφαγῆς αὐτῶν.
\VS{4}Ἕως πότε πενθήσει ἡ γῆ, καὶ πᾶς ὁ χόρτος τοῦ ἀγροῦ ξηρανθήσεται ἀπὸ κακίας τῶν κατοικούντων ἐν αὐτῆ; ἠφανίσθησαν κτήνη καὶ πετεινὰ, ὅτι εἶπαν, οὐκ ὄψεται ὁ Θεὸς ὁδοὺς ἡμῶν.
\par }{\PP \VS{5}Σοῦ οἱ πόδες τρέχουσι, καὶ ἐκλύουσί σε· πῶς παρασκευάσῃ ἐφʼ ἵπποις; καὶ ἐν γῇ εἰρήνης σου πέποιθας· πῶς ποιήσεις ἐν φρυάγματι τοῦ Ἰορδάνου;
\VS{6}Ὅτι καὶ οἱ ἀδελφοί σου καὶ ὁ οἶκος τοῦ πατρός σου, καὶ οὗτοι ἠθέτησάν σε, καὶ αὐτοὶ ἐβόησαν, ἐκ τῶν ὀπίσω σου ἐπισυνήχθησαν· μὴ πιστεύσῃς ἐν αὐτοῖς, ὅτι λαλήσουσι πρὸς σὲ καλά.
\par }{\PP \VS{7}Ἐγκαταλέλοιπα τὸν οἶκόν μου, ἀφῆκα τὴν κληρονομίαν μου, ἔδωκα τὴν ἠγαπημένην ψυχήν μου εἰς χεῖρας ἐχθρῶν αὐτῆς.
\VS{8}Ἐγενήθη ἡ κληρονομία μου ἐμοὶ ὡς λέων ἐν δρυμῷ, ἔδωκεν ἐπʼ ἐμὲ τὴν φωνὴν αὐτῆς, διατοῦτο ἐμίσησα αὐτήν.
\VS{9}Μὴ σπήλαιον ὑαίνης ἡ κληρονομία μου ἐμοὶ, ἢ σπήλαιον κύκλῳ αὐτῆς; βαδίσατε, συναγάγετε πάντα τὰ θηρία τοῦ ἀγροῦ, καὶ ἐλθέτωσαν τοῦ φαγεῖν αὐτήν.
\par }{\PP \VS{10}Ποιμένες πολλοὶ διέφθειραν τὸν ἀμπελῶνά μου, ἐμόλυναν τὴν μερίδα μου, ἔδωκαν τὴν μερίδα τὴν ἐπιθυμητήν μου εἰς ἔρημον ἄβατον,
\VS{11}ἐτέθη εἰς ἀφανισμὸν ἀπωλείας· διʼ ἐμὲ ἀφανισμῷ ἠφανίσθη πᾶσα ἡ γῆ, ὅτι οὐκ ἔστιν ἀνὴρ τιθέμενος ἐν καρδίᾳ.
\VS{12}Ἐπὶ πᾶσαν διεκβολὴν ἐν τῇ ἐρήμῳ ἦλθον ταλαιπωροῦντες, ὅτι μάχαιρα τοῦ Κυρίου καταφάγεται ἀπʼ ἄκρου τῆς γῆς ἕως ἄκρου τῆς γῆς· οὐκ ἔστιν εἰρήνη πάσῃ σαρκί.
\VS{13}Σπείρατε πυροὺς, καὶ ἀκάνθας θερίζετε· οἱ κλῆροι αὐτῶν οὐκ ὠφελήσουσιν αὐτούς· αἰσχύνθητε ἀπὸ καυχήσεως ὑμῶν, ἀπὸ ὀνειδισμοῦ ἔναντι Κυρίου.
\par }{\PP \VS{14}Ὅτι τάδε λέγει Κύριος περὶ πάντων τῶν γειτόνων τῶν πονηρῶν, τῶν ἁπτομένων τῆς κληρονομίας μου, ἧς ἐμέρισα τῷ λαῷ μου Ἰσραὴλ, ἰδοὺ ἐγὼ ἀποσπῶ αὐτοὺς ἀπὸ τῆς γῆς αὐτῶν, καὶ τὸν Ἰούδαν ἐκβαλῶ ἐκ μέσου αὐτῶν.
\par }{\PP \VS{15}Καὶ ἔσται μετὰ τὸ ἐκβαλεῖν με αὐτοὺς, ἐπιστρέψω καὶ ἐλεήσω αὐτοὺς, καὶ κατοικιῶ αὐτοὺς, ἕκαστον εἰς τὴν κληρονομίαν αὐτοῦ, καὶ ἕκαστον εἰς τὴν γῆν αὐτοῦ.
\VS{16}Καὶ ἔσται ἐὰν μαθόντες μάθωσι τὴν ὁδὸν τοῦ λαοῦ μου, τοῦ ὀμνύειν τῷ ὀνόματί μου, ζῇ Κύριος, καθὼς ἐδίδαξαν τὸν λαόν μου ὀμνύειν τῇ Βάαλ, καὶ οἰκοδομηθήσεται ἐν μέσῳ τοῦ λαοῦ μου.
\VS{17}Ἐὰν δὲ μὴ ἐπιστρέψωσι, καὶ ἐξαρῶ τὸ ἔθνος ἐκεῖνο ἐξάρσει καὶ ἀπωλείᾳ.

\par }\Chap{13}{\PP \VerseOne{1}Τάδε λέγει Κύριος, βάδισον καὶ κτῆσαι σεαυτῷ περίζωμα λινοῦν, καὶ περίθου περὶ τὴν ὀσφύν σου, καὶ ἐν ὕδατι οὐ διελεύσεται.
\VS{2}Καὶ ἐκτησάμην τὸ περίζωμα κατὰ τὸν λόγον Κυρίου, καὶ περιέθηκα περὶ τὴν ὀσφύν μου.
\VS{3}Καὶ ἐγενήθη λόγος Κυρίου πρὸς μὲ, λέγων,
\VS{4}λάβε τὸ περίζωμα τὸ περὶ τὴν ὀσφύν σου, καὶ ἀνάστηθι, καὶ βάδισον ἐπὶ τὸν Εὐφράτην, καὶ κατάκρυψον αὐτὸ ἐκεῖ ἐν τῇ τρυμαλιᾷ τῆς πέτρας.
\VS{5}Καὶ ἐπορεύθην, καὶ ἔκρυψα αὐτὸ ἐν τῷ Εὐφράτῃ, καθὼς ἐνετείλατό μοι Κύριος.
\VS{6}Καὶ ἐγένετο μεθʼ ἡμέρας πολλὰς, καὶ εἶπε Κύριος πρὸς μὲ, ἀνάστηθι, βάδισον ἐπὶ τὸν Εὐφράτην, καὶ λάβε ἐκεῖθεν τὸ περίζωμα, ὃ ἐνετειλάμην σοι τοῦ κατακρύψαι ἐκεῖ.
\VS{7}Καὶ ἐπορεύθην ἐπὶ τὸν Εὐφράτην ποταμὸν, καὶ ὤρυξα, καὶ ἔλαβον τὸ περίζωμα ἐκ τοῦ τόπου οὗ κατώρυξα αὐτὸ ἐκεῖ, καὶ ἰδοὺ διεφθαρμένον ἦν, ὃ οὐ μὴ χρησθῇ εἰς οὐθέν.
\par }{\PP \VS{8}Καὶ ἐγενήθη λόγος Κυρίου πρὸς μὲ, λέγων,
\VS{9}Τάδε λέγει Κύριος, οὕτω φθερῶ τὴν ὕβριν Ἰούδα καὶ τὴν ὕβριν Ἱερουσαλὴμ,
\VS{10}τὴν πολλὴν ταύτην ὕβριν, τοὺς μὴ βουλομένους ὑπακούειν τῶν λόγων μου, καὶ πορευθέντας ὀπίσω θεῶν ἀλλοτρίων τοῦ δουλεύειν αὐτοῖς, καὶ τοῦ προσκυνεῖν αὐτοῖς· καὶ ἔσονται ὥσπερ τὸ περίζωμα τοῦτο, ὃ οὐ χρησθήσεται εἰς οὐθέν.
\VS{11}Ὅτι καθάπερ κολλᾶται τὸ περίζωμα περὶ τὴν ὀσφὺν τοῦ ἀνθρώπου, οὕτως ἐκόλλησα πρὸς ἐμαυτὸν τὸν οἶκον τοῦ Ἰσραὴλ, καὶ πάντα οἶκον Ἰούδα, τοῦ γενέσθαι μοι εἰς λαὸν ὀνομαστὸν, καὶ εἰς καύχημα, καὶ εἰς δόξαν, καὶ οὐκ εἰσήκουσάν μου.
\par }{\PP \VS{12}Καὶ ἐρεῖς πρὸς τὸν λαὸν τοῦτον, πᾶς ἀσκὸς πληρωθήσεται οἴνου· καὶ ἔσται ἐὰν εἴπωσι πρὸς σὲ, μὴ γνόντες οὐ γνωσόμεθα, ὅτι πᾶς ἀσκὸς πληρωθήσεται οἴνου;
\VS{13}Καὶ ἐρεῖς πρὸς αὐτοὺς, τάδε λέγει Κύριος, ἰδοὺ ἐγὼ πληρῶ τοὺς κατοικοῦντας τὴν γῆν ταύτην, καὶ τοὺς βασιλεῖς αὐτῶν τοὺς καθημένους υἱοὺς τοῦ Δαυὶδ ἐπὶ τοῦ θρόνου αὐτῶν, καὶ τοὺς ἱερεῖς καὶ τοὺς προφήτας, καὶ τὸν Ἰούδαν καὶ πάντας τοὺς κατοικοῦντας ἐν Ἱερουσαλὴμ, μεθύσματι·
\VS{14}Καὶ διασκορπιῶ αὐτοὺς ἄνδρα καὶ τὸν ἀδελφὸν αὐτοῦ, καὶ τοὺς πατέρας αὐτῶν, καὶ τοὺς υἱοὺς αὐτῶν ἐν τῷ αὐτῷ· οὐκ ἐπιποθήσω, λέγει Κύριος, καὶ οὐ φείσομαι, καὶ οὐκ οἰκτειρήσω ἀπὸ διαφθορᾶς αὐτῶν.
\par }{\PP \VS{15}Ἀκούσατε, καὶ ἐνωτίσασθε, καὶ μὴ ἐπαίρεσθε, ὅτι Κύριος ἐλάλησε.
\VS{16}Δότε τῷ Κυρίῳ Θεῷ ὑμῶν δόξαν πρὸ τοῦ συσκοτάσαι, καὶ πρὸ τοῦ προσκόψαι πόδας ὑμῶν ἐπʼ ὄρη σκοτεινά· καὶ ἀναμενεῖτε εἰς φῶς, καὶ ἐκεῖ σκιὰ θανάτου καὶ τεθήσονται εἰς σκότος.
\VS{17}Ἐὰν δὲ μὴ ἀκούσητε, κεκρυμμένως κλαύσεται ἡ ψυχὴ ὑμῶν ἀπὸ προσώπου ὕβρεως, καὶ κατάξουσιν οἱ ὀφθαλμοὶ ὑμῶν δάκρυα, ὅτι συνετρίβη τὸ ποίμνιον Κυρίου.
\par }{\PP \VS{18}Εἴπατε τῷ βασιλεῖ καὶ τοῖς δυναστεύουσι, ταπεινώθητε καὶ καθίσατε, ὅτι καθῃρέθη ἀπὸ κεφαλῆς ὑμῶν στέφανος δόξης ὑμῶν.
\VS{19}Πόλεις αἱ πρὸς Νότον, συνεκλείσθησαν, καὶ οὐκ ἦν ὁ ἀνοίγων· ἀποικίσθη Ἰούδας, συνετέλεσαν ἀποικίαν τελείαν.
\VS{20}Ἀνάλαβε ὀφθαλμούς σου Ἱερουσαλὴμ, καὶ ἴδε τοὺς ἐρχομένους ἀπὸ Βοῤῥᾶ· ποῦ ἐστι τὸ ποίμνιον ὃ ἐδόθη σοι, πρόβατα δόξης σου;
\VS{21}Τί ἐρεῖς ὅταν ἐπισκέπτωνταί σε; καὶ σὺ ἐδίδαξας αὐτοὺς ἐπὶ σὲ μαθήματα εἰς ἀρχήν· οὐκ ὠδῖνες καθέξουσί σε καθὼς γυναῖκα τίκτουσαν;
\VS{22}Καὶ ἐὰν εἴπῃς ἐν τῇ καρδίᾳ σου, διατί ἀπήντησέ μοι ταῦτα; διὰ τὸ πλῆθος τῆς ἀδικίας σου ἀνεκαλύφθη τὰ ὀπίσθιά σου, παραδειγματισθῆναι τὰς πτέρνας σου.
\par }{\PP \VS{23}Εἰ ἀλλάξεται Αἰθίοψ τὸ δέρμα αὐτοῦ, καὶ πάρδαλις τὰ ποικίλματα αὐτῆς, καὶ ὑμεῖς δυνήσεσθε εὐποιῆσαι μεμαθηκότες τὰ κακά.
\VS{24}Καὶ διέσπειρα αὐτοὺς, ὡς φρύγανα φερόμενα ὑπὸ ἀνέμου εἰς ἔρημον.
\VS{25}Οὕτως ὁ κλῆρός σου, καὶ μερὶς τοῦ ἀπειθεῖν ὑμᾶς ἐμοὶ, λέγει Κύριος· ὡς ἐπελάθου μου, καὶ ἤλπισας ἐπὶ ψεύδεσι,
\VS{26}κᾀγὼ ἀποκαλύψω τὰ ὀπίσω σου ἐπὶ τὸ πρόσωπόν σου, καὶ ὀφθήσεται ἡ ἀτιμία σου,
\VS{27}καὶ ἡ μοιχεία σου, καὶ χρεμετισμός σου, καὶ ἡ ἀπαλλοτρίωσις τῆς πορνείας σου· ἐπὶ τῶν βουνῶν, καὶ ἐν τοῖς ἀγροῖς ἑώρακα τὰ βδελύγματά σου· οὐαί σοι Ἱερουσαλὴμ, ὅτι οὐκ ἐκαθαρίσθης ὀπίσω μου· ἕως τίνος ἔτι;

\par }\Chap{14}{\PP \VerseOne{1}ΚΑΙ ἘΓΕΝΕΤΟ ΛΟΓΟΣ ΚΥΡΙΟΥ ΠΡΟΣ ἹΕΡΕΜΙΑΝ ΠΕΡΙ ΤΗΣ ἈΒΡΟΧΙΑΣ.
\par }{\PP \VS{2}Ἐπένθησεν ἡ Ἰουδαία, καὶ αἱ πύλαι αὐτῆς ἐκενώθησαν, καὶ ἐσκοτώθησαν ἐπὶ τῆς γῆς, καὶ ἡ κραυγὴ τῆς Ἱερουσαλὴμ ἀνέβη·
\VS{3}Καὶ οἱ μεγιστᾶνες αὐτῆς ἀπέστειλαν τοὺς νεωτέρους αὐτῶν ἐφʼ ὕδωρ· ἤλθοσαν ἐπὶ τὰ φρέατα, καὶ οὐχ εὕροσαν ὕδωρ, καὶ ἀπέστρεψαν τὰ ἀγγεῖα αὐτῶν κενά.
\VS{4}Καὶ τὰ ἔργα τῆς γῆς ἐξέλιπεν, ὅτι οὐκ ἦν ὑετός· ᾐσχύνθησαν οἱ γεωργοὶ, ἐπεκάλυψαν τὰς κεφαλὰς αὐτῶν.
\VS{5}Καὶ ἔλαφοι ἐν ἀγρῷ ἔτεκον, καὶ ἐγκατέλιπον, ὅτι οὐκ ἦν βοτάνη.
\VS{6}Ὄνοι ἄγριοι ἔστησαν ἐπὶ νάπας, καὶ εἵλκυσαν ἄνεμον, ἐξέλιπον οἱ ὀφθαλμοὶ αὐτῶν, ὅτι οὐκ ἦν χόρτος.
\par }{\PP \VS{7}Αἱ ἁμαρτίαι ἡμῶν ἀντέστησαν ἡμῖν· Κύριε ποίησον ἡμῖν ἕνεκέν σου, ὅτι πολλαὶ αἱ ἁμαρτίαι ἡμῶν ἐναντίον σου, ὅτι σοι ἡμάρτομεν.
\VS{8}Ὑπομονὴ Ἰσραὴλ Κύριε, καὶ σώζεις ἐν καιρῷ κακῶν· ἱνατί ἑγενήθης ὡσεὶ πάροικος ἐπὶ τῆς γῆς, καὶ ὡς αὐτόχθων ἐκκλίνων εἰς κατάλυμα;
\VS{9}Μὴ ἔσῃ ὥσπερ ἄνθρωπος ὑπνῶν, ἢ ὡς ἀνὴρ οὐ δυνάμενος σώζειν; καὶ σὺ ἐν ἡμῖν εἶ Κύριε, καὶ τὸ ὄνομά σου ἐπικέκληται ἐφʼ ἡμᾶς· μὴ ἐπιλάθῃ ἡμῶν.
\par }{\PP \VS{10}Οὕτως λέγει Κύριος τῷ λαῷ τούτῳ, ἠγάπησαν κινεῖν πόδας αὐτῶν, καὶ οὐκ ἐφείσαντο, καὶ ὁ Θεὸς οὐκ εὐώδωσεν ἐν αὐτοῖς· νῦν μνησθήσεται τῆς ἀδικίας αὐτῶν.
\VS{11}Καὶ εἶπε Κύριος πρὸς μὲ, μὴ προσεύχου περὶ τοῦ λαοῦ τούτου εἰς ἀγαθὰ,
\VS{12}ὅτι ἐὰν νηστεύσωσιν, οὐκ εἰσακούσομαι τῆς δεήσεως αὐτῶν· καὶ ἐὰν προσενέγκωσιν ὁλοκαυτώματα καὶ θυσίας, οὐκ εὐδοκήσω ἐν αὐτοῖς, ὅτι ἐν μαχαίρᾳ καὶ ἐν λιμῷ καὶ ἐν θανάτῳ ἐγὼ συντελέσω αὐτούς.
\par }{\PP \VS{13}Καὶ εἶπα, ὁ ὢν Κύριε, ἰδοὺ οἱ προφῆται αὐτῶν προφητεύουσι, καὶ λέγουσιν, οὐκ ὄψεσθε μάχαιραν, οὐδὲ λιμὸς ἔσται ἐν ὑμῖν, ὅτι ἀλήθειαν καὶ εἰρήνην δώσω ἐπὶ τῆς γῆς, καὶ ἐν τῷ τόπῳ τούτῳ.
\par }{\PP \VS{14}Καὶ εἶπε Κύριος πρὸς μὲ, ψευδῆ οἱ προφῆται προφητεύουσιν ἐπὶ τῷ ὀνόματί μου, οὐκ ἀπέστειλα αὐτοὺς, καὶ οὐκ ἐνετειλάμην αὐτοῖς, καὶ οὐκ ἐλάλησα πρὸς αὐτούς· ὅτι ὁράσεις ψευδεῖς, καὶ μαντείας, καὶ οἰωνίσματα, καὶ προαιρέσεις καρδίας αὐτῶν αὐτοὶ προφητεύουσιν ὑμῖν.
\VS{15}Διατοῦτο τάδε λέγει Κύριος περὶ τῶν προφητῶν τῶν προφητευόντων ἐπὶ τῷ ὀνόματί μου ψευδῆ, καὶ ἐγὼ οὐκ ἀπέστειλα αὐτοὺς, οἳ λέγουσι, μάχαιρα καὶ λιμὸς οὐκ ἔσται ἐπὶ τῆς γῆς ταύτης· ἐν θανάτῳ νοσερῷ ἀποθανοῦνται, καὶ ἐν λιμῷ συντελεσθήσονται οἱ προφῆται,
\VS{16}καὶ ὁ λαὸς, οἷς αὐτοὶ προφητεύουσιν αὐτοῖς, καὶ ἔσονται ἐῤῥιμμένοι ἐν ταῖς ὁδοῖς Ἱερουσαλὴμ, ἀπὸ προσώπου μαχαίρας καὶ τοῦ λιμοῦ, καὶ οὐκ ἔσται ὁ θάπτων αὐτοὺς, καὶ αἱ γυναῖκες αὐτῶν, καὶ οἱ υἱοὶ αὐτῶν, καὶ αἱ θυγατέρες αὐτῶν, καὶ ἐκχεῶ ἐπʼ αὐτοὺς τὰ κακὰ αὐτῶν.
\par }{\PP \VS{17}Καὶ ἐρεῖς πρὸς αὐτοὺς τὸν λόγον τοῦτον, καταγάγετε ἐπ ὀφθαλμοὺς ὑμῶν δάκρυα ἡμερας καὶ νυκτὸς, καὶ μὴ διαλιπέτωσαν, ὅτι συντρίμματι συνετρίβη θυγάτηρ λαοῦ μου, καὶ πληγὴ ὀδυνηρὰ σφόδρα.
\VS{18}Ἐὰν ἐξέλθω εἰς τὸ πεδίον, καὶ ἰδοὺ τραυματίαι μαχαίρας· καὶ ἐὰν εἰσέλθω εἰς τὴν πόλιν, καὶ ἰδοὺ πόνος λιμοῦ· ὅτι ἱερεὺς καὶ προφήτης ἐπορεύθησαν εἰς γῆν, ἣν οὐκ ᾔδεισαν.
\par }{\PP \VS{19}Μὴ ἀποδοκιμάζων ἀπεδοκίμασας τὸν Ἰούδαν, καὶ ἀπὸ Σιὼν ἀπέστη ἡ ψυχή σου; ἱνατί ἔπαισας ἡμᾶς, καὶ οὐκ ἔστιν ἡμῖν ἴασις; ὑπεμείναμεν εἰς εἰρήνην, καὶ οὐκ ἦν ἀγαθὰ, εἰς καιρὸν ἰάσεως, καὶ ἰδοὺ ταραχή.
\VS{20}Ἔγνωμεν Κύριε ἁμαρτήματα ἡμῶν, ἀδικίας πατέρων ἡμῶν, ὅτι ἡμάρτομεν ἐναντίον σου.
\VS{21}Κόπασον διὰ τὸ ὄνομά σου, μὴ ἀπολέσῃς θρόνον δόξης σου· μνήσθητι, μὴ διασκεδάσῃς τὴν διαθήκην σου τὴν μεθʼ ἡμῶν.
\VS{22}Μὴ ἔστιν ἐν εἰδώλοις τῶν ἐθνῶν ὑετίζων; καὶ εἰ ὁ οὐρανὸς δώσει πλησμονὴν αὐτοῦ, οὐχὶ σὺ εἶ αὐτός; καὶ ὑπομενοῦμέν σε Κύριε, ὅτι σὺ ἐποίησας πάντα ταῦτα.

\par }\Chap{15}{\PP \VerseOne{1}Καὶ εἶπε Κύριος πρὸς μὲ, ἐὰν στῇ Μωυσῆς καὶ Σαμουὴλ πρὸ προσώπου μου, οὐκ ἔστιν ἡ ψυχή μου πρὸς αὐτούς· ἐξαπόστειλον τὸν λαὸν τοῦτον, καὶ ἐξελθέτωσαν.
\VS{2}Καὶ ἔσται ἐὰν εἴπωσι πρὸς σὲ, ποῦ ἐξελευσόμεθα; καὶ ἐρεῖς πρὸς αὐτοὺς, τάδε λέγει Κύριος, ὅσοι εἰς θάνατον, εἰς θάνατον· καὶ ὅσοι εἰς μάχαιραν, εἰς μάχαιραν· καὶ ὅσοι εἰς λιμὸν, εἰς λιμόν· καὶ ὅσοι εἰς αἰχμαλωσίαν, εἰς αἰχμαλωσίαν.
\VS{3}Καὶ ἐκδικήσω ἐπʼ αὐτοὺς τέσσαρα εἴδη, λέγει Κύριος· τὴν μάχαιραν εἰς σφαγὴν, καὶ τοὺς κύνας εἰς διασπασμὸν, καὶ τὰ θηρία τῆς γῆς, καὶ τὰ πετεινὰ τοῦ οὐρανοῦ εἰς βρῶσιν καὶ διαφθοράν.
\VS{4}Καὶ παραδώσω αὐτοὺς εἰς ἀνάγκας πάσαις ταῖς βασιλείαις τῆς γῆς, διὰ Μανασσῆ υἱὸν Ἐζεκίου βασιλέως Ἰούδα, περὶ πάντων ὧν ἐποίησεν ἐν Ἱερουσαλήμ.
\par }{\PP \VS{5}Τίς φείσεται ἐπὶ σοὶ Ἱερουσαλήμ; καὶ τίς δειλιάσει ἐπὶ σοί; ἢ τίς ἀνακάμψει εἰς εἰρήνην σοι;
\VS{6}Σὺ ἀπεστράφης με, λέγει Κύριος, ὀπίσω πορεύσῃ· καὶ ἐκτενῶ τὴν χεῖρά μου, καὶ διαφθερῶ σε, καὶ οὐκέτι ἀνήσω αὐτούς·
\VS{7}Καὶ διασπερῶ αὐτοὺς ἐν διασπορᾶ ἐν πύλαις λαοῦ μου ἠτεκνώθησαν, ἀπώλεσαν τὸν λαόν μου διὰ τὰς κακίας αὐτῶν.
\par }{\PP \VS{8}Ἐπληθύνθησαν αἱ χῆραι αὐτῶν ὑπὲρ τὴν ἄμμον τῆς θαλάσσης· ἐπήγαγον ἐπὶ μητέρα νεανίσκους, ταλαιπωρίαν ἐν μεσημβρίᾳ, ἐπέῤῥιψα ἐπʼ αὐτὴν ἐξαίφνης τρόμον καὶ σπουδήν.
\VS{9}Ἐκενώθη ἡ τίκτουσα ἑπτὰ, ἀπεκάκησεν ἡ ψυχὴ αὐτῆς, ἐπέδυ ὁ ἥλιος αὐτῇ ἔτι μεσούσης τῆς ἡμέρας, κατῃσχύνθη καὶ ὠνειδίσθη· τοὺς καταλοίπους αὐτῶν εἰς μάχαιραν δώσω ἐναντίον τῶν ἐχθρῶν αὐτῶν.
\par }{\PP \VS{10}Οἴμοι ἐγὼ, μῆτερ, ὡς τινά με ἔτεκες ἄνδρα δικαζόμενον, καὶ διακρινόμενον πάσῃ τῇ γῇ· οὔτε ὠφέλησα, οὔτε ὠφέλησέ με οὐδείς· ἡ ἰσχύς μου ἐξέλιπεν ἐν τοῖς καταρωμένοις με.
\VS{11}Γένοιτο δέσποτα κατευθυνόντων αὐτῶν· εἰ μὴ παρέστην σοι ἐν καιρῷ τῶν κακῶν αὐτῶν, καὶ ἐν καιρῷ θλίψεως αὐτῶν, εἰς ἀγαθὰ πρὸς τὸν ἐχθρόν.
\VS{12}Εἰ γνωσθήσεται σίδηρος; καὶ περιβόλαιον χαλκοῦν ἡ ἰσχύς σου.
\VS{13}Καὶ τοὺς θησαυρούς σου εἰς προνομὴν δώσω ἀντάλλαγμα, διὰ πάσας τὰς ἁμαρτίας σου, καὶ ἐν πᾶσι τοῖς ὁρίοις σου.
\VS{14}Καὶ καταδουλώσω σε κύκλῳ τοῖς ἐχθροῖς σου, ἐν τῇ γῇ ᾗ οὐκ ᾔδεις· ὅτι πῦρ ἐκκέκαυται ἐκ τοῦ θυμοῦ μου, ἐφʼ ὑμᾶς καυθήσεται.
\par }{\PP \VS{15}Κύριε μνήσθητί μου, καὶ ἐπίσκεψαί με, καὶ ἀθώωσόν με ἀπὸ τῶν καταδιωκόντων με, μὴ εἰς μακροθυμίαν· γνῶθι, ὡς ἔλαβον περὶ σοῦ ὀνειδισμὸν
\VS{16}ὑπὸ τῶν ἀθετούντων τοὺς λόγους σου· συντέλεσον αὐτούς· καὶ ἔσται ὁ λόγος σου ἐμοὶ εἰς εὐφροσύνην καὶ χαρὰν καρδίας μου, ὅτι ἐπικέκληται τὸ ὄνομά σου ἐπʼ ἐμοὶ, Κύριε παντοκράτωρ.
\VS{17}Οὐκ ἐκάθισα ἐν συνεδρίῳ αὐτῶν παιζόντων, ἀλλὰ εὐλαβούμην ἀπὸ προσώπου χειρός σου· καταμόνας ἐκαθήμην, ὅτι πικρίας ἐνεπλήσθην.
\par }{\PP \VS{18}Ἱνατί οἱ λυποῦντές με κατισχύουσί μου; ἡ πληγή μου στερεὰ, πόθεν ἰαθήσομαι; γινομένη ἐγενήθη μοι ὡς ὕδωρ ψευδὲς, οὐκ ἔχον πίστιν.
\par }{\PP \VS{19}Διατοῦτο τάδε λέγει Κύριος, ἐὰν ἐπιστρέψῃς, καὶ ἀποκαταστήσω σε, καὶ πρὸ προσώπου μου στήσῃ· καὶ ἐὰν ἐξαγάγῃς τίμιον ἀπὸ ἀναξίου, ὡς τὸ στόμα μου ἔσῃ· καὶ ἀναστρέψουσιν αὐτοὶ πρὸς σὲ, καὶ σὺ οὐκ ἀναστρέψεις πρὸς αὐτούς.
\VS{20}Καὶ δώσω σε τῷ λαῷ τούτῳ, ὡς τεῖχος ὀχυρὸν, χαλκοῦν· καὶ πολεμήσουσι πρὸς σὲ, καὶ οὐ μὴ δύνωνται πρὸς σὲ, διότι μετὰ σοῦ εἰμι τοῦ σώζειν σε,
\VS{21}καὶ τοῦ ἐξαιρεῖσθαί σε ἐκ χειρὸς πονηρῶν, καὶ λυτρώσομαί σε ἐκ χειρὸς λοιμῶν.

\par }\Chap{16}{\PP \VerseOne{1}Καὶ σὺ μὴ λάβῃς γυναῖκα, λέγει Κύριος ὁ Θεὸς Ἰσραὴλ,
\VS{2}καὶ οὐ γεννηθήσεταί σοι υἱὸς, οὐδὲ θυγάτηρ ἐν τῷ τόπῳ τούτῳ.
\VS{3}Ὅτι τάδε λέγει Κύριος περὶ τῶν υἱῶν καὶ περὶ τῶν θυγατέρων τῶν γεννωμένων ἐν τῷ τόπῳ τούτῳ, καὶ περὶ τῶν μητέρων αὐτῶν τῶν τετοκυιῶν αὐτοὺς, καὶ περὶ τῶν πατέρων αὐτῶν τῶν γεγεννηκότων αὐτοὺς ἐν τῇ γῇ ταύτῃ,
\VS{4}ἐν θανάτῳ νοσερῷ ἀποθανοῦνται, οὐ κοπήσονται, καὶ οὐ ταφήσονται· εἰς παράδειγμα ἐπὶ προσώπου τῆς γῆς ἔσονται· καὶ τοῖς θηρίοις τῆς γῆς ἔσονται καὶ τοῖς πετεινοῖς τοῦ οὐρανοῦ· ἐν μαχαίρᾳ πεσοῦνται, καὶ ἐν λιμῷ συντελεσθήσονται.
\par }{\PP \VS{5}Τάδε λέγει Κύριος, μὴ εἰσέλθῃς εἰς θίασον αὐτῶν, καὶ μὴ πορευθῇς τοῦ κόψασθαι, καὶ μὴ πενθήσῃς αὐτοὺς, ὅτι ἀφέστακα τὴν εἰρήνην μου ἀπὸ τοῦ λαοῦ τούτου·
\VS{6}Οὐ μὴ κόψονται αὐτοὺς, οὐδὲ ἐντομίδας οὐ μὴ ποιήσουσι, καὶ οὐ ξυρηθήσονται,
\VS{7}καὶ οὐ μὴ κλασθῇ ἄρτος ἐν πένθει αὐτῶν εἰς παράκλησιν ἐπὶ τεθνηκότι· οὐ ποτιοῦσιν αὐτὸν ποτήριον εἰς παράκλησιν ἐπὶ πατρὶ καὶ μητρὶ αὐτοῦ.
\par }{\PP \VS{8}Εἰς οἰκίαν πότου οὐκ εἰσελεύσῃ σὺ, συγκαθίσαι μετʼ αὐτῶν τοῦ φαγεῖν καὶ πιεῖν.
\VS{9}Διότι τάδε λέγει Κύριος ὁ Θεὸς Ἰσραὴλ, ἰδοὺ ἐγὼ καταλύω ἐκ τοῦ τόπου τούτου ἐνώπιον τῶν ὀφθαλμῶν ὑμῶν, καὶ ἐν ταῖς ἡμέραις ὑμῶν, φωνὴν χαρᾶς, καὶ φωνὴν εὐφροσύνης, φωνὴν νυμφίου, καὶ φωνὴν νύμφης.
\par }{\PP \VS{10}Καὶ ἔσται ὅταν ἀναγγείλῃς τῷ λαῷ τούτῳ ἅπαντα τὰ ῥήματα ταῦτα, καὶ εἴπωσι πρὸς σὲ, διατί ἐλάλησε Κύριος ἐφʼ ἡμᾶς πάντα τὰ κακὰ ταῦτα; τίς ἡ ἀδικία ἡμῶν; καὶ τίς ἡ ἁμαρτία ἡμῶν, ἣν ἡμάρτομεν ἔναντι Κυρίου τοῦ Θεοῦ ἡμῶν;
\VS{11}Καὶ ἐρεῖς πρὸς αὐτοὺς, ἀνθʼ ὧν ἐγκατέλιπόν με οἱ πατέρες ὑμῶν, λέγει Κύριος, καὶ ᾤχοντο ὀπισω θεῶν ἀλλοτρίων, καὶ ἐδούλευσαν αὐτοῖς, καὶ προσεκύνησαν αὐτοῖς, καὶ ἐμὲ ἐγκατέλιπον, καὶ τὸν νόμον μου οὐκ ἐφυλάξαντο·
\VS{12}Καὶ ὑμεῖς ἐπονηρεύσασθε ὑπὲρ τοὺς πατέρας ὑμῶν· καὶ ἰδοὺ ὑμεῖς πορεύεσθε ἕκαστος ὀπίσω τῶν ἀρεστῶν τῆς καρδίας ὑμῶν τῆς πονηρᾶς, τοῦ μὴ ὑπακούειν μου.
\VS{13}Καὶ ἀποῤῥίψω ὑμᾶς ἀπὸ τῆς γῆς ταύτης εἰς τὴν γῆν, ἣν οὐκ ᾔδειτε ὑμεῖς καὶ οἱ πατέρες ὑμῶν, καὶ δουλεύσετε ἐκεῖ θεοῖς ἑτέροις, οἳ οὐ δώσουσιν ὑμῖν ἔλεος.
\par }{\PP \VS{14}Διατοῦτο ἰδοὺ ἡμέραι ἔρχονται, λέγει Κύριος, καὶ οὐκ ἐροῦσιν ἔτι, ζῇ Κύριος ὁ ἀναγαγὼν τοὺς υἱοὺς Ἰσραὴλ ἐκ γῆς Αἰγύπτου,
\VS{15}ἀλλὰ, ζῇ Κύριος, ὃς ἀνήγαγε τὸν οἶκον Ἰσραὴλ ἀπὸ γῆς Βοῤῥᾶ, καὶ ἀπὸ πασῶν τῶν χωρῶν οὗ ἐξώσθησαν ἐκεῖ· καὶ ἀποκαταστήσω αὐτοὺς εἰς τὴν γῆν αὐτῶν, ἣν ἔδωκα τοῖς πατράσιν αὐτῶν.
\par }{\PP \VS{16}Ἰδοὺ ἐγὼ ἀποστέλλω τοὺς ἁλιεῖς τοὺς πολλοὺς, λέγει Κύριος, καὶ ἁλιευσουσιν αὐτούς· καὶ μετὰ ταῦτα ἀποστέλλω τοὺς πολλοὺς θηρευτὰς, καὶ θηρεύσουσιν αὐτοὺς ἐπάνω παντὸς ὄρους, καὶ ἐπάνω παντὸς βουνοῦ, καὶ ἐκ τῶν τρυμαλιῶν τῶν πετρῶν.
\VS{17}Ὅτι οἱ ὀφθαλμοί μου ἐπὶ πάσας τὰς ὁδοὺς αὐτῶν, καὶ οὐκ ἐκρύβη τὰ ἀδικήματα αὐτῶν ἀπέναντι τῶν ὀφθαλμῶν μου.
\VS{18}Καὶ ἀνταποδώσω διπλὰς τὰς κακίας αὐτῶν, καὶ τὰς ἁμαρτίας αὐτῶν, ἐφʼ αἷς ἐβεβήλωσαν τὴν γῆν μου ἐν τοῖς θνησιμαίοις τῶν βδελυγμάτων αὐτῶν, καὶ ἐν ταῖς ἀνομίαις αὐτῶν, ἐν αἷς ἐπλημμέλησαν τὴν κληρονομίαν μου.
\par }{\PP \VS{19}Κύριε σὺ ἰσχύς μου, καὶ βοήθειά μου, καὶ καταφυγή μου ἐν ἡμέραις κακῶν· πρὸς σὲ ἔθνη ἥξουσιν ἀπʼ ἐσχάτου τῆς γῆς, καὶ ἐροῦσιν, ὡς ψευδῆ ἐκτήσαντο οἱ πατέρες ἡμῶν εἴδωλα, καὶ οὐκ ἔστιν ἐν αὐτοῖς ὠφέλημα;
\VS{20}Εἰ ποιήσει ἑαυτῷ ἄνθρωπος θεοὺς, καὶ οὗτοι οὐκ εἰσὶ θεοί.
\VS{21}Διατοῦτο ἰδοὺ ἐγὼ δηλώσω αὐτοῖς ἐν τῷ καιρῷ τούτῳ τὴν χεῖρά μου, καὶ γνωριῶ αὐτοῖς τὴν δύναμίν μου, καὶ γνώσονται, ὅτι ὄνομά μοι Κύριος.

\par }\Chap{17}{\PP \VS{5}Ἐπικατάρατος ὁ ἄνθρωπος, ὃς τὴν ἐλπίδα ἔχει ἐπʼ ἄνθρωπον, καὶ στηρίσει σάρκα βραχίονος αὐτοῦ ἐπʼ αὐτὸν, καὶ ἀπὸ Κυρίου ἀποστῇ ἡ καρδία αὐτοῦ.
\VS{6}Καὶ ἔσται ὡς ἡ ἀγριομυρίκη ἡ ἐν τῇ ἐρήμῳ, οὐκ ὄψεται ὅταν ἔλθῃ τὰ ἀγαθὰ, καὶ κατασκηνώσει ἐν ἁλίμοις, καὶ ἐν ἐρήμῳ, ἐν γῇ ἁλμυρᾷ ἥτις οὐ κατοικεῖται.
\VS{7}Καὶ εὐλογημένος ὁ ἄνθρωπος, ὃς πέποιθεν ἐπὶ τῷ Κυρίῳ, καὶ ἔσται Κύριος ἐλπὶς αὐτοῦ·
\VS{8}Καὶ ἔσται ὡς ξύλον εὐθηνοῦν παρʼ ὕδατα, καὶ ἐπὶ ἰκμάδα βαλεῖ ῥίζαν αὐτοῦ· οὐ φοβηθήσεται ὅταν ἔλθῃ καῦμα, καὶ ἔσται ἐπʼ αὐτῷ στελέχη ἀλσώδη, ἐν ἐνιαυτῷ ἀβροχίας οὐ φοβηθήσεται, καὶ οὐ διαλείψει ποιῶν καρπόν.
\par }{\PP \VS{9}Βαθεία ἡ καρδία παρὰ πάντα, καὶ ἄνθρωπός ἐστι, καὶ τίς γνώσεται αὐτόν;
\VS{10}Ἐγὼ Κύριος ἐτάζων καρδίας, καὶ δοκιμάζων νεφροὺς, τοῦ δοῦναι ἑκάστῳ κατὰ τὰς ὁδοὺς αὐτοῦ, καὶ κατὰ τοὺς καρποὺς τῶν ἐπιτηδευμάτων αὐτοῦ.
\par }{\PP \VS{11}Ἐφώνησε πέρδιξ, συνήγαγεν ἃ οὐκ ἔτεκε, ποιῶν πλοῦτον αὐτοῦ οὐ μετὰ κρίσεως· ἐν ἡμίσει ἡμερῶν αὐτοῦ ἐγκαταλείψουσιν αὐτὸν, καὶ ἐπʼ ἐσχάτων αὐτοῦ ἔσται ἄφρων.
\par }{\PP \VS{12}Θρόνος δόξης ὑψωμένος, ἁγίασμα ἡμῶν,
\VS{13}ὑπομονὴ Ἰσραήλ· Κύριε, πάντες οἱ καταλιπόντες σε καταισχυνθήτωσαν, ἀφεστηκότες ἐπὶ τῆς γῆς γραφήτωσαν, ὅτι ἐγκατέλιπον πηγὴν ζωῆς, τὸν Κύριον.
\par }{\PP \VS{14}Ἴασαί με Κύριε, καὶ ἰαθήσομαι· σῶσόν με, καὶ σωθήσομαι, ὅτι καύχημά μου σὺ εἶ.
\par }{\PP \VS{15}Ἰδοὺ αὐτοὶ λέγουσι πρὸς μὲ, ποῦ ἐστιν ὁ λόγος Κυρίου; ἐλθέτω.
\VS{16}Ἐγὼ δὲ οὐκ ἐκοπίασα κατακολουθῶν ὀπίσω σου, καὶ ἡμέραν ἀνθρώπου οὐκ ἐπεθύμησα· σὺ ἐπίστῃ· τὰ ἐκπορευόμενα διὰ τῶν χειλέων μου, πρὸ προσώπου σου ἐστί.
\VS{17}Μὴ γενηθῇς μοι εἰς ἀλλοτρίωσιν, φειδόμενός μου ἐν ἡμέρᾳ πονηρᾷ.
\VS{18}Καταισχυνθήτωσαν οἱ διώκοντές με, καὶ μὴ καταισχυνθείην ἐγώ· πτοηθείησαν αὐτοὶ, καὶ μὴ πτοηθείην ἐγώ· ἐπάγαγε ἐπʼ αὐτοὺς ἡμέραν πονηρὰν, δισσὸν σύντριμμα σύντριψον αὐτούς.
\par }{\PP \VS{19}Τάδε λέγει Κύριος, βάδισον, καὶ στῆθι ἐν ταῖς πύλαις υἱῶν λαοῦ σου, ἐν αἷς εἰσπορεύονται ἐν αὐταῖς βασιλεῖς Ἰούδα, καὶ ἐν αἷς ἐκπορεύονται ἐν αὐταῖς, καὶ ἐν πάσαις ταῖς πύλαις Ἱερουσαλὴμ,
\VS{20}καὶ ἐρεῖς αὐτοῖς, ἀκούσατε τὸν λόγον Κυρίου βασιλεῖς Ἰούδα, καὶ πᾶσα Ἰουδαία, καὶ πᾶσα Ἱερουσαλὴμ, οἱ εἰσπορευόμενοι ἐν ταῖς πύλαις ταύταις·
\VS{21}Τάδε λέγει Κύριος, φυλάσσεσθε τὰς ψυχὰς ὑμῶν, καὶ μὴ αἴρετε βαστάγματα ἐν τῇ ἡμέρᾳ τῶν σαββάτων, καὶ μὴ ἐκπορεύεσθε ταῖς πύλαις Ἱερουσαλὴμ,
\VS{22}καὶ μὴ ἐκφέρετε βαστάγματα ἐξ οἰκιῶν ὑμῶν ἐν τῇ ἡμέρᾳ τῶν σαββάτων, καὶ πᾶν ἔργον οὐ ποιήσετε· ἁγιάσατε τὴν ἡμέραν τῶν σαββάτων, καθὼς ἐνετειλάμην τοῖς πατράσιν ὑμῶν. Καὶ οὐκ ἤκουσαν, καὶ οὐκ ἔκλιναν τὸ οὖς αὐτῶν,
\VS{23}καὶ ἐσκλήρυναν τὸν τράχηλον αὐτῶν ὑπὲρ τοὺς πατέρας αὐτῶν, τοῦ μὴ ἀκοῦσαί μου, καὶ τοῦ μὴ δέξασθαι παιδείαν.
\par }{\PP \VS{24}Καὶ ἓσται, ἐὰν εἰσακούσητέ μου, λέγει Κύριος, τοῦ μὴ εἰσφέρειν βαστάγματα διὰ τῶν πυλῶν τῆς πόλεως ταύτης ἐν τῇ ἡμέρᾳ τῶν σαββάτων καὶ ἁγιάζειν τὴν ἡμέραν τῶν σαββάτων, τοῦ μὴ ποιεῖν πᾶν ἔργον,
\VS{25}καὶ εἰσελεύσονται διὰ τῶν πυλῶν τῆς πόλεως ταύτης βασιλεῖς καὶ ἄρχοντες καθήμενοι ἐπὶ θρόνου Δαυὶδ, καὶ ἐπιβεβηκότες ἐφʼ ἅρμασι καὶ ἵπποις αὐτῶν, αὐτοὶ καὶ οἱ ἄρχοντες αὐτῶν, ἄνδρες Ἰούδα καὶ οἱ κατοικοῦντες ἐν Ἱερουσαλήμ· καὶ κατοικισθήσεται ἡ πόλις αὕτη εἰς τὸν αἰῶνα.
\VS{26}Καὶ ἥξουσιν ἐκ τῶν πόλεων Ἰούδα, καὶ κυκλόθεν Ἱερουσαλὴμ, καὶ ἐκ γῆς Βενιαμὶν, καὶ ἐκ γῆς πεδινῆς, καὶ ἐκ τοῦ ὄρους, καὶ ἐκ τῆς πρὸς Νότον, φέροντες ὁλοκαυτώματα καὶ θυσίας καὶ θυμιάματα καὶ μάννα καὶ λίβανον, φέροντες αἴνεσιν εἰς οἶκον Κυρίου.
\par }{\PP \VS{27}Καὶ ἔσται ἐὰν μὴ ἀκούσητέ μου τοῦ ἁγιάζειν τὴν ἡμέραν τῶν σαββάτων, τοῦ μὴ αἴρειν βαστάγματα, καὶ μὴ εἰσπορεύεσθαι ταῖς πύλαις Ἱερουσαλὴμ ἐν τῇ ἡμέρᾳ τῶν σαββάτων, καὶ ἀνάψω πῦρ ἐν ταῖς πύλαις αὐτῆς, καὶ καταφάγεται ἄμφοδα Ἱερουσαλὴμ, καὶ οὐ σβεσθήσεται.

\par }\Chap{18}{\PP \VerseOne{1}Ὁ λόγος ὁ γενόμενος παρὰ Κυρίου πρὸς Ἱερεμίαν, λέγων,
\VS{2}ἀνάστηθι, καὶ κατάβηθι εἰς οἶκον τοῦ κεραμέως, καὶ ἐκεῖ ἀκούσῃ τοὺς λόγους μου.
\VS{3}Καὶ κατέβην εἰς τὸν οἶκον τοῦ κεραμέως, καὶ ἰδοὺ αὐτὸς ἐποίει ἔργον ἐπὶ τῶν λίθων.
\VS{4}Καὶ ἔπεσε τὸ ἀγγεῖον, ὃ αὐτὸς ἐποίει ἐν ταῖς χερσὶν αὐτοῦ· καὶ πάλιν αὐτὸς ἐποίησεν αὐτὸ ἀγγεῖον ἕτερον, καθὼς ἤρεσεν ἐνώπιον αὐτοῦ ποιῆσαι.
\VS{5}Καὶ ἐγένετο λόγος Κυρίου πρὸς μὲ, λέγων,
\par }{\PP \VS{6}Εἰ καθὼς ὁ κεραμεὺς οὗτος οὐ δυνήσομαι τοῦ ποιῆσαι ὑμᾶς οἶκος Ἰσραήλ; ἰδοὺ, ὡς ὁ πηλὸς τοῦ κεραμέως, ὑμεῖς ἐστε ἐν χερσί μου.
\VS{7}Πέρας λαλήσω ἐπὶ ἔθνος, ἢ ἐπὶ βασιλείαν τοῦ ἐξᾶραι αὐτοὺς, καὶ τοῦ ἀπολλύειν,
\VS{8}καὶ ἐπιστραφῇ τὸ ἔθνος ἐκεῖνο ἀπὸ πάντων τῶν κακῶν αὐτῶν, καὶ μετανοήσω περὶ τῶν κακῶν, ὧν ἐλογισάμην, τοῦ ποιῆσαι αὐτοῖς.
\VS{9}Καὶ πέρας λαλήσω ἐπὶ ἔθνος καὶ βασιλείαν, τοῦ ἀνοικοδομεῖσθαι καὶ τοῦ καταφυτεύεσθαι,
\VS{10}καὶ ποιήσωσι τὰ πονηρὰ ἐναντίον μου, τοῦ μὴ ἀκούειν τῆς φωνῆς μου, καὶ μετανοήσω περὶ τῶν ἀγαθῶν ὧν ἐλάλησα, τοῦ ποιῆσαι αὐτοῖς.
\par }{\PP \VS{11}Καὶ νῦν εἰπὸν πρὸς ἄνδρας Ἰούδα, καὶ πρὸς τοὺς κατοικοῦντας Ἱερουσαλὴμ, ἰδοὺ ἐγὼ πλάσσω ἐφʼ ὑμᾶς κακὰ, καὶ λογίζομαι ἐφʼ ὑμᾶς λογισμόν· ἀποστραφήτω δὴ ἕκαστος ἀπὸ ὁδοῦ αὐτοῦ τῆς πονηρᾶς, καὶ καλλίονα ποιήσατε τὰ ἐπιτηδεύματα ὑμῶν.
\VS{12}Καὶ εἶπαν, ἀνδριούμεθα, ὅτι ὀπίσω τῶν ἀποστροφῶν ἡμῶν πορευσόμεθα, καὶ ἕκαστος τὰ ἀρεστὰ τῆς καρδίας αὐτοῦ τῆς πονηρᾶς ποιήσομεν.
\par }{\PP \VS{13}Διατοῦτο τάδε λέγει Κύριος, ἐρωτήσατε δὴ ἐν ἔθνεσι, τίς ἤκουσε τοιαῦτα φρικτὰ ἃ ἐποίησε σφόδρα παρθένος Ἰσραήλ;
\VS{14}Μὴ ἐκλείψουσιν ἀπὸ πέτρας μαστοὶ, ἢ χιὼν ἀπὸ τοῦ Λιβάνου; μὴ ἐκκλίνῃ ὕδωρ βιαίως ἀνέμῳ φερόμενον;
\VS{15}Ὅτι ἐπελάθοντό μου λαός μου, εἰς κενὸν ἐθυμίασαν καὶ ἀσθενήσουσιν ἐν ταῖς ὁδοῖς αὐτῶν σχοίνους αἰωνίους, τοῦ ἐπιβῆναι τρίβους οὐκ ἔχοντας ὁδὸν εἰς πορείαν,
\VS{16}τοῦ τάξαι τὴν γῆν αὐτῶν εἰς ἀφανισμὸν, καὶ σύριγμα αἰώνιον· πάντες οἱ διαπορευόμενοι διʼ αὐτῆς ἐκστήσονται, καὶ κινήσουσι τὴν κεφαλὴν αὐτῶν.
\VS{17}Ὡς ἄνεμον καύσωνα διασπερῶ αὐτοὺς κατὰ πρόσωπον ἐχθρῶν αὐτῶν, δείξω αὐτοῖς ἡμέραν ἀπωλείας αὐτῶν.
\par }{\PP \VS{18}Καὶ εἶπεν, δεῦτε καὶ λογισώμεθα ἐπὶ Ἱερεμίαν λογισμὸν, ὅτι οὐκ ἀπολεῖται νόμος ἀπὸ ἱερέως, καὶ βουλὴ ἀπὸ συνετοῦ, καὶ λόγος ἀπὸ προφήτου· δεῦτε καὶ πατάξωμεν αὐτὸν ἐν γλώσσῃ, καὶ ἀκουσόμεθα πάντας τοὺς λόγους αὐτοῦ.
\par }{\PP \VS{19}Εἰσάκουσόν μου Κύριε, καὶ εἰσάκουσον τῆς φωνῆς τοῦ δικαιώματός μου.
\VS{20}Εἰ ἀνταποδίδοται ἀντὶ ἀγαθῶν κακὰ, ὅτι συνελάλησαν ῥήματα κατὰ τῆς ψυχῆς μου, καὶ τὴν κόλασιν αὐτῶν ἔκρυψάν μοι· μνήσθητι ἑστηκότος μου κατὰ πρόσωπόν σου, τοῦ λαλῆσαι ὑπὲρ αὐτῶν ἀγαθὰ, τοῦ ἀποστρέψαι τὸν θυμόν σου ἀπʼ αὐτῶν.
\VS{21}Διατοῦτο δὸς τοὺς υἱοὺς αὐτῶν εἰς λιμὸν, καὶ ἄθροισον αὐτοὺς εἰς χεῖρας μαχαίρας· γενέσθωσαν αἱ γυναῖκες αὐτῶν ἄτεκνοι καὶ χῆραι, καὶ οἱ ἄνδρες αὐτῶν γενέσθωσαν ἀνῃρημένοι θανάτῳ, καὶ οἱ νεανίσκοι αὐτῶν πεπτωκότες μαχαίρᾳ ἐν πολέμῳ.
\VS{22}Γενηθήτω κραυγὴ ἐν ταῖς οἰκίαις αὐτῶν· ἐπάξεις ἐπʼ αὐτοὺς λῃστὰς ἄφνω, ὅτι ἐνεχείρησαν λόγον εἰς σύλληψίν μου, καὶ παγίδας ἔκρυψαν ἐπʼ ἐμέ.
\par }{\PP \VS{23}Καὶ σὺ, Κύριε, ἔγνως ἅπασαν τὴν βουλὴν αὐτῶν ἐπʼ ἐμὲ εἰς θάνατον· μὴ ἀθωώσῃς τὰς ἀδικίας αὐτῶν, καὶ τὰς ἁμαρτίας αὐτῶν ἀπὸ προσώπου σου μὴ ἐξαλείψῃς· γενέσθω ἡ ἀσθένεια αὐτῶν ἐναντίον σου, ἐν καιρῷ θυμοῦ σου ποίησον ἐν αὐτοῖς.

\par }\Chap{19}{\PP \VerseOne{1}Τότε εἶπε Κύριος πρὸς μὲ, βάδισον, καὶ κτῆσαι βικὸν πεπλασμένον ὀστράκινον, καὶ ἄξεις ἀπὸ τῶν πρεσβυτέρων τοῦ λαοῦ καὶ ἀπὸ τῶν ἱερέων,
\VS{2}καὶ ἐξελεύσῃ εἰς τὸ πολυάνδριον υἱῶν τῶν τέκνων αὐτῶν, ὅ ἐστιν ἐπὶ τῶν προθύρων πύλης τῆς Χαρσείθ· καὶ ἀνάγνωθι ἐκεῖ πάντας τοὺς λόγους τούτους, οὓς ἂν λαλήσω πρὸς σὲ,
\VS{3}καὶ ἐρεῖς αὐτοῖς,
\par }{\PP Ἀκούσατε τὸν λόγον Κυρίου, βασιλεῖς Ἰούδα, καὶ ἄνδρες Ἰούδα, καὶ οἱ κατοικοῦντες ἐν Ἱερουσαλὴμ καὶ οἱ εἰσπορευόμενοι ἐν ταῖς πύλαις ταύταις· τάδε λέγει Κύριος ὁ Θεὸς Ἰσραὴλ, ἰδοὺ ἐγὼ ἐπάγω ἐπὶ τὸν τόπον τοῦτον κακὰ, ὥστε παντὸς ἀκούοντος αὐτὰ ἠχήσει τὰ ὦτα αὐτοῦ·
\VS{4}Ἀνθʼ ὧν ἐγκατέλιπόν με, καὶ ἀπηλλοτρίωσαν τὸν τόπον τοῦτον, καὶ ἐθυμίασαν ἐν αὐτῷ θεοῖς ἀλλοτρίοις, οἷς οὐκ ᾔδεισαν αὐτοὶ καὶ οἱ πατέρες αὐτῶν· καὶ οἱ βασιλεῖς Ἰούδα ἔπλησαν τὸν τόπον τοῦτον αἱμάτων ἀθώων,
\VS{5}καὶ ᾠκοδόμησαν ὑψηλὰ τῇ Βάαλ, τοῦ κατακαίειν τοὺς υἱοὺς αὐτῶν ἐν πυρὶ, ἃ οὐκ ἐνετειλάμην, οὐδὲ διενοήθην ἐν τῇ καρδίᾳ μου.
\par }{\PP \VS{6}Διατοῦτο ἰδοὺ ἡμέραι ἔρχονται, λέγει Κύριος, καὶ οὐ κληθήσεται τῷ τόπῳ τούτῳ ἔτι Διάπτωσις καὶ Πολυάνδριον υἱοῦ Ἐννὼμ, ἀλλʼ ἢ Πολυάνδριον τῆς σφαγῆς.
\VS{7}Καὶ σφάξω τὴν βουλὴν Ἰούδα, καὶ τὴν βουλὴν Ἱερουσαλὴμ ἐν τῷ τόπῳ τούτῳ, καὶ καταβαλῶ αὐτοὺς ἐν μαχαίρᾳ ἐναντίον τῶν ἐχθρῶν αὐτῶν, καὶ ἐν χερσὶ τῶν ζητούντων τὰς ψυχὰς αὐτῶν· καὶ δώσω τοὺς νεκροὺς αὐτῶν εἰς βρῶσιν τοῖς πετεινοῖς τοῦ οὐρανοῦ καὶ τοῖς θηρίοις τῆς γῆς·
\VS{8}Καὶ κατάξω τὴν πόλιν ταύτην εἰς ἀφανισμὸν, καὶ εἰς συρισμόν· πᾶς ὁ παραπορευόμενος ἐπʼ αὐτῆς σκυθρωπάσει, καὶ συριεῖ ὑπὲρ πάσης τῆς πληγῆς αὐτῆς.
\VS{9}Καὶ ἔδονται τὰς σάρκας τῶν υἱῶν αὐτῶν, καὶ τὰς σάρκας τῶν θυγατέρων αὐτῶν· καὶ ἕκαστος τὰς σάρκας τοῦ πλησίον αὐτοῦ ἔδονται ἐν τῇ περιοχῇ, καὶ ἐν τῇ πολιορκίᾳ ᾗ πολιορκήσουσιν αὐτοὺς οἱ ἐχθροὶ αὐτῶν.
\par }{\PP \VS{10}Καὶ συντρίψεις τὸν βικὸν κατʼ ὀφθαλμοὺς τῶν ἀνδρῶν τῶν ἐκπορευομένων μετὰ σοῦ,
\VS{11}καὶ ἐρεῖς, τάδε λέγει Κύριος, οὕτως συντρίψω τὸν λαὸν τοῦτον, καὶ τὴν πόλιν ταύτην, καθὼς συντρίβεται ἄγγος ὀστράκινον, ὃ οὐ δυνήσεται ἰαθῆναι ἔτι.
\VS{12}Οὕτως ποιήσω, λέγει Κύριος, τῷ τόπῳ τούτῳ, καὶ τοῖς κατοικοῦσιν ἐν αὐτῷ, τοῦ δοθῆναι τὴν πόλιν ταύτην, ὡς τὴν διαπίπτουσαν.
\VS{13}Καὶ οἶκοι Ἱερουσαλὴμ, καὶ οἶκοι βασιλέων Ἰούδα ἔσονται καθὼς ὁ τόπος ὁ διαπίπτων, ἀπὸ τῶν ἀκαθαρσιῶν αὐτῶν ἐν πάσαις ταῖς οἰκίαις, ἐν αἷς ἐθυμίασαν ἐπὶ τῶν δωμάτων αὐτῶν πάσῃ τῇ στρατιᾷ τοῦ οὐρανοῦ, καὶ ἔσπεισαν σπονδὰς θεοῖς ἀλλοτρίοις.
\par }{\PP \VS{14}Καὶ ἦλθεν Ἱερεμίας ἀπὸ τῆς διαπτώσεως, οὗ ἀπέστειλεν αὐτὸν Κύριος ἐκεῖ, τοῦ προφητεῦσαι· καὶ ἔστη ἐν τῇ αὐλῇ οἴκου Κυρίου, καὶ εἶπε πρὸς πάντα τὸν λαὸν,
\VS{15}τάδε λέγει Κύριος, ἰδοὺ ἐγὼ ἐπάγω ἐπὶ τὴν πόλιν ταύτην, καὶ ἐπὶ πάσας τὰς πόλεις αὐτῆς, καὶ ἐπὶ τὰς κώμας αὐτῆς, ἅπαντα τὰ κακὰ ἃ ἐλάλησα ἐπʼ αὐτὴν, ὅτι ἐσκλήρυναν τὸν τράχηλον αὐτῶν, τοῦ μὴ εἰσακούειν τῶν ἐντολῶν μου.

\par }\Chap{20}{\PP \VerseOne{1}Καὶ ἤκουσε Πασχὼρ ὁ υἱὸς Ἑμμὴρ ὁ ἱερεὺς, καὶ οὗτος ἦν καθεσταμένος ἡγούμενος οἴκου Κυρίου, τοῦ Ἱερεμίου προφητεύοντος τοὺς λόγους τούτους.
\VS{2}Καὶ ἐπάταξεν αὐτὸν, καὶ ἐνέβαλεν αὐτὸν εἰς τὸν καταράκτην, ὃς ἦν ἐν πύλῃ οἴκου ἀποτεταγμένου τοῦ ὑπερῴου, ὃς ἦν ἐν οἴκῳ Κυρίου.
\par }{\PP \VS{3}Καὶ ἐξήγαγε Πασχὼρ τὸν Ἱερεμίαν ἐκ τοῦ καταράκτου· καὶ εἶπεν αὐτῷ Ἱερεμίας, οὐχὶ Πασχὼρ ἐκάλεσε τὸ ὄνομά σου, ἀλλʼ ἢ Μέτοικον.
\VS{4}Διότι τάδε λέγει Κύριος, ἰδοὺ ἐγὼ δίδωμί σε εἰς μετοικίαν σὺν πᾶσι τοῖς φίλοις σου· καὶ πεσοῦνται ἐν μαχαίρᾳ ἐχθρῶν αὐτῶν, καὶ οἱ ὀφθαλμοί σου ὄψονται· καὶ σὲ καὶ πάντα Ἰούδα δώσω εἰς χεῖρας βασιλέως Βαβυλῶνος, καὶ μετοικιοῦσιν αὐτοὺς, καὶ κατακόψουσιν ἐν μαχαίραις.
\VS{5}Καὶ δώσω τὴν πᾶσαν ἰσχὺν τῆς πόλεως ταύτης, καὶ πάντας τοὺς πόνους αὐτῆς, καὶ πάντας τοὺς θησαυροὺς τοῦ βασιλέως Ἰούδα εἰς χεῖρας ἐχθρῶν αὐτοῦ, καὶ ἄξουσιν αὐτοὺς εἰς Βαβυλῶνα.
\VS{6}Καὶ σὺ καὶ πάντες οἱ κατοικοῦντες ἐν τῷ οἴκῳ σου, πορεύσεσθε ἐν αἰχμαλωσίᾳ, καὶ ἐν Βαβυλῶνι ἀποθανῇ, καὶ ἐκεῖ ταφήσῃ σὺ καὶ πάντες οἱ φίλοι σου, οἷς ἐπροφήτευσας αὐτοῖς ψευδῆ.
\par }{\PP \VS{7}Ἠπάτησάς με Κύριε, καὶ ἠπατήθην, ἐκράτησας, καὶ ἠδυνάσθης· ἐγενόμην εἰς γέλωτα, πᾶσαν ἡμέραν διετέλεσα μυκτηριζόμενος·
\VS{8}Ὅτι πικρῷ λόγῳ μου γελάσομαι, ἀθεσίαν καὶ ταλαιπωρίαν ἐπικαλέσομαι, ὅτι ἐγενήθη λόγος Κυρίου εἰς ὀνειδισμὸν ἐμοὶ καὶ εἰς χλευασμὸν πᾶσαν ἡμέραν μου.
\VS{9}Καὶ εἶπα, οὐ μὴ ὀνομάσω τὸ ὄνομα Κυρίου, καὶ οὐ μὴ λαλήσω ἔτι ἐπὶ τῷ ὀνόματι αὐτοῦ· καὶ ἐγένετο ὡς πῦρ καιόμενον φλέγον ἐν τοῖς ὀστοῖς μου, καὶ παρεῖμαι πάντοθεν, καὶ οὐ δύναμαι φέρειν,
\VS{10}ὅτι ἤκουσα ψόγον πολλῶν συναθροιζομένων κυκλόθεν, ἐπισύστητε, καὶ ἐπισυστῶμεν ἐπʼ αὐτῷ πάντες ἄνδρες φιλοι αὐτοῦ· τηρήσατε τὴν ἐπίνοιαν αὐτοῦ, εἰ ἀπατηθήσεται, καὶ δυνησόμεθα αὐτῷ, καὶ ληψόμεθα τὴν ἐκδίκησιν ἡμῶν ἐξ αὐτοῦ.
\par }{\PP \VS{11}Ὁ δὲ κύριος μετʼ ἐμοῦ καθὼς μαχητὴς ἰσχύων· διατοῦτο ἐδίωξαν, καὶ νοῆσαι οὐκ ἠδύναντο· ᾐσχύνθησαν σφόδρα, ὅτι οὐκ ἐνόησαν ἀτιμίας αὐτῶν, αἳ διʼ αἰῶνος οὐκ ἐπιλησθήσονται.
\par }{\PP \VS{12}Κύριε δοκιμάζων δίκαια, συνιῶν νεφροὺς καὶ καρδίας, ἴδοιμι τὴν παρὰ σοῦ ἐκδίκησιν ἐν αὐτοῖς, ὅτι πρὸς σὲ ἀπεκάλυψα τὰ ἀπολογήματά μου.
\VS{13}Ἄσατε τῷ Κυρίῳ, αἰνέσατε αὐτῷ, ὅτι ἐξείλατο τὴν ψυχὴν πένητος ἐκ χειρὸς πονηρευομένων.
\par }{\PP \VS{14}Ἐπικατάρατος ἡ ἡμέρα ἐν ᾗ ἐτέχθην ἐν αὐτῇ· ἡ ἡμέρα ἐν ᾗ ἔτεκέν με ἡ μήτηρ μου, μὴ ἔστω ἐπευκτή.
\VS{15}Ἐπικατάρατος ὁ ἄνθρωπος ὁ εὐαγγελισάμενος τῷ πατρί μου λέγων, ἐτέχθη σοι ἄρσην· εὐφραινόμενος
\VS{16}ἔστω ὁ ἄνθρωπος ἐκεῖνος, ὡς αἱ πόλεις ἃς κατέστρεψε Κύριος ἐν θυμῷ καὶ οὐ μετεμελήθη· ἀκουσάτω κραυγῆς τῷ πρωὶ, καὶ ἀλαλαγμοῦ μεσημβρίας,
\VS{17}ὅτι οὐκ ἀπέκτεινέ με ἐν μήτρᾳ, καὶ ἐγένετό μοι ἡ μήτηρ μου τάφος μου, καὶ ἡ μήτρα συλλήψεως αἰωνίας.
\VS{18}Ἱνατί τοῦτο ἐξῆλθον ἐκ μήτρας, τοῦ βλέπειν κόπους καὶ πόνους, καὶ διετέλεσαν ἐν αἰσχύνῃ αἱ ἡμέραι μου;

\par }\Chap{21}{\PP \VerseOne{1}Ὁ ΛΟΓΟΣ Ὁ ΓΕΝΟΜΕΝΟΣ ΠΑΡΑ ΚΥΡΙΟΥ ΠΡΟΣ ἹΕΡΕΜΙΑΝ, ὍΤΕ ἈΠΕΣΤΕΙΛΕ ΠΡΟΣ ΑΥΤΟΝ Ὁ ΒΑΣΙΛΕΥΣ ΣΕΔΕΚΙΑΣ ΤΟΝ ΠΑΣΧΩΡ ΥΙΟΝ ΜΕΓΧΙΟΥ, ΚΑΙ ΣΟΦΟΝΙΑΝ ΥΙΟΝ ΒΑΣΑΙΟΥ ΤΟΝ ἹΕΡΕΑ, ΛΕΓΩΝ,
\VS{2}ἐπερώτησον περὶ ἡμῶν τὸν Κύριον· ὅτι βασιλεὺς Βαβυλῶνος ἐφέστηκεν ἐφʼ ἡμᾶς· εἰ ποιήσει Κύριος κατὰ πάντα τὰ θαυμάσια αὐτοῦ, καὶ ἀπελεύσεται ἀφʼ ἡμῶν.
\par }{\PP \VS{3}Καὶ εἶπε πρὸς αὐτοὺς Ἱερεμίας, οὕτως ἐρεῖτε πρὸς Σεδεκίαν βασιλέα Ἰούδα,
\VS{4}τὰδε λέγει Κύριος, ἰδοὺ ἐγὼ μεταστρέφω τὰ ὅπλα τὰ πολεμικὰ, ἐν οἷς ὑμεῖς πολεμεῖτε ἐν αὐτοῖς πρὸς τοὺς Χαλδαίους τοὺς συγκεκλεικότας ὑμᾶς ἔξωθεν τοῦ τείχους· καὶ συνάξω αὐτοὺς εἰς τὸ μέσον τῆς πόλεως ταύτης,
\VS{5}καὶ πολεμήσω ἐγὼ ὑμᾶς ἐν χειρὶ ἐκτεταμένῃ καὶ ἐν βραχίονι κραταιῷ μετὰ θυμοῦ καὶ ὀργῆς μεγάλης.
\VS{6}Καὶ πατάξω πάντας τοὺς κατοικοῦντας ἐν τῇ πόλει ταύτῃ, τοὺς ἀνθρώπους καὶ τὰ κτήνη ἐν θανάτῳ μεγάλῳ, καὶ ἀποθανοῦνται.
\VS{7}Καὶ μετὰ ταῦτα οὕτως λέγει Κύριος, δώσω τὸν Σεδεκίαν βασιλέα Ἰούδα, καὶ τοὺς παῖδας αὐτοῦ, καὶ τὸν λαὸν καταλειφθέντα ἐν τῇ πόλει ταύτῃ ἀπὸ τοῦ θανάτου καὶ ἀπὸ τοῦ λιμοῦ καὶ ἀπὸ τῆς μαχαίρας, εἰς χεῖρας ἐχθρῶν αὐτῶν, τῶν ζητούντων τὰς ψυχὰς αὐτῶν, καὶ κατακόψουσιν αὐτοὺς ἐν στόματι μαχαίρας· οὐ φείσομαι ἐπʼ αὐτοῖς, καὶ οὐ μὴ οἰκτειρήσω αὐτούς.
\par }{\PP \VS{8}Καὶ πρὸς τὸν λαὸν τοῦτον ἐρεῖς, τάδε λέγει Κύριος, ἰδοὺ ἐγὼ δέδωκα πρὸ προσώπου ὑμῶν τὴν ὁδὸν τῆς ζωῆς, καὶ τὴν ὁδὸν τοῦ θανάτου.
\VS{9}Ὁ καθήμενος ἐν τῇ πόλει ταύτῃ, ἀποθανεῖται ἐν μαχαίρᾳ καὶ ἐν λιμῷ· καὶ ὁ ἐκπορευόμενος προσχωρῆσαι πρὸς τοὺς Χαλδαίους τοὺς συγκεκλεικότας ὑμᾶς, ζήσεται, καὶ ἔσται ἡ ψυχὴ αὐτοῦ εἰς σκῦλα, καὶ ζήσεται.
\VS{10}Διότι ἐστήρικα τὸ πρόσωπόν μου ἐπὶ τὴν πόλιν ταύτην εἰς κακὰ, καὶ οὐκ εἰς ἀγαθά· εἰς χεῖρας βασιλέως Βαβυλῶνος παραδοθήσεται, καὶ κατακαύσει αὐτὴν ἐν πυρί.
\par }{\PP \VS{11}Ὁ οἶκος βασιλέως Ἰούδα, ἀκούσατε λόγον Κυρίου.
\VS{12}Οἶκος Δαυὶδ, τάδε λέγει Κύριος, κρίνατε πρωῒ κρίμα, καὶ κατευθύνατε, καὶ ἐξέλεσθε διηρπασμένον ἐκ χειρὸς ἀδικοῦντος αὐτὸν, ὅπως μὴ ἀναφθῇ ὡς πῦρ ἡ ὀργή μου, καὶ καυθήσεται, καὶ οὐκ ἔσται ὁ σβέσων.
\VS{13}Ἰδοὺ ἐγὼ πρὸς σὲ τὸν κατοικοῦντα τὴν κοιλάδα σὸρ, τὴν πεδεινὴν, τοὺς λέγοντας, τίς πτοήσει ἡμᾶς; ἢ τίς εἰσελεύσεται πρὸς τὸ κατοικητήριον ἡμῶν;
\VS{14}Καὶ ἀνάψω πῦρ ἐν τῷ δρυμῷ αὐτῆς, καὶ ἔδεται πάντα τὰ κύκλῳ αὐτῆς.

\par }\Chap{22}{\PP \VerseOne{1}Τάδε λέγει Κύριος, πορεύου καὶ κατάβηθι εἰς τὸν οἶκον τοῦ βασιλέως Ἰούδα, καὶ λαλήσεις ἐκεῖ τὸν λόγον τοῦτον,
\VS{2}καὶ ἐρεῖς,
\par }{\PP Ἄκουε λόγον Κυρίου βασιλεῦ Ἰούδα, ὁ καθήμενος ἐπὶ θρόνου Δαυὶδ, σὺ καὶ ὁ οἶκός σου καὶ ὁ λαός σου, καὶ οἱ εἰσπορευόμενοι ταῖς πύλαις ταύταις·
\VS{3}Τάδε λέγει Κύριος, ποιεῖτε κρίσιν καὶ δικαιοσύνην, καὶ ἐξαιρεῖσθε διηρπασμένον ἐκ χειρὸς ἀδικοῦντος αὐτὸν, καὶ προσήλυτον καὶ ὀρφανὸν καὶ χήραν μὴ καταδυναστεύετε, καὶ μὴ ἀσεβεῖτε, καὶ αἷμα ἀθῶον μὴ ἐκχέητε ἐν τῷ τόπῳ τούτῳ.
\VS{4}Διότι ἐὰν ποιοῦντες ποιήσητε τὸν λόγον τοῦτον, καὶ εἰσελεύσονται ἐν ταῖς πύλαις τοῦ οἴκου τούτου βασιλεῖς καθήμενοι ἐπὶ θρόνου Δαυὶδ, καὶ ἐπιβεβηκότες ἐφʼ ἁρμάτων καὶ ἵππων, αὐτοὶ καὶ οἱ παῖδες αὐτῶν, καὶ ὁ λαὸς αὐτῶν.
\VS{5}Ἐὰν δὲ μὴ ποιήσητε τοὺς λόγους τούτους, κατʼ ἐμαυτοῦ ὤμοσα, λέγει Κύριος, ὅτι εἰς ἐρήμωσιν ἔσται ὁ οἶκος οὗτος.
\par }{\PP \VS{6}Ὅτι τάδε λέγει Κύριος κατὰ τοῦ οἴκου βασιλέως Ἰούδα, Γαλαὰδ σύ μοι ἀρχὴ τοῦ Λιβάνου, ἐὰν μὴ θῶ σε εἰς ἔρημον, πόλεις μὴ κατοικηθησομένας,
\VS{7}καὶ ἐπάξω ἐπὶ σὲ ὀλοθρεύοντα ἄνδρα, καὶ τὸν πέλεκυν αὐτοῦ, καὶ ἐκκόψουσι τὰς ἐκλεκτὰς κέδρους σου, καὶ ἐμβαλοῦσιν εἰς τὸ πῦρ.
\VS{8}Καὶ διελεύσονται ἔθνη διὰ τῆς πόλεως ταύτης, καὶ ἐρεῖ ἕκαστος πρὸς τὸν πλησίον αὐτοῦ, διατί ἐποίησε Κύριος οὕτως τῇ πόλει ταύτῃ τῇ μεγάλῃ;
\VS{9}Καὶ ἐροῦσιν, ἀνθʼ ὧν ἐγκατέλιπον τὴν διαθήκην Κυρίου Θεοῦ αὐτῶν, καὶ προσεκύνησαν θεοῖς ἀλλοτρίοις, καὶ ἐδούλευσαν αὐτοῖς.
\par }{\PP \VS{10}Μὴ κλαίετε τὸν τεθνηκότα, μηδὲ θρηνεῖτε αὐτόν· κλαύσατε κλαυθμῷ τὸν ἐκπορευόμενον, ὅτι οὐκ ἐπιστρέψει ἔτι, οὐδὲ ὄψεται τὴν γῆν πατρίδος αὐτοῦ.
\VS{11}Διότι τάδε λέγει Κύριος ἐπὶ Σελλὴμ υἱὸν Ἰωσία τὸν βασιλεύοντα ἀντὶ Ἰωσίου τοῦ πατρὸς αὐτοῦ, ὃς ἐξῆλθεν ἐκ τοῦ τόπου τούτου· οὐκ ἀναστρέψει ἐκεῖ ἔτι,
\VS{12}ἀλλʼ ἢ ἐν τῷ τόπῳ τούτῳ οὗ μετῴκισα αὐτὸν, ἐκεῖ ἀποθανεῖται, καὶ τὴν γῆν ταύτην οὐκ ὄψεται ἔτι.
\par }{\PP \VS{13}Ὁ οἰκοδομῶν οἰκίαν αὐτοῦ οὐ μετὰ δικαιοσύνης, καὶ τὰ ὑπερῷα αὐτοῦ οὐκ ἐν κρίματι, παρὰ τῷ πλησίον αὐτοῦ ἐργᾶται δωρεὰν, καὶ τὸν μισθὸν αὐτοῦ οὐ μὴ ἀποδώσει αὐτῷ.
\VS{14}Ὠκοδόμησας σεαυτῷ οἶκον σύμμετρον, ὑπερῷα ῥιπιστὰ διεσταλμένα θυρίσι, καὶ ἐξυλωμένα ἐν κέδρῳ, καὶ κεχρισμένα ἐν μίλτῳ.
\VS{15}Μὴ βασιλεύσῃς, ὅτι σὺ παροξυνῇ ἐν Ἄχαζ τῷ πατρί σου; οὐ φάγονται, καὶ οὐ πίονται· βέλτιόν σε ποιεῖν κρίμα καὶ δικαιοσύνην.
\VS{16}Οὐκ ἔγνωσαν, οὐκ ἔκριναν κρίσιν ταπεινῷ, οὐδὲ κρίσιν πένητος· οὐ τοῦτό ἐστι τὸ μὴ γηῶναί σε ἐμὲ; λέγει Κύριος.
\VS{17}Ἰδοὺ οὐκ εἰσὶν οἱ ὀφθαλμοί σου, οὐδὲ ἡ καρδία σου καλὴ, ἀλλὰ εἰς τὴν πλεονεξίαν σου, καὶ εἰς τὸ αἷμα τὸ ἀθῶον τοῦ ἐκχέαι αὐτὸ, καὶ εἰς ἀδικήματα καὶ εἰς φόνον, τοῦ ποιεῖν αὐτά.
\par }{\PP \VS{18}Διατοῦτο τάδε λέγει Κύριος ἐπὶ Ἰωακεὶμ υἱὸν Ἰωσία βασιλέα Ἰούδα, καὶ ἐπὶ τὸν ἄνδρα τοῦτον, οὐ κόψονται αὐτὸν, ὦ ἀδελφὲ, οὐδὲ μὴ κλαύσονται αὐτὸν, οἴμοι Κύριε.
\VS{19}Ταφὴν ὄνου ταφήσεται, συμψησθεὶς ῥιφήσεται ἐπέκεινα τῆς πύλης Ἱερουσαλήμ.
\par }{\PP \VS{20}Ἀνάβηθι εἰς τὸν Λίβανον, καὶ κράξον, καὶ εἰς τὴν Βασὰν δὸς τὴν φωνήν σου, καὶ βόησον εἰς τὸ πέρας τῆς θαλάσσης, ὅτι συνετρίβησαν πάντες οἱ ἐρασταί σου.
\VS{21}Ἐλάλησα πρὸς σὲ ἐν τῇ παραπτώσει σου, καὶ εἶπας, οὐκ ἀκούσομαι· αὕτη ἡ ὁδός σου ἐκ νεότητός σου, οὐκ ἤκουσας τῆς φωνῆς μου.
\VS{22}Πάντας τοὺς ποιμένας σου ποιμανεῖ ἄνεμος, καὶ οἱ ἐρασταί σου ἐν αἰχμαλωσίᾳ ἐξελεύσονται, ὅτι τότε αἰσχυνθήσῃ καὶ ἀτιμωθήσῃ ἀπὸ πάντων τῶν φιλούντων σε.
\VS{23}Κατοικοῦσα ἐν τῷ Λιβάνῳ, ἐννοσσεύουσα ἐν ταῖς κέδροις, καταστενάξεις ἐν τῷ ἐλθεῖν σοι ὀδύνας ὡς τικτούσης.
\par }{\PP \VS{24}Ζῶ ἐγὼ, λέγει Κύριος, ἐὰν γενόμενος γένηται Ἰεχονίας υἱὸς Ἰωακεὶμ βασιλεὺς Ἰούδα ἀποσφράγισμα ἐπὶ τῆς χειρὸς τῆς δεξιᾶς μου, ἐκεῖθεν ἐκσπάσω σε,
\VS{25}καὶ παραδώσω σε εἰς χεῖρας τῶν ζητούντων τὴν ψυχήν σου, ὧν σὺ εὐλαβῇ ἀπὸ προσώπου αὐτῶν, εἰς χεῖρας τῶν Χαλδαίων,
\VS{26}καὶ ἀποῤῥίψω σε καὶ τὴν μητέρα σου τὴν τεκοῦσάν σε εἰς γῆν, οὗ οὐκ ἐτέχθης ἐκεῖ, καὶ ἐκεῖ ἀποθανεῖσθε·
\VS{27}Εἰς δὲ τὴν γῆν ἣν αὐτοὶ εὔχονται ταῖς ψυχαῖς αὐτῶν, οὐ μὴ ἀποστρέψωσιν.
\VS{28}Ἠτιμώθη Ἰεχονίας ὡς σκεῦος οὗ οὐκ ἔστι χρεία αὐτοῦ, ὅτι ἐξεῤῥίφη, καὶ ἐξεβλήθη εἰς γῆν ἣν οὐκ ᾔδει.
\par }{\PP \VS{29}Γῆ, γῆ ἄκουε λόγον Κυρίου,
\VS{30}γράψον τὸν ἄνδρα τοῦτον ἐκκήρυκτον ἄνθρωπον, ὅτι οὐ μὴ αὐξηθῇ ἐκ τοῦ σπέρματος αὐτοῦ καθήμενος ἐπὶ θρόνου Δαυὶδ, ἄρχων ἔτι ἐν τῷ Ἰούδα.

\par }\Chap{23}{\PP \VerseOne{1}Ὢ ποιμένες οἱ ἀπολλύοντες καὶ διασκορπίζοντες τὰ πρόβατα τῆς νομῆς αὐτῶν.
\VS{2}Διατοῦτο τάδε λέγει Κύριος ἐπὶ τοὺς ποιμαίνοντας τὸν λαόν μου, ὑμεῖς διεσκορπίσατε τὰ πρόβατά μου, καὶ ἐξώσατε αὐτὰ, καὶ οὐκ ἐπεσκέψασθε αὐτὰ, ἰδοὺ ἐγὼ ἐκδικῶ ἐφʼ ὑμᾶς κατὰ τὰ πονηρὰ ἐπιτηδεύματα ὑμῶν.
\VS{3}Καὶ ἐγὼ εἰσδέξομαι τοὺς καταλοίπους τοῦ λαοῦ μου ἐπὶ πάσης τῆς γῆς, οὗ ἔξωσα αὐτοὺς ἐκεῖ, καὶ καταστήσω αὐτοὺς εἰς τὴν νομὴν αὐτῶν, καὶ αὐξηθήσονται, καὶ πληθυνθήσονται.
\VS{4}Καὶ ἀναστήσω αὐτοῖς ποιμένας, οἳ ποιμανοῦσιν αὐτοὺς, καὶ οὐ φοβηθήσονται ἔτι, οὐδὲ πτοηθήσονται, λέγει Κύριος.
\par }{\PP \VS{5}Ἰδοὺ ἡμέραι ἔρχονται, λέγει Κύριος, καὶ ἀναστήσω τῷ Δαυὶδ ἀνατολὴν δικαίαν, καὶ βασιλεύσει βασιλεὺς, καὶ συνήσει, καὶ ποιήσει κρίμα καὶ δικαιοσύνην ἐπὶ τῆς γῆς.
\VS{6}Ἐν ταῖς ἡμέραις αὐτοῦ καὶ σωθήσεται Ἰούδας, καὶ Ἰσραὴλ κατασκηνώσει πεποιθὼς, καὶ τοῦτο τὸ ὄνομα αὐτοῦ, ὃ καλέσει αὐτὸν Κύριος, Ἰωσεδὲκ ἐν τοῖς προφήταις.
\par }{\PP \VS{9}Συνετρίβη ἡ καρδία μου ἐν ἐμοὶ, ἐσαλεύθη πάντα τὰ ὀστᾶ μου, ἐγενήθην ὡς ἀνὴρ συντετριμμένος, καὶ ὡς ἄνθρωπος συνεχόμενος ἀπὸ οἶνου ἀπὸ προσώπου Κυρίου καὶ ἀπὸ προσώπου εὐπρεπείας δόξης αὐτοῦ.
\VS{10}Ὅτι ἀπὸ προσώπου τούτων ἐπένθησεν ἡ γῆ, ἐξηράνθησαν αἱ νομαὶ τῆς ἐρήμου· καὶ ἐγένετο ὁ δρόμος αὐτῶν πονηρὸς, καὶ ἡ ἰσχὺς αὐτῶν οὕτως.
\VS{11}Ὅτι ἱερεὺς καὶ προφήτης ἐμολύνθησαν, καὶ ἐν τῷ οἴκῳ μου εἶδον πονηρίας αὐτῶν.
\VS{12}Διατοῦτο γενέσθω ἡ ὁδὸς αὐτῶν αὐτοῖς εἰς ὀλίσθημα ἐν γνόφῳ, καὶ ὑποσκελισθήσονται, καὶ πεσοῦνται ἐν αὐτῇ· διότι ἐπάξω ἐπʼ αὐτοὺς κακὰ, ἐν ἐνιαυτῷ ἐπισκέψεως αὐτῶν.
\par }{\PP \VS{13}Καὶ ἐν τοῖς προφήταις Σαμαρείας εἶδον ἀνομήματα· ἐπροφήτευσαν διὰ τῆς Βάαλ, καὶ ἐπλάνησαν τὸν λαόν μου Ἰσραήλ.
\VS{14}Καὶ ἐν τοῖς προφήταις Ἱερουσαλὴμ ἑώρακα φρικτὰ, μοιχωμένους, καὶ πορευομένους ἐν ψεύδεσι, καὶ ἀντιλαμβανομένους χειρῶν πολλῶν, τοῦ μὴ ἀποστραφῆναι ἕκαστον ἀπὸ τῆς ὁδοῦ αὐτοῦ τῆς πονηρᾶς· ἐγενήθησάν μοι πάντες ὡς Σόδομα, καὶ οἱ κατοικοῦντες αὐτὴν ὥσπερ Γόμοῤῥα.
\par }{\PP \VS{15}Διατοῦτο τάδε λέγει Κύριος, ἰδοὺ ἐγὼ ψωμιῶ αὐτοὺς ὀδύνην, καὶ ποτιῶ αὐτοὺς ὕδωρ πικρὸν, ὅτι ἀπὸ τῶν προφητῶν Ἱερουσαλὴμ ἐξῆλθε μολυσμὸς πάσῃ τῇ γῇ.
\par }{\PP \VS{16}Οὕτως λέγει Κύριος παντοκράτωρ, μὴ ἀκούετε τοὺς λόγους τῶν προφητῶν, ὅτι ματαιοῦσιν ἑαυτοῖς ὅρασιν, ἀπὸ καρδίας αὐτῶν λαλοῦσι, καὶ οὐκ ἀπὸ στόματος Κυρίου.
\VS{17}Λέγουσι τοῖς ἀπωθουμένοις τὸν λόγον Κυρίου, εἰρήνη ἔσται ὑμῖν, καὶ πᾶσι τοῖς πορευομένοις τοῖς θελήμασιν αὐτῶν, καὶ παντὶ τῷ πορευομένῳ πλάνῃ καρδίας αὐτοῦ, εἶπαν, οὐχ ἥξει ἐπὶ σὲ κακά.
\VS{18}Ὅτι τίς ἔστη ἐν ὑποστήματι Κυρίου, καὶ εἶδε τὸν λόγον αὐτοῦ; τίς ἠνωτίσατο, καὶ ἤκουσεν;
\VS{19}Ἰδοὺ σεισμὸς παρὰ Κυρίου, καὶ ὀργὴ ἐκπορεύεται εἰς συσσεισμὸν, συστρεφομένη ἐπὶ τοὺς ἀσεβεῖς ἥξει.
\VS{20}Καὶ οὐκ ἔτι ἀποστρέψει ὁ θυμὸς Κυρίου, ἕως ποιήσῃ αὐτὸ, καὶ ἕως ἂν στήσῃ αὐτὸ, ἀπὸ ἐγχειρήματος καρδίας αὐτοῦ· ἐπʼ ἐσχάτου τῶν ἡμερῶν νοήσουσιν αὐτό.
\par }{\PP \VS{21}Οὐκ ἀπέστελλον τοὺς προφήτας, καὶ αὐτοὶ ἔτρεχον· οὐδὲ ἐλάλησα πρὸς αὐτοὺς, καὶ αὐτοὶ ἐπροφήτευον.
\VS{22}Καὶ εἰ ἔστησαν ἐν τῇ ὑποστάσει μου, καὶ εἰ ἤκουσαν τῶν λόγων μου, καὶ τὸν λαόν μου ἂν ἀπέστρεφον αὐτοὺς ἀπὸ τῶν πονηρῶν ἐπιτηδευμάτων αὐτῶν.
\par }{\PP \VS{23}Θεὸς ἐγγίζων ἐγώ εἰμι, λέγει Κύριος, καὶ οὐχὶ Θεὸς πόῤῥωθεν.
\VS{24}Εἰ κρυβήσεταί τις ἐν κρυφαίοις, καὶ ἐγὼ οὐκ ὄψομαι αὐτόν; μὴ οὐχὶ τὸν οὐρανὸν καὶ τὴν γῆν ἐγὼ πληρῶ; λέγει Κύριος.
\par }{\PP \VS{25}Ἤκουσα ἃ λαλοῦσιν οἱ προφῆται, ἃ προφητεύουσιν ἐπὶ τῷ ὀνόματί μου, ψευδῆ λέγοντες, ἠνυπνιασάμην ἐνύπνιον.
\VS{26}Ἕως πότε ἔσται ἐν καρδίᾳ τῶν προφητῶν τῶν προφητευόντων ψευδῆ, ἐν τῷ προφητεύειν αὐτοὺς τὰ θελήματα τῆς καρδίας αὐτῶν,
\VS{27}τῶν λογιζομένων τοῦ ἐπιλαθέσθαι τοῦ νόμου μου ἐν τοῖς ἐνυπνίοις αὐτῶν, ἃ διηγοῦντο ἕκαστος τῷ πλησίον αὐτοῦ, καθάπερ ἐπελάθοντο οἱ πατέρες αὐτῶν τοῦ ὀνόματός μου ἐν τῇ Βάαλ;
\VS{28}Ὁ προφήτης ἐν ᾧ τὸ ἐνύπνιόν ἐστι, διηγησάσθω τὸ ἐνύπνιον αὐτοῦ, καὶ ἐν ᾧ ὁ λόγος μου πρὸς αὐτὸν, διηγησάσθω τὸν λόγον μου ἐπʼ ἀληθείας· τί τὸ ἄχυρον πρὸς τὸν σῖτον; οὕτως οἱ λόγοι μου, λέγει Κύριος.
\VS{29}Οὐκ ἰδοὺ οἱ λόγοι μου, ὥσπερ πῦρ, λέγει Κύριος, καὶ ὡς πέλυξ κόπτων πέτραν;
\par }{\PP \VS{30}Ἰδοὺ ἐγὼ διατοῦτο πρὸς τοὺς προφήτας, λέγει Κύριος ὁ Θεὸς, τοὺς κλέπτοντας τοὺς λόγους μου ἕκαστον παρὰ τοῦ πλησίον αὐτοῦ.
\VS{31}Ἰδοὺ ἐγὼ πρὸς τοὺς προφήτας τοὺς ἐκβάλλοντας προφητείας γλώσσης, καὶ νυστάζοντας νυσταγμὸν αὐτῶν.
\VS{32}Διατοῦτο ἰδοὺ ἐγὼ πρὸς τοὺς προφήτας τοὺς προφητεύοντας ἐνύπνια ψευδῆ, καὶ οὐ διηγοῦντο αὐτὰ, καὶ ἐπλάνησαν τὸν λαόν μου ἐν τοῖς ψεύδεσιν αὐτῶν, καὶ ἐν τοῖς πλάνοις αὐτῶν, καὶ ἐγὼ οὐκ ἀπέστειλα αὐτοὺς, καὶ οὐκ ἐνετειλάμην αὐτοῖς, καὶ ὠφέλειαν οὐκ ὠφελήσουσι τὸν λαὸν τοῦτον.
\par }{\PP \VS{33}Καὶ ἐὰν ἐρωτήσωσιν ὁ λαὸς οὗτος, ἢ ἱερεὺς, ἢ προφήτης, τί τὸ λῆμμα Κυρίου; καὶ ἐρεῖς αὐτοῖς, ὑμεῖς ἐστε τὸ λῆμμα, καὶ ῥάξω ὑμᾶς, λέγει Κύριος.
\VS{34}Ὁ προφήτης, καὶ οἱ ἱερεῖς, καὶ ὁ λαὸς, οἳ ἂν εἴπωσι, λῆμμα Κυρίου, καὶ ἐκδικήσω τὸν ἄνθρωπον ἐκεῖνον, καὶ τὸν οἶκον αὐτοῦ.
\VS{35}Οὕτως ἐρεῖτε ἕκαστος πρὸς τὸν πλησίον αὐτοῦ, καὶ ἕκαστος πρὸς τὸν ἀδελφὸν αὐτοῦ, τί ἀπεκρίθη Κύριος, καὶ τί ἐλάλησε Κύριος;
\VS{36}Καὶ λῆμμα Κυρίου μὴ ὀνομάζετε ἔτι, ὅτι τὸ λῆμμα τῷ ἀνθρώπῳ ἔσται ὁ λόγος αὐτοῦ.
\VS{37}Καὶ διατί ἐλάλησε Κύριος ὁ Θεὸς ἡμῶν;
\VS{38}Διατοῦτο τάδε λέγει Κύριος ὁ Θεὸς ἡμῶν ἄνθʼ ὧν εἴπατε τὸν λόγον τοῦτον, λῆμμα Κυρίου, καὶ ἀπέστειλα πρὸς ὑμᾶς, λέγων, οὐκ ἐρεῖτε, λῆμμα Κυρίου·
\VS{39}διατοῦτο ἰδοὺ ἐγὼ λαμβάνω, καὶ ῥάσσω ὑμᾶς, καὶ τὴν πόλιν ἣν ἔδωκα ὑμῖν καὶ τοῖς πατράσιν ὑμῶν.
\VS{40}Καὶ δώσω ἐφʼ ὑμᾶς ὀνειδισμὸν αἰώνιον, καὶ ἀτιμίαν αἰώνιον, ἥτις οὐκ ἐπιλησθήσεται.
\par }{\PP \VS{40a}Διατοῦτο ἰδοὺ ἡμέραι ἔρχονται, λέγει Κύριος, καὶ οὐκ ἐροῦσιν ἔτι, ζῇ Κύριος, ὃς ἀνήγαγε τὸν οἶκον Ἰσραὴλ ἐκ γῆς Αἰγύπτου,
\VS{40b}ἀλλὰ, ζῇ Κύριος, ὃς συνήγαγε πᾶν τὸ σπέρμα Ἰσραὴλ ἀπὸ γῆς Βοῤῥᾶ, καὶ ἀπὸ πασῶν τῶν χωρῶν, οὗ ἔξωσεν αὐτοὺς ἐκεῖ, καὶ ἀπεκατέστησεν αὐτοὺς εἰς τὴν γῆν αὐτῶν.

\par }\Chap{24}{\PP \VerseOne{1}Ἔδειξέ μοι Κύριος δύο καλάθους σύκων, κειμένους κατὰ πρόσωπον ναοῦ Κυρίου, μετὰ τὸ ἀποικίσαι Ναβουχοδονόσορ βασιλέα Βαβυλῶνος τὸν Ἰεχονίαν υἱὸν Ἰωακεὶμ βασιλέα Ἰούδα, καὶ τοὺς ἄρχοντας, καὶ τοὺς τεχνίτας, καὶ τοὺς δεσμώτας, καὶ τοὺς πλουσίους ἐξ Ἱερουσαλὴμ, καὶ ἤγαγεν αὐτοὺς εἰς Βαβυλῶνα.
\VS{2}Ὁ κάλαθος ὁ εἷς σύκων χρηστῶν σφόδρα, ὡς τὰ σύκα τὰ πρώϊμα· καὶ ὁ κάλαθος ὁ ἕτερος σύκων πονηρῶν σφόδρα, ἃ οὐ βρωθήσεται ἀπὸ πονηρίας αὐτῶν.
\VS{3}Καὶ εἶπε Κύριος πρὸς μὲ, τί σὺ ὁρᾷς Ἱερεμία; καὶ εἶπα, σύκα· σύκα τὰ χρηστὰ, χρηστὰ λίαν· καὶ τὰ πονηρὰ, πονηρὰ λίαν, ἃ οὐ βρωθήσεται ἀπὸ πονηρίας αὐτῶν.
\par }{\PP \VS{4}Καὶ ἐγένετο λόγος Κυρίου πρὸς μὲ, λέγων,
\VS{5}Τάδε λέγει Κύριος ὁ Θεὸς Ἰσραὴλ, ὡς τὰ σύκα τὰ χρηστὰ ταῦτα, οὕτως ἐπιγνώσομαι τοὺς ἀποικισθέντας Ἰουδαίους, οὓς ἐξαπέσταλκα ἐκ τοῦ τόπου τούτου εἰς γῆν Χαλδαίων εἰς ἀγαθά.
\VS{6}Καὶ στηριῶ τοὺς ὀφθαλμούς μου ἐπʼ αὐτοὺς εἰς ἀγαθὰ, καὶ ἀποκαταστήσω αὐτοὺς εἰς τὴν γῆν ταύτην εἰς ἀγαθά· καὶ ἀνοικοδομήσω αὐτοὺς, καὶ οὐ μὴ καθελῶ αὐτοὺς, καὶ καταφυτεύσω αὐτοὺς, καὶ οὐ μὴ ἐκτίλω.
\par }{\PP \VS{7}Καὶ δώσω αὐτοῖς καρδίαν τοῦ εἰδέναι αὐτοὺς ἐμὲ, ὅτι ἐγώ εἰμι Κύριος, καὶ ἔσονταί μου εἰς λαὸν, καὶ ἐγὼ ἔσομαι αὐτοῖς εἰς Θεὸν, ὅτι ἐπιστραφήσονται ἐπʼ ἐμὲ ἐξ ὅλης τῆς καρδίας αὐτῶν.
\par }{\PP \VS{8}Καὶ ὡς τὰ σύκα τὰ πονηρὰ, ἃ οὐ βρωθήσονται ἀπὸ πονηρίας αὐτῶν, τάδε λέγει Κύριος, οὕτως παραδώσω τὸν Σεδεκίαν βασιλέα Ἰούδα, καὶ τοὺς μεγιστᾶνας αὐτοῦ, καὶ τὸ κατάλοιπον Ἱερουσαλὴμ τοὺς ὑπολελειμμένους ἐν τῇ γῇ ταύτῃ, καὶ τοὺς κατοικοῦντας ἐν Αἰγύπτῳ.
\VS{9}Καὶ δώσω αὐτοὺς εἰς διασκορπισμὸν εἰς πάσας τὰς βασιλείας τῆς γῆς, καὶ ἔσονται εἰς ὀνειδισμὸν, καὶ εἰς παραβολὴν, καὶ εἰς μῖσος, καὶ εἰς κατάραν ἐν παντὶ τόπῳ οὗ ἔξωσα αὐτοὺς ἐκεῖ.
\VS{10}Καὶ ἀποστελῶ εἰς αὐτοὺς τὸν λιμὸν, καὶ τὸν θάνατον, καὶ τὴν μάχαιραν, ἕως ἂν ἐκλείπωσιν ἀπὸ τῆς γῆς ἧς ἔδωκα αὐτοῖς.

\par }\Chap{25}{\PP \VerseOne{1}Ὁ ΛΟΓΟΣ Ὁ ΓΕΝΟΜΕΝΟΣ ΠΡΟΣ ἹΕΡΕΜΙΑΝ ἐπὶ πάντα τὸν λαὸν Ἰούδα ἐν τῷ ἔτει τῷ τετάρτῳ τοῦ Ἰωακεὶμ υἱοῦ Ἰωσία βασιλέως Ἰούδα,
\VS{2}ὃν ἐλάλησε πρὸς πάντα τὸν λαὸν Ἰούδα, καὶ πρὸς τοὺς κατοικοῦντας Ἱερουσαλὴμ, λέγων,
\par }{\PP \VS{3}Ἐν τρισκαιδεκάτῳ ἔτει Ἰωσία υἱοῦ Ἀμὼς βασιλέως Ἰούδα, καὶ ἕως τῆς ἡμέρας ταύτης εἴκοσι καὶ τρία ἔτη, καὶ ἐλάλησα πρὸς ὑμᾶς ὀρθρίζων, καὶ λέγων,
\VS{4}καὶ ἀπέστελλον πρὸς ὑμᾶς τοὺς δούλους μου τοὺς προφήτας, ὄρθρου ἀποστέλλων, καὶ οὐκ εἰσηκούσατε, καὶ οὐ προσέσχετε τοῖς ὠσὶν ὑμῶν,
\VS{5}λέγων, ἀποστράφητε ἕκαστος ἀπὸ τῆς ὁδοῦ αὐτοῦ τῆς πονηρᾶς, καὶ ἀπὸ τῶν πονηρῶν ἐπιτηδευμάτων ὑμῶν, καὶ κατοικήσετε ἐπὶ τῆς γῆς ἧς ἔδωκα ὑμῖν καὶ τοῖς πατράσιν ὑμῶν, ἀπʼ αἰῶνος καὶ ἕως αἰῶνος.
\VS{6}Μὴ πορεύεσθε ὀπίσω θεῶν ἀλλοτρίων, τοῦ δουλεύειν αὐτοῖς, καὶ τοῦ προσκυνεῖν αὐτοῖς, ὅπως μὴ παροργίζητέ με ἐν τοῖς ἔργοις τῶν χειρῶν ὑμῶν, τοῦ κακῶσαι ὑμᾶς.
\VS{7}Καὶ οὐκ ἠκούσατέ μου.
\par }{\PP \VS{8}Διατοῦτο τάδε λέγει Κύριος, ἐπειδὴ οὐκ ἐπιστεύσατε τοῖς λόγοις μου,
\VS{9}ἰδοὺ ἐγὼ ἀποστέλλω, καὶ λήψομαι πατριὰν ἀπὸ Βοῤῥᾶ, καὶ ἄξω αὐτοὺς ἐπὶ τὴν γῆν ταύτην, καὶ ἐπὶ τοὺς κατοικοῦντας αὐτὴν, καὶ ἐπὶ πάντα τὰ ἔθνη τὰ κύκλῳ αὐτῆς, καὶ ἐξερημώσω αὐτοὺς, καὶ δώσω αὐτοὺς εἰς ἀφανισμὸν, καὶ εἰς συριγμὸν, καὶ εἰς ὀνειδισμὸν αἰώνιον.
\VS{10}Καὶ ἀπολῶ ἀπʼ αὐτῶν φωνὴν χαρᾶς, καὶ φωνὴν εὐφροσύνης, φωνὴν νυμφίου καὶ φωνὴν νύμφης, ὀσμὴν μύρου, καὶ φῶς λύχνου.
\VS{11}Καὶ ἔσται πᾶσα ἡ γῆ εἰς ἀφανισμὸν, καὶ δουλεύσουσιν ἐν τοῖς ἔθνεσιν ἑβδομήκοντα ἔτη.
\par }{\PP \VS{12}Καὶ ἐν τῷ πληρωθῆναι τὰ ἑβδομήκοντα ἔτη, ἐκδικήσω τὸ ἔθνος ἐκεῖνο, καὶ θήσομαι αὐτοὺς εἰς ἀφανισμὸν αἰώνιον,
\VS{13}καὶ ἐπάξω ἐπὶ τὴν γῆν ἐκείνην πάντας τοὺς λόγους μου, οὓς ἐλάλησα κατʼ αὐτῆς, πάντα τὰ γεγραμμένα ἐν τῷ βιβλίῳ τούτῳ·
\par }{\PP \VS{14}Ἃ ἘΠΡΟΦΗΤΕΥΣΕΝ ἹΕΡΕΜΙΑΣ ἘΠΙ ΤΑ ἜΘΝΗ ΤΑ ΑΙΛΑΜ.
\par }{\PP \VS{15}Τάδε λέγει Κύριος, συνετρίβη τὸ τόξον Αἰλὰμ, ἀρχὴ δυναστείας αὐτῶν.
\VS{16}Καὶ ἐπάξω ἐπὶ Αἰλὰμ τέσσαρας ἀνέμους ἐκ τῶν τεσσάρων ἄκρων τοῦ οὐρανοῦ, καὶ διασπερῶ αὐτοὺς ἐν πᾶσι τοῖς ἀνέμοις τούτοις, καὶ οὐκ ἔσται ἔθνος ὃ οὐχ ἥξει ἐκεῖ, οἱ ἐξωσμένοι Αἰλάμ.
\VS{17}Καὶ πτοήσω αὐτοὺς ἐναντίον τῶν ἐχθρῶν αὐτῶν, τῶν ζητούντων τὴν ψυχὴν αὐτῶν, καὶ ἐπάξω ἐπʼ αὐτοὺς κατὰ τὴν ὀργὴν τοῦ θυμοῦ μου, καὶ ἐπαποστελῶ ὀπίσω αὐτῶν τὴν μάχαιράν μου, ἕως τοῦ ἐξαναλῶσαι αὐτούς.
\VS{18}Καὶ θήσω τὸν θρόνον μου ἐν Αἰλὰμ, καὶ ἐξαποστελῶ ἐκεῖθεν βασιλέα καὶ μεγιστᾶνας.
\VS{19}Καὶ ἔσται ἐπʼ ἐσχάτου τῶν ἡμερῶν, καὶ ἀποστρέψω τὴν αἰχμαλωσίαν Αἰλὰμ, λέγει Κύριος.
\par }{\PP \VS{20}Ἐν ἀρχῇ βασιλεύοντος Σεδεκίου βασιλέως, ἐγένετο ὁ λόγος οὗτος περὶ Αἰλάμ·

\par }\Chap{26}{\PP \VS{2}ΤΗ ΑΙΓΥΠΤΩ ἘΠΙ ΔΥΝΑΜΙΝ ΦΑΡΑΩ ΝΕΧΑΩ ΒΑΣΙΛΕΩΣ ΑΙΓΥΠΤΟΥ, ὃς ἦν ἐπὶ τῷ ποταμῷ Εὐφράτῃ ἐν Χαρμεὶς, ὃν ἐπάταξε Ναβουχοδονόσορ βασιλεὺς Βαβυλῶνος, ἐν τῷ ἔτει τῷ τετάρτῳ Ἰωακεὶμ βασιλέως Ἰούδα.
\par }{\PP \VS{3}Ἀναλάβετε ὅπλα καὶ ἀσπίδας, καὶ προσαγάγετε εἰς πόλεμον,
\VS{4}καὶ ἐπισάξατε τοὺς ἵππους, ἐπίβητε οἱ ἱππεῖς, καὶ κατάστητε ἐν ταῖς περικεφαλαίαις ὑμῶν, προσβάλετε τὰ δόρατα, καὶ ἐνδύσασθε τοὺς θώρακας ὑμῶν.
\par }{\PP \VS{5}Τί ὅτι αὐτοὶ πτοοῦνται, καὶ ἀποχωροῦσιν εἰς τὸ ὀπίσω; διότι οἱ ἰσχυροὶ αὐτῶν κοπήσονται, φυγῇ ἔφυγον, καὶ οὐκ ἀνέστρεψαν περιεχόμενοι κυκλόθεν, λέγει Κύριος.
\VS{6}Μὴ φευγέτω ὁ κοῦφος, καὶ μὴ ἀνασωζέσθω ὁ ἰσχυρὸς ἐπὶ Βοῤῥᾶν· τὰ παρὰ τὸν Εὐφράτην ἠσθένησε, καὶ πεπτώκασι.
\par }{\PP \VS{7}Τίς οὗτος ὡς ποταμὸς ἀναβήσεται, καὶ ὡς ποταμοὶ κυμαίνουσιν ὕδωρ;
\VS{8}Ὕδατα Αἰγύπτου ὡς ποταμὸς ἀναβήσεται· καὶ εἶπεν, ἀναβήσομαι, καὶ κατακαλύψω τὴν γῆν, καὶ ἀπολῶ τοὺς κατοικοῦντας ἐν αὐτῇ.
\VS{9}Ἐπίβητε ἐπὶ τοὺς ἵππους, παρασκευάσατε τὰ ἅρματα, ἐξέλθατε οἱ μαχηταὶ Αἰθιόπων, καὶ Λίβυες καθωπλισμένοι ὅπλοις, καὶ Λυδοὶ ἀνάβητε, ἐντείνατε τόξον.
\VS{10}Καὶ ἡ ἡμέρα ἐκείνη Κυρίῳ τῷ Θεῷ ἡμῶν ἡμέρα ἐκδικήσεως, τοῦ ἐκδικῆσαι τοὺς ἐχθροὺς αὐτοῦ· καὶ καταφάγεται ἡ μάχαιρα Κυρίου, καὶ ἐμπλησθήσεται, καὶ μεθυσθήσεται ἀπὸ τοῦ αἵματος αὐτῶν, ὅτι θυσία τῷ Κυρίῳ ἀπὸ γῆς Βοῤῥᾶ ἐπὶ ποταμῷ Εὐφράτῃ.
\par }{\PP \VS{11}Ἀνάβηθι Γαλαὰδ, καὶ λάβε ῥητίνην τῇ παρθένῳ θυγατρὶ Αἰγύπτου· εἰς τὸ κενὸν ἐπλήθυνας ἰάματά σου, ὠφέλεια οὐκ ἔστιν ἐν σοί.
\VS{12}Ἤκουσαν ἔθνη φωνήν σου, καὶ τῆς κραυγῆς σου ἐπλήσθη ἡ γῆ, ὅτι μαχητὴς πρὸς μαχητὴν ἠσθένησαν, ἐπιτοαυτὸ ἔπεσαν ἀμφότεροι.
\par }{\PP \VS{13}Ἃ ἘΛΑΛΗΣΕ ΚΥΡΙΟΣ ἐν χειρὶ Ἱερεμίου, τοῦ ἐλθεῖν τὸν βασιλέα Βαβυλῶνος τοῦ κόψαι γῆν Αἰγύπτου.
\par }{\PP \VS{14}Ἀναγγείλατε εἰς Μαγδωλὸν, καὶ παραγγείλατε εἰς Μέμφιν· εἴπατε, ἐπίστηθι, καὶ ἑτοίμασον, ὅτι κατέφαγε μάχαιρα τὴν σμίλακά σου.
\par }{\PP \VS{15}Διατί ἔφυγεν ἀπὸ σοῦ ὁ Ἄπις; ὁ μόσχος ὁ ἐκλεκτός σου οὐκ ἔμεινεν· ὅτι Κύριος παρέλυσεν αὐτόν.
\VS{16}Καὶ τὸ πλῆθός σου ἠσθένησε, καὶ ἔπεσε· καὶ ἕκαστος πρὸς τὸν πλησίον αὐτοῦ ἐλάλει, ἀναστῶμεν, καὶ ἀναστρέψωμεν πρὸς τὸν λαὸν ἡμῶν εἰς τὴν πατρίδα ἡμῶν, ἀπὸ προσώπου μαχαίρας Ἑλληνικῆς.
\VS{17}Καλέσατε τὸ ὄνομα Φαραὼ Νεχαὼ βασιλέως Αἰγύπτου, Σαὼν ἑσβειὲ μωήδ.
\VS{18}Ζῶ ἐγὼ, λέγει Κύριος ὁ Θεὸς, ὅτι ὡς τὸ Ἰταβύριον ἐν τοῖς ὄρεσι, καὶ ὡς ὁ Κάρμηλος ὁ ἐν τῇ θαλάσσῃ, ἥξει.
\VS{19}Σκεύη ἀποικισμοῦ ποίησον σεαυτῇ κατοικοῦσα θύγατερ Αἰγύπτου, ὅτι Μέμφις εἰς ἀφανισμὸν ἔσται, καὶ κληθήσεται Οὐαὶ, διὰ τὸ μὴ ὑπάρχειν κατοικοῦντας ἐν αὐτῇ.
\par }{\PP \VS{20}Δάμαλις κεκαλλωπισμένη Αἴγυπτος, ἀπόσπασμα ἀπὸ Βοῤῥᾶ ἦλθεν ἐπʼ αὐτὴν,
\VS{21}καὶ οἱ μισθωτοὶ αὐτῆς ἐν αὐτῇ, ὥσπερ μόσχοι σιτευτοὶ τρεφόμενοι ἐν αὐτῇ· διότι καὶ αὐτοὶ ἐπεστράφησαν, καὶ ἔφυγον ὁμοθυμαδόν· οὐκ ἔστησαν, ὅτι ἡμέρα ἀπωλείας ἦλθεν ἐπʼ αὐτοὺς, καὶ καιρὸς ἐκδικήσεως αὐτῶν.
\VS{22}Φωνὴ αὐτῶν ὡς ὄφεως συρίζοντος, ὅτι ἐν ἄμμῳ πορεύονται, ἐν ἀξίναις ἥξουσιν ἐπʼ αὐτήν·
\VS{23}ὡς κόπτοντες ξύλα ἐκκόψουσι τὸν δρυμὸν αὐτῆς, λέγει Κύριος, ὅτι οὐ μὴ εἰκασθῇ, ὅτι πληθύνει ὑπὲρ ἀκρίδα, καὶ οὐκ ἔστιν αὐτοῖς ἀριθμός.
\VS{24}Κατῃσχύνθη ἡ θυγάτηρ Αἰγύπτου, παρεδόθη εἰς χεῖρας λαοῦ ἀπὸ Βοῤῥᾶ.
\par }{\PP \VS{25}Ἰδοὺ ἐγὼ ἐκδικῶ τὸν Ἀμμὼν τὸν υἱὸν αὐτῆς ἐπὶ Φαραὼ, καὶ ἐπὶ τοὺς πεποιθότας ἐπʼ αὐτῷ.
\par }{\PP \VS{27}Σὺ δὲ μὴ φοβηθῇς δοῦλός μου Ἰακὼβ, μηδὲ πτοηθῇς Ἰσραήλ· διότι ἐγὼ ἰδοὺ σώζων σε μακρόθεν, καὶ τὸ σπέρμα σου ἐκ τῆς αἰχμαλωσίας αὐτῶν· καὶ ἀναστρέψει Ἰακὼβ, καὶ ἡσυχάσει καὶ ὑπνώσει, καὶ οὐκ ἔσται ὁ παρενοχλῶν αὐτόν.
\VS{28}Μὴ φοβοῦ παῖς μου Ἰακὼβ, λέγει Κύριος, ὅτι μετὰ σοῦ ἐγώ εἰμι· ἡ ἀπτόητος καὶ τρυφερὰ παρεδόθη· ὅτι ποιήσω ἔθνει συντέλειαν ἐν παντὶ ἔθνει εἰς οὓς ἔξωσά σε ἐκεῖ, σὲ δὲ οὐ μὴ ποιήσω ἐκλιπεῖν, καὶ παιδεύσω σε εἰς κρίμα, καὶ ἀθῶον οὐκ ἀθωώσω σε.

\par }\Chap{27}{\PP \VerseOne{1}ΛΟΓΟΣ ΚΥΡΙΟΥ ὋΝ ἘΛΑΛΗΣΕΝ ἘΠΙ ΒΑΒΥΛΩΝΑ.
\par }{\PP \VS{2}Ἀναγγείλατε ἐν τοῖς ἔθνεσι, καὶ ἀκουστὰ ποιήσατε, καὶ μὴ κρύψητε· εἴπατε, ἑάλωκε Βαβυλὼν, κατῃσχύνθη Βῆλος, ἡ ἀπτόητος, ἡ τρυφερὰ παρεδόθη Μαιρωδὰχ,
\VS{3}ὅτι ἀνέβη ἐπʼ αὐτὴν ἔθνος ἀπὸ Βοῤῥᾶ, οὗτος θήσει τὴν γῆν αὐτῆς εἰς ἀφανισμὸν, καὶ οὐκ ἔσται ὁ κατοικῶν ἐν αὐτῇ ἀπὸ ἀνθρώπου καὶ ἕως κτήνους.
\par }{\PP \VS{4}Ἐν ταῖς ἡμέραις ἐκείναις, καὶ ἐν τῷ καιρῷ ἐκείνῳ, ἥξουσιν οἱ υἱοὶ Ἰσραὴλ αὐτοὶ, καὶ οἱ υἱοὶ Ἰούδα ἐπὶ τὸ αὐτὸ, βαδίζοντες καὶ κλαίοντες πορεύσονται, τὸν Κύριον Θεὸν αὐτῶν ζητοῦντες.
\VS{5}Ἕως Σιὼν ἐρωτήσουσι τὴν ὁδὸν, ὧδε γὰρ τὸ πρόσωπον αὐτῶν δώσουσι, καὶ ἥξουσι, καὶ καταφεύξονται πρὸς Κύριον τὸν Θεόν· διαθήκη γὰρ αἰώνιος οὐκ ἐπιλησθήσεται.
\par }{\PP \VS{6}Πρόβατα ἀπολωλότα ἐγενήθη ὁ λαός μου, οἱ ποιμένες αὐτῶν ἔξωσαν αὐτοὺς, ἐπὶ τὰ ὄρη ἀπεπλάνησαν αὐτοὺς, ἐξ ὄρους ἐπὶ βουνὸν ᾤχοντο, ἐπελάθοντο κοίτης αὐτῶν.
\VS{7}Πάντες οἱ εὑρίσκοντες αὐτοὺς, ἀνήλισκον αὐτούς· οἱ ἐχθροὶ αὐτῶν εἶπαν, Μὴ ἀνῶμεν αὐτοὺς, ἀνθʼ ὧν ἥμαρτον τῷ Κυρίῳ· νομὴ δικαιοσύνης τῷ συναγαγόντι τοὺς πατέρας αὐτῶν.
\par }{\PP \VS{8}Ἀπαλλοτριώθητε ἐκ μέσου Βαβυλῶνος καὶ ἀπὸ γῆς Χαλδαίων, καὶ ἐξέλθατε, καὶ γένεσθε ὥσπερ δράκοντες κατὰ πρόσωπον προβάτων.
\VS{9}Ὅτι ἰδοὺ ἐγὼ ἐγείρω ἐπὶ Βαβυλῶνα συναγωγὰς ἐθνῶν ἐκ γῆς Βοῤῥᾶ, καὶ παρατάξονται αὐτῇ· ἐκεῖθεν ἁλώσεται, ὡς βολὶς μαχητοῦ συνετοῦ οὐκ ἐπιστρέψει κενή.
\VS{10}Καὶ ἔσται ἡ Χαλδαία εἰς προνομὴν, πάντες οἱ προνομεύοντες αὐτὴν ἐμπλησθήσονται.
\par }{\PP \VS{11}Ὅτι ηὐφραίνεσθε, καὶ κατεκαυχᾶσθε, διαρπάζοντες τὴν κληρονομίαν μου, διότι ἐσκιρτᾶτε, ὡς βοΐδια ἐν βοτάνῃ, καὶ ἐκερατίζετε ὡς ταῦροι.
\VS{12}Ἠσχύνθη ἡ μήτηρ ὑμῶν σφόδρα, ἐνετράπη ἡ τεκοῦσα ὑμᾶς μήτηρ ἐπʼ ἀγαθὰ, ἐσχάτη ἐθνῶν, ἔρημος
\VS{13}ἀπὸ ὀργῆς Κυρίου, οὐ κατοικηθήσεται καὶ ἔσται εἰς ἀφανισμὸν πᾶσα· καὶ πᾶς ὁ διοδεύων διὰ Βαβυλῶνος σκυθρωπάσει, καὶ συριοῦσιν ἐπὶ πᾶσαν τὴν πληγὴν αὐτῆς.
\par }{\PP \VS{14}Παρατάξασθε ἐπὶ Βαβυλῶνα κύκλῳ πάντες τείνοντες τόξον, τοξεύσατε ἐπʼ αὐτὴν, μὴ φείσησθε ἐπὶ τοῖς τοξεύμασιν ὑμῶν,
\VS{15}καὶ κατακρατήσατε αὐτήν· παρελύθησαν αἱ χεῖρες αὐτῆς, ἔπεσαν αἱ ἐπάλξεις αὐτῆς, καὶ κατεσκάφη τὸ τεῖχος αὐτῆς, ὅτι ἐκδίκησις παρὰ Θεοῦ ἐστιν· ἐκδικεῖτε ἐπʼ αὐτὴν, καθὼς ἐποίησε, ποιήσατε αὐτῇ.
\VS{16}Ἐξολοθρεύσασθε σπέρμα ἐκ Βαβυλῶνος, κατέχοντα δρέπανον ἐν καιρῷ θερισμοῦ· ἀπὸ προσώπου μαχαίρας Ἑλληνικῆς, ἕκαστος εἰς τὸν λαὸν αὐτοῦ ἀποστρέψουσι, καὶ ἕκαστος εἰς τὴν γῆν αὐτοῦ φεύξεται.
\par }{\PP \VS{17}Πρόβατον πλανώμενον Ἰσραὴλ, λέοντες ἔξωσαν αὐτόν· ὁ πρῶτος ἔφαγεν αὐτὸν βασιλεὺς Ἀσσοὺρ, καὶ οὗτος ὕστερον τὰ ὀστᾶ αὐτοῦ βασιλεὺς Βαβυλῶνος.
\VS{18}Διατοῦτο τάδε λέγει Κύριος, ἰδοὺ ἐγὼ ἐκδικῶ ἐπὶ τὸν βασιλέα Βαβυλῶνος, καὶ ἐπὶ τὴν γῆν αὐτοῦ, καθὼς ἐξεδίκησα ἐπὶ τὸν βασιλέα Ἀσσούρ.
\VS{19}Καὶ ἀποκαταστήσω τὸν Ἰσραὴλ εἰς τὴν νομὴν αὐτοῦ, καὶ νεμήσεται ἐν τῷ Καρμήλῳ, καὶ ἐν ὄρει Ἐφραῒμ, καὶ ἐν τῷ Γαλαὰδ, καὶ πλησθήσεται ἡ ψυχὴ αὐτοῦ.
\VS{20}Ἐν ταῖς ἡμέραις ἐκείναις, καὶ ἐν τῷ καιρῷ ἐκείνῳ, ζητήσουσι τὴν ἀδικίαν Ἰσραὴλ, καὶ οὐχ ὑπάρξει, καὶ τὰς ἁμαρτίας Ἰούδα, καὶ οὐ μὴ εὑρεθῶσιν, ὅτι ἵλεως ἔσομαι τοῖς ὑπολελειμμένοις ἐπὶ τῆς γῆς,
\VS{21}λέγει Κύριος.
\par }{\PP Πικρῶς ἐπίβηθι ἐπʼ αὐτὴν, καὶ ἐπὶ τοὺς κατοικοῦντας ἐπʼ αὐτήν· ἐκδίκησον μάχαιρα, καὶ ἀφάνισον, λέγει Κύριος, καὶ ποίει κατὰ πάντα ὅσα ἐντέλλομαί σοι.
\VS{22}Φωνὴ πολέμου καὶ συντριβὴ μεγάλη ἐν γῇ Χαλδαίων.
\VS{23}Πῶς ἐκλάσθη καὶ συνετρίβη ἡ σφυρὰ πάσης τῆς γῆς; πῶς ἐγενήθη εἰς ἀφανισμὸν Βαβυλὼν ἐν ἔθνεσιν;
\VS{24}Ἐπιβήσονταί σοι, καὶ οὐ γνώσῃ, ὡς Βαβυλὼν καὶ ἁλώσῃ· εὑρέθης, καὶ ἐλήφθης, ὅτι τῷ Κυρίῳ ἀντέστης.
\par }{\PP \VS{25}Ἤνοιξε Κύριος τὸν θησαυρὸν αὐτοῦ, καὶ ἐξήνεγκε τὰ σκεύη ὀργῆς αὐτοῦ, ὅτι ἔργον τῷ Κυρίῳ Θεῷ ἐν γῇ Χαλδαίων,
\VS{26}ὅτι ἐληλύθασιν οἱ καιροὶ αὐτῆς· ἀνοίξατε τὰς ἀποθήκας αὐτῆς, ἐρευνήσατε αὐτὴν ὡς σπήλαιον, καὶ ἐξολοθρεύσατε αὐτήν· μὴ γενέσθω αὐτῆς κατάλειμμα·
\VS{27}Ἀναξηράνατε αὐτῆς πάντας τοὺς καρποὺς, καὶ καταβήτωσαν εἰς σφαγήν· οὐαὶ αὐτοῖς, ὅτι ἥκει ἡ ἡμέρα αὐτῶν, καὶ καιρὸς ἐκδικήσεως αὐτῶν.
\VS{28}Φωνὴ φευγόντων, καὶ ἀνασωζομένων ἐκ γῆς Βαβυλῶνος, τοῦ ἀναγγεῖλαι εἰς Σιὼν τὴν ἐκδίκησιν παρὰ Κυρίου Θεοῦ ἡμῶν.
\par }{\PP \VS{29}Παραγγείλατε ἐπὶ Βαβυλῶνα πολλοῖς, παντὶ ἐντείνοντι τόξον, παρεμβάλλετε ἐπʼ αὐτὴν κυκλόθεν· μὴ ἔστω αὐτῆς ἀνασωζόμενος· ἀνταπόδοτε αὐτῇ κατὰ τὰ ἔργα αὐτῆς, κατὰ πάντα ὅσα ἐποίησε, ποιήσατε αὐτῇ, ὅτι πρὸς Κύριον ἀντέστη Θεὸν ἅγιον τοῦ Ἰσραήλ.
\VS{30}Διατοῦτο πεσοῦνται οἱ νεανίσκοι αὐτῆς ἐν ταῖς πλατείαις αὐτῆς, καὶ πάντες οἱ ἄνδρες οἱ πολεμισταὶ αὐτῆς ῥιφήσονται, εἶπε Κύριος.
\par }{\PP \VS{31}Ἰδοὺ ἐγὼ ἐπὶ σὲ τὴν ὑβρίστριαν, λέγει Κύριος· ὅτι ἥκει ἡ ἡμέρα σου, καὶ ὁ καιρὸς ἐκδικήσεως σου.
\VS{32}Καὶ ἀσθενήσει ἡ ὕβρις σου, καὶ πεσεῖται· καὶ οὐδεὶς ἔσται ὁ ἀνιστῶν αὐτήν· καὶ ἀνάψω πῦρ ἐν τῷ δρυμῷ αὐτῆς, καὶ καταφάγεται πάντα τὰ κύκλῳ αὐτῆς.
\par }{\PP \VS{33}Τάδε λέγει Κύριος, καταδεδυνάστευνται οἱ υἱοὶ Ἰσραὴλ, καὶ οἱ υἱοὶ Ἰούδα, ἅμα πάντες οἱ αἰχμαλωτεύσαντες αὐτοὺς, κατεδυνάστευσαν αὐτοὺς, ὅτι οὐκ ἠθέλησαν ἐξαποστεῖλαι αὐτούς.
\VS{34}Καὶ ὁ λυτρούμενος αὐτοὺς ἰσχυρὸς, Κύριος παντοκράτωρ ὄνομα αὐτῷ· κρίσιν κρινεῖ πρὸς τοὺς ἀντιδίκους αὐτοῦ, ὅπως ἐξάρῃ τὴν γῆν· καὶ παροξυνεῖ τοῖς κατοικοῦσι Βαβυλῶνα,
\VS{35}μάχαιραν ἐπὶ τοὺς Χαλδαίους, καὶ ἐπὶ τοὺς κατοικοῦντας Βαβυλῶνα, καὶ ἐπὶ τοὺς μεγιστᾶνας αὐτῆς, καὶ ἐπὶ τοὺς συνετοὺς αὐτῆς·
\VS{36}Μάχαιραν ἐπὶ τοὺς μαχητὰς αὐτῆς, καὶ παραλυθήσονται· μάχαιραν ἐπὶ τοὺς ἵππους αὐτῶν, καὶ ἐπὶ τὰ ἅρματα αὐτῶν·
\VS{37}Μάχαιραν ἐπὶ τοὺς μαχητὰς αὐτῶν, καὶ ἐπὶ τὸν σύμμικτον τὸν ἐν μέσῳ αὐτῆς, καὶ ἔσονται ὡσεὶ γυναῖκες· μάχαιραν ἐπὶ τοὺς θησαυροὺς, καὶ διασκορπισθήσονται
\VS{38}ἐπὶ τῷ ὕδατι αὐτῆς, καὶ καταισχυνθήσονται, ὅτι γῆ τῶν γλυπτῶν ἐστι, καὶ ἐν ταῖς νήσοις, οὗ κατεκαυχῶντο.
\VS{39}Διατοῦτο κατοικήσουσιν ἰνδάλματα ἐν ταῖς νήσοις, καὶ κατοικήσουσιν ἐν αὐτῇ θυγατέρες σειρήνων, οὐ μὴ κατοικηθῇ οὐκέτι εἰς τὸν αἰῶνα.
\VS{40}Καθὼς κατέστρεψεν ὁ Θεὸς Σόδομα καὶ Γόμοῤῥα, καὶ τὰς ὁμορούσας αὐταῖς, εἶπε Κύριος, οὐ μὴ κατοικήσει ἐκεῖ ἄνθρωπος, καὶ οὐ μὴ παροικήσει ἐκεῖ υἱὸς ἀνθρώπου.
\par }{\PP \VS{41}Ἰδοὺ λαὸς ἔρχεται ἀπὸ Βοῤῥᾶ, καὶ ἔθνος μέγα, καὶ βασιλεῖς πολλοὶ ἐξεγερθήσονται ἀπʼ ἐσχάτου τῆς γῆς,
\VS{42}τόξον καὶ ἐγχειρίδιον ἔχοντες, ἰταμός ἐστι, καὶ οὐ μὴ ἐλεήσῃ· ἡ φωνὴ αὐτῶν ὡς θάλασσα ἠχήσει, ἐφʼ ἵπποις ἱππάσονται, παρεσκευασμένοι, ὥσπερ πῦρ, εἰς πόλεμον, πρὸς σὲ, θύγατερ Βαβυλῶνος.
\VS{43}Ἤκουσε βασιλεὺς Βαβυλῶνος τὴν ἀκοὴν αὐτῶν, καὶ παρελύθησαν αἱ χεῖρες αὐτοῦ· θλῖψις κατεκράτησεν αὐτοῦ, ὠδῖνες ὡς τικτούσης.
\VS{44}Ἰδοὺ ὥσπερ λέων ἀναβήσεται ἀπὸ τοῦ Ἰορδάνου εἰς Γαιθὰν, ὅτι ταχέως ἐκδιώξω αὐτοὺς ἀπʼ αὐτῆς, καὶ πάντα νεανίσκον ἐπʼ αὐτὴν ἐπιστήσω· ὅτι τίς ὥσπερ ἐγώ; καὶ τίς ἀντιστήσεταί μοι; καὶ τίς οὗτος ποιμὴν, ὃς στήσεται κατὰ πρόσωπόν μου;
\par }{\PP \VS{45}Διατοῦτο ἀκούσατε τὴν βουλὴν Κυρίου ἣν βεβούλευται ἐπὶ Βαβυλῶνα, καὶ λογισμοὺς αὐτοῦ οὓς ἐλογίσατο ἐπὶ τοὺς κατοικοῦντας Χαλδαίους· ἐὰν μὴ διαφθαρῇ τὰ ἀρνία τῶν προβάτων αὐτῶν, ἐὰν μὴ ἀφανισθῇ νομὴ ἀπʼ αὐτῶν.
\VS{46}Ὅτι ἀπὸ φωνῆς ἁλώσεως Βαβυλῶνος σεισθήσεται ἡ γῆ, καὶ κραυγὴ ἐν ἔθνεσιν ἀκουσθήσεται.

\par }\Chap{28}{\PP \VerseOne{1}Τάδε λέγει Κύριος, ἰδοὺ ἐγὼ ἐξεγείρω ἐπὶ Βαβυλῶνα καὶ ἐπὶ τοὺς κατοικοῦντας Χαλδαίους ἄνεμον καύσωνα διαφθείροντα.
\VS{2}Καὶ ἐξαποστελῶ εἰς Βαβυλῶνα ὑβριστὰς, καὶ καθυβρίσουσιν αὐτὴν, καὶ λυμανοῦνται τὴν γῆν αὐτῆς· οὐαὶ ἐπὶ Βαβυλῶνα κυκλόθεν ἐν ἡμέρᾳ κακώσεως αὐτῆς.
\VS{3}Τεινέτω ὁ τείνων τὸ τόξον αὐτοῦ, καὶ περιθέσθω ᾧ ἐστιν ὅπλα αὐτοῦ, καὶ μὴ φείσησθε ἐπὶ τοὺς νεανίσκους αὐτῆς, καὶ ἀφανίσατε πᾶσαν τὴν δύναμιν αὐτῆς.
\VS{4}Καὶ πεσοῦνται τραυματίαι ἐν γῇ Χαλδαίων, καὶ κατακεκεντημένοι ἔξωθεν αὐτῆς.
\par }{\PP \VS{5}Διότι οὐκ ἐχήρευσεν Ἰσραὴλ καὶ Ἰούδας ἀπὸ Θεοῦ αὐτῶν, ἀπὸ Κυρίου παντοκράτορος, ὅτι ἡ γῆ αὐτῶν ἐπλήσθη ἀδικίας ἀπὸ τῶν ἁγίων Ἰσραήλ.
\VS{6}Φεύγετε ἐκ μέσου Βαβυλῶνος, καὶ ἀνασώζετε ἕκαστος τὴν ψυχὴν αὐτοῦ, καὶ μὴ ἀποῤῥιφῆτε ἐν τῇ ἀδικίᾳ αὐτῆς, ὅτι καιρὸς ἐκδικήσεως αὐτῆς ἐστι παρὰ Κυρίου, ἀνταπόδομα αὐτὸς ἀνταποδίδωσιν αὐτῇ.
\VS{7}Ποτήριον χρυσοῦν Βαβυλὼν ἐν χειρὶ Κυρίου, μεθύσκον πᾶσαν τὴν γῆν, ἀπὸ τοῦ οἴνου αὐτῆς ἐπίοσαν ἔθνη, διατοῦτο ἐσαλεύθησαν.
\VS{8}Καὶ ἄφνω ἔπεσε Βαβυλὼν, καὶ συνετρίβη· θρηνεῖτε αὐτὴν, λάβετε ῥητίνην τῇ διαφθορᾷ αὐτῆς, εἴπως ἰαθήσεται.
\VS{9}Ἰατρεύσαμεν τὴν Βαβυλῶνα, καὶ οὐκ ἰάθη· ἐγκαταλίπωμεν αὐτὴν, καὶ ἀπέλθωμεν ἕκαστος εἰς τὴν γῆν αὐτοῦ, ὅτι ἤγγικεν εἰς οὐρανὸν τὸ κρίμα αὐτῆς, ἐξῇρεν ἕως τῶν ἄστρων.
\VS{10}Ἐξήνεγκε Κύριος τὸ κρίμα αὐτοῦ· δεῦτε, καὶ ἀναγγείλωμεν ἐν Σιὼν τὰ ἔργα Κυρίου τοῦ Θεοῦ ἡμῶν.
\par }{\PP \VS{11}Παρασκευάζετε τὰ τοξεύματα, πληροῦτε τὰς φαρέτρας· ἤγειρε Κύριος τὸ πνεῦμα βασιλέως Μήδων, ὅτι εἰς Βαβυλῶνα ἡ ὀργὴ αὐτοῦ, τοῦ ἐξολοθρεῦσαι αὐτὴν, ὅτι ἐκδίκησις Κυρίου ἐστὶν, ἐκδίκησις λαοῦ αὐτοῦ ἐστιν.
\VS{12}Ἐπὶ τειχέων Βαβυλῶνος ἄρατε σημεῖον, ἐπιστήσατε φαρέτρας, ἐγείρατε φυλακὰς, ἑτοιμάσατε ὅπλα, ὅτι ἐνεχείρισε, καὶ ποιήσει Κύριος ἃ ἐλάλησεν ἐπὶ τοὺς κατοικοῦντας Βαβυλῶνα,
\VS{13}κατασκηνοῦντας ἐφʼ ὕδασι πολλοῖς, καὶ ἐπὶ πλήθει θησαυρῶν αὐτῆς· ἥκει τὸ πέρας σου ἀληθῶς εἰς τὰ σπλάγχνα σου.
\VS{14}Ὅτι ὤμοσε Κύριος κατὰ τοῦ βραχίονος αὐτοῦ, διότι πληρώσω σε ἀνθρώπων ὡσεὶ ἀκρίδων, καὶ φθέγξονται ἐπὶ σὲ οἱ καταβαίνοντες.
\par }{\PP \VS{15}Κύριος ποιῶν γῆν ἐν τῇ ἰσχύϊ αὐτοῦ, ἑτοιμάζων οἰκουμένην ἐν τῇ σοφίᾳ αὐτοῦ, ἐν τῇ συνέσει αὐτοῦ ἐξέτεινε τὸν ουρανὸν,
\VS{16}εἰς φωνὴν ἔθετο ἦχος ὕδατος ἐν οὐρανῷ, καὶ ἀνήγαγε νεφέλας ἀπʼ ἐσχάτου τῆς γῆς· ἀστραπὰς εἰς ὑετὸν ἐποίησε, καὶ ἐξήγαγε φῶς ἐκ τῶν θησαυρῶν αὐτοῦ.
\VS{17}Ἐματαιώθη πᾶς ἄνθρωπος ἀπὸ γνώσεως, κατῃσχύνθη πᾶς χρυσοχόος ἀπὸ τῶν γλυπτῶν αὐτοῦ, ὅτι ψευδῆ ἐχώνευσαν, οὐκ ἔστι πνεῦμα ἐν αὐτοῖς.
\VS{18}Μάταιά ἐστιν ἔργα μεμωκημένα, ἐν καιρῷ ἐπισκέψεως αὐτῶν ἀπολοῦνται.
\VS{19}Οὐ τοιαύτη μερὶς τῷ Ἰακὼβ, ὅτι ὁ πλάσας τὰ πάντα, αὐτός ἐστι κληρονομία αὐτοῦ, Κύριος ὄνομα αὐτῷ.
\par }{\PP \VS{20}Διασκορπίζεις σύ μοι σκεύη πολέμου, καὶ διασκορπιῶ ἐν σοὶ ἔθνη, καὶ ἐξαρῶ ἐκ σοῦ βασιλεῖς.
\VS{21}Καὶ διασκορπιῶ ἐν σοὶ ἵππον καὶ ἐπιβάτην αὐτοῦ, καὶ διασκορπιῶ ἐν σοὶ ἅρματα καὶ ἀναβάτας αὐτῶν.
\VS{22}Καὶ διασκορπιῶ ἐν σοὶ νεανίσκον καὶ παρθένον, καὶ διασκορπιῶ ἐν σοὶ ἄνδρα καὶ γυναῖκα.
\VS{23}Καὶ διασκορπιῶ ἐν σοὶ ποιμένα καὶ τὸ ποίμνιον αὐτοῦ, καὶ διασκορπιῶ ἐν σοὶ γεωργὸν καὶ τὸ γεώργιον αὐτοῦ, καὶ διασκορπιῶ ἐν σοὶ ἡγεμόνας καὶ στρατηγούς σου.
\VS{24}Καὶ ἀνταποδώσω τῇ Βαβυλῶνι καὶ πᾶσι τοῖς κατοικοῦσι Χαλδαίοις πάσας τὰς κακίας αὐτῶν, ἃς ἐποίησαν ἐπὶ Σιὼν κατʼ ὀφθαλμοὺς ὑμῶν, λέγει Κύριος.
\par }{\PP \VS{25}Ἰδοὺ ἐγὼ πρὸς σὲ τὸ ὄρος τὸ διεφθαρμένον, τὸ διαφθεῖρον πᾶσαν τὴν γῆν, καὶ ἐκτενῶ τὴν χεῖρά μου ἐπὶ σὲ, καὶ κατακυλιῶ σε ἐπὶ τῶν πετρῶν, καὶ δώσω σε ὡς ὄρος ἐμπεπυρισμένον·
\VS{26}Καὶ οὐ μὴ λάβωσιν ἀπὸ σοῦ λίθον εἰς γωνίαν, καὶ λίθον εἰς θεμέλιον, ὅτι εἰς ἀφανισμὸν ἔσῃ εἰς τὸν αἰῶνα, λέγει Κύριος.
\par }{\PP \VS{27}Ἄρατε σημεῖον ἐπὶ τῆς γῆς, σαλπίσατε ἐν ἔθνεσι σάλπιγγι, ἁγιάσατε ἐπʼ αὐτὴν ἔθνη, παραγγείλατε ἐπʼ αὐτὴν, βασιλεῖς ἄρατε παρʼ ἐμοῦ, καὶ τοῖς Ἀχαναζέοις· ἐπιστήσατε ἐπʼ αὐτὴν βελοστάσεις, ἀναβιβάσατε ἐπʼ αὐτὴν ἵππον ὡς ἀκρίδων πλῆθος.
\VS{28}Ἀναβιβάσατε ἐπʼ αὐτὴν ἔθνη, τὸν βασιλέα τῶν Μήδων καὶ πάσης τῆς γῆς, τοὺς ἡγουμένους αὐτοῦ, καὶ πάντας τοὺς στρατηγοὺς αὐτοῦ.
\VS{29}Ἐσείσθη ἡ γῆ, καὶ ἐπόνεσε, διότα ἐξανέστη ἐπὶ Βαβυλῶνα λογισμὸς Κυρίου, τοῦ θεῖναι τὴν γῆν Βαβυλῶνος εἰς ἀφανισμὸν, καὶ μὴ κατοικεῖσθαι αὐτήν.
\par }{\PP \VS{30}Ἐξέλιπε μαχητὴς Βαβυλῶνος τοῦ πολεμεῖν, καθήσονται ἐκεῖ ἐν περιοχῇ, ἐθραύσθη ἡ δυναστεία αὐτῶν, ἐγενήθησαν ὡσεὶ γυναῖκες· ἐνεπυρίσθη τὰ σκηνώματα αὐτῆς, συνετρίβησαν οἱ μοχλοὶ αὐτῆς.
\VS{31}Διώκων εἰς ἀπάντησιν διώκοντος διώξεται, καὶ ἀναγγέλλων εἰς ἀπάντησιν ἀναγγέλλοντος, τοῦ ἀναγγεῖλαι τῷ βασιλεῖ Βαβυλῶνος, ὅτι ἑάλωκεν ἡ πόλις αὐτοῦ.
\VS{32}Ἀπʼ ἐσχάτου τῶν διαβάσεων αὐτοῦ ἐλήφθησαν, καὶ τὰ συστήματα αὐτῶν ἐνέπρησαν ἐν πυρὶ, καὶ ἄνδρες αὐτοῦ οἱ πολεμισταὶ ἐξέρχονται.
\par }{\PP \VS{33}Διότι τάδε λέγει Κύριος, οἶκοι βασιλέως Βαβυλῶνος, ὡς ἅλων ὥριμος ἀλοηθήσονται· ἔτι μικρὸν, καὶ ἥξει ὁ ἀμητὸς αὐτῆς.
\par }{\PP \VS{34}Κατέφαγέ με, ἐμερίσατό με, κατέλαβέ με σκότος λεπτὸν, Ναβουχοδονόσορ βασιλεὺς Βαβυλῶνος κατέπιέ με, ὡς δράκων ἔπλησε τὴν κοιλίαν αὐτοῦ ἀπὸ τῆς τρυφῆς μου·
\VS{35}Ἔξωσάν με οἱ μόχθοι μου καὶ αἱ ταλαιπωρίαι μου εἰς Βαβυλῶνα, ἐρεῖ κατοικοῦσα Σιὼν, καὶ τὸ αἷμά μου ἐπὶ τοὺς κατοικοῦντας Χαλδαίους, ἐρεῖ Ἱερουσαλήμ.
\par }{\PP \VS{36}Διατοῦτο τάδε λέγει Κύριος, ἰδοὺ ἐγὼ κρινῶ τὴν ἀντίδικόν σου, καὶ ἐκδικήσω τὴν ἐκδίκησίν σου, καὶ ἐρημώσω τὴν θάλασσαν αὐτῆς, καὶ ξηρανῶ τὴν πηγὴν αὐτῆς.
\VS{37}Καὶ ἔσται Βαβυλὼν εἰς ἀφανισμὸν, καὶ οὐ κατοικηθήσεται.
\VS{38}Ὅτι ἅμα ὡς λέοντες ἐξηγέρθησαν, καὶ ὡς σκύμνοι λεόντων.
\VS{39}Ἐν τῇ θερμασίᾳ αὐτῶν δώσω πότημα αὐτοῖς, καὶ μεθύσω αὐτοὺς, ὅπως καρωθῶσι, καὶ ὑπνώσωσιν ὕπνον αἰώνιον, καὶ οὐ μὴ ἐξεγερθῶσι, λέγει Κύριος.
\VS{40}Καὶ καταβίβασον αὐτοὺς ὡς ἄρνας εἰς σφαγὴν, καὶ ὡς κριοὺς μετʼ ἐρίφων.
\par }{\PP \VS{41}Πῶς ἑάλω καὶ ἐθηρεύθη τὸ καύχημα πάσης τῆς γῆς; πῶς ἐγένετο Βαβυλὼν εἰς ἀφανισμὸν ἐν τοῖς ἔθνεσιν;
\VS{42}Ἀνέβη ἐπὶ Βαβυλῶνα ἡ θάλασσα ἐν ἤχῳ κυμάτων αὐτῆς, καὶ κατεκαλύφθη,
\VS{43}ἐγενήθησαν αἱ πόλεις αὐτῆς ὡς γῆ ἄνυδρος καὶ ἄβατος, οὐ κατοικήσει ἐν αὐτῇ οὐδὲ εἷς, οὐδὲ μὴ καταλύσει ἐν αὐτῇ υἱὸς ἀνθρώπου.
\VS{44}Καὶ ἐκδικήσω ἐπὶ Βαβυλῶνα, καὶ ἐξοίσω ἃ κατέπιεν ἐκ τοῦ στόματος αὐτῆς, καὶ οὐ μὴ συναχθῶσι πρὸς αὐτὴν ἔτι τὰ ἔθνη,
\VS{49}καὶ ἐν Βαβυλῶνι πεσοῦνται τραυματίαι πάσης τῆς γῆς.
\VS{50}Ἀνασωζόμενοι ἐκ γῆς πορεύεσθε, καὶ μὴ ἵστασθε· οἱ μακρόθεν μνήσθητε τοῦ Κυρίου, καὶ Ἱερουσαλὴμ ἀναβήτω ἐπὶ τὴν καρδίαν ὑμῶν.
\VS{51}Ἠσχύνθημεν ὅτι ἠκούσαμεν ὀνειδισμὸν ἡμῶν, κατεκάλυψεν ἀτιμία τὸ πρόσωπον ἡμῶν, εἰσῆλθον ἀλλογενεῖς εἰς τὰ ἅγια ἡμῶν, εἰς οἶκον Κυρίου.
\par }{\PP \VS{52}Διατοῦτο ἰδοὺ ἡμέραι ἔρχονται, λέγει Κύριος, καὶ ἐκδικήσω ἐπὶ τὰ γλυπτὰ αὐτῆς, καὶ ἐν πάσῃ τῇ γῇ αὐτῆς πεσοῦνται τραυματίαι.
\VS{53}Ὅτι ἐὰν ἀναβῇ Βαβυλὼν ὡς ὁ οὐρανὸς, καὶ ὅτι ἐὰν ὀχυρώσῃ τὰ τείχη ἰσχύϊ αὐτῆς, παρʼ ἐμοῦ ἥξουσιν ἐξολοθρεύοντες αὐτὴν, λέγει Κύριος.
\VS{54}Φωνὴ κραυγῆς ἐν Βαβυλῶνι, καὶ συντριβὴ μεγάλη ἐν γῇ Χαλδαίων,
\VS{55}ὅτι ἐξωλόθρευσε Κύριος τὴν Βαβυλῶνα, καὶ ἀπώλεσεν ἀπʼ αὐτῆς φωνὴν μεγάλην ἠχοῦσαν ὡς ὕδατα πολλά· ἔδωκεν εἰς ὄλεθρον φωνὴν αὐτῆς.
\VS{56}Ὅτι ἦλθεν ἐπὶ Βαβυλῶνα ταλαιπωρία, ἑάλωσαν οἱ μαχηταὶ αὐτῆς, ἐπτόηται τὸ τόξον αὐτῶν, ὅτι ὁ Θεὸς ἀνταποδίδωσιν αὐτοῖς.
\VS{57}Κύριος ἀνταποδίδωσι, καὶ μεθύσει μέθῃ τοὺς ἡγεμόνας αὐτῆς, καὶ τοὺς σοφοὺς αὐτῆς, καὶ τοὺς στρατηγοὺς αὐτῆς, λέγει ὁ βασιλεὺς, Κύριος παντοκράτωρ ὄνομα αὐτῷ.
\par }{\PP \VS{58}Τάδε λέγει Κύριος, τεῖχος Βαβυλῶνος ἐπλατύνθη, κατασκαπτόμενον κατασκαφήσεται, καὶ αἱ πύλαι αὐτῆς αἱ ὑψηλαὶ ἐμπυρισθήσονται, καὶ οὐ κοπιάσουσι λαοὶ εἰς κενὸν, καὶ ἔθνη ἐν ἀρχῇ ἐκλείψουσιν.
\par }{\PP \VS{59}Ὁ ΛΟΓΟΣ ὋΝ ἘΝΕΤΕΙΛΑΤΟ ΚΥΡΙΟΣ ἹΕΡΕΜΙΑ ΤΩ ΠΡΟΦΗΤΗ εἰπεῖν τῷ Σαραίᾳ υἱῷ Νηρείου, υἱοῦ Μαασαίου, ὅτε ἐπορεύετο παρὰ Σεδεκίου βασιλέως Ἰούδα εἰς Βαβυλῶνα, ἐν τῷ ἔτει τῷ τετάρτῳ τῆς βασιλείας αὐτοῦ· καὶ Σαραίας ἄρχων δώρων.
\VS{60}Καὶ ἔγραψεν Ἱερεμίας πάντα τὰ κακὰ ἃ ἥξει ἐπὶ Βαβυλῶνα ἐν βιβλίῳ, πάντας τοὺς λόγους τούτους τοὺς γεγραμμένους ἐπὶ Βαβυλῶνα.
\VS{61}Καὶ εἶπεν Ἱερεμίας πρὸς Σαραίαν, ὅταν ἔλθῃς εἰς Βαβυλῶνα, καὶ ὄψῃ καὶ ἀναγνώσῃ πάντας τοὺς λόγους τούτους,
\VS{62}καὶ ἐρεῖς, Κύριε Κύριε, σὺ ἐλάλησας ἐπὶ τὸν τόπον τοῦτον, τοῦ ἐξολοθρεῦσαι αὐτὸν, καὶ τοῦ μὴ εἶναι ἐν αὐτῷ κατοικοῦντας ἀπὸ ἀνθρώπου ἕως κτήνους, ὅτι ἀφανισμὸς εἰς τὸν αἰῶνα ἔσται.
\VS{63}Καὶ ἔσται ὅταν παύσῃ τοῦ ἀναγινώσκειν τὸ βιβλίον τοῦτο, καὶ ἐπιδήσεις ἐπʼ αὐτὸ λίθον, καὶ ῥίψεις αὐτὸ εἰς μέσον τοῦ Εὐφράτου,
\VS{64}καὶ ἐρεῖς, οὕτως καταδύσεται Βαβυλὼν, καὶ οὐ μὴ ἀναστῇ ἀπὸ προσώπου τῶν κακῶν, ὧν ἐγὼ ἐπάγω ἐπʼ αὐτήν.

\par }\Chap{29}{\PP \VerseOne{1}ἘΠΙ ΤΟΥΣ ἈΛΛΟΦΥΛΟΥΣ ΤΑΔΕ ΛΕΓΕΙ ΚΥΡΙΟΣ.
\par }{\PP \VS{2}Ἰδοὺ ὕδατα ἀναβαίνει ἀπὸ Βοῤῥᾶ, καὶ ἔσται εἰς χειμάῤῥουν κατακλύζοντα, καὶ κατακλύσει γῆν καὶ τὸ πλήρωμα αὐτῆς, πόλιν καὶ τοὺς κατοικοῦντας ἐν αὐτῇ· καὶ κεκράξονται οἱ ἄνθρωποι, καὶ ἀλαλάξουσιν ἅπαντες οἱ κατοικοῦντες τὴν γῆν,
\VS{3}ἀπὸ φωνῆς ὁρμῆς αὐτοῦ, ἀπὸ τῶν ὅπλων τῶν ποδῶν αὐτοῦ, καὶ ἀπὸ σεισμοῦ τῶν ἁρμάτων αὐτοῦ, ἤχου τροχῶν αὐτοῦ· οὐκ ἐπέστρεψαν πατέρες ἐφʼ υἱοὺς αὐτῶν ἀπὸ ἐκλύσεως χειρῶν αὐτῶν
\VS{4}ἐν τῇ ἡμέρᾳ τῇ ἐπερχομένῃ τοῦ ἀπολέσαι πάντας τοὺς ἀλλοφύλους· καὶ ἀφανιῶ τὴν Τύρον, καὶ τὴν Σιδῶνα, καὶ πάντας τοὺς καταλοίπους τῆς βοηθείας αὐτῶν, ὅτι ἐξολοθρεύσει Κύριος τοὺς καταλοίπους τῶν νήσων.
\VS{5}Ἥκει φαλάκρωμα ἐπὶ Γάζαν, ἀπεῤῥίφη Ἀσκάλων, καὶ οἱ κατάλοιποι Ἐνακίμ.
\par }{\PP \VS{6}Ἕως τίνος κόψεις ἡ μάχαιρα τοῦ Κυρίου; ἕως τίνος οὐχ ἡσυχάσεις; ἀποκατάστηθι εἰς τὸν κολεόν σου, ἀνάπαυσαι, καὶ ἐπάρθητι.
\par }{\PP \VS{7}Πῶς ἡσυχάσει, καὶ Κύριος ἐνετείλατο αὐτῇ ἐπὶ τὴν Ἀσκάλωνα, καὶ ἐπὶ τὰς παραθαλασσίους, ἐπὶ τὰς καταλοίπους ἐπεγερθῆναι;

\par }\Chap{30}{\PP \VerseOne{1}ΤΗ ἸΔΟΥΜΑΙΑ, τάδε λέγει Κύριος, οὐκ ἔστιν ἔτι σοφία ἐν Θαιμὰν, ἀπώλετο βουλὴ ἐκ συνετῶν, ᾤχετο σοφία αὐτῶν, ἠπατήθη ὁ τόπος αὐτῶν·
\VS{2}βαθύνατε εἰς κάθισιν οἱ κατοικοῦντες ἐν Δαιδὰμ, ὅτι δύσκολα ἐποίησεν· ἤγαγον ἐπʼ αὐτὸν ἐν χρόνῳ ᾧ ἐπεσκεψάμην ἐπʼ αὐτόν.
\VS{3}Ὅτι τρυγηταὶ ἦλθον, οἳ οὐ καταλείψουσί σοι κατάλειμμα· ὡς κλέπται ἐν νυκτὶ, ἑπιθήσουσι χεῖρα αὐτῶν.
\par }{\PP \VS{4}Ὅτι ἐγὼ κατέσυρα τὸν Ἡσαῦ, ἀνεκάλυψα τὰ κρυπτὰ αὐτῶν, κρυβῆναι οὐ μὴ δύνωνται, ὤλοντο διὰ χεῖρα ἀδελφοῦ αὐτοῦ, γείτονός μου, καὶ οὐκ ἔστιν
\VS{5}ὑπολείπεσθαι ὀρφανόν σου, ἵνα ζήσηται· καὶ ἐγὼ ζήσομαι, καὶ αἱ χῆραι ἐπʼ ἐμὲ πεποίθασιν·
\par }{\PP \VS{6}Ὅτι τάδε εἶπε Κύριος, οἷς οὐκ ἦν νόμος πιεῖν τὸ ποτήριον, ἔπιον· καὶ σὺ ἀθωωμένη οὐ μὴ ἀθωωθῇς,
\VS{7}ὅτι κατʼ ἐμαυτοῦ ὤμοσα, λέγει Κύριος, ὅτι εἰς ἄβατον καὶ εἰς ὀνειδισμὸν, καὶ εἰς κατάρασιν ἔσῃ ἐν μέσῳ αὐτῆς, καὶ πᾶσαι αἱ πόλεις αὐτῆς ἔσονται ἔρημοι εἰς αἰῶνα.
\par }{\PP \VS{8}Ἀκοὴν ἤκουσα παρὰ Κυρίου, καὶ ἀγγέλους εἰς ἔθνη ἀπέστειλε, συνάχθητε, καὶ παραγένεσθε εἰς αὐτὴν, ἀνάστητε εἰς πόλεμον.
\VS{9}Μικρὸν ἔδωκά σε ἐν ἔθνεσιν, εὐκαταφρόνητον ἐν ἀνθρώποις.
\VS{10}Ἡ παιγνία σου ἐνεχείρησέ σοι, ἰταμία καρδίας σου κατέλυσε τρυμαλιὰς πετρῶν, συνέλαβεν ἰσχὺν βουνοῦ ὑψηλοῦ· ὅτι ὕψωσεν ὥσπερ ἀετὸς νοσσιὰν αὐτοῦ, ἐκεῖθεν καθελῶ σε.
\par }{\PP \VS{11}Καὶ ἔσται ἡ Ἰδουμαία εἰς ἄβατον, πᾶς ὁ παραπορευόμενος ἐπʼ αὐτὴν συριεῖ.
\VS{12}Ὥσπερ κατεστράφη Σόδομα καὶ Γόμοῤῥα, καὶ αἱ πάροικοι αὐτῆς, εἶπε Κύριος παντοκράτωρ, οὐ μὴ καθίσει ἐκεῖ ἄνθρωπος, καὶ οὐ μὴ κατοικήσει ἐκεῖ υἱὸς ἀνθρώπου.
\VS{13}Ἰδοὺ ὥσπερ λέων ἀναβήσεται ἐκ μέσου τοῦ Ἰορδάνου εἰς τόπον Αἰθὰμ, ὅτι ταχὺ ἐκδιώξω αὐτοὺς ἀπʼ αὐτῆς, καὶ τοὺς νεανίσκους ἐπʼ αὐτὴν ἐπιστήσατε· ὅτι τίς ὥσπερ ἐγώ; καὶ τίς ἀντιστήσεταί μοι; καὶ τίς οὗτος ποιμὴν, ὃς στήσεται κατὰ πρόσωπόν μου;
\par }{\PP \VS{14}Διατοῦτο ἀκούσατε βουλὴν Κυρίου, ἣν ἐβουλεύσατο ἐπὶ τὴν Ἰδουμαίαν, καὶ λογισμὸν αὐτοῦ, ὃν ἐλογίσατο ἐπὶ τοὺς κατοικοῦντας Θαιμὰν, ἐὰν μὴ συμψησθῶσι τὰ ἐλάχιστα τῶν προβάτων, ἐὰν μὴ ἀβατωθῇ ἐπʼ αὐτοὺς κατάλυσις αὐτῶν,
\VS{15}ὅτι ἀπὸ φωνῆς πτώσεως αὐτῶν ἐφοβήθη ἡ γῆ, καὶ κραυγὴ θαλάσσης οὐκ ἠκούσθη.
\VS{16}Ἰδοὺ ὥσπερ ἀετὸς ὄψεται, καὶ ἐκτενεῖ τὰς πτέρυγας ἐπʼ ὀχυρώματα αὐτῆς· καὶ ἔσται ἡ καρδία τῶν ἰσχυρῶν τῆς Ἰδουμαίας ἐν τῇ ἡμέρᾳ ἐκείνῃ, ὡς καρδία γυναικὸς ὠδινούσης.
\par }{\PP \VS{17}ΤΟΙΣ ΥΙΟΙΣ ἈΜΜΩΝ οὕτως εἶπε Κύριος, μὴ υἱοὶ οὐκ εἰσὶν ἐν Ἰσραὴλ, ἢ παραληψόμενος οὐκ ἔστιν αὐτοῖς; διατί παρέλαβε Μελχὸλ τὴν Γαλαὰδ, καὶ ὁ λαὸς αὐτῶν ἐν πόλεσιν αὐτῶν ἐνοικήσει;
\VS{18}Διατοῦτο ἰδοὺ ἡμέραι ἔρχονται, φησὶ Κύριος, καὶ ἀκουτιῶ ἐπὶ Ῥαββὰθ θόρυβον πολέμων, καὶ ἔσονται εἰς ἄβατον καὶ εἰς ἀπώλειαν, καὶ βωμοὶ αὐτῆς ἐν πυρὶ κατακαυθήσονται, καὶ παραλήψεται Ἰσραὴλ τὴν ἀρχὴν αὐτοῦ.
\VS{19}Ἀλάλαξον Ἐσεβὼν, ὅτι ὤλετο Γαΐ· κεκράξατε θυγατέρες Ῥαββὰθ, περιζώσασθε σάκκους καὶ κόψασθε, ὅτι Μελχὸλ βαδιεῖται ἐν ἀποικίᾳ, οἱ ἱερεῖς αὐτοῦ καὶ οἱ ἄρχοντες αὐτοῦ ἅμα.
\par }{\PP \VS{20}Τί ἀγαλλιᾶσθε ἐν τοῖς πεδίοις Ἐνακεὶμ, θύγατερ ἰταμίας, ἡ πεποιθυῖα ἐπὶ θησαυροῖς, ἡ λέγουσα, τίς εἰσελεύσεται ἐπʼ ἐμέ;
\VS{21}Ἰδοὺ ἐγὼ φέρω φόβον ἐπὶ σὲ, εἶπε Κύριος, ἀπὸ πάσης τῆς περιοίκου σου, καὶ διασπαρήσεσθε ἕκαστος εἰς τὸ πρόσωπον αὐτοῦ, καὶ οὐκ ἔστιν ὁ συνάγων.
\par }{\PP \VS{23}ΤΗ ΚΗΔΑΡ ΤΗ ΒΑΣΙΛΙΣΣΗ ΤΗΣ ΑΥΛΗΣ, ἫΝ ἘΠΑΤΑΞΕ ΝΑΒΟΥΧΟΔΟΝΟΣΟΡ ΒΑΣΙΛΕΥΣ ΒΑΒΥΛΩΝΟΣ, οὕτως εἶπε Κύριος,
\par }{\PP Ἀνάστητε, καὶ ἀνάβητε ἐπὶ Κηδὰρ, καὶ πλήσατε τοὺς υἱοὺς Κεδέμ.
\VS{24}Σκηνὰς αὐτῶν, καὶ τὰ πρόβατα αὐτῶν λήψονται· ἱμάτια αὐτῶν, καὶ πάντα τὰ σκεύη αὐτῶν, καὶ καμήλους αὐτῶν λήψονται ἑαυτοῖς· καὶ καλέσατε ἐπʼ αὐτοὺς ἀπώλειαν κυκλόθεν.
\VS{25}Φεύγετε, λίαν ἐμβαθύνατε εἰς κάθισιν, καθήμενοι ἐν τῇ αὐλῇ, ὅτι ἐβουλεύσατο ἐφʼ ὑμᾶς βασιλεὺς Βαβυλῶνος βουλὴν, καὶ ἐλογίσατο λογισμόν.
\par }{\PP \VS{26}Ἀνάστηθι, καὶ ἀνάβηθι ἐπʼ ἔθνος εὐσταθοῦν, καθήμενον εἰς ἀναψυχὴν, οἷς οὐκ εἰσὶ θύραι, οὐ βάλανοι, οὐ μοχλοὶ, μόνοι καταλύουσι.
\VS{27}Καὶ ἔσονται κάμηλοι αὐτῶν εἰς προνομὴν, καὶ πλῆθος κτηνῶν αὐτῶν εἰς ἀπώλειαν, καὶ λικμήσω αὐτοὺς παντὶ πνεύματι κεκαρμένους πρὸ προσώπου αὐτῶν, ἐκ παντὸς πέραν αὐτῶν οἴσω τὴν τροπὴν αὐτῶν, εἶπε Κύριος.
\VS{28}Καὶ ἔσται ἡ αὐλὴ διατριβὴ στρουθῶν, καὶ ἄβατος ἕως αἰῶνος, οὐ μὴ καθίσῃ ἐκεῖ ἄνθρωπος, καὶ οὐ μὴ κατοικήσει ἐκεῖ υἱὸς ἀνθρώπου.
\par }{\PP \VS{29}ΤΗ ΔΑΜΑΣΚΩ. Κατῃσχύνθη Ἠμὰθ, καὶ Ἀρφὰθ, ὅτι ἤκουσαν ἀκοὴν πονηρὰν, ἐξέστησαν, ἐθυμώθησαν, ἀναπαύσασθαι οὐ μὴ δύνωνται.
\VS{30}Ἐξελύθη Δαμασκὸς, ἀπεστράφη εἰς φυγὴν, τρόμος ἐπελάβετο αὐτῆς.
\VS{31}Πῶς οὐχὶ ἐγκατέλιπε πόλιν ἐμὴν, κώμην ἠγάπησαν;
\par }{\PP \VS{32}Διατοῦτο πεσοῦνται νεανίσκοι ἐν πλατείαις σου, καὶ πάντες οἱ ἄνδρες οἱ πολεμισταί σου πεσοῦνται, φησὶ Κύριος·
\VS{33}Καὶ καύσω πῦρ ἐν τείχει Δαμασκοῦ, καὶ καταφάγεται ἄμφοδα υἱοῦ Ἄδερ.

\par }\Chap{31}{\PP \VerseOne{1}ΤΗ ΜΩΑΒ οὕτως εἶπε Κύριος, οὐαὶ ἐπὶ Ναβαῦ, ὅτι ὤλετο, ἐλήφθη Καριαθαὶμ, ᾐσχύνθη Ἀμὰθ καὶ Ἀγάθ.
\VS{2}Οὐκ ἔστιν ἔτι ἰατρεία Μωὰβ, γαυρίαμα ἐν Ἐσεβὼν, ἐλογίσατο ἐπʼ αὐτὴν κακά· ἐκόψαμεν αὐτὴν ἀπὸ ἔθνους, καὶ παῦσιν παύσεται· ὄπισθέν σου βαδιεῖται μάχαιρα,
\VS{3}ὅτι φωνὴ κεκραγότων ἐξ Ὠρωναὶμ, ὄλεθρον καὶ σύντριμμα μέγα·
\VS{4}Συνετρίβη Μωὰβ, ἀναγγείλατε εἰς Ζογόρα,
\VS{5}ὅτι ἐπλήσθη Ἀλὼθ ἐν κλαυθμῷ· ἀναβήσεται κλαίων ἐν ὁδῷ Ὠρωναὶμ, κραυγὴν συντρίμματος ἠκούσατε.
\par }{\PP \VS{6}Φεύγετε καὶ σώσατε τὰς ψυχὰς ὑμῶν, καὶ ἔσεσθε ὥσπερ ὄνος ἄγριος ἐν ἐρήμῳ.
\VS{7}Ἐπειδὴ ἐπεποίθεις ἐν ὀχυρώματί σου, καὶ σὺ συλληφθήσῃ· καὶ ἐξελεύσεται Χαμὼς ἐν ἀποικίᾳ, καὶ οἱ ἱερεῖς αὐτοῦ, καὶ οἱ ἄρχοντες αὐτοῦ ἅμα.
\VS{8}Καὶ ἥξει ὄλεθρος ἐπὶ πᾶσαν πόλιν, οὐ μὴ σωθῇ, καὶ ἀπολεῖται ὁ αὐλῶν, καὶ ἐξολοθρευθήσεται ἡ πεδινὴ, καθὼς εἶπε Κύριος.
\VS{9}Δότε σημεῖα τῇ Μωὰβ, ὅτι ἁφῇ ἁφθήσεται, καὶ πᾶσαι αἱ πόλεις αὐτῆς εἰς ἄβατον ἔσονται· πόθεν ἔνοικος αὐτῇ;
\VS{10}Ἐπικατάρατος ὁ ποιῶν τὰ ἔργα Κυρίου ἀμελῶς, ἐξαίρων μάχαιραν αὐτοῦ ἀφʼ αἵματος.
\par }{\PP \VS{11}Ἀνεπαύσατο Μωὰβ ἐκ παιδαρίου, καὶ πεποιθὼς ἦν ἐπὶ τῇ δόξῃ αὐτοῦ, οὐκ ἐνέχεεν ἐξ ἀγγείου εἰς ἀγγεῖον, καὶ εἰς ἀποικισμὸν οὐκ ᾤχετο· διατοῦτο ἔστη γεῦμα αὐτοῦ ἐν αὐτῷ, καὶ ὀσμὴ αὐτοῦ οὐκ ἐξέλιπε.
\VS{12}Διατοῦτο ἰδοὺ ἡμέραι αὐτοῦ ἔρχονται, φησὶ Κύριος, καὶ ἀποστελῶ αὐτῷ κλίνοντας, καὶ κλινοῦσιν αὐτὸν, καὶ τὰ σκεύη αὐτοῦ λεπτυνοῦσι, καὶ τὰ κέρατα αὐτοῦ συγκόψουσι.
\VS{13}Καὶ καταισχυνθήσεται Μωὰβ ἀπὸ Χαμὼς, ὥσπερ κατῃσχύνθη οἶκος Ἰσραὴλ ἀπὸ Βαιθὴλ ἐλπίδος αὐτῶν πεποιθότες ἐπʼ αὐτοῖς.
\par }{\PP \VS{14}Πῶς ἐρεῖτε, ἰσχυροί ἐσμεν, καὶ ἄνθρωπος ἰσχύων εἰς τὰ πολεμικά;
\VS{15}Ὤλετο Μωὰβ πόλις αὐτοῦ, καὶ ἐκλεκτοὶ νεανίσκοι αὐτοῦ κατέβησαν εἰς σφαγήν.
\VS{16}Ἐγγὺς ἡμέρα Μωὰβ ἐλθεῖν, καὶ πονηρία αὐτοῦ ταχεῖα σφόδρα.
\VS{17}Κινήσατε αὐτῷ πάντες κυκλόθεν αὐτοῦ, πάντες ἔκδοτε ὄνομα αὐτοῦ· εἴπατε, πῶς συνετρίβη βακτηρία εὐκλεὴς, ῥάβδος μεγαλώματος;
\par }{\PP \VS{18}Κατάβηθι ἀπὸ δόξης, καὶ κάθισον ἐν ὑγρασίᾳ καθημένη· Δαιβὼν ἐκτριβήσεται, ὅτι ὤλετο Μωὰβ, ἀνέβη εἰς σὲ λυμαινόμενος ὀχύρωμά σου.
\VS{19}Ἐφʼ ὁδοῦ στῆθι, καὶ ἔπιδε καθημένη ἐν Ἀρὴρ, καὶ ἐρώτησον φεύγοντα, καὶ σωζόμενον, καὶ εἰπὸν, τί ἐγένετο;
\par }{\PP \VS{20}Κατῃσχύνθη Μωὰβ, ὅτι συνετρίβη· ὀλόλυξον καὶ κέκραξον, ἀνάγγειλον ἐν Ἀρνὼν, ὅτι ὤλετο Μωὰβ,
\VS{21}καὶ κρίσις ἔρχεται εἰς τὴν γῆν Μεισὼρ ἐπὶ Χελὼν, καὶ Ῥεφὰς, καὶ Μωφὰς,
\VS{22}καὶ ἐπὶ Δαιβὼν, καὶ ἐπὶ Ναβαῦ, καὶ ἐπʼ οἶκον Δαιθλαθαὶμ,
\VS{23}καὶ ἐπὶ Καριαθαὶμ, καὶ ἐπʼ οἶκον Γαιμὼλ, καὶ ἐπʼ οἶκον Μαὼν,
\VS{24}καὶ ἐπὶ Καριὼθ, καὶ ἐπὶ Βοσὸρ, καὶ ἐπὶ πάσας τὰς πόλεις Μωὰβ τὰς πόῤῥω καὶ τὰς ἐγγύς.
\VS{25}Κατεάχθη κέρας Μωὰβ, καὶ τὸ ἐπίχειρον αὐτοῦ συνετρίβη.
\par }{\PP \VS{26}Μεθύσατε αὐτὸν, ὅτι ἐπὶ Κύριον ἐμεγαλύνθη· καὶ ἐπικρούσει Μωὰβ ἐν χειρὶ αὐτοῦ, καὶ ἔσται εἰς γέλωτα καὶ αὐτός.
\VS{27}Καὶ εἰ μὴ εἰς γελοιασμὸν ἦν σοι Ἰσραὴλ, καὶ ἐν κλοπαῖς σου εὑρέθη, ὅτι ἐπολέμεις αὐτόν.
\VS{28}Κατέλιπον τὰς πόλεις, καὶ ᾤκησαν ἐν πέτραις οἱ κατοικοῦντες Μωάβ· ἐγενήθησαν ὥσπερ περιστεραὶ νοσσεύουσαι ἐν πέτραις, στόματι βοθύνου.
\par }{\PP \VS{29}Καὶ ἤκουσα ὕβριν Μωὰβ, ὕβρισε λίαν ὕβριν αὐτοῦ, καὶ ὑπερηφανίαν αὐτοῦ· καὶ ὑψώθη ἡ καρδία αὐτοῦ.
\VS{30}Ἐγὼ δὲ ἔγνων ἔργα αὐτοῦ· οὐχὶ τὸ ἱκανὸν αὐτῷ οὐχ οὕτως ἐποίησε;
\par }{\PP \VS{31}Διατοῦτο ἐπὶ Μωὰβ ὀλολύζετε πάντοθεν· βοήσατε ἐπʼ ἄνδρας κειράδας αὐχμοῦ.
\VS{32}Ὡς κλαυθμὸν Ἰαζὴρ ἀποκλαύσομαί σοι ἄμπελος Ἀσερημὰ, κλήματά σου διῆλθε θάλασσαν, πόλεις Ἰαζὴρ ἥψαντο, ἐπὶ ὀπώραν σου, ἐπὶ τρυγηταῖς σου ὄλεθρος ἐπέπεσε.
\VS{33}Συνεψήσθη χαρμοσύνη καὶ εὐφροσύνη ἐκ τῆς Μωαβίτιδος· καὶ οἶνος ἦν ἐπὶ ληνοῖς σου, πρωῒ οὐκ ἐπάτησαν, οὐδὲ δείλης οὐκ ἐποίησαν,
\VS{34}αἱ δὲ ἀπὸ κραυγῆς Ἐσεβὼν ἕως Αἰτὰμ αἱ πόλεις αὐτῶν ἔδωκαν φωνὴν αὐτῶν, ἀπὸ Ζογὸρ ἕως Ὠρωναὶμ, καὶ ἀγγελίαν σαλασία, ὅτι καὶ τὸ ὕδωρ Νεβρεὶν εἰς κατάκαυμα ἔσται.
\par }{\PP \VS{35}Καὶ ἀπολῶ τὸν Μωὰβ, φησὶ Κύριος, ἀναβαίνοντα ἐπὶ τὸν βωμὸν, καὶ θυμιῶντα θεοῖς αὐτοῦ.
\VS{36}Διατοῦτο καρδία τοῦ Μωὰβ, ὥσπερ αὐλοὶ βουβήσουσι, καρδία μου ἐπʼ ἀνθρώπους κειράδας ὥσπερ αὐλὸς βομβήσει· διατοῦτο ἃ περιεποιήσατο, ἀπώλετο ἀπὸ ἀνθρώπου.
\VS{37}Πᾶσαν κεφαλὴν ἐν παντὶ τόπῳ ξυρηθήσονται, καὶ πᾶς πώγων ξυρηθήσεται, καὶ πᾶσαι χεῖρες κόψονται, καὶ ἐπὶ πάσης ὀσφύος σάκκος.
\VS{38}Καὶ ἐπὶ πάντων τῶν δωμάτων Μωὰβ, καὶ ἐπὶ ταῖς πλατείαις αὐτῆς, ὅτι συνέτριψα, φησὶ Κύριος, ὡς ἀγγεῖον, οὗ οὐκ ἔστι χρεία αὐτοῦ.
\VS{39}Πῶς κατήλλαξε; πῶς ἔστρεψε νῶτον Μωάβ; ᾐσχύνθη, καὶ ἐγένετο Μωὰβ εἰς γέλωτα, καὶ ἐγκότημα πᾶσι τοῖς κύκλῳ αὐτῆς.
\par }{\PP \VS{40}Ὅτι οὕτως εἶπε Κύριος,
\VS{41}Ἐλήφθη Καριὼθ, καὶ τὰ ὀχυρώματα συνελήφθη,
\VS{42}καὶ ἀπολεῖται Μωὰβ ἀπὸ ὄχλου, ὅτι ἐπὶ τὸν Κύριον ἐμεγαλύνθη.
\VS{43}Παγὶς καὶ φόβος καὶ βόθυνος ἐπὶ σὲ καθήμενος Μωάβ.
\VS{44}Ὁ φεύγων ἀπὸ προσώπου τοῦ φόβου, ἐμπεσεῖται εἰς τὸν βόθυνον· καὶ ὁ ἀναβαίνων ἐκ τοῦ βοθύνου, καὶ συλληφθήσεται ἐν τῇ παγίδι· ὅτι ἐπάξω ταῦτα ἐπὶ Μωὰβ ἐν ἐνιαυτῷ ἐπισκέψεως αὐτῶν.

\par }\Chap{32}{\PP \VS{15}Οὕτως εἶπε Κύριος ὁ Θεὸς Ἰσραὴλ, λάβε τὸ ποτήριον τοῦ οἴνου τοῦ ἀκράτου τούτου ἐκ χειρός μου, καὶ ποτιεῖς πάντα τὰ ἔθνη, πρὸς ἃ ἐγὼ ἀποστέλλω σε πρὸς αὐτούς.
\VS{16}Καὶ πίονται, καὶ ἐξεμοῦνται, καὶ ἐκμανήσονται ἀπὸ προσώπου τῆς μαχαίρας, ἧς ἐγὼ ἀποστέλλω ἀναμέσον αὐτῶν.
\par }{\PP \VS{17}Καὶ ἔλαβον τὸ ποτήριον ἐκ χειρὸς Κυρίου, καὶ ἐπότισα τὰ ἔθνη, πρὸς ἃ ἀπέστειλέ με Κύριος πρὸς αὐτὰ,
\VS{18}τὴν Ἱερουσαλὴμ, καὶ τὰς πόλεις Ἰούδα, καὶ βασιλεῖς Ἰούδα καὶ ἄρχοντας αὐτοῦ, τοῦ θεῖναι αὐτὰς εἰς ἐρήμωσιν, καὶ εἰς ἄβατον, καὶ εἰς συριγμὸν,
\VS{19}καὶ τὸν Φαραὼ βασιλέα Αἰγύπτου, καὶ τοὺς παῖδας αὐτοῦ, καὶ τοὺς μεγιστᾶνας αὐτοῦ,
\VS{20}καὶ πάντα τὸν λαὸν αὐτοῦ, καὶ πάντας τοὺς συμμίκτους, καὶ πάντας τοὺς βασιλεῖς ἀλλοφύλων, καὶ τὴν Ἀσκάλωνα, καὶ τὴν Γάζαν, καὶ τὴν Ἀκκάρων, καὶ τὸ ἐπίλοιπον Ἀζώτου,
\VS{21}καὶ τὴν Ἰδουμαίαν, καὶ τὴν Μωαβῖτιν, καὶ τοὺς υἱοὺς Ἀμμὼν,
\VS{22}καὶ βασιλεῖς Τύρου, καὶ βασιλεῖς Σιδῶνος, καὶ βασιλεῖς τοὺς ἐν τῷ πέραν τῆς θαλάσσης,
\VS{23}καὶ τὴν Δαιδὰν, καὶ τὴν Θαιμὰν, καὶ τὴν Ῥῶς, καὶ πᾶν περικεκαρμένον κατὰ πρόσωπον αὐτοῦ,
\VS{24}καὶ πάντας τοὺς συμμίκτους τοὺς καταλύοντας ἐν τῇ ἐρήμῳ,
\VS{25}καὶ πάντας βασιλεῖς Αἰλὰμ, καὶ πάντας βασιλεῖς Περσῶν,
\VS{26}καὶ πάντας βασιλεῖς ἀπὸ ἀπηλιώτου τοὺς πόῤῥω καὶ τοὺς ἐγγὺς, ἕκαστον πρὸς τὸν ἀδελφὸν αὐτοῦ, καὶ πάσας βασιλείας τὰς ἐπὶ προσώπου τῆς γῆς.
\par }{\PP \VS{27}Καὶ ἐρεῖς αὐτοῖς, οὕτως εἶπε Κύριος παντοκράτωρ, πίετε, μεθύσθητε, καὶ ἐξεμέσετε, καὶ πεσεῖσθε, καὶ οὐ μὴ ἀναστῆτε ἀπὸ προσώπου τῆς μαχαίρας, ἧς ἐγὼ ἀποστέλλω ἀναμέσον ὑμῶν.
\VS{28}Καὶ ἔσται ὅταν μὴ βούλωνται δέξασθαι τὸ ποτήριον ἐκ τῆς χειρός σου, ὥστε πιεῖν, καὶ ἐρεῖς, οὕτως εἶπε Κύριος, πιόντες πίεσθε,
\VS{29}ὅτι ἐν πόλει ἐν ᾗ ὠνομάσθη τὸ ὄνομά μου ἐπʼ αὐτὴν, ἐγὼ ἄρχομαι κακῶσαι, καὶ ὑμεῖς καθάρσει οὐ μὴ καθαρισθῆτε, ὅτι μάχαιραν ἐγὼ καλῶ ἐπὶ πάντας τοὺς καθημένους ἐπὶ τῆς γῆς.
\par }{\PP \VS{30}Καὶ σὺ προφητεύσεις ἐπʼ αὐτοὺς τοὺς λόγους τούτους, καὶ ἐρεῖς, Κύριος ἀφʼ ὑψηλοῦ χρηματιεῖ, ἀπὸ τοῦ ἁγίου αὐτοῦ δώσει φωνὴν αὐτοῦ, λόγον χρηματιεῖ ἐπὶ τοῦ τόπου αὐτοῦ· καὶ οἵδε ὥσπερ τρυγῶντες ἀποκριθήσονται· καὶ ἐπὶ καθημένους ἐπὶ τὴν γῆν
\VS{31}ἥκει ὄλεθρος ἐπὶ μέρος τῆς γῆς, ὅτι κρίσις τῷ Κυρίῳ ἐν τοῖς ἔθνεσι· κρίνεται αὐτὸς πρὸς πᾶσαν σάρκα, οἱ δὲ ἀσεβεῖς ἐδόθησαν εἰς μάχαιραν, λέγει Κύριος.
\par }{\PP \VS{32}Οὕτως εἶπε Κύριος, ἰδοὺ κακὰ ἔρχεται ἀπὸ ἔθνους ἐπὶ ἔθνος, καὶ λαίλαψ μεγάλη ἐκπορεύεται ἀπʼ ἐσχάτου τῆς γῆς.
\VS{33}Καὶ ἔσονται τραυματίαι ὑπὸ Κυρίου ἐν ἡμέρᾳ Κυρίου, ἐκ μέρους τῆς γῆς, καὶ ἕως εἰς μέρος τῆς γῆς· οὐ μὴ κατορυγῶσιν, εἰς κόπρια ἐπὶ προσώπου τῆς γῆς ἔσονται.
\VS{34}Ἀλαλάξατε ποιμένες, καὶ κεκράξατε, καὶ κόπτεσθε οἱ κριοὶ τῶν προβάτων, ὅτι ἐπληρώθησαν αἱ ἡμέραι ὑμῶν εἰς σφαγὴν, καὶ πεσεῖσθε ὥσπερ οἱ κριοὶ οἱ ἐκλεκτοὶ,
\VS{35}καὶ ἀπολεῖται φυγὴ ἀπὸ τῶν ποιμένων, καὶ σωτηρία ἀπὸ τῶν κριῶν τῶν προβάτων.
\VS{36}Φωνὴ κραυγῆς τῶν ποιμένων, καὶ ἀλαλαγμὸς τῶν προβάτων καὶ τῶν κριῶν, ὅτι ὠλόθρευσε Κύριος τὰ βοσκήματα αὐτῶν·
\VS{37}Καὶ παύσεται τὰ κατάλοιπα τῆς εἰρήνης ἀπὸ προσώπου ὀργῆς θυμοῦ μου.
\VS{38}Ἐγκατέλιπεν ὥσπερ λέων κατάλυμα αὐτοῦ, ὅτι ἐγενήθη ἡ γῆ αὐτῶν εἰς ἄβατον ἀπὸ προσώπου τῆς μαχαίρας τῆς μεγάλης.

\par }\Chap{33}{\PP \VerseOne{1}ἘΝ ἈΡΧΗ ΒΑΣΙΛΕΩΣ ἸΩΑΚΕΙΜ ΥΙΟΥ ἸΩΣΙΑ, ἘΓΕΝΗΘΗ Ὁ ΛΟΓΟΣ ΟΥΤΟΣ ΠΑΡΑ ΚΥΡΙΟΥ·
\VS{2}Οὕτως εἶπε Κύριος, στῆθι ἐν αὐλῇ οἴκου Κυρίου, καὶ χρηματιεῖς ἅπασι τοῖς Ἰουδαίοις, καὶ πᾶσι τοῖς ἐρχομένοις προσκυνεῖν ἐν οἴκῳ Κυρίου, ἅπαντας τοὺς λόγους οὓς συνέταξά σοι χρηματίσαι αὐτοῖς, μὴ ἀφέλῃς ῥῆμα.
\VS{3}Ἴσως ἀκούσονται, καὶ ἀποστραφήσονται ἕκαστος ἀπὸ τῆς ὁδοῦ αὐτοῦ τῆς πονηρᾶς, καὶ παύσομαι ἀπὸ τῶν κακῶν ὧν ἐγὼ λογίζομαι τοῦ ποιῆσαι αὐτοῖς ἕνεκεν τῶν πονηρῶν ἐπιτηδευμάτων αὐτῶν.
\VS{4}Καὶ ἐρεῖς, οὕτως εἶπε Κύριος, ἐὰν μὴ ἀκούσητέ μου, τοῦ πορεύεσθαι ἐν τοῖς νομίμοις μου οἷς ἔδωκα κατὰ πρόσωπον ὑμῶν,
\VS{5}εἰσακούειν τῶν λόγων τῶν παίδων μου τῶν προφητῶν, οὓς ἐγὼ ἀποστέλλω πρὸς ὑμᾶς ὄρθρου, καὶ ἀπέστειλα, καὶ οὐκ ἠκούσατέ μου,
\VS{6}καὶ δώσω τὸν οἶκον τοῦτον ὥσπερ Σηλὼ, καὶ τὴν πόλιν δώσω εἰς κατάραν πᾶσι τοῖς ἔθνεσι πάσης τῆς γῆς.
\par }{\PP \VS{7}Καὶ ἤκουσαν οἱ ἱερεῖς, καὶ οἱ ψευδοπροφῆται, καὶ πᾶς ὁ λαὸς τοῦ Ἱερεμίου λαλοῦντος τοὺς λόγους τούτους ἐν οἴκῳ Κυρίου.
\VS{8}Καὶ ἐγένετο Ἱερεμίου παυσαμένου λαλοῦντος πάντα ἃ συνέταξε Κύριος αὐτῷ λαλῆσαι παντὶ τῷ λαῷ, καὶ συνελάβοσαν αὐτὸν οἱ ἱερεῖς, καὶ οἱ ψευδοπροφῆται, καὶ πᾶς ὁ λαὸς, λέγων, θανάτῳ ἀποθανῇ,
\VS{9}ὅτι ἐπροφήτευσας τῷ ὀνόματι Κυρίου, λέγων, ὥσπερ Σηλὼ ἔσται ὁ οἶκος οὗτος, καὶ ἡ πόλις αὕτη ἐρημωθήσεται ἀπὸ κατοικούντων·
\par }{\PP Καὶ ἐξεκκλησιάσθη πᾶς ὁ λαὸς ἐπὶ Ἱερεμίαν ἐν οἴκῳ Κυρίου.
\VS{10}Καὶ ἤκουσαν οἱ ἄρχοντες Ἰούδα τὸν λόγον τοῦτον, καὶ ἀνέβησαν ἐξ οἴκου τοῦ βασιλέως εἰς οἶκον Κυρίου, καὶ ἐκάθισαν ἐν προθύροις πύλης τῆς καινῆς.
\VS{11}Καὶ εἶπαν οἱ ἱερεῖς καὶ οἱ ψευδοπροφῆται πρὸς τοὺς ἄρχοντας, καὶ παντὶ τῷ λαῷ, κρίσις θανάτου τῷ ἀνθρώπῳ τούτῳ, ὅτι ἐπροφήτευσε κατὰ τῆς πόλεως ταύτης, καθὼς ἠκούσατε ἐν τοῖς ὠσὶν ὑμῶν.
\par }{\PP \VS{12}Καὶ εἶπεν Ἱερεμίας πρὸς τοὺς ἄρχοντας, καὶ παντὶ τῷ λαῷ, λέγων, Κύριος ἀπέστειλέ με προφητεῦσαι ἐπὶ τὸν οἶκον τοῦτον, καὶ ἐπὶ τὴν πόλιν ταύτην, πάντας τοὺς λόγους οὓς ἠκούσατε.
\VS{13}Καὶ νῦν βελτίους ποιήσατε τὰς ὁδοὺς ὑμῶν, καὶ τὰ ἔργα ὑμῶν, καὶ ἀκούσατε τῆς φωνῆς Κυρίου, καὶ παύσεται Κύριος ἀπὸ τῶν κακῶν ὧν ἐλάλησεν ἐφʼ ὑμᾶς.
\VS{14}Καὶ ἰδοὺ ἐγὼ ἐν χερσὶν ὑμῶν, ποιήσατέ μοι ὡς συμφέρει, καὶ ὡς βέλτιον ὑμῖν.
\VS{15}Ἀλλʼ ἢ γνόντες γνώσεσθε, ὅτι εἰ ἀναιρεῖτέ με, αἷμα ἀθῶον δίδοτε ἐφʼ ὑμᾶς, καὶ ἐπὶ τὴν πόλιν ταύτην, καὶ ἐπὶ τοὺς κατοικοῦντας ἐν αὐτῇ· ὅτι ἐν ἀληθείᾳ ἀπέσταλκέ με Κύριος πρὸς ὑμᾶς λαλῆσαι εἰς τὰ ὦτα ὑμῶν πάντας τοὺς λόγους τούτους.
\par }{\PP \VS{16}Καὶ εἶπον οἱ ἄρχοντες καὶ πᾶς ὁ λαὸς πρὸς τοὺς ἱερεῖς καὶ πρὸς τοὺς ψευδοπροφήτας, οὐκ ἔστι τῷ ἀνθρώπῳ τούτῳ κρίσις θανάτου, ὅτι ἐπὶ τῷ ὀνόματι Κυρίου τοῦ Θεοῦ ἡμῶν ἐλάλησε πρὸς ἡμᾶς.
\VS{17}Καὶ ἀνέστησαν ἄνδρες τῶν πρεσβυτέρων τῆς γῆς, καὶ εἶπαν πάσῃ τῇ συναγωγῇ τοῦ λαοῦ,
\VS{18}Μιχαίας ὁ Μωραθίτης ἦν ἐν ταῖς ἡμέραις Ἐζεκίου βασιλέως Ἰούδα, καὶ εἶπε παντὶ τῷ λαῷ Ἰούδα, οὕτως εἶπε Κύριος, Σιὼν ὡς ἀγρὸς ἀροτριαθήσεται, καὶ Ἱερουσαλὴμ εἰς ἄβατον ἔσται, καὶ τὸ ὄρος τοῦ οἴκου εἰς ἄλσος δρυμοῦ.
\VS{19}Μὴ ἀνελὼν ἀνεῖλεν αὐτὸν Ἐζεκίας καὶ πᾶς Ἰούδα; οὐχ ὅτι ἐφοβήθησαν τὸν Κύριον, καὶ ὅτι ἐδεήθησαν τοῦ προσώπου Κυρίου, καὶ ἐπαύσατο Κύριος ἀπὸ τῶν κακῶν ὧν ἐλάλησεν ἐπʼ αὐτούς; καὶ ἡμεῖς ἐποιήσαμεν κακὰ μεγάλα ἐπὶ ψυχὰς ἡμῶν.
\par }{\PP \VS{20}Καὶ ἄνθρωπος ἦν προφητεύων τῷ ὀνόματι Κυρίου, Οὐρίας υἱὸς Σαμαίου ἐκ Καριαθιαρὶμ, καὶ ἐπροφήτευσε περὶ τῆς γῆς ταύτης κατὰ πάντας τοὺς λόγους Ἱερεμίου.
\VS{21}Καὶ ἤκουσεν ὁ βασιλεὺς Ἰωακεὶμ καὶ πάντες οἱ ἄρχοντες πάντας τοὺς λόγους αὐτοῦ, καὶ ἐζήτουν ἀποκτεῖναι αὐτόν· καὶ ἤκουσεν Οὐρίας, καὶ εἰσῆλθεν εἰς Αἴγυπτον.
\VS{22}Καὶ ἐξαπέστειλεν ὁ βασιλεὺς ἄνδρας εἰς Αἴγυπτον,
\VS{23}καὶ ἐξηγάγοσαν αὐτὸν ἐκεῖθεν, καὶ εἰσηγάγοσαν αὐτὸν πρὸς τὸν βασιλέα, καὶ ἐπάταξεν αὐτὸν ἐν μαχαίρᾳ, καὶ ἔῤῥιψεν αὐτὸν εἰς τὸ μνῆμα υἱῶν λαοῦ αὐτοῦ.
\VS{24}Πλὴν χεὶρ Ἀχεικὰμ υἱοῦ Σαφὰν ἦν μετὰ Ἱερεμίου, τοῦ μὴ παραδοῦναι αὐτὸν εἰς χεῖρας τοῦ λαοῦ, μὴ ἀνελεῖν αὐτόν.

\par }\Chap{34}{\PP \VS{2}Οὕτως εἶπε Κύριος, ποίησον σεαυτῷ δεσμοὺς καὶ κλοιοὺς, καὶ περίθου περὶ τὸν τράχηλόν σου.
\VS{3}Καὶ ἀποστελεῖς αὐτοὺς πρὸς βασιλέα Ἰδουμαίας, καὶ πρὸς βασιλέα Μωὰβ, καὶ πρὸς βασιλέα υἱῶν Ἀμμὼν, καὶ πρὸς τὸν βασιλέα Τύρου, καὶ πρὸς βασιλέα Σιδῶνος, ἐν χερσὶν ἀγγέλων αὐτῶν τῶν ἐρχομένων εἰς ἀπάντησιν αὐτῶν εἰς Ἱερουσαλὴμ πρὸς Σεδεκίαν βασιλέα Ἰούδα.
\VS{4}Καὶ συντάξεις αὐτοῖς πρὸς τοὺς κυρίους αὐτῶν εἰπεῖν, οὕτως εἶπε Κύριος ὁ Θεὸς Ἰσραὴλ, οὕτως ἐρεῖτε πρὸς τοὺς κυρίους ὑμῶν,
\VS{5}ὅτι ἐγὼ ἐποίησα τὴν γῆν ἐν τῇ ἰσχύϊ μου τῇ μεγάλῃ, καὶ ἐν τῷ ἐπιχείρῳ μου τῷ ὑψηλῷ, καὶ δώσω αὐτὴν ᾧ ἐὰν δόξῃ ἐν ὀφθαλμοῖς μου·
\VS{6}Ἔδωκα τὴν γῆν τῷ Ναβουχοδονόσορ βασιλεῖ Βαβυλῶνος δουλεύειν αὐτῷ, καὶ τὰ θηρία τοῦ ἀγροῦ ἐργάζεσθαι αὐτῷ.
\VS{8}Καὶ τὸ ἔθνος καὶ ἡ βασιλεία, ὅσοι ἐὰν μὴ ἐμβάλωσι τὸν τράχηλον αὐτῶν ὑπὸ τὸν ζυγὸν βασιλέως Βαβυλῶνος, ἐν μαχαίρᾳ καὶ ἐν λιμῷ ἐπισκέψομαι αὐτοὺς, εἶπε Κύριος, ἕως ἐκλίπωσιν ἐν χειρὶ αὐτοῦ.
\par }{\PP \VS{9}Καὶ ὑμεῖς μὴ ἀκούετε τῶν ψευδοπροφητῶν ὑμῶν, καὶ τῶν μαντευομένων ὑμῖν, καὶ τῶν ἐνυπνιαζομένων ὑμῖν, καὶ τῶν οἰωνισμάτων ὑμῶν, καὶ τῶν φαρμακῶν ὑμῶν, τῶν λεγόντων, οὐ μὴ ἐργάσησθε τῷ βασιλεῖ Βαβυλῶνος·
\VS{10}Ὅτι ψευδῆ αὐτοὶ προφητεύουσιν ὑμῖν, πρὸς τὸ μακρῦναι ὑμᾶς ἀπὸ τῆς γῆς ὑμῶν.
\VS{11}Καὶ τὸ ἔθνος ὃ ἐὰν εἰσαγάγῃ τὸν τράχηλον αὐτοῦ ὑπὸ τὸν ζυγὸν βασιλέως Βαβυλῶνος, καὶ ἐργάσηται αὐτῷ, καὶ καταλείψω αὐτὸν ἐπὶ τῆς γῆς αὐτοῦ, καὶ ἐργᾶται αὐτῷ, καὶ ἐνοικήσει ἐν αὐτῇ.
\par }{\PP \VS{12}Καὶ πρὸς Σεδεκίαν βασιλέα Ἰούδα ἐλάλησα κατὰ πάντας τοὺς λόγους τούτους, λέγων, εἰσαγάγετε τὸν τράχηλον ὑμῶν,
\VS{14}καὶ ἐργάσασθε τῷ βασιλεῖ Βαβυλῶνος, ὅτι ἄδικα αὐτοὶ προφητεύουσιν ὑμῖν,
\VS{15}ὅτι οὐκ ἀπέστειλα αὐτοὺς, φησὶ Κύριος, καὶ προφητεύουσι τῷ ὀνόματί μου ἐπ ἀδίκῳ, πρὸς τὸ ἀπολέσαι ὑμᾶς, καὶ ἀπολεῖσθε ὑμεῖς, καὶ οἱ προφῆται ὑμῶν, οἱ προφητεύοντες ὑμῖν ἐπʼ ἀδίκῳ ψευδῆ.
\par }{\PP \VS{16}Ὑμῖν, καὶ παντὶ τῷ λαῷ τούτῳ, καὶ τοῖς ἱερεῦσιν ἐλάλησα, λέγων, οὕτως εἶπε Κύριος, μὴ ἀκούετε τῶν λόγων τῶν προφητῶν, τῶν προφητευόντων ὑμῖν, λεγόντων, ἰδοὺ σκεύη οἴκου Κυρίου ἐπιστρέψει ἐκ Βαβυλῶνος· ὅτι ἄδικα αὐτοὶ προφητεύουσιν ὑμῖν. Οὐκ ἀπέστειλα αὐτούς.
\VS{18}Εἰ προφῆταί εἰσι, καὶ εἰ ἔστι λόγος Κυρίου ἐν αὐτοῖς, ἀπαντησάτωσάν μοι, ὅτι οὕτως εἶπε Κύριος.
\par }{\PP \VS{19}Καὶ τῶν ἐπιλοίπων σκευῶν,
\VS{20}ὧν οὐκ ἔλαβε βασιλεὺς Βαβυλῶνος, ὅτε ἀπῴκισε τὸν Ἰεχονίαν ἐξ Ἱερουσαλὴμ,
\VS{22}εἰς Βαβυλῶνα εἰσελεύσεται, λέγει Κύριος.

\par }\Chap{35}{\PP \VerseOne{1}Καὶ ἐγένετο ἐν τῷ τετάρτῳ ἔτει Σεδεκία βασιλέως Ἰούδα ἐν μηνὶ τῷ πέμπτῳ, εἶπέ μοι Ἀνανίας υἱὸς Ἀζὼρ ὁ ψευδοπροφήτης ἀπὸ Γαβαὼν ἐν οἴκῳ Κυρίου, κατʼ ὀφθαλμοὺς τῶν ἱερέων, καὶ παντὸς τοῦ λαοῦ, λέγων,
\VS{2}Οὕτως εἶπε Κύριος, συνέτριψα τὸν ζυγὸν τοῦ βασιλέως Βαβυλῶνος.
\VS{3}Ἔτι δύο ἔτη ἡμερῶν, καὶ ἐγὼ ἀποστρέψω εἰς τὸν τόπον τοῦτον τὰ σκεύη οἴκου Κυρίου,
\VS{4}καὶ Ἰεχονίαν, καὶ τὴν ἀποικίαν Ἰούδα, ὅτι συντρίψω τὸν ζυγὸν βασιλέως Βαβυλῶνος.
\par }{\PP \VS{5}Καὶ εἶπεν Ἱερεμίας πρὸς Ἀνανίαν κατʼ ὀφθαλμοὺς παντὸς τοῦ λαοῦ καὶ κατʼ ὀφθαλμοὺς τῶν ἱερέων τῶν ἑστηκότων ἐν οἴκῳ Κυρίου,
\VS{6}καὶ εἶπεν Ἱερεμίας, ἀληθῶς οὕτω ποιήσαι Κύριος, στήσαι τὸν λόγον σου ὃν σὺ προφητεύεις, τοῦ ἐπιστρέψαι τὰ σκεύη οἴκου Κυρίου καὶ πᾶσαν τὴν ἀποικίαν ἐκ Βαβυλῶνος εἰς τὸν τόπον τοῦτον.
\VS{7}Πλὴν ἀκούσατε τὸν λόγον Κυρίου, ὃν ἐγὼ λέγω εἰς τὰ ὦτα ὑμῶν καὶ εἰς τὰ ὦτα παντὸς τοῦ λαοῦ.
\VS{8}Οἱ προφῆται οἱ γεγονότες πρότεροί μου καὶ πρότεροι ὑμῶν ἀπὸ τοῦ αἰῶνος, καὶ ἐπροφήτευσαν ἐπὶ γῆς πολλῆς, καὶ ἐπὶ βασιλείας μεγάλας εἰς πόλεμον.
\VS{9}Ὁ προφήτης ὁ προφητεύσας εἰς εἰρήνην, ἐλθόντος τοῦ λόγου, γνώσονται τὸν προφήτην ὃν ἀπέστειλεν αὐτοῖς Κύριος ἐν πίστει.
\par }{\PP \VS{10}Καὶ ἔλαβεν Ἀνανίας ἐν ὀφθαλμοῖς παντὸς τοῦ λαοῦ τοὺς κλοιοὺς ἀπὸ τοῦ τραχήλου Ἱερεμίου, καὶ συνέτριψεν αὐτούς.
\VS{11}Καὶ εἶπεν Ἀνανίας κατʼ ὀφθαλμοὺς παντὸς τοῦ λαοῦ, λέγων, οὕτως εἶπε Κύριος, οὕτως συντρίψω τὸν ζυγὸν βασιλέως Βαβυλῶνος ἀπὸ τραχήλων πάντων τῶν ἐθνῶν· καὶ ᾤχετο Ἱερεμίας εἰς τὴν ὁδὸν αὐτοῦ.
\par }{\PP \VS{12}Καὶ ἐγένετο λόγος Κυρίου πρὸς Ἱερεμίαν, μετὰ τὸ συντρίψαι Ἀνανίαν τοὺς κλοιοὺς ἀπὸ τοῦ τραχήλου αὐτοῦ, λέγων,
\VS{13}Βάδιζε, καὶ εἰπὸν πρὸς Ἀνανίαν, λέγων, οὕτως εἶπε Κύριος, κλοιοὺς ξυλίνους συνέτριψας, καὶ ποιήσω ἀντʼ αὐτῶν κλοιοὺς σιδηροῦς·
\VS{14}Ὅτι οὕτως εἶπε Κύριος, ζυγὸν σιδηροῦν ἔθηκα ἐπὶ τὸν τράχηλον πάντων τῶν ἐθνῶν, ἐργάζεσθαι τῷ βασιλεῖ Βαβυλῶνος.
\VS{15}Καὶ εἶπεν Ἱερεμίας τῷ Ἀνανίᾳ, οὐκ ἀπέσταλκέ σε Κύριος, καὶ πεποιθέναι ἐποίησας τὸν λαὸν τοῦτον ἐπʼ ἀδίκῳ.
\VS{16}Διατοῦτο οὕτως εἶπε Κύριος, ἰδοὺ ἐγὼ ἐξαποστέλλω σε ἀπὸ προσώπου τῆς γῆς, τούτῳ τῷ ἐνιαυτῷ ἀποθανῇ.
\VS{17}Καὶ ἀπέθανεν ἐν τῷ μηνὶ τῷ ἑβδόμῳ.

\par }\Chap{36}{\PP \VerseOne{1}Καὶ οὗτοι οἱ λόγοι τῆς βίβλου οὓς ἀπέστειλεν Ἱερεμίας ἐξ Ἱερουσαλὴμ πρὸς τοὺς πρεσβυτέρους τῆς ἀποικίας, καὶ πρὸς τοὺς ἱερεῖς, καὶ πρὸς τοὺς ψευδοπροφήτας, ἐπιστολὴν εἰς Βαβυλῶνα τῇ ἀποικίᾳ, καὶ πρὸς ἅπαντα τὸν λαὸν,
\VS{2}ὕστερον ἐξελθόντος Ἰεχονίου τοῦ βασιλέως, καὶ τῆς βασιλίσσης, καὶ τῶν εὐνούχων, καὶ παντὸς ἐλευθέρου, καὶ δεσμώτου, καὶ τεχνίτου ἐξ Ἱερουσαλὴμ,
\VS{3}ἐν χειρὶ Ἐλεασὰν υἱοῦ Σαφὰν, καὶ Γαμαρίου υἱοῦ Χελκίου, ὃν ἀπέστειλε Σεδεκίας βασιλεὺς Ἰούδα, πρὸς βασιλέα Βαβυλῶνος εἰς Βαβυλῶνα, λέγων,
\par }{\PP \VS{4}Οὕτως εἶπε Κύριος ὁ Θεὸς Ἰσραὴλ ἐπὶ τὴν ἀποικίαν ἣν ἀπῴκισα ἀπὸ Ἱερουσαλὴμ,
\VS{5}οἰκοδομήσατε οἴκους, καὶ κατοικήσατε, καὶ φυτεύσατε παραδείσους, καὶ φάγετε τοὺς καρποὺς αὐτῶν,
\VS{6}καὶ λάβετε γυναῖκας, καὶ τεκνοποιήσατε υἱοὺς καὶ θυγατέρας, καὶ λάβετε τοῖς υἱοῖς ὑμῶν γυναῖκας, καὶ τὰς θυγατέρας ὑμῶν δότε ἀνδράσι, καὶ πληθύνεσθε, καὶ μὴ σμικρυνθῆτε·
\VS{7}Καὶ ζητήσατε εἰς εἰρήνην τῆς γῆς, εἰς ἣν ἀπῴκισα ὑμᾶς ἐκεῖ· καὶ προσεύξεσθε περὶ αὐτῶν πρὸς Κύριον, ὅτι ἐν εἰρήνῃ αὐτῆς εἰρήνη ὑμῖν.
\par }{\PP \VS{8}Ὅτι οὕτως εἶπε Κύριος, μὴ ἀναπειθέτωσαν ὑμᾶς οἱ ψευδοπροφῆται οἱ ἐν ὑμῖν, καὶ μὴ ἀναπειθέτωσαν ὑμᾶς οἱ μάντεις ὑμῶν, καὶ μὴ ἀκούετε εἰς τὰ ἐνύπνια ὑμῶν, ἃ ὑμεῖς ἐνυπνιάζεσθε,
\VS{9}ὅτι ἄδικα αὐτοὶ προφητεύουσιν ὑμῖν ἐπὶ τῷ ὀνόματί μου, καὶ οὐκ ἀπέστειλα αὐτούς.
\VS{10}Ὅτι οὕτως εἶπε Κύριος, ὅταν μέλλῃ πληροῦσθαι Βαβυλῶνι ἑβδομήκοντα ἔτη, ἐπισκέψομαι ὑμᾶς, καὶ ἐπιστήσω τοὺς λόγους μου ἐφʼ ὑμᾶς, τοῦ ἀποστρέψαι τὸν λαὸν ὑμῶν εἰς τὸν τόπον τοῦτον.
\VS{11}Καὶ λογιοῦμαι ἐφʼ ὑμᾶς λογισμὸν εἰρήνης, καὶ οὐ κακὰ, τοῦ δοῦναι ὑμῖν ταῦτα.
\VS{12}Καὶ προσεύξασθε πρὸς μὲ, καὶ εἰσακούσομαι ὑμῶν.
\VS{13}Καὶ ἐκζητήσατέ με, καὶ εὑρήσετέ με· ὅτι ζητήσετέ με ἐν ὅλῃ καρδίᾳ ὑμῶν,
\VS{14}καὶ ἐπιφανοῦμαι ὑμῖν·
\VS{15}Ὅτι εἴπατε, κατέστησεν ἡμῖν Κύριος προφήτας ἐν Βαβυλῶνι·
\par }{\PP \VS{21}Οὕτως εἶπε Κύριος ἐπὶ Ἀχιὰβ, καὶ ἐπὶ Σεδεκίαν, ἰδοὺ ἐγὼ δίδωμι αὐτοὺς εἰς χεῖρας βασιλέως Βαβυλῶνος, καὶ πατάξει αὐτοὺς κατʼ ὀφθαλμοὺς ὑμῶν.
\VS{22}Καὶ λήψονται ἀπʼ αὐτῶν κατάραν ἐν πάσῃ τῇ ἀποικίᾳ Ἰούδα ἐν Βαβυλῶνι, λέγοντες, ποιήσαι σε Κύριος, ὡς Σεδεκίαν ἐποίησε, καὶ ὡς Ἀχιὰβ, οὓς ἀπετηγάνισε βασιλεὺς Βαβυλῶνος ἐν πυρὶ,
\VS{23}διʼ ἣν ἐποίησαν ἀνομίαν ἐν Ἰσραὴλ, καὶ ἐμοιχῶντο τὰς γυναῖκας τῶν πολιτῶν αὐτῶν, καὶ λόγον ἐχρημάτισαν ἐν τῷ ὀνόματί μου, ὃν οὐ συνέταξα αὐτοῖς· καὶ ἐγὼ μάρτυς, φησὶ Κύριος.
\par }{\PP \VS{24}Καὶ πρὸς Σαμαίαν τὸν Αἰλαμίτην ἐρεῖς,
\VS{25}οὐκ ἀπέστειλά σε τῷ ὀνόματί μου· καὶ πρὸς Σοφονίαν υἱὸν Μαασαίου τὸν ἱερέα εἰπὲ,
\VS{26}Κύριος ἔδωκέ σε ἱερέα ἀντὶ Ἰωδαὲ τοῦ ἱερέως, γενέσθαι ἐπιστάτην ἐν τῷ οἴκῳ Κυρίου παντὶ ἀνθρώπῳ προφητεύοντι, καὶ παντὶ ἀνθρώπῳ μαινομένῳ, καὶ δώσεις αὐτὸν εἰς τὸ ἀπόκλεισμα, καὶ εἰς τὸν καταράκτην.
\VS{27}Καὶ νῦν διατί συνελοιδορήσατε Ἱερεμίαν τὸν ἐξ Ἀναθὼθ, τὸν προφητεύσαντα ὑμῖν;
\VS{28}Οὐ διατοῦτο ἀπέστειλεν; ὅτι διὰ τοῦ μηνὸς τούτου ἀπέστειλε πρὸς ὑμᾶς εἰς Βαβυλῶνα, λέγων, μακράν ἐστιν, οἰκοδομήσατε οἰκίας, καὶ κατοικήσατε, καὶ φυτεύσατε κήπους, καὶ φάγεσθε τὸν καρπὸν αὐτῶν.
\VS{29}Καὶ ἀνέγνω Σοφονίας τὸ βιβλίον εἰς τὰ ὦτα Ἱερεμίου.
\par }{\PP \VS{30}Καὶ ἐγένετο λόγος Κυρίου πρὸς Ἱερεμίαν, λέγων,
\VS{31}ἀπόστειλον πρὸς τὴν ἀποικίαν, λέγων, οὕτως εἶπε Κύριος ἐπὶ Σαμαίαν τὸν Αἰλαμίτην, ἐπειδὴ ἐπροφήτευσεν ὑμῖν Σαμαίας, καὶ ἐγὼ οὐκ ἀπέστειλα αὐτὸν, καὶ πεποιθέναι ἐποίησεν ὑμᾶς ἐπʼ ἀδίκοις,
\VS{32}διατοῦτο οὕτως εἶπε Κύριος, ἰδοὺ ἐγὼ ἐπισκέψομαι ἐπὶ Σαμαίαν, καὶ ἐπὶ τὸ γένος αὐτοῦ, καὶ οὐκ ἔσται αὐτῶν ἄνθρωπος ἐν μέσῳ ὑμῶν, τοῦ ἰδεῖν τὰ ἀγαθὰ, ἃ ἐγὼ ποιήσω ὑμῖν, οὐκ ὄψονται.

\par }\Chap{37}{\PP \VerseOne{1}Ὁ ΛΟΓΟΣ Ὁ ΓΕΝΟΜΕΝΟΣ ΠΡΟΣ ἹΕΡΕΜΙΑΝ ΠΑΡΑ ΚΥΡΙΟΥ, ΕΙΠΕΙΝ,
\VS{2}Οὕτως εἶπε Κύριος ὁ Θεὸς Ἰσραὴλ, λέγων,
\par }{\PP Γράψον πάντας τοὺς λόγους οὓς ἐχρημάτισα πρὸς σὲ ἐπὶ βιβλίου.
\VS{3}Ὅτι ἰδοὺ ἡμέραι ἔρχονται, φησὶ Κύριος, καὶ ἀποστρέψω τὴν ἀποικίαν λαοῦ μου Ἰσραὴλ καὶ Ἰούδα, εἶπε Κύριος, καὶ ἀποστρέψω αὐτοὺς εἰς τὴν γῆν ἣν ἔδωκα τοῖς πατράσιν αὐτῶν, καὶ κυριεύσουσιν αὐτῆς.
\par }{\PP \VS{4}ΚΑΙ ΟΥΤΟΙ ΟΙ ΛΟΓΟΙ ΟΥΣ ἘΛΑΛΗΣΕ ΚΥΡΙΟΣ ἘΠΙ ἸΣΡΑΗΛ ΚΑΙ ἸΟΥΔΑ.
\par }{\PP \VS{5}Οὕτως εἶπε Κύριος, φωνὴν φόβου ἀκούσεσθε· φόβος, καὶ οὐκ ἔστιν εἰρήνη.
\VS{6}Ἐρωτήσατε, καὶ ἴδετε, εἰ ἔτεκεν ἄρσεν; καὶ περὶ φόβου, ἐν ᾧ καθέξουσιν ὀσφὺν καὶ σωτηρίαν; διότι ἑώρακα πάντα ἄνθρωπον, καὶ αἱ χεῖρες αὐτοῦ ἐπὶ τῆς ὀσφύος αὐτοῦ· ἐστράφησαν πρόσωπα εἰς ἴκτερον.
\VS{7}Ὅτι ἐγενήθη μεγάλη ἡ ἡμέρα ἐκείνη, καὶ οὐκ ἔστι τοιαύτη· καὶ χρόνος στενός ἐστι τῷ Ἰακὼβ,
\VS{8}καὶ ἀπὸ τούτου σωθήσεται. Ἐν τῇ ἡμέρᾳ ἐκείνῃ, εἶπε Κύριος, συντρίψω τὸν ζυγὸν ἀπὸ τοῦ τραχήλου αὐτῶν, καὶ τοὺς δεσμοὺς αὐτῶν διαῤῥήξω· καὶ οὐκ ἐργῶνται αὐτοὶ ἔτι ἀλλοτρίοις,
\VS{9}καὶ ἐργῶνται τῷ Κυρίῳ Θεῷ αὐτῶν· καὶ τὸν Δαυὶδ βασιλέα αὐτῶν ἀναστήσω αὐτοῖς.
\par }{\PP \VS{12}Οὕτως εἶπε Κύριος, ἀνέστησα σύντριμμα, ἀλγηρὰ ἡ πληγή σου,
\VS{13}οὐκ ἔστι κρίνων κρίσιν σου, εἰς ἀλγηρὸν ἰατρεύθης, ὠφέλειά σοι οὐκ ἔστι.
\VS{14}Πάντες οἱ φίλοι σου ἐπελάθοντό σου, οὐ μὴ ἐπερωτήσωσιν· ὅτι πληγὴν ἐχθροῦ ἔπαισά σε, παιδείαν στερεάν· ἐπὶ πᾶσαν ἀδικίαν σου ἐπλήθυναν αἱ ἁμαρτίαι σου.
\VS{16}Διατοῦτο πάντες οἱ ἔσθοντές σε βρωθήσονται, καὶ πάντες οἱ ἐχθροί σου κρέας αὐτῶν πᾶν ἔδονται. Ἐπὶ πλῆθος ἀδικιῶν σου ἐπληθύνθησαν αἱ ἁμαρτίαι σου· ἐποίησαν ταῦτά σοι· Καὶ ἔσονται οἱ διαφοροῦντές σε εἰς διαφόρημα, καὶ πάντας τοὺς προνομεύσαντάς σε δώσω εἰς προνομήν.
\VS{17}Ὅτι ἀνάξω τὸ ἴαμά σου, ἀπὸ πληγῆς ὀδυνηρᾶς ἰατρεύσω σε, φησὶ Κύριος· ὅτι Ἐσπαρμένη ἐκλήθης, θήρευμα ὑμῶν ἐστιν, ὅτι ζητῶν οὐκ ἔστιν αὐτήν.
\par }{\PP \VS{18}Οὕτως εἶπε Κύριος, ἰδοὺ ἐγὼ ἀποστρέψω τὴν ἀποικιαν Ἰακὼβ, καὶ τὴν αἰχμαλωσίαν αὐτοῦ ἐλεήσω· καὶ οἰκοδομηθήσεται πόλις ἐπὶ τὸ ὕψος αὐτῆς, καὶ ὁ λαὸς κατὰ τὸ κρίμα αὐτοῦ καθεδεῖται,
\VS{19}καὶ ἐξελεύσονται ἀπʼ αὐτῶν ᾄδοντες, φωνὴ παιζόντων· καὶ πλεονάσω αὐτοὺς, καὶ οὐ μὴ ἐλαττωθῶσι.
\VS{20}Καὶ εἰσελεύσονται οἱ υἱοὶ αὐτῶν ὡς τὸ πρότερον, καὶ τὰ μαρτύρια αὐτῶν κατὰ πρόσωπόν μου ὀρθωθήσεται· καὶ ἐπισκέψομαι τοὺς θλίβοντας αὐτοὺς.
\VS{21}Καὶ ἔσονται ἰσχυρότεροι αὐτοῦ ἐπʼ αὐτοὺς, καὶ ὁ ἄρχων αὐτοῦ ἐξ αὐτοῦ ἐξελεύσεται· καὶ συνάξω αὐτοὺς, καὶ ἀποστρέψουσι πρὸς μέ· ὅτι τίς ἐστιν οὗτος ὃς ἔδωκε τὴν καρδίαν αὐτοῦ, ἀποστρέψαι πρὸς μέ; φησὶ Κύριος·
\par }{\PP \VS{23}Ὅτι ὀργὴ Κυρίου ἐξῆλθε θυμώδης, ἐξῆλθεν ὀργὴ στρεφομένη, ἐπʼ ἀσεβεῖς ἥξει.
\VS{24}Οὐ μὴ ἀποστραφῇ ὀργὴ θυμοῦ Κυρίου, ἕως ποιήσει, καὶ ἕως καταστήσῃ ἐγχείρημα καρδίας αὐτοῦ· ἐπʼ ἐσχάτων τῶν ἡμερῶν γνώσεσθε αὐτά.

\par }\Chap{38}{\PP \VerseOne{1}Ἐν τῷ χρόνῳ ἐκείνῳ, εἶπε Κύριος, ἔσομαι εἰς Θεὸν τῷ γένει Ἰσραὴλ, καὶ αὐτοὶ ἔσονταί μοι εἰς λαόν.
\VS{2}Οὕτως εἶπε Κύριος, εὗρον θερμὸν ἐν ἐρήμῳ μετὰ ὀλωλότων ἐν μαχαίρᾳ· βαδίσατε, καὶ μὴ ὀλέσητε τὸν Ἰσραὴλ.
\VS{3}Κύριος πόῤῥωθεν ὤφθη αὐτῷ· ἀγάπησιν αἰώνιον ἠγάπησά σε· διατοῦτο εἵλκυσά σε εἰς οἰκτείρημα.
\VS{4}Ὅτι οἰκοδομήσω σε, καὶ οἰκοδομηθήσῃ παρθένος Ἰσραήλ· ἔπι λήψῃ τύμπανόν σου, καὶ ἐξελεύσῃ μετὰ συναγωγῆς παιζόντων.
\VS{5}Ὅτι ἐφυτεύσατε ἀμπελῶνας ἐν ὄρεσι Σαμαρείας, φυτεύσατε καὶ αἰνέσατε,
\VS{6}ὅτι ἐστὶν ἡμέρα κλήσεως ἀπολογουμένων ἐν ὄρεσιν Ἐφραὶμ, ἀνάστητε, καὶ ἀνάβητε εἰς Σιὼν πρὸς Κύριον τὸν Θεὸν ὑμῶν.
\par }{\PP \VS{7}Ὅτι οὕτως εἶπε Κύριος τῷ Ἰακὼβ, εὐφράνθητε, καὶ χρεμετίσατε ἐπὶ κεφαλὴν ἐθνῶν· ἀκουστὰ ποιήσατε, καὶ αἰνέσατε· εἴπατε, ἔσωσε Κύριος τὸν λαὸν αὐτοῦ, τὸ κατάλοιπον τοῦ Ἰσραήλ.
\VS{8}Ἰδοὺ ἐγὼ ἄγω αὐτοὺς ἀπὸ Βοῤῥᾶ, καὶ συνάξω αὐτοὺς ἀπʼ ἐσχάτου τῆς γῆς ἐν ἑορτῇ φασέκ· καὶ τεκνοποιήσει ὄχλον πολὺν, καὶ ἀποστρέψουσιν ὧδε.
\VS{9}Ἐν κλαυθμῷ ἐξῆλθον, καὶ ἐν παρακλήσει ἀνάξω αὐτοὺς, αὐλίζων ἐπὶ διώρυγας ὑδάτων ἐν ὁδῷ ὀρθῇ, καὶ οὐ μὴ πλανηθῶσιν ἐν αὐτῇ· ὅτι ἐγενόμην τῷ Ἰσραὴλ εἰς πατέρα, καὶ Ἐφραὶμ πρωτότοκός μου ἐστίν.
\par }{\PP \VS{10}Ἀκούσατε λόγους Κυρίου ἔθνη, καὶ ἀναγγείλατε εἰς νήσους τὰς μακρόθεν· εἴπατε, ὁ λικμήσας τὸν Ἰσραὴλ καὶ συνάξει αὐτὸν, καὶ φυλάξει αὐτὸν, ὡς ὁ βόσκων ποίμνιον αὐτοῦ.
\VS{11}Ὅτι ἐλυτρώσατο Κύριος τὸν Ἰακὼβ, ἐξείλατο αὐτὸν ἐκ χειρὸς στερεωτέρων αὐτοῦ.
\VS{12}Καὶ ἥξουσι, καὶ εὐφρανθήσονται ἐν τῷ ὄρει Σιὼν, καὶ ἥξουσιν ἐπʼ ἀγαθὰ Κυρίου, ἐπὶ γῆν σίτου καὶ οἴνου, καὶ καρπῶν, καὶ κτηνῶν, καὶ προβάτων· καὶ ἔσται ἡ ψυχὴ αὐτῶν ὥσπερ ξύλον ἔγκαρπον, καὶ οὐ πεινάσουσιν ἔτι.
\VS{13}Τότε χαρήσονται παρθένοι ἐν συναγωγῇ νεανίσκων, καὶ πρεσβύται χαρήσονται, καὶ στρέψω τὸ πένθος αὐτῶν εἰς χαρμονὴν, καὶ ποιήσω αὐτοὺς εὐφραινομένους.
\VS{14}Μεγαλυνῶ καὶ μεθύσω τὴν ψυχὴν τῶν ἱερεων υἱῶν Λευὶ, καὶ ὁ λαός μου τῶν ἀγαθῶν μου ἐμπλησθήσεται· οὕτως εἶπε Κύριος.
\par }{\PP \VS{15}Φωνὴ ἐν Ῥαμᾷ ἠκούσθη θρήνου, καὶ κλαυθμοῦ, καὶ ὀδυρμοῦ· Ῥαχὴλ ἀποκλαιομένη οὐκ ἤθελε παύσασθαι ἐπὶ τοῖς υἱοῖς αὐτῆς, ὅτι οὐκ εἰσίν.
\par }{\PP \VS{16}Οὕτως εἶπε Κύριος, διαλειπέτω ἡ φωνή σου ἀπὸ κλαυθμοῦ, καὶ οἱ ὀφθαλμοί σου ἀπὸ δακρύων σου, ὅτι ἔστι μισθὸς τοῖς σοῖς ἔργοις, καὶ ἐπιστρέψουσιν ἐκ γῆς ἐχθρῶν,
\VS{17}μόνιμον τοῖς σοῖς τέκνοις.
\par }{\PP \VS{18}Ἀκοὴν ἤκουσα Ἐφραὶμ ὀδυρομένου, ἐπαίδευσάς με, καὶ ἐπαιδεύθην· ἐγὼ ὥσπερ μόσχος οὐκ ἐδιδάχθην· ἐπίστρεψόν με, καὶ ἐπιστρέψω, ὅτι σὺ Κύριος ὁ Θεός μου.
\VS{19}Ὅτι ὕστερον αἰχμαλωσίας μου μετενόησα, καὶ ὕστερον τοῦ γνῶναί με, ἐστέναξα ἐφʼ ἡμέρας αἰσχύνης, καὶ ὑπέδειξά σοι, ὅτι ἔλαβον ὀνειδισμὸν ἐκ νεότητός μου.
\VS{20}Υἱὸς ἀγαπητὸς Ἐφραὶμ, ἐμοὶ παιδίον ἐντρυφῶν, ὅτι ἀνθʼ ὧν οἱ λόγοι μου ἐν αὐτῷ, μνείᾳ μνησθήσομαι αὐτοῦ· διατοῦτο ἔσπευσα ἐπʼ αὐτῷ, ἐλεῶν ἐλεήσω αὐτὸν, φησὶ Κύριος.
\par }{\PP \VS{21}Στῆσον σεαυτὴν Σιὼν, ποίησον τιμωρίαν, δὸς καρδίαν σου εἰς τοὺς ὤμους· ὁδὸν ᾗ ἐπορεύθης, ἀποστράφηθι παρθένος Ἰσραὴλ, ἀποστράφητι εἰς τὰς πόλεις σου πενθοῦσα.
\VS{22}Ἕως πότε ἀποστρέψεις θυγάτηρ ἠτιμωμένη; ὅτι ἔκτισε Κύριος σωτηρίαν εἰς καταφύτευσιν καινὴν, ἐν σωτηρίᾳ περιελεύσονται ἄνθρωποι.
\par }{\PP \VS{23}Ὅτι οὕτως εἶπε Κύριος, ἔτι ἐροῦσι τὸν λόγον τοῦτον ἐν γῇ Ἰούδα, καὶ ἐν πόλεσιν αὐτοῦ, ὅταν ἀποστρέψω τὴν αἰχμαλωσίαν αὐτοῦ, εὐλογημένος Κύριος ἐπὶ δίκαιον ὄρος τὸ ἅγιον αὐτοῦ·
\VS{24}Καὶ ἐνοικοῦντες ἐν ταῖς πόλεσιν Ἰούδα, καὶ ἐν πάσῃ τῇ γῇ αὐτοῦ, ἅμα γεωργῷ, καὶ ἀρθήσεται ἐν ποιμνίῳ.
\VS{25}Ὅτι ἐμέθυσα πᾶσαν ψυχὴν διψῶσαν, καὶ πᾶσαν ψυχὴν πεινῶσαν ἐνέπλησα.
\VS{26}Διατοῦτο ἐξηγέρθην, καὶ εἶδον, καὶ ὁ ὕπνος μου ἡδύς μοι ἐγενήθη.
\par }{\PP \VS{27}Διατοῦτο ἰδοὺ ἡμέραι ἔρχονται, φησὶ Κύριος, καὶ σπερῶ τὸν Ἰσραὴλ καὶ τὸν Ἰούδαν, σπέρμα ἀνθρώπου καὶ σπέρμα κτήνους.
\VS{28}Καὶ ἔσται ὥσπερ ἐγρηγόρουν ἐπʼ αὐτοὺς καθαιρεῖν καὶ κακοῦν, οὕτως γρηγορήσω ἐπʼ αὐτοὺς τοῦ οἰκοδομεῖν καὶ καταφυτεύειν, φησὶ Κύριος.
\VS{29}Ἐν ταῖς ἡμέραις ἐκείναις οὐ μὴ εἴπωσιν, οἱ πατέρες ἔφαγον ὄμφακα, καὶ οἱ ὀδόντες τῶν τέκνων ᾑμωδίασαν.
\VS{30}Ἀλλʼ ἢ ἕκαστος ἐν τῇ ἑαυτοῦ ἁμαρτίᾳ ἀποθανεῖται, καὶ τοῦ φαγόντος τὸν ὄμφακα αἱμωδιάσουσιν οἱ ὀδόντες αὐτοῦ.
\par }{\PP \VS{31}Ἰδοὺ ἡμέραι ἔρχονται, φησὶ Κύριος, καὶ διαθήσομαι τῷ οἴκῳ Ἰσραὴλ καὶ τῷ οἴκῳ Ἰούδα διαθήκην καινὴν,
\VS{32}οὐ κατὰ τὴν διαθήκην ἣν διεθέμην τοῖς πατράσιν αὐτῶν, ἐν ἡμέρᾳ ἐπιλαβομένου μου τῆς χειρὸς αὐτῶν, ἐξαγαγεῖν αὐτοὺς ἐκ γῆς Αἰγύπτου, ὅτι αὐτοὶ οὐκ ἐνέμειναν ἐν τῇ διαθήκῃ μου, καὶ ἐγὼ ἠμέλησα αὐτῶν, φησὶ Κύριος.
\VS{33}Ὅτι αὕτη ἡ διαθήκη μου, ἣν διαθήσομαι τῷ οἴκῳ Ἰσραὴλ, μετὰ τὰς ἡμέρας ἐκείνας, φησὶ Κύριος, διδοὺς δώσω νόμους μου εἰς τὴν διάνοιαν αὐτῶν, καὶ ἐπὶ καρδίας αὐτῶν γράψω αὐτοὺς, καὶ ἔσομαι αὐτοῖς εἰς Θεὸν, καὶ αὐτοὶ ἔσονταί μοι εἰς λαόν.
\VS{34}Καὶ οὐ μὴ διδάξωσιν ἕκαστος τὸν πολίτην αὐτοῦ, καὶ ἕκαστος τὸν ἀδελφὸν αὐτοῦ, λέγων, γνῶθι τὸν Κύριον· ὅτι πάντες εἰδήσουσί με ἀπὸ μικροῦ αὐτῶν ἕως μεγάλου αὐτῶν, ὅτι ἵλεως ἔσομαι ταῖς ἀδικίαις αὐτῶν, καὶ τῶν ἁμαρτιῶν αὐτῶν οὐ μὴ μνησθῶ ἔτι.
\par }{\PP \VS{35}Ἐὰν ὑψωθῇ ὁ οὐρανὸς εἰς τὸ μετέωρον, φησὶ Κύριος, καὶ ἐὰν ταπεινωθῇ τὸ ἔδαφος τῆς γῆς κάτω, καὶ ἐγὼ οὐκ ἀποδοκιμῶ τὸ γένος Ἰσραὴλ, φησὶ Κύριος, περὶ πάντων ὧν ἐποίησαν.
\par }{\PP \VS{36}Οὕτως εἶπε Κύριος, ὁ δοὺς τὸν ἥλιον εἱς φῶς τῆς ἡμέρας, σελήνην καὶ ἀστέρας εἰς φῶς τῆς νυκτὸς, καὶ κραυγὴν ἐν θαλάσσῃ, καὶ ἐβόμβησε τὰ κύματα αὐτῆς, Κύριος παντοκράτωρ ὄνομα αὐτῷ·
\VS{37}Ἐὰν παύσωναι οἱ νόμοι οὗτοι ἀπὸ προσώπου μου, φησὶ Κύριος, καὶ τὸ γένος Ἰσραὴλ παύσεται γενέσθαι ἔθνος κατὰ πρόσωπόν μου πάσας τὰς ἡμέρας.
\par }{\PP \VS{38}Ἰδοὺ ἡμέραι ἔρχονται, φησὶ Κύριος, καὶ οἰκοδομηθήσεται πόλις τῷ Κυρίῳ ἀπὸ πύργου Ἁναμεὴλ, ἕως πύλης τῆς γωνίας.
\VS{39}Καὶ ἐξελεύσεται ἡ διαμέτρησις αὐτῆς ἀπέναντι αὐτῶν ἕως βουνῶν Γαρὴβ, καὶ περικυκλωθήσεται κύκλῳ ἐξ ἐκλεκτῶν λίθων,
\VS{40}καὶ πάντες Ἀσαρημὼθ ἕως Νάχαλ Κέδρων, ἕως γωνίας πύλης ἵππων ἀνατολῆς, ἁγίασμα τῷ Κυρίῳ, καὶ οὐκέτι οὐ μὴ ἐκλίπῃ, καὶ οὐ μὴ καθαιρεθῇ ἕως τοῦ αἰῶνος.

\par }\Chap{39}{\PP \VerseOne{1}Ὁ ΛΟΓΟΣ ὁ γενόμενος παρὰ Κυρίου πρὸς Ἱερεμίαν ἐν τῷ ἐνιαυτῷ δεκάτῳ βασιλεῖ Σεδεκίᾳ, οὗτος ἐνιαυτὸς ὀκτωκαιδέκατος τῷ βασιλεῖ Ναβουχοδονόσορ βασιλεῖ Βαβυλῶνος.
\par }{\PP \VS{2}Καὶ δύναμις βασιλέως Βαβυλῶνος ἐχαράκωσεν ἐπὶ Ἱερουσαλὴμ, καὶ Ἱερεμίας ἐφυλάσσετο ἐν αὐλῇ τῆς φυλακῆς, ἥ ἐστιν ἐν οἴκῳ βασιλέως,
\VS{3}ἐν ᾗ κατέκλεισεν αὐτὸν ὁ βασιλεὺς Σεδεκίας, λέγων, διατί σὺ προφητεύεις, λέγων, οὕτως εἶπε Κύριος, ἰδοὺ ἐγὼ δίδωμι τὴν πόλιν ταύτην ἐν χερσὶ βασιλέως Βαβυλῶνος, καὶ λήψεται αὐτὴν,
\VS{4}καὶ Σεδεκίας οὐ μὴ σωθῇ ἐκ χειρὸς τῶν Χαλδαίων, ὅτι παραδόσει παραδοθήσεται εἰς χεῖρας βασιλέως Βαβυλῶνος, καὶ λαλήσει στόμα αὐτοῦ πρὸς στόμα αὐτοῦ, καὶ οἱ ὀφθαλμοὶ αὐτοῦ τοὺς ὀφθαλμοὺς αὐτοῦ ὄψονται·
\VS{5}Καὶ εἰσελεύσεται Σεδεκίας εἰς Βαβυλῶνα, καὶ ἐκεῖ καθίεται;
\par }{\PP \VS{6}ΚΑΙ Ὁ ΛΟΓΟΣ ΚΥΡΙΟΥ ἘΓΕΝΗΘΗ ΠΡΟΣ ἹΕΡΕΜΙΑΝ, ΛΕΓΩΝ,
\VS{7}Ἰδοὺ Ἀναμεὴλ υἱὸς Σαλῶμ ἀδελφοῦ πατρός σου ἔρχεται πρὸς σὲ, λέγων, κτῆσαι σεαυτῷ τὸν ἀγρόν μου τὸν ἐν Ἀναθὼθ, ὅτι σοὶ κρίσις παραλαβεῖν εἰς κτῆσιν.
\VS{8}Καὶ ἦλθε πρὸς μὲ Ἀναμεὴλ υἱὸς Σαλὼμ, ἀδελφοῦ πατρός μου, εἰς τὴν αὐλὴν τῆς φυλακῆς, καὶ εἶπε, κτῆσαι σεαυτῷ τὸν ἀγρόν μου τὸν ἐν γῇ Βενιαμὶν τὸν ἐν Ἀναθὼθ, ὅτι σοὶ κρίμα κτήσασθαι αὐτὸν, καὶ σὺ πρεσβύτερος. καὶ ἔγνων, ὅτι λόγος Κυρίου ἐστὶ,
\VS{9}καὶ ἐκτησάμην τὸν ἀγρὸν Ἀναμεὴλ υἱοῦ ἀδελφοῦ πατρός μου, καὶ ἔστησα αὐτῷ ἑπτὰ σίκλους καὶ δέκα ἀργυρίου,
\VS{10}καὶ ἔγραψα εἰς βιβλίον, καὶ ἐσφραγισάμην, καὶ διεμαρτυράμην μάρτυρας, καὶ ἔστησα τὸ ἀργύριον ἐν ζυγῷ.
\VS{11}Καὶ ἔλαβον τὸ βιβλίον τῆς κτήσεως τὸ ἐσφραγισμένον,
\VS{12}καὶ ἔδωκα αὐτὸ τῷ Βαροὺχ υἱῷ Νηρίου υἱῷ Μαασαίου, κατʼ ὀφθαλμοὺς Ἀναμεὴλ υἱοῦ ἀδελφοῦ πατρός μου, καὶ κατʼ ὀφθαλμοὺς τῶν ἀνδρῶν τῶν παρεστηκότων καὶ γραφόντων ἐν τῷ βιβλίῳ τῆς κτήσεως, καὶ κατʼ ὀφθαλμοὺς τῶν Ἰουδαίων τῶν ἐν τῇ αὐλῇ τῆς φυλακῆς.
\VS{13}Καὶ συνέταξα τῷ Βαροὺχ κατʼ ὀφθαλμοὺς αὐτῶν, λέγων,
\VS{14}Οὕτως εἶπε Κύριος παντοκράτωρ, λάβε τὸ βιβλίον τῆς κτήσεως τοῦτο, καὶ τὸ βιβλίον τὸ ἀνεγνωσμένον, καὶ θήσεις αὐτὸ εἰς ἀγγεῖον ὀστράκινον, ἵνα διαμείνῃ ἡμέρας πλείους.
\VS{15}Ὅτι οὕτως εἶπε Κύριος, ἔτι κτιηθήσονται ἀγροὶ, καὶ οἰκίαι, καὶ ἀμπελῶνες ἐν τῇ γῇ ταύτῃ.
\par }{\PP \VS{16}Καὶ προσευξάμην πρὸς Κύριον μετὰ τὸ δοῦναί με τὸ βιβλίον τῆς κτήσεως πρὸς Βαροὺχ υἱὸν Νηρίου, λέγων,
\par }{\PP \VS{17}Ὁ ὢν Κύριε, σὺ ἐποίησας τὸν οὐρανὸν καὶ τὴν γῆν τῇ ἰσχύϊ σου τῇ μεγάλῃ, καὶ τῷ βραχίονί σου τῷ ὑψηλῷ καὶ τῷ μετεώρῳ, οὐ μὴ ἀποκρυβῇ ἀπὸ σοῦ οὐθὲν,
\VS{18}ποιῶν ἔλεος εἰς χιλιάδας, καὶ ἀποδιδοὺς ἁμαρτίας πατέρων εἰς κόλπους τέκνων αὐτῶν μετʼ αὐτούς· ὁ Θεὸς ὁ μέγας, ὁ ἰσχυρὸς,
\VS{19}Κύριος μεγάλης βουλῆς, καὶ δυνατὸς τοῖς ἔργοις, ὁ Θεὸς ὁ μέγας ὁ παντοκράτωρ, καὶ μεγαλώνυμος Κύριος· οἱ ὀφθαλμοί σου εἰς τὰς ὁδοὺς τῶν υἱῶν τῶν ἀνθρώπων, δοῦναι ἑκάστῳ κατὰ τὴν ὁδὸν αὐτοῦ·
\VS{20}Ὃς ἐποίησας σημεῖα καὶ τέρατα ἐν γῇ Αἰγύπτῳ ἕως τῆς ἡμέρας ταύτης, καὶ ἐν Ἰσραὴλ, καὶ ἐν τοῖς γηγενέσι· καὶ ἐποίησας σεαυτῷ ὄνομα, ὡς ἡμέρα αὕτη,
\VS{21}καὶ ἐξήγαγες τὸν λαόν σου Ἰσραὴλ ἐκ γῆς Αἰγύπτου ἐν σημείοις καὶ ἐν τέρασιν, ἐν χειρὶ κραταιᾷ,
\VS{22}καὶ ἐν βραχίονι ὑψηλῷ, καὶ ἐν ὁράμασι μεγάλοις, καὶ ἔδωκας αὐτοῖς τὴν γῆν ταύτην, ἣν ὤμοσας τοῖς πατράσιν αὐτῶν, γῆν ῥέουσαν γάλα καὶ μέλι.
\VS{23}Καὶ εἰσήλθοσαν καὶ ἔλαβον αὐτὴν, καὶ οὐκ ἤκουσαν τῆς φωνῆς σου, καὶ ἐν τοῖς προστάγμασί σου οὐκ ἐπορεύθησαν· ἅπαντα ἃ ἐνετείλω αὐτοῖς οὐκ ἐποίησαν, καὶ ἐποὶησαν συμβῆναι αὐτοῖς πάντα τὰ κακὰ ταῦτα.
\VS{24}Ἰδοὺ ὄχλος ἥκει εἰς τὴν πόλιν συλλαβεῖν αὐτὴν, καὶ ἡ πόλις ἐδόθη εἰς χεῖρας Χαλδαίων τῶν πολεμούντων αὐτὴν ἀπὸ προσώπου μαχαίρας, καὶ τοῦ λιμοῦ ὡς ἐλάλησας, οὕτως ἐγένετο.
\VS{25}Καὶ σὺ λέγεις πρὸς μέ κτῆσαι σεαυτῷ τὸν ἀγρὸν ἀργυρίου· καὶ ἔγραψα βιβλίον, καὶ ἐσφραγισάμην, καὶ ἐπεμαρτυράμην μάρτυρας, καὶ ἡ πόλις ἐδόθη εἰς χεῖρας Χαλδαίων.
\par }{\PP \VS{26}Καὶ ἐγένετο λόγος Κυρίου πρὸς μὲ, λέγων,
\par }{\PP \VS{27}Ἐγὼ Κύριος ὁ Θεὸς πάσης σαρκός, μὴ ἀπʼ ἐμοῦ κρυβήσεταί τι;
\VS{28}Διατοῦτο οὕτως εἶπε Κύριος ὁ Θεὸς Ἰσραὴλ, δοθεῖσα παραδοθήσεται ἡ πόλις αὕτη εἰς χεῖρας βασιλέως Βαβυλῶνος, καὶ λήψεται αὐτὴν,
\VS{29}καὶ ἥξουσιν οἱ Χαλδαῖοι πολεμοῦντες ἐπὶ τὴν πόλιν ταύτην, καὶ καύσουσι τὴν πόλιν ταύτην ἐν πυρί, καὶ κατακαύσουσι τὰς οἰκίας ἐν αἷς ἐθυμιῶσαν ἐπὶ τῶν δωμάτων αὐτῶν τῇ Βάαλ, καὶ ἔσπενδον σπονδὰς θεοῖς ἑτέροις, πρὸς τὸ παραπικράναι με·
\VS{30}Ὅτι ἦσαν οἱ υἱοὶ Ἰσραὴλ καὶ οἱ υἱοὶ Ἰούδα μόνοι ποιοῦντες τὸ πονηρὸν κατʼ ὀφθαλμούς μου ἐκ νεὁτητος αὐτῶν·
\VS{31}Ὅτι ἐπὶ τὴν ὀργήν μου, καὶ ἐπὶ τὸν θυμόν μου ἦν ἡ πόλις αὕτη, ἀφʼ ἧς ἡμέρας ᾠκοδόμησαν αὐτὴν καὶ ἕως τῆς ἡμέρας ταύτης, ἀπαλλάξαι αὐτὴν ἀπὸ προσώπου μου,
\VS{32}διὰ πάσας τὰς πονηρίας τῶν υἱῶν Ἰσραὴλ καὶ Ἰούδα, ὧν ἐποίησαν πικράναι με, αὐτοὶ καὶ οἱ βασιλεῖς αὐτῶν, καὶ οἱ ἄρχοντες αὐτῶν, καὶ οἱ ἱερεῖς αὐτῶν, καὶ οἱ προφῆται αὐτῶν, ἄνδρες Ἰούδα, καὶ οἱ κατοικοῦντες ἐν Ἱερουσαλὴμ,
\VS{33}καὶ ἀπέστρεψαν πρὸς μὲ νῶτον, καὶ οὐ πρόσωπον· καὶ ἐδίδαξα αὐτοὺς ὄρθρου, καὶ οὐκ ἤκουσαν ἔτι λαβεῖν παιδείαν.
\VS{34}Καὶ ἔθηκαν τὰ μιάσματα αὐτῶν ἐν τῷ οἴκῳ, οὗ ἐπεκλήθη τὸ ὄνομά μου ἐπʼ αὐτῷ, ἐν ἀκαθαρσίαις αὐτῶν.
\VS{35}Καὶ ᾠκοδόμησαν τοὺς βωμοὺς τῇ Βάαλ τοὺς ἐν φάραγγι υἱοὺ Ἑννὸμ, τοῦ ἀναφέρειν τοὺς υἱοὺς αὐτῶν καὶ τὰς θυγατέρας αὐτῶν τῷ Μολὸχ βασιλεῖ, ἃ οὐ συνέταξα αὐτοῖς, καὶ οὐκ ἀνέβη ἐπὶ καρδίαν μου τοῦ ποιῆσαι τὸ βδέλυγμα τοῦτο, πρὸς τὸ ἐφαμαρτεῖν τὸν Ἰούδαν.
\par }{\PP \VS{36}Καὶ νῦν οὕτως εἶπε Κύριος ὁ Θεὸς Ἰσραὴλ ἐπὶ τὴν πόλιν, ἣν σὺ λέγεις, παραδοθήσεται εἰς χεῖρας βασιλέως Βαβυλῶνος ἐν μαχαίρᾳ, καὶ ἐν λιμῷ καὶ ἐν ἀποστολῇ·
\VS{37}Ἰδοὺ ἐγὼ συνάγω αὐτοὺς ἐκ πάσης τῆς γῆς, οὗ διέσπειρα αὐτοὺς ἐκεῖ ἐν ὀργῇ μου, καὶ τῷ θυμῷ μου, καὶ ἐν παροξυσμῷ μεγάλῳ· καὶ ἐπιστρέψω αὐτοὺς εἰς τὸν τόπον τοῦτον, καὶ καθιῶ αὐτοὺς πεποιθότας·
\VS{38}Καὶ ἔσονταί μοι εἰς λαὸν, καὶ ἐγὼ ἔσομαι αὐτοῖς εἰς Θεόν.
\VS{39}Καὶ δώσω αὐτοῖς ὁδὸν ἑτέραν καὶ καρδίαν ἑτέραν, φοβηθῆναί με πάσας τὰς ἡμέρας, καὶ εἰς ἀγαθὸν αὐτοῖς καὶ τοῖς τέκνοις αὐτῶν μετʼ αὐτούς.
\VS{40}Καὶ διαθήσομαι αὐτοῖς διαθήκην αἰωνίαν, ἣν οὐ μὴ ἀποστρέψω ὄπισθεν αὐτῶν· καὶ τὸν φόβον μου δώσω εἰς τὴν καρδίαν αὐτῶν, πρὸς τὸ μὴ ἀποστῆναι αὐτοὺς ἀπʼ ἐμοῦ.
\VS{41}Καὶ ἐπισκέψομαι τοῦ ἀγαθῶσαι αὐτοὺς, καὶ φυτεύσω αὐτοὺς ἐν τῇ γῇ ταύτῃ ἐν πίστει, καὶ ἐν πάσῃ καρδίᾳ μου, καὶ ἐν πάσῃ ψυχῇ.
\par }{\PP \VS{42}Ὅτι οὕτως εἶπε Κύριος, καθὰ ἐπήγαγον ἐπὶ τὸν λαὸν τοῦτον πάντα τὰ κακὰ τὰ μεγάλα ταῦτα, οὕτως ἐγὼ ἐπάξω ἐπʼ αὐτοὺς πάντα τὰ ἀγαθὰ, ἃ ἐλάλησα ἐπʼ αὐτούς.
\VS{43}Καὶ κτηθήσονται ἔτι ἀγροὶ ἐν τῇ γῇ, ᾗ σὺ λέγεις, ἄβατος ἔσται ἀπὸ ἀνθρώπων καὶ κτήνους, καὶ παρεδόθησαν εἰς χεῖρας Χαλδαίων.
\VS{44}Καὶ κτήσονται ἀγροὺς ἐν ἀργυρίῳ· καὶ γράψεις βιβλίον καὶ σφραγιῇ, καὶ διαμαρτύρῃ μάρτυρας ἐν γῇ Βενιαμὶν, καὶ κύκλῳ τῆς Ἱερουσαλὴμ, καὶ ἐν πόλεσιν Ἰούδα, καὶ ἐν πόλεσι τοῦ ὄρους, καὶ ἐν πόλεσι τῆς σεφηλὰ, καὶ ἐν πόλεσι τῆς ναγὲβ, ὅτι ἀποστρέψω τὰς ἀποικίας αὐτῶν.

\par }\Chap{40}{\PP \VerseOne{1}Καὶ ἐγένετο λόγος Κυρίου πρὸς Ἱερεμίαν δεύτερον, καὶ αὐτὸς ἦν ἔτι δεδεμένος ἐν τῇ αὐλῇ τῆς φυλακῆς, λέγων,
\par }{\PP \VS{2}Οὕτως εἶπε Κύριος, ποιῶν γῆν, καὶ πλάσσων αὐτὴν, τοῦ ἀνορθῶσαι αὐτὴν, Κύριος ὄνομα αὐτῷ·
\VS{3}Κέκραξον πρὸς μὲ, καὶ ἀποκριθήσομαί σοι, καὶ ἀπαγγελῶ σοι μεγάλα καὶ ἰσχυρὰ, ἃ οὐκ ἔγνως αὐτά.
\VS{4}Ὅτι οὕτως εἶπε Κύριος περὶ οἴκων τῆς πόλεως ταύτης, καὶ περὶ οἴκων βασιλεως Ἰούδα τῶν καθῃρημένων εἰς χάρακας καὶ προμαχῶνας,
\VS{5}τοῦ μάχεσθαι πρὸς τοὺς Χαλδαίους, καὶ πληρῶσαι αὐτὴν τῶν νεκρῶν τῶν ἀνθρώπων, οὓς ἐπάταξα ἐν ὀργῇ μου, καὶ ἐν θυμῷ μου· καὶ ἀπέστρεψα τὸ πρόσωπόν μου ἀπʼ αὐτῶν, περὶ πασῶν τῶν πονηριῶν αὐτῶν.
\VS{6}Ἰδοὺ ἐγὼ ἀνάγω αὐτῇ συνούλωσιν καὶ ἴαμα, καὶ φανερώσω αὐτοῖς, καὶ ἰατρεύσω αὐτὴν, καὶ ποιήσω καὶ εἰρήνην καὶ πίστιν.
\par }{\PP \VS{7}Καὶ ἀποστρέψω τὴν ἀποικίαν Ἰούδα, καὶ ἀποικίαν Ἰσραὴλ, καὶ οἰκοδομήσω αὐτοὺς καθὼς καὶ τοπρότερον.
\VS{8}Καὶ καθαριῶ αὐτοὺς ἀπὸ πασῶν τῶν ἀδικιῶν αὐτῶν, ὧν ἡμάρτοσάν μοι, καὶ οὐ μὴ μνησθήσομαι ἁμαρτιῶν αὐτῶν, ὧν ἥμαρτόν μοι, καὶ ἀπέστησαν ἀπʼ ἐμοῦ.
\VS{9}Καὶ ἔσται εἰς εὐφροσύνην καὶ αἴνεσιν, καὶ εἰς μεγαλειότητα παντὶ τῷ λαῷ τῆς γῆς, οἵτινες ἀκούσονται πάντα τὰ ἀγαθὰ ἃ ἐγὼ ποιήσω, καὶ φοβηθήσονται καὶ πικρανθήσονται περὶ πάντων τῶν ἀγαθῶν, καὶ περὶ πάσης τῆς εἰρήνης ἧς ἐγὼ ποιήσω αὐτοῖς.
\par }{\PP \VS{10}Οὕτως εἶπε Κύριος, ἔτι ἀκουσθήσεται ἐν τῷ τόπῳ τούτῳ, ᾧ ὑμεῖς λέγετε, ἔρημός ἐστιν ἀπὸ ἀνθρώπων καὶ κτηνῶν, ἐν πόλεσιν Ἰούδα, καὶ ἔξωθεν Ἱερουσαλὴμ, ταῖς ἠρημωμέναις, παρὰ τὸ μὴ εἶναι ἄνθρωπον καὶ κτήνη·
\VS{11}Φωνὴ εὐφροσύνης, καὶ φωνὴ χαρμοσύνης, φωνὴ νυμφίου, καὶ φωνὴ νύμφης, φωνὴ λεγόντων, ἐξομολογεῖσθε Κυρίῳ παντοκράτορι, ὅτι χρηστὸς Κύριος, ὅτι εἰς τὸν αἰῶνα τὸ ἔλεος αὐτοῦ· καὶ εἰσοίσουσι δῶρα εἰς οἶκον Κυρίου, ὅτι ἀποστρέψω πᾶσαν τὴν ἀποικίαν τῆς γῆς ἐκείνης κατὰ τοπρότερον, εἶπε Κύριος.
\VS{12}Οὕτως εἶπε Κύριος τῶν δυνάμεων, ἔτι ἔσται ἐν τῷ τόπῳ τούτῳ τῷ ἐρήμῳ, παρὰ τὸ μὴ εἶναι ἄνθρωπον καὶ κτῆνος, ἐν πάσαις ταῖς πόλεσιν αὐτοῦ καταλύματα ποιμένων κοιταζόντων πρόβατα,
\VS{13}ἐν πόλεσι τῆς ὀρεινῆς, καὶ ἐν πόλεσι τῆς σεφηλὰ, καὶ ἐν πόλεσι τῆς ναγὲβ, καὶ ἐν γῇ Βενιαμὶν, καὶ ἐν ταῖς κύκλῳ Ἱερουσαλὴμ, καὶ ἐν πόλεσιν Ἰούδα· ἔτι παρελεύσεται πρόβατα ἐπὶ χεῖρα ἀριθμοῦντος, εἶπε Κύριος.

\par }\Chap{41}{\PP \VerseOne{1}Ὁ ΛΟΓΟΣ ὁ γενόμενος πρὸς Ἱερεμίαν παρὰ Κυρίου, (καὶ Ναβουχοδονόσορ βασιλεὺς Βαβυλῶνος, καὶ πᾶν τὸ στρατόπεδον αὐτοῦ, καὶ πᾶσα ἡ γῆ ἀρχῆς αὐτοῦ ἐπολέμουν ἐπὶ Ἱερουσαλὴμ, καὶ ἐπὶ πάσας τὰς πόλεις Ἰούδα,) λέγων,
\par }{\PP \VS{2}Οὕτως εἶπε Κύριος, βάδισον πρὸς Σεδεκίαν βασιλέα Ἰούδα, καὶ ἐρεῖς αὐτῷ, οὕτως εἶπε Κύριος, παραδόσει παραδοθήσεται ἡ πόλις αὕτη εἰς χεῖρας βασιλέως Βαβυλῶνος, καὶ συλλήψεται αὐτὴν, καὶ καύσει αὐτὴν ἐν πυρὶ,
\VS{3}καὶ σὺ οὐ μὴ σωθῇς ἐκ χειρὸς αὐτοῦ, καὶ συλλήψει συλληφθήσῃ, καὶ εἰς χεῖρας αὐτοῦ δοθήσῃ, καὶ ὀφθαλμοί σου τοὺς ὀφθαλμοὺς αὐτοῦ ὄψονται, καὶ εἰς Βαβυλῶνα εἰσελεύσῃ.
\par }{\PP \VS{4}Ἀλλʼ ἄκουσον τὸν λόγον Κυρίου, Σεδεκία βασιλεῦ Ἰούδα· οὕτως λέγει Κύριος,
\VS{5}ἐν εἰρήνῃ ἀποθανῇ· καὶ ὡς ἔκλαυσαν τοὺς πατέρας σου τοὺς βασιλεύσαντας πρότερόν σου, κλαύσονται καὶ σὲ, οὐαὶ Κύριε, καὶ ἕως ᾅδου κόψονταί σε, ὅτι λόγον ἐγὼ ἐλάλησα, εἶπε Κύριος.
\par }{\PP \VS{6}Καὶ ἐλάλησεν Ἱερεμίας πρὸς τὸν βασιλέα Σεδεκίαν πάντας τοὺς λόγους τούτους ἐν Ἱερουσαλήμ.
\VS{7}Καὶ ἡ δύναμις βασιλέως Βαβυλῶνος ἐπολέμει ἐπὶ Ἱερουσαλὴμ, καὶ ἐπὶ τὰς πόλεις Ἰούδα, καὶ ἐπὶ Λαχὶς, καὶ ἐπὶ Ἄζηκα, ὅτι αὗται κατελείφθησαν ἐν πόλεσιν Ἰούδα πόλεις ὀχυραί.
\par }{\PP \VS{8}Ὁ λόγος ὁ γενόμενος πρὸς Ἱερεμίαν παρὰ Κυρίου, μετὰ τὸ συντελέσαι τὸν βασιλέα Σεδεκίαν διαθήκην πρὸς τὸν λαὸν, τοῦ καλέσαι ἄφεσιν,
\VS{9}τοῦ ἐξαποστεῖλαι ἕκαστον τὸν παῖδα αὐτοῦ, καὶ ἕκαστον τὴν παιδίσκην αὐτοῦ, τὸν Ἑβραῖον καὶ τὴν Ἑβραίαν ἐλευθέρους, πρὸς τὸ μὴ δουλεύειν ἄνδρα ἐξ Ἰούδα.
\VS{10}Καὶ ἐπεστράφησαν πάντες οἱ μεγιστᾶνες, καὶ πᾶς ὁ λαὸς οἱ εἰσελθόντες ἐν τῇ διαθήκῃ, τοῦ ἀποστεῖλαι ἕκαστον τὸν παῖδα αὐτοῦ, καὶ ἕκαστον τὴν παιδίσκην αὐτοῦ,
\VS{11}καὶ ἐῶσαν αὐτοὺς εἰς παῖδας καὶ παιδίσκας.
\par }{\PP \VS{12}Καὶ ἐγενήθη λόγος Κυρίου πρὸς Ἱερεμίαν, λέγων,
\VS{13}Οὕτως εἶπε Κύριος, ἐγὼ διεθέμην διαθήκην πρὸς τοὺς πατέρας ὑμῶν ἐν τῇ ἡμέρᾳ ᾗ ἐξειλάμην αὐτοὺς ἐκ γῆς Αἰγύπτου, ἐκ οἴκου δουλείας, λέγων,
\VS{14}ὅταν πληρωθῇ ἓξ ἔτη, ἀποστελεῖς τὸν ἀδελφόν σου τὸν Ἑβραῖον, ὃς πραθήσεταί σοι, καὶ ἐργᾶταί σοι ἓξ ἔτη, καὶ ἐξαποστελεῖς αὐτὸν ἐλεύθερον· καὶ οὐκ ἤκουσάν μου, καὶ οὐκ ἔκλιναν τὸ οὖς αὐτῶν.
\VS{15}Καὶ ἐπέστρεψαν σήμερον ποιῆσαι τὸ εὐθὲς πρὸ ὀφθαλμῶν μου, τοῦ καλέσαι ἄφεσιν ἕκαστον τοῦ πλησίον αὐτοῦ· καὶ συνετέλεσαν διαθήκην κατὰ πρόσωπόν μου, ἐν τῷ οἴκῳ οὗ ἐπεκλήθη τὸ ὄνομά μου ἐπʼ αὐτῷ.
\VS{16}Καὶ ἐπεστρέψατε, καὶ ἐβεβηλώσατε τὸ ὄνομά μου, τοῦ ἐπιστρέψαι ἕκαστον τὸν παῖδα αὐτοῦ, καὶ ἕκαστον τὴν παιδίσκην αὐτοῦ, οὓς ἐξαπεστείλατε ἐλευθέρους τῇ ψυχῇ αὐτῶν, τοῦ εἶναι ὑμῖν εἰς παῖδας καὶ παιδίσκας.
\par }{\PP \VS{17}Διατοῦτο οὕτως εἶπε Κύριος, ὑμεῖς οὐκ ἠκούσατέ μου, τοῦ καλέσαι ἄφεσιν ἕκαστος πρὸς τὸν πλησίον αὐτοῦ· ἰδοὺ ἐγὼ καλῶ ἄφεσιν ὑμῖν εἰς μάχαιραν, καὶ εἰς τὸν θάνατον, καὶ εἰς τὸν λιμὸν, καὶ δώσω ὑμᾶς εἰς διασπορὰν πάσαις ταῖς βασιλείαις τῆς γῆς·
\VS{18}Καὶ δώσω τοὺς ἄνδρας τοὺς παρεληλυθότας τὴν διαθήκην μου, τοὺς μὴ στήσαντας τὴν διαθήκην μου, ἣν ἐποίησαν κατὰ πρόσωπόν μου, τὸν μόσχον ὃν ἐποίησαν ἐργάζεσθαι αὐτῷ,
\VS{19}τοὺς ἄρχοντας Ἰούδα, καὶ τοὺς δυνάστας, καὶ τοὺς ἱερεῖς, καὶ τὸν λαόν·
\VS{20}Καὶ δώσω αὐτοὺς τοῖς ἐχθροῖς αὐτῶν, καὶ ἔσται τὰ θνησιμαῖα αὐτῶν βρῶσις τοῖς πετεινοῖς τοῦ οὐρανοῦ καὶ τοῖς θηρίοις τῆς γῆς.
\VS{21}Καὶ τὸν Σεδεκίαν βασιλέα τῆς Ἰουδαίας, καὶ τοὺς ἄρχοντας αὐτῶν δώσω εἰς χεῖρας ἐχθρῶν αὐτῶν, καὶ δύναμις βασιλέως Βαβυλῶνος τοῖς ἀποτρέχουσιν ἀπʼ αὐτῶν.
\VS{22}Ἰδοὺ ἐγὼ συντάσσω, φησὶ Κύριος, καὶ ἐπιστρέψω αὐτοὺς εἰς τὴν γῆν ταύτην, καὶ πολεμήσουσιν ἐπʼ αὐτὴν, καὶ λήψονται αὐτὴν, καὶ κατακαύσουσιν αὐτὴν ἐν πυρὶ, καὶ τὰς πόλεις Ἰούδα, καὶ δώσω αὐτὰς ἐρήμους ἀπὸ τῶν κατοικούντων.

\par }\Chap{42}{\PP \VerseOne{1}Ὁ ΛΟΓΟΣ Ὁ ΓΕΝΟΜΕΝΟΣ ΠΡΟΣ ἹΕΡΕΜΙΑΝ παρὰ Κυρίου ἐν ἡμέραις Ἰωακεὶμ βασιλέως Ἰούδα, λέγων,
\VS{2}Βάδισον εἰς οἶκον Ἀρχαβεὶν, καὶ ἄξεις αὐτοὺς εἰς οἶκον Κυρίου, εἰς μίαν τῶν αὐλῶν, καὶ ποτιεῖς αὐτοὺς οἶνον.
\par }{\PP \VS{3}Καὶ ἐξήγαγον τὸν Ἰεχονιαν υἱὸν Ἱερεμὶν υἱοῦ Χαβασὶν, καὶ τοὺς ἀδελφοὺς αὐτοῦ, καὶ τοὺς υἱοὺς αὐτοῦ, καὶ πᾶσαν τὴν οἰκίαν Ἀρχαβεὶν,
\VS{4}καὶ εἰσήγαγον αὐτοὺς εἰς οἶκον Κυρίου εἰς τὸ παστοφόριον υἱῶν Ἰωνὰν, υἱοῦ Ἀνανίου, υἱοῦ Γοδολίου ἀνθρώπου τοῦ Θεοῦ, ὅς ἐστιν ἐγγὺς τοῦ οἴκου τῶν ἀρχόντων τῶν ἐπάνω τοῦ οἴκου Μαασαίου υἱοῦ Σελὼμ, τοῦ φυλάσσοντος τὴν αὐλήν.
\VS{5}Καὶ ἔδωκα κατὰ πρόσωπον αὐτῶν κεράμιον οἴνου, καὶ ποτήρια, καὶ εἶπα, πίετε οἴνον.
\par }{\PP \VS{6}Καὶ εἶπον, οὐ μὴ πίωμεν οἶνον, ὅτι Ἰωναδὰβ υἱὸς Ῥηχὰβ ὁ πατὴρ ἡμῶν ἐνετείλατο ἡμῖν, λέγων, οὐ μὴ πίητε οἶνον ὑμεῖς καὶ οἱ υἱοὶ υμῶν ἕως αἰῶνος,
\VS{7}καὶ οἰκίας οὐ μὴ οἰκοδομήσητε, καὶ σπέρμα οὐ μὴ σπείρητε, καὶ ἀμπελὼν οὐκ ἔσται ὑμῖν, ὅτι ἐν σκηναῖς οἰκήσετε πάσας τὰς ἡμέρας ὑμῶν, ὅπως ἂν ζήσητε ἡμέρας πολλὰς ἐπὶ τῆς γῆς, ἐφʼ ἧς διατρίβετε ὑμεῖς ἐπʼ αὐτῆς.
\VS{8}Καὶ ἠκούσαμεν τῆς φωνῆς Ἰωναδὰβ τοῦ πατρὸς ἡμῶν, πρὸς τὸ μὴ πιεῖν οἶνον πάσας τὰς ἡμέρας ἡμῶν, ἡμεῖς καὶ αἱ γυναῖκες ἡμῶν, καὶ οἱ υἱοὶ ἡμῶν, καὶ αἱ θυγατέρες ἡμῶν,
\VS{9}καὶ πρὸς τὸ μὴ οἰκοδομεῖν οἰκίας τοῦ κατοικεῖν ἐκεῖ, καὶ ἀμπελὼν καὶ ἀγρὸς καὶ σπέρμα οὐκ ἐγένετο ἡμῖν.
\VS{10}Καὶ ᾠκήσαμεν ἐν σκηναῖς, καὶ ἠκούσαμεν, καὶ ἐποιήσαμεν κατὰ πάντα ἃ ἐνετείλατο ἡμῖν Ἰωναδὰβ ὁ πατὴρ ἡμῶν.
\VS{11}Καὶ ἐγενήθη ὅτε ἀνέβη Ναβουχοδονόσορ ἐπὶ τὴν γῆν, καὶ εἴπαμεν εἰσελθεῖν, καὶ εἰσήλθομεν εἰς Ἱερουσαλὴμ, ἀπὸ προσώπου τῆς δυνάμεως τῶν Χαλδαίων, καὶ ἀπὸ προσώπου τῆς δυνάμεως τῶν Ἀσσυρίων, καὶ ᾠκοῦμεν ἐκεῖ.
\par }{\PP \VS{12}Καὶ ἐγένετο λόγος Κυρίου πρὸς μὲ, λέγων,
\VS{13}οὕτως λέγει Κύριος, πορεύου, καὶ εἰπὸν ἀνθρώπῳ Ἰούδα, καὶ τοῖς κατοικοῦσιν Ἱερουσαλὴμ, οὐ μὴ λάβητε παιδείαν τοῦ ἀκούειν τοὺς λόγους μου;
\VS{14}Ἔστησαν ῥῆμα υἱοὶ Ἰωναδὰβ υἱοῦ Ῥηχὰβ, ὃ ἐνετείλατο τοῖς τέκνοις αὐτοῦ πρὸς τὸ μὴ πιεῖν οἶνον, καὶ οὐκ ἐπίοσαν· καὶ ἐγὼ ἐλάλησα πρὸς ὑμᾶς ὄρθρου, καὶ οὐκ ἠκούσατε.
\VS{15}Καὶ ἀπέστειλα πρὸς ὑμᾶς τοὺς παῖδάς μου τοὺς προφήτας, λέγων, ἀποστράφητε ἕκαστος ἀπὸ τῆς ὁδοῦ αὐτοῦ τῆς πονηρᾶς, καὶ βελτίω ποιήσατε τὰ ἐπιτηδεύματα ὑμῶν, καὶ οὐ πορεύεσθε ὀπίσω θεῶν ἑτέρων τοῦ δουλεύειν αὐτοῖς, καὶ οἰκήσετε ἐπὶ τῆς γῆς, ἧς ἔδωκα ὑμῖν, καὶ τοῖς πατράσιν ὑμῶν· καὶ οὐκ ἐκλίνατε τὰ ὦτα ὑμῶν, καὶ οὐκ εἰσηκούσατε.
\VS{16}Καὶ ἔστησαν υἱοὶ Ἰωναδὰβ υἱοῦ Ῥηχὰβ τὴν ἐντολὴν τοῦ πατρὸς αὐτῶν, ὁ δὲ λαὸς οὗτος οὐκ ἤκουσέ μου.
\VS{17}Διατοῦτο οὕτως εἶπε Κύριος, ἰδοὺ ἐγὼ φέρω ἐπὶ Ἰούδαν καὶ ἐπὶ τοὺς κατοικοῦντας Ἱερουσαλὴμ πάντα τὰ κακὰ ἃ ἐλάλησα ἐπʼ αὐτούς.
\par }{\PP \VS{18}Διατοῦτο οὕτως εἶπε Κύριος, ἐπειδὴ ἤκουσαν υἱοὶ Ἰωναδὰβ υἱοῦ Ῥηχὰβ τὴν ἐντολὴν τοῦ πατρὸς αὐτῶν ποιεῖν καθότι ἐνετείλατο αὐτοῖς ὁ πατὴρ αὐτῶν,
\VS{19}οὐ μὴ ἐκλείπῃ ἀνὴρ τῶν υἱῶν Ἰωναδὰβ υἱοῦ Ῥηχὰβ παρεστηκὼς κατὰ πρόσωπόν μου πάσας τὰς ἡμέρας τῆς γῆς.

\par }\Chap{43}{\PP \VerseOne{1}ἘΝ ΤΩ ἘΝΙΑΥΤΩ ΤΩ ΤΕΤΑΡΤΩ ἸΩΑΚΕΙΜ υἱοῦ Ἰωσία βασιλέως Ἰούδα, ἐγενήθη λόγος Κυρίου πρὸς μὲ, λέγων,
\par }{\PP \VS{2}Λάβε σεαυτῷ χαρτίον βιβλίου, καὶ γράψον ἐπʼ αὐτοῦ πάντας τοὺς λόγους οὓς ἐλάλησα πρὸς σὲ ἐπὶ Ἱερουσαλὴμ, καὶ ἐπὶ Ἰούδα, καὶ ἐπὶ πάντα τὰ ἔθνη, ἀφʼ ἧς ἡμέρας λαλήσαντός μου πρὸς σὲ ἀφʼ ἡμερῶν Ἰωσία βασιλέως Ἰούδα, καὶ ἕως τῆς ἡμέρας ταύτης.
\VS{3}Ἴσως ἀκούσεται ὁ οἶκος Ἰούδα πάντα τὰ κακὰ ἃ ἐγὼ λογίζομαι ποιῆσαι αὐτοῖς, ἵνα ἀποστρέψωσιν ἀπὸ τῆς ὁδοῦ αὐτῶν τῆς πονηρᾶς, καὶ ἵλεως ἔσομαι ταῖς ἀδικίαις αὐτῶν καὶ ταῖς ἁμαρτίαις αὐτῶν.
\par }{\PP \VS{4}Καὶ ἐκάλεσεν Ἱερεμίας τὸν Βαροὺχ υἱὸν Νηρίου· καὶ ἔγραψεν ἀπὸ στόματος Ἱερεμίου πάντας τοὺς λόγους Κυρίου, οὓς ἐλάλησε πρὸς αὐτὸν, εἰς χαρτίον βιβλίου.
\VS{5}Καὶ ἐνετείλατο Ἱερεμίας τῷ Βαροὺχ, λέγων, ἐγὼ φυλάσσομαι, οὐ μὴ δύνωμαι εἰσελθεῖν εἰς οἶκον Κυρίου·
\VS{6}Καὶ ἀναγνώσῃ ἐν τῷ χαρτίῳ τούτῳ εἰς τὰ ὦτα τοῦ λαοῦ ἐν οἴκῳ Κυρίου, ἐν ἡμέρᾳ νηστείας, καὶ ἐν ὠσὶ παντὸς Ἰούδα τῶν ἐρχομένων ἐκ πόλεων αὐτῶν, ἀναγνώσῃ αὐτοῖς.
\VS{7}Ἴσως πεσεῖται ἔλεος αὐτῶν κατὰ πρόσωπον Κυρίου, καὶ ἀποστρέψουσιν ἐκ τῆς ὁδοῦ αὐτῶν τῆς πονηρᾶς, ὅτι μέγας ὁ θυμὸς καὶ ἡ ὀργὴ Κυρίου, ἣν ἐλάλησεν ἐπὶ τὸν λαὸν τοῦτον.
\par }{\PP \VS{8}Καὶ ἐποίησε Βαροὺχ κατὰ πάντα ἃ ἐνετείλατο αὐτῷ Ἱερεμίας, τοῦ ἀναγνῶναι ἐν τῷ βιβλίῳ τοὺς λόγους Κυρίου ἐν οἴκῳ Κυρίου.
\VS{9}Καὶ ἐγενήθη ἐν τῷ ἔτει τῷ ὀγδόῳ τῷ βασιλεῖ Ἰωακεὶμ ἐν τῷ μηνὶ τῷ ἐννάτῳ, ἐξεκκλησίασαν νηστείαν κατὰ πρόσωπον Κυρίου πᾶς ὁ λαὸς ἐν Ἱερουσαλὴμ, καὶ οἴκος Ἰούδα.
\VS{10}Καὶ ἀνεγίνωσκε Βαροὺχ ἐν τῷ βιβλίῳ τοὺς λόγους Ἱερεμίου ἐν οἴκῳ Κυρίου, ἐν οἴκῳ Γαμαρίου υἱοῦ Σαφὰν τοῦ γραμματέως, ἐν τῇ αὐλῇ τῇ ἐπάνω ἐν προθύροις πύλης τοῦ οἴκου Κυρίου τῆς καινῆς, καὶ ἐν ὠσὶ παντὸς τοῦ λαοῦ.
\par }{\PP \VS{11}Καὶ ἤκουσε Μιχαίας υἱὸς Γαμαρίου υἱοῦ Σαφὰν ἅπαντας τοὺς λόγους Κυρίου, ἐκ τοῦ βιβλίου·
\VS{12}Καὶ κατέβη εἰς οἰκίαν τοῦ βασιλέως, εἰς τὸν οἶκον τοῦ γραμματέως, καὶ ἰδοὺ ἐκεῖ πάντες οἱ ἄρχοντες ἐκάθηντο, Ἐλισαμὰ ὁ γραμματεὺς, καὶ Δαλαίας υἱὸς Σελεμίου, καὶ Ἰωνάθαν υἱὸς Ἀκχοβὼρ, καὶ Γαμαρίας υἱὸς Σαφὰν, καὶ Σεδεκίας υἱὸς Ἀνανίου, καὶ πάντες οἱ ἄρχοντες·
\VS{13}Καὶ ἀνήγγειλεν αὐτοῖς Μιχαίας πάντας τοὺς λόγους οὓς ἤκουσεν ἀναγινώσκοντος Βαροὺχ εἰς τὰ ὦτα τοῦ λαοῦ.
\par }{\PP \VS{14}Καὶ ἀπέστειλαν πάντες οἱ ἄρχοντες πρὸς Βαροὺχ υἱὸν Νηρίου, τὸν Ἰουδὶν υἱὸν Ναθανίου, υἱοῦ Σελεμίου, υἱοῦ Χουσὶ, λέγοντες, τὸ χαρτίον ἐν ᾧ σὺ ἀναγινώσκεις ἐν αὐτῷ ἐν ὠσὶ τοῦ λαοῦ, λάβε αὐτὸ εἰς τὴν χεῖρά σου, καὶ ἧκε· καὶ ἔλαβε Βαροὺχ τὸ χαρτίον, καὶ κατέβη πρὸς αὐτούς.
\VS{15}Καὶ εἶπον αὐτῷ, πάλιν ἀνάγνωθι εἰς τὰ ὦτα ἡμῶν· καὶ ἀνέγνω Βαρούχ.
\VS{16}Καὶ ἐγενήθη ὡς ἤκουσαν πάντας τοὺς λόγους, συνεβουλεύσαντο ἕκαστος πρὸς τὸν πλησίον αὐτοῦ, καὶ εἶπον, ἀναγγέλλοντες ἀναγγείλωμεν τῷ βασιλεῖ ἅπαντας τοὺς λόγους τούτους.
\VS{17}Καὶ τὸν Βαροὺχ ἠρώτησαν, λέγοντες, ποῦ ἔγραψας πάντας τοὺς λόγους τούτους;
\VS{18}Καὶ εἶπε Βαροὺχ, ἀπὸ στόματος αὐτοῦ ἀνήγγειλέ μοι Ἱερεμίας πάντας τοὺς λόγους τούτους, καὶ ἔγραφον ἐν βιβλίῳ.
\VS{19}Καὶ εἶπον τῷ Βαροὺχ, βάδισον, καὶ κατακρύβηθι σὺ καὶ Ἱερεμίας, ἄνθρωπος μὴ γνώτω ποῦ ὑμεῖς.
\par }{\PP \VS{20}Καὶ εἰσῆλθον πρὸς τὸν βασιλέα εἰς τὴν αὐλὴν, καὶ τὸ χαρτίον ἔδωκαν φυλάσσειν ἐν οἴκῳ Ἐλισαμά· καὶ ἀνήγγειλαν τῷ βασιλεῖ πάντας τοὺς λόγους τούτους.
\VS{21}Καὶ ἀπέστειλεν ὁ βασιλεὺς τὸν Ἰουδὶν, λαβεῖν τὸ χαρτίον· καὶ ἔλαβεν αὐτὸ ἐξ οἴκου Ἐλισαμά· καὶ ἀνέγνω Ἰουδὶν εἰς τὰ ὦτα τοῦ βασιλέως, καὶ εἰς τὰ ὦτα πάντων τῶν ἀρχόντων, τῶν ἑστηκότων περὶ τὸν βασιλέα.
\VS{22}Καὶ ὁ βασιλεὺς ἐκάθητο ἐν οἴκῳ χειμερινῷ, καὶ ἐσχάρα πυρὸς κατὰ πρόσωπον αὐτοῦ.
\VS{23}Καὶ ἐγενήθη ἀναγινώσκοντος Ἰουδὶν τρεῖς σελίδας καὶ τέσσαρας, ἀπέτεμεν αὐτὰς τῷ ξυρῷ τοῦ γραμματέως, καὶ ἔῤῥιπτεν εἰς τὸ πῦρ τὸ ἐπὶ τῆς ἐσχάρας, ἕως ἐξέλιπε πᾶς ὁ χάρτης εἰς τὸ πῦρ τὸ ἐπὶ τῆς ἐσχάρας.
\VS{24}Καὶ οὐκ ἐζήτησαν, καὶ οὐ διέῤῥηξαν τὰ ἱμάτια αὐτῶν ὁ βασιλεὺς καὶ οἱ παῖδες αὐτοῦ οἱ ἀκούοντες πάντας τοὺς λόγους τούτους.
\VS{25}Καὶ Ἐλνάθαν καὶ Γοδολίας ὑπέθεντο τῷ βασιλεῖ, πρὸς τὸ κατακαῦσαι τὸ χαρτίον.
\par }{\PP \VS{26}Καὶ ἐνετείλατο ὁ βασιλεὺς τῷ Ἱερεμεὴλ υἱῷ τοῦ βασιλέως, καὶ τῷ Σαραίᾳ υἱῷ Ἐσριὴλ, συλλαβεῖν τὸν Βαροὺχ, καὶ τὸν Ἱερεμίαν, καὶ κατεκρύβησαν.
\par }{\PP \VS{27}Καὶ ἐγένετο λόγος Κυρίου πρὸς Ἱερεμίαν, μετὰ τὸ κατακαῦσαι τὸν βασιλέα τὸ χαρτίον, πάντας τοὺς λόγους, οὓς ἔγραψε Βαροὺχ ἀπὸ στόματος Ἱερεμίου, λέγων,
\VS{28}πάλιν λάβε σὺ χαρτίον ἕτερον, καὶ γράψον πάντας τοὺς λόγους, τοὺς ὄντας ἐπὶ τοῦ χαρτίου, οὓς κατέκαυσεν ὁ βασιλεὺς Ἰωακεὶμ,
\VS{29}καὶ ἐρεῖς, οὕτως εἶπε Κύριος, σὺ κατέκαυσας τὸ χαρτίον τοῦτο, λέγων, διατί ἔγραψας ἐπʼ αὐτῷ, λέγων, εἰσπορευόμενος εἰσπορεύσεται βασιλεὺς Βαβυλῶνος, καὶ ἐξολοθρεύσει τὴν γῆν ταύτην, καὶ ἐκλείψει ἐπʼ αὐτῆς ἄνθρωπος καὶ κτήνη;
\par }{\PP \VS{30}Διατοῦτο οὕτως εἶπε Κύριος ἐπὶ Ἰωακεὶμ βασιλέα Ἰούδα, οὐκ ἔσται αὐτῷ καθήμενος ἐπὶ θρόνου Δαυὶδ, καὶ τὸ θνησιμαῖον αὐτοῦ ἔσται ἐῤῥιμμένον ἐν τῷ καύματι τῆς ἡμέρας, καὶ ἐν τῷ παγετῷ τῆς νυκτός·
\VS{31}Καὶ ἐπισκέψομαι ἐπʼ αὐτὸν, καὶ ἐπὶ τὸ γένος αὐτοῦ, καὶ ἐπὶ τοὺς παῖδας αὐτοῦ, καὶ ἐπάξω ἐπʼ αὐτὸν, καὶ ἐπὶ τοὺς κατοικοῦντας Ἱερουσαλὴμ, καὶ ἐπὶ γῆν Ἰούδα, πάντα τὰ κακὰ ἃ ἐλάλησα πρὸς αὐτοὺς, καὶ οὐκ ἤκουσαν.
\par }{\PP \VS{32}Καὶ ἔλαβε Βαροὺχ χαρτίον ἕτερον, καὶ ἔγραψεν ἐπʼ αὐτῷ ἀπὸ στόματος Ἱερεμίου ἅπαντας τοὺς λόγους τοῦ βιβλίου, οὓς κατέκαυσεν Ἰωακείμ· καὶ ἔτι προσετέθησαν αὐτῷ λόγοι πλείονες, ὡς οὗτοι.

\par }\Chap{44}{\PP \VerseOne{1}Καὶ ἐβασίλευσε Σεδεκίας υἱὸς Ἰωσεία ἀντὶ Ἰωακείμ, ὃν ἐβασίλευσε Ναβουχοδονόσορ βασιλεύειν τοῦ Ἰούδα.
\VS{2}Καὶ οὐκ ἤκουσαν αὐτὸς καὶ οἱ παῖδες αὐτοῦ καὶ ὁ λαὸς τῆς γῆς τοῦς λόγους Κυρίου, οὓς ἐλάλησεν ἐν χειρὶ Ἱερεμίου.
\par }{\PP \VS{3}Καὶ ἀπέστειλεν ὁ βασιλεὺς Σεδεκίας τὸν Ἰωάχαλ υἱὸν Σελεμίου, καὶ τὸν Σοφονίαν υἱὸν Μαασαίου τὸν ἱερέα πρὸς Ἱερεμίαν λέγων, πρόσευξαι δὴ περὶ ἡμῶν πρὸς Κύριον.
\VS{4}Καὶ Ἱερεμίας ἦλθε καὶ διῆλθε διὰ μέσου τῆς πόλεως, καὶ οὐκ ἔδωκαν αὐτὸν εἰς τὸν οἶκον τῆς φυλακῆς.
\VS{5}Καὶ δύναμις Φαραὼ ἐξῆλθεν ἐξ Αἰγύπτου, καὶ ἤκουσαν οἱ Χαλδαῖοι τὴν ἀκοὴν αὐτῶν, καὶ ἀνέβησαν ἐπὶ Ἰερουσαλήμ.
\par }{\PP \VS{6}Καὶ ἐγένετο λόγος Κυρίου πρὸς Ἱερεμίαν, λέγων,
\VS{7}οὕτως εἶπε Κύριος, οὕτως ἐρεῖς πρὸς βασιλέα Ἰούδα τὸν ἀποστείλαντα πρὸς σὲ, τοῦ ἐκζητῆσαί με, ἰδοὺ δύναμις· Φαραὼ ἡ ἐξελθοῦσα ὑμῖν εἰς βοήθειαν· ἀποστρέψουσιν εἰς γῆν Αἰγύπτου,
\VS{8}καὶ ἀναστρέψουσιν αὐτοὶ οἱ Χαλδαῖοι, καὶ πολεμήσουσιν ἐπὶ τὴν πόλιν ταύτην, καὶ συλλήψονται αὐτὴν, καὶ καύσουσιν αὐτὴν ἐν πυρί.
\VS{9}Ὅτι οὕτως εἶπε Κύριος, μὴ ὑπολάβητε ταῖς ψυχαῖς ὑμῶν, λέγοντες, ἀποτρέχοντες ἀπελεύσονται ἀφʼ ἡμῶν οἱ Χαλδαῖοι· ὅτι οὐ μὴ ἀπέλθωσι.
\VS{10}Καὶ ἐὰν πατάξητε πᾶσαν δύναμιν τῶν Χαλδαίων τοὺς πολεμοῦντας ὑμᾶς, καὶ καταλειφθῶσί τινες ἐκκεκεντημένοι, ἕκαστος ἐν τῷ τόπῳ αὐτοῦ οὗτοι ἀναστήσονται, καὶ καύσουσι τὴν πόλιν ταύτην ἐν πυρί.
\par }{\PP \VS{11}Καὶ ἐγένετο ὅτε ἀνέβη ἡ δύναμις τῶν Χαλδαίων ἀπὸ Ἱερουσαλὴμ ἀπὸ προσώπου τῆς δυνάμεως Φαραὼ,
\VS{12}ἐξῆλθεν Ἱερεμίας ἀπὸ Ἱερουσαλὴμ τοῦ πορευθῆναι εἰς γῆν Βενιαμεὶν, τοῦ ἀγοράσαι ἐκεῖθεν ἐν μέσῳ τοῦ λαοῦ·
\VS{13}Καὶ ἐγένετο αὐτὸς ἐν πύλῃ Βενιαμὶν, καὶ ἐκεῖ ἄνθρωπος παρʼ ᾧ κατέλυε, Σαρουΐα υἱὸς Σελεμίου, υἱοῦ Ἀνανίου, καὶ συνέλαβε τὸν Ἱερεμίαν, λέγων, πρὸς τοὺς Χαλδαίους σὺ φεύγεις.
\VS{14}Καὶ εἶπε, ψεῦδος, οὐκ εἰς τοὺς Χαλδαίους ἐγὼ φεύγω· καὶ οὐκ εἰσήκουσεν αὐτοῦ· καὶ συνέλαβε Σαρουΐα τὸν Ἱερεμίαν, καὶ εἰσήγαγεν αὐτὸν πρὸς τοὺς ἄρχοντας.
\VS{15}Καὶ ἐπικράνθησαν οἱ ἄρχοντες ἐπὶ Ἱερεμίαν, καὶ ἐπάταξαν αὐτὸν, καὶ ἀπέστειλαν αὐτὸν εἰς τὴν οἰκίαν Ἰωνάθαν τοῦ γραμματέως, ὅτι ταύτην ἐποίησαν εἰς οἰκίαν φυλακῆς.
\par }{\PP \VS{16}Καὶ ἦλθεν Ἱερεμίας εἰς οἰκίαν τοῦ λάκκου, καὶ εἰς τὴν χερὲθ, καὶ ἐκάθισεν ἐκεῖ ἡμέρας πολλάς.
\VS{17}Καὶ ἀπέστειλε Σεδεκίας, καὶ ἐκάλεσεν αὐτὸν, καὶ ἠρώτα αὐτὸν ὁ βασιλεὺς κρυφαίως εἰπεῖν, εἰ ἔστιν ὁ λόγος παρὰ Κυρίου; καὶ εἶπεν, ἔστιν· εἰς χεῖρας βασιλέως Βαβυλῶνος παραδοθήσῃ.
\VS{18}Καὶ εἶπεν Ἱερεμίας τῷ βασιλεῖ, τί ἠδίκησά σε, καὶ τοὺς παῖδάς σου, καὶ τὸν λαὸν τοῦτον, ὅτι σὺ δίδως με εἰς οἰκίαν φυλακῆς;
\VS{19}Καὶ ποῦ εἰσιν οἱ προφῆται ὑμῶν οἱ προφητεύσαντες ὑμῖν, λέγοντες, ὅτι οὐ μὴ ἔλθῃ βασιλεὺς Βαβυλῶνος ἐπὶ τὴν γῆν ταύτην;
\VS{20}Καὶ νῦν, Κύριε βασιλεῦ, πεσέτω τὸ ἔλεός μου κατὰ πρόσωπόν σου· καὶ τί ἀποστρέφεις με εἰς οἰκίαν Ἰωνάθαν τοῦ γραμματέως; καὶ οὐ μὴ ἀποθάνω ἐκεῖ.
\VS{21}Καὶ συνέταξεν ὁ βασιλεὺς, καὶ ἐνεβάλοσαν αὐτὸν εἰς οἰκίαν τῆς φυλακῆς, καὶ ἐδίδοσαν αὐτῷ ἄρτον ἕνα τῆς ἡμέρας, ἔξωθεν οὗ πέσσουσιν, ἕως ἐξέλιπον οἱ ἄρτοι ἐκ τῆς πόλεως· καὶ ἐκάθισεν Ἱερεμίας ἐν τῇ αὐλῇ τῆς φυλακῆς.

\par }\Chap{45}{\PP \VerseOne{1}Καὶ ἤκουσε Σαφανίας υἱὸς Νάθαν, καὶ Γοδολίας υἱὸς Πασχὼρ, καὶ Ἰωάχαλ υἱὸς Σεμελίου, τοὺς λόγους οὓς Ἱερεμίας ἐλάλει ἐπὶ τὸν λαὸν, λέγων,
\par }{\PP \VS{2}Οὕτως εἶπε Κύριος, ὁ κατοικῶν ἐν τῇ πόλει ταύτῃ, ἀποθανεῖται ἐν ῥομφαίᾳ, καὶ ἐν λιμῷ· καὶ ὁ ἐκπορευόμενος πρὸς τοὺς Χαλδαίους, ζήσεται, καὶ ἔσται ἡ ψυχὴ αὐτοῦ εἰς εὕρημα, καὶ ζήσεται.
\VS{3}Ὅτι οὕτως εἶπε Κύριος, παραδιδομένη παραδοθήσεται ἡ πόλις αὕτη εἰς χεῖρας δυνάμεως βασιλέως Βαβυλῶνος, καὶ συλλήψεται αὐτήν.
\VS{4}Καὶ εἶπον τῷ βασιλεῖ, ἀναιρεθήτω δὴ ὁ ἄνθρωπος ἐκεῖνος, ὅτι αὐτὸς ἐκλύει τὰς χεῖρας τῶν ἀνθρώπων τῶν πολεμούντων τῶν καταλειπομένων ἐν τῇ πόλει, καὶ τὰς χεῖρας παντὸς τοῦ λαοῦ, λαλῶν πρὸς αὐτοὺς κατὰ τοὺς λόγους τούτους· ὅτι ὁ ἄνθρωπος οὗτος οὐ χρησμολογεῖ εἰρήνην τῷ λαῷ τούτῳ ἀλλʼ ἢ πονηρά.
\VS{5}Καὶ εἶπεν ὁ βασιλεὺς, ἰδοὺ αὐτὸς ἐν χερσὶν ὑμῶν· ὅτι οὐκ ἠδύνατο ὁ βασιλεὺς πρὸς αὐτούς.
\VS{6}Καὶ ἔῤῥιψαν αὐτὸν εἰς λάκκον Μελχίου υἱοῦ τοῦ βασιλέως, ὃς ἦν ἐν τῇ αὐλῇ τῆς φυλακῆς, καὶ ἐχάλασαν αὐτὸν εἰς τὸν λάκκον, καὶ ἐν τῷ λάκκῳ οὐκ ἦν ὕδωρ, ἀλλʼ ἢ βόρβορος, καὶ ἦν ἐν τῷ βορβόρῳ.
\par }{\PP \VS{7}Καὶ ἤκουσεν Ἀβδεμέλεχ ὁ Αἰθίοψ, καὶ αὐτὸς ἐν οἰκίᾳ τοῦ βασιλέως, ὅτι ἔδωκαν Ἱερεμίαν εἰς τὸν λάκκον· καὶ ὁ βασιλεὺς ἦν ἐν τῇ πύλῃ Βενιαμὶν,
\VS{8}καὶ ἐξῆλθε πρὸς αὐτὸν, καὶ ἐλάλησε πρὸς τὸν βασιλέα, καὶ εἶπεν,
\VS{9}ἐπονηρεύσω, ἃ ἐποίησας τοῦ ἀποκτεῖναι τὸν ἄνθρωπον τοῦτον, ἀπὸ προσώπου τοῦ λιμοῦ, ὅτι οὐκ εἰσὶν ἔτι ἄρτοι ἐν τῇ πόλει.
\VS{10}Καὶ ἐνετείλατο ὁ βασιλεὺς τῷ Ἀβδεμέλεχ, λέγων, λάβε εἰς τὰς χεῖράς σου ἐντεῦθεν τριάκοντα ἀνθρώπους, καὶ ἀνάγαγε αὐτὸν ἐκ τοῦ λάκκου, ἵνα μὴ ἀποθάνῃ.
\VS{11}Καὶ ἔλαβεν Ἀβδεμέλεχ τοὺς ἀνθρώπους, καὶ εἰσῆλθεν εἰς τὴν οἰκίαν τοῦ βασιλέως τὴν ὑπόγαιον, καὶ ἔλαβεν ἐκεῖθεν παλαιὰ ῥάκη καὶ παλαιὰ σχοινία, καὶ ἔῤῥιψεν αὐτὰ πρὸς Ἱερεμίαν εἰς τὸν λάκκον,
\VS{12}καὶ εἶπε, ταῦτα θὲς ὑποκάτω τῶν σχοινίων· καὶ ἐποίησεν Ἱερεμίας οὕτως.
\VS{13}Καὶ εἵλκυσαν αὐτὸν τοῖς σχοινίοις, καὶ ἀνήγαγον αὐτὸν ἐκ τοῦ λάκκου· καὶ ἐκάθισεν Ἱερεμίας ἐν τῇ αὐλῇ τῆς φυλακῆς.
\par }{\PP \VS{14}Καὶ ἀπέστειλεν ὁ βασιλεὺς, καὶ ἐκάλεσεν αὐτὸν προς ἑαυτὸν εἰς οἰκίαν Ἀσελεισὴλ, τὴν ἐν οἴκῳ Κυρίου· καὶ εἶπεν αὐτῷ ὁ βασιλεὺς, ἐρωτήσω σε λόγον, καὶ μὴ δὴ κρύψῃς ἀπʼ ἐμοῦ ῥῆμα.
\par }{\PP \VS{15}Καὶ εἶπεν Ἱερεμίας τῷ βασιλεῖ, ἐὰν ἀναγγείλω σοι, οὐχὶ θανάτῳ με θανατώσεις; καὶ ἐὰν συμβουλεύσω σοι, οὐ μὴ ἀκούσῃς μου.
\VS{16}Καὶ ὤμοσεν αὐτῷ ὁ βασιλεὺς, λέγων, ζῇ Κύριος ὃς ἐποίησεν ἡμῖν τὴν ψυχὴν ταύτην, εἰ ἀποκτενῶ σε, καὶ εἰ δώσω σε εἰς χεῖρας τῶν ἀνθρώπων τούτων.
\par }{\PP \VS{17}Καὶ εἶπεν αὐτῷ Ἱερεμίας, οὕτως εἶπε Κύριος, ἐὰν ἐξελθὼν ἐξέλθῃς πρὸς ἡγεμόνας βασιλέως Βαβυλῶνος, ζήσεται ἡ ψυχή σου, καὶ ἡ πόλις αὕτη οὐ μὴ κατακαυθῇ ἐν πυρὶ, καὶ ζήσῃ σὺ καὶ ἡ οἰκία σου.
\VS{18}Καὶ ἐὰν μὴ ἐξέλθῃς, δοθήσεται ἡ πόλις αὕτη εἰς χεῖρας τῶν Χαλδαίων, καὶ καύσουσιν αὐτὴν ἐν πυρὶ καὶ σὺ οὐ μὴ σωθῇς.
\par }{\PP \VS{19}Καὶ εἶπεν ὁ βασιλεὺς τῷ Ἱερεμίᾳ, ἐγὼ λόγον ἔχω τῶν Ἰουδαίων τῶν πεφευγότων πρὸς τοὺς Χαλδαίους, μὴ δώσειν με εἰς χεῖρας αὐτῶν, καὶ καταμωκήσονταί μου.
\par }{\PP \VS{20}Καὶ εἶπεν Ἱερεμίας, οὐ μὴ παραδῶσί σε· ἄκουσον τὸν λόγον Κυρίου, ὃν ἐγὼ λέγω πρὸς σὲ, καὶ βέλτιον ἔσται σοι, καὶ ζήσεται ἡ ψυχή σου.
\VS{21}Καὶ εἰ μὴ θέλῃς σὺ ἐξελθεῖν, οὗτος ὁ λόγος ὃν ἔδειξέ μοι Κύριος·
\VS{22}Καὶ ἰδοὺ πᾶσαι αἱ γυναῖκες αἱ καταλειφθεῖσαι ἐν οἰκίᾳ βασιλέως Ἰούδα, ἐξήγοντο πρὸς ἄρχοντας βασιλέως Βαβυλῶνος· καὶ αὗται ἔλεγον, ἠπάτησάν σε, καὶ δυνήσονταί σοι ἄνδρες εἰρηνικοί σου· καὶ καταλύσουσιν ἐν ὀλισθήμασι πόδα σου, ἀπέστρεψαν ἀπὸ σοῦ,
\VS{23}καὶ τᾶς γυναῖκάς σου καὶ τὰ τέκνα σου ἐξάξουσι πρὸς τοὺς Χαλδαίους· καὶ σὺ οὐ μὴ σωθῇς, ὅτι ἐν χειρὶ βασιλέως Βαβυλῶνος συλληφθήσῃ, καὶ ἡ πόλις αὕτη κατακαυθήσεται.
\par }{\PP \VS{24}Καὶ εἶπεν αὐτῷ ὁ βασιλεὺς, ἄνθρωπος μὴ γνώτω ἐκ τῶν λόγων τούτων, καὶ σὺ οὐ μὴ ἀποθάνῃς.
\VS{25}Καὶ ἐὰν οἱ ἄρχοντες ἀκούσωσιν ὅτι ἐλάλησά σοι, καὶ ἔλθωσι πρὸς σὲ, καὶ εἴπωσί σοι, ἀνάγγειλον ἡμῖν, τί ἐλάλησέ σοι ὁ βασιλεύς; μὴ κρύψῃς ἀφʼ ἡμῶν, καὶ οὐ μὴ ἀνέλωμέν σε· καὶ τί ἐλάλησε πρὸς σὲ ὁ βασιλεύς;
\VS{26}Καὶ ἐρεῖς αὐτοῖς, ῥίπτω ἐγὼ τὸ ἔλεός μου κατʼ ὀφθαλμοὺς τοῦ βασιλέως, πρὸς τὸ μὴ ἀποστρέψαι με εἰς οἰκίαν Ἰωνάθαν, ἀποθανεῖν με ἐκεῖ.
\par }{\PP \VS{27}Καὶ ἤλθοσαν πάντες οἱ ἄρχοντες πρὸς Ἰερεμίαν, καὶ ἠρώτησαν αὐτόν· καὶ ἀνήγγειλεν αὐτοῖς κατὰ πάντας τοὺς λόγους τούτους, οὓς ἐνετείλατο αὐτῷ ὁ βασιλεύς· καὶ ἀπεσιώπησαν, ὅτι οὐκ ἠκούσθη ὁ λόγος Κυρίου.
\VS{28}Καὶ ἐκάθισεν Ἱερεμίας ἐν τῇ αὐλῇ τῆς φυλακῆς, ἕως χρόνου οὗ συνελήφθη Ἱερουσαλήμ.

\par }\Chap{46}{\PP \VerseOne{1}Καὶ ἐγένετο τῷ μηνὶ τῷ ἐννάτῳ τοῦ Σεδεκία βασιλέως Ἰούδα, παρεγένετο Ναβουχοδονόσορ βασιλεὺς Βαβυλῶνος, καὶ πᾶσα ἡ δύναμις αὐτοῦ ἐπὶ Ἱερουσαλὴμ, καὶ ἐπολιόρκουν αὐτήν.
\VS{2}Καὶ ἐν τῷ ἑνδεκάτῳ ἔτει τοῦ Σεδεκία, ἐν τῷ μηνὶ τῷ τετάρτῳ, ἐννάτῃ τοῦ μηνὸς, ἐῤῥάγη ἡ πόλις,
\VS{3}καὶ εἰσῆλθον πάντες οἱ ἡγούμενοι βασιλέως Βαβυλῶνος, καὶ ἐκάθισαν ἐν πύλῃ τῇ μέσῃ Μαργανασὰρ, καὶ Σαμαγὼθ, καὶ Ναβουσάχαρ, καὶ Ναβουσαρεὶς, Ναγαργᾶς, Νασεῤῥαβαμὰθ, καὶ οἱ κατάλοιποι ἡγεμόνες βασιλέως Βαβυλῶνος.
\VS{14}Καὶ ἀπέστειλαν, καὶ ἔλαβον τὸν Ἱερεμίαν ἐξ αὐλῆς τῆς φυλακῆς, καὶ ἔδωκαν αὐτὸν πρὸς τὸν Γοδολίαν υἱὸν Ἀχεικὰμ, υἱοῦ Σαφὰν, καὶ ἐξήγαγον αὐτὸν, καὶ ἐκάθισεν ἐν μέσῳ τοῦ λαοῦ.
\par }{\PP \VS{15}Καὶ πρὸς Ἱερεμίαν ἐγένετο λόγος Κυρίου ἐν τῇ αὐλῇ τῆς φυλακῆς, λέγων,
\VS{16}πορεύου καὶ εἰπὲ πρὸς Ἀβδεμέλεχ τὸν Αἰθίοπα, οὕτως εἶπε Κύριος ὁ Θεὸς Ἰσραὴλ, ἰδοὺ ἐγὼ φέρω τοὺς λόγους μου ἐπὶ τὴν πόλιν ταύτην εἰς κακὰ καὶ οὐκ εἰς ἀγαθά.
\VS{17}Καὶ σώσω σε ἐν τῇ ἡμέρᾳ ἐκείνῃ, καὶ οὐ μὴ δώσω σε εἰς χεῖρας τῶν ἀνθρώπων ὧν σὺ φοβῇ ἀπὸ προσώπου αὐτῶν,
\VS{18}ὅτι σώζων σώσω σε, καὶ ἐν ῥομφαίᾳ οὐ μὴ πέσῃς· καὶ ἔσται ἡ ψυχή σου εἰς εὕρημα, ὅτι ἐπεποίθεις ἐπʼ ἐμοὶ, φησὶ Κύριος.

\par }\Chap{47}{\PP \VerseOne{1}Ὁ λόγος ὁ γενόμενος παρὰ Κυρίου πρὸς Ἱερεμίαν, μετὰ τὸ ἀποστεῖλαι αὐτὸν Ναβουζαρδὰν τὸν ἀρχιμάγειρον τὸν ἐκ Ῥαμὰ, ἐν τῷ λαβεῖν αὐτὸν ἐν χειροπέδαις, ἐν μέσῳ ἀποικίας Ἰούδα τῶν ἠγμένων εἰς Βαβυλῶνα.
\par }{\PP \VS{2}Καὶ ἔλαβεν αὐτὸν ὁ ἀρχιμάγειρος, καὶ εἶπεν αὐτῷ, Κύριος ὁ Θεός σου ἐλάλησε τὰ κακὰ ταῦτα ἐπὶ τὸν τόπον τοῦτον·
\VS{3}Καὶ ἐποίησε Κύριος, ὅτι ἡμάρτετε αὐτῷ, καὶ οὐκ ἠκούσατε τῆς φωνῆς αὐτοῦ.
\VS{4}Ἰδοὺ ἔλυσά σε ἀπὸ τῶν χειροπέδων τῶν ἐπὶ τὰς χεῖράς σου· εἰ καλὸν ἐναντίον σου ἐλθεῖν μετʼ ἐμοῦ εἰς Βαβυλῶνα, καὶ θήσω τοὺς ὀφθαλμούς μου ἐπὶ σέ.
\VS{5}Εἰ δὲ μὴ, ἀπότρεχε, ἀνάστρεψον πρὸς τὸν Γοδολίαν υἱὸν Ἀχεικὰμ, υἱοῦ Σαφὰν, ὃν κατέστησε βασιλεὺς Βαβυλῶνος ἐν γῇ Ἰούδα, καὶ οἴκησον μετʼ αὐτοῦ ἐν μέσῳ τοῦ λαοῦ ἐν γῇ Ἰούδα, εἰς ἅπαντα τὰ ἀγαθὰ ἐν ὀφθαλμοῖς σου τοῦ πορευθῆναι ἐκεῖ, καὶ πορεύου· καὶ ἔδωκεν αὐτῷ ὁ ἀρχιμάγειρος δῶρα, καὶ ἀπέστειλεν αὐτόν.
\VS{6}Καὶ ἦλθε πρὸς Γοδολίαν εἰς Μασσηφὰ, καὶ ἐκάθισεν ἐν μέσῳ τοῦ λαοῦ αὐτοῦ, τοῦ καταλειφθέντος ἐν τῇ γῇ.
\par }{\PP \VS{7}Καὶ ἤκουσαν πάντες οἱ ἡγεμόνες τῆς δυνάμεως τῆς ἐν ἀγρῷ, αὐτοὶ καὶ οἱ ἄνδρες αὐτῶν, ὅτι κατέστησε βασιλεὺς Βαβυλῶνος τὸν Γοδολίαν ἐν τῇ γῇ, καὶ παρακατέθεντο αὐτῷ ἄνδρας καὶ γυναῖκας αὐτῶν, οὓς οὐ κατῴκισεν εἰς Βαβυλῶνα.
\VS{8}Καὶ ἦλθε πρὸς Γοδολίαν εἰς Μασσηφὰ Ἰσραὴλ υἱὸς Ναθανίου, καὶ Ἰωάναν υἱὸς Κάρηε, καὶ Σαραίας υἱὸς Θαναεμὲθ, καὶ υἱοὶ Ἰωφὲ τοῦ Νετωφαθὶ, καὶ Ἐζονίας υἱὸς τοῦ Μωχαθὶ, αὐτοὶ, καὶ οἱ ἄνδρες αὐτῶν.
\par }{\PP \VS{9}Καὶ ὤμοσεν αὐτοῖς Γοδολίας, καὶ τοῖς ἀνδράσιν αὐτῶν, λέγων, μὴ φοβηθῆτε ἀπὸ προσώπου τῶν παίδων τῶν Χαλδαίων· κατοικήσατε ἐν τῇ γῇ, καὶ ἐργάσασθε τῷ βασιλεῖ Βαβυλῶνος, καὶ βέλτιον ἔσται ὑμῖν.
\VS{10}Καὶ ἰδοὺ ἐγὼ κάθημαι ἐναντίον ὑμῶν εἰς Μασσηφὰ, στῆναι κατὰ πρόσωπον τῶν Χαλδαίων, οἳ ἂν ἔλθωσιν ἐφʼ ὑμᾶς· καὶ ὑμεῖς συνάγετε οἶνον καὶ ὀπώραν καὶ ἔλαιον, καὶ βάλετε εἰς τὰ ἀγγεῖα ὑμῶν, καὶ οἰκήσατε ἐν ταῖς πόλεσιν αἷς κατεκρατήσατε.
\par }{\PP \VS{11}Καὶ πάντες οἱ Ἰουδαῖοι οἱ ἐν Μωὰβ, καὶ ἐν υἱοῖς Ἀμμὼν, καὶ οἱ ἐν τῇ Ἰδουμαίᾳ, καὶ οἱ ἐν πάσῃ τῇ γῇ, ἤκουσαν ὅτι ἔδωκε βασιλεὺς Βαβυλῶνος κατάλειμμα τῷ Ἰούδᾳ, καὶ ὅτι κατέστησεν ἐπʼ αὐτοὺς τὸν Γοδολίαν υἱὸν Ἀχεικάμ.
\VS{12}Καὶ ἦλθον πρὸς Γοδολίαν εἰς γῆν Ἰούδα εἰς Μασσηφὰ, καὶ συνήγαγον οἶνον, καὶ ὀπώραν πολλὴν σφόδρα, καὶ ἔλαιον.
\par }{\PP \VS{13}Καὶ Ἰωάναν υἱὸς Κάρηε, καὶ πάντες οἱ ἡγεμόνες τῆς δυνάμεως, οἱ ἐν τοῖς ἀγροῖς, ἦλθον πρὸς τὸν Γοδολίαν εἰς Μασσηφὰ,
\VS{14}καὶ εἶπον αὐτῷ, εἰ γνώσει γινώσκεις, ὅτι Βελεισσὰ βασιλεὺς υἱὸς Ἀμμὼν ἀπέστειλε πρὸς σὲ τὸν Ἰσμαὴλ πατάξαι σου ψυχήν; καὶ οὐκ ἐπίστευσεν αὐτοῖς Γοδολίας.
\VS{15}Καὶ εἶπεν Ἰωάναν τῷ Γοδολίᾳ κρυφαίως ἐν Μασσηφᾷ, πορεύσομαι δὴ καὶ πατάξω τὸν Ἰσμαὴλ, καὶ μηδεὶς γνώτω, μὴ πατάξῃ σου ψυχὴν, καὶ διασπαρῇ πᾶς Ἰούδα οἱ συνηγμένοι πρὸς σὲ, καὶ ἀπολοῦνται οἱ κατάλοιποι Ἰούδα.
\VS{16}Καὶ εἶπε Γοδολίας πρὸς Ἰωάναν, Μῆ ποιήσῃς τὸ πρᾶγμα, ὅτι ψευδῆ σὺ λέγεις ὑπὲρ Ἰσμαήλ.

\par }\Chap{48}{\PP \VerseOne{1}Καὶ ἐγένετο τῷ μηνὶ τῷ ἑβδόμῳ, ἦλθεν Ἰσμαὴλ υἱὸς Ναθανίου υἱοῦ Ἐλεασὰ, ἀπὸ γένους τοῦ βασιλέως, καὶ δέκα ἄνδρες μετʼ αὐτοῦ πρὸς Γοδολίαν εἰς Μασσηφὰ, καὶ ἔφαγον ἐκεῖ ἄρτον ἅμα.
\VS{2}Καὶ ἀνέστη Ἰσμαὴλ, καὶ οἱ δέκα ἄνδρες οἳ ἦσαν μετʼ αὐτοῦ, καὶ ἐπάταξαν τὸν Γοδολίαν, ὃν κατέστησε βασιλεὺς Βαβυλῶνος ἐπὶ τῆς γῆς,
\VS{3}καὶ πάντας τοὺς Ἰουδαίους τοὺς ὄντας μετʼ αὐτοῦ ἐν Μασσηφὰ, καὶ πάντας τοὺς Χαλδαίους τοὺς εὑρεθέντας ἐκεῖ.
\par }{\PP \VS{4}Καὶ ἐγένετο τῇ ἡμέρᾳ τῇ δευτέρᾳ πατάξαντος αὐτοῦ τὸν Γοδολίαν, καὶ ἄνθρωπος οὐκ ἔγνω.
\VS{5}Καὶ ἤλθοσαν ἄνδρες ἀπὸ Συχὲμ, καὶ ἀπὸ Σαλὴμ, καὶ ἀπὸ Σαμαρίας, ὀγδοήκοντα ἄνδρες, ἐξυρημένοι πώγωνας, καὶ διεῤῥηγμένοι τὰ ἱμάτια, καὶ κοπτόμενοι, καὶ μάννα, καὶ λίβανος ἐν χερσὶν αὐτῶν, τοῦ εἰσενεγκεῖν εἰς οἶκον Κυρίου.
\VS{6}Καὶ ἐξῆλθεν εἰς ἀπάντησιν αὐτοῖς Ἰσμαήλ· αὐτοὶ ἐπορεύοντο, καὶ ἔκλαιον· καὶ εἶπεν αὐτοῖς, εἰσέλθετε πρὸς Γοδολίαν.
\VS{7}Καὶ ἐγένετο, εἰσελθόντων αὐτῶν εἰς τὸ μέσον τῆς πόλεως, ἔσφαξεν αὐτοὺς εἰς τὸ φρέαρ.
\VS{8}Καὶ δέκα ἄνδρες εὑρέθησαν ἐκεῖ, καὶ εἶπον τῷ Ἱσμαὴλ, μὴ ἀνέλῃς ἡμᾶς, ὅτι εἰσὶν ἡμῖν θησαυροὶ ἐν ἀγρῷ, πυροὶ καὶ κριθαὶ, μέλι καὶ ἔλαιον· καὶ παρῆλθε, καὶ οὐκ ἀνεῖλεν αὐτοὺς ἐν μέσῳ τῶν ἀδελφῶν αὐτῶν.
\par }{\PP \VS{9}Καὶ τὸ φρέαρ εἰς ὃ ἔῤῥιψεν ἐκεῖ Ἰσμαὴλ πάντας οὓς ἐπάταξε, φρέαρ μέγα τοῦτό ἐστιν, ὃ ἐποίησεν ὁ βασιλεὺς Ἀσὰ ἀπὸ προσώπου Βαασὰ βασιλέως Ἰσραὴλ, τοῦτο ἐνέπλησεν Ἰσμαὴλ τραυματιῶν.
\par }{\PP \VS{10}Καὶ ἀπέστρεψεν Ἰσμαὴλ πάντα τὸν λαὸν τὸν καταλειφθέντα εἰς Μασσηφὰ, καὶ τὰς θυγατέρας τοῦ βασιλέως, ἃς παρκατέθετο ὁ ἀρχιμάγειρος τῷ Γοδολίᾳ υἱῷ Ἀχεικὰμ, καὶ ᾤχετο εἰς τὸ πέραν υἱῶν Ἀμμών.
\par }{\PP \VS{11}Καὶ ἤκουσεν Ἰωάναν υἱὸς Κάρηε, καὶ πάντες οἱ ἡγεμόνες τῆς δυνάμεως οἱ μετʼ αὐτοῦ, πάντα τὰ κακὰ ἃ ἐποίησεν Ἰσμαὴλ,
\VS{12}καὶ ἤγαγον ἅπαν τὸ στρατόπεδον αὐτῶν, καὶ ᾤχοντο πολεμεῖν αὐτὸν, καὶ εὗρον αὐτὸν ἐπὶ ὕδατος πολλοῦ ἐν Γαβαών.
\VS{13}Καὶ ἐγένετο, ὅτε εἶδε πᾶς ὁ λαὸς ὁ μετὰ Ἰσμαὴλ τὸν Ἰωάναν καὶ τοὺς ἡγεμόνας τῆς δυνάμεως τῆς μετʼ αὐτοῦ,
\VS{14}καὶ ἀνέστρεψαν πρὸς Ἰωάναν.
\VS{15}Καὶ Ἰσμαὴλ ἐσώθη σὺν ὀκτὼ ἀνθρώποις, καὶ ᾤχετο πρὸς τοὺς υἱοὺς Ἀμμών.
\par }{\PP \VS{16}Καὶ ἔλαβεν Ἰωάναν, καὶ πάντες οἱ ἡγεμόνες τῆς δυνάμεως οἱ μετʼ αὐτοῦ, πάντας τοὺς καταλοίπους τοῦ λαοῦ, οὓς ἀπέστρέψεν ἀπὸ Ἰσμαὴλ, δυνατοὺς ἄνδρας ἐν πολέμῳ, καὶ τὰς γυναῖκας, καὶ τὰ λοιπὰ, καὶ τοὺς εὐνούχους, οὓς ἀπέστρεψαν ἀπὸ Γαβαὼν,
\VS{17}καὶ ᾤχοντο, καὶ ἐκάθισαν ἐν Γαβηρωχαμάα, τῇ πρὸς Βηθλεὲμ, τοῦ πορευθῆναι εἰς Αἴγυπτον
\VS{18}ἀπὸ προσώπου τῶν Χαλδαίων· ὅτι ἐφοβήθησαν ἀπὸ προσώπου αὐτῶν, ὅτι ἐπάταξεν Ἰσμαὴλ τὸν Γοδολίαν, ὃν κατέστησεν ὁ βασιλεὺς Βαβυλῶνος ἐν τῇ γῇ.

\par }\Chap{49}{\PP \VerseOne{1}Καὶ προσῆλθον πάντες οἱ ἡγεμόνες τῆς δυνάμεως, καὶ Ἰωάναν, καὶ Ἀζαρίας υἱὸς Μαασαίου, καὶ πᾶς ὁ λαὸς ἀπὸ μικροῦ καὶ ἕως μεγάλου,
\VS{2}πρὸς Ἱερεμίαν τὸν προφήτην, καὶ εἶπαν αὐτῷ, πεσέτω δὴ τὸ ἔλεος ἡμῶν κατὰ πρόσωπόν σου, καὶ πρόσευξαι πρὸς Κύριον τὸν Θεόν σου περὶ τῶν καταλοίπων τούτων· ὅτι κατελείφθημεν ὀλίγοι ἀπὸ πολλῶν, καθὼς οἱ ὀφθαλμοί σου βλέπουσι.
\VS{3}Καὶ ἀναγγειλάτω ἡμῖν Κύριος ὁ Θεός σου τὴν ὁδὸν ᾗ πορευσόμεθα ἐν αὐτῇ, καὶ λόγον ὃν ποιήσομεν.
\par }{\PP \VS{4}Καὶ εἶπεν αὐτοῖς Ἱερεμίας, ἤκουσα, ἰδοὺ ἐγὼ προσεύξομαι ὑπὲρ ὑμῶν πρὸς Κύριον τὸν Θεὸν ἡμῶν, κατὰ τοὺς λόγους ὑμῶν· καὶ ἔσται, ὁ λόγος ὃν ἂν ἀποκριθήσεται Κύριος ὁ Θεὸς, ἁναγγελῶ ὑμῖν, οὐ μὴ κρύψω ἀφʼ ὑμῶν ῥῆμα.
\par }{\PP \VS{5}Καὶ αὐτοὶ εἶπαν τῷ Ἱερεμίᾳ, ἔστω Κύριος ἐν ἡμῖν εἰς μάρτυρα δίκαιον καὶ πιστὸν, εἰ μὴ κατὰ πάντα τὸν λόγον, ὃν ἐὰν ἀποστείλῃ Κύριος πρὸς ἡμᾶς, οὕτως ποιήσωμεν.
\VS{6}Καὶ ἐὰν ἀγαθὸν καὶ ἐὰν κακὸν, τὴν φωνὴν Κυρίου τοῦ Θεοῦ ἡμῶν, οὗ ἡμεῖς ἀποστέλλομέν σε πρὸς αὐτὸν, ἀκουσόμεθα, ἵνα βέλτιον ἡμῖν γένηται, ὅτι ἀκουσόμεθα τῆς φωνῆς Κυρίου τοῦ Θεοῦ ἡμῶν.
\par }{\PP \VS{7}Καὶ ἐγενήθη μεθʼ ἡμέρας δέκα, ἐγενήθη λόγος Κυρίου πρὸς Ἱερεμίαν.
\VS{8}Καὶ ἐκάλεσε τὸν Ἰωάναν, καὶ τοὺς ἡγεμόνας τῆς δυνάμεως, καὶ πάντα τὸν λαὸν ἀπὸ μικροῦ καὶ ἕως μεγάλου,
\VS{9}καὶ εἶπεν αὐτοῖς, οὕτως εἶπε Κύριος,
\VS{10}ἐὰν καθίσαντες καθίσητε ἐν τῇ γῇ ταύτῃ, οἰκοδομήσω ὑμᾶς, καὶ οὐ μὴ καθελῶ, καὶ φυτεύσω ὑμᾶς, καὶ οὐ μὴ ἐκτιλῶ, ὅτι ἀναπέπαυμαι ἐπὶ τοῖς κακοῖς οἷς ἐποίησα ὑμῖν.
\VS{11}Μὴ φοβηθῆτε ἀπὸ προσώπου βασιλέως Βαβυλῶνος, οὗ ὑμεῖς φοβεῖσθε· ἀπὸ προσώπου αὐτοῦ μὴ φοβηθῆτε, φησὶ Κύριος, ὅτι μεθʼ ὑμῶν ἐγὼ, ἐξαιρεῖσθαι ὑμᾶς, καὶ σώζειν ὑμᾶς ἐκ χειρὸς αὐτῶν.
\VS{12}Καὶ δώσω ὑμῖν ἔλεος, καὶ ἐλεήσω ὑμᾶς, καὶ ἐπιστρέψω ὑμᾶς εἰς τὴν γῆν ὑμῶν.
\par }{\PP \VS{13}Καὶ εἰ λέγετε ὑμεῖς, οὐ μὴ καθίσωμεν ἐν τῇ γῇ ταύτῃ, πρὸς τὸ μὴ ἀκοῦσαι φωνῆς Κυρίου,
\VS{14}ὅτι εἰς γῆν Αἰγύπτου εἰσελευσόμεθα, καὶ οὐ μὴ ἴδωμεν πόλεμον, καὶ φωνὴν σάλπιγγος οὐ μὴ ἀκούσωμεν, καὶ ἐν ἄρτοις οὐ μὴ πεινάσωμεν, καὶ ἐκεῖ οἰκήσομεν·
\VS{15}διατοῦτο ἀκούσατε λόγον Κυρίου· οὕτως εἶπε Κύριος, ἐὰν ὑμεῖς δῶτε τὸ πρόσωπον ὑμῶν εἰς Αἴγυπτον, καὶ εἰσέλθητε ἐκεῖ κατοικεῖν,
\VS{16}καὶ ἔσται, ἡ ῥομφαία ἣν ὑμεῖς φοβεῖσθε ἀπὸ προσώπου αὐτῆς εὑρήσει ὑμᾶς ἐν γῇ Αἰγύπτου, καὶ ὁ λιμὸς οὗ ὑμεῖς λόγον ἔχετε ἀπὸ προσώπου αὐτοῦ καταλήψεται ὑμᾶς ὀπίσω ὑμῶν ἐν Αἰγύπτῳ, καὶ ἐκεῖ ἀποθανεῖσθε.
\VS{17}Καὶ ἔσονται πάντες οἱ ἄνθρωποι, καὶ πάντες οἱ ἀλλογενεῖς, οἱ θέντες τὸ πρόσωπον αὐτῶν εἰς γῆν Αἰγύπτου ἐνοικεῖν ἐκεῖ, ἐκλείψουσιν ἐν τῇ ῥομφαίᾳ, καὶ ἐν τῷ λιμῷ, καὶ οὐκ ἔσται αὐτῶν οὐδεὶς σωζόμενος ἀπὸ τῶν κακῶν ὧν ἐγὼ ἐπάγω ἐπʼ αὐτούς.
\par }{\PP \VS{18}Ὅτι οὕτως εἶπε Κύριος, καθὼς ἔσταξεν ὁ θυμός μου ἐπὶ τοὺς κατοικοῦντας Ἱερουσαλὴμ, οὕτως στάξει ὁ θυμός μου ἐφʼ ὑμᾶς, εἰσελθόντων ὑμῶν εἰς Αἴγυπτον· καὶ ἔσεσθε εἰς ἄβατον, καὶ ὑποχείριοι, καὶ εἰς ἀρὰν, καὶ εἰς ὀνειδισμὸν, καὶ οὐ μὴ ἴδητε οὐκέτι τὸν τόπον τοῦτον.
\par }{\PP \VS{19}Ἃ ἐλάλησε Κύριος ἐφʼ ὑμᾶς τοὺς καταλοίπους Ἰούδα· μὴ εἰσέλθητε εἰς Αἴγυπτον· καὶ νῦν γνόντες γνώσεσθε,
\VS{20}ὅτι ἐπονηρεύσασθε ἐν ψυχαῖς ὑμῶν, ἀποστείλαντές με, λέγοντες, πρόσευξαι περὶ ἡμῶν πρὸς Κύριον, καὶ κατὰ πάντα ἃ ἐὰν λαλήσει σοι Κύριος ποιήσομεν.
\VS{21}Καὶ οὐκ ἠκούσατε τῆς φωνῆς Κυρίου, ἧς ἀπέστειλέ με πρὸς ὑμᾶς.
\VS{22}Καὶ νῦν ἐν ῥομφαίᾳ καὶ ἐν λιμῷ ἐκλείψετε ἐν τῷ τόπῳ ᾧ ὑμεῖς βούλεσθε εἰσελθεῖν κατοικεῖν ἐκεῖ.

\par }\Chap{50}{\PP \VerseOne{1}Καὶ ἐγενήθη, ὡς ἐπαύσατο Ἱερεμίας λέγων πρὸς τὸν λαὸν πάντας τοὺς λόγους Κυρίου, οὓς ἀπέστειλεν αὐτὸν Κύριος πρὸς αὐτοὺς, πάντας τοὺς λόγους τούτους,
\VS{2}καὶ εἶπεν Ἀζαρίας υἱὸς Μαασαίου, καὶ Ἰωάναν υἱὸς Κάρηε, καὶ πάντες οἱ ἄνδρες, οἱ εἰπόντες τῷ Ἱερεμίᾳ, λέγοντες, ψεύδη, οὐκ ἀπέστειλέ σε Κύριος πρὸς ἡμᾶς, λέγων, μὴ εἰσέλθητε εἰς Αἴγυπτον οἰκεῖν ἐκεῖ,
\VS{3}ἀλλʼ ἢ Βαροὺχ υἱὸς Νηρίου συμβάλλει σε πρὸς ἡμᾶς, ἵνα δῷς ἡμᾶς εἰς χεῖρας τῶν Χαλδαίων, τοῦ θανατῶσαι ἡμᾶς, καὶ ἀποικισθῆναι ἡμᾶς εἰς Βαβυλῶνα.
\VS{4}Καὶ οὐκ ἤκουσεν Ἰωάναν, καὶ πάντες ἡγεμόνες τῆς δυνάμεως, καὶ πᾶς ὁ λαὸς τῆς φωνῆς Κυρίου, κατοικῆσαι ἐν γῇ Ἰούδα.
\par }{\PP \VS{5}Καὶ ἔλαβεν Ἰωάναν, καὶ πάντες οἱ ἡγεμόνες τῆς δυνάμεως πάντας τοὺς καταλοίπους Ἰούδα, τοὺς ἀποστρέψαντας κατοικεῖν ἐν τῇ γῃ,
\VS{6}τοὺς δυνατοὺς ἄνδρας, καὶ τὰς γυναῖκας, καὶ τὰ νήπια τὰ λοιπὰ, καὶ τὰς θυγατέρας τοῦ βασιλέως, καὶ τὰς ψυχὰς ἃς κατέλιπε Ναβουζαρδὰν μετὰ Γοδολίου υἱοῦ Ἀχεικὰμ, καὶ Ἱερεμίαν τὸν προφήτην, καὶ Βαροὺχ, υἱὸν Νηρίου,
\VS{7}καὶ εἰσῆλθον εἰς Αἴγυπτον, ὅτι οὐκ ἤκουσαν τῆς φωνῆς Κυρίου, καὶ εἰσῆλθον εἰς Τάφνας.
\par }{\PP \VS{8}Καὶ ἐγένετο λόγος Κυρίου πρὸς Ἱερεμίαν ἐν Τάφνας, λέγων,
\VS{9}λάβε σεαυτῷ λίθους μεγάλους, καὶ κατάκρυψον αὐτοὺς ἐν προθύροις, ἐν πύλῃ τῆς οἰκίας Φαραὼ ἐν Τάφνας, κατʼ ὀφθαλμοὺς ἀνδρῶν Ἰούδα,
\VS{10}καὶ ἐρεῖς, οὕτως εἶπε Κύριος, ἰδοὺ ἐγὼ ἀποστέλλω, καὶ ἄξω Ναβουχοδονόσορ βασιλέα Βαβυλῶνος, καὶ θήσει αὐτοῦ τὸν θρόνον ἐπάνω τῶν λίθων τούτων ὧν κατέκρυψας, καὶ ἀρεῖ τᾶ ὅπλα ἐπʼ αὐτοὺς,
\VS{11}καὶ εἰσελεύσεται, καὶ πατάξει γῆν Αἰγύπτου, οὓς εἰς θάνατον εἰς θάνατον, καὶ οὓς εἰς ἀποικισμὸν εἰς ἀποικισμὸν, καὶ οὓς εἰς ῥομφαίαν εἰς ῥομφαίαν.
\VS{12}Καὶ καύσει πῦρ ἐν οἰκίαις τῶν θεῶν αὐτῶν, καὶ ἐμπυριεῖ αὐτὰς, καὶ ἀποικιεῖ αὐτοὺς, καὶ φθειριεῖ γῆν Αἰγύπτου, ὥσπερ φθειρίζει ποιμὴν τὸ ἱμάτιον αὐτοῦ· καὶ ἐξελεύσεται ἐν εἰρήνῃ,
\VS{13}καὶ συντρίψει τοὺς στύλους Ἡλιουπόλεως τοὺς ἐν Ὢν, καὶ τὰς οἰκίας αὐτῶν κατακαύσει ἐν πυρί.

\par }\Chap{51}{\PP \VerseOne{1}Ὁ ΛΟΓΟΣ Ὁ ΓΕΝΟΜΕΝΟΣ ΠΡΟΣ ἹΕΡΕΜΙΑΝ ἅπασι τοῖς Ἰουδαίοις τοῖς κατοικοῦσιν ἐν γῇ Αἰγύπτου, καὶ τοῖς καθημένοις ἐν Μαγδωλῷ, καὶ ἐν Τάφνας, καὶ ἐν γῇ Παθούρης, λέγων,
\par }{\PP \VS{2}Οὕτως εἶπε Κύριος ὁ Θεὸς Ἰσραὴλ, ὑμεῖς ἑωράκατε πάντα τὰ κακὰ ἃ ἐπήγαγον ἐπὶ Ἱερουσαλὴμ, καὶ ἐπὶ τὰς πόλεις Ἰούδα· καὶ ἰδού εἰσιν ἔρημοι ἀπὸ ἐνοίκων
\VS{3}ἀπὸ προσώπου πονηρίας αὐτῶν, ἧς ἐποίησαν παραπικράναι με, πορευθέντες θυμιᾷν θεοῖς ἑτέροις, οἷς οὐκ ἔγνωτε.
\VS{4}Καὶ ἀπέστειλα πρὸς ὑμᾶς τοὺς παῖδάς μου τοὺς προφήτας ὄρθρου, καὶ ἀπέστειλα, λέγων, μὴ ποιήσητε τὸ πρᾶγμα τῆς μολύνσεως ταύτης, ἧς ἐμίσησα.
\par }{\PP \VS{5}Καὶ οὐκ ἤκουσάν μου, καὶ οὐκ ἔκλιναν τὸ οὖς αὐτῶν ἀποστρέψαι ἀπὸ τῶν κακῶν αὐτῶν, πρὸς τὸ μὴ θυμιᾷν θεοῖς ἑτέροις.
\VS{6}Καὶ ἔσταξεν ἡ ὀργή μου, καὶ ὁ θυμός μου, καὶ ἐξεκαύθη ἐν πύλαις Ἰούδα, καὶ ἔξωθεν Ἱερουσαλήμ· καὶ ἐγενήθησαν εἰς ἐρήμωσιν καὶ εἰς ἄβατον ὡς ἡ ἡμέρα αὕτη.
\par }{\PP \VS{7}Καὶ νῦν οὕτως εἶπε Κύριος παντοκράτωρ, ἱνατί ὑμεῖς ποιεῖτε κακὰ μεγάλα ἐπὶ ψυχαῖς ὑμῶν; ἐκκόψαι ὑμῶν ἄνθρωπον καὶ γυναῖκα, νήπιον καὶ θηλάζοντα ἐκ μέσου Ἰούδα, πρὸς τὸ μὴ καταλειφθῆναι ὑμῶν μηδένα,
\VS{8}παραπικράναι με ἐν τοῖς ἔργοις τῶν χειρῶν ὑμῶν, θυμιᾷν θεοῖς ἑτέροις ἐν γῇ Αἰγύπτῳ, εἰς ἣν ἤλθετε κατοικεῖν ἐκεῖ, ἵνα κοπῆτε, καὶ ἵνα γένησθε εἰς κατάραν καὶ εἰς ὀνειδισμὸν ἐν πᾶσι τοῖς ἔθνεσι τῆς γῆς;
\VS{9}Μὴ ἐπιλέλησθε ὑμεῖς τῶν κακῶν τῶν πατέρων ὑμῶν, καὶ τῶν κακῶν τῶν βασιλέων Ἰούδα, καὶ τῶν κακῶν τῶν ἀρχόντων ὑμῶν, καὶ τῶν κακῶν τῶν γυναικῶν ὑμῶν, ὧν ἐποίησαν ἐν γῇ Ἰούδα, καὶ ἔξωθεν Ἱερουσαλήμ;
\VS{10}Καὶ οὐκ ἐπαύσαντο ἕως τῆς ἡμέρας ταύτης, καὶ οὐκ ἀντείχοντο τῶν προσταγμάτων μου, ὧν ἔδωκα κατὰ πρόσωπον τῶν πατέρων αὐτῶν.
\par }{\PP \VS{11}Διατοῦτο οὕτως εἶπε Κύριος, ἰδοὺ ἐγὼ ἐφίστημι τὸ πρόσωπόν μου,
\VS{12}τοῦ ἀπολέσαι πάντας τοὺς καταλοίπους τοὺς ἐν Αἰγύπτῳ, καὶ πεσοῦνται ἐν ῥομφαίᾳ καὶ ἐν λιμῷ, καὶ ἐκλείψουσιν ἀπὸ μικροῦ ἕως μεγάλου, καὶ ἔσονται εἰς ὀνειδισμὸν, καὶ εἰς ἀπώλειαν, καὶ εἰς κατάραν.
\VS{13}Καὶ ἐπισκέψομαι ἐπὶ τοὺς καθημένους ἐν γῇ Αἰγύπτῳ, ὡς ἐπεσκεψάμην ἐπὶ Ἱερουσαλὴμ, ἐν ῥομφαίᾳ καὶ ἐν λιμῷ,
\VS{14}καὶ οὐκ ἔσται σεσωσμένος οὐδεὶς τῶν ἐπιλοίπων Ἰούδα τῶν παροικούντων ἐν γῇ Αἰγύπτῳ, τοῦ ἐπιστρέψαι εἰς γῆν Ἰούδα, ἐφʼ ἣν αὐτοὶ ἐλπίζουσι ταῖς ψυχαῖς αὐτῶν τοῦ ἐπιστρέψαι ἐκεῖ· οὐ μὴ ἐπιστρέψωσιν, ἀλλʼ ἢ οἱ ἀνασεσωσμένοι.
\par }{\PP \VS{15}Καὶ ἀπεκρίθησαν τῷ Ἱερεμίᾳ πάντες οἱ ἄνδρες οἱ γνόντες ὅτι θυμιῶσιν αἱ γυναῖκες αὐτῶν, καὶ πᾶσαι αἱ γυναῖκες, συναγωγὴ μεγάλη, καὶ πᾶς ὁ λαὸς οἱ καθήμενοι ἐν γῇ Αἰγύπτῳ, ἐν Παθουρῇ, λέγοντες,
\par }{\PP \VS{16}Ὁ λόγος ὃν ἐλάλησας πρὸς ἡμᾶς τῷ ὀνόματι Κυρίου, οὐκ ἀκούσομέν σου,
\VS{17}ὅτι ποιοῦντες ποιήσομεν πάντα τὸν λόγον ὃς ἐξελεύσεται ἐκ τοῦ στόματος ἡμῶν, θυμιᾷν τῇ βασιλίσσῃ τοῦ οὐρανοῦ, καὶ σπένδειν αὐτῇ σπονδὰς, καθὰ ἐποιήσαμεν ἡμεῖς καὶ οἱ πατέρες ἡμῶν, καὶ οἱ βασιλεῖς ἡμῶν, καὶ οἱ ἄρχοντες ἡμῶν, ἐν πόλεσιν Ἰούδα καὶ ἔξωθεν Ἱερουσαλήμ· καὶ ἐπλήσθημεν ἄρτων, καὶ ἐγενόμεθα χρηστοὶ, καὶ κακὰ οὐκ εἴδομεν.
\VS{18}Καὶ ὡς διελίπομεν θυμιῶντες τῇ βασιλίσσῃ τοῦ οὐρανοῦ, ἠλαττώθημεν πάντες, καὶ ἐν ῥομφαίᾳ καὶ ἐν λιμῷ ἐξελίπομεν.
\VS{19}Καὶ ὅτι ἡμεῖς ἐθυμιῶμεν τῇ βασιλίσσῃ τοῦ οὐρανοῦ, καὶ ἐσπείσαμεν αὐτῇ σπονδὰς, μὴ ἄνευ τῶν ἀνδρῶν ἡμῶν ἐποιήσαμεν αὐτῇ χαυῶνας, καὶ ἐσπείσαμεν αὐτῇ σπονδάς;
\par }{\PP \VS{20}Καὶ εἶπεν Ἱερεμίας παντὶ τῷ λαῷ, τοῖς δυνατοῖς καὶ ταῖς γυναιξὶ καὶ παντὶ τῷ λαῷ τοῖς ἀποκριθεῖσιν αὐτῷ λόγους, λέγων,
\VS{21}οὐχὶ τοῦ θυμιάματος οὗ ἐθυμιάσατε ἐν ταῖς πόλεσιν Ἰούδα, καὶ ἔξωθεν Ἱερουσαλὴμ, ὑμεῖς καὶ οἱ πατέρες ὑμῶν, καὶ οἱ βασιλεῖς ὑμῶν, καὶ οἱ ἄρχοντες ὑμῶν, καὶ ὁ λαὸς τῆς γῆς, ἐμνήσθη Κύριος; καὶ ἀνέβη ἐπὶ τὴν καρδίαν αὐτοῦ;
\VS{22}Καὶ οὐκ ἠδύνατο Κύριος ἔτι φέρειν ἀπὸ προσώπου πονηρίας πραγμάτων ὑμῶν, καὶ ἀπὸ τῶν βδελυγμάτων ὑμῶν ὧν ἐποιήσατε· καὶ ἐγενήθη ἡ γῆ ὑμῶν εἰς ἐρήμωσιν, καὶ εἰς ἄβατον, καὶ εἰς ἀρὰν, ὡς ἐν τῇ ἡμέρᾳ ταύτῃ,
\VS{23}ἀπὸ προσώπου ὧν ἐθυμιᾶτε, καὶ ὧν ἡμάρτετε τῷ Κύριῳ· καὶ οὐκ ἠκούσατε τῆς φωνῆς Κυρίου, καὶ ἐν τοῖς προστάγμασιν αὐτοῦ, καὶ ἐν τῷ νόμῳ καὶ ἐν τοῖς μαρτυρίοις αὐτοῦ οὐκ ἐπορεύθητε, καὶ ἐπελάβετο ὑμῶν τὰ κακὰ ταῦτα.
\par }{\PP \VS{24}Καὶ εἶπεν Ἱερεμίας τῷ λαῷ, καὶ ταῖς γυναιξὶν, ἀκουσατε λόγον Κυρίου.
\VS{25}Οὕτως εἶπε Κύριος ὁ Θεὸς Ἰσραὴλ, ὑμεῖς γυναῖκες τῷ στόματι ὑμῶν ἐλαλήσατε, καὶ ταῖς χερσὶν ὑμῶν ἐπληρώσατε, λέγουσαι, ποιοῦσαι ποιήσομεν τὰς ὁμολογίας ἡμῶν ἃς ὡμολογήκαμεν, θυμιᾷν τῇ βασιλίσσῃ τοῦ οὐρανοῦ καὶ σπένδειν αὐτῇ σπονδάς· ἐμμείνασαι ἐνεμείνατε ταῖς ὁμολογίαις ὑμῶν, καὶ ποιοῦσαι ἐποιήσατε.
\VS{26}Διατοῦτο ἀκούσατε λόγον Κυρίου, πᾶς Ἰούδα οἱ καθήμενοι ἐν γῇ Αἰγύπτῳ, ἰδοὺ ὤμοσα τῷ ὀνόματί μου τῷ μεγάλῳ, εἶπε Κύριος, ἐὰν γένηται ἔτι ὄνομά μου ἐν τῷ στόματι παντὸς Ἰούδα εἰπεῖν, ζῇ Κύριος, ἐπὶ πάση γῇ Αἰγύπτῳ.
\VS{27}Ὅτι ἐγὼ ἐγρήγορα ἐπʼ αὐτοὺς, τοῦ κακῶσαι αὐτοὺς, καὶ οὐκ ἀγαθῶσαι· καὶ ἐκλείψουσι πᾶς Ἰούδα, οἱ κατοικοῦντες ἐν γῇ Αἰγύπτῳ, ἐν ῥομφαίᾳ καὶ ἐν λιμῷ, ἕως ἂν ἐκλείπωσι.
\VS{28}Καὶ οἱ σεσωσμένοι ἀπὸ ῥομφαίας ἐπιστρέψουσιν εἰς γῆν Ἰούδα ὀλίγοι ἀριθμῷ, καὶ γνώσονται οἱ κατάλοιποι Ἰούδα οἱ καταστάντες ἐν γῇ Αἰγύπτῳ, κατοικῆσαι ἐκεῖ, λόγος τίνος ἐμμενεῖ.
\par }{\PP \VS{29}Καὶ τοῦτο τὸ σημεῖον ὑμῖν, ὅτι ἐπισκέψομαι ἐγὼ ἐφʼ ὑμᾶς εἰς πονηρά.
\VS{30}Οὕτως εἶπε Κύριος, ἰδοὺ ἐγὼ δίδωμι τὸν Οὐαφρῆ βασιλέα Αἰγύπτου εἰς χεῖρας ἐχθροῦ αὐτοῦ, καὶ εἰς χεῖρας ζητοῦντος τὴν ψυχὴν αὐτοῦ, καθὰ ἔδωκα τὸν Σεδεκίαν βασιλέα Ἰούδα εἰς χεῖρας Ναβουχοδονόσορ βασιλέως Βαβυλῶνος ἐχθροῦ αὐτοῦ, καὶ ζητοῦντος τὴν ψυχὴν αὐτοῦ.
\par }{\PP \VS{31}Ὁ ΛΟΓΟΣ ὋΝ ἘΛΑΛΗΣΕΝ ἹΕΡΕΜΙΑΣ Ὁ ΠΡΟΦΗΤΗΣ πρὸς Βαροὺχ υἱὸν Νηρίου, ὅτε ἔγραφε τοὺς λόγους τούτους ἐν τῷ βιβλίῳ ἀπὸ στόματος Ἱερεμίου, ἐν τῷ ἐνιαυτῷ τῷ τετάρτῳ Ἰωακεὶμ υἱῷ Ἰωσία βασιλέως Ἰούδα.
\par }{\PP \VS{32}Οὕτως εἶπε Κύριος ἐπὶ σοὶ Βαροὺχ,
\VS{33}ὅτι εἶπας, οἴμοι οἴμοι, ὅτι προσέθηκε Κύριος κόπον ἐπίπονόν μοι, ἐκοιμήθην ἐν στεναγμοῖς, ἀνάπαυσιν οὐχ εὗρον·
\VS{34}Εἰπὸν αὐτῷ, οὕτως εἶπε Κύριος, ἰδοὺ οὓς ἐγὼ ᾠκοδόμησα, ἐγὼ καθαιρῶ· καὶ οὓς ἐγὼ ἐφύτευσα, ἐγὼ ἐκτίλλω.
\VS{35}Καὶ σὺ ζητήσεις σεαυτῷ μεγάλα; μὴ ζητήσῃς, ὅτι ἰδοὺ ἐγὼ ἐπάγω κακὰ ἐπὶ πᾶσαν σάρκα, λέγει Κύριος· καὶ δώσω τὴν ψυχήν σου εἰς εὕρημα ἐν παντὶ τόπῳ οὗ ἐὰν βαδίσῃς ἐκεῖ.

\par }\Chap{52}{\PP \VerseOne{1}Ὄντος εἰκοστοῦ καὶ ἑνὸς ἔτους Σεδεκίου, ἐν τῷ βασιλεύειν αὐτὸν, καὶ ἕνδεκα ἔτη ἐβασίλευσεν ἐν Ἱερουσαλήμ· καὶ ὄνομα τῇ μητρὶ αὐτοῦ Ἀμειτάαλ, θυγάτηρ Ἱερεμίου, ἐκ Λοβενά.
\par }{\PP \VS{4}Καὶ ἐγένετο τῷ ἔτει τῷ ἐννάτῳ τῆς βασιλείας αὐτοῦ, ἐν μηνὶ τῷ ἐννάτῳ, δεκάτῃ τοῦ μηνὸς, ἦλθε Ναβουχοδονόσορ βασιλεὺς Βαβυλῶνος, καὶ πᾶσα ἡ δύναμις αὐτοῦ ἐπὶ Ἱερουσαλὴμ, καὶ περιεχαράκωσαν αὐτὴν, καὶ περιῳκοδόμησαν αὐτὴν τετραπέδοις λίθοις κύκλῳ.
\par }{\PP \VS{5}Καὶ ἦλθεν ἡ πόλις εἰς συνοχὴν, ἕως ἑνδεκάτου ἔτους τῷ βασιλεῖ Σεδεκίᾳ,
\VS{6}ἐν τῇ ἐννάτῃ τοῦ μηνὸς, καὶ ἐστερεώθη ὁ λιμὸς ἐν τῇ πόλει, καὶ οὐκ ἦσαν ἄρτοι τῷ λαῷ τῆς γῆς.
\VS{7}Καὶ διεκόπη ἡ πόλις, καὶ πάντες οἱ ἄνδρες οἱ πολεμισταὶ ἐξῆλθον νυκτὸς κατὰ τὴν ὁδὸν τῆς πύλης, ἀναμέσον τοῦ τείχους, καὶ τοῦ προτειχίσματος, ὃ ἦν κατὰ τὸν κῆπον τοῦ βασιλέως, καὶ οἱ Χαλδαῖοι ἐπὶ τῆς πόλεως κύκλῳ, καὶ ἐπορεύθησαν ὁδὸν τὴν εἰς ἄραβα,
\VS{8}Καὶ κατεδίωξεν ἡ δύναμις τῶν Χαλδαίων ὀπίσω τοῦ βασιλέως, καὶ κατέλαβον αὐτὸν ἐν τῷ πέραν Ἱερειχὼ, καὶ πάντες οἱ παῖδες αὐτοῦ διεσπάρησαν ἀπʼ αὐτοῦ.
\VS{9}Καὶ συνέλαβον τὸν βασιλέα, καὶ ἤγαγον αὐτὸν πρὸς τὸν βασιλέα Βαβυλῶνος εἰς Δεβλαθὰ, καὶ ἐλάλησεν αὐτῷ μετὰ κρίσεως.
\VS{10}Καὶ ἔσφαξε βασιλεὺς Βαβυλῶνος τοὺς υἱοὺς Σεδεκίου κατʼ ὀφθαλμοὺς αὐτοῦ, καὶ πάντας τοὺς ἄρχοντας Ἰούδα ἔσφαξεν ἐν Δεβλαθά.
\VS{11}Καὶ τοὺς ὀφθαλμοὺς Σεδεκίου ἐξετύφλωσε, καὶ ἔδησεν αὐτὸν ἐν πέδαις· καὶ ἤγαγεν αὐτὸν βασιλεὺς Βαβυλῶνος εἰς Βαβυλῶνα, καὶ ἔδωκεν αὐτὸν εἰς οἰκίαν μύλωνος, ἕως ἡμέρας ἧς ἀπέθανε.
\par }{\PP \VS{12}Καὶ ἐν μηνὶ πέμπτῳ, δεκάτῃ τοῦ μηνὸς, ἦλθε Ναβουζαρδὰν ὁ ἀρχιμάγειρος, ἑστηκὼς κατὰ πρόσωπον τοῦ βασιλέως Βαβυλῶνος, εἰς Ἱερουσαλὴμ,
\VS{13}καὶ ἐνέπρησε τὸν οἶκον Κυρίου, καὶ τὸν οἶκον τοῦ βασιλέως, καὶ πάσας τὰς οἰκίας τῆς πόλεως, καὶ πᾶσαν οἰκίαν μεγάλην ἐνέπρησεν ἐν πυρί·
\VS{14}Καὶ πᾶν τεῖχος Ἱερουσαλὴμ κύκλῳ καθεῖλεν ἡ δύναμις τῶν Χαλδαίων, ἡ μετὰ τοῦ ἀρχιμαγείρου.
\VS{16}Καὶ τοὺς καταλοίπους τοῦ λαοῦ κατέλιπεν ὁ ἀρχιμάγειρος εἰς ἀμπελουργοὺς καὶ εἰς γεωργούς.
\par }{\PP \VS{17}Καὶ τοὺς στύλους τοὺς χαλκοῦς τοὺς ἐν οἴκῳ Κυρίου, καὶ τὰς βάσεις, καὶ τὴν θάλασσαν τὴν χαλκῆν τὴν ἐν οἴκῳ Κυρίου συνέτριψαν οἱ Χαλδαῖοι, καὶ ἔλαβον τὸν χαλκὸν αὐτῶν, καὶ ἀπήνεγκαν εἰς Βαβυλῶνα.
\VS{18}Καὶ τὴν στεφάνην, καὶ τὰς φιάλας, καὶ τὰς κρεάγρας, καὶ πάντα τὰ σκεύη τὰ χαλκᾶ, ἐν οἷς ἐλειτούργουν ἐν αὐτοῖς,
\VS{19}καὶ τὰς ἀπφὼθ, καὶ τὰς μασμαρὼθ, καὶ τοὺς ὑποχυτῆρας, καὶ τὰς λυχνίας, καὶ τὰς θυΐσκας, καὶ τοὺς κυάθους, ἃ ἦν χρυσᾶ χρυσᾶ, καὶ ἃ ἦν ἀργυρᾶ ἀργυρᾶ, ἔλαβεν ὁ ἀρχιμάγειρος.
\VS{20}Καὶ οἱ στύλοι δύο, καὶ ἡ θάλασσα μία, καὶ οἱ μόσχοι δώδεκα χαλκοῖ ὑποκάτω τῆς θαλάσσης, ἃ ἐποίησεν ὁ βασιλεὺς Σαλωμὼν εἰς οἶκον Κυρίου, οὗ οὐκ ἦν σταθμὸς τοῦ χαλκοῦ αὐτῶν.
\par }{\PP \VS{21}Καὶ οἱ στύλοι τριακονταπέντε πηχῶν ὕψος τοῦ στύλου τοῦ ἑνὸς, καὶ σπαρτίον δώδεκα πήχεων περιεκύκλου αὐτὸν, καὶ τὸ πάχος αὐτοῦ δακτύλων τεσσάρων κύκλῳ,
\VS{22}καὶ γεῖσος ἐπʼ αὐτοῖς χαλκοῦν, καὶ πέντε πήχεων τὸ μῆκος, ὑμεροχὴ τοῦ γείσους τοῦ ἑνός, καὶ δίκτυον καὶ ῥοαὶ ἐπὶ τοῦ γείσους κύκλῳ τὰ πάντα χαλκᾶ, καὶ κατὰ ταῦτα τῷ στύλῳ τῷ δευτέρῳ ὀκτὼ ῥοαὶ τῷ πήχει τοῖς δώδεκα πήχεσι.
\VS{23}Καὶ ἦσαν αἱ ῥοαὶ ἐννενηκονταὲξ τὸ ἓν μέρος, καὶ ἦσαν αἱ πᾶσαι ῥοαὶ ἑκατὸν ἐπὶ τοῦ δικτύου κύκλῳ.
\par }{\PP \VS{24}Καὶ ἔλαβεν ὁ ἀρχιμάγειρος τὸν ἱερέα τὸν πρῶτον, καὶ τὸν ἱερέα τὸν δευτεροῦντα, καὶ τοὺς φυλάττοντας τὴν ὁδὸν,
\VS{25}καὶ εὐνοῦχον ἕνα ὃς ἦν ἐπιστάτης τῶν ἀνδρῶν τῶν πολεμιστῶν, καὶ εὐνοῦχον ἕνα ὃς ἦν ἐπιστάτης ἀνδρῶν τῶν πολεμιστῶν, καὶ ἑπτὰ ἄνδρας ὀνομαστοὺς, τοὺς ἐν προσώπῳ τοῦ βασιλέως, τοὺς εὑρεθέντας ἐν τῇ πόλει, καὶ τὸν γραμματέα τῶν δυνάμεων, τὸν γραμματεύοντα τῷ λαῷ τῆς γῆς, καὶ ἑξήκοντα ἀνθρώπους ἐκ τοῦ λαοῦ τῆς γῆς, τοὺς εὑρεθέντας ἐν μέσῳ τῆς πόλεως·
\VS{26}Καὶ ἔλαβεν αὐτοὺς Ναβουζαρδὰν ὁ ἀρχιμάγειρος τοῦ βασιλέως, καὶ ἢγαγεν αὐτοὺς πρὸς βασιλέα Βαβυλῶνος εἰς Δεβλαθά.
\VS{27}Καὶ ἐπάταξεν αὐτοὺς βασιλεὺς Βαβυλῶνος ἐν Δεβλαθὰ, ἐν γῇ Αἱμάθ.
\par }{\PP \VS{31}Καὶ ἐγένετο ἐν τῷ τριακοστῷ καὶ ἑβδόμῳ ἔτει, ἀποικισθέντος τοῦ Ἰωακεὶμ βασιλέως Ἰούδα, ἐν τῷ δωδεκάτῳ μηνὶ, ἐν τῇ τετράδι καὶ εἰκάδι τοῦ μηνὸς, ἔλαβεν Οὐλαιμαδάχὰρ βασιλεὺς Βαβυλῶνος, ἐν τῷ ἐνιαυτῷ ᾧ ἐβασίλευσε, τὴν κεφαλὴν Ἰωακεὶμ βασιλέως Ἰούδα, καὶ ἔκειρεν αὐτὸν, καὶ ἐξήγαγεν αὐτὸν ἐξ οἰκίας ἧς ἐφυλάσσετο,
\VS{32}καὶ ἐλάλησεν αὐτῷ χρηστὰ, καὶ ἔδωκε τὸν θρόνον αὐτοῦ ἐπάνω τῶν βασιλέων τῶν μετʼ αὐτοῦ ἐν Βαβυλῶνι,
\VS{33}καὶ ἤλλαξε τὴν στολὴν τῆς φυλακῆς αὐτοῦ, καὶ ἤσθιεν ἄρτον διαπαντὸς κατὰ πρόσωπον αὐτοῦ πάσας τὰς ἡμέρας ἃς ἔζησε.
\VS{34}Καὶ ἡ σύνταξις αὐτῷ ἐδίδοτο διαπαντὸς παρὰ τοῦ βασιλέως Βαβυλῶνος ἐξ ἡμέρας εἰς ἡμέραν, ἕως ἡμέρας ἧς ἀπέθανε.
\par }