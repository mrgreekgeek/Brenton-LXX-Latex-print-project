\NormalFont\ShortTitle{ΕΞΟΔΟΣ}
{\MT ΕΞΟΔΟΣ

\par }\ChapOne{1}{\PP \VerseOne{1}ΤΑΥΤΑ τὰ ὀνόματα τῶν υἱῶν Ἰσραὴλ τῶν εἰσπεπορευμένων εἰς Αἴγυπτον ἅμα Ἰακὼβ τῷ πατρὶ αὐτῶν, ἕκαστος πανοικὶ αὐτῶν εἰσήλθοσαν.
\VS{2}Ῥουβὴν, Συμεών, Λευὶ, Ἰούδας,
\VS{3}Ἰσσάχαρ, Ζαβουλὼν, Βενιαμὶν,
\VS{4}Δὰν, καὶ Νεφθαλὶ, Γὰδ, καὶ Ἀσήρ.
\VS{5}Ἰωσὴφ δὲ ἦν ἐν Αἰγύπτῳ· ἦσαν δὲ πᾶσαι ψυχαὶ ἐξ Ἰακὼβ, πέντε καὶ ἑβδομήκοντα.
\VS{6}Ἐτελεύτησε δὲ Ἰωσὴφ, καὶ πάντες οἱ ἀδελφοὶ αὐτοῦ, καὶ πᾶσα ἡ γενεὰ ἐκείνη.
\VS{7}Οἱ δὲ υἱοὶ Ἰσραὴλ ηὐξήθησαν, καὶ ἐπληθύνθησαν, καὶ χυδαῖοι ἐγένοντο, καὶ κατίσχυον σφόδρα σφόδρα· ἐπλήθυνε δὲ ἡ γῆ αὐτούς.
\VS{8}Ἀνέστη δὲ βασιλεὺς ἕτερος ἐπʼ Αἴγυπτον, ὃς οὐκ ᾔδει τὸν Ἰωσήφ.
\VS{9}Εἶπε δὲ τῷ ἔθνει αὐτοῦ, ἰδοὺ τὸ γένος τῶν υἱῶν Ἰσραὴλ μέγα πλῆθος, καὶ ἰσχύει ὑπὲρ ἡμᾶς.
\VS{10}Δεῦτε οὖν κατασοφισώμεθα αὐτοὺς, μήποτε πληθυνθῇ, καὶ ἡνίκα ἂν συμβῇ ἡμῖν πόλεμος, προστεθήσονται καὶ οὗτοι πρὸς τοὺς ὑπεναντίους, καὶ ἐκπολεμήσαντες ἡμᾶς, ἐξελεύσονται ἐκ τῆς γῆς.
\VS{11}Καὶ ἐπέστησεν αὐτοῖς ἐπιστάτας τῶν ἔργων, ἵνα κακώσωσιν αὐτοὺς ἐν τοῖς ἔργοις. Καὶ ᾠκοδόμησαν πόλεις ὀχυρὰς τῷ Φαραῷ, τήν τε Πειθὼ, καὶ Ῥαμεσσῆ, καὶ Ὢν, ἥ ἐστιν Ἡλιού πολις.
\VS{12}Καθότι δὲ αὐτοὺς ἐταπείνουν, τοσούτῳ πλείους ἐγίνοντο, καὶ ἴσχυον σφόδρα σφόδρα· καὶ ἐβδελύσσοντο οἱ Αἰγύπτιοι ἀπὸ τῶν υἱῶν Ἰσραήλ.
\VS{13}Καὶ κατεδυνάστευον οἱ Αἰγύπτιοι τοὺς υἱοὺς Ἰσραὴλ βίᾳ.
\VS{14}Καὶ κατωδύνων αὐτῶν τὴν ζωὴν ἐν τοῖς ἔργοις τοῖς σκληροῖς, τῷ πηλῷ καὶ τῇ πλινθείᾳ, καὶ πᾶσι τοῖς ἔργοις τοῖς ἐν τοῖς πεδίοις, κατὰ πάντα τὰ ἔργα, ὧν κατεδουλοῦντο αὐτοὺς μετὰ βίας.
\par }{\PP \VS{15}Καὶ εἶπεν ὁ βασιλεὺς τῶν Αἰγυπτίων ταῖς μαίαις τῶν Ἐβραίων, τῇ μιᾷ αὐτῶν ὄνομα Σεπφώρα, καὶ τὸ ὄνομα τῆς δευτέρας Φουά·
\VS{16}Καὶ εἶπεν, ὅταν μαιοῦσθε τὰς Ἐβραίας, καὶ ὦσι πρὸς τῷ τίκτειν, ἐὰν μὲν ἄρσεν ᾖ, ἀποκτείνατε αὐτό· ἐὰν δὲ θῆλυ, περιποιεῖσθε αὐτό.
\VS{17}Ἐφοβήθησαν δὲ αἱ μαῖαι τὸν Θεὸν, καὶ οὐκ ἐποίησαν καθότι συνέταξεν αὐταῖς ὁ βασιλεὺς Αἰγύπτου, καὶ ἐζωογόνουν τὰ ἄρσενα.
\VS{18}Ἐκάλεσε δὲ ὁ βασιλεὺς Αἰγύπτου τὰς μαίας, καὶ εἶπεν αὐταῖς, τί ὅτι ἐποιήσατε τὸ πρᾶγμα τοῦτο, καὶ ἐζωογονεῖτε τὰ ἄρσενα;
\VS{19}Εἶπαν δὲ αἱ μαῖαι τῷ Φαραῷ, οὐχ ὡς γυναῖκες Αἰγύπτου αἱ Ἐβραῖαι· τίκτουσι γὰρ πρὶν ἢ εἰσελθεῖν πρὸς αὐτὰς τὰς μαίας· καὶ ἔτικτον.
\VS{20}Εὖ δὲ ἐποίει ὁ Θεὸς ταῖς μαίαις· καὶ ἐπλήθυνεν ὁ λαὸς, καὶ ἴσχυε σφόδπα.
\VS{21}Ἐπεὶ δὲ ἐφοβοῦντο αἱ μαῖαι τὸν Θεὸν, ἐποίησαν ἑαυταῖς οἰκίας.
\VS{22}Συνέταξε δὲ Φαραὼ παντὶ τῷ λαῷ αὐτοῦ, λέγων, πᾶν ἄρσεν, ὃ ἐὰν τεχθῇ τοῖς Ἑβραίοις, εἰς τὸν ποταμὸν ῥίψατε, καὶ πᾶν θῆλυ, ζωογονεῖτε αὐτό.

\par }\Chap{2}{\PP \VerseOne{1}Ἦν δέ τις ἐκ τῆς φυλῆς Λευὶ, ὃς ἔλαβεν τῶν θυγατέρων Λευί.
\VS{2}Καὶ ἐν γαστρὶ ἔλαβε, καὶ ἔτεκεν ἄρσεν· ἰδόντες δὲ αὐτὸ ἀστεῖον, ἐσπέπασαν αὐτὸ μῆνας τρεῖς.
\VS{3}Ἐπεὶ δὲ οὐκ ἐδύναντο αὐτὸ ἔτι κρύπτειν, ἔλαβεν αὐτῷ ἡ μήτηρ αὐτοῦ θῖβιν, καὶ κατέχρισεν αὐτὴν ἀσφαλτοπίσσῃ, καὶ ἐνέβαλε τὸ παιδίον εἰς αὐτήν, καὶ ἔθηκεν αὐτὴν εἰς τὸ ἕλος παρὰ τὸν ποταμόν.
\VS{4}Καὶ κατεσκόπευεν ἡ ἀδελφὴ αὐτοῦ μακρόθεν, μαθεῖν τί τὸ ἀποβησόμενον αὐτῷ.
\par }{\PP \VS{5}Κατέβη δὲ ἡ θυγάτηρ Φαραὼ λούσασθαι ἐπὶ τὸν ποταμὸν, καὶ αἱ ἅβραι αὐτῆς παρεπορεύοντο παρὰ τὸν ποταμόν· καὶ ἰδοῦσα τὴν θίβιν ἐν τῷ ἕλει, ἀποστείλασα τὴν ἅβραν, ἀνείλατο αὐτήν.
\VS{6}Ἀνοίξασα δὲ ὁρᾷ παιδίον κλαῖον ἐν τῇ θίβει· καὶ ἐφείσατο αὐτοῦ ἡ θυγάτηρ Φαραὼ, καὶ ἔφη, ἀπὸ τῶν παιδίων τῶν Ἐβραίων τοῦτο.
\VS{7}Καὶ εἶπεν ἡ ἀδελφὴ αὐτοῦ τῇ θυγατρὶ Φαραὼ, θέλεις καλέσω σοι γυναῖκα τροφεύουσαν ἐκ τῶν Ἐβραίων, καὶ θηλάσει σαι τὸ παιδὶον σοι τὸ παιδίον;
\VS{8}Ἡ δὲ εἶπεν ἡ θυγάτηρ Φαραὼ, πορεύου· ἐλθοῦσα δὲ νεᾶνις ἐκάλεσε τὴν μητέρα τοῦ παιδίου.
\VS{9}Εἶπεν δὲ πρὸς αὐτὴν ἡ θυγάτηρ Φαραὼ, διατήρησόν μοι τὸ παιδίον τοῦτο, καὶ θήλασόν μοι αὐτὸ, ἐγὼ δὲ δώσω σοι τὸν μισθόν· ἔλαβε δὲ ἡ γυνὴ τὸ παιδίον, καὶ ἐθήλαζεν αὐτό.
\VS{10}Ἁδρυνθέντος δὲ τοῦ παιδίου, εἰσήγαγεν αὐτὸ πρὸς τὴν θυγατέρα Φαραὼ, καὶ ἐγενήθη αὐτῇ εἰς υἱόν· ἐπωνόμασε δὲ τὸ ὄνομα αὐτοῦ Μωυσῆν, λέγουσα, ἐκ τοῦ ὕδατος αὐτὸν ἀνειλόμην.
\par }{\PP \VS{11}Ἐγένετο δὲ ἐν ταῖς ἡμέραις ταῖς πολλαῖς ἐκείναις μέγας γενόμενος Μωυσῆς, ἐξῆλθε πρὸς τοὺς ἀδελφοὺς αὐτοῦ τοὺς υἱοὺς Ἰσραήλ· κατανοήσας δὲ τὸν πόνον αὐτῶν, ὁρᾷ ἄνθρωπον Αἰγύπτιον τύπτοντα τινὰ Ἐβραῖον, τῶν ἑαυτοῦ ἀδελφῶν τῶν υἱῶν Ἰσραήλ.
\VS{12}Περιβλεψάμενος δὲ ὧδε καὶ ὧδε οὐχ ὁρᾷ οὐδένα, καὶ πατάξας τὸν Αἰγύπτιον, ἔκρυψεν αὐτὸν ἐν τῇ ἄμμῳ.
\VS{13}Ἐξελθὼν δὲ τῇ ἡμέρᾳ τῇ δευτέρᾳ, ὁρᾷ δύο ἄνδρας Ἐβραίους διαπληκτιζομένους· καὶ λέγει τῷ ἀδικοῦντι, διὰ τί σὺ τύπτεις τὸν πλησίον;
\VS{14}Ὁ δὲ εἶπε, τίς σε κατέστησεν ἄρχοντα καὶ δικαστὴν ἐφʼ ἡμῶν; μὴ ἀνελεῖν με σὺ θέλεις, ὃν τρόπον ἀνεῖλες χθὲς τὸν Αἰγύπτιον; ἐφοβήθη δὲ Μωυσῆς, καὶ εἶπεν, εἰ οὕτως ἐμφανὲς γέγονε τὸ ῥῆμα τοῦτο.
\VS{15}Ἤκουσε δὲ Φαραὼ τὸ ῥῆμα τοῦτο, καὶ ἐζήτει ἀνελεῖν Μωυσῆν. Ἀνεχώρησε δὲ Μωυσῆς ἀπὸ προσώπου Φαραὼ, καὶ ᾤκησεν ἐν γῇ Μαδιάμ· ἐλθὼν δὲ εἰς γῆν Μαδιὰμ, ἐκάθισεν ἐπὶ τοῦ φρέατος.
\VS{16}Τῷ δὲ ἱερεῖ Μαδιὰμ ἦσαν ἑπτὰ θυγατέρες, ποιμαίνουσαι τὰ πρόβατα τοῦ πατρὸς αὐτῶν Ἰοθόρ· παραγενόμεναι δὲ ἤντλουν, ἕως ἔπλησαν τὰς δεξαμενάς, ποτίσαι τὰ πρόβατα τοῦ πατρὸς αὐτῶν Ἰοθόρ.
\VS{17}Παραγενόμενοι δὲ οἱ ποιμένες ἐξέβαλλον αὐτάς· ἀναστὰς δὲ Μωυσῆς ἐῤῥύσατο αὐτὰς, καὶ ἤντλησεν αὐταῖς, καὶ ἐπότισε τὰ πρόβατα αὐτῶν.
\VS{18}Παρεγένοντο δὲ πρὸς Ῥαγουὴλ τὸν πατέρα αὐτῶν· ὁ δὲ εἶπεν αὐταῖς, διατί ἐταχύνατε τοῦ παραγενέσθαι σήμερον;
\VS{19}Αἱ δὲ εἶπαν, ἄνθρωπος Αἰγύπτιος ἐῤῥύσατο ἡμᾶς ἀπὸ τῶν ποιμένων, καὶ ἤντλησεν ἡμῖν, καὶ ἐπότισε τὰ πρόβατα ἡμῶν.
\VS{20}Ὁ δὲ εἶπε ταῖς θυγατράσιν αὐτοῦ, καὶ ποῦ ἐστιν; καὶ ἱνατί καταλελοίπατε τὸν ἄνθρωπον; καλέσατε οὖν αὐτὸν, ὅπως φάγῃ ἄρτον.
\VS{21}Κατῳκίσθη δὲ Μωυσῆς παρὰ τῷ ἀνθρώπῳ· καὶ ἐξέδοτο Σεπφώραν τὴν θυγατέρα αὐτοῦ Μωυσῇ γυναῖκα.
\VS{22}Ἐν γαστρὶ δὲ λαβοῦσα ἡ γυνὴ ἔτεκεν υἱόν· καὶ ἐπωνόμασε Μωυσῆς τὸ ὄνομα αὐτοῦ Γηρσάμ, λέγων, ὅτι παροικός εἰμι ἐν γῇ ἀλλοτρίᾳ.
\VS{23}Μετὰ δὲ τὰς ἡμέρας τὰς πολλὰς ἐκείνας, ἐτελεύτησεν ὁ βασιλεὺς Αἰγύπτου, καὶ κατεστέναξαν οἱ υἱοὶ Ἰσραὴλ ἀπὸ τῶν ἔργων, καὶ ἀνεβόησαν· καὶ ἀνέβη ἡ βοὴ αὐτῶν πρὸς τὸν Θεὸν ἀπὸ τῶν ἔργων.
\VS{24}Καὶ εἰσήκουσεν ὁ Θεὸς τὸν στεναγμὸν αὐτῶν· καὶ ἐμνήσθη ὁ Θεὸς τῆς διαθήκης αὐτοῦ τῆς πρὸς Ἀβραὰμ, καὶ Ἰσαὰκ, καὶ Ἰακώβ.
\VS{25}Καὶ ἐπεῖδεν ὁ Θεὸς τοὺς υἱοὺς Ἰσραὴλ, καὶ ἐγνώσθη αὐτοῖς.

\par }\Chap{3}{\PP \VerseOne{1}Καὶ Μωυσῆς ἦν ποιμαίνων τὰ πρόβατα Ἰοθὸρ τοῦ γαμβροῦ αὐτοῦ, τοῦ ἱερέως Μαδιὰμ, καὶ ἤγαγεν τὰ πρόβατα ὑπὸ τὴν ἔρημον, καὶ ἦλθεν εἰς τὸ ὄρος Χωρήβ.
\VS{2}Ὤφθη δὲ αὐτῷ Ἄγγελος Κυρίου ἐν πυρὶ φλογὸς ἐκ τοῦ βάτου· καὶ ὁρᾷ ὅτι ὁ βάτος καίεται πυρί, ὁ δὲ βάτος οὐ κατεκαίετο.
\VS{3}Εἶπε δὲ Μωυσῆς, παρελθὼν ὄψομαι τὸ ὅραμα τὸ μέγα τοῦτο, ὅτι οὐ κατακαίεται ὁ βάτος.
\VS{4}Ὡς δὲ εἶδεν Κύριος ὅτι προσάγει ἰδεῖν, ἐκάλεσεν αὐτὸν Κύριος ἐκ τοῦ βάτου, λέγων, Μωυσῆ, Μωυσῆ· ὁ δὲ εἶπε, τί ἐστιν;
\VS{5}Ὁ δὲ εἶπε, μὴ ἐγγίσῃς ὧδε· λύσαι τὸ ὑπόδημα ἐκ τῶν ποδῶν σου, ὁ γὰρ τόπος. ἐν ᾧ σὺ ἕστηκας, γῆ ἁγία ἐστί.
\VS{6}Καὶ εἶπεν, ἐγώ εἰμι ὁ Θεὸς τοῦ πατρός σου, Θεὸς Ἁβραὰμ, καὶ Θεὸς Ἰσαὰκ, καὶ Θεὸς Ἰακώβ· ἀπέστρεψε δὲ Μωυσῆς τὸ πρόσωπον αὐτοῦ, εὐλαβεῖτο γὰρ κατεμβλέψαι ἐνώπιον τοῦ Θεοῦ.
\VS{7}Εἶπε δὲ Κύριος πρὸς Μωυσῆν, ἰδὼν εἶδον τὴν κάκωσιν τοῦ λαοῦ μου τοῦ ἐν Αἰγύπτῳ, καὶ τῆς κραυγῆς αὐτῶν ἀκήκοα ἀπὸ τῶν ἐργοδιωκτῶν· οἶδα γὰρ τὴν ὀδύνην αὐτων,
\VS{8}καὶ κατέβην ἐξελέσθαι αὐτοὺς ἐκ χειρὸς τῶν Αἰγυπτίων, καὶ ἐξαγαγεῖν αὐτοὺς ἐκ τῆς γῆς ἐκείνης, καὶ εἰσαγαγεῖν αὐτοὺς εἰς γῆν ἀγαθὴν καὶ πολλήν, εἰς γῆν ῥέουσαν γάλα καὶ μέλι, εἰς τὸν τόπον τῶν Χαναναίων, καὶ Χετταίων, καὶ Ἀμοῤῥαίων, καὶ Φερεζαίων, καὶ Γεργεσαὶων, καὶ Εὐαίων, καὶ Ἰεβουσαίων.
\VS{9}Καὶ νῦν ἰδοὺ κραυγὴ τῶν υἱῶν Ἰσραὴλ ἥκει πρὸς με· κᾀγὼ ἑώρακα τὸν θλιμμὸν, ὃν οἱ Αἰγύπτιοι θλίβουσιν αὐτούς.
\VS{10}Καὶ νῦν δεῦρο, ἀποστείλω σε πρὸς Φαραὼ βασιλέα Αἰγύπτου, καὶ ἐξάξεις τὸν λαόν μου τοὺς υἱοὺς Ἰσραὴλ ἐκ γῆς Αἰγύπτου.
\par }{\PP \VS{11}Καὶ εἶπε Μωυσῆς πρὸς τὸν Θεὸν, τίς εἰμι ἐγὼ, ὅτι πορεύσομαι πρὸς Φαραὼ βασιλέα Αἰγύπτου, καὶ ὅτι ἐξάξω τοὺς υἱοὺς Ἰσραὴλ ἐκ γῆς Αἰγύπτου;
\VS{12}Εἶπε δὲ ὁ Θεὸς Μωυσῇ, λέγων, ὅτι ἔσομαι μετὰ σοῦ· καὶ τοῦτό σοι τὸ σημεῖον ὅτι ἐγώ σε ἐξαποστελῶ, ἐν τῷ ἐξαγαγεῖν σε τὸν λαόν μου ἐξ Αἰγύπτου, καὶ λατρεύσετε τῷ Θεῷ ἐν τῷ ὄρει τοῦτῳ.
\VS{13}Καὶ εἶπε Μωυσῆς πρὸς τὸν Θεὸν, ἰδοὺ ἐγὼ ἐξελεύσομαι πρὸς τοὺς υἱοὺς Ἰσραὴλ, καὶ ἐρῶ πρὸς αὐτοὺς, ὁ Θεὸς τῶν πατέρων ἡμῶν ἀπέσταλκέ με πρὸς ὑμᾶς· ἐρωτήσουσί με, τί ὄνομα αὐτῷ; τί ἐρῶ πρὸς αὐτούς;
\VS{14}Καὶ εἶπεν ὁ Θεὸς πρὸς Μωυσῆν, λέγων, ἐγώ εἰμι ὁ Ὤν· καὶ εἶπεν, οὕτως ἐρεῖς τοῖς υἱοῖς Ἰσραὴλ, ὁ Ὢν ἀπέσταλκέ με πρὸς ὑμᾶς.
\VS{15}Καὶ εἶπεν ὁ Θεὸς πάλιν πρὸς Μωυσῆν, οὕτως ἐρεῖς τοῖς υἱοῖς Ἰσραήλ, Κύριος ὁ Θεὸς τῶν πατέρων ἡμῶν, Θεὸς Ἀβραὰμ, καὶ Θεὸς Ἰσαὰκ, καὶ Θεὸς Ἰακὼβ, ἀπέσταλκέ με πρὸς ὑμᾶς· τοῦτό μου ἐστὶν ὄνομα αἰώνιον, καὶ μνημόσυνον γενεῶν γενεαῖς.
\VS{16}Ἐλθὼν οὐν συνάγαγε τὴν γερουσίαν τῶν υἱῶν Ἰσραὴλ, καὶ ἐρεῖς πρὸς αὐτοὺς, Κύριος ὁ Θεὸς τῶν πατέρων ἡμων ὦπταί μοι, Θεὸς Ἀβραὰμ, καὶ Θεὸς Ἰσαὰκ, καὶ Θεὸς Ἰακὼβ, λέγων, ἐπισκοπῇ ἐπέσκεμμαι ὑμᾶς, καὶ ὅσα συμβέβηκεν ὑμῖν ἐν Αἰγύπτῳ.
\VS{17}Καὶ εἶπεν, ἀναβιβάσω ὑμᾶς ἐκ τῆς κακώσεως τῶν Αἰγυπτίων, εἰς τὴν γῆν τῶν Χαναναίων, καὶ Χετταίων, καὶ Ἀμοῤῥαίων, καὶ Φερεζαίων, καὶ Γεργεσαίων, καὶ Εὑαίων, καὶ Ἰεβουσαίων, εἰς γῆν ῥέουσαν γάλα καὶ μέλι.
\VS{18}Καὶ εἰσακούσονταί σου τῆς φωνῆς· καὶ εἰσελεύσῃ σὺ, καὶ ἡ γερουσία Ἰσραὴλ, πρὸς Φαραὼ βασιλέα Αἰγύπτου, καὶ ἐρεῖς πρὸς αὐτὸν ὁ Θεὸς τῶν Ἑβραίων προσκέκληται ἡμᾶς· πορευσόμεθα οὖν ὁδὸν τριῶν ἡμερῶν εἰς τὴν ἔρημον, ἵνα θύσωμεν τῷ Θεῷ ἡμῶν.
\VS{19}Ἐγὼ δὲ οἶδα ὅτι οὐ προήσεται ὑμᾶς Φαραὼ βασιλεὺς Αἰγύπτου πορευθῆναι, ἐὰν μὴ μετὰ χειρὸς κραταιᾶς.
\VS{20}Καὶ ἐκτείνας τὴν χεῖρα, πατάξω τοὺς Αἰγυπτίους ἐν πᾶσι τοῖς θαυμασίοις μου, οἷς ποιήσω ἐν αὐτοῖς· καὶ μετὰ ταῦτα ἐξαποστελεῖ ὑμᾶς.
\VS{21}Καὶ δώσω χάριν τῷ λαῷ τούτῳ ἐναντίον τῶν Αἰγυπτίων· ὅταν δὲ ἀποτρέχητε, οὐκ ἀπελεύσεσθε κενοί·
\VS{22}Ἀλλὰ αἰτήσει γυνὴ παρὰ γείτονος καὶ συσκήνου αὐτῆς σκεύη ἀργυρᾶ, καὶ χρυσᾶ, καὶ ἱματισμόν· καὶ ἐπιθήσετε ἐπὶ τοὺς υἱοὺς ὑμῶν, καὶ ἐπὶ τὰς θυγατέρας ὑμῶν, καὶ σκυλεύσατε τοὺς Αἰγυπτίους.

\par }\Chap{4}{\PP \VerseOne{1}Ἀπεκρίθη δὲ Μωυσῆς, καὶ εἶπεν, ἐὰν μὴ πιστεύσωσί μοι, μηδὲ εἰσακούσωσι τῆς φωνῆς μου, ἐροῦσι γὰρ, ὅτι οὐκ ὦπταί σοι ὁ Θεὸς, τί ἐρῶ πρὸς αὐτούς;
\VS{2}Εἶπε δὲ αὐτῳ Κύριος, τί τοῦτό ἐστι τὸ ἐν τῇ χειρί σου; ὁ δὲ εἶπε, ῥάβδος.
\VS{3}Καὶ εἶπε, ῥίψον αὐτὴν ἐπὶ τὴν γῆν· καὶ ἔῤῥιψεν αὐτὴν ἐπὶ τὴν γῆν, καὶ ἐγένετο ὄφις· καὶ ἔφυγε Μωυσῆς ἀπʼ αὐτοῦ.
\VS{4}Καὶ εἶπε Κύριος πρὸς Μωυσῆν, ἔκτεινον τῆν χεῖρα, καὶ ἐπιλαβοῦ τῆς κέρκου· ἐκτείνας οὖν τὴν χεῖρα ἐπελάβετο τῆς κέρκου· καὶ ἐγένετο ῥάβδος ἐν τῇ χειρὶ αὐτοῦ.
\VS{5}Ἵνα πιστεύσωσί σοι, ὅτι ὦπταί σοι ὁ Θεὸς τῶν πατέρων αὐτῶν, Θεὸς Ἀβραὰμ, καὶ Θεὸς Ἰσαὰκ, καὶ Θεὸς Ἰακώβ.
\VS{6}Εἶπε δὲ αὐτῷ Κύριος πάλιν, εἰσένεγκον τὴν χεῖρά σου εἰς τὸν κόλπον σου· καὶ εἰσήνεγκε τὴν χεῖρα αὐτοῦ εἰς τὸν κόλπον αὐτοῦ· καὶ ἐξήνεγκεν τὴν χεῖρα αὐτοῦ ἐκ τοῦ κόλπου αὐτοῦ, καὶ ἐγενήθη ἡ χεὶρ αὐτοῦ ὡσεὶ χιών.
\VS{7}Καὶ εἶπεν πάλιν, εἰσένεγκον τὴν χεῖρά σου εἰς τὸν κόλπον σου· καὶ εἰσήνεγκε τὴν χεῖρα εἰς τὸν κόλπον αὐτοῦ· καὶ ἐξήνεγκεν αὐτὴν ἐκ τοῦ κόλπου αὐτοῦ, καὶ πάλιν ἀπεκατέστη εἰς τὴν χρόαν τῆς σαρκὸς αὐτῆς.
\VS{8}Ἐὰν δὲ μὴ πιστεύσωσί σοι, μηδὲ εἰσακούσωσι τῆς φωνῆς τοῦ σημείου τοῦ πρώτου, πιστεύσουσί σοι τῆς φωνῆς τοῦ σημείου τοῦ δετέρου.
\VS{9}Καὶ ἔσται ἐὰν μὴ πιστεύσωσί σοι τοῖς δυσὶ σημείοις τούτοις, μηδὲ εἰσακούσωσι τῆς φωνῆς σου, λήψῃ ἀπὸ τοῦ ὕδατος τοῦ ποταμοῦ, καὶ ἐκχεεῖς ἐπὶ τὸ ξηρόν· καὶ ἔσται τὸ ὕδωρ, ὃ ἐὰν λάβῃς ἀπὸ τοῦ ποταμοῦ, αἷμα ἐπὶ τοῦ ξηροῦ.
\VS{10}Εἶπε δὲ Μωυσῆς πρὸς Κύριον, δέομαι, Κύριε· οὐχ ἱκανός εἰμι πρὸ τῆς χθὲς οὐδὲ πρὸ τῆς τρίτης ἡμέρας, οὐδὲ ἀφʼ οὗ ἤρξω λαλεῖν τῷ θεράποντί σου· ἰσχνόφωνος καὶ βραδύγλωσσος ἐγώ εἰμι.
\VS{11}Εἶπε δὲ Κύριος πρὸς Μωυσῆν, τίς ἔδωκε στόμα ἀνθρώπῳ; καὶ τίς ἐποιήσε δύσκωφον καὶ κωφὸν, βλέποντα καὶ τυφλόν; οὐκ ἐγὼ ὁ Θεός;
\VS{12}Καὶ νῦν πορεύου, καὶ ἐγὼ ἀνοίξω τὸ στόμα σου, καὶ συμβιβάσω σε ὃ μέλλεις λαλῆσαι.
\VS{13}Καὶ εἶπε Μωυσῆς, δέομαι, Κύριε· προχείρισαι δυνάμενον ἄλλον, ὃν ἀποστελεῖς.
\VS{14}Καὶ θυμωθεὶς ὀργῇ Κύριος ἐπὶ Μωυσῆν, εἶπεν, οὐκ ἰδοὺ Ἀαρὼν ὁ ἄδελφός σου ὁ Λευίτης; ἐπίσταμαι ὅτι λαλῶν λαλήσει αὐτός σοι· καὶ ἰδοὺ αὐτὸς ἐξελεύσεται εἰς συνάντησίν σοι, καὶ ἰδών σε χαρήσεται ἐν ἑαυτῷ.
\VS{15}Καὶ ἐρεῖς πρὸς αὐτὸν, καὶ δώσεις τὰ ῥήματά μου εἰς τὸ στόμα αὐτοῦ, καὶ ἐγὼ ἀνοίξω τὸ στόμα σου καὶ τὸ στόμα αὐτοῦ, καί συμβιβάσω ὑμᾶς ἃ ποιήσετε.
\VS{16}Καὶ αὐτός σοι λαλήσει πρὸς τὸν λαὸν, καὶ αὐτὸς ἔσται σου στόμα· σὺ δὲ αὐτῷ ἔσῃ τὰ πρὸς τὸν Θεόν.
\VS{17}Καὶ τὴν ῥάβδον ταύτην, τὴν στραφεῖσαν εἰς ὄφιν, λήψῃ ἐν τῇ χειρί σου, ἐν ᾗ ποιήσεις ἐν αὐτῇ τὰ σημεῖα.
\par }{\PP \VS{18}Ἐπορεύθη δὲ Μωυσῆς, καὶ ἀπέστρεψε πρὸς Ἰοθὸρ τὸν γαμβρὸν αὐτοῦ, καὶ λέγει, πορεύσομαι καὶ ἀποστρέψω πρὸς τοὺς ἀδελφούς μου τοὺς ἐν Αἰγύπτῳ, καὶ ὄψομαι εἰ ἔτι ζῶσι· καὶ εἶπεν Ἰοθὸρ Μωυσῇ, βάδιζε ὑγιαίνων· μετὰ δὲ τὰς ἡμέρας τὰς πολλὰς ἐκείνας ἐτελεύτησεν ὁ βασιλεὺς Αἰγύπτου.
\VS{19}Εἶπε δὲ Κύριος πρὸς Μωυσῆν ἐν Μαδιὰμ, βάδιζε, ἄπελθε εἰς Αἴγυπτον, τεθνήκασι γὰρ πάντες οἱ ζητοῦντες σου τὴν ψυχήν.
\VS{20}Ἀναλαβὼν δὲ Μωυσῆς τὴν γυναῖκα καὶ τὰ παιδία, ἀνεβίβασεν αὐτὰ ἐπὶ τὰ ὑποζύγια, καὶ ἐπέστρεψεν εἰς Αἴγυπτον· ἔλαβε δὲ Μωυσῆς τὴν ῥάβδον τὴν παρὰ τοῦ Θεοῦ ἐν τῇ χειρὶ αὐτοῦ.
\VS{21}Εἶπε δὲ Κύριος πρὸς Μωυσῆν, πορευομένου σου καὶ ἀποστρέφοντος εἰς Αἴγυπτον, ὅρα πάντα τὰ τέρατα ἃ δέδωκα ἐν ταῖς χερσί σου, ποιήσεις αὐτὰ ἐναντίον Φαραώ· ἐγὼ δὲ σκληρυνῶ τὴν καρδίαν αὐτοῦ, καὶ οὐ μὴ ἐξαποστείλῃ τὸν λαόν.
\VS{22}Σὺ δὲ ἐρεῖς τῷ Φαραῴ, τάδε λέγει Κύριος, υἱὸς πρωτότοκός μου Ἰσραήλ.
\VS{23}Εἶπα δέ σοι, ἐξαπόστειλον τὸν λαόν μου, ἵνα μοι λατρεύσῃ· εἰ μὲν οὖν μὴ βούλει ἐξαποστεῖλαι αὐτούς, ὅρα οὖν, ἐγὼ ἀποκτένῶ τὸν υἱόν σου τὸν πρωτότοκον.
\VS{24}Ἐγένετο δὲ ἐν τῇ ὁδῷ ἐν τῷ καταλύματι συνήντησεν αὐτῷ Ἄγγελος Κυρίου, καὶ ἐζήτει αὐτὸν ἀποκτεῖναι.
\VS{25}Καὶ λαβοῦσα Σεπφώρα ψῆφον, περιέτεμε τὴν ἀκροβυστίαν τοῦ υἱοῦ αὐτῆς· καὶ προσέπεσε πρὸς τοὺς πόδας αὐτοῦ, καὶ εἶπεν, ἔστη τὸ αἷμα τῆς περιτομῆς τοῦ παιδίου μου.
\VS{26}Καὶ ἀπῆλθεν ἀπʼ αὐτοῦ, διότι εἶπεν, ἔστη τὸ αἷμα τῆς περιτομῆς τοῦ παιδίου μου.
\VS{27}Εἶπε δὲ Κύριος πρὸς Ἀαρὼν, πορεύθητι εἰς συνάντησιν Μωυσῇ εἰς τὴν ἔρημον· καὶ ἐπορεύθη, καὶ συνήντησεν αὐτῷ ἐν τῷ ὄρει τοῦ Θεοῦ, καὶ κατεφίλησαν ἀλλήλους.
\VS{28}Καὶ ἀνήγγειλε Μωυσῆς τῷ Ἀαρὼν πάντας τοὺς λόγους Κυρίου, οὓς ἀπέστειλε, καὶ πάντα τὰ ῥήματα, ἃ ἐνετείλατο αὐτῷ.
\VS{29}Ἐπορεύθη δὲ Μωυσῆς καὶ Ἀαρὼν, καὶ συνήγαγον τὴν γερουσίαν τῶν υἱῶν Ἰσραήλ.
\VS{30}Καὶ ἐλάλησεν Ἀαρὼν πάντα τὰ ῥήματα ταῦτα, ἃ ἐλάλησεν ὁ Θεὸς πρὸς Μωυσῆν, καὶ ἐποίησε τὰ σημεῖα ἐναντίον τοῦ λαοῦ.
\VS{31}Καὶ ἐπίστευσεν ὁ λαὸς καὶ ἐχάρη, ὅτι ἐπεσκέψατο ὁ Θεὸς τοὺς υἱοὺς Ἰσραὴλ, καὶ ὅτι εἶδεν αὐτῶν τὴν θλίψιν· κύψας δὲ ὁ λαὸς προσεκύνησε.

\par }\Chap{5}{\PP \VerseOne{1}Καὶ μετὰ ταῦτα εἰσῆλθε Μωυσῆς καὶ Ἀαρὼν πρὸς Φαραὼ, καὶ εἶπαν αὐτῷ, τάδε λέγει Κύριος ὁ Θεὸς Ἰσραὴλ, ἐξαπόστειλον τὸν λαόν μου, ἵνα μοι ἑορτάσωσιν ἐν τῇ ἐρήμῳ.
\VS{2}Καὶ εἶπε Φαραὼ, τίς ἐστιν οὗ εἰσακούσομαι τῆς φωνῆς αὐτοῦ, ὥστε ἐξαποστεῖλαι τοὺς υἱοὺς Ἰσραήλ; οὐκ οἶδα τὸν Κύριον, καὶ τὸν Ἰσραὴλ οὐκ ἐξαποστέλλω.
\VS{3}Καὶ λέγουσιν αὐτῷ, ὁ Θεὸς τῶν Ἐβραίων προσκέκληται ἡμᾶς· πορευσόμεθα οὖν ὁδὸν τριῶν ἡμερῶν εἰς τὴν ἔρημον, ὅπως θύσωμεν Κυρίῳ τῷ Θεῷ ἡμῶν, μή ποτε συναντήσῃ ἡμῖν θάνατος ἢ φόνος.
\VS{4}Καὶ εἶπεν αὐτοῖς ὁ βασιλεὺς Αἰγύπτου, ἱνατί Μωυσῆς καὶ Ἀαρών διαστρέφετε τὸν λαὸν ἀπὸ τῶν ἔργων; ἀπέλθατε ἕκαστος ὑμῶν πρὸς τὰ ἔργα αὐτοῦ.
\VS{5}Καὶ εἶπεν Φαραὼ, ἰδοὺ νῦν πολυπληθεῖ ὁ λαὸς, μὴ οὖν καταπαύσωμεν αὐτοὺς ἀπὸ τῶν ἔργων.
\VS{6}Συνέταξε δὲ Φαραὼ τοῖς ἐργοδιώκταις τοῦ λαοῦ, καὶ τοῖς γραμματεῦσι, λέγων,
\VS{7}οὐκέτι προστεθήσεσθε διδόναι ἄχυρον τῷ λαῷ εἰς τὴν πλινθουργίαν, καθάπερ χθὲς καὶ τρίτην ἡμέραν· ἀλλ αὐτοὶ πορευέσθωσαν καὶ συναγαγέτωσαν ἑαυτοῖς ἄχυρα.
\VS{8}Καὶ τὴν σύνταξιν τῆς πλινθείας, ἧς αὐτοὶ ποιοῦσι, καθʼ ἑκάστην ἡμέραν ἐπιβαλεῖς αὐτοῖς· οὐκ ἀφελεῖς οὐδέν· σχολάζουσι γάρ· διὰ τοῦτο κεκράγασι, λέγοντες, ἐγερθῶμεν, καὶ θύσωμεν τῷ Θεῷ ἡμῶν.
\VS{9}Βαρυνέσθω τὰ ἔργα τῶν ἀνθρώπων τούτων, καὶ μεριμνάτωσαν ταῦτα, καὶ μὴ μεριμνάτωσαν ἐν λόγοις κενοῖς.
\par }{\PP \VS{10}Κατέσπευδον δὲ αὐτοὺς οἱ ἐργοδιῶκται καὶ οἱ γραμματεῖς, καὶ ἔλεγον πρὸς τὸν λαὸν, λέγοντες, τάδε λέγει Φαραὼ, οὐκέτι δίδωμι ὑμῖν ἄχυρα.
\VS{11}Αὐτοὶ ὑμεῖς πορευόμενοι συλλέγετε ἑαυτοῖς ἄχυρα, ὅθεν ἐὰν εὕρητε· οὐ γὰρ ἀφαιρεῖται ἀπὸ τῆς συντάξεως ὑμῶν οὐθέν.
\VS{12}Καὶ διεσπάρη ὁ λαὸς ἐν ὅλῃ γῇ Αἰγύπτῳ συναγαγεῖν καλάμην εἰς ἄχυρα.
\VS{13}Οἱ δὲ ἐργοδιῶκται κατέσπευδον αὐτοὺς, λέγοντες, συντελεῖτε τὰ ἔργα τὰ καθήκοντα καθʼ ἡμέραν, καθάπερ καὶ ὅτε τὸ ἄχυρον ἐδίδοτο ὑμῖν.
\VS{14}Καὶ ἐμαστιγώθησαν οἱ γραμματεῖς τοῦ γένους τῶν υἱῶν Ἰσραὴλ, οἱ κατασταθέντες ἐπʼ αὐτοὺς, ὑπὸ τῶν ἐπιστατῶν τοῦ Φαραὼ, λέγοντες, διατί οὐ συνετελέσατε τὰς συντάξεις ὑμῶν τῆς πλινθείας καθάπερ χθὲς καὶ τρίτην ἡμέραν, καὶ τὸ τῆς σήμερον;
\VS{15}Εἰσελθόντες δὲ οἱ γραμματεῖς τῶν υἱῶν Ἰσραὴλ κατεβόησαν πρὸς Φαραὼ, λέγοντες, ἱνατί σὺ οὕτως ποιεῖς τοῖς σοῖς οἰκέταις;
\VS{16}Ἄχυρον οὐ δίδοται τοῖς οἰκέταις σου, καὶ τὴν πλίνθον ἡμῖν λέγουσι ποιεῖν· καὶ ἰδοὺ οἱ παῖδές σου μεμαστίγωνται, ἀδικήσεις οὖν τὸν λαόν σου.
\VS{17}Καὶ εἶπεν αὐτοῖς, σχολάζετε, σχολασταί ἐστε· διὰ τοῦτο λέγετε, πορευθῶμεν, θύσωμεν τῷ Θεῷ ἡμῶν.
\VS{18}Νῦν οὖν πορευθέντες, ἐργάζεσθε· τὸ γὰρ ἄχυρον οὐ δοθήσεται ὑμῖν, καὶ τὴν σύνταξιν τῆς πλινθείας ἀποδώσετε.
\VS{19}Ἑώρων δὲ οἱ γραμματεῖς τῶν υἱῶν Ἰσραὴλ ἑαυτοὺς ἐν κακοῖς, λέγοντες, οὐκ ἀπολείψετε τῆς πλινθείας τὸ καθῆκον τῇ ἡμέρᾳ.
\VS{20}Συνήντησαν δὲ Μωυσῇ καὶ Ἀαρὼν ἐρχομένοις εἰς συνάντησιν αὐτοῖς, ἐκπορευομένων αὐτῶν ἀπὸ Φαραώ,
\VS{21}Καὶ εἶπαν αὐτοῖς, ἴδοι ὁ Θεὸς ὑμᾶς καὶ κρίναι, ὅτι ἐβδελύξατε τὴν ὀσμὴν ἡμῶν ἐναντίον Φαραὼ, καὶ ἐναντίον τῶν θεραπόντων αὐτοῦ, δοῦναι ῥομφαίαν εἰς τὰς χεῖρας αὐτοῦ, ἀποκτεῖναι ἡμᾶς.
\VS{22}Ἐπέστρεψε δὲ Μωυσῆς πρὸς Κύριον, καὶ εἶπε, δέομαι, Κύριε· τί ἐκάκωσας τὸν λαὸν τοῦτον; καὶ ἱνατί ἀπέσταλκάς με;
\VS{23}Καὶ ἀφʼ οὗ πεπόρευμαι πρὸς Φαραὼ, λαλῆσαι ἐπὶ τῷ σῷ ὀνόματι, ἐκάκωσε τὸν λαὸν τοῦτον· καὶ οὐκ ἐῤῥύσω τὸν λαόν σου.

\par }\Chap{6}{\PP \VerseOne{1}Καὶ εἶπε Κύριος πρὸς Μωυσῆν, ἤδη ὄψει ἃ ποιήσω τῷ Φαραῷ· ἐν γὰρ χειρὶ κραταίᾳ ἐξαποστελεῖ αὐτούς, καὶ ἐν βραχίονι ὑψηλῷ ἐκβαλεῖ αὐτοὺς ἐκ τῆς γῆς αὐτοῦ.
\VS{2}Ἐλάλησε δὲ ὁ Θεὸς πρὸς Μωυσῆν, καὶ εἶπε πρὸς αὐτὸν, ἐγὼ Κύριος.
\VS{3}Καὶ ὤφθην πρὸς Ἀβραὰμ καὶ Ἰσαὰκ καὶ Ἰακὼβ, Θεὸς ὢν αὐτῶν· καὶ τὸ ὄνομά μου Κύριος οὐκ ἐδήλωσα αὐτοῖς.
\VS{4}Καὶ ἔστησα τὴν διαθήκην μου πρὸς αὐτοὺς, ὥστε δοῦναι αὐτοῖς τὴν γῆν τῶν Χαναναίων, τὴν γῆν ἣν παρῳκήκασιν, ἐν ᾗ καὶ παρῴκησαν ἐπʼ αὐτῆς.
\VS{5}Καὶ ἐγὼ εἰσήκουσα τὸν στεναγμὸν τῶν υἱῶν Ἰσραήλ, ὃν οἱ Αἰγύπτιοι καταδουλοῦνται αὐτούς, καὶ ἐμνήσθην τῆς διαθήκης ὑμῶν.
\VS{6}Βάδιζε, εἶπον τοῖς υἱοῖς Ἰσραὴλ, λέγων, ἐγὼ Κύριος· καὶ ἐξάξω ὑμᾶς ἀπὸ τῆς δυναστείας τῶν Αἰγυπτίων, καὶ ῥύσομαι ὑμᾶς ἐκ τῆς δουλείας, καὶ λυτρώσομαι ὑμᾶς ἐν βραχίονι ὑψηλῷ καὶ κρίσει μεγάλῃ·
\VS{7}Καὶ λὴψομαι ἐμαυτῷ ὑμᾶς λαὸν ἐμοὶ, καὶ ἔσομαι ὑμῶν Θεός· καὶ γνώσεσθε ὅτι ἐγὼ Κύριος ὁ Θεὸς ὑμῶν, ὁ ἐξαγαγὼν ὑμᾶς ἐκ τῆς καταδυναστείας τῶν Αἰγυπτίων.
\VS{8}Καὶ εἰσάξω ὑμᾶς εἰς τὴν γῆν, εἰς ἣν ἐξέτεινα τὴν χεῖρά μου, δοῦναι αὐτὴν τῷ Ἀβραὰμ, καὶ Ἰσαὰκ, καὶ Ἰακὼβ, καὶ δώσω ὑμῖν αὐτὴν ἐν κληρῷ· ἐγὼ Κύριος.
\VS{9}Ἐλάλησε δὲ Μωυσῆς οὕτω τοῖς υἱοῖς Ἰσραήλ· καὶ οὐκ εἰσήκουσαν Μωυσῇ ἀπὸ τῆς ὀλιγοψυχίας, καὶ ἀπὸ τῶν ἔργων τῶν σκληρῶν.
\VS{10}Εἶπε δὲ Κύριος πρὸς Μωυσῆν λέγων,
\VS{11}εἴσελθε, λάλησον Φαραῷ βασιλεῖ Αἰγύπτου, ἵνα ἐξαποστείλῃ τοὺς υἱοὺς Ἰσραὴλ ἐκ τῆς γῆς αὐτοῦ.
\VS{12}Ἐλάλησε δὲ Μωυσῆς ἔναντι Κυρίου, λέγων, ἰδοὺ οἱ υἱοὶ Ἰσραὴλ οὐκ εἰσήκουσάν μου, καὶ πῶς εἰσακούσεταί μου Φαραώ; ἐγὼ δὲ ἄλογός εἰμι.
\VS{13}Εἶπε δὲ Κύριος πρὸς Μωυσῆν καὶ Ἀαρὼν, καὶ συνέταξεν αὐτοῖς πρὸς Φαραὼ βασιλέα Αἰγύπτου, ὥστε ἐξαποστεῖλαι τοὺς υἱοὺς Ἰσραὴλ ἐκ γῆς Αἰγύπτου.
\par }{\PP \VS{14}Καὶ οὗτοι ἀρχηγοὶ οἴκων πατριῶν αὐτῶν· υἱοὶ Ῥουβὴν, πρωτοτόκου Ἰσραήλ· Ἑνὼχ, καὶ Φαλλοὺς, Ἀσρὼν, καὶ Χαρμεί· αὕτη ἡ συγγένεια Ῥουβήν.
\VS{15}Καὶ υἱοὶ Συμεών· Ἰεμουὴλ, καὶ Ἰαμεὶμ, καὶ Ἀὼδ, καὶ Ἰαχεὶν, καὶ Σαὰρ, καὶ Σαοὺλ ὁ ἐκ τῆς Φοινίσσης· αὗται αἱ πατριαὶ τῶν υἱῶν Συμεών.
\VS{16}Καὶ ταῦτα τὰ ὀνόματα τῶν υἱῶν Λευὶ κατὰ συγγενείας αὐτῶν· Γεδσὼν, Καὰθ, καὶ Μεραρεί· καὶ τὰ ἔτη τῆς ζωῆς Λευὶ ἑκατὸν τριάκοντα ἑπτά.
\VS{17}Καὶ οὗτοι υἱοὶ Γεδσών· Λοβενεὶ, καὶ Σεμεεί· οἶκοι πατριᾶς αὐτῶν.
\VS{18}Καὶ υἱοὶ Καάθ· Ἀμβρὰμ, καὶ Ἰσσαάρ, Χεβρὼν, καὶ Ὀζειήλ· καὶ τὰ ἔτη τῆς ζωῆς Καὰθ ἑκατὸν τριάκοντα τρία ἔτη.
\VS{19}Καὶ υἱοὶ Μεραρεί· Μοολεὶ, καὶ Ὀμουσεί. οὗτοι οἱ οἶκοι πατριῶν Λευὶ κατὰ συγγενείας αὐτῶν.
\VS{20}Καὶ ἔλαβεν Ἀμβρὰν τὴν Ἰωχαβὲδ, θυγατέρα τοῦ ἀδελφοῦ τοῦ πατρὸς αὐτοῦ, ἑαυτῷ εἰς γυναῖκα· καὶ ἐγέννησεν αὐτῷ τόν τε Ἀαρὼν καὶ τὸν Μωυσῆν, καὶ Μαριὰμ τὴν ἀδελφὴν αὐτῶν· τὰ δὲ ἔτη τῆς ζωῆς Ἀμβρὰμ, ἑκατὸν τριάκοντα δύο ἔτη.
\VS{21}Καὶ υἱοὶ Ἰσσαάρ· Κορὲ, καὶ Ναφὲκ, καὶ Ζεχρεί.
\VS{22}Καὶ υἱοὶ Ὀζειήλ· Μισαὴλ, καὶ Ἐλισαφὰν, καὶ Σεγρεί.
\VS{23}Ἔλαβε δὲ Ἀαρὼν τὴν Ἐλισαβὲθ θυγατέρα Ἀμιναδὰβ, ἀδελφὴν Ναασσὼν, αὐτῷ γυναῖκα· καὶ ἔτεκεν αὐτῷ τόν τε Ναδὰβ, καὶ Ἀβιοὺδ, καὶ τὸν Ἐλεάζαρ, καὶ Ἰθάμαρ.
\VS{24}Υἱοὶ δὲ Κορέ· Ἀσεὶρ, καὶ Ἑλκανὰ, καὶ Ἀβιασάρ· αὗται αἱ γενέσεις Κορέ.
\VS{25}Καὶ Ἐλεάζαρ ὁ τοῦ Ἀαρὼν ἔλαβε τῶν θυγατέρων Φουτιὴλ αὐτῷ γυναῖκα· καὶ ἔτεκεν αὐτῷ τὸν Φινεές· αὗται αἱ ἀρχαὶ πατριᾶς Λευιτῶν, κατὰ γενέσεις αὐτῶν.
\VS{26}Οὗτος Ἀαρὼν καὶ Μωυσῆς, οἷς εἶπεν αὐτοῖς ὁ Θεὸς ἐξαγαγεῖν τοὺς υἱοὺς Ἰσραὴλ ἐκ γῆς Αἰγύπτου σὺν δυνάμει αὐτῶν.
\VS{27}Οὗτοί εἰσιν οἱ διαλεγόμενοι πρὸς Φαραὼ βασιλέα Αἰγύπτου· καὶ ἐξήγαγον τοὺς υἱοὺς Ἰσραὴλ ἐκ γῆς Αἰγύπτου αὐτὸς Ἀαρὼν καὶ Μωυσὴς,
\VS{28}ᾗ ἡμέρᾳ ἐλάλησε Κύριος Μωυσῇ ἐν γῇ Αἰγύπτῳ.
\VS{29}Καὶ ἐλάλησε Κύριος πρὸς Μωυσῆν, λέγων, ἐγὼ Κύριος· λάλησον πρὸς Φαραὼ βασιλέα Αἰγύπτου ὅσα ἐγὼ λέγω πρὸς σέ.
\VS{30}Καὶ εἶπε Μωυσῆς ἐναντίον Κυρίου, ἰδοὺ ἐγὼ ἰσχνόφωνός εἰμι, καὶ πῶς εἰσακούσεταί μου Φαραώ,

\par }\Chap{7}{\PP \VerseOne{1}Καὶ εἶπε Κύριος πρὸς Μωυσῆν, λέγων, ἰδοὺ δέδωκά σε θεὸν Φαραὼ, καὶ Ἀαρὼν ὁ ἀδελφός σου ἔσται σου προφήτης.
\VS{2}Σὺ δὲ λαλήσεις αὐτῷ πάντα ὅσα σοι ἐντέλλομαι· ὁ δὲ Ἀαρὼν ὁ ἀδελφός σου λαλήσει πρὸς Φαραὼ, ὥστε ἐξαποστεῖλαι τοὺς υἱοὺς Ἰσραὴλ ἐκ τῆς γῆς αὐτοῦ.
\VS{3}Ἐγὼ δὲ σκληρυνῶ τὴν καρδίαν Φαραὼ, καὶ πληθυνῶ τὰ σημεῖά μου καὶ τὰ τέρατα ἐν γῇ Αἰγύπτῳ.
\VS{4}Καὶ οὐκ εἰσακούσεται ὑμῶν Φαραώ· καὶ ἐπιβαλῶ τὴν χεῖρά μου ἐπʼ Αἴγυπτον, καὶ ἐξάξω σὺν δυνάμει μου τὸν λαόν μου τοὺς υἱοὺς Ἰσραὴλ ἐκ γῆς Αἰγύπτου σὺν ἐκδικήσει μεγάλῃ.
\VS{5}Καὶ γνώσονται πάντες οἱ Αἰγύπτιοι ὅτι ἐγώ εἰμι Κύριος, ἐκτείνων τὴν χεῖρά μου ἐπʼ Αἴγυπτον, καὶ ἐξάξω τοὺς υἱοὺς Ἰσραὴλ ἐκ μέσον αὐτῶν.
\VS{6}Ἐποίησε δὲ Μωυσῆς καὶ Ἀαρὼν καθάπερ ἐνετείλατο αὐτοῖς Κύριος, οὕτως ἐποίησαν.
\VS{7}Μωυσῆς δὲ ἦν ἐτῶν ὀγδοήκοντα, Ἀαρὼν δὲ ὁ ἀδελφὸς αὐτοῦ ἐτῶν ὀγδοήκοντατριῶν, ἡνίκα ἐλάλησεν πρὸς Φαραώ.
\VS{8}Καὶ εἶπε Κύριος πρὸς Μωυσῆν καὶ Ἀαρὼν, λέγων,
\VS{9}καὶ ἐὰν λαλήσῃ πρὸς ὑμᾶς Φαραὼ, λέγων, δότε ἡμῖν σημεῖον ἢ τέρας, καὶ ἐρεῖς Ἀαρὼν τῷ ἀδελφῷ σου, λάβε τὴν ῥάβδον, καὶ ῥίψον ἐπὶ τὴν γῆν ἐναντίον Φαραὼ, καὶ ἐναντίον τῶν θεραπόντων αὐτοῦ, καὶ ἔσται δράκων.
\VS{10}Εἰσῆλθε δὲ Μωυσῆς καὶ Ἀαρὼν ἐναντίον Φαραὼ, καὶ τῶν θεραπόντων αὐτοῦ· καὶ ἐποίησαν οὕτως, καθάπερ ἐνετείλατο αὐτοῖς Κύριος· καὶ ἔῤῥιψεν Ἀαρὼν τὴν ῥάβδον ἐναντίον Φαραὼ, καὶ ἐναντίον τῶν θεραποντων αὐτοῦ, καὶ ἐγένετο δράκων.
\VS{11}Συνεκάλεσε δὲ Φαραὼ τοὺς σοφιστὰς Αἰγύπτου, καὶ τοὺς φαρμακούς· καὶ ἐποίησαν καὶ οἱ ἐπαοιδοὶ τῶν Αἰγυπτίων ταῖς φαρμακίαις αὐτῶν ὡσαύτως.
\VS{12}Καὶ ἔῤῥιψαν ἔκαστος τὴν ῥάβδον αὐτῶν, καὶ ἐγένοντο δράκοντες· καὶ κατέπιεν ἡ ῥάβδος ἡ Ἀαρὼν τὰς ἐκείνων ῥάβδους.
\VS{13}Καὶ κατίσχυσεν ἡ καρδία Φαραὼ, καὶ οὐκ εἰσήκουσεν αὐτῶν, καθάπερ ἐνετείλατο αὐτοῖς Κύριος.
\par }{\PP \VS{14}Εἶπε δὲ Κύριος πρὸς Μωυσῆν, βεβάρηται ἡ καρδία Φαραὼ, τοῦ μὴ ἐξαποστεῖλαι τὸν λαόν.
\VS{15}Βάδισον πρὸς Φαραὼ τὸ πρωΐ· ἰδοὺ αὐτὸς ἐκπορεύεται ἐπὶ τὸ ὕδωρ, καὶ ἔσῃ συναντῶν αὐτῷ ἐπὶ τὸ χεῖλος τοῦ ποταμοῦ· καὶ τὴν ῥάβδον τὴν στραφεῖσαν εἰς ὄφιν λήψῃ ἐν τῇ χειρί σου.
\VS{16}Καὶ ἐρεῖς πρὸς αὐτὸν, Κύριος ὁ Θεὸς τῶν Ἐβραίων ἀπέσταλκέ με πρὸς σὲ, λέγων, ἐξαπόστειλον τὸν λαόν μου, ἵνα μοι λατρεύσῃ ἐν τῇ ἐρήμῳ· καὶ ἰδοὺ οὐκ εἰσήκουσας ἕως τούτου.
\VS{17}Τάδε λέγει Κύριος, ἐν τούτῳ γνώσῃ ὅτι ἐγὼ Κύριος· ἰδοὺ ἑγὼ τύπτω τῇ ῥάβδῳ τῇ ἐν τῇ χειρί μου ἐπὶ τὸ ὕδωρ τὸ ἐν τῷ ποταμῷ, καὶ μεταβαλεῖ εἰς αἷμα.
\VS{18}Καὶ οἱ ἰχθύες οἱ ἐν τῷ ποταμῷ τελευτήσουσι· καὶ ἐποζέσει ὁ ποταμὸς, καὶ οὐ δυνήσονται οἱ Αἰγύπτιοι πιεῖν ὕδωρ ἀπὸ τοῦ ποταμοῦ.
\VS{19}Εἶπε δὲ Κύριος πρὸς Μωυσῆν, εἶπὸν Ἀαρὼν τῷ ἀδελφῷ σου, λάβε τὴν ῥάβδον σου ἐν τῇ χειρί σου, καὶ ἔκτεινον τὴν χεῖρά σου ἐπὶ τὰ ὕδατα Αἰγύπτου, καὶ ἐπὶ τοὺς ποταμοὺς αὐτῶν, καὶ ἐπὶ τὰς διώρυγας αὐτῶν, καὶ ἐπὶ τὰ ἕλη αὐτῶν, καὶ ἐπὶ πᾶν συνεστηκὸς ὕδωρ αὐτῶν, καὶ ἔσται αἷμα· καὶ ἐγένετο αἷμα ἐν πάσῃ γῇ Αἰγύπτου, ἔν τε τοῖς ξύλοις καὶ ἐν τοῖς λίθοις.
\VS{20}Καὶ ἐποίησαν οὕτως Μωυσῆς καὶ Ἀαρὼν, καθάπερ ἐνετείλατο αὐτοῖς Κύριος· καὶ ἐπάρας τῇ ῥάβδῳ αὐτοῦ ἐπάταξε τὸ ὕδωρ τὸ ἐν τῷ ποταμῷ ἐναντίον Φαραὼ, καὶ ἐναντίον τῶν θεραπόντων αὐτοῦ· καὶ μετέβαλε πᾶν τὸ ὕδωρ τὸ ἐν τῷ ποταμῷ εἰς αἷμα.
\VS{21}Καὶ οἱ ἰχθύες οἱ ἐν τῷ ποταμῷ ἐτελεύτησαν· καὶ ἐπώζεσεν ὁ ποταμὸς, καὶ οὐκ ἠδύναντο οἱ Αἰγύπτιοι πιεῖν ὕδωρ ἐκ τοῦ ποταμοῦ· καὶ ἦν τὸ αἷμα ἐν πάσῃ γῇ Αἰγύπτου.
\VS{22}Ἐποίησαν δὲ ὡσαύτως καὶ οἱ ἐπαοιδοὶ τῶν Αἰγυπτίων ταῖς φαρμακίαις αὐτῶν· καὶ ἐσκληρύνθη ἡ καρδία Φαραὼ, καὶ οὐκ εἰσήκουσεν αὐτῶν, καθάπερ εἶπε Κύριος.
\VS{23}Ἐπιστραφεὶς δὲ Φαραὼ εἰσῆλθεν εἰς τὸν οἶκον αὐτοῦ· καὶ οὐκ ἐπέστησε τὸν νοῦν αὐτοῦ οὐδὲ ἐπὶ τούτῳ.
\VS{24}Ὤρυξαν δὲ πάντες οἱ Αἰγύπτιοι κύκλῳ τοῦ ποταμοῦ, ὥστε πιεῖν ὕδωρ· καὶ οὐκ ἠδύναντο πιεῖν ὕδωρ ἀπὸ τοῦ ποταμοῦ.
\VS{25}Καὶ ἀνεπληρώθησαν ἑπτὰ ἡμέραι, μετὰ τὸ πατάξαι Κύριον τὸν ποταμόν.
\par }{\PP \VS{26}Εἶπε δὲ Κύριος πρὸς Μωυσὴν, εἴσελθε πρὸς Φαραὼ, καὶ ἐρεῖς πρὸς αὐτὸν, τάδε λεγέι Κύριος, ἐξαπόστειλον τὸν λαόν μου, ἵνα μοι λατρεύσωσιν.
\VS{27}Εἰ δὲ μὴ βούλει σὺ ἐξαποστεῖλαι, ἰδοὺ ἐγὼ τύπτω πάντα τὰ ὅριά σου τοῖς βατράχοις.
\VS{28}Καὶ ἐξερεύξεται ὁ ποταμὸς βατράχους· καὶ ἀναβάντες εἰσελεύσονται εἰς τοὺς οἴκους σου, καὶ εἰς τὰ ταμιεῖα τῶν κοιτώνων σου, καὶ ἐπὶ τῶν κλινῶν σου, καὶ ἐπὶ τοὺς οἴκους τῶν θεραπόντων σου, καὶ τοῦ λαοῦ σου, καὶ ἐν τοῖς φυράμασί σου, καὶ ἐν τοῖς κλιβάνοις σου.
\VS{29}Καὶ ἐπὶ σὲ, καὶ ἐπὶ τοὺς θεράποντάς σου, καὶ ἐπὶ τὸν λαόν σου, ἀναβήσονται οἱ βάτραχοι.

\Chap{8}\VerseOne{1}Εἶπε δὲ Κύριος πρὸς Μωυσῆν, εἶπον Ἀαρὼν τῷ ἀδελφῷ σου, ἔκτεινον τῇ χειρὶ τὴν ῥάβδον σου ἐπὶ τοὺς ποταμοὺς, καὶ ἐπὶ τὰς διώρυγας, καὶ ἐπὶ τὰ ἕλη, καὶ ἀνάγαγε τοὺς βατράχους.
\VS{2}Καὶ ἐξέτεινεν Ἀαρὼν τὴν χεῖρα ἐπὶ τὰ ὕδατα Αἰγύπτου, καὶ ἀνήγαγε τοὺς βατράχους· καὶ ἀνεβιβάσθη ὁ βάτραχος, καὶ ἐκάλυψε τὴν γῆν Αἰγύπτου.
\VS{3}Ἐποίησαν δὲ ὡσαύτως καὶ οἱ ἐπαοιδοὶ τῶν Αἰγυπτίων ταῖς φαρμακίαις αὐτῶν, καὶ ἀνήγαγον τοὺς βατράχους ἐπὶ γῆν Αἰγύπτου.
\VS{4}Καὶ ἐκάλεσε Φαραὼ Μωυσῆν καὶ Ἀαρὼν, καὶ εἶπεν, εὔξασθε περὶ ἐμοῦ πρὸς Κύριον, καὶ περιελέτω τοὺς βατράχους ἀπʼ ἐμοῦ, καὶ ἀπὸ τοῦ ἐμοῦ λαοῦ· καὶ ἐξαποστελῶ αὐτοὺς, καὶ θύσωσι τῷ Κυρίῳ.
\VS{5}Εἶπε δὲ Μωυσῆς πρὸς Φαραὼ, τάξαι πρὸς με πότε εὔξομαι περὶ σοῦ, καὶ περὶ τῶν θεραπόντων σου, καὶ τοῦ λαοῦ σου, ἀφανίσαι τοὺς βατράχους ἀπὸ σοῦ, καὶ ἀπὸ τοῦ λαοῦ σου, καὶ ἐκ τῶν οἰκιῶν ὑμῶν, πλὴν ἐν τῷ ποταμῷ ὑπολειφθήσονται.
\VS{6}Ὁ δὲ εἶπεν, εἰς αὔριον· εἶπεν οὖν, ὡς εἴρηκας· ἵνα εἰδῇς ὅτι οὐκ ἔστιν ἄλλος πλὴν Κυρίου·
\VS{7}Καὶ περιαιρεθήσονται οἱ βάτραχοι ἀπὸ σοῦ, καὶ ἀπὸ τῶν οἰκιῶν ὑμῶν, καὶ ἀπὸ τῶν ἐπαύλεων, καὶ ἀπὸ τῶν θεραπόντων σου, καὶ ἀπὸ τοῦ λαοῦ σου, πλὴν ἐν τῷ ποταμῷ ὑπολειφθήσονται.
\VS{8}Ἐξῆλθε δὲ Μωυσῆς καὶ Ἀαρὼν ἀπὸ Φαραώ· καὶ ἐβόησε Μωυσῆς πρὸς Κύριον περὶ τοῦ ὁρισμοῦ τῶν βατράχων, ὡς ἐτάξατο Φαραώ.
\VS{9}Ἐποιήσε δὲ Κύριος καθάπερ εἶπε Μωυσῆς· καὶ ἐτελεύτησαν οἱ βάτραχοι ἐκ τῶν οἰκιῶν, καὶ ἐκ τῶν ἐπαύλεων, καὶ ἐκ τῶν ἀγρῶν.
\VS{10}Καὶ συνήγαγον αὐτοὺς, θημωνίας θημωνίας· καὶ ὤζεσεν ἡ γῆ.
\VS{11}Ἰδὼν δὲ Φαραὼ ὅτι γέγονεν ἀνάψυξις, ἐβαρύνθη ἡ καρδία αὐτοῦ, καὶ οὐκ εἰσήκουσεν αὐτῶν, καθάπερ ἐλάλησε Κύριος.
\VS{12}Εἶπε δὲ Κύριος πρὸς Μωυσῆν, εἶπον Ἀαρὼν, ἔκτεινον τῇ χειρὶ τὴν ῥάβδον σου, καὶ πάταξον τὸ χῶμα τῆς γῆς· καὶ ἔσονται σκνίφες ἔν τε τοῖς ἀνθρώποις, καὶ ἐν τοῖς τετράποσι, καὶ ἐν πάσῃ γῇ Αἰγύπτου.
\VS{13}Ἐξέτεινεν οὖν Ἀαρὼν τῇ χειρὶ τὴν ῥάβδον, καὶ ἐπάταξε τὸ χῶμα τῆς γῆς· καὶ ἐγένοντο οἱ σκνίφες ἐν τοῖς ἀνθρώποις, ἔν τε τοῖς τετράποσι, καὶ ἐν παντὶ χώματι τῆς γῆς ἐγένοντο οἱ σκνίφες.
\VS{14}Ἐποίησαν δὲ ὡσαύτως καὶ οἱ ἐπαοιδοὶ ταῖς φαρμακίαις αὐτῶν, ἐξαγαγεῖν τὸν σκνῖφα, καὶ οὐκ ἠδύναντο· καὶ ἐγένοντο οἱ σκνίφες ἔν τε τοῖς ἀνθρώποις, καὶ ἐν τοῖς τετράποσιν.
\VS{15}Εἶπαν οὖν οἱ ἐπαοιδοὶ τῷ Φαραῷ, δάκτυλος Θεοῦ ἐστι τοῦτο· καὶ ἐσκληρύνθη ἡ καρδία Φαραὼ, καὶ οὐκ εἰσήκουσεν αὐτῶν, καθάπερ ἐλάλησε Κύριος.
\VS{16}Εἶπε δὲ Κύριος πρὸς Μωυσῆν, ὄρθρισον τὸ πρωΐ, καὶ στῆθι ἐναντίον Φαραώ· καὶ ἰδοὺ αὐτὸς ἐξελεύσεται ἐπὶ τὸ ὕδωρ· καὶ ἐρεῖς πρὸς αὐτὸν, τάδε λέγει Κύριος, ἐξαπόστειλον τὸν λαόν μου, ἵνα μοι λατρεύσωσιν ἐν τῇ ἐρήμῳ.
\VS{17}Ἐὰν δὲ μὴ βούλει ἐξαποστεῖλαι τὸν λαόν μου, ἰδοὺ ἐγὼ ἐξαποστέλλω ἐπὶ σὲ, καὶ ἐπὶ τοὺς θεράποντάς σου, καὶ ἐπὶ τὸν λαόν σου, καὶ ἐπὶ τοὺς οἴκους ὑμῶν, κυνόμυιαν· καὶ πλησθήσονται αἱ οἰκίαι τῶν Αἰγυπτίων τῆς κυνομυίης, καὶ εἰς τὴν γῆν ἐφʼ ἧς εἰσιν ἐπʼ αὐτῆς.
\VS{18}Καὶ παραδοξάσω ἐν τῇ ἡμέρᾳ ἐκείνῃ τὴν γῆν Γεσὲμ, ἐφʼ ἧς ὁ λαός μου ἔπεστιν ἐπʼ αὐτῆς, ἐφʼ ἧς οὐκ ἔσται ἐκεῖ ἡ κυνόμυια· ἵνα εἰδῇς ὅτι ἐγώ εἰμι Κύριος ὁ Θεὸς πάσης τῆς γῆς.
\VS{19}Καὶ δώσω διαστολὴν ἀνὰ μέσον τοῦ ἐμοῦ λαοῦ, καὶ ἀνὰ μέσον τοῦ σου λαοῦ· ἐν δὲ τῇ αὔριον ἔσται τοῦτο ἐπὶ τῆς γῆς.
\VS{20}Ἐποίησε δὲ Κύριος οὕτως· καὶ παρεγένετο ἡ κυνόμυια πλῆθος εἰς τοὺς οἴκους Φαραὼ, καὶ εἰς τοὺς οἴκους τῶν θεραπόντων αὐτοῦ, καὶ εἰς πᾶσαν τὴν γῆν Αἰγύπτου· καὶ ἐξωλοθρεύθη ἡ γῆ ἀπὸ τῆς κυνομυίης.
\par }{\PP \VS{21}Ἐκάλεσε δὲ Φαραὼ Μωυσῆν καὶ Ἀαρὼν, λέγων, ἐλθόντες θύσατε Κυρίῳ τῷ Θεῷ ὑμῶν ἐν τῇ γῇ.
\VS{22}Καὶ εἶπε Μωυσῆς, οὐ δυνατὸν γενέσθαι οὕτως· τὰ γὰρ βδελύγματα τῶν Αἰγυπτίων θύσομεν Κυρίῳ τῷ Θεῷ ἡμῶν· ἐὰν γὰρ θύσωμεν τὰ βδελύγματα τῶν Αἰγυπτίων ἐναντίον αὐτῶν, λιθοβοληθησόμεθα.
\VS{23}Ὁδὸν τριῶν ἡμερῶν πορευσόμεθα εἰς τὴν ἔρημον· καὶ θύσομεν τῷ Θεῷ ἡμῶν, καθάπερ εἶπεν Κύριος ἡμῖν.
\VS{24}Καὶ εἶπε Φαραὼ, ἐγὼ ἀποστέλλω ὑμᾶς, καὶ θύσατε τῷ Θεῷ ὑμῶν ἐν τῇ ἐρήμῳ· ἀλλʼ οὐ μακρὰν ἀποτενεῖτε πορευθῆναι· εὔξασθε οὖν περὶ ἐμοῦ πρὸς Κύριον.
\VS{25}Εἶπε δὲ Μωυσῆς, ὁ δὲ ἐγὼ ἐξελεύσομαι ἀπὸ σοῦ, καὶ εὔξομαι πρὸς τὸν Θεὸν, καὶ ἀπελεύσεται ἡ κυνόμυια καὶ ἀπὸ τῶν θεραπόντων σου, καὶ ἀπὸ τοῦ λαοῦ σου αὔριον· μὴ προσθῇς ἔτι Φαραὼ ἐξαπατῆσαι, τοῦ μὴ ἐξαποστεῖλαι τὸν λαὸν θῦσαι Κυρίῳ.
\VS{26}Ἐξῆλθε δὲ Μωυσῆς ἀπὸ Φαραὼ, καὶ ηὔξατο πρὸς τὸν Θεόν.
\VS{27}Ἐποίησε δὲ Κύριος καθάπερ εἶπε Μωυσῆς· καὶ περιεῖλε τὴν κυνόμυιαν ἀπὸ Φαραὼ, καὶ τῶν θεραπόντων αὐτοῦ, καὶ τοῦ λαοῦ αὐτοῦ, καὶ οὐ κατελείφθη οὐδεμία.
\VS{28}Καὶ ἐβάρυνε Φαραὼ τὴν καρδίαν αὐτοῦ καὶ ἐπὶ τοῦ καιροῦ τούτου, καὶ οὐκ ἠθέλησεν ἐξαποστεῖλαι τὸν λαόν.

\par }\Chap{9}{\PP \VerseOne{1}Εἶπε δὲ Κύριος πρὸς Μωυσῆν, εἴσελθε πρὸς Φαραὼ, καὶ ἐρεῖς αὐτῷ, τάδε λέγει Κύριος ὁ Θεὸς τῶν Ἑβραίων, ἐξαπόστειλον τὸν λαόν μου, ἵνα μοι λατρεύσωσι.
\VS{2}Εἰ μὲν οὖν μὴ βούλει ἐξαποστεῖλαι τὸν λαόν μου, ἀλλὰ ἔτι ἐγκρατεῖς αὐτοῦ,
\VS{3}Ἰδοὺ, χεὶρ Κυρίου ἐπέσται ἐν τοῖς κτήνεσί σου τοῖς ἐν τοῖς πεδίοις, ἔν τε τοῖς ἵπποις, καὶ ἐν τοῖς ὑποζυγίοις, καὶ ταῖς καμήλοις, καὶ βουσὶ, καὶ προβάτοις, θάνατος μέγας σφόδρα.
\VS{4}Καὶ παραδοξάσω ἐγὼ ἐν τῷ καιρῷ ἐκείνῳ ἀνὰ μέσον τῶν κτηνῶν τῶν Αἰγυπτίων, καὶ ἀνὰ μέσον τῶν κτηνῶν τῶν υἱῶν Ἰσραήλ· οὐ τελευτήσει ἀπὸ πάντων τῶν τοῦ Ἰσραὴλ υἱῶν ῥητόν.
\VS{5}Καὶ ἔδωκεν ὁ Θεὸς ὅρον, λέγων, ἐν τῇ αὔριον ποιήσει Κύριος τὸ ῥῆμα τοῦτο ἐπὶ τῆς γῆς.
\VS{6}Καὶ ἐποίησε Κύριος τὸ ῥῆμα τοῦτο τῇ ἐπαύριον· καὶ ἐτελεύτησε πάντα τὰ κτήνη τῶν Αἰγυπτίων· ἀπὸ δὲ τῶν κτηνῶν τῶν υἱῶν Ἰσραὴλ οὐκ ἐτελεύτησεν οὐδέν.
\VS{7}Ἰδὼν δὲ Φαραὼ ὅτι οὐκ ἐτελεύτησεν ἀπὸ πάντων τῶν κτηνῶν τῶν υἱῶν Ἰσραὴλ οὐδὲν, ἐβαρύνθη ἡ καρδία Φαραὼ, καὶ οὐκ ἐξαπέστειλε τὸν λαόν.
\VS{8}Εἶπε δὲ Κύριος πρὸς Μωυσῆν καὶ Ἀαρὼν, λέγων, λάβετε ὑμεῖς πληρεῖς τὰς χεῖρας αἰθάλης καμιναίας, καὶ πασάτω Μωυσῆς εἰς τὸν οὐρανὸν ἐναντίον Φαραὼ, καὶ ἐναντίον τῶν θεραπόντων αὐτοῦ.
\VS{9}Καὶ γενηθήτω κονιορτὸς ἐπὶ πᾶσαν τὴν γῆν Αἰγύπτου· καὶ ἔσται ἐπὶ τοὺς ἀνθρώπους, καὶ ἐπὶ τὰ τετράποδα, ἕλκη, φλυκτίδες ἀναζέουσαι ἔν τε τοῖς ἀνθρώποις, καὶ ἐν τοῖς τετράποσιν, ἐν πάσῃ γῇ Αἰγύπτου.
\VS{10}Καὶ ἔλαβεν τὴν αἰθάλην τῆς καμιναίας ἐναντίον Φαραὼ, καὶ ἔπασεν αὐτὴν Μωυσῆς εἰς τὸν οὐρανόν· καὶ ἐγένετο ἕλκη, φλυκτίδες ἀναζέουσαι, ἔν τε τοῖς ἀνθρώποις, καὶ ἐν τοῖς τετράποσι.
\VS{11}Καὶ οὐκ ἠδύναντο οἱ φαρμακοὶ στῆναι ἐναντίον Μωυσῆ διὰ τὰ ἕλκη· ἐγένετο γὰρ τὰ ἕλκη ἐν τοῖς φαρμακοῖς, καὶ ἐν πάσῃ γῇ Αἰγύπτου.
\VS{12}Ἐσκλήρυνε δὲ Κύριος τὴν καρδίαν Φαραὼ, καὶ οὐκ εἰσήκουσεν αὐτῶν, καθὰ συνέταξε Κύριος.
\VS{13}Εἶπε δὲ Κύριος πρὸς Μωυσῆν, ὄρθρισον τὸ πρωῒ, καὶ στῆθι ἐναντίον Φαραὼ, καὶ ἐρεῖς πρὸς αὐτὸν, τάδε λέγει Κύριος ὁ Θεὸς τῶν Ἑβραίων, ἐξαπόστειλον τὸν λαόν μου, ἵνα λατρεύσωσί μοι.
\VS{14}Ἐν τῷ γὰρ νῦν καιρῷ ἐγὼ ἐξαποστέλλω πάντα τὰ συναντήματά μου εἰς τὴν καρδίαν σου, καὶ τῶν θεραπόντων σου, καὶ τοῦ λαοῦ σου, ἵνα εἴδῇς ὅτι οὐκ ἔστιν, ὡς ἐγὼ, ἄλλος ἐν πάσῃ τῇ γῇ.
\VS{15}Νῦν γὰρ ἀποστείλας τὴν χεῖρα πατάξω σε, καὶ τὸν λαόν σου θανατώσω, καὶ ἐκτριβήσῃ ἀπὸ τῆς γῆς.
\VS{16}Καὶ ἕνεκεν τούτου διετηρήθης, ἵνα ἐνδείξωμαι ἐν σοὶ τὴν ἰσχύν μου, καὶ ὅπως διαγγελῇ τὸ ὄνομά μου ἐν πάσῃ τῇ γῇ.
\VS{17}Ἔτι οὖν σὺ ἐνποιῇ τοῦ λαοῦ μου, τοῦ μὴ ἐξαποστεῖλαι αὐτούς;
\VS{18}Ἰδοὺ ἐγὼ ὕω ταύτην τὴν ὥραν αὔριον χάλαζαν πολλὴν σφόδρα, ἥτις τοιαύτη οὐ γέγονεν ἐν Αἰγύπτῳ, ἀφʼ ἧς ἡμέρας ἔκτισται, ἕως τῆς ἡμέρας ταύτης.
\VS{19}Νῦν οὖν κατάσπευσον συναγαγεῖν τὰ κτήνη σου, καὶ ὅσα σοι ἐστὶν ἐν τῷ πεδίῳ· πάντες γὰρ οἱ ἄνθρωποι, καὶ τὰ κτήνη, ὅσα σοί ἐστιν ἐν τῷ πεδίῳ· πὰντες γὰρ οἱ ἄνθρωποι, καὶ τὰ κτήνη, ὅσα ἐὰν εὑρεθῇ ἐν τοῖς πεδίοις, καὶ μὴ εἰσέλθῃ εἰς οἰκίαν, πεσῇ δὲ ἐπʼ αὐτὰ ἡ χάλαζα, τελευτήσει.
\VS{20}Ὁ φοβούμενος τὸ ῥῆμα Κυρίου τῶν θεραπόντων Φαραὼ, συνήγαγε τὰ κτήνη αὐτοῦ εἰς τοὺς οἴκους.
\VS{21}Ὃς δὲ μὴ πρόσεσχεν τῇ διανοίᾳ εἰς τὸ ῥῆμα Κυρίου, ἀφῆκε τὰ κτήνη ἐν τοῖς πεδίοις.
\par }{\PP \VS{22}Εἶπε δὲ Κύριος πρὸς Μωυσῆν, ἔκτεινον τὴν χεῖρά σου εἰς τὸν οὐρανὸν, καὶ ἔσται χάλαζα ἐπὶ πᾶσαν γῆν Αἰγύπτου, ἐπί τε τοὺς ἀνθρώπους, καὶ τὰ κτήνη, καὶ ἐπὶ πᾶσαν βοτάνην τὴν ἐπὶ τῆς γῆς.
\VS{23}Ἐξέτεινε δὲ Μωυσῆς τὴν χεῖρα εἰς τὸν οὐρανὸν, καὶ Κύριος ἔδωκε φωνὰς καὶ χάλαζαν· καὶ διέτρεχε τὸ πῦρ ἐπὶ τῆς γῆς· καὶ ἔβρεξε Κύριος χάλαζαν ἐπὶ πᾶσαν γῆν Αἰγύπτου.
\VS{24}Ἦν δὲ ἡ χάλαζα καὶ τὸ πῦρ φλογίζον ἐν τῇ χαλάζῃ· ἡ δὲ χάλαζα πολλὴ σφόδρα, ἥτις τοιαύτη οὐ γέγονεν ἐν Αἰγύπτῳ, ἀφʼ ἧς ἡμέρας γεγένηται ἐπʼ αὐτῆς ἔθνος.
\VS{25}Ἐπάταξε δὲ ἡ χάλαζα ἐν πάσῃ γῇ Αἰγύπτου, ἀπὸ ἀνθρώπου ἕως κτήνους· καὶ πᾶσαν βοτάνην τὴν ἐν τῷ πεδίῳ ἐπάταξεν ἡ χάλαζα· καὶ πάντα τὰ ξύλα τὰ ἐν τοῖς πεδίοις συνέτριψεν ἡ χάλαζα.
\VS{26}Πλὴν ἐν γῇ Γεσὲμ, οὗ ἦσαν οἱ υἱοὶ Ἰσραὴλ, οὐκ ἐγένετο ἡ χάλαζα.
\VS{27}Ἀποστείλας δὲ Φαραὼ ἐκάλεσε Μωυσῆν καὶ Ἀαρὼν, καὶ εἶπεν αὐτοῖς, ἡμάρτηκα τὸ νῦν· ὁ Κύριος δίκαιος, ἐγὼ δὲ καὶ ὁ λαός μου ἀσεβεῖς.
\VS{28}Εὔξασθε οὖν περὶ ἐμοῦ πρὸς Κύριον, καὶ παυσάσθω τοῦ γενηθῆναι φωνὰς Θεοῦ, καὶ χάλαζαν, καὶ πῦρ· καὶ ἐξαποστελῶ ὑμᾶς, καὶ οὐκέτι προστεθήσεσθε μένειν.
\VS{29}Εἶπε δὲ αὐτῷ Μωυσῆς, ὡς ἂν ἐξέλθω τὴν πόλιν, ἐκπετάσω τὰς χεῖράς μου πρὸς τὸν Κύριον, καὶ αἱ φωναὶ παύσονται, καὶ ἡ χὰλαζα καὶ ὁ ὑετὸς οὐκ ἔσται ἔτι, ἵνα γνῷς ὅτι τοῦ Κυρίου ἡ γῆ.
\VS{30}Καὶ σὺ καὶ οἱ θεράποντές σου, ἐπίσταμαι ὅτι οὐδέπω πεφόβησθε τὸν Κύριον.
\VS{31}Τὸ δὲ λίνον καὶ ἡ κριθὴ ἐπλήγη· ἡ γὰρ κριθὴ παρεστηκυῖα, τὸ δὲ λίνον σπερματίζον.
\VS{32}Ὁ δὲ πυρὸς καὶ ἡ ὀλύρα οὐκ ἐπληγησαν, ὄψιμα γὰρ ἦν.
\VS{33}Ἐξῆλθε δὲ Μωυσῆς ἀπὸ Φαραὼ ἐκτὸς τῆς πόλεως, καὶ ἐξέτεινε τὰς χεῖρας πρὸς Κύριον· καὶ αἱ φωναὶ ἐπαύσαντο, καὶ ἡ χάλαζα καὶ ὁ ὑετὸς οὐκ ἔσταξεν ἔτι ἐπὶ τὴν γῆν.
\VS{34}Ἰδὼν δὲ Φαραὼ ὅτι πέπαυται ὁ ὑετὸς καὶ ἡ χάλαζα καὶ αἱ φωναὶ, προσέθετο τοῦ ἁμαρτάνειν· καὶ ἐβάρυνεν αὐτοῦ τὴν καρδίαν, καὶ τῶν θεραπόντων αὐτοῦ.
\VS{35}Καὶ ἐσκληρύνθη ἡ καρδία Φαραὼ, καὶ οὐκ ἐξαπέστειλε τοὺς υἱοὺς Ἰσραὴλ, καθάπερ ἐλάλησε Κύριος τῷ Μωυσῇ.

\par }\Chap{10}{\PP \VerseOne{1}Εἶπε δὲ Κύριος πρὸς Μωυσῆν, λέγων, εἴσελθε πρὸς Φαραὼ, ἐγὼ γὰρ ἐσκλήρυνα αὐτοῦ τὴν καρδίαν καὶ τῶν θεραπόντων αὐτοῦ, ἵνα ἑξῆς ἐπέλθῃ τὰ σημεῖα ταῦτα ἐπʼ αὐτούς·
\VS{2}ὅπως διηγήσησθε εἰς τὰ ὦτα τῶν τέκνων ὑμῶν, καὶ τοῖς τέκνοις τῶν τέκνων ὑμῶν, ὅσα ἐμπέπαιχα τοῖς Αἰγυπτίοις, καὶ τὰ σημεῖά μου, ἃ ἐποίησα ἐν αὐτοῖς· καὶ γνώσεσθε ὅτι ἐγὼ Κύριος.
\VS{3}Εἰσῆλθε δὲ Μωυσῆς καὶ Ἀαρὼν ἐναντίον Φαραὼ, καὶ εἶπαν αὐτῷ, τάδε λέγει Κύριος ὁ Θεὸς τῶν Ἐβραίων, ἕως τίνος οὐ βούλει ἐντραπῆναί με; ἔξαπόστειλον τὸν λαόν μου, ἵνα λατρεύσωσί μοι.
\VS{4}Ἐὰν δὲ μὴ θέλῃς σὺ ἐξαποστεῖλαι τὸν λαόν μου, ἰδοὺ ἐγὼ ἐπάγω ταύτην τὴν ὥραν αὔριον ἀκρίδα πολλὴν ἐπὶ πάντα τὰ ὅριά σου.
\VS{5}Καὶ καλύψει τὴν ὄψιν τῆς γῆς, καὶ οὐ δυνήσῃ κατιδεῖν τὴν γῆν· καὶ κατέδεται πᾶν τὸ περισσὸν τῆς γῆς τὸ καταλειφθὲν, ὃ κατέλιπεν ὑμῖν ἡ χάλαζα, καὶ κατέδεται πᾶν ξύλον τὸ φυόμενον ὑμῖν ἐπὶ τῆς γῆς.
\VS{6}Καὶ πλησθήσονταί σου αἱ οἰκίαι, καὶ αἱ οἰκίαι τῶν θεραπόντων σου, καὶ πᾶσαι αἱ οἰκίαι ἐν πάσῃ γῇ τῶν Αἰγυπτίων· ἃ οὐδέποτε ἑωράκασιν οἱ πατέρες σου, οὐδʼ οἱ πρόπαπποι αὐτῶν, ἀφʼ ἧς ἡμέρας γεγόνασιν ἐπὶ τῆς γῆς, ἔως τῆς ἡμέρας ταύτης· καὶ ἐκκλίνας Μωυσῆς ἐξῆλθεν ἀπὸ Φαραώ.
\VS{7}Καὶ λέγουσιν οἱ θεράποντες Φαραὼ πρὸς αὐτὸν, ἕως τίνος ἔσται τοῦτο ἡμῖν σκῶλον; ἐξαπόστειλον τοὺς ἀνθρώπους, ὅπως λατρεύσωσι τῷ Θεῷ αὐτῶν· ἢ εἰδέναι βούλει ὅτι ἀπόλωλεν Αἴγυπτος;
\VS{8}Καὶ ἀπέστρεψαν τόν τε Μωυσῆν καὶ Ἀαρὼν πρὸς Φαραὼ, καὶ εἶπεν αὐτοῖς, πορεύεσθε καὶ λατρεύσατε Κυρίῳ τῷ Θεῷ ὑμῶν· τίνες δὲ καὶ τίνες εἰσιν οἱ πορευόμενοι;
\VS{9}Καὶ λέγει Μωυσῆς, σὺν τοῖς νεανίσκοις καὶ πρεσβυτέροις πορευσόμεθα, σὺν τοῖς υἱοῖς καὶ θυγατράσι, καὶ προβάτοις, καὶ βουσὶν ἡμῶν· ἔστι γὰρ ἑορτὴ Κυρίου.
\VS{10}Καὶ εἶπε πρὸς αὐτοὺς, ἔστω οὕτω Κύριος μεθʼ ὑμῶν· καθότι ἀποστέλλω ὑμᾶς, μὴ καὶ τὴν ἀποσκευὴν ὑμῶν; ἴδετε ὅτι πονηρία πρόσκειται ὑμῖν.
\VS{11}Μὴ οὕτως· πορευέσθωσαν δὲ οἱ ἄνδρες, καὶ λατρευσάτωσαν τῷ Θεῷ· τοῦτο γὰρ αὐτοὶ ἐκζητεῖτε· ἐξέβαλον δὲ αὐτοὺς ἀπὸ προσώπου Φαραώ.
\VS{12}Εἶπε δὲ Κύριος πρὸς Μωυσῆν, ἔκτεινον τὴν χεῖρα ἐπὶ γῆν Αἰγύπτου· καὶ ἀναβήτω ἀκρὶς ἐπὶ τὴν γῆν, καὶ κατέδεται πᾶσαν βοτάνην τῆς γῆς, καὶ πάντα τὸν καρπὸν τῶν ξύλων, ὃν ὑπελίπετο ἡ χάλαζα.
\VS{13}Καὶ ἐπῇρε Μωυσῆς τὴν ῥάβδον εἰς τὸν οὐρανὸν, καὶ Κύριος ἐπήγαγεν ἄνεμον νότον ἐπὶ τὴν γῆν, ὅλην τὴν ἡμέραν ἐκείνην, καὶ ὅλην τὴν νύκτα· τὸ πρωῒ ἐγενήθη, καὶ ὁ ἄνεμος ὁ νότος ἀνέλαβεν τὴν ἀκρίδα,
\VS{14}καὶ ἀνήγαγεν αὐτὴν ἐπὶ πᾶσαν γῆν Αἰγύπτου· καὶ κατέπαυσεν ἐπὶ πάντα τὰ ὅρια Αἰγύπτου πολλὴ σφόδρα· προτέρα αὐτῆς οὐ γέγονε τοιαύτη ἀκρὶς, καὶ μετὰ ταῦτα οὐκ ἔσται οὕτως.
\VS{15}Καὶ ἐκάλυψε τὴν ὄψιν τῆς γῆς, καὶ ἐφθάρη ἡ γῆ· καὶ κατέφαγε πᾶσαν βοτάνην τῆς γῆς, καὶ πάντα τόν καρπὸν τῶν ξύλων, ὃς ὑπελείφθη ἀπὸ τῆς χαλάζης· οὐχ ὑπελείφθη χλωρὸν οὐδὲν ἐν τοῖς ξύλοις, καὶ ἐν πάσῃ βοτάνῃ τοῦ πεδίου, ἐν πάσηῃ γῇ Αἰγύπτου.
\par }{\PP \VS{16}Κατέσπευδε δὲ Φαραὼ καλέσαι Μωυσῆν καὶ Ἀαρὼν, λέγων, ἡμάρτηκα ἐναντίον Κυρίου τοῦ Θεοῦ ὑμῶν, καὶ εἰς ὑμᾶς.
\VS{17}Προσδέξασθε οὖν μου τὴν ἁμαρτίαν ἔτι νῦν, καὶ προσεύξασθε πρὸς Κύριον τὸν Θεὸν ὑμῶν, καὶ περιελέτω ἀπʼ ἐμοῦ τὸν θάνατον τοῦτον.
\VS{18}Ἐξῆλθε δὲ Μωυσῆς ἀπὸ Φαραὼ, καὶ ηὔξατο πρὸς τὸν Θεόν.
\VS{19}Καὶ μετέβαλε Κύριος ἄνεμον ἀπὸ θαλάσσης σφοδρὸν, καὶ ἀνέλαβε τὴν ἀκρίδα, καὶ ἔβαλεν αὐτὴν εἰς τὴν ἐρυθρὰν θαλάσσαν· καὶ οὐχ ὑπελείφη ἀκρὶς μία ἐν πάσῃ γῇ Αἰγύπτου.
\VS{20}Καὶ ἐσκλήρυνε Κύριος τὴν καρδίαν Φαραὼ, καὶ οὐκ ἐξαπέστειλε τοὺς υἱοὺς Ἰσραήλ.
\VS{21}Εἶπε δὲ Κύριος πρὸς Μωυσῆν, ἔκτεινον τὴν χεῖρά σου εἰς τὸν οὐρανὸν, καὶ γενηθήτω σκότος ἐπὶ γῆς Αἰγύπτου, ψηλαφητὸν σκότος.
\VS{22}Ἐξέτεινε δὲ Μωυσῆς τὴν χεῖρα εἰς τὸν οὐρανόν· καὶ ἐγένετο σκότος γνόφος, θύελλα ἐπὶ πᾶσαν γῆν Αἰγύπτου τρεῖς ἡμέρας·
\VS{23}Καὶ οὐκ εἶδεν οὐδεὶς τὸν ἀδελφὸν αὐτοῦ τρεῖς ἡμέρας· καὶ οὐκ ἐξανέστη οὐδεὶς ἐκ τῆς κοίτης αὐτοῦ τρεῖς ἡμέρας· πᾶσι δὲ τοῖς υἱοῖς Ἰσραὴλ φῶς ἦν ἐν πᾶσιν οἷς κατεγίνοντο.
\VS{24}Καὶ ἐκάλεσε Φαραὼ Μωυσῆν καὶ Ἀαρὼν, λέγων, Βαδίζετε, λατρεύσατε Κυρίῳ τῷ Θεῷ ὑμῶν, πλὴν τῶν προβάτων καὶ τῶν βοῶν ὑπολείπεσθε· καὶ ἡ ἀποσκευὴ ὑμῶν ἀποτρεχέτω μεθʼ ὑμῶν.
\VS{25}Καὶ εἶπε Μωυσῆς, ἀλλὰ καὶ σὺ δώσεις ἡμῖν ὁλοκαυτώματα καὶ θυσίας, ἂ ποιήσομεν Κυρίῳ τῷ Θεῷ ἡμῶν.
\VS{26}Καὶ τὰ κτήνη ἡμῶν πορεύσεται μεθʼ ἡμῶν, καὶ οὐχ ὑπολειψόμεθα ὁπλήν· ἀπʼ αὐτῶν γὰρ ληψόμεθα λατρεῦσαι Κυρίῳ τῷ Θεῷ ἡμῶν· ἡμεῖς δὲ οὐκ οἴδαμεν τί λατρεύσομεν Κυρίῳ τῷ Θεῷ ἡμῶν, ἕως τοῦ ἐλθεῖν ἡμᾶς ἐκεῖ.
\VS{27}Ἐσκλήρυνε δὲ Κύριος τὴν καρδίαν Φαραὼ, καὶ οὐκ ἐβουλήθη ἐξαποστεῖλαι αὐτούς.
\VS{28}Καὶ λέγει Φαραὼ, ἄπελθε ἀπʼ ἐμοῦ· πρόσεχε σεαυτῷ ἔτι προσθεῖναι ἰδεῖν μου τὸ πρόσωπον· ᾗ δʼ ἂν ἡμέρᾳ ὀφθῇς μοι, ἀποθανῇ.
\VS{29}Λέγει δὲ Μωυσῆς, εἴρηκας· οὐκ ἔτι ὀφθήσομαί σοι εἰς πρόσωπον.

\par }\Chap{11}{\PP \VerseOne{1}Εἶπε δὲ Κύριος πρὸς Μωυσῆν, ἔτι μίαν πληγὴν ἐγὼ ἐπάξω ἐπὶ Φαραὼ, καὶ ἐπʼ Αἴγυπτον, καὶ μετὰ ταῦτα ἐξαποστελεῖ ὑμᾶς ἐντεῦθεν· ὅταν δὲ ἐξαποστέλλῃ ὑμᾶς σὺν παντὶ, ἐκβαλεῖ ὑμᾶς ἐκβολῇ.
\VS{2}Λάλησον οὖν κρυφῇ εἰς τὰ ὦτα τοῦ λαοῦ, καὶ αἰτησάτω ἕκαστος παρὰ τοῦ πλησίον σκεύη ἀργυρᾶ καὶ χρυσὰ καὶ ἱματισμόν.
\VS{3}Κύριος δὲ ἔδωκε τὴν χάριν τῷ λαῷ αὐτοῦ ἐναντίον τῶν Αἰγυπτίων, καὶ ἔχρησαν αὐτοῖς· καὶ ὁ ἄνθρωπος Μωυσῆς μέγας ἐγενήθη σφόδρα ἐναντίον τῶν Αἰγυπτίων, καὶ ἐναντίον Φαραὼ, καὶ ἐναντίον τῶν θεραπόντων ἀτοῦ.
\VS{4}Καὶ εἶπε Μωυσῆς, τάδε λέγει Κύριος, περὶ μέσας νύκτας ἐγὼ εἰσπορεύομαι εἰς μέσον Αἰγύπτου·
\VS{5}Καὶ τελευτήσει πᾶν πρωτότοκον ἐν γῇ Αἰγύπτῳ, ἀπὸ πρωτοτόκου Φαραὼ, ὃς κάθηται ἐπὶ τοῦ θρόνου, καὶ ἕως πρωτοτόκου τῆς θεραπαίνης τῆς παρὰ τὸν μύλον, καὶ ἕως πρωτοτοκου παντος κτήνους.
\VS{6}Καὶ ἔσται κραυγὴ μεγάλη κατὰ πᾶσαν γῆν Αἰγύπτου, ἥτις τοιαύτη οὐ γέγονε, καὶ τοιαύτη οὐκ ἔτι προστεθήσεται.
\VS{7}Καὶ ἐν πᾶσι τοῖς υἱοῖς Ἰσραὴλ οὐ γρύξει κύων τῇ γλώσσῃ αὐτοῦ, ἀπὸ ἀνθρώπου ἕως κτήνους· ὅπως εἰδῇς ὅσα παραδοξάσει Κύριος ἀνὰ μέσον τῶν Αἰγυπτίων καὶ τοῦ Ἰσραήλ.
\VS{8}Καὶ καταβήσονται πάντες οἱ παῖδές σου οὗτοι πρός με, καὶ προσκυνήσουσί με, λέγοντες, ἔξελθε σὺ, καὶ πᾶς ὁ λαός σου, οὗ σὺ ἀφηγῇ· καὶ μετὰ ταῦτα ἐξελεύσομαι· ἐξῆλθε δὲ Μωυσῆς ἀπὸ Φαραὼ μετὰ θυμοῦ.
\VS{9}Εἶπε δὲ Κύριος πρὸς Μωυσῆν, οὐκ εἰσακούσεται ὑμῶν Φαραὼ, ἵνα πληθύνων πληθυνῶ μου τὰ σημεῖα, καὶ τὰ τέρατα ἐν γῇ Αἰγύπτῳ.
\VS{10}Μωσῆς δὲ καὶ Ἀαρὼν ἐποίησαν πάντα τὰ σημεῖα καὶ τὰ τέρατα ταῦτα ἐν γῇ Αἰγύπτῳ ἐναντίον Φαραώ· ἐσκλήρυνε δὲ Κύριος τὴν καρδίαν Φαραὼ, καὶ οὐκ εἰσήκουσεν ἐξαποστεῖλαι τοὺς υἱοὺς Ἰσραὴλ ἐκ γῆς Αἰγύπτου.

\par }\Chap{12}{\PP \VerseOne{1}Εἶπε δὲ Κύριος πρὸς Μευσῆν καὶ Ἀαρὼν ἐν γῇ Αἰγύπτου, λέγων,
\VS{2}ὁ μὴν οὗτος ὑμῖν ἀρχὴ μηνῶν· πρῶτός ἐστιν ὑμῖν ἐν τοῖς μησὶ τοῦ ἐνιαυτοῦ.
\VS{3}Λάλησον πρὸς πᾶσαν συναγωγὴν υἱῶν Ἰσραὴλ, λέγων, τῇ δεκάτῃ τοῦ μηνὸς τούτου λαβέτωσαν ἕκαστος πρόβατον κατʼ οἴκους πατριῶν, ἕκαστος πρόβατον κατʼ οἰκίαν.
\VS{4}Ἐὰν δὲ ὀλιγοστοὶ ὦσιν ἐν τῇ οἰκίᾳ, ὥστε μὴ εἶναι ἱκανοὺς εἰς πρόβατον, συλλήψεται μεθʼ ἑαυτοῦ τὸν γείτονα τὸν πλησίον αὐτοῦ· κατὰ ἀριθμὸν ψυχῶν, ἕκαστος τὸ ἀρκοῦν αὐτῷ συναριθμήσεται εἰς πρόβατον.
\VS{5}Πρόβατον τέλειον, ἄρσεν, ἐνιαύσιον ἔσται ὑμῖν· ἀπὸ τῶν ἀρνῶν καὶ τῶν ἐρίφων λὴψεσθε.
\VS{6}Καὶ ἔσται ὑμῖν διατετηρημένον ἕως τῆς τεσσαρεσκαιδεκάτης τοῦ μηνὸς τούτου· καὶ σφάξουσιν αὐτὸ πᾶν τὸ πλῆθος συναγωγῆς υἱῶν Ἰσραὴλ πρὸς ἑσπέραν.
\VS{7}Καὶ λήψονται ἀπὸ τοῦ αἵματος, καὶ θήσουσιν ἐπὶ τῶν δύο σταθμῶν καὶ ἐπὶ τὴν φλιὰν, ἐν τοῖς οἴκοις ἐν οἷς ἐὰν φάγωσιν αὐτὰ ἐν αὐτοῖς.
\VS{8}Καὶ φάγονται τὰ κρέα τῇ νυκτὶ ταύτῃ ὀπτὰ πυρὶ, καὶ ἄζυμα ἐπὶ πικρίδων ἔδονται.
\VS{9}Οὐκ ἔδεσθε ἀπʼ αὐτῶν ὠμὸν, οὐδὲ ἡψημένον ἐν ὕδατι, ἀλλʼ ἢ ὀπτὰ πυρὶ, κεφαλὴν σὺν τοῖς ποσὶ καὶ τοῖς ἐνδοσθίοις.
\VS{10}Οὐκ ἀπολείψεται ἀπʼ αὐτοῦ ἕως πρωΐ· καὶ ὀστοῦν οὐ συντρίψετε ἀπʼ αὐτοῦ· τὰ δὲ καταλειπόμενα ἀπʼ αὐτοῦ ἕως πρωῒ ἐν πυρὶ κατακαύσετε.
\VS{11}Οὕτω δὲ φάγεσθε αὐτό· αἱ ὀσφύες ὑμῶν περιεζωσμέναι, καὶ τὰ ὑποδήματα ἐν τοῖς ποσὶν ὑμῶν, καὶ αἱ βακτηρίαι ἐν ταῖς χερσὶν ὑμῶν· καὶ ἔδεσθε αὐτὸ μετὰ σπουδῆς· Πάσχα ἐστὶ Κυρίῳ.
\VS{12}Καὶ διελεύσομαι ἐν γῇ Αἰγύπτῳ ἐν τῇ νυκτὶ ταύτῃ, καὶ πατάξω πᾶν πρωτότοκον ἐν γῇ Αἰγύπτῳ ἀπὸ ἀνθρώπου ἕως κτήνους· καὶ ἐν πᾶσι τοῖς θεοῖς τῶν Αἰγυπτίων ποιήσω τὴν ἐκδίκησιν· ἐγὼ Κύριος.
\VS{13}Καὶ ἔσται τὸ αἷμα ὑμῖν ἐν σημείῳ ἐπὶ τῶν οἰκιῶν, ἐν αἷς ὑμεῖς ἔστε ἐκεῖ· καὶ ὄψομαι τὸ αἷμα, καὶ σκεπάσω ὑμᾶς, καὶ οὐκ ἔσται ἐν ὑμῖν πληγὴ τοῦ ἐκτριβῆναι ὅταν παίω ἐν γῇ Αἰγύπτῳ.
\par }{\PP \VS{14}Καὶ ἔσται ἡ ἡμέρα ὑμῖν αὕτη μνημόσυνον, καὶ ἑορτάσετε αὐτὴν ἑορτὴν Κυρίῳ εἰς πάσας τὰς γενεὰς ὑμῶν· νόμιμον αἰώνιον ἑορτάσετε αὐτήν.
\VS{15}Ἑπτὰ ἡμέρας ἄζυμα ἔδεσθε· ἀπὸ δὲ τῆς ἡμέρας τῆς πρώτης, ἀφανιεῖτε ζύμην ἐκ τῶν οἰκιῶν ὑμῶν· πᾶς ὃς ἂν φάγῃ ζύμην, ἐξολοθρευθήσεται ἡ ψυχὴ ἐκείνη ἐξ Ἰσραήλ, ἀπὸ τῆς ἡμέρας τῆς πρώτης ἕως τῆς ἡμέρας τῆς ἑβδόμης.
\VS{16}Καὶ ἡ ἡμέρα ἡ πρώτη, κληθήσεται ἁγία· καὶ ἡ ἡμέρα ἡ ἑβδόμη, κλητὴ ἁγία ἔσται ὑμῖν· πᾶν ἔργον λατρευτὸν οὐ ποιήσετε ἐν αὐταῖς, πλὴν ὅσα ποιηθήσεται πάσῃ ψυχῇ, τοῦτο μόνον ποιηθήσεται ὑμῖν.
\VS{17}Καὶ φυλάξετε τὴν ἐντολὴν ταύτην· ἐν γὰρ τῇ ἡμέρᾳ ταύτῃ ἐξάξω τὴν δύναμιν ὑμῶν ἐκ γῆς Αἰγύπτου, καὶ ποιήσετε τὴν ἡμέραν ταύτην εἰς γενεὰς ὑμῶν νόμιμον αἰώνιον,
\VS{18}ἐναρχόμενοι τῇ τεσσαρεσκαιδεκάτῃ ἡμέρᾳ τοῦ μηνὸς τοῦ πρώτου, ἀφʼ ἑσπέρας ἔδεσθε ἄζυμα, ἕως ἡμέρας μίας καὶ εἰκάδος τοῦ μηνὸς, ἕως ἑσπέρας.
\VS{19}Ἑπτὰ ἡμέρας ζύμη οὐχ εὑρεθήσεται ἐν ταῖς οἰκιαῖς ὑμῶν· πᾶς ὃς ἂν φάγῃ ζυμωτὸν, ἐξολοθρευθήσεται ἡ ψυχὴ ἐκείνη ἐκ συναγωγῆς Ἰσραήλ· ἔν τε τοῖς γειώραις, καὶ αὐτόχθοσι τῆς γῆς.
\VS{20}Πᾶν ζυμωτὸν οὐκ ἔδεσθε, ἐν παντὶ δὲ κατοικητηρίῳ ὑμῶν ἔδεσθε ἄζυμα.
\par }{\PP \VS{21}Ἐκάλεσε δὲ Μωυσῆς πᾶσαν γερουσίαν υἱῶν Ἰσραὴλ, καὶ εἶπε πρὸς αὐτοὺς, ἀπελθόντες λάβετε ὑμῖν αὐτοῖς πρόβατον κατὰ συγγενείας ὑμῶν, καὶ θύσατε τὸ πάσχα.
\VS{22}Λήψεσθε δὲ δέσμην ὑσσώπου, καὶ βάψαντες ἀπὸ τοῦ αἵματος τοῦ παρὰ τὴν θύραν, καθίξετε τῆς φλιᾶς, καὶ ἐπʼ ἀμφοτέρων τῶν σταθμῶν, ἀπὸ τοῦ αἵματος ὅ ἐστι παρὰ τὴν θύραν· ὑμεῖς δὲ οὐκ ἐξελεύσεσθε ἕκαστος τὴν θύραν τοῦ οἴκου αὐτοῦ ἕως πρωΐ.
\VS{23}Καὶ παρελεύσεται Κύριος πατάξαι τοὺς Αἰγυπτίους, καὶ ὄψεται τὸ αἷμα ἐπὶ τῆς φλιᾶς, καὶ ἐπʼ ἀμφοτέρων τῶν σταθμῶν· καὶ παρελεύσεται Κύριος τὴν θύραν, καὶ οὐκ ἀφήσει τὸν ὀλοθρεύοντα εἰσελθεῖν εἰς τὰς οἰκίας ὑμῶν πατάξαι.
\VS{24}Καὶ φυλάξασθε τὸ ῥῆμα τοῦτο νόμιμον σεαυτῷ, καὶ τοῖς υἱοῖς σου, ἕως αἰῶνος.
\VS{25}Ἐὰν δὲ εἰσέλθητε εἰς τὴν γῆν, ἣν ἂν δῷ Κύριος ὑμῖν, καθότι ἐλάλησε, φυλάξασθε τὴν λατρείαν ταύτην.
\VS{26}Καὶ ἐσται ἐὰν λέγωσι πρὸς ὑμᾶς οἱ υἱοὶ ὑμῶν, τίς ἡ λατρεία αὕτη;
\VS{27}Καὶ ἐρεῖτε αὐτοῖς, θυσία τὸ πάσχα τοῦτο Κυρίῳ, ὡς ἐσκέπασε τοὺς οἴκους τῶν υἱῶν Ἰσραὴλ ἐν Αἰγύπτῳ, ἡνίκα ἐπάταξε τοὺς Αἰγυπτίους, τοὺς δὲ οἴκους ἡμῶν ἐῤῥύσατο· καὶ κύψας ὁ λαὸς προσεκύνησε.
\VS{28}Καὶ ἀπελθόντες ἐποίησαν οἱ υἱοὶ Ἰσραὴλ, καθὰ ἐνετείλατο Κύριος τῷ Μωυσῇ καὶ Ἀαρῶν, οὕτως ἐποίησαν.
\par }{\PP \VS{29}Ἐγενήθη δὲ μεσούσης τῆς νυκτὸς, καὶ Κύριος ἐπάταξε πᾶν πρωτότοκον ἐν γῇ Αἰγύπτῳ, ἀπὸ πρωτοτόκου Φαραὼ τοῦ καθημένου ἐπὶ τοῦ θρόνου, ἕως πρωτοτόκου τῆς αἰχμαλωτίδος τῆς ἐν τῷ λάκκῳ, καὶ ἕως πρωτοτόκου παντὸς κτήνους.
\VS{30}Καὶ ἀναστὰς Φαραὼ νυκτὸς, καὶ οἱ θεράποντες αὐτοῦ, καὶ πάντες οἱ Αἰγύπτιοι, καὶ ἔγενήθη κραυγὴ μεγάλη ἐν πάσῃ γῇ Αἰγύπτῳ· οὐ γὰρ ἦν οἰκία, ἐν ᾗ οὐκ ἦν ἐν αὐτῇ τεθνηκώς.
\VS{31}Καὶ ἐκάλεσε Φαραὼ Μωυσῆν καὶ Ἀαρὼν νυκτὸς, καὶ εἶπεν αὐτοῖς, ἀνάστητε, καὶ ἐξέλθατε ἐκ τοῦ λαοῦ μου, καὶ ὑμεῖς, καὶ οἱ υἱοὶ Ἰσραήλ· βαδίζετε καὶ λατρεύσατε Κυρίῳ τῷ Θεῷ ὑμῶν, καθὰ λέγετε.
\VS{32}Καὶ τὰ πρόβατα καὶ τοὺς βόας ὑμῶν ἀναλαβόντες πορεύεσθε· εὐλογήσατε δὴ κᾀμέ.
\VS{33}Καὶ κατεβιάζοντο οἱ Αἰγύπτιοι τὸν λαὸν σπουδῇ ἐκβαλεῖν αὐτοὺς ἐν τῆς γῆς· εἶπαν γὰρ, ὅτι πάντες ἡμεῖς ἀποθνήσκομεν.
\VS{34}Ἀνέλαβε δὲ ὁ λαὸς τὸ σταῖς αὐτῶν, πρὸ τοῦ ζυμωθῆναι τὰ φυράματα αὐτῶν, ἐνδεδεμένα ἐν τοῖς ἱματίοις αὐτῶν ἐπὶ τῶν ὤμαν.
\VS{35}Οἱ δὲ υἱοὶ Ἰσραὴλ ἐποίησαν, καθὰ συνέταξεν αὐτοῖς Μωυσῆς, καὶ ᾔτησαν παρὰ τῶν Αἰγυπτίων σκεύη ἀργυρᾶ καὶ χρυσᾶ καὶ ἱματισμόν.
\VS{36}Καὶ ἔδωκε Κύριος τὴν χάριν τῷ λαῷ αὐτοῦ ἐναντίον τῶν Αἰγυπτίων, καὶ ἔχρησαν αὐτοῖς· καὶ ἐσκύλευσαν τοὺς Αἰγυπτίους.
\par }{\PP \VS{37}Ἀπάραντες δὲ υἱοὶ Ἰσραὴλ ἐκ Ῥαμεσσῆ εἰς Σοκχὼθ εἰς ἑξακοσίας χιλιάδας πεζῶν, οἱ ἄνδρες, πλὴν τῆς ἀποσκευῆς.
\VS{38}Καὶ ἐπίμικτος πολὺς συνανέβη αὐτοῖς, καὶ πρόβατα, καὶ βόες, καὶ κτήνη πολλὰ σφόδρα.
\VS{39}Καὶ ἔπεψαν τὸ σταῖς ὃ ἐξήνεγκαν ἐξ Αἰγύπτου, ἐγκρυφίας ἀζύμους, οὐ γὰρ ἐζυμώθη· ἐξέβαλον γὰρ αὐτοὺς οἱ Αἰγύπτιοι, καὶ οὐκ ἠδυνήθησαν ἐπιμεῖναι, οὐδὲ ἐπισιτισμὸν ἐποίησαν ἑαυτοῖς εἰς τὴν ὁδόν.
\VS{40}Ἡ δὲ κατοίκησις τῶν υἱῶν Ἰσραὴλ, ἣν κατῴκησαν ἐν γῇ Αἰγύπτῳ καὶ ἐν γῇ Χαναὰν, ἔτη τετρακόσια τριάκοντα.
\VS{41}Καὶ ἐγένετο μετὰ τὰ τετρακόσια τριάκοντα ἔτη, ἐξῆλθε πᾶσα ἡ δύναμις Κυρίου ἐκ γῆς Αἰγύπτου νυκτός.
\VS{42}Προφυλακή ἐστι τῷ Κυρίῳ, ὥστε ἐξαγαγεῖν αὐτοὺς ἐκ γῆς Αἰγύπτου· ἐκείνη ἡ νὺξ αὕτη, προφυλακὴ Κυρίῳ, ὥστε πᾶσι τοῖς υἱοῖς Ἰσραὴλ εἶναι εἰς γενεὰς αὐτῶν.
\VS{43}Εἶπε δὲ Κύριος πρὸς Μωυσῆν καὶ Ἀαρὼν, οὗτος ὁ νόμος τοῦ πάσχα· πᾶς ἀλλογενὴς οὐκ ἔδεται ἀπʼ αὐτοῦ·
\VS{44}Καὶ πάντα οἰκέτην ἢ ἀργυρώνητον περιτεμεῖς αὐτόν· καὶ τότε φάγεται ἀπʼ αὐτοῦ.
\VS{45}Πάροικος ἢ μισθωτὸς οὐκ ἔδεται ἀπʼ αὐτοῦ.
\VS{46}Ἐν οἰκίᾳ μιᾷ βρωθήσεται, καὶ οὐκ ἐξοίσετε ἐκ τῆς οἰκίας τῶν κρεῶν ἔξω· καὶ ὀστοῦν οὐ συντρίψετε ἀπʼ αὐτοῦ.
\VS{47}Πᾶσα συναγωγὴ υἱῶν Ἰσραὴλ ποιήσει αὐτό.
\VS{48}Ἐὰν δέ τις προσέλθῃ πρὸς ὑμᾶς προσήλυτος ποιῆσαι τὸ πάσχα Κυρίῳ, περιτεμεῖς αὐτοῦ πᾶν ἀρσενικόν, καὶ τότε προσελεύσεται ποιῆσαι αὐτό· καὶ ἔσται ὥσπερ καὶ ὁ αὐτόχθων τῆς γῆς· πᾶς ἀπερίτμητος οὐκ ἔδεται ἀπʼ αὐτοῦ.
\VS{49}Νόμος εἷς ἔσται τῷ ἐγχωρίῳ, καὶ τῷ προσελθόντι προσηλύτῳ ἐν ὑμῖν.
\VS{50}Καὶ ἐποίησαν οἱ υἱοὶ Ἰσραὴλ καθὰ ἐνετείλατο Κύριος τῷ Μωυσῇ καὶ Ἀαρὼν πρὸς αὐτούς, οὕτως ἐποίησαν.
\VS{51}Καὶ ἐγένετο ἐν τῇ ἡμέρᾳ ἐκείνῃ, ἐξήγαγε Κύριος τοὺς υἱοὺς Ἰσραὴλ ἐκ γῆς Αἰγύπτου σὺν δυνάμει αὐτῶν.

\par }\Chap{13}{\PP \VerseOne{1}Εἶπε δὲ Κύριος πρὸς Μωυσῆν, λέγων,
\VS{2}ἁγίασόν μοι πᾶν πρωτότοκον πρωτογενὲς διανοῖγον πᾶσαν μήτραν ἐν τοῖς υἱοῖς Ἰσραὴλ ἀπὸ ἀνθρώπου ἕως κτήνους, ἐμοί ἐστιν.
\VS{3}Εἶπε δὲ Μωυσῆς πρὸς τὸν λαὸν, μνημονεύετε τὴν ἡμέραν ταύτην, ἐν ᾗ ἐξήλθατε ἐκ γῆς Αἰγύπτου, ἐξ οἴκου δουλείας· ἐν γὰρ χειρὶ κραταιᾷ ἐξήγαγεν ὑμᾶς Κύριος ἐντεῦθεν· καὶ οὐ βρωθήσεται ζύμη.
\VS{4}Ἐν γὰρ τῇ σήμερον ὑμεῖς ἐκπορεύεσθε ἐν μηνὶ τῶν νέων.
\VS{5}Καὶ ἔσται ἡνίκα ἐὰν εἰσαγάγῃ σε Κύριος ὁ Θεός σου εἰς τὴν γῆν τῶν Χαναναίων, καὶ Χετταίων, καὶ Ἀμοῤῥαίων, καὶ Εὐαίων, καὶ Ἰεβουσαίων, καὶ Γεργεσαίων, καὶ Φερεζαίων, ἣν ὤμοσε τοῖς πατράσι σου, δοῦναί σοι γῆν ῥέουσαν γάλα καὶ μέλι· καὶ ποιήσεις τὴν λατρείαν ταύτην ἐν τῷ μηνὶ τούτῳ.
\VS{6}Ἓξ ἡμέρας ἔδεσθε ἄζυμα, τῇ δὲ ἡμέρᾳ τῇ ἑβδόμῃ ἑορτὴ Κυρίου.
\VS{7}Ἄζυμα ἔδεσθε ἑπτὰ ἡμέρας· οὐκ ὀφθήσεταί σοι ζυμωτὸν, οὐδὲ ἔσται σοι ζύμη ἐν πᾶσι τοῖς ὁρίοις σου.
\VS{8}Καὶ ἀναγγελεῖς τῷ υἱῷ σου ἐν τῇ ἡμέρᾳ ἐκείνῃ, λέγων, διὰ τοῦτο ἐποίησε Κύριος ὁ Θεός μοι, ὡς ἐξεπορευόμην ἐξ Αἰγύπτου.
\VS{9}Καὶ ἔσται σοι σημεῖον ἐπὶ τῆς χειρός σου, καὶ μνημόσυνον πρὸ ὀφθαλμῶν σου, ὅπως ἂν γένηται ὁ νόμος Κυρίου ἐν τῷ στόματί σου· ἐν γὰρ χειρὶ κραταιᾷ ἐξήγαγέ σε Κύριος ὁ Θεὸς ἐξ Αἰγύπτου.
\VS{10}Καὶ φυλάξασθε τὸν νόμον τοῦτον κατὰ καιροὺς ὡρῶν, ἀφʼ ἡμερῶν εἰς ἡμέρας.
\par }{\PP \VS{11}Καὶ ἔσται ὡς ἂν εἰσαγάγῃ σε Κύριος ὁ Θεός σου εἰς τὴν γῆν τῶν Χαναναίων, ὃν τρόπον ὤμοσε τοῖς πατράσι σου, καὶ δώσει σοι αὐτήν.
\VS{12}Καὶ ἀφελεῖς πᾶν διανοῖγον μήτραν, τὰ ἀρσενικὰ τῷ Κυρίῳ· πᾶν διανοῖγον μήτραν ἐκ βουκολίων ἢ ἐν τοῖς κτήνεσί σου, ὅσα ἐὰν γένηταί σοι, τὰ ἀρσενικὰ ἁγιάσεις τῷ Κυρίῳ.
\VS{13}Πᾶν διανοῖγον μήτραν ὄνου, ἀλλάξεις προβάτῳ· ἐὰν δὲ μὴ ἀλλάξῃς, λυτρώσῃ αὐτό· πᾶν πρωτότοκον ἀνθρώπου τῶν υἱῶν σου λυτρώσῃ.
\VS{14}Ἐὰν δὲ ἐρωτήσῃ σε ὁ υἱός σου μετὰ ταῦτα, λέγων, τί τοῦτο; καὶ ἐρεῖς αὐτῷ, ὅτι ἐν χειρὶ κραταιᾷ ἐξήγαγεν Κύριος ἡμᾶς ἐκ γῆς Αἰγύπτου, ἐξ οἴκου δουλείας.
\VS{15}Ἡνίκα δὲ ἐσκλήρυνε Φαραὼ ἐξαποστεῖλαι ἡμᾶς, ἀπέκτεινε πᾶν πρωτότοκον ἐν γῇ Αἰγύπτῳ, ἀπὸ πρωτοτόκων ἀνθρώπων ἕως πρωτοτόκων κτηνῶν· διὰ τοῦτο ἐγὼ θύω πᾶν διανοῖγον μήτραν, τὰ ἀρσενικὰ τῷ Κυρίῳ, καὶ πᾶν πρωτότοκον τῶν υἱῶν μου λυτρώσομαι.
\VS{16}Καὶ ἔσται εἰς σημεῖον ἐπὶ τῆς χειρός σου, καὶ ἀσαλευτον πρὸ ὀφθαλμων σου· ἐν γὰρ χειρὶ κραταιᾷ ἐξήγαγέ σε Κύριος ἐξ Αἰγύπτου.
\par }{\PP \VS{17}Ὡς δὲ ἐξαπέστειλε Φαραὼ τὸν λαὸν, οὐχ ὡδήγησεν αὐτοὺς ὁ Θεὸς ὁδὸν γῆς· Φυλιστιεὶμ, ὅτι ἐγγὺς ἦν· εἶπε γὰρ ὁ Θεὸς, μήποτε μεταμελήσῃ τῷ λαῷ ἰδόντι πόλεμον, καὶ ἀποστρέψῃ εἰς Αἴγυπτον.
\VS{18}Καὶ ἐκύκλωσεν ὁ Θεὸς τὸν λαὸν ὁδὸν τὴν εἰς τὴν ἔρημον, εἰς τὴν ἐρυθρὰν θάλασσαν· πέμπτῃ δὲ γενεᾷ ἀνέβησαν οἱ υἱοὶ Ἰσραὴλ ἐκ γῆς Αἰγύπτου.
\VS{19}Καὶ ἔλαβε Μωυσῆς τὰ ὀστᾶ Ἰωσὴφ μεθʼ ἑαυτοῦ· ὅρκῳ γὰρ ὥρκισεν τοὺς υἱοὺς Ἰσραὴλ, λέγων, ἐπισκοπῇ ἐπισκέψεται ὑμᾶς Κύριος, καὶ συνανοίσετε μου τὰ ὀστᾶ ἐντεῦθεν μεθʼ ὑμῶν.
\par }{\PP \VS{20}Ἐξάραντες δὲ οἱ υἱοὶ Ἰσραὴλ ἐκ Σοκχὼθ, ἐστρατοπέδευσαν ἐν Ὀθὼμ παρὰ τὴν ἔρημον.
\VS{21}Ὁ δὲ Θεὸς ἡγεῖτο αὐτῶν, ἡμέρας μὲν ἐν στύλῳ νεφέλης, δεῖξαι αὐτοῖς τὴν ὁδόν· τὴν δὲ νύκτα ἐν στύλῳ πυρός.
\VS{22}Οὐκ ἐξέλιπεν δὲ ὁ στύλος τῆς νεφέλης ἡμέρας, καὶ ὁ στύλος τοῦ πυρὸς νυκτὸς, ἐναντίον τοῦ λαοῦ παντός.

\par }\Chap{14}{\PP \VerseOne{1}Καὶ ἐλάλησε Κύριος πρὸς Μωυσῆν, λέγων,
\VS{2}Λάλησον τοῖς υἱοῖς Ἰσραὴλ, καὶ ἀποστρέψαντες στρατοπεδευσάτωσαν ἀπέναντι τῆς ἐπαύλεως, ἀνὰ μέσον Μαγδώλου καὶ ἀνὰ μέσον τῆς θαλάσσης, ἐξεναντίας Βεελσεπφῶν· ἐνώπιον αὐτῶν στρατοπεδεύσεις ἐπὶ τῆς θαλάσσης.
\VS{3}Καὶ ἐρεῖ Φαραὼ τῷ λαῷ αὐτοῦ, οἱ υἱοὶ Ἰσραὴλ πλανῶνται οὗτοι ἐν τῇ γῇ, συγκέκλεικε γὰρ αὐτοὺς ἡ ἔρημος.
\VS{4}Ἐγὼ δὲ σκληρυνῶ τὴν καρδίαν Φαραὼ, καὶ καταδιώξεται ὀπίσω αὐτῶν· καὶ ἐνδοξασθήσομαι ἐν Φαραῷ, καὶ ἐν πάσῃ τῇ στρατίᾳ αὐτοῦ· καὶ γνώσονται πάντες οἱ Αἰγύπτιοι ὅτι ἐγώ εἰμι Κύριος· καὶ ἐποίησαν οὕτως.
\VS{5}Καὶ ἀνηγγέλη τῷ βασιλεῖ τῶν Αἰγυπτίων ὅτι πέφευγεν ὁ λαός· καὶ μετεστράφη ἡ καρδία Φαραὼ, καὶ τῶν θεραπόντων αὐτοῦ, ἐπὶ τὸν λαὸν, καὶ εἶπαν, τί τοῦτο ἐποιήσαμεν, τοῦ ἐξαποστεῖλαι τοὺς υἱοὺς Ἰσραὴλ, τοῦ μὴ δουλεύειν ἡμῖν;
\VS{6}Ἔζευξεν οὖν Φαραὼ τὰ ἅρματα αὐτοῦ, καὶ πάντα τὸν λαὸν αὐτοῦ συναπήγαγε μεθʼ ἑαυτοῦ,
\VS{7}καὶ λαβὼν ἑξακόσια ἅρματα ἐκλεκτὰ, καὶ πᾶσαν τὴν ἵππον τῶν Αἰγυπτίων, καὶ τριστάτας ἐπὶ πάντων.
\VS{8}Καὶ ἐσκλήρυνε Κύριος τὴν καρδίαν Φαραὼ βασιλέως Αἰγύπτου, καὶ τῶν θεραπόντων αὐτοῦ, καὶ κατεδίωξεν ὀπίσω τῶν υἱῶν Ἰσραήλ· οἱ δὲ υἱοὶ Ἰσραὴλ ἐξεπορεύοντο ἐν χειρὶ ὑψηλῇ.
\VS{9}Καὶ κατεδίωξαν οἱ Αἰγύπτιοι ὀπίσω αὐτῶν, καὶ εὕροσαν αὐτοὺς παρεμβεβληκότας παρὰ τὴν θάλασσαν· καὶ πᾶσα ἡ ἵππος καὶ τὰ ἅρματα Φαραὼ, καὶ οἱ ἱππεῖς, καὶ ἡ στρατιὰ αὐτοῦ ἀπέναντι τῆς ἐπαύλεως, ἐξεναντίας Βεελσεπφῶν.
\VS{10}Καὶ Φαραὼ προσῆγε· καὶ ἀναβλέψαντες οἱ υἱοὶ Ἰσραὴλ τοῖς ὀφθαλμοῖς ὁρῶσι, καὶ οἱ Αἰγύπτιοι ἐστρατοπέδευσαν ὀπίσω αὐτῶν· καὶ ἐφοβήθησαν σφόδρα· ἀνεβόησαν δὲ οἱ υἱοὶ Ἰσραὴλ πρὸς Κύριον.
\VS{11}Καὶ εἶπαν πρὸς Μωυσῆν, παρὰ τὸ μὴ ὑπάρχειν μνήματα ἐν γῇ Αἰγύπτῳ, ἐξήγαγες ἡμᾶς θανατῶσαι ἐν τῇ ἐρήμῳ· τί τοῦτο ἐποίησας ἡμῖν, ἐξαγαγὼν ἐξ Αἰγύπτου;
\VS{12}Οὐ τοῦτο ἦν τὸ ῥῆμα, ὃ ἐλαλήσαμεν πρὸς σὲ ἐν Αἰγύπτῳ, λέγοντες, πάρες ἡμᾶς, ὅπως δουλεύσωμεν τοῖς Αἰγυπτίοις; κρεῖσσον γὰρ ἡμᾶς δουλεύειν τοῖς Αἰγυπτίοις, ἢ ἀποθανεῖν ἐν τῇ ἐρήμῳ ταύτῃ.
\par }{\PP \VS{13}Εἶπε δὲ Μωυσῆς πρὸς τὸν λαὸν, θαρσεῖτε, στῆτε καὶ ὁρᾶτε τὴν σωτηρίαν τὴν παρὰ τοῦ Κυρίου, ἣν ποιήσει ἡμῖν σήμερον· ὃν τρόπον γὰρ ἑωράκατε τοὺς Αἰγυπτίους σήμερον, οὐ προσθήσεσθε ἔτι ἰδεῖν αὐτοὺς εἰς τὸν αἰῶνα χρόνον·
\VS{14}Κύριος πολεμήσει περὶ ὑμῶν, καὶ ὑμεῖς σιγήσετε.
\VS{15}Εἶπε δὲ Κύριος πρὸς Μωυσῆν, τί βοᾷς πρός με; λάλησον τοῖς υἱοῖς Ἰσραὴλ, καὶ ἀναζευξάτωσαν.
\VS{16}Καὶ σὺ ἔπαρον τῇ ῥάβδῳ σου, καὶ ἔκτεινον τὴν χεῖρά σου ἐπὶ τὴν θάλασσαν, καὶ ῥῆξον αὐτήν· καὶ εἰσελθάτωσαν οἱ υἱοὶ Ἰσραὴλ εἰς μέσον τῆς θαλάσσης κατὰ τὸ ξηρόν.
\VS{17}Καὶ ἰδοὺ ἐγὼ σκληρυνῶ τὴν καρδίαν Φαραὼ, καὶ τῶν Αἰγυπτίων πάντων, καὶ εἰσελεύσονται ὀπίσω αὐτῶν· καὶ ἐνδοξασθήσομαι ἐν Φαραῷ, καὶ ἐν πάσῃ τῇ στρατιᾷ αὐτοῦ, καὶ ἐν τοῖς ἅρμασι, καὶ ἐν τοῖς ἵπποις αὐτοῦ.
\VS{18}Καὶ γνώσονται πάντες οἱ Αἰγύπτιοι ὅτι ἐγώ εἰμὶ Κύριος, ἐνδοξαζομένου μου ἐν Φαραῷ, καὶ ἐν τοῖς ἅρμασι, καὶ ἵπποις αὐτοῦ.
\VS{19}Ἐξῇρε δὲ ὁ Ἄγγελος τοῦ Θεοῦ ὁ προπορευόμενος τῆς παρεμβολῆς τῶν υἱῶν Ἰσραὴλ, καὶ ἐπορεύθη ἐκ τῶν ὄπισθεν· ἐξῇρε δὲ καὶ ὁ στύλος τὴς νεφέλης ἀπὸ προσώπου αὐτῶν, καὶ ἔστη ἐκ τῶν ὀπίσω αὐτῶν.
\VS{20}Καὶ εἰσῆλθεν ἀνὰ μέσον τῆς παρεμβολῆς τῶν Αἰγυπτίων, καὶ ἀνὰ μέσον τῆς παρεμβολῆς τῶν Αἰγυπίων, καὶ ἀνὰ μέσον τῆς παρεμβολῆς Ἰσραὴλ, καὶ ἔστη· καὶ ἐγένετο σκότος καὶ γνόφος· καὶ διῆλθεν ἡ νύξ· καὶ οὐ συνέμιξαν ἀλλήλοις ὅλην τὴν νύκτα.
\VS{21}Ἐξέτεινε δὲ Μωυσῆς τὴν χεῖρα ἐπὶ τὴν θάλασσαν· καὶ ὑπήγαγε Κύριος τὴν θάλασσαν ἐν ἀνέμῳ νότῳ βιαίῳ ὅλην τὴν νύκτα, καὶ ἐποίησε τὴν θάλασσαν ξηράν· καὶ ἐσχίσθη τὸ ὕδωρ.
\VS{22}Καὶ εἰσῆλθον οἱ υἱοὶ Ἰσραὴλ εἰς μέσον τῆς θαλάσσης κατὰ τὸ ξηρόν· καὶ τὸ ὕδωρ αὐτῆς τεῖχος ἐκ δεξιῶν, καὶ τεῖχος ἐξ εὐωνύμων.
\par }{\PP \VS{23}Καὶ κάτεδίωξαν οἱ Αἰγύπτιοι, καὶ εἰσῆλθον ὀπίσω αὐτῶν καὶ πᾶς ἵππος Φαραὼ, καὶ τὰ ἅρματα, καὶ οἱ ἀναβάται, εἰς μέσον τῆς θαλάσσης.
\VS{24}Ἐγενήθη δὲ ἐν τῇ φυλακῇ τῇ ἑωθινῇ, καὶ ἐπίβλεψε Κύριος ἐπὶ τὴν παρεμβολὴν τῶν Αἰγυπτίων ἐν στύλῳ πυρὸς καὶ νεφέλης, καὶ συνετάραξε τὴν παρεμβολὴν τῶν Αἰγυπτίων,
\VS{25}καὶ συνέδησε τοὺς ἄξονας τῶν ἁρμάτων αὐτῶν, καὶ ἤγαγεν αὐτοὺς μετὰ βίας· καὶ εἶπαν οἱ Αἰγύπτιοι, φυγωμεν ἀπὸ προσώπου Ἰσραήλ· ὁ γὰρ Κύριος πολεμεῖ περὶ αὐτῶν τοὺς Αἰγυπτίους.
\VS{26}Εἶπε δὲ Κύριος πρὸς Μωυσῆν, ἔκτεινον τὴν χεῖρά σου ἐπὶ τὴν θάλασσαν, καὶ ἀποκαταστήτω τὸ ὕδωρ, καὶ ἐπικαλυψάτω τοὺς Αἰγυπτίους, ἐπί τε τὰ ἅρματα καὶ τοὺς ἀναβάτας.
\VS{27}Ἐξέτεινε δὲ Μωυσῆς τὴν χεῖρα ἐπὶ τὴν θάλασσαν, καὶ ἀπεκατέστη τὸ ὕδωρ πρὸς ἡμέραν ἐπὶ χώρας· οἱ δὲ Αἰγύπτιοι ἔφυγον ὑπὸ τὸ ὕδωρ· καὶ ἐξετίναξε Κύριος τοὺς Αἰγυπτίους μέσον τῆς θαλάσοης.
\VS{28}Καὶ ἐπαναστραφὲν τὸ ὕδωρ ἐκάλυψε τὰ ἅρματα καὶ τοὺς ἀναβάτας, καὶ πᾶσαν τὴν δύναμιν Φαραὼ, τοὺς εἰσπεπορευμένους ὀπίσω αὐτῶν εἰς τὴν θάλασσαν· καὶ οὐ κατελείφθη ἐξ αὐτῶν οὐδὲ εἷς.
\VS{29}Οἱ δὲ υἱοὶ Ἰσραὴλ ἐπορεύθησαν διὰ ξηρᾶς ἐν μέσῳ τῆς θάλασσης· τὸ δὲ ὕδωρ αὐτοῖς τεῖχος ἐκ δεξιῶν, καὶ τεῖχος ἐξ εὐωνύμων.
\VS{30}Καὶ ἐῤῥύσατο Κύριος τὸν Ἰσραὴλ ἐν τῇ ἡμέρᾳ ἐκείνῃ ἐκ χειρὸς τῶν Αἰγυπτίων· καὶ εἶδεν Ἰσραὴλ τοὺς Αἰγυπτίους τεθνηκότας παρὰ τὸ χεῖλος τῆς θαλάσσης.
\VS{31}Εἶδε δὲ Ἰσραὴλ τὴν χεῖρα τὴν μεγάλην, ἃ ἐποίησε Κύριος τοῖς Αἰγυπτίοις· ἐφοβήθη δὲ ὁ λαὸς τὸν Κύριον, καὶ ἐπίστευσαν τῷ Θεῷ, καὶ Μωυσῇ τῷ θεράποντι αὐτοῦ.

\par }\Chap{15}{\PP \VerseOne{1}Τότε ᾖσε Μωυσῆς καὶ οἱ υἱοὶ Ἰσραὴλ τὴν ᾠδὴν ταύτην τῷ Θεῷ, καὶ εἶπαν, λέγοντες, ᾄσωμεν τῷ Κυρίῳ, ἐνδόξως γὰρ δεδόξασται· ἵππον καὶ ἀναβάτην ἔῤῥιψεν εἰς θάλασσαν.
\VS{2}Βοηθὸς καὶ σκεπαστὴς ἐγένετό μοι εἰς σωτηρίαν· οὗτός μου Θεὸς, καὶ δοξάσω αὐτόν· Θεὸς τοῦ πατρός μου, καὶ ὑψώσω αὐτόν.
\VS{3}Κύριος συντρίβων πολέμους, Κύριος ὄνομα αὐτῷ.
\VS{4}Ἅρματα Φαραὼ, καὶ τὴν δύναμιν αὐτοῦ, ἔῤῥιψεν εἰς θάλασσαν, ἐπιλέκτους ἀναβάτας τριστάτας· κατεπόθησαν ἐν ἐρυθρᾷ θαλάσσῃ·
\VS{5}Πόντῳ ἐκάλυψεν αὐτούς· κατέδυσαν εἰς βυθὸν ὡσεὶ λίθος.
\VS{6}Ἡ δεξιά σου, Κύριε, δεδόξασται ἐν ἰσχύϊ· ἡ δεξιά σου χεὶρ, Κύριε, ἔθραυσεν ἐχθρούς.
\VS{7}Καὶ τῷ πλήθει τῆς δόξης σου συνέτριψας τοὺς ὑπεναντίους· ἀπέστειλας τὴν ὀργήν σου κατέφαγεν αὐτοὺς ὡς καλάμην.
\VS{8}Καὶ διὰ πνεύματος τοῦ θυμοῦ σου διέστη τὸ ὕδωρ· ἐπάγη ὡσεὶ τεῖχος τὰ ὕδατα· ἐπάγη τὰ κύματα ἐν μέσῳ τῆς θαλάσσης.
\VS{9}Εἶπεν ὁ ἐχθρὸς, διώξας καταλήψομαι, μεριῶ σκῦλα· ἐμπλήσω ψυχήν μου, ἀνελῶ τῇ μαχαίρῃ μου, κυριεύσει ἡ χείρ μου.
\VS{10}Ἀπέστειλας τὸ πνεῦμά σου· ἐκάλυψεν αὐτοὺς θάλασσα· ἔδυσαν ὡσεὶ μόλιβος ἐν ὕδατι σφοδρῷ.
\VS{11}Τίς ὅμοιός σοι ἐν θεοῖς, Κύριε; τίς ὅμοιός σοι; δεδοξασμένος ἐν ἁγίοις, θαυμαστὸς ἐν δόξαις, ποιῶν τέρατα.
\VS{12}Ἐξέτεινας τὴν δεξιάν σου· κατέπιεν αὐτοὺς γῆ.
\VS{13}Ὡδήγησας τῇ δικαιοσύνῃ σου τὸν λαόν σου τοῦτον, ὃν ἐλυτρώσω· παρεκάλεσας τῇ ἰσχύϊ σου εἰς κατάλυμα ἅγιόν σου.
\VS{14}Ἤκουσαν ἔθνη, καὶ ὠργίσθησαν· ὠδῖνες ἔλαβον κατοικοῦντας Φυλιστιείμ.
\VS{15}Τότε ἔσπευσαν ἡγεμόνες Ἐδὼμ, καὶ ἄρχοντες Μωαβιτῶν· ἔλαβεν αὐτοὺς τρόμος· ἐτάκησαν πάντες οἱ κατοικοῦντες Χαναάν.
\VS{16}Ἐπιπέσοι ἐπʼ αὐτοὺς τρόμος καὶ φόβος· μεγέθει βραχίονός σου ἀπολιθωθήτωσαν, ἕως ἂν παρέλθῃ ὁ λαός σου, Κύριε· ἕως ἂν παρέλθῃ ὁ λαός σου οὗτος, ὃν ἐκτήσω.
\VS{17}Εἰσαγαγὼν καταφύτευσον αὐτοὺς εἰς ὄρος κληρονομίας σου, εἰς ἕτοιμον κατοικητήριόν σου, ὃ κατηρτίσω, Κύριε, ἁγίασμα, Κύριε, ὃ ἡτοίμασαν αἱ χεῖρές σου.
\VS{18}Κύριος βασιλεύων τὸν αἰῶνα, καὶ ἐπʼ αἰῶνα, καὶ ἔτι.
\VS{19}Ὅτι εἰσῆλθεν ἵππος Φαραὼ σὺν ἅρμασιν καὶ ἀναβάταις εἰς θάλασσαν, καὶ ἐπήγαγεν ἐπʼ αὐτοὺς Κύριος τὸ ὕδωρ τῆς θαλάσσης· οἱ δὲ υἱοὶ Ἰσραὴλ ἐπορεύθησαν διὰ ξηρᾶς ἐν μέσῳ τῆς θαλάσσης.
\par }{\PP \VS{20}Λαβοῦσα δὲ Μαριὰμ ἡ προφῆτις ἡ ἀδελφὴ Ἀαρὼν τὸ τύμπανον ἐν τῇ χειρὶ αὐτῆς, καὶ ἐξήλθοσαν πᾶσαι αἱ γυναῖκες ὀπίσω αὐτῆς μετὰ τυμπάνων καὶ χορῶν.
\VS{21}Ἐξῆρχε δὲ αὐτῶν Μαριὰμ, λέγουσα, ᾄσωμεν τῷ Κυρίῳ, ἐνδόξως γὰρ δεδόξασται· ἵππον καὶ ἀναβάτην ἔῤῥιψεν εἰς θάλασσαν.
\VS{22}Ἐξῆρε δὲ Μωυσῆς τοὺς υἱοὺς Ἰσραὴλ ἀπὸ θαλάσσης ἐρυθρᾶς, καὶ ἤγαγεν αὐτοὺς εἰς τὴν ἔρημον Σούρ· καὶ ἐπορεύοντο τρεῖς ἡμέρας ἐν τῇ ἐρήμῳ, καὶ οὐχ ηὕρισκον ὕδωρ, ὥστε πιεῖν.
\VS{23}Ἦλθον δὲ εἰς Μεῤῥᾶ, καὶ οὐκ ἠδύναντο πιεῖν ἐκ Μεῤῥᾶς· πικρὸν γὰρ ἦν· διὰ τοῦτο ἐπωνόμασε τὸ ὄνομα τοῦ τόπου ἐκείνου, Πικρία.
\VS{24}Καὶ διεγόγγυζεν ὁ λαὸς ἐπὶ Μωυσῇ, λέγοντες, τί πιόμεθα;
\VS{25}Ἐβόησε δὲ Μωυσῆς πρὸς Κύριον· καὶ ἔδειξεν αὐτῷ Κύριος ξύλον, καὶ ἐνέβαλεν αὐτὸ εἰς τὸ ὕδωρ, καὶ ἐγλυκάνθη τὸ ὕδωρ· ἐκεῖ ἔθετο αὐτῷ δικαιώματα καὶ κρίσεις· καὶ ἐκεῖ αὐτὸν ἐπείρασε,
\VS{26}καὶ εἶπεν, ἐὰν ἀκοῇ ἀκούσῃς τῆς φωνῆς Κυρίου τοῦ Θεοῦ σου, καὶ τὰ ἀρεστὰ ἐναντίον αὐτοῦ ποιήσῃς, καὶ ἐνωτίσῃ ταῖς ἐντολαῖς αὐτοῦ, καὶ φυλάξῃς πάντα τὰ δικαιώματα αὐτοῦ, πᾶσαν νόσον, ἣν ἐπήγαγον τοῖς Αἰγυπτίοις, οὐκ ἐπάξω ἐπὶ σέ· ἐγὼ γάρ εἰμι Κύριος ὁ Θεός σου ὁ ἰώμενός σε.
\VS{27}Καὶ ἤλθοσαν εἰς Αἰλείμ· καὶ ἦσαν ἐκεῖ δώδεκα πηγαὶ ὑδάτων, καὶ ἑβδομήκοντα στελέχη φοινίκων· παρενέβαλον δὲ ἐκεῖ παρὰ τὰ ὕδατα.

\par }\Chap{16}{\PP \VerseOne{1}Ἀπῄραν δὲ ἐξ Αἰλεὶμ, καὶ ἤλθοσαν πᾶσα συναγωγὴ υἱῶν Ἰσραὴλ εἰς τὴν ἔρημον Σὶν, ὅ ἐστιν ἀνὰ μέσον Αἰλεὶμ, καὶ ἀνὰ μέσον Σινά. τῇ δὲ πεντεκαιδεκάτῃ ἡμέρᾳ, τῷ μηνὶ τῷ δευτέρῳ ἐξεληλυθότων αὐτῶν ἐκ γῆς Αἰγύπτου,
\VS{2}διεγόγγυζε πᾶσα συναγωγὴ υἱῶν Ἰσραὴλ ἐπὶ Μωυσὴν καὶ Ἀαρών.
\VS{3}Καὶ εἶπεν πρὸς αὐτοὺς οἱ υἱοὶ Ἰσραήλ, ὄφελον ἀπεθάνομεν πληγέντες ὑπὸ Κυρίου ἐν γῇ Αἰγύπτῳ, ὅταν ἐκαθίσαμεν ἐπὶ τῶν λεβήτων τῶν κρεῶν, καὶ ἠσθίομεν ἄρτους εἰς πλησμονήν· ὅτι ἐξηγάγετε ἡμᾶς εἰς τὴν ἔρημον ταύτην, ἀποκτεῖναι πᾶσαν τὴν συναγωγὴν ταύτην ἐν λιμῷ.
\VS{4}Εἶπε δὲ Κύριος πρὸς Μωυσῆν, ἰδοὺ ἐγὼ ὕω ὑμῖν ἄρτους ἐκ τοῦ οὐρανοῦ· καὶ ἐξελεύσεται ὁ λαὸς, καὶ συλλέξουσι τὸ τῆς ἡμέρας εἰς ἡμέραν, ὅπως πειράσω αὐτοὺς εἰ πορεύσονται τῷ νόμῳ μου, ἢ οὔ·
\VS{5}Καὶ ἔσται ἐν τῇ ἡμέρᾳ τῇ ἕκτῃ, καὶ ἑτοιμάσουσιν ὃ ἐὰν εἰσενέγκωσι· καὶ ἔσται διπλοῦν ὃ ἐὰν συναγάγωσι τὸ καθʼ ἡμέραν εἰς ἡμέραν.
\VS{6}Καὶ εἶπε Μωυσῆς καὶ Ἀαρὼν πρὸς πάσαν συναγωγὴν υἱῶν Ἰσραὴλ, ἑσπέρας γνώσεσθε, ὅτι Κύριος ἐξήγαγεν ὑμᾶς ἐκ γῆς Αἰγύπτου,
\VS{7}καὶ πρωῒ ὄψεσθε τὴν δόξαν Κυρίου ἐν τῷ εἰσακοῦσαι τὸν γογγυσμὸν ὑμῶν ἐπὶ τῷ Θεῷ· ἡμεῖς δὲ τί ἐσμεν, ὅτι διαγογγύζετε καθʼ ἡμῶν;
\VS{8}Καὶ εἶπε Μωυσῆς, ἐν τῷ διδόναι Κύριον ὑμῖν ἑσπέρας κρέα φαγεῖν, καὶ ἄρτους τὸ πρωῒ εἰς πλησμονὴν, διὰ τὸ εἰσακοῦσαι Κύριον τὸν γογγυσμὸν ὑμῶν, ὃν ὑμεῖς διαγογγύζετε καθʼ ἡμῶν· ἡμεῖς δὲ τί ἐσμεν; οὐ γὰρ καθʼ ἡμῶν ἐστιν ὁ γογγυσμὸς ὑμῶν, ἀλλʼ ἢ κατὰ τοῦ Θεοῦ.
\par }{\PP \VS{9}Εἶπε δὲ Μωυσῆς πρὸς Ἀαρὼν, εἶπον πάσῃ συναγωγῇ υἱῶν Ἰσραὴλ, προσέλθετε ἐναντίον τοῦ Θεοῦ· εἰσακήκοε γὰρ τὸν γογγυσμὸν ὑμῶν.
\VS{10}Ἡνίκα δὲ ἐλάλει Ἀαρὼν πάσῃ συναγωγῇ υἱῶν Ἰσραὴλ, καὶ ἐπεστράφησαν εἰς τὴν ἔρημον, καὶ ἡ δόξα Κυρίου ὤφθη ἐν νεφέλῃ.
\VS{11}καὶ ἐλάλησε Κύριος πρὸς Μωυσῆν, λέγων,
\VS{12}εἰσακήκοα τὸν γογγυσμὸν τῶν υἱῶν Ἰσραήλ· λάλησον πρὸς αὐτοὺς, λέγων, τὸ πρὸς ἑσπέραν ἔδεσθε κρέα, καὶ τὸ πρωῒ πλησθήσεσθε ἄρτων· καὶ γνώσεσθε, ὅτι ἐγὼ Κύριος ὁ Θεὸς ὑμῶν.
\VS{13}Ἐγένετο δὲ ἑσπέρα· καὶ ἀνέβη ὀρτυγομήτρα, καὶ ἐκάλυψε τὴν παρεμβολήν· τὸ πρωῒ ἐγένετο καταπαυομένης τῆς δρόσου κύκλῳ τῆς παρεμβολῆς.
\VS{14}Καὶ ἰδοὺ ἐπὶ πρόσωπον τῆς ἐρήμου λεπτὸν ὡσεὶ κόριον λευκὸν, ὡσεὶ πάγος ἐπὶ τῆς γῆς.
\VS{15}Ἰδόντες δὲ αὐτὸ οἱ υἱοὶ Ἰσραὴλ, εἶπαν ἕτερος τῷ ἑτέρῳ, τί ἐστι τοῦτο; οὐ γὰρ ᾔδεισαν τί ἦν· εἶπε δὲ Μωυσῆς αὐτοῖς, οὗτος ὁ ἄρτος, ὃν ἔδωκε Κύριος ὑμῖν φαγεῖν.
\VS{16}Τοῦτο τὸ ῥῆμα ὃ συνέταξε Κύριος· συναγάγετε ἀπʼ αὐτοῦ ἕκαστος εἰς τοὺς καθήκοντας γομὸρ, κατὰ κεφαλὴν κατὰ ἀριθμὸν ψυχῶν ὑμῶν, ἕκαστος σὺν τοῖς συσκηνίοις ὑμῶν συλλέξατε.
\VS{17}Ἐποίησαν δὲ οὕτως οἱ υἱοὶ Ἰσραήλ· καὶ συνέλεξαν ὁ τὸ πολὺ καὶ ὁ τὸ ἔλαττον.
\VS{18}Καὶ μετρήσαντες γομὸρ, οὐκ ἐπλεόνασεν ὁ τὸ πόλυ, καὶ ὁ τὸ ἔλαττον οὐκ ἠλαττόνησεν· ἕκαστος εἰς τοὺς καθήκοντας παρʼ ἑαυτῷ συνέλεξαν.
\VS{19}Εἶπε δὲ Μωυσῆς πρὸς αὐτοὺς, μηδεὶς καταλειπέτω ἀπʼ αὐτοῦ εἰς τὸ πρωΐ.
\par }{\PP \VS{20}Καὶ οὐκ εἰσήκουσαν Μωυσῆ, ἀλλὰ κατέλιπόν τινες ἀπʼ αὐτοῦ εἰς τὸ πρωΐ· καὶ ἐξέζεσε σκώληκας, καὶ ἐπώζεσε· καὶ ἐπικράνθη ἐπʼ αὐτοῖς Μωυσῆς.
\VS{21}Καὶ συνέλεξαν αὐτὸ πρωῒ πρωῒ, ἕκαστος τὸ καθῆκον αὐτῷ· ἡνίκα δὲ διεθέρμαινεν ὁ ἥλιος, ἐτήκετο.
\VS{22}Ἐγένετο δὲ τῇ ἡμέρᾳ τῇ ἕκτῃ, συνέλεξαν τὰ δέοντα διπλᾶ, δύο γομὸρ τῷ ἑνί· εἰσήλθοσαν δὲ πάντες οἱ ἄρχοντες τῆς συναγωγῆς, καὶ ἀνήγγειλαν Μωυσῇ.
\VS{23}Εἶπε δὲ Μωυσῆς πρὸς αὐτούς, οὐ τοῦτο τὸ ῥῆμά ἐστιν ὃ ἐλάλησε Κύριος; σάββατα ἀνάπαυσις ἁγία τῷ Κυρίῳ αὔριον· ὅσα ἐὰν πέσσητε, πέσσετε· καὶ ὅσα ἐὰν ἕψητε, ἕψετε· καὶ πᾶν τὸ πλεονάζον καταλείπετε αὐτὸ εἰς ἀποθήκην εἰς τὸ πρωΐ.
\VS{24}Καὶ κατελίποσαν ἀπʼ αὐτοῦ εἰς ἕως πρωῒ, καθὼς συνέταξεν αὐτοῖς Μωυσῆς· καὶ οὐκ ἐπώζεσεν, οὐδὲ σκώληξ ἐγένετο ἐν αὐτῷ.
\VS{25}Εἶπε δὲ Μωυσῆς, φάγετε σήμερον· ἔστι γὰρ σάββατα αήμερον τῷ Κυρίῳ· οὐχ εὑρεθήσεται ἐν τῷ πεδίῳ.
\VS{26}Ἓξ ἡμέρας συλλέξετε· τῇ δὲ ἡμέρᾳ τῇ ἑβδόμῃ σάββατα, ὅτι οὐκ ἔσται ἐν αὐτῇ.
\VS{27}Ἐγένετο δὲ ἐν τῇ ἡμέρᾳ τῇ ἑβδόμῃ ἐξήθλοσάν τινες ἐκ τοῦ λαοῦ συλλέξαι, καὶ οὐχ εὗρον.
\VS{28}Εἶπε δὲ Κύριος πρὸς Μωυσῆν, ἕως τίνος οὐ βούλεσθε εἰσακούειν τὰς ἐντολάς μου, καὶ τὸν νόμον μου;
\VS{29}Ἴδετε, ὁ γὰρ Κύριος ἔδωκεν ὑμῖν σάββατα τὴν ἡμέραν ταύτην· διὰ τοῦτο αὐτὸς ἔδωκεν ὑμῖν τῇ ἡμέρᾳ τῇ ἕκτῃ ἄρτους δύο ἡμερῶν· καθίσεσθε ἕκαστος εἰς τοὺς οἴκους ὑμῶν· μηδεὶς ἐκπορευέσθω ἐκ τοῦ τόπου αὐτοῦ τῇ ἡμέρᾳ τῇ ἑβδόμῃ.
\VS{30}Καὶ ἐσαββάτισεν ὁ λαὸς τῇ ἡμέρᾳ τῇ ἑβδόμῃ.
\VS{31}Καὶ ἐπωνόμασαν αὐτὸ οἱ υἱοὶ Ἰσραὴλ τὸ ὄνομα αὐτοῦ, Μάν· ἦν δὲ ὡσεὶ σπέρμα κορίου λευκόν· τὸ δὲ γεῦμα αὐτοῦ ὡς ἐγκρὶς ἐν μέλιτι.
\VS{32}Εἶπε δὲ Μωυσῆς, τοῦτο τὸ ῥῆμα, ὃ συνέταξε Κύριος, πλήσατε τὸ γομὸρ τοῦ μὰν, εἰς ἀποθήκην εἰς τὰς γενεὰς ὑμῶν· ἵνα ἴδωσι τὸν ἄρτον, ὃν ἐφάγετε ὑμεῖς ἐν τῇ ἐρήμῳ, ὡς ἐξήγαγεν ὑμᾶς Κύριος ἐκ γῆς Αἰγύπτου.
\VS{33}Καὶ εἶπε Μωυσῆς πρὸς Ἀαρὼν, λάβε στάμνον χρυσοῦν ἕνα, καὶ ἔμβαλε εἰς αὐτὸν πλῆρες τὸ γομὸρ τοῦ μὰν, καὶ ἀποθήσεις αὐτὸ ἐναντίον τοῦ Θεοῦ, εἰς διατήρησιν εἰς τὰς γενεὰς ὑμῶν,
\VS{34}ὃν τρόπον συνέταξε Κύριος τῷ Μωυσῇ· καὶ ἀπέθηκεν Ἀαρὼν ἐναντίον τοῦ μαρτυρίου εἰς διατήρησιν.
\VS{35}Οἱ δὲ υἱοὶ Ἰσραὴλ ἔφαγον τὸ μὰν ἔτη τεσσαράκοντα, ἕως ἦλθον εἰς τὴν οἰκουμένην ἐφάγοσαν τὸ μὰν, ἕως παρεγένοντο εἰς μέρος τῆς Φοινίκης.
\VS{36}Τὸ δὲ γομὸρ τὸ δέκατον τῶν τριῶν μέτρων ἦν.

\par }\Chap{17}{\PP \VerseOne{1}Καὶ ἀπῇρε πᾶσα συναγωγὴ υἱῶν Ἰσραὴλ ἐκ τῆς ἐρήμου Σὶν κατὰ παρεμβολὰς αὐτῶν, διὰ ῥήματος Κυρίου· καὶ παρενεβάλοσαν ἐν Ῥαφιδείν· οὐκ ἦν δὲ ὕδωρ τῷ λαῷ πιεῖν.
\VS{2}Καὶ ἐλοιδορεῖτο ὁ λαὸς πρὸς Μωυσῆν, λέγοντες, δὸς ἡμῖν ὕδωρ, ἵνα πίωμεν· καὶ εἶπεν αὐτοῖς Μωυσῆς, τί λοιδορεῖσθέ μοι, καὶ τί πειράζετε Κύριον;
\VS{3}Ἐδίψησε δὲ ἐκεῖ ὁ λαὸς ὕδατι· καὶ διεγόγγυσεν ἐκεῖ ὁ λαὸς πρὸς Μωυσῆν, λέγοντες, ἱνατί τοῦτο; ἀνεβίβασας ἡμᾶς ἐξ Αἰγύπτου ἀποκτεῖναι ἡμᾶς καὶ τὰ τέκνα ἡμῶν καὶ τὰ κτήνη τῷ δίψει;
\VS{4}Ἐβόησε δὲ Μωυσῆς πρὸς Κύριον, λέγων, τί ποιήσω τῷ λαῷ τούτῳ; ἔτι μικρὸν, καὶ καταλιθοβολὴσουσί με.
\VS{5}Καὶ εἶπε Κύριος πρὸς Μωυσῆν, προπορεύου τοῦ λαοῦ τούτου· λάβε δὲ σεαυτῷ ἀπὸ τῶν πρεσβυτέρων τοῦ λαοῦ· καὶ τὴν ῥάβδον, ἐν ᾗ ἐπάταξας τὸν ποταμὸν, λάβε ἐν τῇ χειρί σου, καὶ πορεύσῃ.
\VS{6}Ὅδε ἐγὼ ἕστηκα ἐκεῖ πρὸ τοῦ σὲ ἐπὶ τῆς πέτρας ἐν Χωρήβ· καὶ πατάξεις τὴν πέτραν, καὶ ἐξελεύσεται ἐξ αὐτῆς ὕδωρ, καὶ πίεται ὁ λαός. Ἐποίησε δὲ Μωυσῆς οὕτως ἐναντίον τῶν υἱῶν Ἰσραήλ.
\VS{7}Καὶ ἐπωνόμασε τὸ ὄνομα τοῦ τόπου ἐκείνου, Πειρασμὸς, καὶ Λοιδόρησις, διὰ τὴν λοιδορίαν τῶν υἱῶν Ἰσραὴλ, καὶ διὰ τὸ πειράζειν Κύριον, λέγοντας, εἰ ἔστι Κύριος ἐν ἡμῖν, ἢ οὔ;
\par }{\PP \VS{8}Ἦλθε δὲ Ἀμαλὴκ καὶ ἐπολέμει Ἰσραὴλ ἐν Ῥαφιδείν.
\VS{9}Εἶπε δὲ Μωυσῆς τῷ Ἰησοῖ, Ἐπίλεξον σεαυτῷ ἄνδρας δυνατοὺς, καὶ ἐξελθὼν παράταξαι τῷ Ἀμαλὴκ αὔριον· καὶ ἰδοὺ ἐγὼ ἕστηκα ἐπὶ τῆς κορυφῆς τοῦ βουνοῦ, καὶ ἡ ῥάβδος τοῦ Θεοῦ ἐν τῇ χειρί μου.
\VS{10}Καὶ ἐποίησεν Ἰησοῦς καθάπερ εἶπεν αὐτῷ Μωυσῆς, καὶ ἐξελθὼν παρετάξατο τῷ Ἀμαλήκ· καὶ Μωυσῆς καὶ Ἀαρὼν καὶ Ὢρ ἀνέβησαν ἐπὶ τὴν κορυφὴν τοῦ βουνοῦ.
\VS{11}Καὶ ἐγένετο ὅταν ἐπῇρε Μωυσῆς τὰς χεῖρας, κατίσχυεν Ἰσραήλ· ὅταν δὲ καθῆκε τὰς χεῖρας, κατίσχυεν Ἀμαλήκ.
\VS{12}Αἱ δὲ χεῖρες Μωυσῆ βαρεῖαι· καὶ λαβόντες λίθον ὑπέθηκαν ὑπʼ αὐτὸν, καὶ ἐκάθητο ἐπʼ αὐτοῦ· καὶ Ἀαρὼν καὶ Ὢρ ἐστήριζον τὰς χεῖρας αὐτοῦ ἐντεῦθεν εἷς, καὶ ἐντεῦθεν εἷς· καὶ ἐγένοντο αἱ χεῖρες Μωυσῆ ἐστηριγμέναι ἕως δυσμῶν ἡλίου.
\VS{13}Καὶ ἐτρέψατο Ἰησοῦς τὸν Ἀμαλὴκ, καὶ πάντα τὸν λαὸν αὐτοῦ ἐν φόνῳ μαχαίρας.
\VS{14}Εἶπε δὲ Κύριος πρὸς Μωυσῆν, Κατάγραψον τοῦτο εἰς μνημόσυνον εἰς βιβλίον, καὶ δὸς εἰς τὰ ὦτα Ἰησοῖ· ὅτι ἀλοιφῇ ἐξαλείψω τὸ μνημόσυνον Ἀμαλὴκ ἐκ τῆς ὑπὸ τὸν οὐρανόν.
\VS{15}Καὶ ᾠκοδόμησε Μωυσῆς θυσιαστήριον Κυρίῳ· καὶ ἐπωνόμασε τὸ ὄνομα αὐτοῦ, Κύριος καταφυγή μου.
\VS{16}Ὅτι ἐν χειρὶ κρυφαίᾳ πολεμεῖ Κύριος ἐπὶ Ἀμαλὴκ ἀπὸ γενεῶν εἰς γενεάς.

\par }\Chap{18}{\PP \VerseOne{1}Ἤκουσε δὲ Ἰοθὸρ ἱερεὺς Μαδιὰμ ὁ γαμβρὸς Μωυσῆ πάντα ὅσα ἐποίησε Κύριος Ἰσραὴλ τῷ ἑαυτοῦ λαῷ· ἐξήγαγε γὰρ Κύριος τὸν Ἰσραὴλ ἐξ Αἰγύπτου.
\VS{2}Ἔλαβε δὲ Ἰοθὸρ ὁ γαμβρὸς Μωυσῆ Σεπφώραν τὴν γυναῖκα Μωυσῆ μετὰ τὴν ἄφεσιν αὐτῆς,
\VS{3}καὶ τοὺς δύο υἱοὺς αὐτῆς· ὄνομα τῷ ἑνὶ αὐτῶν Γηρσάμ, λέγων, πάροικος ἤμην ἐν γῇ ἀλλοτρίᾳ·
\VS{4}καὶ τὸ ὄνομα τοῦ δευτέρου Ἐλίεζερ, λέγων, ὁ γὰρ Θεὸς τοῦ πατρός μου βοηθός μου, καὶ ἐξείλατό με ἐκ χειρὸς Φαραώ.
\VS{5}Καὶ ἐξῆλθεν Ἰοθὸρ ὁ γαμβρὸς Μωυσῆ καὶ οἱ υἱοὶ καὶ ἡ γυνὴ πρὸς Μωυσῆν εἰς τὴν ἔρημον, οὗ παρενέβαλεν ἐπʼ ὄρους τοῦ Θεοῦ.
\VS{6}Ἀνηγγέλη δὲ Μωυσῇ, λέγοντες, ἰδοὺ ὁ γαμβρός σου Ἰοθὸρ παραγίνεται πρὸς σέ, καὶ ἡ γυνὴ καὶ οἱ δύο υἱοί σου μετʼ αὐτοῦ.
\VS{7}Ἐξῆλθε δὲ Μωυσῆς εἰς συνάντησιν τῷ γαμβρῷ, καὶ προσεκύνησεν αὐτῷ, καὶ ἐφίλησεν αυτὸν, καὶ ἠσπάσαντο ἀλλήλους, καὶ εἰσήγαγεν αὐτοὺς εἰς τὴν σκηνήν.
\VS{8}Καὶ διηγήσατο Μωυσῆς τῷ γαμβρῷ πάντα ὅσα ἐποίησε Κύριος τῷ Φαραῷ καὶ πᾶσι τοῖς Αἰγυπτίοις ἕνεκεν τοῦ Ἰσραήλ, καὶ πάντα τὸν μόχθον τὸν γενόμενον αὐτοῖς ἐν τῇ ὁδῷ, καὶ ὅτι ἐξείλατο αὐτοὺς Κύριος ἐκ χειρὸς Φαραὼ, καὶ ἐκ χειρὸς τῶν Αἰγυπτίων.
\VS{9}Ἐξέστη δὲ Ἰοθὸρ ἐπὶ πᾶσι τοῖς ἀγαθοῖς οἷς ἐποίησεν αὐτοῖς Κύριος, ὅτι ἐξείλατο αὐτοὺς ἐκ χειρὸς Αἰγυπτίων καὶ ἐκ χειρὸς Φαραώ.
\VS{10}Καὶ εἶπεν Ἰοθὸρ, εὐλογητὸς Κύριος, ὅτι ἐξείλατο αὐτοὺς ἐκ χειρὸς Αἰγυπτίων καὶ ἐκ χειρὸς Φαραώ.
\VS{11}Νῦν ἔγνων ὅτι μέγας Κύριος παρὰ πάντας τοὺς θεούς ἕνεκεν τούτου, ὅτι ἐπέθεντο αὐτοῖς.
\VS{12}Καὶ ἔλαβεν Ἰοθὸρ ὁ γαμβρὸς Μωυσῆ ὁλοκαυτώματα καὶ θυσίας τῷ Θεῷ· παρεγένετο δὲ Ἀαρὼν καὶ πάντες οἱ πρεσβύτεροι Ἰσραὴλ συμφαγεῖν ἄρτον μετὰ τοῦ γαμβροῦ Μωυσῆ, ἐναντίον τοῦ Θεοῦ.
\par }{\PP \VS{13}Καὶ ἐγένετο μετὰ τὴν ἐπαύριον συνεκάθισε Μωυσῆς κρίνειν τὸν λαόν· παρειστήκει δὲ πᾶς ὁ λαὸς Μωυσῇ ἀπὸ πρωΐθεν ἕως δείλης.
\VS{14}Καὶ ἰδὼν Ἰοθὸρ πάντα ὅσα ποιεῖ τῷ λαῷ, λέγει, τί τοῦτο ὃ σὺ ποιεῖς τῷ λαῷ; διατί σὺ κάθησαι μόνος, πᾶς δὲ ὁ λαὸς παρέστηκέ σοι ἀπὸ πρωΐθεν ἕως δείλης;
\VS{15}Καὶ λέγει Μωυσῆς τῷ γαμβρῶ, Ὅτι παραγίνεται πρός με ὁ λαὸς ἐκζητῆσαι κρίσιν παρὰ τοῦ Θεοῦ.
\VS{16}Ὅταν γὰρ γένηται αὐτοῖς ἀντιλογία, καὶ ἔλθωσι πρός με, διακρίνω ἕκαστον, καὶ συμβιβάζω αὐτοὺς τὰ προστάγματα τοῦ Θεοῦ καὶ τὸν νόμον αὐτοῦ.
\VS{17}Εἶπε δὲ ὁ γαμβρὸς Μωυσῆ πρὸς αὐτὸν, οὐκ ὀρθῶς σὺ ποιεῖς τὸ ῥῆμα τοῦτο.
\VS{18}Φθορᾷ καταφθαρήσῃ ἀνυπομονήτῳ καὶ σὺ, καὶ πᾶς ὁ λαὸς οὗτος, ὅς ἐστι μετὰ σοῦ· βαρύ σοι τὸ ῥῆμα τοῦτο· οὐ δυνήσῃ ποιεῖν σὺ μόνος.
\VS{19}Νῦν οὖν ἄκουσόν μου, καὶ συμβουλεύσω σοι, καὶ ἔσται ὁ Θεὸς μετὰ σοῦ· γίνου σὺ τῷ λαῷ τὰ πρὸς τὸν Θεὸν, καὶ ἀνοίσεις τοὺς λόγους αὐτῶν πρὸς τὸν Θεόν.
\VS{20}Καὶ διαμαρτύρῇ αὐτοῖς τὰ προστάγματα τοῦ Θεοῦ καὶ τὸν νόμον αὐτοῦ, καὶ σημανεῖς αὐτοῖς τὰς ὁδοὺς ἐν αἷς πορεύσονται ἐν αὐταῖς, καὶ τὰ ἔργα ἃ ποιήσουσι.
\VS{21}Καὶ σὺ σεαυτῷ σκέψαι ἀπὸ παντὸς τοῦ λαοῦ ἄνδρας δυνατοὺς, θεοσεβεῖς, ἄνδρας δικαίους, μισοῦντας ὑπερηφανίαν, καὶ καταστήσεις ἐπʼ αὐτῶν χιλιάρχους καὶ ἑκατοντάρχους καὶ πεντηκοντάρχους καὶ δεκαδάρχους.
\VS{22}Καὶ κρινοῦσι τὸν λαὸν πᾶσαν ὥραν· τὸ δὲ ῥῆμα τὸ ὑπέρογκον ἀνοίσουσιν ἐπὶ σὲ· τὰ δὲ βραχέα τῶν κριμάτων κρινοῦσιν αὐτοί· καὶ κουφιοῦσιν ἀπὸ σοῦ, καὶ συναντιλήψονταί σοι.
\VS{23}Ἐὰν τὸ ῥῆμα τοῦτο ποιήσῃς, κατισχύσει σε ὁ Θεὸς, καὶ δυνήσῃ παραστῆναι, καὶ πᾶς ὁ λαὸς οὗτος εἰς τὸν ἑαυτοῦ τόπον μετʼ εἰρήνης ἥξει.
\VS{24}Ἤκουσε δὲ Μωυσῆς τῆς φωνῆς τοῦ γαμβροῦ, καὶ ἐποίησεν ὅσα εἶπεν αὐτῷ.
\VS{25}Καὶ ἐπέλεξε Μωυσῆς ἄνδρας δυνατοὺς ἀπὸ παντὸς Ἰσραὴλ, καὶ ἐποίησεν αὐτοὺς ἐπʼ αὐτῶν χιλιάρχους καὶ ἑκατοντάρχους καὶ πεντηκοντάρχους καὶ δεκαδάρχους.
\VS{26}Καὶ ἐκρίνοσαν τὸν λαὸν πᾶσαν ὥραν· πᾶν δὲ ῥῆμα ὑπέρογκον ἀνεφέροσαν ἐπὶ Μωυσῆν· πᾶν δὲ ῥῆμα ἐλαφρὸν ἐκρίνοσαν αὐτοί.
\VS{27}Ἐξαπέστειλε δὲ Μωυσῆς τὸν ἑαυτοῦ γαμβρὸν, καὶ ἀπῆλθεν εἰς τὴν γῆν αὐτοῦ.

\par }\Chap{19}{\PP \VerseOne{1}Τοῦ δὲ μηνὸς τοῦ τρίτου τῆς ἐξόδου τῶν υἱῶν Ἰσραὴλ ἐκ γῆς Αἰγύπτου τῇ ἡμέρᾳ ταύτῃ, ἤλθοσαν εἰς τὴν ἔρημον τοῦ Σινά.
\VS{2}Καὶ ἀπῆραν ἐκ Ῥαφιδεὶν, καὶ ἤλθοσαν εἰς τὴν ἔρημον τοῦ Σινὰ, καὶ παρενέβαλεν ἐκεῖ Ἰσραὴλ κατέναντι τοῦ ὄρους.
\VS{3}Καὶ Μωυσῆς ἀνέβη εἰς τὸ ὄρος τοῦ Θεοῦ· καὶ ἐκάλεσεν αὐτὸν ὁ Θεὸς ἐκ τοῦ ὄρους, λέγων, τάδε ἐρεῖς τῷ οἴκῳ Ἰακὼβ, καὶ ἀναγγελεῖς τοῖς υἱοῖς Ἰσραήλ.
\VS{4}Αὐτοὶ ἑωράκατε ὅσα πεποίηκα τοῖς Αἱγυπτίοις, καὶ ἀνέλαβον ὑμᾶς ὡσεὶ ἐπὶ πτερύγων ἀετῶν, καὶ προσηγαγόμην ὑμᾶς πρὸς ἐμαυτόν.
\VS{5}Καὶ νῦν ἐὰν ἀκοῇ ἀκούσητε τῆς ἐμῆς φωνῆς, καὶ φυλάξητε τὴν διαθήκην μου, ἔσεσθέ μοι λαὸς περιούσιος ἀπὸ πάντων τῶν ἐθνῶν· ἐμὴ γάρ ἐστι πᾶσα ἡ γῆ.
\VS{6}Ὑμεῖς δὲ ἔσεσθέ μοι βασίλειον ἱεράτευμα καὶ ἔθνος ἅγιον· ταῦτα τὰ ῥήματα ἐρεῖς τοῖς υἱοῖς Ἰσραήλ.
\VS{7}Ἦλθε δὲ Μωυσῆς, καὶ ἐκάλεσεν τοὺς πρεσβυτέρους τοῦ λαοῦ· καὶ παρέθηκεν αὐτοῖς πάντας τοὺς λόγους τούτους, οὓς συνέταξεν αὐτοῖς ὁ Θεός.
\VS{8}Ἀπεκρίθη δὲ πᾶς ὁ λαὸς ὁμοθυμαδὸν, καὶ εἶπαν, πάντα ὅσα εἶπεν ὁ Θεὸς, ποιήσομεν καὶ ἀκουσόμεθα· ἀνήνεγκε δὲ Μωυσῆς τοὺς λόγους τούτους πρὸς τὸν Θεόν.
\VS{9}Εἶπε δὲ Κύριος πρὸς Μωυσῆν, ἰδοὺ ἐγὼ παραγίνομαι πρὸς σὲ ἐν στύλῳ νεφέλης, ἵνα ἀκούσῃ ὁ λαὸς λαλοῦντός μου πρὸς σὲ, καὶ σοὶ πιστεύσωσιν εἰς τὸν αἰῶνα· ἀνήγγειλε δὲ Μωυσῆς τὰ ῥήματα τοῦ λαοῦ πρὸς Κύριον.
\VS{10}Εἶπε δὲ Κύριος πρὸς Μωυσῆν, Καταβὰς διαμάρτυραι τῷ λαῷ, καὶ ἅγνισον αὐτοὺς σήμερον καὶ αὔριον, καὶ πλυνάτωσαν τὰ ἱμάτια,
\VS{11}καὶ ἔστωσαν ἕτοιμοι εἰς τὴν ἡμέραν τὴν τρίτην· τῇ γὰρ ἡμέρᾳ τῇ τρίτῃ καταβήσεται Κύριος ἐπὶ τὸ ὄρος τὸ Σινὰ, ἐναντίον παντὸς τοῦ λαοῦ.
\VS{12}Καὶ ἀφοριεῖς τὸν λαὸν κύκλῳ, λέγων, προσέχετε ἑαυτοῖς τοῦ ἀναβῆναι εἰς τὸ ὄρος, καὶ θίγειν τι αὐτοῦ· πᾶς ὁ ἁψάμενος τοῦ ὄρους, θανάτῳ τελευτήσει.
\VS{13}Οὐχ ἅψεται αὐτοῦ χείρ· ἐν γὰρ λίθοις λιθοβοληθήσεται, ἢ βολίδι κατατοξευθήσεται· ἐάν τε κτῆνος ἐάν τε ἄνθρωπος, οὐ ζήσεται· ὅταν αἱ φωναὶ καὶ αἱ σάλπιγγες καὶ ἡ νεφέλη ἀπέλθῃ ἀπὸ τοῦ ὄρους, ἐκεῖνοι ἀναβήσονται ἐπὶ τὸ ὄρος.
\par }{\PP \VS{14}Κατέβη δὲ Μωυσῆς ἐκ τοῦ ὄρους πρὸς τὸν λαὸν, καὶ ἡγίασεν αὐτούς· καὶ ἔπλυναν τὰ ἱμάτια.
\VS{15}Καὶ εἶπε τῷ λαῷ, γίνεσθε ἕτοιμοι, τρεῖς ἡμέρας μὴ προσέλθητε γυναικί.
\VS{16}Ἐγένετο δὲ τῇ ἡμέρᾳ τῇ τρίτῃ γενηθέντος πρὸς ὄρθρον, καὶ ἐγένοντο φωναὶ καὶ ἀστραπαὶ καὶ νεφέλη γνοφώδης ἐπʼ ὄρους Σινά· φωνὴ τῆς σάλπιγγος ἤχει μέγα· καὶ ἐπτοήθη πᾶς ὁ λαὸς ὁ ἐν τῇ παρεμβολῇ.
\VS{17}Καὶ ἐξήγαγε Μωυσῆς τὸν λαὸν εἰς συνάντησιν τοῦ Θεοῦ ἐκ τῆς παρεμβολῆς· καὶ παρέστησαν ὑπὸ τὸ ὄρος.
\VS{18}Τὸ ὄρος τὸ Σινὰ ἐκαπνίζετο ὅλον, διὰ τὸ καταβεβηκέναι ἐπʼ αὐτὸ τὸν Θεὸν ἐν πυρί· καὶ ἀνέβαινεν ὁ καπνὸς, ὡσεὶ καπνὸς καμίνου· καὶ ἐξέστη πᾶς ὁ λαὸς σφόδρα.
\VS{19}Ἐγίνοντο δὲ αἱ φωναὶ τῆς σάλπιγγος προβαίνουσαι ἰσχυρότεραι σφόδρα. Μωυσῆς ἐλάλησεν, ὁ δὲ Θεὸς ἀπεκρίνατο αὐτῷ φωνῇ.
\VS{20}Κατέβη δὲ Κύριος ἐπὶ τὸ ὄρος τὸ Σινὰ ἐπὶ τὴν κορυφὴν τοῦ ὄρους· καὶ ἐκάλεσε Κύριος Μωυσῆν ἐπὶ τὴν κορυφὴν τοῦ ὄρους· καὶ ἀνέβη Μωυσῆς.
\VS{21}Καὶ εἶπεν ὁ Θεὸς πρὸς Μωυσῆν, λέγων, καταβὰς διαμάρτυραι τῷ λαῷ, μή ποτε ἐγγίσωσι πρὸς τὸν Θεὸν κατανοῆσαι, καὶ πέσωσιν ἐξ αὐτῶν πλῆθος·
\VS{22}Καὶ οἱ ἱερεῖς οἱ ἐγγίζοντες Κυρίῳ τῷ Θεῷ ἁγιασθήτωσαν, μήποτε ἀπαλλάξῃ ἀπʼ αὐτῶν Κύριος.
\par }{\PP \VS{23}Καὶ εἶπε Μωυσῆς πρὸς τὸν Θεὸν, οὐ δυνήσεται ὁ λαὸς προσαναβῆναι πρὸς τὸ ὄρος τὸ Σινά· σὺ γὰρ διαμεμαρτύρησαι ἡμῖν, λέγων, ἀφόρισαι τὸ ὄρος, καὶ ἁγίασαι αὐτό.
\VS{24}Εἴπε δὲ αὐτῷ Κύριος, βάδιζε, κατάβηθι, καὶ ἀνάβηθι σὺ καὶ Ἀαρὼν μετὰ σοῦ· οἱ δὲ ἱερεῖς καὶ ὁ λαὸς μὴ βιαζέσθωσαν ἀναβῆναι πρὸς τὸν Θεὸν, μὴ ποτε ἀπολέσῃ ἀπʼ αὐτῶν Κύριος.
\VS{25}Κατέβη δὲ Μωυσῆς πρὸς τὸν λαὸν, καὶ εἶπεν αὐτοῖς.

\par }\Chap{20}{\PP \VerseOne{1}Καὶ ἐλάλησε Κύριος πάντας τοὺς λόγους τούτους, λέγων,
\VS{2}ἐγώ εἰμι Κύριος ὁ Θεός σου, ὅστις ἐξήγαγόν σε ἐκ γῆς Αἰγύπτου, ἐξ οἴκου δουλείας.
\VS{3}Οὐκ ἔσονταί σοι θεοὶ ἕτεροι πλὴν ἐμοῦ.
\VS{4}Οὐ ποιήσεις σεαυτῷ εἴδωλον, οὐδὲ παντὸς ὁμοίωμα, ὅσα ἐν τῷ οὐρανῷ ἄνω, καὶ ὅσα ἐν τῇ γῇ κάτω, καὶ ὅσα ἐν τοῖς ὕδασιν ὑποκάτω τῆς γῆς.
\VS{5}Οὐ προσκυνήσεις αὐτοῖς, οὐδὲ μὴ λατρεύσεις αὐτοῖς· ἐγὼ γάρ εἰμι Κύριος ὁ Θεός σου, Θεὸς ζηλωτὴς, ἀποδιδοὺς ἁμαρτίας πατέρων ἐπὶ τέκνα, ἕως τρίτης καὶ τετάρτης γενεᾶς τοῖς μισοῦσί με,
\VS{6}καὶ ποιῶν ἔλεος εἰς χιλιάδας τοῖς ἀγαπῶσί με, καὶ τοῖς φυλάσσουσι τὰ προστάγματά μου.
\VS{7}Οὐ λήψῃ τὸ ὄνομα Κυρίου τοῦ Θεοῦ σου ἐπὶ ματαίῳ· οὐ γὰρ μὴ καθαρίσῃ Κύριος ὁ Θεός σου τὸν λαμβάνοντα τὸ ὄνομα αὐτοῦ ἐπὶ ματαίῳ.
\VS{8}Μνήσθητι τὴν ἡμέραν τῶν σαββάτων ἁγιάζειν αὐτήν.
\VS{9}Ἓξ ἡμέρας ἐργᾷ, καὶ ποιήσεις πάντα τὰ ἔργα σου.
\VS{10}Τῇ δὲ ἡμέρᾳ τῇ ἑβδόμῃ, σάββατα Κυρίῳ τῷ Θεῷ σου· οὐ ποιήσεις ἐν αὐτῇ πᾶν ἔργον σὺ, καὶ ὁ υἱός σου, καὶ ἡ θυγάτηρ σου, ὁ παῖς σου, καὶ ἡ παιδίσκη σου, ὁ βοῦς σου, καὶ τὸ ὑποζύγιόν σου, καὶ πᾶν κτῆνός σου, καὶ ὁ προσήλυτος ὁ παροικῶν ἐν σοί.
\VS{11}Ἐν γὰρ ἓξ ἡμέραις ἐποίησε Κύριος τὸν οὐρανὸν καὶ τὴν γῆν καὶ τὴν θὰλασσαν καὶ πάντα τὰ ἐν αὐτοῖς, καὶ κατέπαυσε τῇ ἡμέρᾳ τῇ ἑβδόμῃ· διὰ τοῦτο εὐλόγησε Κύριος τὴν ἡμέραν τὴν ἑβδόμην, καὶ ἡγίασεν αὐτήν.
\VS{12}Τίμα τὸν πατέρα σου, καὶ τὴν μητέρα σου, ἵνα εὖ σοι γένηται, καὶ ἵνα μακροχρόνιος γένῃ ἐπὶ τῆς γῆς τῆς ἀγαθῆς, ἧς Κύριος ὁ Θεός σου δίδωσί σοι.
\VS{13}Οὐ μοιχεύσεις.
\VS{14}Οὐ κλέψεις.
\VS{15}Οὐ φονεύσεις.
\VS{16}Οὐ ψευδομαρτυρήσεις κατὰ τοῦ πλησίον σου μαρτυρίαν ψευδῆ.
\VS{17}Οὐκ ἐπιθυμήσεις τὴν γυναῖκα τοῦ πλησίον σου· οὐκ ἐπιθυμήσεις τὴν οἰκίαν τοῦ πλησίον σου, οὔτε τὸν ἀγρὸν αὐτοῦ, οὔτε τὸν παῖδα αὐτοῦ, οὔτε τὴν παιδίσκην αὐτοῦ, οὔτε τοῦ βοὸς αὐτοῦ, οὔτε τοῦ ὑποζυγίου αὐτοῦ, οὔτε παντὸς κτήνους αὐτοῦ, οὔτε ὅσα τῷ πλησίον σου ἐστί.
\par }{\PP \VS{18}Καὶ πᾶς ὁ λαὸς ἑώρα τὴν φωνὴν, καὶ τὰς λαμπάδας, καὶ τὴν φωνὴν τῆς σάλπιγγος, καὶ τὸ ὄρος τὸ καπνίζον· φοβηθέντες δὲ πᾶς ὁ λαὸς ἔστησαν μακρόθεν.
\VS{19}Καὶ εἶπαν πρὸς Μωυσῆν, λάλησον σὺ ἡμῖν, καὶ μὴ λαλείτω πρὸς ἡμᾶς ὁ Θεὸς, μὴ ἀποθάνωμεν.
\VS{20}Καὶ λέγει αὐτοῖς Μωυσῆς, θαρσεῖτε· ἕνεκεν γὰρ τοῦ πειράσαι ὑμᾶς παρεγενήθη ὁ Θεὸς πρὸς ὑμᾶς, ὅπως ἂν γένηται ὁ φόβος αὐτοῦ ἐν ὑμῖν, ἵνα μὴ ἁμαρτάνητε.
\VS{21}Εἱστήκει δὲ ὁ λαὸς μακρόθεν, Μωυσῆς δὲ εἰσῆλθεν εἰς τὸν γνόφον, οὗ ἦν ὁ Θεός.
\VS{22}Εἶπε δὲ Κύριος πρὸς Μωυσῆν, τάδε ἐρεῖς τῷ οἴκῳ Ἰακὼβ, καὶ ἀναγγελεῖς τοῖς υἱοῖς Ἰσραήλ· ὑμεῖς ἑωράκατε, ὅτι ἐκ τοῦ οὐρανοῦ λελάληκα πρὸς ὑμᾶς.
\VS{23}Οὐ ποιήσετε ὑμῖν αὐτοῖς θεοὺς ἀργυροῦς, καὶ θεοὺς χρυσοῦς οὐ ποιήσετε ὑμῖν αὑτοῖς.
\VS{24}Θυσιαστήριον ἐκ γῆς ποιήσετέ μοι, καὶ θύσετε ἐπʼ αὐτοῦ τὰ ὁλοκαυτώματα ὑμῶν, καὶ τὰ σωτήρια ὑμῶν, καὶ τὰ πρόβατα, καὶ τοὺς μόσχους ὑμῶν ἐν παντὶ τόπῳ, οὗ ἐὰν ἐπονομάσω τὸ ὄνομά μου ἐκεῖ, καὶ ἥξω πρὸς σὲ, καὶ εὐλογήσω σε.
\VS{25}Ἐὰν δὲ θυσιαστήριον ἐκ λίθων ποιῇς μοι, οὐκ οἰκοδομήσεις αὐτοὺς τμητούς· τὸ γὰρ ἐγχειρίδιόν σου ἐπιβέβληκας ἐπʼ αὐτοὺς, καὶ μεμίανται.
\VS{26}Οὐκ ἀναβήσῃ ἐν ἀναβαθμίσιν ἐπὶ τὸ θυσιαστήριόν μου, ὅπως ἂν μὴ ἀποκαλύψῃς τὴν ἀσχημοσύνην σου ἐπʼ αὐτοῦ.

\par }\Chap{21}{\PP \VerseOne{1}Καὶ ταῦτα τὰ δικαιώματα, ἃ παραθήσῃ ἐνώπιον αὐτῶν.
\VS{2}Ἐὰν κτήσῃ παῖδα Ἐβραῖον, ἓξ ἔτη δουλεύσει σοι· τῷ δὲ ἑβδόμῳ ἔτει ἀπελεύσεται ἐλεύθερος δωρεάν.
\VS{3}Ἐὰν αὐτὸς μόνος εἰσέλθῃ, καὶ μόνος ἐξελεύσεται· ἐὰν δὲ γυνὴ συνεισέλθῃ μετʼ αὐτοῦ, ἐξελεύσεται καὶ ἡ γυνὴ αὐτοῦ.
\VS{4}Καὶ ἐὰν δὲ ὁ κύριος δῷ αὐτῷ γυναῖκα, καὶ τέκῃ αὐτῷ υἱοὺς ἢ θυγατέρας, ἡ γυνὴ καὶ τὰ παιδία ἔσται τῷ κυρίῳ αὐτοῦ, αὐτὸς δὲ μόνος ἐξελεύσεται.
\VS{5}Ἐὰν δὲ ἀποκριθεὶς εἴπῃ ὁ παῖς, ἠγάπηκα τὸν κύριόν μου, καὶ τὴν γυναῖκα, καὶ τὰ παιδία, οὐκ ἀποτρέχω ἐλεύθερος·
\VS{6}προσάξει αὐτὸν ὁ κύριος αὐτοῦ πρὸς τὸ κριτήριον τοῦ Θεοῦ, καὶ τότε προσάξει αὐτὸν ἐπὶ τὴν θύραν ἐπὶ τὸν σταθμὸν, καὶ τρυπήσει ὁ κύριος αὐτοῦ τὸ οὖς τῷ ὀπητίῳ, καὶ δουλεύσει αὐτῷ εἰς τὸν αἰῶνα.
\par }{\PP \VS{7}Ἐὰν δέ τις ἀποδῶται τὴν ἑαυτοῦ θυγατέρα οἰκέτιν, οὐκ ἀπελεύσεται, ὥσπερ ἀποτρέχουσιν αἱ δοῦλαι.
\VS{8}Ἐὰν μὴ εὐαρεστήσῃ τῷ κυρίῳ αὐτῆς, ἣ αὐτῷ καθωμολογήσατο, ἀπολυτρώσει αὐτήν· ἔθνει δὲ ἀλλοτρίῳ οὐ κύριός ἐστι πωλεῖν αὐτὴν, ὅτι ἠθέτησεν ἐν αὐτῇ.
\VS{9}Ἐὰν δὲ τῷ υἱῷ καθομολογήσηται αὐτὴν, κατὰ τὸ δικαίωμα τῶν θυγατέρων ποιήσει αὐτῇ.
\VS{10}Ἐὰν δὲ ἄλλην λάβῃ ἑαυτῷ, τὰ δέοντα καὶ τὸν ἱματισμὸν καὶ τὴν ὁμιλίαν αὐτῆς οὐκ ἀποστερήσει.
\VS{11}Ἐὰν δὲ τὰ τρία ταῦτα μὴ ποιήσῃ αὐτῇ, ἐξελεύσεται δωρεὰν ἄνευ ἀργυρίου.
\VS{12}Ἐὰν δὲ πατάξῃ τις τινὰ, καὶ ἀποθάνῃ, θανάτῳ θανατούσθω.
\VS{13}Ὁ δὲ οὐχ ἑκὼν, ἀλλὰ ὁ Θεὸς παρέδωκεν εἰς τὰς χεῖρας αὐτοῦ, δώσω σοι τόπον οὗ φεύξεται ἐκεῖ ὁ φονεύσας.
\VS{14}Ἐὰν δέ τις ἐπιθῆται τῷ πλησίον ἀποκτεῖναι αὐτὸν δόλῳ, καὶ καταφύγῃ, ἀπὸ τοῦ θυσιαστηρίου μου λήψῃ αὐτὸν θανατῶσαι.
\VS{15}Ὃς τύπτει πατέρα αὐτοῦ ἢ μητέρα αὐτοῦ, θανάτῳ θανατούσθω.
\VS{16}Ὁ κακολογῶν πατέρα αὐτοῦ ἢ μητέρα αὐτοῦ, τελευτήσει θανάτῳ.
\VS{17}Ὃς ἐὰν κλέψῃ τις τινὰ τῶν υἱῶν Ἰσραὴλ, καὶ καταδυναστεύσας αὐτὸν ἀποδῶται, καὶ εὑρεθῇ ἐν αὐτῷ, θανάτῳ τελευτάτω.
\VS{18}Ἐὰν δὲ λοιδορῶνται δύο ἄνδρες, καὶ πατάξωσι τὸν πλησίον λίθῳ ἢ πυγμῇ, καὶ μὴ ἀποθάνῃ, κατακλιθῇ δὲ ἐπὶ τὴν κοίτην,
\VS{19}ἐὰν ἐξαναστὰς ὁ ἄνθρωπος περιπατήσῃ ἔξω ἐπὶ ῥάβδου, ἀθῶος ἔσται ὁ πατάξας· πλὴν τῆς ἀργείας αὐτοῦ ἀποτίσει, καὶ τὰ ἰατρεῖα.
\VS{20}Ἐὰν δέ τις πατάξῃ τὸν παῖδα αὐτοῦ ἢ τὴν παιδίσκην αὐτοῦ ἐν ῥάβδῳ, καὶ ἀποθάνῃ ὑπὸ τὰς χεῖρας αὐτοῦ, δίκῃ ἐκδικηθήσεται.
\VS{21}Ἐὰν δὲ διαβιώσῃ ἡμέραν μίαν ἢ δύο, οὐκ ἐκδικηθήτω· τὸ γὰρ ἀργύριον αὐτοῦ ἐστιν.
\VS{22}Ἐὰν δὲ μάχωνται δύο ἄνδρες, καὶ πατάξωσι γυναῖκα ἐν γαστρὶ ἔχουσαν, καὶ ἐξέλθῃ τὸ παιδίον αὐτῆς μὴ ἐξεικονισμένον, ἐπιζήμιον ζημιωθήσεται· καθότι ἂν ἐπιβάλῃ ὁ ἀνὴρ τῆς γυναικὸς, δώσει μετὰ ἀξιώματος.
\VS{23}Ἐὰν δὲ ἐξεικονισμένον ᾖ, δώσει ψυχὴν ἀντὶ ψυχῆς,
\VS{24}Ὀφθαλμὸν ἀντὶ ὀφθαλμοῦ, ὀδόντα ἀντὶ ὀδόντος, χεῖρα ἀντὶ χειρὸς, πόδα ἀντὶ ποδὸς,
\VS{25}κατάκαυμα ἀντὶ κατακαύματος, τραῦμα ἀντὶ τραύματος, μώλωπα ἀντὶ μώλωπος.
\VS{26}Ἐὰν δέ τις πατάξῃ τὸν ὀφθαλμὸν τοῦ οἰκέτου αὐτοῦ, ἢ τὸν ὀφθαλμὸν τῆς θεραπαίνης αὐτοῦ, καὶ ἐκτυφλώσῃ, ἐλευθέρους ἐξαποστελεῖ αὐτοὺς ἀντὶ τοῦ ὀφθαλμοῦ αὐτῶν.
\VS{27}Ἐὰν δὲ τὸν ὀδόντα τοῦ οἰκέτου, ἢ τὸν ὀδόντα τῆς θεραπαίνης αὐτοῦ ἐκκόψῃ, ἐλευθέρους ἐξαποστελεῖ αὐτοὺς ἀντὶ τοῦ ὀδόντος αὐτῶν.
\VS{28}Ἐὰν δὲ κερατίσῃ ταῦρος ἄνδρα ἢ γυναῖκα καὶ ἀποθάνῃ, λίθοις λιθοβοληθήσεται ὁ ταῦρος, καὶ οὐ βρωθήσεται τὰ κρέα αὐτοῦ· ὁ δὲ κύριος τοῦ ταύρου ἀθῶος ἔσται.
\VS{29}Ἐὰν δὲ ὁ ταῦρος κερατιστὴς ᾖ πρὸ τῆς χθὲς καὶ πρὸ τῆς τρίτης, καὶ διαμαρτύρωνται τῷ κυρίῳ αὐτοῦ, καὶ μὴ ἀφανίσῃ αὐτὸν, ἀνέλῃ δὲ ἄνδρα ἢ γυναῖκα, ὁ ταῦρος λιθοβοληθήσεται, καὶ ὁ κύριος αὐτοῦ προσαποθανεῖται.
\VS{30}Ἐὰν δὲ λύτρα ἐπιβληθῇ αὐτῷ, δώσει λύτρα τῆς ψυχῆς αὐτοῦ ὅσα ἐὰν ἐπιβάλωσιν αὐτῶ.
\VS{31}Ἐὰν δὲ υἱὸν ἢ θυγατέρα κερατίσῃ, κατὰ τὸ δικαίωμα τοῦτο ποιήσωσιν αὐτῷ.
\VS{32}Ἐὰν δὲ παῖδα κερατίσῃ ὁ ταῦρος ἢ παιδίσκην, ἀργυρίου τριάκοντα δίδραχμα δώσει τῷ κυρίῳ αὐτῶν, καὶ ὁ ταῦρος λιθοβοληθήσεται.
\VS{33}Ἐὰν δέ τις ἀνοίξῃ λάκκον ἢ λατομήσῃ λάκκον, καὶ μὴ καλύψῃ αὐτὸν, καὶ ἐμπέσῃ ἐκεῖ μόσχος ἢ ὄνος,
\VS{34}ὁ κύριος τοῦ λάκκου ἀποτίσει, ἀργύριον δώσει τῷ κυρίῳ αὐτῶν· τὸ δὲ τετελευτηκὸς αὐτῷ ἔσται.
\VS{35}Ἐὰν δὲ κερατίσῃ τινὸς ταῦρος τόν ταῦρον τοῦ πλησίον, καὶ τελευτήσῃ, ἀποδώσονται τὸν ταῦρον τὸν ζῶντα, καὶ διελοῦνται τὸ ἀργύριον αὐτοῦ, καὶ τὸν ταῦρον τὸν τεθνηκότα διελοῦνται.
\VS{36}Ἐὰν δὲ γνωρίζηται ὁ ταῦρος ὅτι κερατιστής ἐστι πρὸ τῆς χθὲς καὶ πρὸ τῆς τρίτης ἡμέρας, καὶ διαμεμαρτυρημένοι ὦσι τῷ κυρίῳ αὐτοῦ· καὶ μὴ ἀφανίσῃ αὐτὸν, ἀποτίσει ταῦρον ἀντὶ ταύρου, ὁ δὲ τετελευτηκὼς αὐτῷ ἔσται.
\par }{\PP \VS{37}Ἐὰν δέ τις κλέψῃ μόσχον ἢ πρόβατον, καὶ σφάξῃ ἢ ἀποδῶται, πέντε μόσχους ἀποτίσει ἀντὶ τοῦ μόσχου, καὶ τέσσερα πρόβατα ἀντὶ τοῦ προβάτου.

\Chap{22}\VerseOne{1}Ἐὰν δὲ ἐν τῷ διορύγματι εὑρεθῇ ὁ κλέπτης, καὶ πληγεὶς ἀποθάνῃ, οὐκ ἔστιν αὐτῷ φόνος.
\VS{2}Ἐὰν δὲ ἀνατείλῃ ὁ ἥλιος ἐπʼ αὐτῷ, ἔνοχός ἐστιν, ἀνταποθανεῖται· ἐὰν δὲ μὴ ὑπάρχῃ αὐτῷ, πραθήτω ἀντὶ τοῦ κλέμματος.
\VS{3}Ἐὰν δὲ καταλειφθῇ καὶ εὑρεθῇ ἐν τῇ χειρὶ αὐτοῦ τὸ κλέμμα ἀπό τε ὄνου ἕως προβάτου ζῶντα, διπλᾶ αὐτὰ ἀποτίσει.
\VS{4}Ἐὰν δὲ καταβοσκήσῃ τις ἀγρὸν ἢ ἀμπελῶνα, καὶ ἀφῇ τὸ κτῆνος αὐτοῦ καταβοσκῆσαι ἀγρὸν ἕτερον, ἀποτίσει ἐκ τοῦ ἀγροῦ αὐτοῦ κατὰ τὸ γέννημα αὐτοῦ· ἐὰν δὲ πάντα τὸν ἀγρὸν καταβοσκήσῃ, τὰ βέλτιστα τοῦ ἀγροῦ αὐτοῦ καὶ τὰ βέλτιστα τοῦ ἀμπελῶνος αὐτοῦ ἀποτίσει.
\VS{5}Ἐὰν δὲ ἐξελθὸν πῦρ εὕρῃ ἀκάνθας, καὶ προσεμπρήσῃ ἅλωνας ἢ στάχυς ἢ πεδίον, ἀποτίσει ὁ τὸ πῦρ ἐκκαύσας.
\par }{\PP \VS{6}Ἐὰν δέ τις δῷ τῷ πλησίον ἀργύριον ἢ σκεύη φυλάξαι, καὶ κλαπῇ ἐκ τῆς οἰκίας τοῦ ἀνθρώπου, ἐὰν εὑρεθῇ ὁ κλέψας, ἀποτίσει τὸ διπλοῦν.
\VS{7}Ἐὰν δὲ μὴ εὑρεθῇ ὁ κλέψας, προσελεύσεται ὁ κύριος τῆς οἰκίας ἐνώπιον τοῦ Θεοῦ, καὶ ὀμεῖται ἦ μὴν μὴ αὐτὸν πεπονηρεῦσθαι ἐφʼ ὅλης τῆς παρακαταθήκης τοῦ πλησίον,
\VS{8}κατὰ πᾶν ῥητὸν ἀδίκημα, περί τε μόσχου, καὶ ὑποζυγίου, καὶ προβάτου, καὶ ἱματίου, καὶ πάσης ἀπωλίας τῆς ἐνκαλουμένης· ὅ, τι οὖν ἂν ᾖ, ἐνώπιον τοῦ Θεοῦ ἐλεύσεται ἡ κρίσις ἀμφοτέρων, καὶ ὁ ἁλοὺς διὰ τοῦ Θεοῦ, ἀποτίσει διπλοῦν τῷ πλησίον.
\VS{9}Ἐὰν δέ τις δῷ τῷ πλησίον ὑποζύγιον ἢ μόσχον ἢ πρόβατον ἢ πᾶν κτῆνος φυλάξαι, καὶ συντριβῇ ἢ τελευτήσῃ ἢ αἰχμάλωτον γένηται, καὶ μηδεὶς γνῷ,
\VS{10}ὅρκος ἔσται τοῦ Θεοῦ ἀνὰ μέσον ἀμφοτέρων, ἦ μὴν μὴ αὐτὸν πεπονηρεῦσθαι καθόλου τῆς παρακαταθήκης τοῦ πλησίον· καὶ οὕτως προσδέξεται ὁ κύριος αὐτοῦ, καὶ οὐκ ἀποτίσει.
\VS{11}Ἐὰν δὲ κλαπῇ παρʼ αὐτοῦ, ἀποτίσει τῷ κυρίῳ·
\VS{12}Ἐὰν δὲ θηριάλωτον γένηται, ἄξει αὐτὸν ἐπὶ τὴν θήραν, καὶ οὐκ ἀποτίσει.
\VS{13}Ἐὰν δὲ αἰτήσῃ τις παρὰ τοῦ πλησίον, καὶ συντριβῇ ἢ ἀποθάνῃ ἢ αἰχμάλωτον γένηται, ὁ δὲ κύριος μὴ ᾖ μετʼ αὐτοῦ, ἀποτίσει.
\VS{14}Ἐὰν δὲ ὁ κύριος ᾖ μετʼ αὐτοῦ, οὐκ ἀποτίσει· ἐὰν δὲ μισθωτὸς ᾖ, ἔσται αὐτῷ ἀντὶ τοῦ μισθοῦ αὐτοῦ.
\par }{\PP \VS{15}Ἐὰν δὲ ἀπατήσῃ τις παρθένον ἀμνήστευτον, καὶ κοιμηθῇ μετʼ αὐτῆς, φερνῇ φερνιεῖ αὐτὴν αὐτῷ γυναῖκα.
\VS{16}Ἐὰν δὲ ἀνανεύων ἀνανεύσῃ, καὶ μὴ βούληται ὁ πατὴρ αὐτῆς δοῦναι αὐτὴν αὐτῷ γυναῖκα, ἀργύριον ἀποτίσει τῷ πατρὶ καθʼ ὅσον ἐστὶν ἡ φερνὴ τῶν παρθένων.
\VS{17}Φαρμακοὺς οὐ περιποιήσετε.
\VS{18}Πᾶν κοιμώμενον μετὰ κτήνους θανάτῳ ἀποκτενεῖτε αὐτούς.
\VS{19}Ὁ θυσιάζων θεοῖς θανάτῳ ἐξολοθρευθήσεται, πλὴν Κυρίῳ μόνῳ.
\par }{\PP \VS{20}Καὶ προσήλυτον οὐ κακώσετε, οὐδὲ μὴ θλίψητε αὐτόν· ἦτε γάρ προσήλυτοι ἐν γῇ Αἰγύπτῳ.
\VS{21}Πᾶσαν χήραν καὶ ὀρφανὸν οὐ κακώσετε.
\VS{22}Ἐὰν δὲ κακίᾳ κακώσητε αὐτοὺς, καὶ κεκράξαντες καταβοήσωσι πρός με, ἀκοῇ εἰσακούσομαι τῆς φωνῆς αὐτῶν,
\VS{23}καὶ ὀργισθήσομαι θυμῷ, καὶ ἀποκτενῶ ὑμᾶς μαχαίρᾳ, καὶ ἔσονται αἱ γυναῖκες ὑμῶν χῆραι, καὶ τὰ παιδία ὑμῶν ὀρφανά.
\VS{24}Ἐὰν δὲ ἀργύριον ἐκδανείσῃς τῷ ἀδελφῷ τῷ πενιχρῷ παρὰ σοὶ, οὐκ ἔσῃ αὐτὸν κατεπείγων, οὐκ ἐπιθήσεις αὐτῷ τόκον.
\VS{25}Ἐὰν δὲ ἐνεχύρασμα ἐνεχυράσῃς τὸ ἱμάτιον τοῦ πλησίον, πρὸ δυσμῶν ἡλίου ἀποδώσεις αὐτῷ·
\VS{26}Ἔστι γὰρ τοῦτο περιβόλαιον αὐτοῦ, μόνον τοῦτο τὸ ἱμάτιον ἀσχημοσύνης αὐτοῦ· ἐν τίνι κοιμηθήσεται; Ἐὰν οὖν καταβοήσῃ πρός μέ, εἰσακούσομαι αὐτοῦ· ἐλεήμων γάρ εἰμι.
\VS{27}Θεοὺς οὐ κακολογήσεις, καὶ ἄρχοντα τοῦ λαοῦ σου οὐ κακῶς ἐρεῖς.
\VS{28}Ἀπαρχὰς ἅλωνος καὶ ληνοῦ σου οὐ καθυστερήσεις· τὰ πρωτότοκα τῶν υἱῶν σου δώσεις ἐμοί.
\VS{29}Οὕτω ποιήσεις τὸν μόσχον σου καὶ τὸ πρόβατόν σου καὶ τὸ ὑποζύγιόν σου· ἑπτὰ ἡμέρας ἔσται ὑπὸ τὴν μητέρα, τῇ δὲ ὀγδόῃ ἡμέρᾳ ἀποδώσεις μοι αὐτό.
\VS{30}Καὶ ἄνδρες ἅγιοι ἔσεσθέ μοι· καὶ κρέας θηριάλωτον οὐκ ἔδεσθε, τῷ κυνὶ ἀποῤῥίψατε αὐτό.

\par }\Chap{23}{\PP \VerseOne{1}Οὐ παραδέξῃ ἀκοὴν ματαίαν· οὐ συγκαταθήσῃ μετὰ τοῦ ἀδίκου γενέσθαι μάρτυς ἄδικος.
\VS{2}Οὐκ ἔσῃ μετὰ πλειόνων ἐπὶ κακίᾳ· οὐ προστεθήσῃ μετὰ πλήθους ἐκκλῖναι μετὰ τῶν πλειόνων, ὥστε ἑκκλεῖσαι κρίσιν.
\VS{3}Καὶ πένητα οὐκ ἐλεήσεις ἐν κρίσει.
\VS{4}Ἐὰν δὲ συναντήσῃς τῷ βοῒ τοῦ ἐχθροῦ σου, ἢ τῷ ὑποζυγίῳ αὐτοῦ πλανωμένοις, ἀποστρέψας ἀποδώσεις αὐτῷ.
\VS{5}Ἐὰν δὲ ἴδῃς τὸ ὑποζύγιον τοῦ ἐχθροῦ σου πεπτωκὸς ὑπὸ τὸν γόμον αὐτοῦ, οὐ παρελεύσῃ αὐτὸ, ἀλλὰ συναρεῖς αὐτὸ μετʼ αὐτοῦ.
\par }{\PP \VS{6}Οὐ διαστρέψεις κρίμα πένητος ἐν κρίσει αὐτοῦ.
\VS{7}Ἀπὸ παντὸς ῥήματος ἀδίκου ἀποστήσῃ· ἀθῷον καὶ δίκαιον οὐκ ἀποκτενεῖς· καὶ οὐ δικαιώσεις τὸν ἀσεβῆ ἕνεκεν δώρων.
\VS{8}Καὶ δῶρα οὐ λήψῃ· τὰ γὰρ δῶρα ἐκτυφλοῖ ὀφθαλμοὺς βλεπόντων, καὶ λυμαῖνεται ῥήματα δίκαια.
\VS{9}Καὶ προσήλυτον οὐ θλίψετε· ὑμεῖς γὰρ οἴδατε τὴν ψυχὴν τοῦ προσηλύτου· αὐτοὶ γὰρ προσήλυτοι ἦτε ἐν γῇ Αἰγύπτῳ.
\VS{10}Ἓξ ἔτη σπερεῖς τὴν γῆν σου, καὶ συνάξεις τὰ γεννήματα αὐτῆς.
\VS{11}Τῷ δὲ ἑβδόμῳ ἄφεσιν ποιήσεις, καὶ ἀνήσεις αὐτὴν, καὶ ἔδονται οἱ πτωχοὶ τοῦ ἔθνους σου· τὰ δὲ ὑπολειπόμενα ἔδεται τὰ ἄγρια θηρία· οὕτως ποιήσεις τὸν ἀμπελῶνά σου, καὶ τὸν ἐλαιῶνά σου.
\VS{12}Ἓξ ἡμέρας ποιήσεις τὰ ἔργα σου, τῇ δὲ ἡμέρᾳ τῇ ἑβδόμῃ, ἀνάπαυσις· ἵνα ἀναπαύσηται ὁ βοῦς σου, καὶ τὸ ὑποζύγιόν σου, καὶ ἵνα ἀναψύξῃ ὁ υἱὸς τῆς παιδίσκης σου καὶ ὁ προσήλυτος.
\VS{13}Πάντα ὅσα εἴρηκα πρὸς ὑμᾶς, φυλάξασθε· καὶ ὄνομα θεῶν ἑτέρων οὐκ ἀναμνησθήσεσθε, οὐδὲ μὴ ἀκουσθῇ ἐκ τοῦ στόματος ὑμῶν.
\par }{\PP \VS{14}Τρεῖς καιροὺς τοῦ ἐνιαυτοῦ ἑορτάσατέ μοι.
\VS{15}Τὴν ἑορτὴν τῶν ἀζύμων φυλάξασθε ποιεῖν· ἑπτὰ ἡμέρας ἔδεσθε ἄζυμα, καθάπερ ἐνετειλάμην σοι κατὰ τὸν καιρὸν τοῦ μηνὸς τῶν νέων· ἐν γὰρ αὐτῷ ἐξῆλθες ἐξ Αἰγύπτου· οὐκ ὀφθήσῃ ἐνώπίον μου κενός.
\VS{16}Καὶ ἑορτὴν θερισμοῦ πρωτογεννημάτων ποιήσεις τῶν ἔργων σου, ὧν ἐὰν σπείρῃς ἐν τῷ ἀγρῷ σου, καὶ ἑορτὴν συντελείας ἐπʼ ἐξόδου τοῦ ἐνιαυτοῦ ἐν τῇ συναγωγῇ τῶν ἔργων σου τῶν ἐκ τοῦ ἀγροῦ σου.
\VS{17}Τρεῖς καιροὺς τοῦ ἐνιαυτοῦ ὀφθήσεται πᾶν ἀρσενικόν σου ἐνώπιον Κυρίου τοῦ Θεοῦ σου.
\VS{18}Ὅταν γὰρ ἐκβάλω τὰ ἔθνη ἀπὸ προσώπου σου, καὶ ἐμπλατύνω τὰ ὅριά σου, οὐ θύσεις ἐπὶ ζύμῃ αἷμα θυμιάματός μου, οὐδὲ μὴ κοιμηθῇ στέαρ τῆς ἑορτῆς μου ἕως πρωΐ.
\VS{19}Τὰς ἀπαρχὰς τῶν πρωτογενημάτων τῆς γνς σου εἰσοίσεις εἰς τὸν οἶκον Κυρίου τοῦ Θεοῦ σου· οὐχ ἑψήσεις ἄρνα ἐν γάλακτι μητρὸς αὐτοῦ.
\VS{20}Καὶ ἰδοὺ ἐγὼ ἀποστέλλω τὸν ἄγγελόν μου πρὸ προσώπου σου, ἵνα φυλάξῃ σε ἐν τῇ ὁδῷ, ὅπως εἰσαγάγῃ σε εἰς τὴν γῆν, ἣν ἡτοίμασά σοι.
\VS{21}Πρόσεχε σεαυτῷ, καὶ εἰσάκουε αὐτοῦ, καὶ μὴ ἀπείθει αὐτῷ, οὐ γὰρ μὴ ὑποστείληταί σε· τὸ γὰρ ὄνομά μου ἐστὶν ἐπʼ αὐτῷ.
\VS{22}Ἐὰν ἀκοῇ ἀκούσητε τῆς ἐμῆς φωνῆς, καὶ ποιήσῃς πάντα ὅσα ἂν ἐντείλωμαί σοι, καὶ φυλάξητε τὴν διαθήκην μου, ἔσεσθέ μοι λαὸς περιούσιος ἀπὸ πάντων τῶν ἐθνῶν· ἐμὴ γάρ ἐστι πᾶσα ἡ γῆ· ὑμεῖς δὲ ἔσεσθέ μοι βασίλειον ἱεράτευμα, καὶ ἔθνος ἅγιον· ταῦτα τὰ ῥήματα ἐρεῖς τοῖς υἱοῖς Ἰσραὴλ, ἐὰν ἀκοῇ ἀκούσητε τῆς φωνῆς μου, καὶ ποιήσητε πάντα ὅσα ἂν εἴπω σοι, ἐχθρεύσω τοῖς ἐχθροῖς σου, καὶ ἀντικείσομαι τοῖς ἀντικειμένοις σοι.
\VS{23}Πορεύσεται γὰρ ὁ ἄγγελός μου ἡγούμενός σου, καὶ εἰσάξει σε πρὸς τὸν Ἀμοῤῥαῖον, καὶ Χετταῖον, καὶ Φερεζαῖον, καὶ Χαναναῖον, καὶ Γεργεσαῖον, καὶ Εὑαῖον, καὶ Ἰεβουσαῖον, καὶ ἐκτρίψω αὐτούς.
\VS{24}Οὐ προσκυνήσεις τοῖς θεοῖς αὐτῶν, οὐδὲ μὴ λατρεύσῃς αὐτοῖς· οὐ ποιήσεις κατὰ τὰ ἔργα αὐτῶν· ἀλλὰ καθαιρέσει καθελεῖς, καὶ συντρίβων συντρίψεις τὰς στήλας αὐτῶν.
\VS{25}Καὶ λατρεύσεις Κυρίῳ τῷ Θεῷ σου· καὶ εὐλογήσω τὸν ἄρτον σου καὶ τὸν οἶνόν σου καὶ τὸ ὕδωρ σου, καὶ ἀποστρέψω μαλακίαν ἀφʼ ὑμῶν.
\VS{26}Οὐκ ἔσται ἄγονος, οὐδὲ στεῖρα ἐπὶ τῆς γῆς σου· τὸν ἀριθμὸν τῶν ἡμερῶν σου ἀναπληρῶν ἀναπληρώσω.
\VS{27}Καὶ τὸν φόβον ἀποστελῶ ἡγούμενόν σου, καὶ ἐκστήσω πάντα τὰ ἔθνη, εἰς οὓς σὺ εἰσπορεύῃ εἰς αὐτούς· καὶ δώσω πάντας τοὺς ὑπεναντίους σου φυγάδας.
\VS{28}Καὶ ἀποστελῶ τὰς σφηκίας προτέρας σου· καὶ ἐκβαλεῖς τοὺς Ἀμοῤῥαίους, καὶ τοὺς Εὑαίους, καὶ τοὺς Χαναναίους, καὶ τοὺς Χετταίους ἀπὸ σοῦ.
\VS{29}Οὐκ ἐκβαλῶ αὐτοὺς ἐν ἐνιαυτῷ ἑνὶ, ἵνα μὴ γένηται ἡ γῆ ἔρημος, καὶ πολλὰ γένηται ἐπὶ σὲ τὰ θηρία τῆς γῆς.
\VS{30}Κατὰ μικρὸν ἐκβαλῶ αὐτοὺς ἀπὸ σοῦ, ἕως ἂν αὐξηθῇς καὶ κληρονομήσῃς τὴν γῆν.
\VS{31}Καὶ θήσω τὰ ὅριά σου ἀπὸ τῆς ἐρυθρᾶς θαλάσσης, ἕως τῆς θαλάσσης τῆς Φυλιστιείμ· καὶ ἀπὸ τῆς ἐρήμου, ἕως τοῦ μεγάλου ποταμοῦ Εὐφράτου· καὶ παραδώσω εἰς τὰς χεῖρας ὑμῶν τοὺς ἐγκαθημένους ἐν τῇ γῇ, καὶ ἐκβαλῶ αὐτοὺς ἀπὸ σοῦ.
\VS{32}Οὐ συγκαταθήσῃ αὐτοῖς καὶ τοῖς θεοῖς αὐτῶν διαθήκην.
\VS{33}Καὶ οὐκ ἐνκαθήσονται ἐν τῇ γῇ σου, ἵνα μὴ ἁμαρτεῖν σε ποιήσωσι πρὸς μέ· ἐὰν γὰρ δουλεύσῃς τοῖς θεοῖς αὐτῶν, οὗτοι ἔσονταί σοι πρόσκομμα.

\par }\Chap{24}{\PP \VerseOne{1}Καὶ Μωυσῇ εἶπεν, ἀνάβηθι πρὸς τὸν Κύριον σὺ καὶ Ἀαρὼν, καὶ Ναδὰβ, καὶ Ἀβιοὺδ, καὶ ἑβδομήκοντα τῶν πρεσβυτέρων Ἰσραήλ· καὶ προσκυνήσουσι μακρόθεν τῷ Κυρίῳ.
\VS{2}Καὶ ἐγγιεῖ Μωσῆς μόνος πρὸς τὸν Θεὸν, αὐτοὶ δὲ οὐκ ἐγγιοῦσιν, ὁ δὲ λαὸς οὐ συναναβήσεται μετʼ αὐτῶν.
\VS{3}Εἰσῆλθε δὲ Μωυσῆς, καὶ διηγήσατο τῷ λαῷ πάντα τὰ ῥήματα τοῦ Θεοῦ καὶ τὰ δικαιώματα· ἀπεκρίθη δὲ πᾶς ὁ λαὸς φωνῇ μιᾷ, λέγοντες, πάντας τοὺς λόγους, οὓς ἐλάλησε Κύριος, ποιήσομεν, καὶ ἀκουσόμεθα.
\VS{4}Καὶ ἔγραψε Μωυσῆς πάντα τὰ ῥήματα Κυρίου· ὀρθρίσας δὲ Μωυσῆς τὸ πρωῒ ᾠκοδόμησε θυσιαστήριον ὑπὸ τὸ ὄρος, καὶ δώδεκα λίθους εἰς τὰς δώδεκα φυλὰς τοῦ Ἰσραήλ.
\VS{5}Καὶ ἐξαπέστειλε τοὺς νεανίσκους τῶν υἱῶν Ἰσραήλ, καὶ ἀνήνεγκαν ὁλοκαυτώματα· καὶ ἔθυσαν θυσίαν σωτηρίου τῷ Θεῷ μοσχάρια.
\VS{6}Λαβὼν δὲ Μωυσῆς τὸ ἥμισυ τοῦ αἵματος, ἐνέχεεν εἰς κρατῆρας, τὸ δὲ ἥμισυ τοῦ αἵματος προσέχεε πρὸς τὸ θυσιαστήριον.
\VS{7}Καὶ λαβὼν τὸ βιβλίον τῆς διαθήκης, ἀνέγνω εἰς τὰ ὦτα τοῦ λαοῦ· καὶ εἶπαν, πάντα ὅσα ἐλάλησε Κύριος, ποιήσομεν καὶ ἀκουσόμεθα.
\VS{8}Λαβὼν δὲ Μωυσῆς τὸ αἷμα, κατεσκέδασε τοῦ λαοῦ, καὶ εἶπεν, ἰδοὺ τὸ αἷμα τῆς διαθήκης, ἧς διέθετο Κύριος πρὸς ὑμᾶς περὶ πάντων τῶν λόγων τούτων.
\par }{\PP \VS{9}Καὶ ἀνέβη Μωυσῆς καὶ Ἀαρὼν, καὶ Ναδὰβ, καὶ Ἀβιοῦδ, καὶ ἑβδομήκοντα τῆς γερουσίας Ἰσραήλ.
\VS{10}Καὶ εἶδον τὸν τόπον οὗ εἱστήκει ὁ Θεὸς τοῦ Ἰσραήλ· καὶ τὰ ὑπὸ τοὺς πόδας αὐτοῦ, ὡσεὶ ἔργον πλίνθου σαπφείρου, καὶ ὥσπερ εἶδος στερεώματος τοῦ οὐρανοῦ τῇ καθαριότητι.
\VS{11}Καὶ τῶν ἐπιλέκτων τοῦ Ἰσραὴλ οὐ διεφώνησεν οὐδὲ εἷς· καὶ ὤφθησαν ἐν τῷ τόπῳ τοῦ Θεοῦ, καὶ ἔφαγον καὶ ἔπιον.
\VS{12}Καὶ εἶπε Κύριος πρὸς Μωυσῆν, ἀνάβηθι πρὸς με εἰς τὸ ὄρος, καὶ ἴσθι ἐκεῖ· καὶ δώσω σοι τὰ πυξία τὰ λίθινα, τὸν νόμον καὶ τὰς ἐντολας, ἃς ἔγραψα νομοθετῆσαι αὐτοῖς.
\VS{13}Καὶ ἀναστὰς Μωυσῆς καὶ Ἰησοῦς ὁ παρεστηκὼς αὐτῷ, ἀνέβησαν εἰς τὸ ὄρος τοῦ Θεοῦ.
\VS{14}Καὶ τοῖς πρεσβυτέροις εἶπαν, ἡσυχάζετε αὐτοῦ, ἕως ἀναστρέψωμεν πρὸς ὑμᾶς· καὶ ἰδοὺ Ἀαρὼν καὶ Ὢρ μεθʼ ὑμῶν· ἐάν τινι συμβῇ κρίσις, προσπορευέσθωσαν αὐτοῖς.
\VS{15}Καὶ ἀνέβη Μωυσῆς καὶ Ἰησοῦς εἰς τὸ ὄρος· καὶ ἐκάλυψεν ἡ νεφέλη τὸ ὄρος.
\VS{16}Καὶ κατέβη ἡ δόξα τοῦ Θεοῦ ἐπὶ τὸ ὄρος τὸ Σινὰ, καὶ ἐκάλυψεν αὐτὸ ἡ νεφέλη ἓξ ἡμέρας· καὶ ἐκάλεσε Κύριος τὸν Μωυσῆν τῇ ἡμέρᾳ τῇ ἑβδόμῃ ἐκ μέσου τῆς νεφέλης.
\VS{17}Τὸ δὲ εἶδος τῆς δόξης Κυρίου, ὡσεὶ πῦρ φλέγον ἐπὶ τῆς κορυφῆς τοῦ ὄρους, ἐναντίον τῶν υἱῶν Ἰσραήλ.
\VS{18}Καὶ εἰσῆλθε Μωυσῆς εἰς τὸ μέσον τῆς νεφέλης, καὶ ἀνέβη εἰς τὸ ὄρος· καὶ ἦν ἐκεῖ ἐν τῷ ὄρει τεσσεράκοντα ἡμέρας καὶ τεσσαράκοντα νύκτας.

\par }\Chap{25}{\PP \VerseOne{1}Καὶ ἐλάλησε Κύριος πρὸς Μωυσῆν, λέγων,
\VS{2}εἶπον τοῖς υἱοῖς Ἰσραὴλ, καὶ λάβετε ἀπαρχὰς παρὰ πάντων, οἷς ἂν δόξῃ τῇ καρδίᾳ, καὶ λήψεσθε τὰς ἀπαρχάς μου.
\VS{3}Καὶ αὕτη ἐστὶν ἡ ἀπαρχὴ, ἣν λήψεσθε παρʼ αὐτῶν· χρυσίον, καὶ ἀργύριον, καὶ χαλκὸν,
\VS{4}καὶ ὑάκινθον, καὶ πορφύραν, καὶ κόκκινον διπλοῦν, καὶ βύσσον κεκλωσμένην, καὶ τρίχας αἰγείας,
\VS{5}καὶ δέρματα κριῶν ἠρυθροδανωμένα, καὶ δέρματα ὑακίνθινα, καὶ ξύλα ἄσηπτα,
\VS{5a}καὶ ἔλαιον εἰς τὴν φαῦσιν, θυμιάματα εἰς τὸ ἔλαιον τῆς χρίσεως, καὶ εἰς τὴν σύνθεσιν τοῦ θυμιάματος,
\VS{7}καὶ λίθους Σαρδίου, καὶ λίθους εἰς τὴν γλυφὴν εἰς τὴν ἐπωμίδα, καὶ τὸν ποδήρη.
\VS{8}Καὶ ποιήεις μοι ἁγίασμα, καὶ ὀφθήσομαι ἐν ὑμῖν.
\VS{9}Καὶ ποιήσεις μοι κατὰ πάντα ὅσα σοι δεικνύω ἐν τῷ ὄρει, τὸ παράδειγμα τῆς σκηνῆς, καὶ τὸ παράδειγμα πάντων τῶν σκευῶν αὐτῆς· οὕτω ποιήσεις.
\VS{10}Καὶ ποιήσεις κιβωτὸν μαρτυρίου ἐκ ξύλων ἀσήπτων, δύο πήχεων καὶ ἡμίσους τὸ μῆκος, καὶ πήχεος καὶ ἡμίσους τὸ πλάτος, καὶ πήχεως καὶ ἡμίσους τὸ ὕψος.
\VS{11}Καὶ καταχρυσώσεις αὐτὴν χρυσίῳ καθαρῷ, ἔσωθεν καὶ ἔξωθεν χρυσώσεις αὐτήν· καὶ ποιήσεις αὐτῇ κυμάτια χρυσᾶ στρεπτὰ κύκλῳ.
\VS{12}Καὶ ἐλάσεις αὐτῇ τέσσαρας δακτυλίους χρυσοῦς, καὶ ἐπιθήσεις ἐπὶ τὰ τέσσαρα κλίτη· δύο δακτυλίους ἐπὶ τὸ κλίτος τὸ ἓν, καὶ δύο δακτυλίους ἐπὶ τὸ κλίτος τὸ δεύτερον.
\VS{13}Ποιήσεις δὲ ἀναφορεῖς ξύλα ἄσηπτα, καὶ καταχρυσώσεις αὐτὰ χρυσίῳ·
\VS{14}Καὶ εἰσάξεις τοὺς ἀναφορεῖς εἰς τοὺς δακτυλίους τοὺς ἐν τοῖς κλίτεσι τῆς κιβωτοῦ, αἴρειν τὴν κιβωτὸν ἐν αὐτοῖς.
\VS{15}Ἐν τοῖς δακτυλίοις τῆς κιβωτοῦ ἔσονται οἱ ἀναφορεῖς ἀκίνητοι.
\VS{16}Καὶ ἐμβαλεῖς εἰς τὴν κιβωτὸν τὰ μαρτύρια, ἃ ἂν δῶ σοι.
\VS{17}Καὶ ποιήσεις ἱλαστήριον ἐπίθεμα χρυσίου καθαροῦ, δύο πήχεων καὶ ἡμίσους τὸ μῆκος, καὶ πήχεως καὶ ἡμίσους τὸ πλάτος.
\VS{18}Καὶ ποιήσεις δύο χερουβὶμ χρυσοτορευτὰ, καὶ ἐπιθήσεις αὐτὰ ἐξ ἀμφοτέρων τῶν κλιτῶν τοῦ ἱλαστηρίου.
\VS{19}Ποιηθήσονται χεροὺβ εἷς ἐκ τοῦ κλίτους τούτου, καὶ χεροὺβ εἷς ἐκ τοῦ κλίτους τοῦ δευτέρου τοῦ ἱλαστηρίου· καὶ ποιήσεις τοὺς δύο χερουβὶμ ἐπὶ τὰ δύο κλίτη.
\VS{20}Ἔσονται οἱ χερουβὶμ ἐκτείνοντες τὰς πτέρυγας ἐπάνωθεν, συσκιάζοντες ἐν ταῖς πτέρυξιν αὐτῶν ἐπὶ τοῦ ἱλαστηρίου, καὶ τὰ πρόσωπα αὐτῶν εἰς ἄλληλα, εἰς τὸ ἱλαστήριον ἔσονται τὰ πρόσωπα τῶν χερουβίμ.
\VS{21}Καὶ ἐπιθήσεις τὸ ἱλαστήριον ἐπὶ τὴν κιβωτὸν ἄνωθεν, καὶ εἰς τὴν κιβωτὸν ἐμβαλεῖς τὰ μαρτύρια, ἃ ἂν δῶ σοι.
\VS{22}Καὶ γνωσθήσομαί σοι ἐκεῖθεν, καὶ λαλήσω σοι ἄνωθεν τοῦ ἱλαστηρίου ἀνὰ μέσον τῶν δύο χερουβὶμ, τῶν ὄντων ἐπὶ τῆς κιβωτοῦ τοῦ μαρτυρίου, καὶ κατὰ πάντα ὅσα ἐὰν ἐντείλωμαί σοι πρὸς τοὺς υἱοὺς Ἰσραήλ.
\VS{23}Καὶ ποιήσεις τράπεζαν χρυσῆν χρυσίου καθαροῦ, δύο πήχεων τὸ μῆκος, καὶ πήχεως τὸ εὖρος, καὶ πήχεως καὶ ἡμίσους τὸ ὕψος.
\VS{24}Καὶ ποιήσεις αὐτῇ στρεπτὰ κυμάτια χρυσᾶ κύκλῳ· καὶ ποιήσεις αὐτῇ στεφάνην παλαιστοῦ κύκλῳ·
\par }{\PP \VS{25}Καὶ ποιήσεις στρεπτὸν κυμάτιον τῇ στεφάνῃ κύκλῳ.
\VS{26}Καὶ ποιήσεις τέσσαρας δακτυλίους χρυσοῦς, καὶ ἐπιθήσεις τοὺς τέσσαρας δακτυλίους ἐπὶ τὰ τέσσαρα μέρη τῶν ποδῶν αὐτῆς ὑπὸ τὴν στεφάνην.
\VS{27}Καὶ ἔσονται οἱ δακτύλιοι εἰς θήκας τοῖς ἀναφορεῦσιν, ὥστε αἴρειν ἐν αὐτοῖς τὴν τράπεζαν.
\VS{28}Καὶ ποιήσεις τοὺς ἀναφορεῖς ἐκ ξύλων ἀσήπτων, καὶ καταχρυσώσεις αὐτοὺς χρυσίῳ καθαρῷ, καὶ ἀρθήσεται ἐν αὐτοῖς ἡ τράπεζα.
\VS{29}Καὶ ποιήσεις τὰ τρυβλία αὐτῆς, καὶ τὰς θυΐσκας, καὶ τὰ σπονδεῖα, καὶ τοὺς κυάθους, ἐν οἷς σπείσεις ἐν αὐτοῖς, ἐκ χρυσίου καθαροῦ ποιήσεις αὐτά.
\VS{30}Καὶ ἐπιθήσεις ἐπὶ τὴν τράπεζαν ἄρτους ἐνωπίους ἐναντίον μου διαπαντός.
\par }{\PP \VS{31}Καὶ ποιήσεις λυχνίαν ἐκ χρυσίου καθαροῦ, τορευτὴν ποιήσεις τὴν λυχνίαν· ὁ καυλὸς αὐτῆς, καὶ ὁ καλαμίσκοι, καὶ οἱ κρατῆρες, καὶ οἱ σφαιρωτῆρες, καὶ τὰ κρίνα ἐξ αὐτῆς ἔσται.
\VS{32}Ἓξ δὲ καλαμίσκοι ἐκπορευόμενοι ἐκ πλαγίων, τρεῖς καλαμίσκοι τῆς λυχνίας ἐκ τοῦ κλίτους τοῦ ἑνὸς αὐτῆς, καὶ τρεῖς καλαμίσκοι τῆς λυχνίας ἐκ τοῦ κλίτους τοῦ δευτέρου.
\VS{33}Καὶ τρεῖς κρατῆρες ἐκτετυπωμένοι καρυΐσκους· ἐν τῷ ἑνὶ καλαμίσκῳ σφαιρωτὴρ καὶ κρίνον· οὕτω τοῖς ἓξ καλαμίσκοις τοῖς ἐκπορευομένοις ἐκ τῆς λυχνίας.
\VS{34}Καὶ ἐν τῇ λυχνίᾳ τέσσαρες κρατῆρες ἐκτετυπωμένοι καρυΐσκους· ἐν τῷ ἑνὶ καλαμίσκῳ σφαιρωτῆρες, καὶ τὰ κρίνα αὐτῆς.
\VS{35}Ὁ σφαιρωτὴρ ὑπὸ τοὺς δύο καλαμίσκους ἐξ αὐτῆς· καὶ σφαιρωτὴρ ὑπὸ τοὺς τέσσαρας καλαμίσκους ἐξ αὐτῆς· οὕτω τοῖς ἓξ καλαμίσκοις τοῖς ἐκπορευομένοις ἐκ τῆς λυχνίας· καὶ ἐν τῇ λυχνίᾳ τέσσαρες κρατῆρες ἐκτετυπωμένοι καρυΐσκους.
\VS{36}Οἱ σφαιρωτῆρες καὶ οἱ καλαμίσκοι ἐξ αὐτῆς ἔστωσαν· ὅλη τορευτὴ ἐξ ἑνὸς χρυσίου καθαροῦ.
\VS{37}Καὶ ποιήσεις τοὺς λύχνους αὐτῆς ἑπτά· καὶ ἐπιθήσεις τοὺς λύχνους, καὶ φανοῦσιν ἐκ τοῦ ἑνὸς προσώπου.
\VS{38}Καὶ τὸν ἐπαρυστῆρα αὐτῆς, καὶ τὰ ὑποθέματα αὐτῆς ἐκ χρυσίου καθαροῦ ποιήσεις.
\VS{39}Πάντα τὰ σκεύη ταῦτα τάλαντον χρυσίου καθαροῦ.
\VS{40}Ὅρα, ποιήσεις κατὰ τὸν τύπον τὸν δεδειγμένον σοι ἐν τῷ ὄρει.

\par }\Chap{26}{\PP \VerseOne{1}Καὶ τὴν σκηνὴν ποιήσεις, δέκα αὐλαίας ἐκ βύσσου κεκλωσμένης, καὶ ὑακίνθου, καὶ πορφύρας, καὶ κοκκίνου κεκλωσμένου χερουβὶμ· ἐργασίᾳ ὑφάντου ποιήσεις αὐτάς.
\VS{2}Μῆκος τῆς αὐλαίας τῆς μιᾶς ὀκτὼ καὶ εἴκοσι πήχεων, καὶ εὖρος τεσσάρων πήχεων ἡ αὐλαία ἡ μία ἔσται· μέτρον τὸ αὐτὸ ἔσται πάσαις ταῖς αὐλαίαις.
\VS{3}Πέντε δὲ αὐλαῖαι ἔσονται ἐξ ἀλλήλων ἐχόμεναι ἡ ἑτέρα ἐκ τῆς ἑτέρας· καὶ πέντε αὐλαῖαι ἔσονται συνεχόμεναι ἑτέρα τῇ ἑτέρᾳ.
\VS{4}Καὶ ποιήσεις αὐταῖς ἀγκύλας ὑακινθίνας ἐπὶ τοῦ χείλους τῆς αὐλαίας τῆς μιᾶς, ἐκ τοῦ ἑνὸς μέρους εἰς τὴν συμβολήν· καὶ οὕτω ποιήσεις ἐπὶ τοῦ χείλους τῆς αὐλαίας τῆς ἐξωτέρας πρὸς τῇ συμβολῇ τῇ δευτέρᾳ.
\VS{5}Πεντήκοντα ἀγκύλας ποιήσεις τῇ αὐλαίᾳ τῇ μιᾷ, καὶ πεντήκοντα ἀγκύλας ποιήσεις ἐκ τοῦ μέρους τῆς αὐλαίας κατὰ τὴν συμβολὴν τῆς δευτέρας, ἀντιπρόσωποι ἀντιπίπτουσαι ἀλλήλαις εἰς ἑκάστην.
\VS{6}Καὶ ποιήσεις κρίκους πεντήκοντα χρυσοῦς· καὶ συνάψεις τὰς αὐλαίας ἑτέραν τῇ ἑτέρα τοῖς κρίκοις· καὶ ἔσται ἡ σκηνὴ μία.
\VS{7}Καὶ ποιήσεις δέῤῥεις τριχίνας σκέπην ἐπὶ τῆς σκηνῆς, ἕνδεκα δέῤῥεις ποιήσεις αὐτάς.
\VS{8}Τὸ μῆκος τῆς δέῤῥεως τῆς μιᾶς, τριάκοντα πήχεων, καὶ τεσσάρων πήχεων τὸ εὖρος τῆς δέῤῥεως τῆς μιᾶς· τὸ αὐτὸ μέτρον ἔσται ταῖς ἕνδεκα δέῤῥεσι.
\VS{9}Καὶ συνάψεις τὰς πέντε δέῤῥεις ἐπὶ τὸ αὐτὸ, καὶ τὰς ἓξ δέῤῥεις ἐπὶ τὸ αὐτό· καὶ ἐπιδιπλώσεις τὴν δέῤῥιν τὴν ἕκτην κατὰ πρόσωπον τῆς σκηνῆς.
\VS{10}Καὶ ποιήσεις ἀγκύλας πεντήκοντα ἐπὶ τοῦ χείλους τῆς δέῤῥεως τῆς μιᾶς, τῆς ἀναμέσον κατὰ συμβολήν· καὶ πεντήκοντα ἀγκύλας ποιήσεις ἐπὶ τοῦ χείλους τῆς δέῤῥεως, τῆς συναπτούσης τῆς δευτέρας.
\par }{\PP \VS{11}Καὶ ποιήσεις κρίκους χαλκοῦς πεντήκοντα· καὶ συνάψεις τοὺς κρίκους ἐκ τῶν ἀγκυλῶν, καὶ συνάψεις τὰς δέῤῥεις, καὶ ἔσται ἕν.
\VS{12}Καὶ ὑποθήσεις τὸ πλεονάζον ἐν ταῖς δέῤῥεσι τῆς σκηνῆς· τὸ ἥμισυ τῆς δέῤῥεως τὸ ὑπολελειμμένον ὑποκαλύψεις εἰς τὸ πλεονάζον τῶν δέῤῥεων τῆς σκηνῆς, ὑποκαλύψεις ὀπίσω τῆς σκηνῆς.
\VS{13}Πῆχυν ἐκ τούτου, καὶ πῆχυν ἐκ τούτου, ἐκ τοῦ ὑπερέχοντος τῶν δέῤῥεων, ἐκ τοῦ μήκους τῶν δέῤῥεων τῆς σκηνῆς· ἔσται συγκαλύπτον ἐπὶ τὰ πλάγια τῆς σκηνῆς ἔνθεν καὶ ἔνθεν, ἵνα καλύπτῃ.
\VS{14}Καὶ ποιήσεις κατακάλυμμα τῇ σκηνῇ δέρματα κριῶν ἠρυθροδανωμένα, καὶ ἐπικαλύμματα δέρματα ὑακίνθινα ἐπάνωθεν.
\par }{\PP \VS{15}Καὶ ποιήσεις στύλους τῆς σκηνῆς ἐκ ξύλων ἀσήπτων.
\VS{16}Δέκα πήχεων ποιήσεις τὸν στύλον τὸν ἕνα, καὶ πήχεως ἑνὸς καὶ ἡμίσους τὸ πλάτος τοῦ στύλου τοῦ ἑνός.
\VS{17}Δύο ἀγκωνίσκους τῷ στύλῳ τῷ ἑνὶ, ἀντιπίπτοντας ἕτερον τῷ ἑτέρῳ· οὕτω ποιήσεις πᾶσι τοῖς στύλοις τῆς σκηνῆς.
\VS{18}Καὶ ποιήσεις στύλους τῇ σκηνῇ, εἴκοσι στύλους ἐκ τοῦ κλίτους τοῦ πρὸς Βοῤῥᾶν.
\VS{19}Καὶ τεσσαράκοντα βάσεις ἀργυρᾶς ποιήσεις τοῖς εἴκοσι στύλοις· δύο βάσεις τῷ στύλῳ τῷ ἑνὶ εἰς ἀμφότερα τὰ μέρη αὐτοῦ· και δύο βάσεις τῷ στύλῳ τῷ ἑνὶ εἰς ἀμφοτέρα τὰ μέρη αὐτοῦ.
\VS{20}Καὶ τὸ κλίτος τὸ δεύτερον τὸ πρὸς Νότον, εἴκοσι στύλους,
\VS{21}καὶ τεσσαράκοντα βάσεις αὐτῶν ἀργυρᾶς· δύο βάσεις τῷ στύλῳ τῷ ἑνὶ εἰς ἀμφότερα τὰ μέρη αὐτοῦ, καὶ δύο βάσεις τῷ στύλῳ τῷ ἑνὶ εἰς ἀμφότερα τὰ μέρη αὐτοῦ.
\VS{22}Καὶ ἐκ τῶν ὀπίσω τῆς σκηνῆς κατὰ τὸ μέρος τὸ πρὸς θάλασσαν ποιήσεις ἓξ στύλους.
\VS{23}Καὶ δύο στύλους ποιήσεις ἐπὶ τῶν γωνιῶν τῆς σκηνῆς ἐκ τῶν ὀπισθίων.
\VS{24}Καὶ ἔσται ἐξ ἴσου κάτωθεν· κατὰ τὸ αὐτὸ ἔσονται ἴσοι ἐκ τῶν κεφαλῶν εἰς σύμβλησιν μίαν· οὕτω ποιήσεῖς ἀμφοτέραις ταῖς δυσὶ γωνίαις· ἴσαι ἔστωσαν.
\VS{25}Καὶ ἔσονται ὀκτὼ στύλοι, καὶ αἱ βάσεις αὐτῶν ἀργυραῖ δεκαέξ· δύο βάσεις τῷ ἑνὶ στύλῳ εἰς ἀμφότερα τὰ μέρη αὐτοῦ, καὶ δύο βάσεις τῷ στύλῳ τῷ ἑνί.
\VS{26}Καὶ ποιήσεις μοχλοὺς ἐκ ξύλων ἀσήπτων· πέντε τῷ ἑνὶ στύλῳ ἐκ τοῦ ἑνὸς μέρους τῆς σκηνῆς,
\VS{27}καὶ πέντε μοχλοὺς τῷ στύλῳ τῷ ἑνὶ κλίτει τῆς σκηνῆς τῷ δευτέρῳ, καὶ πέντε μοχλοὺς τῷ στύλῳ τῷ ὀπισθίῳ τῷ κλίτει τῆς σκηνῆς τῷ πρὸς θάλασσαν.
\VS{28}Καὶ ὁ μοχλὸς ὁ μέσος ἀναμέσον τῶν στύλων διϊκνείσθω ἀπὸ τοῦ ἑνὸς κλίτους εἰς τὸ ἕτερον κλίτος.
\VS{29}Καὶ τοὺς στύλους καταχρυσώσεις χρυσίῳ· καὶ τοὺς δακτυλίους ποιήσεις χρυσοῦς, εἰς οὓς εἰσάξεις τούς μοχλούς· καὶ καταχρυσώσεις τοὺς μοχλοὺς χρυσίῳ.
\VS{30}Καὶ ἀναστήσεις τὴν σκηνὴν κατὰ τὸ εἶδος τὸ δεδειγμένον σοι ἐν τῷ ὄρει.
\par }{\PP \VS{31}Καὶ ποιήσεις καταπέτασμα ἐξ ὑακίνθου, καὶ πορφύρας, καὶ κοκκίνου κεκλωσμένου, καὶ βύσσου νενησμένης· ἔργον ὑφαντὸν ποιήσεις αὐτὸ χερουβίμ.
\VS{32}Καὶ ἐπιθήσεις αὐτὸ ἐπὶ τεσσάρων στύλων ἀσήπτων κεχρυσωμένων χρυσίῳ· καὶ αἱ κεφαλίδες αὐτῶν χρυσαῖ, καὶ αἱ βάσεις αὐτῶν τέσσαρες ἀργυραῖ.
\VS{33}Καὶ θήσεις τὸ καταπέτασμα ἐπὶ τῶν στύλων· καὶ εἰσοίσεις ἐκεῖ ἐσώτερον τοῦ καταπετάσματος τὴν κιβωτὸν τοῦ μαρτυρίου· καὶ διοριεῖ τὸ καταπέτασμα ὑμῖν ἀναμέσον τοῦ ἁγίου καὶ ἀναμέσον τοῦ ἁγίου τῶν ἁγίων.
\VS{34}Καὶ κατακαλύψεις τῷ καταπετάσματι τὴν κιβωτὸν τοῦ μαρτυρίου ἐν τῷ ἁγίῳ τῶν ἁγίων.
\VS{35}Καὶ ἐπιθήσεις τὴν τράπεζαν ἔξωθεν τοῦ καταπετάσματος, καὶ τὴν λυχνίαν ἀπέναντι τῆς τραπέζης ἐπὶ μέρους τῆς σκηνῆς τὸ πρὸς Νότον· καὶ τὴν τράπεζαν θήσεις ἐπὶ μέρους τῆς σκηνῆς τὸ πρὸς Βοῤῥᾶν.
\VS{36}Καὶ ποιήσεις ἐπίσπαστρον τῇ θύρᾳ τῆς σκηνῆς ἐξ ὑακίνθου, καὶ πορφύρας, καὶ κοκκίνου κεκλωσμένου, καὶ βύσσου κεκλωσμένης, ἔργον ποικιλτοῦ.
\VS{37}Καὶ ποιήσεις τῷ καταπετάσματι πέντε στύλους, καὶ χρυσώσεις αὐτοὺς χρυσίῳ· καὶ αἱ κεφαλίδες αὐτῶν χρυσαῖ· καὶ χωνεύσεις αὐτοῖς πέντε βάσεις χαλκᾶς.

\par }\Chap{27}{\PP \VerseOne{1}Καὶ ποιήσεις θυσιαστήριον ἐκ ξύλων ἀσήπτων, πέντε πήχεων τὸ μῆκος, καὶ πέντε πήχεων τὸ εὖρος· τετράγωνον ἔσται τὸ θυσιαστήριον, καὶ τριῶν πήχεων τὸ ὕψος αὐτοῦ.
\VS{2}Καὶ ποιήσεις τὰ κέρατα ἐπὶ τῶν τεσσάρων γωνιῶν· ἐξ αὐτοῦ ἔσται τὰ κέρατα, καὶ καλύψεις αὐτὰ χαλκῷ.
\VS{3}Καὶ ποιήσεις στεφάνην τῷ θυσιαστηρίῳ· καὶ τὸν καλυπτῆρα αὐτοῦ, καὶ τὰς φιάλας αὐτοῦ, καὶ τὰς κρεάγρας αὐτοῦ, καὶ τὸ πυρεῖον αὐτοῦ, καὶ πάντα τὰ σκεύη αὐτοῦ ποιήσεις χαλκᾶ.
\VS{4}Καὶ ποιήσεις αὐτῷ ἐσχάραν ἔργῳ δικτυωτῷ χαλκῆν· καὶ ποιήσεις τῇ ἐσχάρᾳ τέσσαρες δακτυλίους χαλκοῦς ὑπὸ τὰ τέσσαρα κλίτη.
\VS{5}Καὶ ὑποθήσεις αὐτοὺς ὑπὸ τὴν ἐσχάραν τοῦ θυσιαστήριου κάτωθεν· ἔσται δὲ ἡ ἐσχάρα ἕως τοῦ ἡμίσους τοῦ θυσιαστηρίου.
\VS{6}Καὶ ποιήσεις τῷ θυσιαστηρίῳ ἀναφορεῖς ἐκ ξύλων ἀσήπτων, καὶ περιχαλκώσεις αὐτοὺς χαλκῷ.
\VS{7}Καὶ εἰσάξεις τοὺς ἀναφορεῖς εἰς τοὺς δακτυλίους· καὶ ἔστωσαν ἀναφορεῖς κατὰ πλευρὰ τοῦ θυσιαστηρίου ἐν τῷ αἴρειν αὐτό.
\VS{8}Κοῖλον συνιδωτὸν ποιήσεις αὐτό· κατὰ τὸ παραδειχθέν σοι ἐν τῷ ὄρει, οὕτω ποιήσεις αὐτό.
\VS{9}Καὶ ποιήσεις αὐλὴν τῇ σκηνῇ· εἰς τὸ κλίτος τὸ πρὸς Λίβα ἱστία τῆς αὐλῆς ἐκ βύσσου κεκλωσμένης· μῆκος ἑκατὸν πήχεων τῷ ἑνὶ κλίτει.
\VS{10}Καὶ οἱ στύλοι αὐτῶν εἴκοσι, καὶ αἱ βάσεις αὐτῶν εἴκοσι χαλκαῖ, καὶ οἱ κρίκοι αὐτῶν καὶ αἱ ψαλίδες ἀργυραῖ.
\VS{11}Οὕτως τῷ κλίτει τῷ πρὸς ἀπηλιώτην ἱστία ἑκατὸν πήχεων μῆκος· καὶ οἱ στύλοι αὐτῶν εἴκοσι, καὶ αἱ βάσεις αὐτῶν εἴκοσι χαλκαῖ· καὶ οἱ κρίκοι καὶ αἱ ψαλίδες τῶν στύλων, καὶ αἱ βάσεις αὐτῶν περιηργυρωμέναι ἀργυρίῳ.
\VS{12}Τὸ δὲ εὖρος τῆς αὐλῆς τὸ κατὰ θάλασσαν ἱστία πεντήκοντα πήχεων· στύλοι αὐτῶν δέκα, καὶ βάσεις αὐτῶν δέκα.
\VS{13}Καὶ εὖρος τῆς αὐλῆς τῆς πρὸς Νότον ἱστία πεντήκοντα πήχεων· στύλοι αὐτῶν δέκα, καὶ βάσεις αὐτῶν δέκα.
\VS{14}Καὶ πεντεκαίδεκα πήχεων τὸ ὕψος τῶν ἱστίων τῷ κλίτει τῷ ἑνί· στύλοι αὐτῶν τρεῖς, καὶ αἱ βάσεις αὐτῶν τρεῖς.
\VS{15}Καὶ τὸ κλίτος τὸ δεύτερον δεκαπέντε πήχεων τῶν ἱστίων τὸ ὕψος· στύλοι αὐτῶν τρεῖς, καὶ αἱ βάσεις αὐτῶν τρεῖς.
\VS{16}Καὶ τῇ πύλῃ τῆς αὐλῆς κάλυμμα· εἴκοσι πήχεων τὸ ὕψος ἐξ ὑακίνθου, καὶ πορφύρας, καὶ κοκκίνου κεκλωσμένου, καὶ βύσσου κεκλωσμένης τῇ ποικιλίᾳ τοῦ ῥαφιδευτοῦ· στύλοι αὐτῶν τέσσαρες, καὶ αἱ βάσεις αὐτῶν τέσσαρες.
\VS{17}Πάντες οἱ στύλοι τῆς αὐλῆς κύκλῳ κατηργυρωμένοι ἀργυρίῳ, καὶ αἱ κεφαλίδες αὐτῶν ἀργυραῖ, καὶ αἱ βάσεις αὐτῶν χαλκαῖ.
\VS{18}Τὸ δὲ μῆκος τῆς αὐλῆς ἑκατὸν ἐφʼ ἑκατόν· καὶ εὖρος πεντήκοντα ἐπὶ πεντήκοντα· καὶ ὕψος πέντε πήχεῶν ἐκ βύσσου κεκλωσμένης, καὶ βάσεις αὐτῶν χαλκαῖ.
\VS{19}Καὶ πᾶσα ἡ κατασκευὴ καὶ πάντα τὰ ἐργαλεῖα καὶ οἱ πάσσαλοι τῆς αὐλῆς χαλκοῖ.
\par }{\PP \VS{20}Καὶ σὺ σύνταξον τοῖς υἱοῖς Ἰσραὴλ, καὶ λαβέτωσάν σοι ἔλαιον ἐξ ἐλαιῶν ἀτρυγον καθαρὸν κεκομμένον εἰς φῶς καῦσαι, ἵνα καίηται λύχνος διαπαντός
\VS{21}ἐν τῇ σκηνῇ τοῦ μαρτυρίου· ἔξωθεν τοῦ καταπετάσματος τοῦ ἐπὶ τῆς διαθήκης καύσει αὐτὸ Ἀαρὼν καὶ οἱ υἱοὶ αὐτοῦ ἀφʼ ἑσπέρας ἕως πρωῒ, ἐναντίον Κυρίου, νόμιμον αἰώνιον εἰς τὰς γενεὰς ὑμῶν παρὰ τῶν υἱῶν Ἰσραήλ.

\par }\Chap{28}{\PP \VerseOne{1}Καὶ σὺ προσαγάγου πρὸς σεαυτὸν τόν τε Ἀαρὼν τὸν ἀδελφόν σου, καὶ τοὺς υἱοὺς αὐτοῦ, καὶ ἐκ τῶν υἱῶν Ἰσραὴλ, ἱερατεύειν μοι Ἀαρὼν, καὶ Ναδὰβ, καὶ Ἀβιοὺδ, καὶ Ἐλεάζαρ, καὶ Ἰθάμαρ, υἱοὺς Ἀαρών.
\VS{2}Καὶ ποιήσεις στολὴν ἁγίαν Ἀαρὼν τῷ ἀδελφῷ σου εἰς τιμὴν καὶ δόξαν.
\VS{3}Καὶ σύ λάλησον πᾶσι τοῖς σοφοῖς τῇ διανοίᾳ, οὓς ἐνέπλησα πνεύματος σοφίας καὶ αἰσθήσεως· καὶ ποιήσουσι τὴν στολὴν τὴν ἁγίαν Ἀαρὼν εἰς τὸ ἅγιον, ἐν ᾗ ἱερατεύσει μοι.
\VS{4}Καὶ αὗται αἱ στολαὶ, ἃς ποιησουσι· τὸ περιστήθιον, καὶ τὴν ἐπωμίδα, καὶ τὸν ποδήρη, καὶ χιτῶνα κοσυμβωτὸν, καὶ κίδαριν, καὶ ζώνην· καὶ ποιήσουσι στολὰς ἁγίας Ἀαρὼν καὶ τοῖς υἱοῖς αὐτοῦ εἰς τὸ ἱερατεύειν μοι.
\VS{5}Καὶ αὐτοὶ λήψονται τὸ χρυσίον, καὶ τὸν ὑάκινθον, καὶ τὴν πορφύραν, καὶ τὸ κόκκινον, καὶ τὴν βύσσον.
\VS{6}Καὶ ποιήσουσι τὴν ἐπωμίδα ἐκ βύσσου κεκλωσμένης, ἔργον ὑφαντὸν ποικιλτοῦ.
\VS{7}Δύο ἐπωμίδες συνέχουσαι ἔσονται αὐτῷ ἑτέρα τὴν ἑτέραν, ἐπὶ τοῖς δυσὶ μέρεσιν ἐξηρτισμέναι.
\VS{8}Καὶ τὸ ὕφασμα τῶν ἐπωμίδων ὅ ἐστιν ἐπʼ αὐτῷ, κατὰ τὴν ποίησιν ἐξ αὐτοῦ ἔσται ἐκ χρυσίου καθαροῦ, καὶ ὑακίνθου, καὶ πορφύρας, καὶ κοκκίνου διανενησμένου, καὶ βύσσου κεκλωσμένης.
\VS{9}Καὶ λήψῃ τοὺς δύο λίθους, λίθους σμαράγδου, καὶ γλύψεις ἐν αὐτοῖς τὰ ὀνόματα τῶν υἱῶν Ἰσραήλ.
\VS{10}Ἓξ ὀνόματα ἐπὶ τὸν λίθον τὸν ἕνα, καὶ τὰ ἓξ ὀνόματα τὰ λοιπὰ ἐπὶ τὸν λίθον τὸν δεύτερον κατὰ τὰς γενέσεις αὐτῶν.
\VS{11}Ἔργον λιθουργικῆς τέχνης· γλύμμα σφραγίδος διαγλύψεις τοὺς δύο λίθους ἐπὶ τοῖς ὀνόμασι τῶς υἱῶν Ἰσρσήλ.
\VS{12}Καὶ θήσεις τοὺς δύο λίθους ἐπὶ τῶς ὤμων τῆς ἐπωμίδος· λίθοι μνημοσύνου εἰσὶ τοῖς υἱοῖς Ἰσραήλ· καὶ ἀναλήψεται Ἀαρὼν τὰ ὀνόματα τῶν υἱῶν Ἰσραὴλ ἔναντι Κυρίου ἐπὶ τῶν δύο ὤμων αὐτοῦ, μνημόσυνον πεπὶ αὐτῶν.
\VS{13}Καὶ ποιήσεις ἀσπιδίσκας ἐκ χρυσίου καθαροῦ.
\VS{14}Καὶ ποιήσεις δύο κροσωτὰ ἐκ χρυσίου καθαροῦ, καταμεμιγμένα ἐν ἄνθεσιν, ἔργον πλοκῆς· καὶ ἐπιθήσεις τὰ κροσσωτὰ τὰ πεπλεγμένα ἐπὶ τὰς ἀσπιδίσκας, κατὰ τὰς παρωμίδας αὐτῶν ἐκ τῶν ἐμπροσθίων.
\par }{\PP \VS{15}Καὶ ποιήσεις λογεῖον τῶν κρίσεων, ἔργον ποικιλτοῦ· κατὰ τὸν ῥυθμὸν τῆς ἐπωμίδος ποιήσεις αὐτὸ ἐκ χρυσίου, καὶ ὑακίνθου, καὶ πορφύρας, καὶ κοκκίνου κεκλωσμένου, καὶ βύσσου κεκλωσμένης.
\VS{16}Ποιήσεις αὐτό τετράγωνον· ἔσται διπλοῦν, σπιθαμῆς τὸ μῆκος αὐτοῦ, καὶ σπιθαμῆς τὸ εὖρος.
\VS{17}Καὶ καθυφανεῖς ἐν αὐτῷ ὕφασμα κατάλιθον τετράστιχον· στίχος λίθων ἔσται, σάρδιον, τοπάζιον, καὶ σμαράγδος, ὁ στίχος ὁ εἷς.
\VS{18}Καὶ ὁ στίχος ὁ δεύτερος, ἄνθραξ, καὶ σάπφειρος, καὶ ἴασπις.
\VS{19}Καὶ ὁ στίχος ὁ τρίτος, λιγύριον, ἀχάτης, ἀμέθυστος.
\VS{20}Καὶ ὁ στίχος ὁ τέταρτος, χρυσόλιθος, καὶ βηρύλλιον, καὶ ὀνύχιον, περικεκαλυμμένα χρυσίῳ, συνδεδεμένα ἐν χρυσίῳ· ἔστωσαν κατὰ στίχον αὐτῶν.
\VS{21}Καὶ οἱ λίθοι ἔστωσαν ἐκ τῶν ὀνομάτων τῶν υἱῶν Ἰσραὴλ δεκαδύο κατὰ τὰ ὀνόματα αὐτῶν· γλυφαὶ σφραγίδων, ἕκαστος κατὰ τὸ ὄνομα ἔστωσαν εἰς δεκαδύο φυλάς.
\VS{22}Καὶ ποιήσεις ἐπὶ τὸ λογιον κρωσσοὺς συμπεπλεγμένους, ἔργον ἁλυσιδωτὸν ἐκ χρυσίου καθαροῦ.
\VS{29}Καὶ λήψεται Ἀαρὼν τὰ ὀνόματα τῶν υἱῶν Ἰσραὴλ ἐπὶ τοῦ λογείου τῆς κρίσεως ἐπὶ τοῦ στήθους, εἰσιόντι εἰς τὸ ἅγιον μνημόσυνου ἐναντίον τοῦ Θεοῦ.
\VS{29a}Καὶ θήσεις ἐπὶ τὸ λογεῖον τῆς κρίσεως τοὺς κρωσσούς· τὰ ἁλυσιδωτὰ ἐπʼ ἀμφοτέρων τῶν κλιτῶν τοῦ λογείου ἐπιθήσεις. Καὶ τὰς δύο ἀσπιδίσκας ἐπιθήσεις ἐπʼ ἀμφοτέρους τοὺς ὤμους τῆς ἐπωμίδος κατὰ πρόσωπον.
\VS{30}Καὶ ἐπιθήσεις ἐπὶ τὸ λογεῖον τῆς κρίσεως τὴν δήλωσιν καὶ τὴν ἀλήθειαν· καὶ ἔσται ἐπὶ τοῦ στήθους Ἀαρὼν, ὃταν εἰσπορεύεται εἰς τὸ ἅγιον ἔναντὶ Κυρίου· καὶ οἴσει Ἀαρὼν τὰς κρίσεις τῶν υἱῶν Ἰσραὴλ ἐπὶ τοῦ στήθους ἔναντι Κυρίου διαπαντός.
\VS{31}Καὶ ποιήσεις ὑποδύτην ποδήρη ὅλον ὑακίνθινον.
\VS{32}Καὶ ἔσται τὸ περιστόμιον ἐξ αὐτοῦ μέσον, ὤαν ἔχον κύκλῳ τοῦ περιστομίου, ἔργον ὑφαντου, τὴν συμβολὴν συνυφασμένην ἐξ αὐτοῦ, ἵνα μὴ ῥαγῇ.
\VS{33}Καὶ ποιήσεις ὑπὸ τὸ λῶμα τοῦ ὑποδύτου κάτωθεν, ὡσεὶ ἐξανθούσης ῥόας ῥοΐσκους ἐξ ὑακίνθου, καὶ πορφύρας, καὶ κοκκίνου διανενησμένου, καὶ βύσσου κεκλωσμένης, ὑπὸ τοῦ λώματος τοῦ ὑποδύτου κύκλῳ· τὸ αὐτὸ εἶδος ῥοΐσκους χρυσοῦς, καὶ κώδωνας ἀναμέσον τούτων περικύκλῳ.
\VS{34}Παρὰ ῥοΐσκον χρυσοῦν δώδωνα, καὶ ἄνθινον ἐπὶ τοῦ λώματος τοῦ ὑποδύτου κύκλῳ·
\VS{35}Καὶ ἔσται Ἀαρὼν ἐν τῷ λειτουργεῖν ἀκουστὴ ἡ φωνὴ αὐτοῦ, εἰσιόντι εἰς τὸ ἅγιον ἔναντι Κυρίου, καὶ ἐξιόντι, ἵνα μὴ ἀποθάνῃ.
\VS{36}Καὶ ποιήσεις πέταλον χρυσοῦν καθαρόν· καὶ ἐκτυπώσεις ἐν αὐτῷ ἐκτύπωμα σφραγίδος, Ἁγίασμα Κυρίου.
\VS{37}Καὶ ἐπιθήσεις αὐτὸ ἐπὶ ὑακίνθου κεκλωσμένης· καὶ ἔσται ἐπὶ τῆς μίτρας, κατὰ πρόσωπον τῆς μίτρας ἔσται.
\VS{38}Καὶ ἔσται ἐπὶ τοῦ μετώπου Ἀαρών· καὶ ἐξαρεῖ Ἀαρὼν τὰ ἁμαρτήματα τῶν ἁγίων, ὅσα ἂν ἁγιάσωσιν οἱ υἱοὶ Ἰσραὴλ παντὸς δόματος τῶν ἁγίων αὐτῶν· καὶ ἔσται ἐπὶ τοῦ μετώπου Ἀαρὼν διαπαντὸς δεκτὸν αὐτοῖς ἔναντι Κυρίου.
\par }{\PP \VS{39}Καὶ οἱ κοσυμβωτοὶ τῶν χιτώνων ἐκ βύσσου· καὶ ποιήσεις κίδαριν βυσσίνην· καὶ ζώνην ποιήσεις, ἔργον ποικιλτοῦ.
\VS{40}Καὶ τοῖς υἱοῖς Ἀαρὼν ποιήσεις χιτῶνας καὶ ζώνας, καὶ κιδάρεις ποιήσεις αὐτοῖς εἰς τιμὴν καὶ δόξαν.
\VS{41}Καὶ ἐνδύσεις αὐτὰ Ἀαρὼν τὸν ἀδελφόν σου, καὶ τοὺς υἱοὺς αὐτοῦ μετʼ αὐτοῦ· καὶ χρίσεις αὐτοὺς, καὶ ἐμπλήσεις αὐτῶν τὰς χεῖρας· καὶ ἁγιάσεις αὐτοὺς, ἵνα ἱερατεύωσί μοι.
\VS{42}Καὶ ποιήσεις αὐτοῖς περισκελῆ λινᾶ καλύψαι ἀσχημοσύνην χρωτὸς αὐτῶν, ἀπὸ ὀσφύος ἕως μηρῶν ἔσται.
\VS{43}Καὶ ἕξει Ἀαρὼν αὐτὰ καὶ οἱ υἱοὶ αὐτοῦ, ὅταν εἰσπορεύωνται εἰς τὴν σκηνὴν τοῦ μαρτυρίου, ἢ ὅταν προσπορεύωνται λειτουργεῖν πρὸς τὸ θυσιαστήριον τοῦ ἁγίου· καὶ οὐκ ἐπάξονται πρὸς ἑαυτοὺς ἁμαρτίαν, ἵνα μὴ ἀποθάνωσι· νόμιμον αἰώνιον αὐτῷ, καὶ τῷ σπέρματι αὐτοῦ μετʼ αὐτόν.

\par }\Chap{29}{\PP \VerseOne{1}Καὶ ταῦτά ἐστιν, ἃ ποιήσεις αὐτοῖς· ἁγιάσεις αὐτοὺς, ὥστε ἱερατεύειν μοι αὐτούς· λήψῃ δὲ μοσχάριον ἐκ βοῶν ἓν, καὶ κριοὺς ἀμώμους δύο,
\VS{2}καὶ ἄρτους ἀζύμους πεφυραμένους ἑν ἐλαίῳ, καὶ λάγανα ἄζυμα κεχρισμένα ἐν ἐλαίῳ· σεμίδαλιν ἐκ πυρῶν ποιήσεις αὐτά.
\VS{3}Καὶ ἐπιθήσεις αὐτὰ ἐπὶ κανοῦν ἕν· καὶ προσοίσεις αὐτὰ ἐπὶ τῷ κανῷ· καὶ τὸ μοσχάριον, καὶ τοὺς δύο κριούς.
\VS{4}Καὶ Ἀαρὼν καὶ τοὺς υἱοὺς αὐτοῦ προσάξεις ἐπὶ τὰς θύρας τῆς σκηνῆς τοῦ μαρτυρίου, καὶ λούσεις αὐτοὺς ἐν ὕδατι.
\VS{5}Καὶ λαβὼν τὰς στολὰς, ἐνδύσεις Ἀαρὼν τὸν ἀδελφόν σου καὶ τὸν χιτῶνα τὸν ποδήρη, καὶ τὴν ἐπωμίδα, καὶ τὸ λογεῖον· καὶ συνάψεις αὐτῷ τὸ λογεῖον πρὸς τὴν ἐπωμίδα.
\VS{6}Καὶ ἐπιθήσεις τὴν μίτραν ἐπὶ τὴν κεφαλὴν αὐτοῦ, καὶ ἐπιθήσεις τὸ πέταλον τὸ ἁγίασμα ἐπὶ τὴν μίτραν.
\VS{7}Καὶ λήψῃ τοῦ ἐλαίου τοῦ χρίσματος· καὶ ἐπιχεεῖς αὐτὸ ἐπὶ τὴν κεφαλὴν αὐτοῦ, καὶ χρίσεις αὐτόν.
\VS{8}Καὶ τοὺς υἱοὺς αὐτοῦ προσάξεις, καὶ ἐνδύσεις αὐτοὺς χιτῶνας.
\VS{9}Καὶ ζώσεις αὐτοὺς ταῖς ζωναῖς, καὶ περιθήσεις αὐτοῖς τὰς κιδάρεις· καὶ ἔσται αὐτοῖς ἱερατῖα μοι εἰς τὸν αἰῶνα· καὶ τελειώσεις Ἀαρὼν τὰς χεῖρας αὐτοῦ, καὶ τὰς χεῖρας τῶν υἱῶν αὐτοῦ.
\VS{10}Καὶ προσάξεις τὸν μόσχον ἐπὶ τὰς θύρας τῆς σκηνῆς τοῦ μαρτυρίου· καὶ ἐπιθήσουσιν Ἀαρὼν καὶ οἱ υἱοὶ αὐτοῦ τὰς χεῖρας αὐτῶν ἐπὶ τὴν κεφαλὴν τοῦ μόσχου, ἔναντι Κυρίου, παρὰ τὰς θύρας τῆς σκηνῆς τοῦ μαρτυρίου.
\VS{11}Καὶ σφάξεις τὸν μόσχον ἔναντι Κυρίου, παρὰ τὰς θύρας τῆς σκηνῆς τοῦ μαρτυρίου.
\VS{12}Καὶ λήψῃ ἀπὸ τοῦ αἵματος τοῦ μόσχου, καὶ θήσεις ἐπὶ τῶν κεράτων τοῦ θυσιαστηρίου τῷ δακτύλῳ σου· τὸ δὲ λοιπὸν πᾶν αἷμα ἐκχεεῖς παρὰ τὴν βάσιν τοῦ θυσιαστηρίου.
\VS{13}Καὶ λήψῃ πᾶν τὸ στέαρ τὸ ἐπὶ τῆς κοιλίας, καὶ τὸν λοβὸν τοῦ ἥπατος, καὶ τοὺς δύο νεφροὺς, καὶ τὸ στέαρ τὸ ἐπʼ αὐτῶν, καὶ ἐπιθήσεις ἐπὶ τὸ θυσιαστήριον.
\VS{14}Τὰ δὲ κρέατα τοῦ μόσχου, καὶ τὸ δέρμα, καὶ τὴν κόπρον κατακαύσεις πυρὶ ἔξω τῆς παρεμβολῆς· ἁμαρτίας γάρ ἐστι.
\par }{\PP \VS{15}Καὶ τὸν κριὸν λήψῃ τὸν ἕνα, καὶ ἐπιθήσουσιν Ἀαρὼν καὶ οἱ υἱοὶ αὐτοῦ τὰς χεῖρας αὐτῶν ἐπὶ τὴν κεφαλὴν τοῦ κριοῦ.
\VS{16}Καὶ σφάξεις αὐτὸν, καὶ λαβὼν τὸ αἷμα προσχεεῖς πρὸς τὸ θυσιαστήριον κύκλῳ.
\VS{17}Καὶ τὸν κριὸν διχοτομήσεις κατὰ μέλη· καὶ πλυνεῖς τὰ ἐνδόσθια καὶ τοὺς πόδας ὕδατι, καὶ ἐπιθήσεις ἐπὶ τὰ διχοτομήματα σὺν τῇ κεφαλῇ.
\VS{18}Καὶ ἀνοίσεις ὅλον τὸν κριὸν ἐπὶ τὸ θυσιαστήριον, ὁλοκαύτωμα τῷ Κυρίῳ εἰς ὀσμὴν εὐωδίας· θυμίαμα Κυρίῳ ἐστί.
\VS{19}Καὶ λήψῃ τὸν κριὸν τὸν δεύτερον, καὶ ἐπιθήσει Ἀαρὼν καὶ οἱ υἱοὶ αὐτοῦ τὰς χεῖρας αὐτῶν ἐπὶ τὴν κεφαλὴν τοῦ κριοῦ.
\VS{20}Καὶ σφάξεις αὐτὸν, καὶ λήψῃ τοῦ αἵματος αὐτοῦ, καὶ ἐπιθήσεις ἐπὶ τὸν λοβὸν τοῦ ὠτὸς Ἀαρὼν τοῦ δεξιοῦ, καὶ ἐπὶ τὸ ἄκρον τῆς δεξιᾶς χειρὸς, καὶ ἐπὶ τὸ ἄκρον τοῦ ποδὸς τοῦ δεξιοῦ, καὶ ἐπὶ τοὺς λοβοὺς τῶν ὤτων τῶν υἱῶν αὐτοῦ τῶν δεξιῶν, καὶ ἐπὶ τὰ ἄκρα τῶν χειρῶν αὐτῶν τῶν δεξιῶν, καὶ ἐπὶ τὰ ἄκρα τῶν ποδῶν αὐτῶν τῶν δεξιῶν.
\VS{21}Καὶ λήψῃ ἀπὸ τοῦ αἵματος τοῦ ἀπὸ τοῦ θυσιαστηρίου, καὶ ἀπὸ τοῦ ἐλαίου τῆς χρίσεως, καὶ ῥανεῖς ἐπὶ Ἀαρὼν καὶ ἐπὶ τὴν στολὴν αὐτοῦ, καὶ ἐπὶ τοὺς υἱοὺς αὐτοῦ καὶ ἐπὶ τὰς στολὰς τῶν υἱῶν αὐτοῦ μετʼ αὐτοῦ· καὶ ἁγιασθήσεται αὐτὸς καὶ ἡ στολὴ αὐτοῦ, καὶ οἱ υἱοὶ αὐτοῦ καὶ αἱ στολαὶ τῶν υἱῶν αὐτοῦ μετʼ αὐτοῦ· τὸ δὲ αἷμα τοῦ κριοῦ προσχεεῖς πρὸς τὸ θυσιαστήριον κύκλῳ.
\VS{22}Καὶ λήψῃ ἀπὸ τοῦ κριοῦ τὸ στέαρ αὐτοῦ, καὶ τὸ στέαρ τὸ κατακαλύπτον τὴν κοιλίαν, καὶ τὸν λοβὸν τοῦ ἥπατος, καὶ τοὺς δύο νεφροὺς, καὶ τὸ στέαρ τὸ ἐπʼ αὐτῶν, καὶ τὸν βραχίονα τὸν δεξιόν· ἔστι γὰρ τελείωσις αὕτη.
\VS{23}Καὶ ἄρτον ἕνα ἐξ ἐλαίου, καὶ λάγανον ἓν ἀπὸ τοῦ κανοῦ τῶν ἀζύμων τῶν προτεθειμένων ἔναντι Κυρίου.
\VS{24}Καὶ ἐπιθήσεις τὰ πάντα ἐπὶ τὰς χεῖρας Ἀαρὼν, καὶ ἐπὶ τὰς χεῖρας τῶν υἱῶν αὐτοῦ· καὶ ἀφοριεῖς αὐτὰ ἀφόρισμα ἔναντι Κυρίου.
\VS{25}Καὶ λήψῃ αὐτὰ ἐκ τῶν χειρῶν αὐτῶν, καὶ ἀνοίσεις ἐπὶ τὸ θυσιαστήριον τῆς ὁλοκαυτώσεως εἰς ὀσμὴν εὐωδίας ἔναντι Κύριου· κάρπωμά ἐστι Κυρίῳ.
\VS{26}Καὶ λήψῃ τὸ στηθύνιον ἀπὸ τοῦ κριοῦ τῆς τελειώσεως, ὅ ἐστιν Ἀαρών· καὶ ἀφοριεῖς αὐτὸ ἀφόρισμα ἔναντι Κυρίου· καὶ ἔσται σοι ἐν μερίδι.
\VS{27}Καὶ ἁγιάσεις τὸ στηθύνιον ἀφόρισμα, καὶ τὸν βραχίονα τοῦ ἀφαιρέματος, ὃς ἀφώρισται, καὶ ὃς ἀφῄρηται ἀπὸ τοῦ κριοῦ τῆς τελειώσεως ἀπὸ τοῦ Ἀαρὼν, καὶ ἀπὸ τῶν υἱῶν αὐτοῦ.
\VS{28}Καὶ ἔσται Ἀαρὼν καὶ τοῖς υἱοῖς αὐτοῦ νόμιμον αἰώνιον παρὰ τῶν υἱῶν Ἰσραήλ· ἔστι γὰρ ἀφόρισμα τοῦτο· καὶ ἀφαίρεμα ἕσται παρὰ τῶν υἱῶν Ἰσραὴλ ἀπὸ τῶν θυμάτων τῶν σωτηρίων τῶν υἱῶν Ἰσραὴλ, ἀφαίρεμα Κυρίῳ.
\par }{\PP \VS{29}Καὶ ἡ στολὴ τοῦ ἁγίου, ἥ ἐστιν Ἀαρὼν, ἔσται τοῖς υἱοῖς αὐτοῦ μετʼ αὐτὸν, χρισθῆναι αὐτοὺς ἐν αὐτοῖς, καὶ τελειῶσαι τὰς χεῖρας αὐτῶν.
\VS{30}Ἑπτὰ ἡμέρας ἐνδύσεται αὐτὰ ὁ ἱερεὺς ὁ ἀντʼ αὐτοῦ ἐκ τῶν υἱῶν αὐτοῦ, ὃς εἰσελεύσεται εἰς τὴν σκηνὴν τοῦ μαρτυρίου λειτουργεῖν ἐν τοῖς ἁγίοις.
\VS{31}Καὶ τὸν κριὸν τῆς τελειώσεως λήψῃ· καὶ ἑψήσεις τὰ κρέα ἐν τόπῳ ἁγίῳ.
\VS{32}Καὶ ἔδονται Ἀαρὼν καὶ οἱ υἱοὶ αὐτοῦ τὰ κρέα τοῦ κριοῦ, καὶ τοὺς ἄρτους τοὺς ἐν τῷ κανῷ, παρὰ τὰς θύρας τῆς σκηνῆς τοῦ μαρτυρίου.
\VS{33}Ἔδονται αὐτὰ ἐν οἷς ἡγιάσθησαν ἐν αὐτοῖς τελειῶσαι τὰς χεῖρας αὐτῶν, ἁγιάσαι αὐτούς· καὶ ἀλλογενὴς οὐκ ἔδεται ἀπʼ αὐτῶν· ἔστι γὰρ ἅγια.
\VS{34}Ἐὰν δὲ καταλειφθῇ ἀπὸ τῶν κρεῶν τῆς θυσίας τῆς τελειώσεως καὶ τῶν ἄρτων ἕως πρωῒ, κατακαύσεις τὰ λοιπὰ πυρί· οὐ βρωθήσεται· ἁγίασμα γάρ ἐστι.
\par }{\PP \VS{35}Καὶ ποιήσεις Ἀαρὼν καὶ τοῖς υἱοῖς αὐτοῦ οὕτω κατὰ πάντα ὅσα ἐνετειλάμην σοι· ἑπτὰ ἡμέρας τελειώσεις τὰς χεῖρας αὐτῶν.
\VS{36}Καὶ τὸ μοσχάριον τῆς ἁμαρτίας ποιήσεις τῇ ἡμέρᾳ τοῦ καθαρισμοῦ· καὶ καθαριεῖς τὸ θυσιαστήριον ἐν τῷ ἁγιάζειν σε ἐπʼ αὐτῷ· καὶ χρίσεις αὐτὸ ὥστε ἁγιάσαι αὐτό.
\VS{37}Ἑπτὰ ἡμέρας καθαριεῖς τὸ θυσιαστήριον, καὶ ἁγιάσεις αὐτό· καὶ ἔσται τὸ θυσιαστήριον, ἅγιον τοῦ ἁγίου· πᾶς ὁ ἁπτόμενος τοῦ θυσιαστηρίου, ἁγιασθήσεται.
\VS{38}Καὶ ταῦτά ἐστιν, ἃ ποιήσεις ἐπὶ τοῦ θυσιαστηρίου· ἀμνοὺς ἐνιαυσίους ἀμώμους δύο τὴν ἡμέραν ἐπὶ τὸ θυσιαστήριον ἐνδελεχῶς, κάρπωμα ἐνδελεχισμοῦ.
\par }{\PP \VS{39}Τὸν ἀμνὸν τὸν ἕνα ποιήσεις τὸ πρωῒ, καὶ τὸν ἀμνὸν τὸν δεύτερον ποιήσεις τὸ δειλινόν.
\VS{40}Καὶ δέκατον σεμιδάλεως πεφυραμένης ἐν ἐλαίῳ κεκομμένῳ τῷ τετάρτῳ τοῦ εἴν· καὶ σπονδὴν τὸ τέταρτον τοῦ εἲν οἴνου τῷ ἀμνῷ τῷ ἑνί.
\VS{41}Καὶ τὸν ἀμνὸν τὸν δεύτερον ποιήσεις τὸ δειλινὸν, κατὰ τὴν θυσίαν τὴν πρωϊνὴν, καὶ κατὰ τὴν σπονδὴν αὐτοῦ· ποιήσεις εἰς ὀσμὴν εὐωδίας κάρπωμα Κυρίῳ,
\VS{42}θυσίαν ἐνδελεχισμοῦ εἰς γενεὰς ὑμῶν, ἐπὶ θύρας τῆς σκηνῆς τοῦ μαρτυρίου ἔναντι Κυρίου, ἐν οἷς γνωσθήσομαί σοι ἐκεῖθεν, ὥστε λαλῆσαί σοι.
\VS{43}Καὶ τάξομαι ἐκεῖ τοῖς υἱοῖς Ἰσραὴλ, καὶ ἁγιασθήσομαι ἐν δόξῃ μου.
\VS{44}Καὶ ἁγιάσω τὴν σκηνὴν τοῦ μαρτυρίου, καὶ τὸ θυσιαστήριον· καὶ Ἀαρὼν καὶ τοὺς υἱοὺς αὐτοῦ ἁγιάσω, ἱερατεύειν μοι.
\VS{45}Καὶ ἐπικληθήσομαι ἐν τοῖς υἱοῖς Ἰσραὴλ, καὶ ἔσομαι αὐτῶν Θεός.
\VS{46}Καὶ γνώσονται, ὅτι ἐγώ εἰμι Κύριος ὁ Θεὸς αὐτῶν, ὁ ἐξαγαγὼν αὐτοὺς ἐκ γῆς Αἰγύπτου, ἐπικληθῆναι αὐτοῖς, καὶ εἶναι αὐτῶν Θεός.

\par }\Chap{30}{\PP \VerseOne{1}Καὶ ποιήσεις θυσιαστήριον θυμιάματος ἐκ ξύλων ἀσήπτων.
\VS{2}Καὶ ποιήσεις αὐτὸ πήχεος τὸ μῆκος, καὶ πήχεος τὸ εὖρος· τετράγωνον ἔσται· καὶ δύο πήχεων τὸ ὕψος· ἐξ αὐτοῦ ἔσται τὰ κέρατα αὐτοῦ.
\VS{3}Καὶ καταχρυσώσεις χρυσίῳ καθαρῷ τὴν ἐσχάραν αὐτοῦ, καὶ τοὺς τοίχους αὐτοῦ κύκλῳ, καὶ τὰ κέρατα αὐτοῦ· καὶ ποιήσεις αὐτῷ στρεπτὴν στεφάνην χρυσῆν κύκλῳ.
\VS{4}Καὶ δύο δακτυλίους χρυσοῦς καθαροὺς ποιήσεις ὑπὸ τὴν στρεπτὴν στεφάνην αὐτοῦ, εἰς τὰ δύο κλίτη ποιήσεις ἐν τοῖς δυσὶ πλευροῖς· καὶ ἔσονται ψαλίδες ταῖς σκυτάλαις, ὥστε αἴρειν αὐτὸ ἐν αὐταῖς.
\VS{5}Καὶ ποιήσεις σκυτάλας ἐκ ξύλων ἀσήπτων, καὶ καταχρυσώσεις αὐτὰς χρυσίῳ.
\VS{6}Καὶ θήσεις αὐτὸ ἀπέναντι τοῦ καταπετάσματος, τοῦ ὄντος ἐπὶ τῆς κιβωτοῦ τῶν μαρτυρίων, ἐν οἷς γνωσθήσομαί σοι ἐκεῖθεν.
\VS{7}Καὶ θυμιάσει ἐπʼ αὐτοῦ Ἀαρὼν θυμίαμα σύνθετον λεπτὸν τὸ πρωῒ πρωΐ· ὅταν ἐπισκευάζῃ τοὺς λύχνους, θυμιάσει ἐπʼ αὐτοῦ.
\VS{8}Καὶ ὅταν ἐξάπτῃ Ἀαρὼν τοὺς λύχνους ὀψὲ, θυμιάσει ἐπʼ αὐτοῦ. θυμίαμα ἐνδελεχισμοῦ διαπαντὸς ἔναντι Κυρίου εἰς γενεὰς αὐτῶν.
\VS{9}Καὶ οὐκ ἀνοίσει ἐπʼ αὐτοῦ θυμίαμα ἕτερον· κάρπωμα, θυσίαν, και σπονδὴν οὐ σπείσεις ἐπʼ αὐτοῦ.
\VS{10}Καὶ ἐξιλάσεται ἐπʼ αὐτοῦ Ἀαρὼν ἐπὶ τῶν κεράτων αὐτοῦ ἅπαξ τοῦ ἐνιαυτοῦ· ἀπὸ τοῦ αἵματος τοῦ καθαρισμοῦ καθαριεῖ αὐτὸ εἰς γενεὰς αὐτῶν· ἅγιον τῶν ἁγίων ἐστὶ Κυρίῳ.
\par }{\PP \VS{11}Καὶ ἑλάλησε Κύριος πρὸς Μωυσῆν, λέγων,
\VS{12}ἐὰν λάβῃς τὸν συλλογισμὸν τῶν υἱῶν Ἰσραὴλ ἐν τῇ ἐπισκοπῇ αὐτῶν, καὶ δώσουσιν ἕκαστος λύτρα τῆς ψυχῆς αὐτοῦ Κυρίῳ, καὶ οὐκ ἔσται ἐν αὐτοῖς πτῶσις ἐν τῇ ἐπισκοπῇ αὐτῶν.
\VS{13}Καὶ τοῦτό ἐστιν ὅ δώσουσιν ὅσοι ἂν παραπορεύωνται τὴν ἐπίσκεψιν· τὸ ἥμισυ τοῦ διδράχμου ὅ ἐστιν κατὰ τὸ δίδραχμον τὸ ἅγιον, εἴκοσι ὀβολοὶ τὸ δίδραχμον, τὸ δὲ ἥμισυ τοῦ διδράχμου εἰσφορὰ Κυρίῳ.
\VS{14}Πᾶς ὁ παραπορευόμενος εἰς τὴν ἐπίσκεψιν ἀπὸ εἰκοσαετοῦς καὶ ἐπάνω, δώσουσι τὴν εἰσφορὰν Κυρίῳ.
\VS{15}Ὁ πλουτῶν οὐ προσθήσει, καὶ ὁ πενόμενος οὐκ ἐλαττονήσει ἀπὸ τοῦ ἡμίσεως τοῦ διδράχμου ἐν τῷ διδόναι τὴν εἰσφορὰν Κυρίῳ, ἐξιλάσασθαι περὶ τῶν ψυχῶν ὑμῶν.
\VS{16}Καὶ λήψῃ τὸ ἀργύριον τῆς εἰσφορᾶς παρὰ τῶν υἱῶν Ἰσραήλ, καὶ δώσεις αὐτὸ εἰς τὸ κάτεργον τῆς σκηνῆς τοῦ μαρτυρίου· καὶ ἔσται τοῖς υἱοῖς Ἰσραὴλ μνημόσυνον ἔναντι Κυρίου, ἐξιλάσασθαι περὶ τῶν ψυχῶν ὑμῶν.
\VS{17}Καὶ ἐλάλησε Κύριος πρὸς Μωυσῆν, λέγων,
\VS{18}ποίησον λουτῆρα χαλκοῦν, καὶ βάσιν αὐτῷ χαλκῆν, ὥστε νίπτεσθαι· καὶ θήσεις αὐτὸν ἀνὰ μέσον τῆς σκηνῆς τοῦ μαρτυρίου καὶ ἀνὰ μέσον τοῦ θυσιαστηρίου· καὶ ἐκχεεῖς εἰς αὐτὸν ὕδωρ.
\VS{19}Καὶ νίψεται Ἀαρὼν καὶ οἱ υἱοὶ αὐτοῦ ἑξ αὐτοῦ τὰς χεῖρας, καὶ τοὺς πόδας ὕδατι.
\VS{20}Ὅταν εἰσπορεύωνται εἰς τὴν σκηνὴν τοῦ μαρτυρίου, νίψονται ὕδατι, καὶ οὐ μὴ ἀποθάνωσιν, ὅταν προσπορεύωνται πρὸς τὸ θυσιαστήριον λειτουργεῖν καὶ ἀναφέρειν τὰ ὁλοκαυτώματα Κυρίῳ.
\VS{21}Νίψονται τὰς χεῖρας καὶ τοὺς πόδας ὕδατι, ὅταν εἰσπορεύωνται εἰς τὴν σκηνὴν τοῦ μαρτυρίου, νίψονται ὕδατι, ἵνα μὴ ἀποθάνωσι· καὶ ἔσται αὐτοῖς νόμιμον αἰώνιον, αὐτῷ καὶ ταῖς γενεαῖς αὐτοῦ μετʼ αὐτόν.
\VS{22}Καὶ ἐλάλησε Κύριος πρὸς Μωυσῆν, λέγων,
\VS{23}καὶ σὺ λάβε ἡδύσματα, τὸ ἄνθος σμύρνης ἐκλεκτῆς πεντακοσίους σίκλους, καὶ κινναμώμου εὐώδους τὸ ἥμισυ τούτου διακοσίους πεντήκοντα, καὶ καλάμου εὐώδους διακοσίους πεντήκοντα,
\VS{24}καὶ ἴρεως πεντακοσίους σίκλους τοῦ ἁγίου, καὶ ἔλαιον ἐξ ἐλαιῶν εἵν.
\VS{25}Καὶ ποιήσεις αὐτὸ ἔλαιον χρίσμα ἅγιον, μύρον μυρεψικὸν τέχνῃ μυρεψοῦ· ἔλαιον χρίσμα ἅγιον ἔσται.
\VS{26}Καὶ χρίσεις ἐξ αὐτοῦ τὴν σκηνὴν τοῦ μαρτυρίου, καὶ τὴν κιβωτὸν τῆς σκηνῆς τοῦ μαρτυρίου, καὶ πάντα τὰ σκεύη αὐτῆς,
\VS{27}καὶ τὴν λυχνίαν καὶ πάντα τὰ σκεύη αὐτῆς, καὶ τὸ θυσιαστήριον τοῦ θυμιάματος,
\VS{28}καὶ τὸ θυσιαστήριον τῶν ὁλοκαυτωμάτων καὶ πάντα αὐτοῦ τὰ σκεύη, καὶ τὴν τράπεζαν καὶ πάντα τὰ σκεύη αὐτῆς, καὶ τὸν λουτῆρα.
\VS{29}Καὶ ἁγιάσεις αὐτά· καὶ ἔσται ἅγια τῶν ἁγίων· πᾶς ὁ ἁπτόμενος αὐτῶν, ἁγιασθήσεται.
\VS{30}Καὶ Ἀαρὼν καὶ τοὺς υἱοὺς αὐτοῦ χρίσεις, καὶ ἁγιάσεις αὐτοὺς ἱερατεύειν μοι.
\VS{31}Καὶ τοῖς υἱοῖς Ἰσραὴλ λαλήσεις, λέγων, ἔλαιον ἄλειμμα χρίσεως ἅγιον ἔσται τοῦτο ὑμῖν εἰς τὰς γενεὰς ὑμῶν.
\VS{32}Ἐπὶ σάρκα ἀνθρώπου οὐ χρισθήσεται· καὶ κατὰ τὴν σύνθεσιν ταύτην οὐ ποιήσετε ὑμῖν ἑαυτοῖς ὡσαύτως· ἅγιόν ἐστιν, καὶ ἁγίασμα ἔσται ὑμῖν.
\VS{33}Ὃς ἂν ποιήσῃ ὡσαύτως, καὶ ὃς ἂν δῷ ἀπʼ αὐτοῦ ἀλλογενεῖ, ἐξολοθρευθήσεται ἐκ τοῦ λαοῦ αὐτοῦ.
\par }{\PP \VS{34}Καὶ εἶπε Κύριος πρὸς Μωυσῆν, λάβε σεαυτῷ ἡδύσματα, στακτήν, ὄνυχα, χαλβάνην ἡδυσμοῦ καὶ λίβανον διαφανῆ· ἴσον ἴσῳ ἔσται.
\VS{35}Καὶ ποιήσουσιν ἐν αὐτῷ θυμίαμα μυρεψικὸν ἔργον μυρεψοῦ μεμιγμένον, καθαρὸν ἔργον ἅγιον.
\VS{36}Καὶ συνκόψεις ἐκ τούτων λεπτόν, καὶ θήσεις ἀπέναντι τῶν μαρτυρίων ἐν τῇ σκηνῇ τοῦ μαρτυρίου, ὅθεν γνωσθήσομαί σοι ἐκεῖθεν· ἅγιον τῶν ἁγίων ἔσται ὑμῖν θυμίαμα.
\VS{37}Κατὰ τὴν σύνθεσιν ταύτην οὐ ποιήσετε ὑμῖν ἐαυτοῖς· ἁγίασμα ἔσται ὑμῖν Κυρίῳ·
\VS{38}Ὃς ἂν ποιήσῃ ὡσαύτως ὥστε ὀσφραίνεσθαι ἐν αὐτῷ, ἀπολεῖται ἐκ τοῦ λαοῦ αὐτοῦ.

\par }\Chap{31}{\PP \VerseOne{1}Καὶ ἐλάλησε Κύριος πρὸς Μωυσῆν, λέγων,
\VS{2}ἰδοὺ ἀνακέκλημαι ἐξ ὀνόματος τὸν Βεσελεὴλ τὸν τοῦ Οὐρείου τὸν Ὣρ, ἐκ τῆς φυλῆς Ἰούδα.
\VS{3}Καὶ ἐνέπλησα αὐτὸν πνεῦμα θεῖον σοφίας καὶ συνέσεως καὶ ἐπιστήμης, ἐν παντὶ ἔργῳ διανοεῖσθαι,
\VS{4}καὶ ἀρχιτεκτονῆσαι, ἐργάζεσθαι τὸ χρυσίον, καὶ τὸ ἀργύριον, καὶ τὸν χαλκὸν, καὶ τὴν ὑάκινθον, καὶ τὴν πορφύραν, καὶ τὸ κόκκινον τὸ νηστὸν,
\VS{5}καὶ τὰ λιθουργικὰ, καὶ εἰς τὰ ἔργα τὰ τεκτονικὰ τῶν ξύλων, ἐργάζεσθαι κατὰ πάντα τὰ ἔργα.
\VS{6}Καὶ ἐγὼ ἔδωκα αὐτὸν καὶ τὸν Ἐλιὰβ τὸν τοῦ Ἀχισαμὰχ ἐκ φυλῆς Δάν· καὶ παντὶ συνετῷ καρδίᾳ δέδωκα σύνεσιν·
\VS{7}καὶ πονήσουσι πάντα ὅσα συνέταξά σοι, τὴν σκηνὴν τοῦ μαρτυρίου, καὶ τὴν κιβωτὸν τῆς διαθήκης, καὶ τὸ ἱλαστήριον τὸ ἐπʼ αὐτῆς, καὶ τὴν διασκευὴν τῆς σκηνῆς,
\VS{8}καὶ τὰ θυσιαστήρια, καὶ τὴν τράπεζαν καὶ πάντα τὰ σκεύη αὐτῆς, καὶ τὴν λυχνίαν τὴν καθαρὰν καὶ πάντα τὰ σκεύη αὐτῆς
\VS{9}καὶ τὸν λουτῆρα καὶ τὴν βάσιν αὐτοῦ,
\VS{10}καὶ τὰς στολὰς τὰς λειτουργικὰς Ἀαρὼν, καὶ τὰς στολὰς τῶν υἱῶν αὐτοῦ ἱερατεύειν μοι,
\VS{11}καὶ τὸ ἔλαιον τῆς χρίσεως, καὶ τὸ θυμίαμα τῆς συνθέσεως τοῦ ἁγίου· κατὰ πάντα ὅσα ἐγὼ ἐνετειλάμην σοι, ποιήσουσι.
\par }{\PP \VS{12}Καὶ ἐλάλησε Κύριος πρὸς Μωυσῆν, λέγων,
\VS{13}Καὶ σὺ σύνταξον τοῖς υἱοῖς Ἰσραὴλ, λέγων, Ὁρᾶτε, καὶ τὰ σάββατά μου φυλάξεσθε· σημεῖόν ἐστι παρʼ ἐμοὶ καὶ ἐν ὑμῖν εἰς τὰς γενεὰς ὑμῶν, ἵνα γνῶτε ὅτι ἐγὼ Κύριος ὁ ἁγιάξων ὑμᾶς.
\VS{14}καὶ φυλάξεσθε τὰ σάββατα, ὅτι ἅγιον τοῦτό ἐστι Κυρίῳ ὑμῖν· ὁ βεβηλῶν αὐτὸ, θανάτῳ θανατωθήσεται· πᾶς ὃς ποιήσει ἐν αὐτῷ ἔργον, ἐξολοθρευθήσεται ἡ ψυχὴ ἐκείνη ἐκ μέσου τοῦ λαοῦ αὐτοῦ.
\VS{15}ἓξ ἡμέρας ποιήσεις ἔργα, τῇ δὲ ἡμέρᾳ τῇ ἑβδόμῃ σάββατα, ἀνάπαυσις ἁγία τῷ κυρίῳ· πᾶς ὃς ποιήσει ἔργον τῇ ἡμέρᾳ τῇ ἑβδόμῃ θανατωθήσεται.
\VS{16}Καὶ φυλάξουσιν οἱ υἱοὶ Ἰσραὴλ τὰ σάββατα, ποιεῖν αὐτὰ εἰς τὰς γενεὰς αὐτῶν·
\VS{17}Διαθήκη αἰώνιος ἐν ἐμοὶ καὶ τοῖς υἱοῖς Ἰσραὴλ, σημεῖόν ἐστιν ἐν ἐμοὶ αἰώνιον· ὅτι ἓξ ἡμεραις ἐποίησε Κύριος τὸν οὐρανὸν καὶ τὴν γῆν, καὶ τῇ ἡμέρᾳ τῇ ἑβδόμῃ κατέπαυσε, καὶ ἐπαύσατο.
\VS{18}Καὶ ἔδωκε Μωυσῇ ἡνίκα κατέπαυσε λαλῶν αὐτῷ ἐν τῷ ὄρει τῷ Σινὰ, τὰς δύο πλάκας τοῦ μαρτυρίου, πλάκας λιθίνας γεγραμμένας τῷ δακτύλῳ τοῦ Θεοῦ.

\par }\Chap{32}{\PP \VerseOne{1}Καὶ ἰδὼν ὁ λαὸς, ὅτι κεχρόνικε Μωυσῆς καταβῆναι ἐκ τοῦ ὄρους, συνέστη ὁ λαὸς ἐπὶ Ἀαρὼν, καὶ λέγουσιν αὐτῷ ἀνάστηθι, καὶ ποίησον ἡμῖν θεοὺς, οἳ προπορεύσονται ἡμῶν· ὁ γὰρ Μωυσῆς οὗτος ὁ ἄνθρωπος ὃς ἐξήγαγεν ἡμᾶς ἐκ γῆς Αἰγύπτου, οὐκ οἴδαμεν τί γέγονεν αὐτῷ.
\VS{2}Καὶ λέγει αὐτοῖς Ἀαρὼν Περιέλεσθε τὰ ἐνώτια τὰ χρυσᾶ τὰ ἐν τοῖς ὠσὶ τῶν γυναικῶν ὑμῶν καὶ θυγατέρων, καὶ ἐνέγκατε πρός με.
\VS{3}Καὶ περιείλαντο πᾶς ὁ λαὸς τὰ ἐνώτια τὰ χρυσᾶ τὰ ἐν τοῖς ὠσὶν αὐτῶν, καὶ ἤνεγκαν πρὸς Ἀαρών.
\VS{4}Καὶ ἐδέξατο ἐκ τῶν χειρῶν αὐτῶν, καὶ ἔπλασεν αὐτὰ ἐν τῇ γραφίδι· καὶ ἐποίησεν αὐτὰ μόσχον χωνευτὸν καὶ εἶπεν, Οὗτοι οἱ θεοί σου Ἰσραὴλ, οἵτινες ἀνεβίβασάν σε ἐκ γῆς Αἰγύπτου.
\VS{5}Καὶ ἰδὼν Ἀαρὼν ᾠκοδόμησε θυσιαστήριον κατέναντι αὐτοῦ· καὶ ἐκήρυξεν Ἀαρὼν λέγων, ἑορτὴ τοῦ κυρίου αὔριον.
\VS{6}Καὶ ὀρθρίσας τῇ ἐπαύριον ἀνεβίβασεν ὁλοκαυτώματα, καὶ προσήνεγκε θυσίαν σωτηρίου· καὶ ἐκάθισεν ὁ λαὸς φαγεῖν καὶ πιεῖν, καὶ ἀνέστησαν παίζειν.
\par }{\PP \VS{7}Καὶ ἐλάλησε Κύριος πρὸς Μωυσῆν, λέγων, βάδιζε τὸ τάχος, κατάβηθι ἐντεύθεν· ἠνόμησε γὰρ ὁ λαός σου ὃν ἐξήγαγες ἐκ γῆς Αἰγύπτου.
\VS{8}Παρέβησαν ταχὺ ἐκ τῆς ὁδοῦ, ἧς ἐνετείλω αὐτοῖς· ἐποίησαν ἑαυτοῖς μόσχον, καὶ προσκεκυνήκασιν αὐτῷ, καὶ τεθύκασιν αὐτῷ, καὶ εἶπαν, Οὗτοι οἱ θεοί σου Ἰσραὴλ, οἵτινες ἀνεβίβασάν σε ἐκ γῆς Αἰγύπτου.
\VS{10}καὶ νῦν ἔασόν με, καὶ θυμωθεὶς ὀργῇ εἰς αὐτοὺς, ἐκτρίψω αὐτούς· καὶ ποιήσω σὲ εἰς ἔθνος μέγα.
\VS{11}καὶ ἐδεήθη Μωυσῆς ἔναντι Κυρίου τοῦ Θεοῦ, καὶ εἶπεν, ἱνατί, Κύριε, θυμοῖ ὀργῇ εἰς τὸν λαόν σου, οὓς ἐξήγαγες ἐκ γῆς Αἰγύπτου ἐν ἰσχύϊ μεγάλῃ, καὶ ἐν τῷ βραχίονί σου τῷ ὑψηλῷ;
\VS{12}Μή ποτε εἴπωσιν οἱ Αἰγύπτιοι λέγοντες Μετὰ πονηρίας ἐξήγαγεν αὐτοὺς ἀποκτεῖναι ἐν τοῖς ὄρεσιν καὶ ἐξαναλῶσαι αὐτοὺς ἀπὸ τῆς γῆς. παῦσαι τῆς ὀργῆς τοῦ θυμοῦ σου, καὶ ἵλεως γενοῦ ἐπὶ τῇ κακίᾳ τοῦ λαοῦ σου,
\VS{13}μνησθεὶς Ἀβραὰμ καὶ Ἰσαὰκ καὶ Ἰακὼβ τῶν σῶν οἰκετῶν, οἷς ὤμοσας κατὰ σεαυτοῦ, καὶ ἐλάλησας πρὸς αὐτοὺς, λέγων, πολυπληθυνῶ τὸ σπέρμα ὑμῶν ὡσεὶ τὰ ἄστρα τοῦ οὐρανοῦ τῷ πλήθει· καὶ πᾶσαν τὴν γῆν ταύτην ἣν εἶπας δοῦναι αὐτοῖς, καὶ καθέξουσιν αὐτὴν εἰς τὸν αἰῶνα.
\VS{14}Καὶ ἱλάσθη Κύριος περιποιῆσαι τὸν λαὸν αὐτοῦ.
\par }{\PP \VS{15}Καὶ ἀποστρέψας Μωυσῆς, κατέβη ἀπὸ τοῦ ὄρους· καὶ αἱ δύο πλάκες τοῦ μαρτυρίου ἐν ταῖς χερσὶν αὐτοῦ, πλάκες λίθιναι καταγεγραμμέναι ἐξ ἀμφοτέρων τῶν μερῶν αὐτῶν, ἔνθεν καὶ ἔνθεν ἦσανς γεγραμμέναι·
\VS{16}καὶ αἱ πλάκες ἔργον Θεοῦ ἦσαν, καὶ ἡ γραφὴ γραφὴ Θεοῦ κεκολαμμένη ἐν ταῖς πλαξί.
\VS{17}καὶ ἀκούσας Ἰησοῦς τῆς φωνῆς τοῦ λαοῦ κραζόντων, λέγει πρὸς Μωυσὴν, Φωνὴ πολέμου ἐν τῇ παρεμβολῇ.
\VS{18}καὶ λέγει Οὐκ ἔστι φωνὴ ἐξαρχόντων κατʼ ἰσχὺν, οὐδὲ φωνὴ ἐξαρχόντων τροπῆς, ἀλλὰ φωνὴν ἐξαρχόντων οἴνου ἐγὼ ἀκούω.
\par }{\PP \VS{19}καὶ ἡνίκα ἤγγιζε τῇ παρεμβολῇ, ὁρᾷ τὸν μόσχον καὶ τοὺς χορούς· καὶ ὀργισθεὶς θυμῷ Μωυσῆς ἔῤῥιψεν ἀπὸ τῶν χειρῶν αὐτοῦ τὰς δύο πλάκας, καὶ συνέτριψεν αὐτὰς ὑπὸ τὸ ὄρος·
\VS{20}καὶ λαβὼν τὸν μόσχον ὃν ἐποίησαν, κατέκαυσεν αὐτὸν ἐν πυρὶ, καὶ κατήλεσεν αὐτὸν λεπτὸν, καὶ ἔσπειρεν αὐτὸν ὑπὸ τὸ ὕδωρ, καὶ ἐπότισεν αὐτὸ τοὺς υἱοὺς Ἰσραήλ.
\VS{21}καὶ εἶπε Μωυσῆς τῷ Ἀαρὼν, Τί ἐποίησέ σοι ὁ λαὸς οὗτος, ὅτι ἐπήγαγες ἐπʼ αὐτοὺς ἁμαρτίαν μεγάλην;
\VS{22}καὶ εἶπεν Ἀαρὼν πρὸς Μωυσῆν, μὴ ὀργίζου, κύριε· σὺ γὰρ οἶδας τὸ ὅρμημα τοῦ λαοῦ τούτου.
\VS{23}Λέγουσι γάρ μοι, ποιήσον ἡμῖν θεοὺς, οἳ προπορεύσονται ἡμῶν· ὁ γὰρ Μωυσῆς οὗτος ὁ ἄνθρωπος, ὃς ἐξήγαγεν ἡμᾶς ἐξ Αἰγύπτου, οὐκ οἴδαμεν τί γέγονεν αὐτῷ.
\VS{24}καὶ εἶπα αὐτοῖς, εἴ τινι ὑπάρχει χρυσία, περιέλεσθε· καὶ ἔδωκάν μοι· καὶ ἔῤῥιψα εἰς τὸ πῦρ· καὶ ἐξῆλθεν ὁ μόσχος οὗτος.
\VS{25}Καὶ ἰδὼν Μωυσῆς τὸν λαὸν ὅτι διεσκέδασται· (διεσκέδασε γὰρ αὐτοὺς Ἀαρών ἐπίχαρμα τοῖς ὑπεναντίοις αὐτῶν)
\VS{26}ἔστη δὲ Μωυσῆς ἑπὶ τῆς πύλης τῆς παρεμβολῆς, καὶ εἶπε, τίς πρὸς Κύριον; ἴτω πρός με. Συνῆλθον οὖν πρὸς αὐτὸν πάντες οἱ υἱοὶ Λευί.
\VS{27}Καὶ λέγει αὐτοῖς τάδε λέγει Κύριος ὁ Θεὸς Ἰσραήλ θέσθε ἕκαστος τὴν ἑαυτοῦ ῥομφαίανν ἐπὶ τὸν μηρὸν, καὶ διέλθατε καὶ ἀνακάμψατε ἀπὸ πύλης ἐπὶ πύλην διὰ τῆς παρεμβολῆς, καὶ ἀποκτείνατε ἕκαστος τὸν ἀδελφὸν αὐτοῦ, καὶ ἕκαστος τὸν πλησίον αὐτοῦ, καὶ ἓκαστος τὸν ἔγγιστα αὐτοῦ.
\VS{28}Καὶ ἐποίησαν οἱ υἱοὶ Λευεὶ καθὰ ἐλάλησεν αὐτοῖς Μωυσῆς· καὶ ἔπεσαν ἐκ τοῦ λαοῦ ἐν ἐκείνῃ τῇ ἡμέρᾳ εἰς τρισχιλίους ἄνδρας.
\VS{29}Καὶ εἶπεν αὐτοῖς Μωυσῆς, ἐπληρώσατε τὰς χεῖρας ὑμῶν σήμερον Κυρίῳ ἕκαστος ἐν τῷ υἱῷ ἢ ἐν τῷ ἀδελφῷ αὐτοῦ, δοθῆναι ἐφʼ ὑμᾶς εὐλογίαν.
\par }{\PP \VS{30}Καὶ ἐγένετο μετὰ τὴν αὔριον εἶπε Μωυσῆς πρὸς τὸν λαὸν, ὑμεῖς ἡμαρτήκατε ἁμαρτίαν μεγάλη· καὶ νῦν ἀναβήσομαι πρὸς τὸν Θεὸν, ἵνα ἐξιλάσωμαι περὶ τῆς ἁμαρτίας ὑμῶν.
\VS{31}Ὑπέστρεψε δὲ Μωυσῆς πρὸς Κύριον, καὶ εἶπε, δέομαι, κύριε· ἡμάρτηκεν ὁ λαὸς οὗτος ἁμαρτίαν μεγάλην, καὶ ἐποίησαν ἑαυτοῖς θεοὺς χρυσοῦς.
\VS{32}Καὶ νῦν εἰ μὲν ἀφεῖς αὐτοῖς τὴν ἁμαρτίαν αὐτῶν, ἄφες· εἰ δὲ μή, ἐξάλειψόν με ἐκ τῆς βίβλου σου, ἧς ἔγραψας.
\VS{33}Καὶ εἶπε Κύριος πρὸς Μωυσῆν, εἴ τις ἡμάρτηκεν ἐνώπιόν μου, ἐξαλείψω αὐτοὺς ἐκ τῆς βίβλου μου.
\VS{34}Νυνὶ δὲ βάδιζε, κατάβηθι, καὶ ὁδήγησον τὸν λαὸν τοῦτον εἰς τὸν τόπον, ὃν εἶπά σοι· ἰδοὺ ὁ ἄγγελός μου προπορεύσεται πρὸ προσώπου σου· ᾖ δʼ ἂν ἡμέρᾳ ἐπισκέπτωμαι, ἐπάξω ἐπʼ αὐτοὺς τὴν ἁμαρτίαν αὐτῶν
\VS{35}Καὶ ἐπάταξε Κύριος τὸν λαὸν περὶ τῆς ποιήσεως τοῦ μόσχου, οὗ ἐποίησεν Ἀαρών.

\par }\Chap{33}{\PP \VerseOne{1}Καὶ εἶπε Κύριος πρὸς Μωυσῆς, προπορεύου, ἀνάβηθι ἐντεῦθεν σὺ καὶ ὁ λαός σου, οὓς ἐξήγαγες ἐκ γῆς Αἰγύπτου, εἰς τὴν γῆν, ἣν ὤμοσα τῷ Ἀβραὰμ, καὶ Ἰσαὰκ, καὶ Ἰακὼβ, λέγων, Τῷ σπέρματι ὑμῶν δώσω αὐτήν.
\VS{2}Καὶ συναποστελῶ τὸν ἄγγελόν μου πρὸ προσώπου σου· καὶ ἐκβαλεῖ τὸν Ἀμοῤῥαῖον, καὶ Χετταῖον, καὶ Φερεζαῖον, καὶ Γεργεσαῖον, καὶ Εὐαῖον, καὶ Ἰεβουσαῖον, καὶ Χαναναῖον.
\VS{3}Καὶ εἰσάξω σε εἰς γῆν ῥέουσαν γάλα καὶ μέλι· οὐ γὰρ μὴ συναναβῶ μετὰ σου, διὰ τὸ λαὸν σκληροτράχηλόν σε εἶναι, ἵνα μὴ ἐξαναλώσω σεε ἐν τῇ ὁδῷ.
\VS{4}Καὶ ἀκούσας ὁ λαὸς τὸ ῥῆμα τὸ πονηρὸν τοῦτο, κατεπένθησεν ἐν πενθικοῖς.
\VS{5}Καὶ εἶπε Κύριος τοῖς υἱοῖς Ἰσραὴλ, ὑμεῖς λαὸς σκληροτράχηλος· ὁρᾶτε, μὴ πληγὴν ἄλλην ἐπάξω ἐγὼ ἐφʼ ὑμᾶς, καὶ ἐξαναλώσω ὑμᾶς· νῦν οὖν ἀφέλεσθε τὰς στολὰς τῶν δοξῶν ὑμῶν, καὶ τὸν κόσμον, καὶ δείξω σοι ἃ ποιήσω σοι.
\VS{6}Καὶ περιέλαντο οἱ υἱοὶ Ἰσραὴλ τὸν κόσμον αὐτῶν, καὶ τὴν περιστολὴν ἀπὸ τοῦ ὄρους τοῦ Χωρήβ.
\VS{7}Καὶ λαβὼν Μωυσῆς τὴν σκηνὴν αὐτοῦ, ἔπηξεν ἔξω τῆς παρεμβολῆς, μακρὰν ἀπὸ τῆς παρεμβολῆς· καὶ ἐκλήθν Σκηνὴ μαρτυρίου· καὶ ἐγένετο, πᾶς ὁ ζητῶν Κύριον ἐξεπορεύετο εἰς τὴν σκηνὴν τὴν ἔξω τῆς παρεμβολῆς.
\VS{8}Ἡνίκα δʼ ἂν εἰσεπορεύετο Μωυσῆς εἰς τὴν σκηνὴν ἔξω τῆς παρεμβολῆς, εἱστήκει πᾶς ὁ λαὸς σκοπεύοντες ἕκαστος παρὰ τὰς θύρας τῆς σκηνῆς αὐτοῦ· καὶ κατενοοῦσαν ἀπιόντος Μωυσῆ ἕως τοῦ εἰσελθεῖν αὐτὸν εἰς τὴν σκηνὴν.
\VS{9}Ὡς δʼ ἂν εἰσῆλθε Μωσῆς εἰς τὴν σκηνήν, κατέβαινεν ὁ στύλος τῆς νεφέλης, καὶ ἵστατο ἐπὶ τὴν θύραν τῆς σκηνῆς, καὶ ἐλάλει Μωσῇ·
\VS{10}καὶ ἐλάλει Μωυσῇ. Καὶ ἑώρα πᾶς ὁ λαὸς τὸν στύλον τῆς νεφέλης ἑστῶτα ἐπὶ τῆς θύρας τῆς σκηνῆς· καὶ στάντες πᾶς ὁ λαὸς, προσεκύνησαν ἕκαστος ἀπὸ τῆς θύρας τῆς σκηνῆς αὐτοῦ.
\VS{11}Καὶ ἐλάλησε Κύριος πρὸς Μωυσῆν, ἐνώπιος ἐνωπίῳ, ὡς εἴτις λαλήσει πρὸς τὸν ἑαυτοῦ φίλον· καὶ ἀπελύετο εἰς τὴν παρεμβολήν· ὁ δὲ θεράπων Ἰησοῦς υἱὸς Ναυὴ νέος οὐκ ἐξεπορεύετο ἐκ τῆς σκηνῆς.
\par }{\PP \VS{12}Καὶ εἶπε Μωυσῆς, πρὸς Κύριον, Ἰδοὺ σύ μοι λέγεις, ἀνάγαγε τὸν λαὸν τοῦτον, σὺ δὲ οὐκ ἐδήλωσάς μοι, ὃν συναποστελεῖς μετʼ ἐμοῦ· σὺ δέ μοι εἶπας, Οἶδά σε παρὰ πάντας, καὶ χάριν ἔχεις παρʼ ἐμοί.
\VS{13}Εἰ οὖν εὕρηκα χάριν ἐναντίον σου, ἐμφάνισόν μοι σεαυτόν· γνωστῶς ἴδω ἴδω σε, ὅπως ἂν ὦ εὑρηκὼς χάριν ἐναντίον σου,, καὶ ἵνα γνῶ, ὅτι λαός σου τὸ ἔθνος τὸ μέγα τοῦτο.
\VS{14}Καὶ λέγει, αὐτὸς προπορεύσομαί σου, καὶ καταπαύσω σε.
\VS{15}Καὶ λέγει πρὸς αὐτόν, εἰ μὴ αὐτὸς σὺ σνμπορεύῃ, μή με ἀναγάγῃς ἐντεῦθεν.
\VS{16}Καὶ πῶς γνωστὸν ἔσται ἀληθῶς, ὅτι εὕρηκα χάριν παρὰ σοί ἐγώ τε καὶ ὁ λαός σου, ἀλλʼ ἢ συμπορευομένου σου μεθʼ ἡμῶν; καὶ ἐνδοξασθήσομαι ἐγώ τε καὶ ὁ λαός σου παρὰ πάντα τὰ ἔθνη, ὅσα ἐπὶ τῆς γῆς ἐστί.
\VS{17}Καὶ εἶπε Κύριος πρὸς Μωυσῆν, Καὶ τοῦτόν σοι τὸν λόγον, ὃν εἴρηκας ποιήσω· εὕρηκας, ποιήσω· εὕρηκας γὰρ χάριν ἐνώπιον ἐμοῦ, καὶ οἶδά σε παρὰ πάντας.
\VS{18}Καὶ λέγει, ἐμφάνισόν μοι σεαυτόν.
\VS{19}Καὶ εἶπεν, ἐγὼ παρελεύσομαι πρότερός σου τῇ δόξῃ μου, καὶ καλέσω τῷ ὀνόματί μου, Κύριος ἐναντίον σου· καὶ ἐλεήσω, ὃν ἂν ἐλεῶ, καὶ οἰκτειρήσω, ὃν ἂν οἰκτείρῶ.
\VS{20}Καὶ εἶπε, οὐ δυνήσῃ ἰδεῖν τὸ πρόσωπόν μου· οὐ γὰρ μὴ ἴδῃ ἄνθρωπος τὸ πρόσωπόν μου, καὶ ζήσεται.
\VS{21}Καὶ εἶπεν Κύριος, Ἰδοὺ τόπος παρʼ ἐμοί, στήσῃ ἐπὶ τῆς πέτρας·
\VS{22}Ἡνίκα δʼ ἂν παρέλθηνᾑ ᾑ δόξα μου, καί θήσω σε εἰς ὀπὴν τῆς πέτρας, καὶ σκεπάσω τῇ χειρί μου ἐπὶ σὲ, ἕως ἂν παρέλθω.
\VS{23}Καὶ ἀφελῶ τὴν χεῖρα, καὶ τότε ὄψει τὰ ὀπίσω μου· τὸ δὲ πρόσωπόν μου οὐκ ὀφθήσεταί σοι.

\par }\Chap{34}{\PP \VerseOne{1}Καὶ εἶπε Κύριος πρὸς Μωυσῆν λαξευσον σεαυτῷ δύο πλάκας λιθίνας, καθὼς καὶ αἱ πρῶται, καὶ ἀνάβηθι πρὸς μὲ εἰς τὸ ὄρος, καὶ γράψω ἐπὶ τῶν πλακῶν τὰ ῥήματα ἃ ἦν ἐν ταῖς πλαξὶ ταῖς πρώταις, αἷς συνέτριψας.
\VS{2}Καὶ γίνου ἕτοιμος εἰς τὸ πρωί, καὶ ἀναβήσῃ ἐπὶ τὸ ὄρος τὸ Σινά, καὶ στήσῃ μοι ἐκεῖ ἐπʼ ἄκρου τοῦ ὄρους.
\VS{3}Καὶ μηδεὶς ἀναβήτω μετὰ σοῦ μηδὲ ὀφθήτω ἐν παντὶ τῷ ὄρει· καὶ τὰ πρόβατα καὶ βόες μὴ νεμέσθωσαν πλησίον τοῦ ὄρους ἐκείνου.
\VS{4}Καὶ ἐλάξευσε δύο πλάκας λιθίνας, καθάπερ καὶ αἱ πρῶται· καὶ ὀρθρίσας Μωυσῆς, ἀνέβη εἰς τὸ ὄρος τὸ Σινὰ, καθότι συνέταξεν αὐτῷ Κύριος· καὶ ἔλαβε Μωυσῆς τὰς δύο πλάκας τὰς λιθίνας.
\VS{5}Καὶ κατέβη Κύριος ἐν νεφέλῃ, καὶ παρέστη αὐτῷ ἐκεῖ, καὶ ἐκάλεσε τῷ ὀνόματι Κυρίου.
\VS{6}Καὶ παρῆλθε Κύριος πρὸ προσώπου αὐτοῦ, καὶ ἐκάλεσε κύριος ὁ Θεὸς οἰκτείρμων καὶ ἐλεήμων, μακρόθυμος καὶ πολυέλεος καὶ ἀληθινός,
\VS{7}καὶ δικαιοσύνην διατηρῶν καὶ ἔλεος εἰς χιλιάδας, ἀφαιρῶν ἀνομίας καὶ ἀδικίας καὶ ἁμαρτίας, καὶ οὐ καθαριεῖ τὸν ἔνοχον, ἐπάγων ἀνομίας πατέρων ἐπὶ τέκνα καὶ ἐπὶ τέκνα τέκνων ἐπὶ τρίτην καὶ τετάρτην γενεάν.
\VS{8}Καὶ σπεύσας Μωσῆς κύψας ἐπὶ τὴν γῆν προσεκύνησε·
\VS{9}καὶ εἶπεν, εἰ εὕρηκα χάριν ἐνώπιόν σου, συμπορευθήτω ὁ Κύριός μου μεθʼ ἡμῶν· ὁ λαὸς γὰρ σκληροτράχηλός ἐστι, καὶ ἀφελεῖς σὺ τὰς ἁμαρτίας ἡμῶν, καὶ τὰς ἀνομίας ἡμῶν, καὶ ἐσόμεθά σοι.
\par }{\PP \VS{10}Καὶ εἶπε Κύριος πρὸς Μωυσῆν, Ἰδοὺ, ἐγὼ τίθημί σοι διαθήκην ἐνώπιον παντὸς τοῦ λαοῦ σοῦ, ποιήσω ἔνδοξα, ἃ οὐ γέγονεν ἐν πάσῃ τῇ γῇ, καὶ ἐν παντὶ ἔθνει· καὶ ὄψεται πᾶς ὁ λαὸς, ἐν οἷς εἶ σὺ, τὰ ἔργα Κυρίου, ὅτι θαυμαστά ἐστιν, ἃ ἐγὼ ποιήσω σοι.
\VS{11}Πρόσεχε σὺ πάντα ὅσα ἐγὼ ἐντέλλομαί σοι· ἰδοὺ ἐγὼ ἐκβάλλω πρὸ προσώπου ὑμῶν τὸν Ἀμοῤῥαῖον, καὶ Χαναναῖον, καὶ Φερεζαῖον, καὶ Χετταῖον, καὶ Εὑαῖον, καὶ Γεργεσαῖον, καὶ Ἰεβουσαῖον.
\VS{12}Πρόσεχε σεαυτῷ, μή ποτε θῇς διαθήκην τοῖς ἐγκαθημένοις ἐπὶ τῆς γῆς, εἰς ἣν εἰσπορεύῃ εἰς αὐτὴν, μή σοι γένηται πρόσκομμα ἐν ὑμῖν.
\VS{13}Τοὺς βωμοὺς αὐτῶν καθελεῖτε, καὶ τὰς στήλας αὐτῶν συντρίψετε, καὶ τὰ ἄλση αὐτῶν ἐκκόψετε, καὶ τὰ γλυπτὰ τῶν θεῶν αὐτῶν κατακαύσετε ἐν πυρί.
\VS{14}Οὐ γὰρ μὴ προσκυνήσητε θεοῖς ἑτέροις· ὁ γὰρ Κύριος ὁ Θεὸς, ζηλωτὸν ὄνομα, Θεὸς ζηλωτής ἐστι.
\VS{15}Μή ποτε θῇς διαθήκην τοῖς ἔγκαθημένοις ἐπὶ τῆς γῆς, καὶ ἐκπορνεύσωσιν ὀπίσω τῶν θεῶν αὐτῶν, καὶ θύσωσι τοῖς θεοῖς αὐτῶν, καὶ καλέσωσίν σε, καὶ φάγῃς τῶν αὐτῶν,
\VS{16}καὶ λάβῃς τῶν θυγατέρων αὐτῶν τοῖς υἱοῖς σου, καὶ τῶν θυγατέρων σου δῷς τοῖς υἱοῖς αὐτῶν, καὶ ἐκπορνεύσωσιν αἱ θυγατέρες σου ὀπίσω τῶν θεῶν αὐτῶν, καὶ ἐκπορνεύσωσιν οἱ υἱοί σου ὀπίσω τῶν θεῶν αὐτῶν.
\VS{17}Καὶ θεοὺς χωνευτοὺς οὐ ποιήσεις σεαυτῷ.
\VS{18}Καὶ τὴν ἑσρτὴν τῶν ἀζύμων φυλάξῃ· ἑπτὰ ἡμέρας φαγῃ ἄζυμα, καθάπερ ἐντέταλμαί σοι, εἰς τὸν καιρὸν ἐν μηνὶ τῶν νέων· ἐν γὰρ μηνὶ τῶν νέων ἐξῆλθες ἐξ Αἰγύπτου.
\VS{19}Πᾶν διανοῖγον μήτραν, ἐμοὶ τὰ ἀρσενικὰ, πᾶν πρωτότοκον μόσχου, καὶ πρωτότοκον προβάτου.
\VS{20}Καὶ πρωτότοκον ὑποζυγίου λυτρώσῃ προβάτῳ· ἐὰν δὲ μὴ λυτρώσῃ αὐτὸ, τιμὴν δώσεις. πᾶν πρωτότοκον τῶν υἱῶν σου λυτρώσῃ· οὐκ ὀφθήσῃ ἐνώπιόν μου κενός.
\par }{\PP \VS{21}Ἓξ ἡμέρας ἐργᾷ, τῇ δὲ ἑβδόμῃ καταπαύσεις· τῷ σπόρῳ καὶ τῷ ἀμητῷ κατάπαυσις.
\VS{22}Καὶ ἑορτὴν ἑβδομάδων ποιήσεις μοι, ἀρχὴν θερισμοῦ πυροῦ· καὶ ἐορτὴν συναγωγῆς μεσοῦντος τοῦ ἐνιαυτοῦ.
\VS{23}Τρεῖς καιροὺς τοῦ ἐνιαυτοῦ ὀφθήσεται πᾶν ἀρσενικόν σου ἐνώπιον Κυρίου τοῦ Θεοῦ Ἰσραήλ.
\VS{24}Ὅταν γὰρ ἐκβάλω τὰ ἔθνη πρὸ προσώπου σου, καὶ πλατυνῶ τὰ ὅριά σου, οὐκ ἐπιθυμήσει οὐδεὶς τῆς γῆς σου, ἡνίκα ἂν ἀναβαίνῃς ὀφθῆναι ἐναντίον Κυρίου τοῦ Θεοῦ σου, τρεῖς καιροὺς τοῦ ἐνιαυτοῦ.
\VS{25}Οὐ σφάξεις ἐπὶ ζύμῃ αἷμα θυμιαμάτων μου, καὶ οὐ κοιμηθήσεται εἰς τὸ πρωῒ θύματα ἑορτῆς τοῦ πάσχα.
\VS{26}Τὰ πρωτογεννήματα τῆς γῆς σου θήσεις εἰς τὸν οἶκον Κυρίου τοῦ Θεοῦ σου· οὐχ ἑψήσεις ἄρνα ἐν γάλακτι μητρὸς αὐτοῦ.
\VS{27}Καὶ εἶπε Κύριος πρὸς Μωυσῆν, γράψον σεαυτῷ τὰ ῥήματα ταῦτα· ἐπὶ γὰρ τῶν λόγων τούτων τέθειμαί σοι διαθήκην, καὶ τῷ Ἰσραήλ.
\VS{28}Καὶ ἦν ἐκεῖ Μωυσῆς ἐναντίον Κυρίου τεσσεράκοντα ἡμέρας, καὶ τεσσεράκοντα νύκτας· ἄρτον οὐκ ἔφαγε, καὶ ὕδωρ οὐκ ἔπιε· καὶ ἔγραψεν ἐπὶ τῶν πλακῶν τὰ ῥήματα ταῦτα τῆς διαθήκης, τοὺς δέκα λόγους.
\par }{\PP \VS{29}Ὡς δὲ κατέβαινε Μωυσῆς ἐκ τοῦ ὄρους, καὶ αἱ δύο πλάκες ἐπὶ τῶν χειρῶν Μωυσῆ· καταβαίνοντος δὲ αὐτοῦ ἐκ τοῦ ὄρους, Μωυσῆς οὐκ ᾔδει ὅτι δεδόξασται ἡ ὄψις τοῦ χρώματος τοῦ προσώπου αὐτοῦ ἐν τῷ λαλεῖν αὐτὸν αὐτῷ.
\VS{30}Καὶ εἶδεν Ἀαρὼν καὶ πάντες οἱ πρεσβύτεροι Ἰσραὴλ τὸν Μωυσῆν, καὶ ἦν δεδοξασμένη ἡ ὄψις τοῦ χρώματος τοῦ προσώπου αὐτοῦ. καὶ ἐφοβήθησαν ἐγγίσαι αὐτῷ.
\VS{31}Καὶ ἐκάλεσεν αὐτοὺς Μωυσῆς, καὶ ἐπεστράφησαν πρὸς αὐτὸν Ἀαρὼν καὶ πάντες οἱ ἄρχοντες τῆς συναγωγῆς· καὶ ἐλάλησεν αὐτοῖς Μωυσῆς.
\par }{\PP \VS{32}Καὶ μετὰ ταῦτα προσῆλθον πρὸς αὐτὸν πάντες οἱ υἱοὶ Ἰσραήλ. καὶ ἐνετείλατο αὐτοῖς πάντα, ὅσα ἐνετείλατο Κύριος πρὸς αὐτὸν ἐν τῷ ὄρει Σινά.
\VS{33}Καὶ ἐπειδὴ κατέπαυσε λαλῶν πρὸς αὐτοὺς, ἐπέθηκεν ἐπὶ τὸ πρόσωπον αὐτοῦ κάλυμμα.
\VS{34}Ἡνίκα δʼ ἂν εἰσεπορεύετο Μωυσῆς, ἔναντι Κυρίου λαλεῖν αὐτῷ, περιῃρεῖτο τὸ κάλυμμα ἕως τοῦ ἐκπορεύεσθαι· καὶ ἐξελθὼν ἐλάλει πᾶσι τοῖς υἱοῖς Ἰσραὴλ ὅσα ἐνετείλατο αὐτῷ Κύριος.
\VS{35}Καὶ εἶδον οἱ υἱοὶ Ἰσραὴλ τὸ πρόσωπον Μωυσέως, ὅτι δεδόξασται· καὶ περιέθηκε Μωυσῆς κάλυμμα ἐπὶ τὸ πρόσωπον ἑαυτοῦ, ἕως ἂν εἰσέλθῃ συλλαλεῖν αὐτῷ.

\par }\Chap{35}{\PP \VerseOne{1}Καὶ συνήθροισε Μωυσῆς πᾶσαν συναγωγὴν υἱῶν Ἰσραὴλ, καὶ εἶπεν, οὗτοι οἱ λόγοι, οὓς εἶπε Κύριος ποιῆσαι αὐτούς.
\VS{2}Ἓξ ἡμέρας ποιήσεις ἔργα, τῇ δὲ ἡμέρᾳ τῇ ἑβδόμῃ κατάπαυσις· ἅγια, σάββατα· ἀνάπαυσις Κυρίῳ· πᾶς ὁ ποιῶν ἔργον ἐν αὐτῇ, τελευτάτω.
\VS{3}Οὐ καύσετε πῦρ ἐν πάσῃ κατοικίᾳ ὑμῶν τῇ ἡμέρᾳ τῶν σαββάτων· ἐγὼ Κύριος.
\VS{4}Καὶ εἶπε Μωυσῆς πρὸς πᾶσαν συναγωγὴν υἱῶν Ἰσραὴλ, λέγων, τοῦτο τὸ ῥῆμα, ὃ συνέταξε Κύριος, λέγων,
\VS{5}λάβετε παρʼ ὑμῶν αὐτῶν ἀφαίρεμα Κυρίῳ· πᾶς ὁ καταδεχόμενος τῇ καρδίᾳ, οἴσουσι τὰς ἀπαρχὰς Κυρίῳ, χρυσίον, ἀργύριον, χαλκὸν,
\VS{6}ὑάκινθον, πορφύραν, κόκκινον διπλοῦν διανενησμένον, καὶ βύσσον κεκλωσμένην, καὶ τρίχας αἰγείας,
\VS{7}καὶ δέρματα κριῶν ἠρυθροδανωμένα, καὶ δέρματα ὑακινθινα, καὶ ξύλα ἄσηπτα,
\VS{9}καὶ λίθους σαρδίου, καὶ λίθους εἰς τὴν γλυφὴν εἰς τὴν ἐπωμίδα καὶ τὸν ποδήρη.
\VS{10}Καὶ πᾶς σοφὸς τῇ καρδίᾳ ἐν ὑμῖν, ἐλθὼν ἐργαζέσθω πάντα ὅσα συνέταξε Κύριος·
\VS{11}Τὴν σκηνὴν, καὶ τὰ παραρύματα, καὶ τὰ κατακαλύμματα, καὶ τὰ διατόνια, καὶ τοὺς μοχλοὺς, καὶ τοὺς στύλους,
\VS{12}καὶ τὴν κιβωτὸν τοῦ μαρτυρίου, καὶ τοὺς ἀναφορεῖς αὐτῆς, καὶ τὸ ἱλαστήριον αὐτῆς, καὶ τὸ καταπέτασμα,
\VS{12a}καὶ τὰ ἱστία τῆς αὐλῆς, καὶ τοὺς στύλους αὐτῆς, καὶ τοὺς λίθους τοὺς τῆς σμαράγδου, καὶ τὸ θυμίαμα, καὶ τὸ ἔλαιον τοῦ χρίσματος,
\VS{13}καὶ τὴν τράπεζαν καὶ πάντα τὰ σκεύη αὐτῆς,
\VS{14}καὶ τὴν λυχνίαν τοῦ φωτὸς καὶ πάντα τὰ σκεύη αὐτῆς,
\VS{16}καὶ τὸ θυσιαστήριον καὶ πάντα τὰ σκεύη αὐτοῦ,
\VS{19}καὶ τὰς στολὰς τὰς ἁγίας Ἀαρὼν τοῦ ἱερέως, καὶ τὰς στολὰς ἐν αἷς λειτουργήσουσιν ἐν αὐταῖς, καὶ τοὺς χιτῶνας τοῖς υἱοῖς Ἀαρὼν τῆς ἱερατίας, καὶ τὸ ἔλαιον τοῦ χρίσματος, καὶ τὸ θυμίαμα τῆς συνθέσεως.
\par }{\PP \VS{20}Καὶ ἐξῆλθε πᾶσα συναγωγὴ υἱῶν Ἰσραὴλ ἀπὸ Μωυσῆ.
\VS{21}Καὶ ἤνεγκαν ἕκαστος, ὧν ἔφερεν ἡ καρδία αὐτῶν, καὶ ὅσοις ἔδοξε τῇ ψυχῇ αὐτῶν, ἀφαίρεμα· καὶ ἤνεγκαν ἀφαίρεμα Κυρίῳ εἰς πάντα τὰ ἔργα τῆς σκηνῆς τοῦ μαρτυρίου, καὶ εἰς πάντα τὰ κάτεργα αὐτῆς καὶ εἰς πάσας τὰς στολὰς τοῦ ἁγίου.
\VS{22}Καὶ ἤνεγκαν οἱ ἄνδρες παρὰ τῶν γυναικῶν, πᾶς ᾧ ἔδοξε τῇ διανοίᾳ, ἤνεγκαν σφραγίδας, καὶ ἐνώτια, καὶ δακτυλίους, καὶ ἐμπλόκια, καὶ περιδέξια, πᾶν σκεῦος χρυσοῦν.
\VS{23}Καὶ πάντες ὅσοι ἤνεγκαν ἀφαιρέματα χρυσίου Κυρίῳ, καὶ παρʼ ᾧ εὑρέθη βύσσος· καὶ δέρματα ὑακίνθινα καὶ δέρματα κριῶν ἠρυθροδανωμένα ἤνεγκαν.
\VS{24}Καὶ πᾶς ὁ ἀφαιρῶν ἀφαίρεμα, ἤνεγκαν ἀργύριον καὶ χαλκὸν, τὰ ἀφαιρέματα Κυρίῳ· καὶ παρʼ οἷς εὑρέθη ξύλα ἄσηπτα· καὶ εἰς πάντα τὰ ἔργα τῆς παρασκευῆς ἤνεγκαν.
\VS{25}Καὶ πᾶσα γυνὴ σοφὴ τῇ διανοίᾳ ταῖς χερσὶ νήθειν, ἤνεγκαν νενησμένα, τὴν ὑάκινθον, καὶ τὴν πορφύραν, καὶ τὸ κόκκινον, καὶ τὴν βύσσον.
\VS{26}Καὶ πᾶσαι αἱ γυναῖκες, αἷς ἔδοξε τῇ διανοίᾳ αὐτῶν ἐν σοφίᾳ, ἔνησαν τὰς τρίχας τὰς αἰγείας.
\VS{27}Καὶ οἱ ἄρχοντες ἤνεγκαν τοὺς λίθους τῆς σμαράγδου, καὶ τοὺς λίθους τῆς πληρώσεως εἰς τὴν ἐπωμίδα, καὶ τὸ λογεῖον,
\VS{28}καὶ τὰς συνθέσεις, καὶ εἰς τὸ ἔλαιον τῆς χρίσεως, καὶ τὴν συνθεσιν τοῦ θυμιάματος.
\VS{29}Καὶ πᾶς ἀνὴρ καὶ γυνὴ, ὧν ἔφερεν ἡ διάνοια αὐτῶν εἰσελθόντας ποιεῖν πάντα τὰ ἔργα, ὅσα συνέταξε Κύριος ποιῆσαι αὐτὰ διὰ Μωυσῆ, ἤνεγκαν οἱ υἱοὶ Ἰσραὴλ, ἀφαίρεμα Κυρίῳ.
\VS{30}Καὶ εἶπε Μωυσῆς τοῖς υἱοῖς Ἰσραὴλ, Ἰδοὺ ἀνακέκληκεν ὁ Θεὸς ἐξ ὀνόματος τὸν Βεσελεὴλ τὸν τοῦ Οὐρείου τὸν Ὢρ, ἐκ τῆς φυλῆς Ἰούδα,
\VS{31}καὶ ἐνέπλησεν αὐτὸν πνεῦμα θεῖον σοφίας καὶ συνέσεως, καὶ ἐπιστήμης πάντων,
\VS{32}ἀρχιτεκτονεῖν κατὰ πάντα τὰ ἔργα τῆς ἀρχιτεκτονίας, ποιεῖν τὸ χρυσίον καὶ τὸ ἀργύριον καὶ τὸν χαλκόν,
\VS{33}καὶ λιθουργῆσαι τὸν λίθον, καὶ κατεργάζεσθαι τὰ ξύλα, καὶ ποιεῖν ἐν παντὶ ἔργῳ σοφίας.
\VS{34}Καὶ προβιβάσαι γε ἔδωκεν ἐν τῇ διανοίᾳ αὐτῷ τε, καὶ τῷ Ἐλιὰβ τῷ τοῦ Ἀχισαμὰκ, ἐκ φυλῆς Δάν·
\VS{35}Καὶ ἐνέπλησεν αὐτοὺς σοφίας, συνέσεως, διανοίας, πάντα συνιέναι ποιῆσαι τὰ ἔργα τοῦ ἁγίου, καὶ τὰ ὑφαντὰ καὶ ποικιλτὰ ὑφάναι τῷ κοκκίνῳ, καὶ τῇ βύσσῳ, ποιεῖν πᾶν ἔργον ἀρχιτεκτονίας, ποικιλίας.

\par }\Chap{36}{\PP \VerseOne{1}Καὶ ἐποίησε Βεσελεὴλ καὶ Ἐλιὰβ, καὶ πᾶς σοφὸς τῇ διανοὶᾳ, ᾧ ἐδόθη σοφία καὶ ἐπιστήμη ἐν αὑτοῖς, συνιέναι ποιεῖν πάντα τὰ ἔργα, κατὰ τὰ ἅγια καθήκοντα, κατὰ πάντα ὅσα συνέταξε Κύριος.
\VS{2}Καὶ ἐκάλεσε Μωυσῆς Βεσελεὴλ καὶ Ἐλιὰβ, καὶ πάντας τοὺς ἔχοντας τὴν σοφίαν, ᾧ ἔδωκεν ὁ Θεὸς ἐπιστήμην ἐν τῇ καρδίᾳ, καὶ πάντας τοὺς ἑκουσίως βουλομένους προσπορεύεσθαι πρὸς τὰ ἔργα, ὥστε συντελεῖν αὐτά.
\VS{3}Καὶ ἔλαβον παρὰ Μωσῆ πάντα τὰ ἀφαιρέματα, ἃ ἤνεγκαν οἱ υἱοὶ Ἰσραὴλ εἰς πάντα τὰ ἔργα τοῦ ἁγίου ποιεῖν αὐτά· καὶ αὐτοὶ προσεδέχοντο ἔτι τὰ προσφερόμενα παρὰ τῶν φερόντων τὸ πρωΐ.
\VS{4}Καὶ παρεγίνοντο πάντες οἱ σοφοὶ οἱ ποιοῦντες τὰ ἔργα τοῦ ἁγίου, ἕκαστος κατὰ τὸ αὐτοῦ ἔργον, ὃ εἰργάζοντο αὐτοί.
\VS{5}Καὶ εἶπε πρὸς Μωυσῆν, ὅτι πλῆθος φέρει ὁ λαὸς κατὰ τὰ ἔργα ὅσα συνέταξε Κύριος ποιῆσαι.
\VS{6}Καὶ προσέταξε Μωυσῆς, καὶ ἐκήρυξεν ἐν τῇ παρεμβολῇ, λέγων, ἀνὴρ καὶ γυνὴ μηκέτι ἐργαζέσθωσαν εἰς τάς ἀπαρχὰς τοῦ ἁγίου· καὶ ἐκωλύθη ὁ λαὸς ἔτι προσφέρειν.
\VS{7}Καὶ τὰ ἔργα ἦν αὐτοῖς ἱκανὰ εἰς τὴν κατασκευὴν ποιῆσαι, καὶ προσκατέλιπον.
\VS{8}Καὶ ἐποίησε πᾶς σοφὸς ἐν τοῖς ἐργαζομένοις τὰς στολὰς τῶν ἁγίων, αἵ εἰσιν Ἀαρὼν τῷ ἱερεῖ, καθὰ συνέταξε Κύριος τῷ Μωυσῇ.
\VS{9}Καὶ ἐποίησε τὴν ἐπωμίδα ἐκ χρυσίου, καὶ ὑακίνθου, καὶ πορφύρας, καὶ κοκκίνου νενησμένου, καὶ βύσσου κεκλωσμένης·
\VS{10}καὶ ἐτμήθη τὰ πέταλα τοῦ χρυσίου τρίχες, ὥστε συνυφάναι σὺν τῇ ὑακίνθῳ, καὶ τῇ πορφύρᾳ, καὶ σὺν τῷ κοκκίνῳ τῷ διανενησμένῳ, καὶ τῇ βύσσῳ τῇ κεκλωσμένῃ· ἔργον ὑφαντὸν ἐποίησαν αὐτό·
\VS{11}ἐπωμίδας συνεχούσας ἐξ ἀμφοτέρων τῶν μερῶν, ἔργον ὑφαντὸν εἰς ἄλληλα συμπεπλεγμένα καθʼ ἑαυτό.
\VS{12}Ἐξ αὐτοῦ ἐποίησαν αὐτὸ κατὰ τὴν αὐτοῦ ποίησιν, ἐκ χρυσίου, καὶ ὑακίνθου, καὶ πορφύρας, καὶ κοκκίνου διανενησμένου, καὶ βύσσου κεκλωσμένης, καθὰ συνέταξε Κύριος τῷ Μωυσῇ·
\VS{13}καὶ ἐποίησαν ἀμφοτέρους τοὺς λίθους τῆς σμαράγδου συνπεπορπημένους καὶ περισεσιαλωμένους χρυσίῳ, γεγλυμμένους καὶ ἐκκεκολαμμένους ἐγκόλαμμα σφραγίδος ἐκ τῶν ὀνομάτων τῶν υἱῶν Ἰσραήλ·
\VS{14}καὶ ἐπέθηκεν αὐτοὺς ἐπὶ τοὺς ὤμους τῆς ἐπωμίδος, λίθους μνημοσύνου τῶν υἱῶν Ἰσραὴλ, καθὰ συνέταξε Κύριος τῷ Μωυσῇ.
\par }{\PP \VS{15}Καὶ ἐποίησαν λογεῖον, ἔργον ὑφαντὸν ποικιλίᾳ κατὰ τὸ ἔργον τῆς ἐπωμίδος, ἐκ χρυσίου, καὶ ὑακίνθου, καὶ πορφύρας, καὶ κοκκίνου διανενησμένου, καὶ βύσσου κεκλωσμένης·
\VS{16}τετράγωνον διπλοῦν ἐπόησαν τὸ λογεῖον· σπιθαμῆς τὸ μῆκος, καὶ σπιθαμῆς τὸ εὖρος διπλοῦν.
\VS{17}Καὶ συνυφάνθη ἐν αὐτῷ ὕφασμα κατάλιθον τετράστιχον· στίχος λίθων, σάρδιον καὶ τοπάζιον καὶ σμάραγδος, ὁ στίχος ὁ εἷς·
\VS{18}καὶ ὁ στίχος ὁ δεύτερος, ἄνθραξ καὶ σάπφειρος καὶ ἴασπις·
\VS{19}καὶ ὁ στίχος ὁ τρίτος, λιγύριον καὶ ἀχάτης καὶ ἀμέθυστος·
\VS{20}καὶ ὁ στίχος ὁ τέταρτος, χρυσόλιθος καὶ βηρύλλιον καὶ ὀνύχιον περικεκυκλωμένα χρυσίῳ, καὶ συνδεδεμένα χρυσίῳ.
\VS{21}Καὶ οἱ λίθοι ἦσαν ἐκ τῶν ὀνομάτων τῶν υἱῶν Ἰσραὴλ δώδεκα, ἐκ τῶν ὀνομάτων αὐτῶν ἐγγεγλυμμένα εἰς σφραγίδας, ἕκαστος ἐκ τοῦ ἑαυτοῦ ὀνόματος εἰς τὰς δώδεκα φυλάς.
\VS{22}Καὶ ἐποίησαν ἐπὶ τὸ λογεῖον κρωσσοὺς συμπεπλεγμένους, ἔργον ἐμπλοκίου, ἐκ χρυσίου καθαροῦ.
\VS{23}Καὶ ἐποίησαν δύο ἀσπιδίσκας χρυσᾶς, καὶ δύο δακτυλίους χρυσοῦς· καὶ ἐπέθηκαν τοὺς δύο δακτυλίους τοὺς χρυσοῦς ἐπʼ ἀμφοτέρας τὰς ἀρχὰς τοῦ λογείου.
\VS{24}Καὶ ἐπέθηκαν τὰ ἐμπλόκια ἐκ χρυσίου ἐπὶ τοὺς δακτυλίους ἐπʼ ἀμφοτέρων τῶν μερῶν τοῦ λογείου·
\VS{25}καὶ εἰς τὰς δύο συμβολὰς τὰ δύο ἐμπλόκια. Καὶ ἐπέθηκαν ἐπὶ τὰς δύο ἀσπιδίσκας· καὶ ἐπέθηκαν ἐπὶ τοὺς ὤμους τῆς ἐπωμίδος ἐξεναντίας κατὰ πρόσωπον.
\VS{26}Καὶ ἐποίησαν δύο δακτυλίους χρυσοῦς, καὶ ἐπέθηκαν ἐπὶ τὰ δύο πτερύγια ἐπʼ ἄκρου τοῦ λογείου, καὶ ἐπὶ τὸ ἄκρον τοῦ ὀπισθίου τῆς ἐπωμίδος ἔσωθεν·
\VS{27}Καὶ ἐποίησαν δύο δακτυλίους χρυσοῦς, καὶ ἐπέθηκαν ἐπʼ ἀμφοτέρους τοὺς ὤμους τῆς ἐπωμίδος κάτωθεν αὐτοῦ, κατὰ πρόσωπον κατὰ τὴν συμβολὴν ἄνωθεν τῆς συνυφῆς τῆς ἐπωμίδος·
\VS{28}καὶ συνέσφιγξε τὸ λογεῖον ἀπὸ τῶν δακτυλίων τῶν ἐπʼ αὐτοῦ εἰς τοὺς δακτυλίους τῆς ἐπωμίδος, συνεχομένους ἐκ τῆς ὑακίνθου, συμπεπλεγμένους εἰς τὸ ὕφασμα τῆς ἐπωμίδος, ἵνα μὴ χαλᾶται τὸ λογεῖον ἀπὸ τῆς ἐπωμίδος, καθὰ συνέταξε Κύριος τῷ Μωυσῇ.
\VS{29}Καὶ ἐποίησαν τὸν ὑποδύτην ὑπὸ τὴν ἐπωμίδα, ἔργον ὑφαντὸν, ὅλον ὑακίνθινον·
\VS{30}τὸ δὲ περιστόμιον τοῦ ὑποδύτου ἐν τῷ μέσῳ διυφασμένον συμπλεκτὸν, ὤαν ἔχον κύκλῳ τὸ περιστόμιονν ἀδιάλυτον·
\VS{31}Καὶ ἐποίησαν ἐπὶ τοῦ λώματος τοῦ ὑποδύτου κάτωθεν ὡς ἐξανθούσης ῥόας ῥοΐσκους, ἐξ ὑακίνθου, καὶ πορφύρας, καὶ κοκκίνου νενησμένου, καὶ βύσσου κεκλωσμένης.
\VS{32}Καὶ ἐποίησαν κώδωνας χρυσοῦς, καὶ ἐπέθηκαν τοὺς κώδωνας ἐπὶ τὸ λῶμα τοῦ ὑποδύτου κύκλῳ ἀνὰ μέσον τῶν ῥοΐσκων·
\VS{33}κώδων χρυσοῦς καὶ ῥοΐσκος ἐπὶ τοῦ λώματος τοῦ ὑποδύτου κύκλῳ, εἰς τὸ λειτουργεῖν, καθὰ συνέταξε Κύριος τῷ Μωυσῇ.
\VS{34}Καὶ ἐποίησαν χιτῶνας βυσσίνους, ἔργον ὑφαντὸν, Ἀαρὼν καὶ τοῖς υἱοῖς αὐτοῦ,
\VS{35}καὶ τὰς κιδάρεις ἐκ βύσσου, καὶ τὴν μίτραν ἐκ βύσσου, καὶ τὰ περισκελῆ ἐκ βύσσου κεκλωσμένης,
\VS{36}καὶ τὰς ζώνας αὐτῶν ἐκ βύσσου, καὶ ὑακίνθου, καὶ πορφύρας, καὶ κοκκίνου νενησμένου, ἔργον ποικιλτοῦ, ὃν τρόπον συνέταξε Κύριος τῷ Μωυσῇ.
\VS{37}Καὶ ἐποίησαν τὸ πέταλον τὸ χρυσοῦν, ἀφόρισμα τοῦ ἁγίου, χρυσίου καθαροῦ· καὶ ἔγραψεν ἐπʼ αὐτοῦ γράμματα ἐκτετυπωμένα, σφραγίδος, Ἁγίασμα Κυρίῳ·
\VS{38}Καὶ ἐπέθηκαν ἐπὶ τὸ λῶμα ὑακίνθινον, ὥστε ἐπικεῖσθαι ἐπὶ τὴν μίτραν ἄνωθεν, ὅν τρόπον συνέταξε Κύριος τῷ Μωυσῇ.

\par }\Chap{37}{\PP \VerseOne{1}Καὶ ἐποίησαν τῇ σκηνῇ δέκα αὐλαίας·
\VS{2}ὀκτὼ καὶ εἴκοσι πήχεων μῆκος τῆς αὐλαίας τῆς μιᾶς· τὸ αὐτὸ ἦν πάσαις· καὶ τεσσάρων πήχεων τὸ εὖρος τῆς αὐλαίας τῆς μιᾶς.
\VS{3}καὶ ἐποίησαν τὸ καταπέτασμα ἐξ ὑακίνθου, καὶ πορφύρας, καὶ κοκκίνου νενησμένου, καὶ βύσσου κεκλωσμένης, ἔργον ὑφαντὸυ χερουβείμ·
\VS{4}καὶ ἐπέθηκαν αὐτὸ ἐπὶ τέσσαρας στύλους ἀσήπτους κατακεχρυσωμένους ἐν χρυσίῳ· καὶ αἱ κεφαλίδες αὐτῶν χρυσαῖ, καὶ αἱ βάσεις αὐτῶν τέσσαρες ἀργυραῖ.
\VS{5}καὶ ἐποίησαν τὸ καταπέτασμα τῆς θύρας τῆς σκηνῆς τοῦ μαρτυρίου ἐξ ὑακίνθου, καὶ πορφύρας, καὶ κοκκίνου νενησμένου, καὶ βύσσου κεκλωσμένης, ἔργον ὑφαντὸντοῦ χερουβείμ·
\VS{6}καὶ τοὺς στύλους αὐτῶν πέντε, καὶ τοὺς κρίκους· καὶ τὰς κεφαλίδας αὐτῶν, καὶ τὰς ψαλίδας αὐτῶν κατεχρύσωσαν χρυσίῳ· καὶ αἱ βάσεις αὐτῶν πέντε χαλκαῖ.
\par }{\PP \VS{7}Καὶ ἐποίησαν τὴν αὐλῆν τὰ πρὸς Λίβα, ἱστία τῆς αὐλῆς ἐκ βύσσου κεκλωσμένης ἑκατὸν ἐφʼ ἑκατόν·
\VS{8}καὶ οἱ στύλοι αὐτῶν εἴκοσι, καὶ αἱ βάσεις αὐτῶν εἴκοσι.
\VS{9}καὶ τὸ κλίτος τὸ πρὸς Βοῤῥᾶν, ἑκατὸν ἐφʼ ἑκατόν· καὶ τὸ κλίτος τὸ πρὸς Νότον, ἑκατὸν ἐφʼ ἑκατόν· καὶ οἱ στύλοι αὐτῶν εἴκοσι, καὶ αἱ βάσεις αὐτῶν εἴκοσι·
\VS{10}Καὶ τὸ κλίτος τὸ πρὸς θάλασσαν αὐλαῖαι πεντήκοντα πήχεων· στύλοι αὐτῶν δέκα, καὶ αἱ βάσεις αὐτῶν δέκα·
\VS{11}Καὶ τὸ κλίτος τὸ πρὸς ἀνατολὰς πεντήκοντα πήχεων ἱστία, πεντεαίδεκα πήχεων τὸ κατὰ νώτου·
\VS{12}καὶ οἱ στύλοι αὐτῶν τρεῖς, καὶ αἱ βάσεις αὐτῶν τρεῖς·
\VS{13}Καὶ ἐπὶ τοῦ νώτου τοῦ δευτέρου ἔνθεν καὶ ἔνθεν κατὰ τὴν πύλην τῆς αὐλῆς, αὐλαῖαι πεντεκαίδεκα πήχεων· στύλοι αὐτῶν τρεῖς, καὶ αἱ βάσεις αὐτῶν τρεῖς·
\VS{14}πᾶσαι αἱ αὐλαῖαι τῆς σκηνῆς ἐκ βύσσου κεκλωσμένης.
\VS{15}Καὶ αἱ βάσεις τῶν στύλων αὐτῶν χαλκαῖ, καὶ αἱ ἀγκύλαι αὐτῶν ἀργυραῖ, καὶ αἱ κεφαλίδες αὐτῶν περιηργυρωμέναι ἀργυρίῳ, καὶ οἱ στύλοι περιηργυρωμένοι ἀργυρίῳ πάντες οἱ στύλοι τῆς αὐλῆς·
\VS{16}καὶ τὸ καταπέτασμα τῆς πύλης τῆς αὐλῆς ἔργον ποικιλτοῦ ἐξ ὑακίνθου, καὶ πορφύρας, καὶ κοκκίνου νενησμένου, καὶ βύσσου κεκλωσμένης· εἴκοσι πήχεων τὸ μῆκος, καὶ τὸ ὕψος καὶ τὸ εὖρος πέντε πήχεων ἐξισούμενον τοῖς ἱστίοις τῆς αὐλῆς·
\VS{17}καὶ οἱ στύλοι αὐτῶν τέσσαρες, καὶ αἱ βάσεις αὐτῶν τέσσαρες χαλκαῖ, καὶ αἱ ἀγκύλαι αὐτῶν ἀργυραῖ, καὶ αἱ κεφαλίδες αὐτῶν περιηργυρωμέναι ἀργυρίῳ.
\VS{18}Καὶ πάντες οἱ πάσσαλοι τῆς αὐλῆς κύκλῳ χαλκοῖ, καὶ αὐτοὶ περιηργυρωμένοι ἀργυρίῳ.
\VS{19}Καὶ αὕτη ἡ σύνταξις τῆς σκηνῆς τοῦ μαρτυρίου, καθὰ συνετάγη Μωυσῇ, τὴν λειτουργίαν εἶναι τῶν Λευιτῶν διὰ Ἰθάμαρ τοῦ υἱοῦ Ἀαρὼν τοῦ ἱερέως.
\par }{\PP \VS{20}Καὶ Βεσελεὴλ ὁ τοῦ Οὐρείου, ἐκ φυλῆς Ἰούδα, ἐποίησε καθὰ συνέταξε Κύριος τῷ Μωυσῇ,
\VS{21}καὶ Ἐλιὰβ ὁ τοῦ Ἀχισαμὰχ ἐκ φυλῆς Δὰν, ὅς ἠρχιτεκτόνησε τὰ ὑφαντὰ καὶ τὰ ῥαφιδευτὰ καὶ ποικιλτικά, ὑφάναι τῷ κοκκίνῳ καὶ τῇ βύσσῳ.

\par }\Chap{38}{\PP \VerseOne{1}Καὶ ἐποίησε Βεσελεὴλ τὴν κιβωτόν,
\VS{2}καὶ κατεχρύσωσεν αὐτὴν χρωσίῳ καθαρῷ ἔσωθεν καὶ ἔξωθεν·
\VS{3}καὶ ἐχώνευσεν αὐτῇ τέσσαρας δακτυλίους χρυσοῦς· δύο ἐπὶ τὸ κλίτος τὸ ἓν, καὶ δύο ἐπὶ τὸ κλίτος τὸ δεύτερον,
\VS{4}εὐρεῖς τοῖς διωστῆρσιν, ὥστε αἴρειν αὐτὴν ἐν αὐτοῖς.
\VS{5}Καὶ ἐποίησε τὸ ἱλαστήριον ἐπάνωθεν τῆς κιβωτοῦ ἐκ χρυσίου καθαροῦ,
\VS{6}καὶ τοὺς δύο χερουβεὶμ χρυσοῦς·
\VS{7}χεροὺβ ἕνα ἐπὶ τὸ ἄκρον τοῦ ἱλαστηρίου τὸ ἓν, καὶ χεροὺβ ἕνα ἐπὶ τὸ ἄκρον τοῦ ἱλαστηρίου τὸ δεύτερον,
\VS{8}σκιάζοντα ταῖς πτέρυξιν αὐτῶν ἐπὶ τὸ ἱλαστήριον.
\VS{9}Καὶ ἐποίησε τὴν τράπεζαν τὴν προκειμένην ἐκ χρυσίου καθαροῦ,
\VS{10}καὶ ἐχώνευσεν αὐτῇ τέσσαρας δακτυλίους, δύο ἐπὶ τοῦ κλίτους τοῦ ἑνὸς, καὶ δύο ἐπὶ τοῦ κλίτους τοῦ δευτέρου, εὐρεῖς, ὥστε αἴρειν τοῖς διωστῆρσιν ἐν αὐτοῖς.
\VS{11}Καὶ τοὺς διωστῆρας τῆς κιβωτοῦ καὶ τῆς τραπέζης ἐποίησε, καὶ κατεχρύσωσεν αὐτοὺς χρυσίῳ.
\VS{12}Καὶ ἐποίησε τὰ σκεύη τῆς τραπέζης, τά τε τρυβλία, καὶ τὰς θυίσκας, καὶ τοὺς κυάθους, καὶ τὰ σπονδεῖα, ἐν οἷς σπείσει ἐν αὐτοῖς, χρυσᾶ.
\VS{13}Καὶ ἐποίησε τὴν λυχνίαν ἣ φωτίζει, χρυσῆν,
\VS{14}στερεὰν τὸν καυλόν, καὶ τοὺς καλαμίσκους ἐξ ἀμφοτέρων τῶν μερῶν αὐτῆς·
\VS{15}ἐκ τῶν καλαμίσκων αὐτῆς οἱ βλαστοὶ ἐκ τῶν καλαμίσκων αὐτῆς οἱ βλαστοὶ ἐξέχοντες· τρεῖς ἐκ τούτου, καὶ τρεῖς ἐκ τούτου, ἐξισούμενοι ἀλλήλοις.
\VS{16}Καὶ τὰ λαμπάδια αὐτῶν, ἅ ἐστιν ἐπὶ τῶν ἄκρων, καρυωτὰ ἐξ αὐτῶν· καὶ τὰ ἐνθέμια ἐξ αὐτῶν, ἵνα ὦσιν οἱ λύχνοι ἐπʼ αὐτῶν· καὶ τὸ ἐνθέμιον τὸ ἕβδομον, τὸ ἐπʼ ἄκρου τοῦ λαμπαδίου, ἐπὶ τῆς κορυφῆς ἄνωθεν, στερεὸν ὅλον χρυσοῦν·
\VS{17}Καὶ ἑπτὰ λύχνους ἐπʼ αὐτῆς χρυσοῦς, καὶ τὰς λαβίδας αὐτῆς χρυσᾶς, καὶ τὰς ἐπαρυστρίδας αὐτῶν χρυσᾶς.
\VS{18}Οὗτος περιηργύρωσε τοὺς στύλους, καὶ ἐχώνευσε τῷ στύλῳ δακτυλίους χρυσοῦς, καὶ ἐχρύσωσε τοὺς μοχλοὺς χρυσίῳ, καὶ κατεχρύσωσε τοὺς στύλους τοῦ καταπετάσματος χρυσίῳ, καὶ ἐποίησε τὰς ἀγκύλας χρυσᾶς.
\VS{19}Οὗτος ἐποίησε καὶ τοὺς κρίκους τῆς σκηνῆς χρυσοῦς, καὶ τοὺς κρίκους τῆς αὐλῆς, καὶ κρίκους εἰς τὸ ἐκτείνειν τὸ κατακάλυμμα ἄνωθεν χαλκοῦς·
\VS{20}Οὗτος ἐχώνευσε τὰς κεφαλίδας τὰς ἀργυρᾶς τῆς σκηνῆς, καὶ τὰς κεφαλίδας τὰς χαλκᾶς τῆς θύρας τῆς σκηνῆς, καὶ τὴν πύλην τῆς αὐλῆς· καὶ ἀγκύλας ἐποίησε τοῖς στύλοις ἀργυρᾶς, ἐπὶ τῶν στύλων οὗτος περιηργύρωσεν αὐτάς·
\VS{21}Οὗτος ἐποίησε τοὺς πασσάλους τῆς σκηνῆς, καὶ τοὺς πασσάλους τῆς αὐλῆς χαλκοῦς·
\VS{22}Οὗτος ἐποίησε τὸ θυσιαστήριον τὸ χαλκοῦν ἐκ τῶν πυρείων τῶν χαλκῶν, ἃ ἦσαν τοῖς ἀνδράσιν τοῖς καταστασιάσασι μετὰ τῆς Κορὲ συναγωγῆς·
\VS{23}Οὗτος ἐποίησε πάντα τὰ σκεύη τοῦ θυσιαστηρίου, καὶ τὸ πυρεῖον αὐτοῦ, καὶ τὴν βάσιν, καὶ τὰς φιάλας, καὶ τὰς κρεάγρας τὰς χαλκᾶς·
\VS{24}Οὗτος ἐποίησε θυσιαστηρίῳ παράθεμα, ἔργον δικτυωτὸν κάτωθεν τοῦ πυρείου ὑπὸ αὐτὸ ἕως τοῦ ἡμίσους αὐτοῦ· καὶ ἐπέθηκεν αὐτῷ τέσσαρας δακτυλίους ἐκ τῶν τεσσάρων μερῶν τοῦ παραθέματος τοῦ θυσιαστηρίου χαλκοῦς, εὐρεῖς τοῖς μοχλοῖς, ὥστε αἴρειν ἐν αὐτοῖς τὸ θυσιαστήριον·
\VS{25}Οὗτος ἐποίησε τὸ ἔλαιον τῆς χρίσεως τὸ ἅγιον, καὶ τὴν σύνθεσιν τοῦ θυμιάματος καθαρὸν ἔργον μυρεψοῦ·
\VS{26}Οὗτος ἐποίησε τὸν λουτῆρα τὸν χαλκοῦν, καὶ τὴν βάσιν αὐτοῦ χαλκῆν ἐκ τῶν κατόπτρων τῶν νηστευσασῶν, αἳ ἐνήστευσαν παρὰ τὰς θύρας τῆς σκηνῆς τοῦ μαρτυρίου, ἐν ᾗ ἡμέρᾳ ἔπηξεν αὐτήν.
\par }{\PP \VS{27}Καὶ ἐποίησε τὸν λουτῆρα, ἵνα νίπτωνται ἐξ αὐτοῦ Μωυσῆς καὶ Ἀαρὼν καὶ οἱ υἱοὶ αὐτοῦ τὰς χεῖρας αὐτῶν καὶ τοὺς πόδας, εἰσπορευομένων αὐτῶν εἰς τὴν σκηνὴν τοῦ μαρτυρίου, ἢ ὅταν προσπορεύωνται πρὸς τὸ θυσιαστήριον λειτουργεῖν, ἐνίπτοντο ἐξ αὐτοῦ, καθάπερ συνέταξε Κύριος τῷ Μωυσῇ.

\par }\Chap{39}{\PP \VerseOne{1}Πᾶν τὸ χρυσίον, ὃ κατειργάσθη εἰς τὰ ἔργα κατὰ πᾶσαν τὴν ἐργασίαν τῶν ἁγίων, ἐγένετο χρυσίου τοῦ τῆς ἀπαρχῆς, ἐννέα καὶ εἴκοσι τάλαντα, καὶ ἑπτακόσιοι εἴκοσι σίκλοι κατὰ τὸν σίκλον τὸν ἅγιον·
\VS{2}Καὶ ἀργυρίου ἀφαίρεμα παρὰ τῶν ἐπεσκεμμένων ἀνδρῶν τῆς συναγωγῆς ἑκατὸν τάλαντα, καὶ χίλιοι ἑπτακόσιοι ἑβδομηκονταπέντε σίκλοι· δραχμὴ μία τῇ κεφαλῇ τὸ ἥμισυ τοῦ σίκλου, κατὰ τὸν σίκλον τὸν ἅγιον·
\VS{3}Πᾶς ὁ παραπορευόμενος τὴν ἐπίσκεψιν ἀπὸ εἰκοσαετοῦς καὶ ἐπάνω εἰς τὰς ἑξήκοντα μυριάδας, καὶ τρισχίλιοι πεντακόσιοι καὶ πεντήκοντα.
\VS{4}Καὶ ἐγενήθη τὰ ἑκατὸν τάλαντα τοῦ ἀργυρίου εἰς τὴν χώνευσιν τῶν ἑκατὸν κεφαλίδων τῆς σκηνῆς, καὶ εἰς τὰς κεφαλίδας τοῦ καταπετάσματος,
\VS{5}ἑκατὸν κεφαλίδες εἰς τὰ ἑκατὸν τάλαντα, τάλαντον τῇ κεφαλίδι·
\VS{6}Καὶ τοὺς χιλίους ἑπτακοσίους ἑβδομηκοντα πέντε σίκλους ἐποίησεν εἰς τὰς ἀγκύλας τοῖς στύλοις· καὶ κατεχρύσωσε τὰς κεφαλίδας αὐτῶν, καὶ κατεκόσμησεν αὐτούς.
\par }{\PP \VS{7}Καὶ ὁ χαλκὸς τοῦ ἀφαιρέματος ἑβδομήκοντα τάλαντα, καὶ χίλιοι πεντακόσιοι σίκλοι·
\VS{8}Καὶ ἐποίησαν ἐξ αὐτον τὰς βάσεις τῆς θύρας τῆς σκηνῆς τοῦ μαρτυρίου,
\VS{9}καὶ τὰς βάσεις τῆς αὐλῆς κύκλῳ, καὶ τὰς βάσεις τῆς πύλης τῆς αὐλῆς, καὶ τοὺς πασσάλους τῆς σκηνῆς, καὶ τοὺς πασσάλους τῆς αὐλῆς κύκλῳ,
\VS{10}καὶ τὸ παράθεμα τὸ χαλκοῦν τοῦ θυσιαστηρίου, καὶ πάντα τὰ σκεύη τοῦ θυσιαστηρίου, καὶ πάντα τὰ ἐργαλεῖα τῆς σκηνῆς τοῦ μαρτυρίου·
\VS{11}Καὶ ἐποίησαν οἱ υἱοὶ Ἰσραὴλ, καθὰ συνέταξε Κύριος τῷ Μωυσῇ, οὕτως ἐποίησαν·
\VS{12}Τὸ δὲ λοιπὸν χρυσίον τοῦ ἀφαιρέματος ἐποίησαν σκεύη εἰς τὸ λειτουργεῖν ἐν αὐτοῖς ἔναντι Κυρίου·
\VS{13}Καὶ τὴν καταλειφθεῖσαν ὑάκινθον, καὶ πορφύραν, καὶ τὸ κόκκινον ἐποίησαν στολὰς λειτουργικὰς Ἀαρών, ὥστε λειτουργεῖν ἐν αὐταῖς ἐν τῷ ἁγίῳ·
\VS{14}Καὶ ἤνεγκαν τὰς στολὰς πρὸς Μωυσῆν, καὶ τὴν σκηνὴν, καὶ τὰ σκεύη αὐτῆς, τὰς βάσεις καὶ τοὺς μοχλοὺς αὐτῆς, καὶ τοὺς στύλους·
\VS{15}καὶ τὸ θυσιαστήριον, καὶ πάντα τὰ σκεύη αὐτοῦ.
\par }{\PP \VS{16}Καὶ τὸ ἔλαιον τῆς χρίσεως, καὶ τὸ θυμίαμα τῆς συνθέσεως, καὶ τὴν λυχνίαν τὴν καθαρὰν,
\VS{17}καὶ τοὺς λύχνους αὐτῆς, λύχνους τῆς καύσεως, καὶ τὸ ἔλαιον τοῦ φωτός·
\VS{18}Καὶ τὴν τράπεζαν τῆς προθέσεως, καὶ πάντα τὰ σκεύη αὐτῆς· καὶ τοὺς ἄρτους τοὺς προκειμένους·
\VS{19}Καὶ τὰς στολὰς τοῦ ἁγίου, αἵ εἰσιν Ἀαρών, καὶ τὰς στολὰς τῶν υἱῶν αὐτοῦ, εἰς τὴν ἱερατείαν·
\VS{20}Καὶ τὰ ἱστία τῆς αὐλῆς, καὶ τοὺς στύλους· καὶ τὸ καταπέτασμα τῆς θύρας τῆς σκηνῆς, καὶ τῆς πύλης τῆς αὐλῆς·
\VS{21}Καὶ πάντα τὰ σκεύη τῆς σκηνῆς, καὶ πάντα τὰ ἐργαλεῖα αὐτῆς· καὶ τὰς διφθέρας δέρματα κριῶν ἠρυθροδανωμένα, καὶ τὰ καλύμματα ὑακίνθινα, καὶ τῶν λοιπῶν τὰ ἐπικαλύμματα· καὶ τοὺς πασσάλους, καὶ πάντα τὰ ἐργαλεῖα τὰ εἰς τὰ ἔργα τῆς σκηνῆς τοῦ μαρτυρίου·
\VS{22}Ὃσα συνέταξε Κύριος τῷ Μωυσῇ, οὕτως ἐποίησαν οἱ υἱοὶ Ἰσραὴλ πᾶσαν τὴν ἀποσκευήν·
\VS{23}Καὶ εἶδε Μωυσῆς πάντα τὰ ἔργα, καὶ ἦσαν πεποιηκότες αὐτὰ ὃν τρόπον συνέταξε Κύριος τῷ Μωυσῇ, οὕτως ἐποίησαν αὐτὰ, καὶ εὐλόγησεν αὐτοὺς Μωυσῆς.

\par }\Chap{40}{\PP \VerseOne{1}Καὶ ἐλάλησε Κύριος πρὸς Μωυσῆν, λέγων,
\VS{2}ἐν ἡμέρᾳ μιᾷ τοῦ μηνὸς τοῦ πρώτου νουμηνίᾳ, στήσεις τὴν σκηνὴν τοῦ μαρτυρίου.
\VS{3}Καὶ θήσεις τὴν κιβωτὸν τοῦ μαρτυρίου, καὶ σκεπάσεις τὴν κιβωτὸν τῷ καταπετάσματι.
\VS{4}Καὶ εἰσοίσεις τὴν τράπεζαν, καὶ προθήσεις τὴν πρόθεσιν αὐτῆς· καὶ εἰσοίσεις τὴν λυχνίαν, καὶ ἐπιθήσεις τοὺς λύχνους αὐτῆς.
\VS{5}Καὶ θήσεις τὸ θυσιαστήριον τὸ χρυσοῦν, εἰς τὸ θυμιᾷν ἐναντίον τῆς κιβωτοῦ· καὶ ἐπιθήσεις κάλυμμα καταπετάσματος ἐπὶ τὴν θύραν τῆς σκηνῆς τοῦ μαρτυρίου.
\VS{6}Καὶ τὸ θυσιαστήριον τῶν καρπωμάτων θήσεις παρὰ τὰς θύρας τῆς σκηνῆς τοῦ μαρτυρίου·
\VS{8}καὶ περιθήσεις τὴν σκηνήν, καὶ πάντα τὰ αὐτῆς ἁγιάσεις κύκλῳ.
\VS{9}Καὶ λήψῃ τὸ ἔλαιον τοῦ χρίσματος, καὶ χρίσεις τὴν σκηνὴν, καὶ πάντα τὰ ἐν αὐτῇ, καὶ ἁγιάσεις αὐτὴν, καὶ πάντα τὰ σκεύη αὐτῆς, καὶ ἔσται ἁγία.
\VS{10}Καὶ χρίσεις τὸ θυσιαστήριον τῶν καρπωμάτων, καὶ πάντα τὰ σκεύη αὐτοῦ· καὶ ἁγιάσεις τὸ θυσιαστήριον, καὶ ἔσται τὸ θυσιαστήριον ἅγιον τῶν ἁγίων.
\VS{12}Καὶ προσάξεις Ἀαρὼν καὶ τοὺς υἱοὺς αὐτοῦ ἐπὶ τὰς θύρας τῆς σκηνῆς τοῦ μαρτυρίου, καὶ λούσεις αὐτοὺς ὕδατι.
\VS{13}Καὶ ἐνδύσεις Ἀαρὼν τὰς στολὰς τὰς ἁγίας, καὶ χρίσεις αὐτὸν, καὶ ἁγιάσεις αὐτὸν, καὶ ἱερατεύει μοι.
\VS{14}Καὶ τοὺς υἱοὺς αὐτοῦ προσάξεις, καὶ ἐνδύσεις αὐτοὺς χιτῶνας.
\VS{15}Καὶ ἀλείψεις αὐτοὺς ὃν τρόπον ἤλειψας τὸν πατέρα αὐτῶν, καὶ ἱερατεύσουσί μοι· καὶ ἔσται, ὥστε εἶναι αὐτοῖς χρίσμα ἱερατείας εἰς τὸν αἰῶνα, εἰς τὰς γενεὰς αὐτῶν.
\VS{16}Καὶ ἐποίησε Μωυσῆς πάντα, ὅσα ἐνετείλατο αὐτῷ Κύριος, οὕτως ἐποίησε.
\par }{\PP \VS{17}Καὶ ἐγένετο ἐν τῷ μηνὶ τῷ πρώτῳ, τῷ δευτέρῳ ἔτει, ἐκπορευομένων αὐτῶν ἐξ Αἰγύπτου, νουμηνίᾳ ἐστάθη ἡ σκηνή.
\VS{18}Καὶ ἔστησε Μωυσῆς τὴν σκηνὴν, καὶ ἐπέθηκε τὰς κεφαλιδας, καὶ διενέβαλε τοὺς μοχλοὺς, καὶ ἔστησε τοὺς στύλους.
\VS{19}Καὶ ἐξέτεινε τὰς αὐλαίας ἐπὶ τὴν σκηνὴν, καὶ ἐπέθηκε τὸ κατακάλυμμα τῆς σκηνῆς ἐπʼ αὐτὴν ἄνωθεν, καθὰ συνέταξε Κύριος τῷ Μωυσῇ.
\VS{20}Καὶ λαβὼν τὰ μαρτύρια ἐνέβαλεν εἰς τὴν κιβωτόν· καὶ ὑπέθηκε τοὺς διωστῆρας ὑπὸ τὴν κιβωτὸν,
\VS{21}καὶ εἰσήνεγκε τὴν κιβωτὸν εἰς τὴν σκηνὴν, καὶ ἐπέθηκε τὸ κατακάλυμμα τοῦ καταπετάσματος, καὶ ἐσκέπασε τὴν κιβωτὸν τοῦ μαρτυρίου, ὃν τρόπον συνέταξε Κύριος τῷ Μωυσῇ·
\VS{22}Καὶ ἐπέθηκε τὴν τράπεζαν εἰς τὴν σκηνὴν τοῦ μαρτυρίου, τὸ πρὸς Βοῤῥᾶν ἔξωθεν τοῦ καταπετάσματος τῆς σκηνῆς.
\VS{23}Καὶ προσέθηκεν ἐπʼ αὐτῆς ἄρτους τῆς προθέσεως ἔναντι Κυρίου, ὃν τρόπον συνέταξε Κύριος τῷ Μωυσῇ.
\VS{24}Καὶ ἔθηκε τὴν λυχνίαν εἰς τὴν σκηνὴν τοῦ μαρτυρίου, εἰς τὸ κλίτος τῆς σκηνῆς τὸ πρὸς Νότον.
\VS{25}Καὶ ἐπέθηκε τοὺς λύχνους αὐτῆς ἔναντι Κυρίου, ὃν τρόπον συνέταξε Κύριος τῷ Μωυσῇ.
\VS{26}Καὶ ἔθηκε τὸ θυσιαστήριον τὸ χρυσοῦν ἐν τῇ σκηνῇ τοῦ μαρτυρίου ἀπέναντι τοῦ καταπετάσματος,
\VS{27}καὶ ἐθυμίασεν ἐνʼ αὐτοῦ θυμίαμα τῆς συνθέσεως, καθάπερ συνέταξε Κύριος τῷ Μωυσῇ.
\VS{29}Καὶ τὸ θυσιαστήριον τῶν καρπωμάτων ἔθηκε παρὰ τὰς θύρας τῆς σκηνῆς.
\VS{33}Καὶ ἔστησε τὴν αὐλὴν κύκλῳ τῆς σκηνῆς, και τοῦ θυσιαστηρίου· καὶ συνετέλεσε Μωυσῆς πάντα τὰ ἔργα.
\par }{\PP \VS{34}Καὶ ἐκάλυψεν ἡ νεφέλη τὴν σκηνὴν τοῦ μαρτυρίου· καὶ δόξης Κυρίου ἐπλήσθη ἡ σκηνή.
\VS{35}Καὶ οὐκ ἠδυνάσθη Μωυσῆς εἰσελθεῖν εἰς τὴν σκηνὴν τοῦ μαρτυρίου, ὅτι ἐπεσκίαζεν ἐπʼ αὐτὴν ἡ νεφέλη, καὶ δόξης Κυρίου ἐνεπλήσθη ἡ σκηνή.
\VS{36}Ἡνίκα δʼ ἂν ἀνέβη ἡ νεφέλη ἀπὸ τῆς σκηνῆς, ἀνεζεύγνυσαν οἱ υἱοὶ Ἰσραὴλ σὺν τῇ ἀπαρτίᾳ αὐτῶν.
\VS{37}Εἰ δὲ μὴ ἀνέβη ἡ νεφέλη, οὐκ ἀνεζεύγνυσαν ἕως ἡμέρας, ἧς ἀνέβη ἡ νεφέλη.
\VS{38}Νεφέλη γὰρ ἦν ἐπὶ τῆς σκηνῆς ἡμέρας, καὶ πῦρ ἦν ἐπʼ αὐτῆς νυκτὸς ἐναντίον παντὸς Ἰσραὴλ, ἐν πάσαις ταῖς ἀναζυγαῖς αὐτῶν.
\par }