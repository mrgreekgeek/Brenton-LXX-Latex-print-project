\NormalFont\ShortTitle{ΜΑΚΚΑΒΑΙΩΝ Αʹ}
{\MT ΜΑΚΚΑΒΑΙΩΝ Αʹ

\par }\ChapOne{1}{\PP \VerseOne{1}ΚΑΙ ἐγένετο μετὰ τὸ πατάξαι Ἀλέξανδρον τὸν Φιλίππου τὸν Μακεδόνα, ὃς ἐξῆλθεν ἐκ τῆς γῆς Χεττειεὶμ, καὶ ἐπάταξε τὸν Δαρεῖον βασιλέα Περσῶν καὶ Μήδων, καὶ ἐβασίλευσεν ἀντʼ αὐτοῦ πρότερος ἐπὶ τὴν Ἑλλάδα.
\VS{2}Καὶ συνεστήσατο πολέμους πολλοὺς, καὶ ἐκράτησεν ὀχυρωμάτων πολλῶν, καὶ ἔσφαξε βασιλεῖς τῆς γῆς.
\VS{3}Καὶ διῆλθεν ἕως ἄκρων τῆς γῆς, καὶ ἔλαβε σκῦλα πλήθους ἐθνῶν· καὶ ἠσύχασεν ἡ γῆ ἐνώπιον αὐτοῦ· καὶ ὑψώθη, καὶ ἐπῄρθη ἡ καρδία αὐτοῦ.
\VS{4}Καὶ συνήγαγε δύναμιν ἰσχυρὰν σφόδρα, καὶ ἦρξε χωρῶν, καὶ ἐθνῶν, καὶ τυράννων, καὶ ἐγένοντο αὐτῷ εἰς φόρον.
\par }{\PP \VS{5}Καὶ μετὰ ταῦτα ἔπεσεν ἐπὶ τὴν κοίτην, καὶ ἔγνω ὅτι ἀποθνήσκει.
\VS{6}Καὶ ἐκάλεσε τοὺς παῖδας αὐτοῦ τοὺς ἐνδόξους τοὺς συντρόφους αὐτοῦ ἀπὸ νεότητος, καὶ διεῖλεν αὐτοῖς τὴν βασιλείαν αὐτοῦ ἔτι ζῶντος αὐτοῦ.
\VS{7}Καὶ ἐβασίλευσεν Ἀλέξανδρος ἔτη δώδεκα, καὶ ἀπέθανε.
\VS{8}Καὶ ἐπεκράτησαν οἱ παῖδες αὐτοῦ ἕκαστος ἐν τῷ τόπῳ αὐτοῦ.
\VS{9}Καὶ ἐπέθεντο πάντες διαδήματα μετὰ τὸ ἀποθανεῖν αὐτὸν, καὶ οἱ υἱοὶ αὐτῶν ὀπίσω αὐτῶν ἔτη πολλὰ, καὶ ἐπλήθυναν κακὰ ἐν τῇ γῇ.
\par }{\PP \VS{10}Καὶ ἐξῆλθεν ἐξ αὐτῶν ῥίζα ἁμαρτωλὸς Ἀντίοχος Ἐπιφανὴς, υἱὸς Ἀντιόχου βασιλέως, ὃς ἦν ὅμηρα ἐν τῇ Ῥώμῃ· καὶ ἐβασίλευσεν ἐν ἔτει ἑκατοστῷ καὶ τριακοστῷ καὶ ἑβδόμῳ βασιλείας Ἑλλήνων.
\par }{\PP \VS{11}Ἐν ταῖς ἡμέραις ἐκείναις ἐξῆλθον ἐξ Ἰσραὴλ υἱοὶ παράνομοι, καὶ ἀνέπεισαν πολλοὺς, λέγοντες, πορευθῶμεν, καὶ διαθώμεθα διαθήκην μετὰ τῶν ἐθνῶν τῶν κύκλῳ ἡμῶν, ὅτι ἀφʼ ἧς ἐχωρίσθημεν ἀπʼ αὐτῶν, εὗρεν ἡμᾶς κακὰ πολλά.
\VS{12}Καὶ ἠγαθύνθη ὁ λόγος ἐν ὀφθαλμοῖς αὐτῶν.
\VS{13}Καὶ προεθυμήθησάν τινες ἀπὸ τοῦ λαοῦ, καὶ ἐπορεύθησαν πρὸς τὸν βασιλέα· καὶ ἔδωκεν αὐτοῖς ἐξουσίαν ποιῆσαι τὰ δικαιώματα τῶν ἐθνῶν.
\VS{14}Καὶ ᾠκοδόμησαν γυμνάσιον ἐν Ἱεροσολύμοις κατὰ τὰ νόμιμα τῶν ἐθνῶν.
\VS{15}Καὶ ἐποίησαν ἑαυτοῖς ἀκροβυστίας, καὶ ἀπέστησαν ἀπὸ διαθήκης ἁγίας· καὶ ἐζεύχθησαν τοῖς ἔθνεσι, καὶ ἐπράθησαν τοῦ ποιῆσαι τὸ πονηρόν.
\par }{\PP \VS{16}Καὶ ἡτοιμάσθη ἡ βασιλεία ἐναντίον Ἀντιόχου· καὶ ὑπέλαβε βασιλεῦσαι τῆς Αἰγύπτου, ὅπως βασιλεύσῃ ἐπὶ τὰς δύο βασιλείας.
\VS{17}Καὶ εἰσῆλθεν εἰς Αἴγυπτον ἐν ὄχλῳ βαρεῖ, ἐν ἅρμασι, καὶ ἐν ἐλέφασι, καὶ ἐν ἱππεῦσι, καὶ ἐν στόλῳ μεγάλῳ.
\VS{18}Καὶ συνεστήσατο πόλεμον πρὸς Πτολεμαῖον βασιλέα Αἰγύπτου· καὶ ἐνετράπη Πτολεμαῖος ἀπὸ προσώπου αὐτοῦ, καὶ ἔφυγε· καὶ ἔπεσον τραυματίαι πολλοί.
\par }{\PP \VS{19}Καὶ κατελάβοντο τὰς πόλεις τὰς ὀχυρὰς ἐν γῇ Αἰγύπτῳ· καὶ ἔλαβε τὰ σκῦλα γῆς Αἰγύπτου.
\par }{\PP \VS{20}Καὶ ἐπέστρεψεν Ἀντιόχος μετὰ τὸ πατάξαι Αἴγυπτον ἐν τῷ ἑκατοστῷ καὶ τεσσαρακοστῷ καὶ τρίτῳ ἔτει· καὶ ἀνέβη ἐπὶ Ἰσραὴλ, καὶ ἀνέβη εἰς Ἱερουσαλὴμ ἐν ὄχλῳ βαρεῖ.
\VS{21}Καὶ εἰσῆλθεν εἰς τὸ ἁγίασμα ἐν ὑπερηφανείᾳ, καὶ ἔλαβε τὸ θυσιαστήριον τὸ χρυσοῦν, καὶ τὴν λυχνίαν τοῦ φωτὸς, καὶ πάντα τὰ σκεύη αὐτῆς,
\VS{22}καὶ τὴν τράπεζαν τῆς προθέσεως, καὶ τὰ σπονδεῖα, καὶ τὰς φιάλας, καὶ τὰς θυΐσκας τὰς χρυσᾶς, καὶ τὸ καταπέτασμα, καὶ τοὺς στεφάνους, καὶ τὸν κόσμον τὸν χρυσοῦν τὸν κατὰ πρόσωπον τοῦ ναοῦ, καὶ ἐλέπισε πάντα.
\VS{23}Καὶ ἔλαβε τὸ ἀργύριον, καὶ τὸ χρυσίον, καὶ τὰ σκεύη τὰ ἐπιθυμητά· καὶ ἔλαβε τοὺς θησαυροὺς τοὺς ἀποκρύφους οὓς εὗρε.
\par }{\PP \VS{24}Καὶ λαβὼν πάντα ἀπῆλθεν εἰς τὴν γῆν αὐτοῦ· καὶ ἐποίησε φονοκτονίαν, καὶ ἐλάλησεν ὑπερηφανείαν μεγάλην.
\VS{25}Καὶ ἐγένετο πένθος μέγα ἐπὶ Ἰσραὴλ ἐν παντὶ τόπῳ αὐτῶν.
\VS{26}Καὶ ἐστέναξαν ἄρχοντες καὶ πρεσβύτεροι, παρθένοι καὶ νεανίσκοι ἠσθένησαν, καὶ τὸ κάλλος τῶν γυναικῶν ἠλλοιώθη.
\VS{27}Πᾶς νυμφίος ἀνέλαβε θρῆνον, καὶ καθημένη ἐν παστῷ ἐγένετο ἐν πένθει.
\VS{28}Καὶ ἐσείσθη ἡ γῆ ἐπὶ τοὺς κατοικοῦντας αὐτήν· καὶ πᾶς ὁ οἶκος Ἰακὼβ ἐνεδύσατο αἰσχύνην.
\par }{\PP \VS{29}Καὶ μετὰ δύο ἔτη ἡμερῶν ἀπέστειλεν ὁ βασιλεὺς ἄρχοντα φορολογίας εἰς τὰς πόλεις Ἰούδα· καὶ ἦλθεν εἰς Ἱερουσαλὴμ ἐν ὄχλῳ βαρεῖ.
\VS{30}Καὶ ἐλάλησεν αὐτοῖς λόγους εἰρηνικοὺς ἐν δόλῳ· καὶ ἐνεπίστευσαν αὐτῷ· καὶ ἐπέπεσεν ἐπὶ τὴν πόλιν ἐξάπινα, καὶ ἐπάταξεν αὐτὴν πληγὴν μεγάλην, καὶ ἀπώλεσε λαὸν πολὺν ἐξ Ἰσραήλ.
\VS{31}Καὶ ἔλαβε τὰ σκῦλα τῆς πόλεως, καὶ ἐνεπύρισεν αὐτὴν πυρὶ, καὶ καθεῖλε τοὺς οἴκους αὐτῆς καὶ τὰ τείχη αὐτῆς κύκλῳ.
\VS{32}Καὶ ᾐχμαλώτευσαν τὰς γυναῖκας καὶ τὰ τέκνα, καὶ τὰ κτήνη ἐκληρονόμησαν.
\par }{\PP \VS{33}Καὶ ᾠκοδόμησαν τὴν πόλιν Δαυὶδ τείχει μεγάλῳ καὶ ἰσχυρῷ, πύργοις ὀχυροῖς, καὶ ἐγένετο αὐτοῖς εἰς ἄκραν.
\VS{34}καὶ ἔθηκαν ἐκεῖ ἔθνος ἁμαρτωλὸν, ἄνδρας παρανόμους, καὶ ἐνίσχυσαν ἐν αὐτῇ.
\VS{35}Καὶ παρέθεντο ὅπλα καὶ τροφὰς, καὶ συναγαγόντες τὰ σκῦλα Ἱερουσαλὴμ ἀπέθεντο ἐκεῖ· καὶ ἐγένοντο εἰς μεγάλην παγίδα.
\VS{36}Καὶ ἐγένετο εἰς ἔνεδρον τῷ ἁγιάσματι, καὶ εἰς διάβολον πονηρὸν τῷ Ἰσραὴλ διαπαντός.
\par }{\PP \VS{37}Καὶ ἐξέχεαν αἷμα ἀθῷον κύκλῳ τοῦ ἁγιάσματος, καὶ ἐμόλυναν τὸ ἁγίασμα.
\VS{38}Καὶ ἔφυγον οἱ κάτοικοι Ἱερουσαλὴμ διʼ αὐτοὺς, καὶ ἐγένετο κατοικία ἀλλοτρίων· καὶ ἐγένετο ἀλλοτρία τοῖς γεννήμασιν αὐτῆς, καὶ τὰ τέκνα αὐτῆς ἐγκατέλιπον αὐτήν.
\VS{39}Τὸ ἁγίασμα αὐτῆς ἠρημώθη ὡς ἔρημος, αἱ ἑορταὶ αὐτῆς ἐστράφησαν εἰς πένθος, τὰ σάββατα αὐτῆς εἰς ὀνειδισμὸν, ἡ τιμὴ αὐτῆς εἱς ἐξουδένωσιν.
\VS{40}Κατὰ τὴν δόξαν αὐτῆς ἐπληθύνθη ἡ ἀτιμία αὐτῆς, καὶ τὸ ὕψος αὐτῆς ἐστράφη εἰς πένθος.
\par }{\PP \VS{41}Καὶ ἔγραψεν ὁ βασιλεὺς Ἀντίοχος πάσῃ τῇ βασιλείᾳ αὐτοῦ εἶναι πάντας λαὸν ἕνα,
\VS{42}καὶ ἐγκαταλιπεῖν ἕκαστον τὰ νόμιμα αὐτοῦ· καὶ ἐπεδέξατο πάντα τὰ ἔθνη κατὰ τὸν λόγον τοῦ βασιλέως.
\VS{43}Καὶ πολλοὶ ἀπὸ Ἰσραὴλ εὐδόκησαν τῇ λατρείᾳ αὐτοῦ, καὶ ἔθυσαν τοῖς εἰδώλοις, καὶ ἐβεβήλωσαν τὸ σάββατον.
\par }{\PP \VS{44}Καὶ ἀπέστειλεν ὁ βασιλεὺς βιβλία ἐν χειρὶ ἀγγέλων εἰς Ἱερουσαλὴμ καὶ τὰς πόλεις Ἰούδα, πορευθῆναι ὀπίσω νομίμων ἀλλοτρίων τῆς γῆς,
\VS{45}καὶ κωλῦσαι ὁλοκαυτώματα καὶ θυσίαν καὶ σπονδὴν ἐκ τοῦ ἁγιάσματος, καὶ βεβηλῶσαι σάββατα καὶ ἑορτὰς,
\VS{46}καὶ μιᾶναι ἁγίασμα καὶ ἁγίους·
\VS{47}οἰκοδομῆσαι βωμοὺς, καὶ τεμένη, καὶ εἰδωλεῖα, καὶ θύειν ὕεια, καὶ κτήνη κοινὰ,
\VS{48}καὶ ἀφιέναι τοὺς υἱοὺς αὐτῶν ἀπεριτμήτους, βδελύξαι τὰς ψυχὰς αὐτῶν ἐν παντὶ ἀκαθάρτῳ καὶ βεβηλώσει,
\VS{49}ὥστε ἐπιλαθέσθαι τοῦ νόμου, καὶ ἀλλάξαι πάντα τὰ δικαιώματα.
\par }{\PP \VS{50}Καὶ ὃς ἂν μὴ ποιήσῃ κατὰ τὸ ῥῆμα τοῦ βασιλέως, ἀποθανεῖται.
\VS{51}Κατὰ πάντας τοὺς λόγους τούτους ἔγραψε πάσῃ τῇ βασιλείᾳ αὐτοῦ, καὶ ἐποίησεν ἐπισκόπους ἐπὶ πάντα τὸν λαόν· καὶ ἐνετείλατο ταῖς πόλεσιν Ἰούδα θυσιάζειν κατὰ πόλιν καὶ πόλιν.
\VS{52}Καὶ συνηθροίσθησαν ἀπὸ τοῦ λαοῦ πρὸς αὐτοὺς πολλοὶ, πᾶς ὁ ἐγκαταλιπὼν τὸν νόμον· καὶ ἐποιησαν κακὰ ἐν τῇ γῇ.
\VS{53}Καὶ ἔθεντο τὸν Ἰσραὴλ ἐν κρύφοις ἐν παντὶ φυγαδευτηρίῳ αὐτῶν.
\par }{\PP \VS{54}Καὶ τῇ πεντεκαιδεκάτῃ ἡμέρᾳ Χασελεῦ, τῷ πέμπτῳ καὶ τεσσαρακοστῷ καὶ ἑκατοστῷ ἔτει, ᾠκοδόμησαν βδέλυγμα ἐρημώσεως ἐπὶ τὸ θυσιαστήριον, καὶ ἐν πόλεσιν Ἰούδα κύκλῳ ᾠκοδόμησαν βωμούς.
\VS{55}Καὶ ἐπὶ τῶν θυρῶν τῶν οἰκιῶν, καὶ ἐν ταῖς πλατείαις ἐθυμίων.
\par }{\PP \VS{56}Καὶ τὰ βιβλία τοῦ νόμου ἃ εὗρον, ἐνεπύρισαν πυρὶ κατασχίσαντες.
\VS{57}Καὶ ὅπου εὑρίσκετο παρά τινι βιβλίον διαθήκης, καὶ εἴ τις συνευδόκει τῷ νόμῳ, τὸ σύγκριμα τοῦ βασιλέως ἐθανάτου αὐτόν.
\VS{58}Ἐν ἰσχύϊ αὐτῶν ἐποίουν οὕτως τῷ Ἰσραὴλ τοῖς εὑρισκομένοις ἐν παντὶ μηνὶ καὶ μηνὶ ἐν ταῖς πόλεσι.
\VS{59}Καὶ τῇ πέμπτῃ καὶ εἰκάδι τοῦ μηνὸς θυσιάζοντες ἐπὶ τὸν βωμὸν ὃς ἦν ἐπὶ τοῦ θυσιαστηρίου.
\par }{\PP \VS{60}Καὶ τὰς γυναῖκας τὰς περιτετμηκυίας τὰ τέκνα αὐτῶν ἐθανάτωσαν, κατὰ τὸ πρόσταγμα.
\VS{61}Καὶ ἐκρέμασαν τὰ βρέφη ἐκ τῶν τραχήλων αὐτῶν, καὶ τοὺς οἴκους αὐτῶν προενόμευσαν, καὶ τοὺς περιτετμηκότας αὐτοὺς ἐθανάτωσαν.
\VS{62}Καὶ πολλοὶ ἐν Ἰσραὴλ ἐκραταιώθησαν, καὶ ὠχυρώθησαν ἐν ἑαυτοῖς τοῦ μὴ φαγεῖν κοινά.
\VS{63}Καὶ ἐπελέξαντο ἀποθανεῖν, ἵνα μὴ μιανθῶσι τοῖς βρώμασι, καὶ μὴ βεβηλώσωσι διαθήκην ἁγίαν· καὶ ἀπέθανον.
\VS{64}Καὶ ἐγένετο ὀργὴ μεγάλη ἐπὶ Ἰσραὴλ σφόδρα.

\par }\Chap{2}{\PP \VerseOne{1}Ἐν ταῖς ἡμέραις ἐκείναις ἀνέστη Ματταθίας Ἰωάννου τοῦ Συμεὼν, ἱερεὺς τῶν υἱῶν Ἰωαρὶβ ἀπὸ Ἱερουσαλὴμ, καὶ ἐκάθισεν ἐν Μωδεΐν.
\VS{2}Καὶ αὐτῷ υἱοὶ πέντε, Ἰωαννὰν ὁ ἐπικαλούμενος Καδδὶς,
\VS{3}Σίμων ὁ καλούμενος Θασσι,
\VS{4}Ἰούδας ὁ ἐπικαλούμενος Μακκαβαῖος,
\VS{5}Ἐλεάζαρ ὁ ἐπικαλούμενος Αὐαρὰν, Ἰωνάθαν ὁ ἐπικαλούμενος Ἀπφοῦς.
\par }{\PP \VS{6}Καὶ εἶδε τὰς βλασφημίας τὰς γινομένας ἐν Ἰούδα καὶ ἐν Ἱερουσαλὴμ,
\VS{7}καὶ εἶπεν, οἴμοι, ἱνατί τοῦτο ἐγεννήθην ἰδεῖν τὸ σύντριμμα τοῦ λαοῦ μου, καὶ τὸ σύντριμμα τῆς πόλεως τῆς ἁγίας, καὶ καθίσαι ἐκεῖ ἐν τῷ δοθῆναι αὐτὴν ἐν χειρὶ ἐχθρῶν, καὶ τὸ ἁγίασμα ἐν χειρὶ ἀλλοτρίων;
\par }{\PP \VS{8}Ἐγένετο ὁ ναὸς αὐτῆς ὡς ἀνὴρ ἄδοξος,
\VS{9}τὰ σκεύη τῆς δόξης αὐτῆς αἰχμάλωτα ἀπήχθη, ἀπεκτάνθη τὰ νήπια αὐτῆς ἐν ταῖς πλατείαις, οἱ νεανίσκοι αὐτῆς ἐν ῥομφαίᾳ ἐχθροῦ.
\VS{10}Ποῖον ἔθνος οὐκ ἐκληρονόμησε βασιλείαν αὐτῆς, καὶ οὐκ ἐκράτησε τῶν σκύλων αὐτῆς;
\VS{11}Πᾶς ὁ κόσμος αὐτῆς ἀφῃρέθη, ἀντὶ ἐλευθήρας ἐγένετο εἰς δούλην.
\VS{12}Καὶ ἰδοὺ τὰ ἅγια ἡμῶν καὶ ἡ καλλονὴ ἡμῶν καὶ ἡ δόξα ἡμῶν ἠρημώθη, καὶ ἐβεβήλωσαν αὐτὰ τὰ ἔθνη.
\VS{13}Ἱνατί ἡμῖν ἔτι ζῇν;
\par }{\PP \VS{14}Καὶ διέῤῥηξε Ματταθίας καὶ υἱοὶ αὐτοῦ τὰ ἱμάτια αὐτῶν, καὶ περιεβάλοντο σάκκους, καὶ ἐπένθησαν σφόδρα.
\par }{\PP \VS{15}Καὶ ἦλθον οἱ παρὰ τοῦ βασιλέως οἱ καταναγκάζοντες τὴν ἀποστασίαν εἰς Μωδεῒν τὴν πόλιν, ἵνα θυσιάσωσι.
\VS{16}Καὶ πολλοὶ ἀπὸ Ἰσραὴλ πρὸς αὐτοὺς προσῆλθον· καὶ Ματταθίας καὶ οἱ υἱοὶ αὐτοῦ συνήχθησαν.
\par }{\PP \VS{17}Καὶ ἀπεκρίθησαν οἱ παρὰ τοῦ βασιλέως, καὶ εἰπον τῷ Ματταθίᾳ, λέγοντες, ἄρχων καὶ ἔνδοξος καὶ μέγας εἶ ἐν τῇ πόλει ταύτῃ, καὶ ἐστηριγμένος ἐν υἱοῖς καὶ ἀδελφοῖς.
\VS{18}Νῦν οὖν πρόσελθε πρῶτος, καὶ ποίησον τὸ πρόσταγμα τοῦ βασιλέως, ὡς ἐποίησαν πάντα τὰ ἔθνη, καὶ οἱ ἄνδρες Ἰούδα, καὶ οἱ καταλειφθέντες ἐν Ἱερουσαλήμ· καὶ ἔσῃ σὺ καὶ ὁ οἶκός σου τῶν φίλων τοῦ βασιλέως, καὶ σὺ καὶ οἱ υἱοί σου δοξασθήσεσθε ἀργυρίῳ, καὶ χρυσίῳ, καὶ ἀποστολαῖς πολλαῖς.
\par }{\PP \VS{19}Καὶ ἀπεκρίθη Ματταθίας, καὶ εἶπε φωνῇ μεγάλῃ, εἰ πάντα τὰ ἔθνη τὰ ἐν οἴκῳ τῆς βασιλείας τοῦ βασιλέως ἀκούουσιν αὐτοῦ, ἀποστῆναι ἕκαστος ἀπὸ λατρείας πατέρων αὐτοῦ, καὶ ᾑρετίσαντο ἐν ταῖς ἐντολαῖς αὐτοῦ,
\VS{20}ἀλλʼ ἐγὼ καὶ οἱ υἱοί μου καὶ οἱ ἀδελφοί μου πορευσόμεθα ἐν διαθήκῃ πατέρων ἡμῶν.
\VS{21}Ἵλεως ἡμῖν καταλιπεῖν νόμον καὶ δικαιώματὰ.
\VS{22}Τῶν λόγων τοῦ βασιλέως οὐκ ἀκουσόμεθα, τοῦ παρελθεῖν τὴν λατρείαν ἡμῶν, δεξιὰν ἢ ἀριστεράν.
\par }{\PP \VS{23}Καὶ ὡς ἐπαύσατο λαλῶν τοὺς λόγους τούτους, προσῆλθεν ἀνὴρ Ἰουδαῖος ἐν ὀφθαλμοῖς πάντων, θυσιᾶσαι ἐπὶ τοῦ βωμοῦ τοῦ ἐν Μωδεῒν κατὰ τὸ πρόσταγμα τοῦ βασιλέως.
\VS{24}Καὶ εἶδε Ματταθίας καὶ ἐζήλωσε, καὶ ἐτρόμησαν οἱ νεφροὶ αὐτοῦ, καὶ ἀνήνεγκε θυμὸν κατὰ τὸ κρίμα, καὶ δραμῶν ἔσφαξεν αὐτὸν ἐπὶ τὸν βωμόν.
\par }{\PP \VS{25}Καὶ τὸν ἄνδρα τοῦ βασιλέως τὸν ἀναγκάζοντα θύειν, ἀπέκτεινεν ἐν τῷ καιρῷ ἐκείνῳ, καὶ τὸν βωμὸν καθεῖλε.
\VS{26}Καὶ ἐζήλωσε τῷ νόμῳ καθὼς ἐποίησε Φινεὲς τῷ Ζαμβρὶ υἱῷ Σαλώμ.
\par }{\PP \VS{27}Καὶ ἀνέκραξε Ματταθίας ἐν τῇ πόλει φωνῇ μεγάλῃ, λέγων, πᾶς ὁ ζηλῶν τῷ νόμῳ καὶ ἱστῶν διαθήκην, ἐξελθέτω ὀπίσω μου.
\VS{28}Καὶ ἔφυγον αὐτὸς καὶ οἱ υἱοὶ αὐτοῦ εἰς τὰ ὄρη, καὶ ἐγκατέλιπον ὅσα εἶχον ἐν τῇ πόλει.
\par }{\PP \VS{29}Τότε κατέβησαν πολλοὶ ζητοῦντες δικαιοσύνην καὶ κρίμα, εἰς τὴν ἔρημον, καθίσαι ἐκεῖ,
\VS{30}αὐτοὶ καὶ οἱ υἱοὶ αὐτῶν καὶ αἱ γυναῖκες αὐτῶν καὶ τὰ κτήνη αὐτῶν, ὅτι ἐπληθύνθη ἐπʼ αὐτοὺς τὰ κακά.
\par }{\PP \VS{31}Καὶ ἀνηγγέλη τοῖς ἀνδράσι τοῦ βασιλέως καὶ ταῖς δυνάμεσιν αἳ ἦσαν ἐν Ἱερουσαλὴμ πόλει Δαυὶδ, ὅτι κατέβησαν ἄνδρες, οἵτινες διεσκέδασαν τὴν ἐντολὴν τοῦ βασιλέως, εἰς τοὺς κρύφους ἐν τῇ ἐρήμῳ.
\VS{32}Καὶ ἔδραμον ὀπίσω αὐτῶν πολλοί· καὶ καταλαβόντες αὐτοὺς παρενέβαλον ἐπʼ αὐτοὺς, καὶ συνεστήσαντο πρὸς αὐτοὺς πόλεμον ἐν τῇ ἡμέρᾳ τῶν σαββάτων,
\VS{33}καὶ εἶπον πρὸς αὐτοὺς, ἕως τοῦ νῦν ἱκανόν· ἐξέλθετε καὶ ποιήσατε κατὰ τὸν λόγον τοῦ βασιλέως, καὶ ζήσεσθε.
\par }{\PP \VS{34}Καὶ εἶπον, οὐκ ἐξελευσόμεθα, οὐδὲ ποιήσομεν τὸν λόγον τοῦ βασιλέως, τοῦ βεβηλῶσαι τὴν ἡμέραν τῶν σαββάτων.
\VS{35}καὶ ἐτάχυναν ἐπʼ αὐτοὺς πόλεμον.
\VS{36}Καὶ οὐκ ἀπεκρίθησαν αὐτοῖς, οὐδὲ λίθον ἐνετίναξαν αὐτοῖς, οὐδὲ ἐνέφραξαν τοὺς κρύφους,
\VS{37}λέγοντες, ἀποθάνωμεν πάντες ἐν τῇ ἁπλότητι ἡμῶν· μαρτυρεῖ ἐφʼ ἡμᾶς ὁ οὐρανὸς καὶ ἡ γῆ, ὅτι ἀκρίτως ἀπόλλυτε ἡμᾶς.
\VS{38}Καὶ ἀνέστησαν ἐπʼ αὐτοὺς ἐν τῷ πολέμῳ τοῖς σάββασι, καὶ ἀπέθανον αὐτοὶ καὶ αἱ γυναῖκες αὐτῶν, καὶ τὰ τέκνα αὐτῶν, καὶ τὰ κτήνη αὐτῶν, ἕως χιλίων ψυχῶν ἀνθρώπων.
\par }{\PP \VS{39}Καὶ ἔγνω Ματταθίας καὶ οἱ φίλοι αὐτοῦ, καὶ ἐπένθησαν ἐπʼ αὐτοὺς ἕως σφόδρα.
\VS{40}Καὶ εἶπεν ἀνὴρ τῷ πλησίον αὐτοῦ, ἐὰν πάντες ποιήσωμεν ὡς οἱ ἀδελφοὶ ἡμῶν ἐποίησαν, καὶ μὴ πολεμήσωμεν πρὸς τὰ ἔθνη ὑπὲρ τῶν ψυχῶν ἡμῶν καὶ τῶν δικαιωμάτων ἡμῶν, νῦν τάχιον ἡμᾶς ἐξολοθρεύσουσιν ἀπὸ τῆς γῆς.
\par }{\PP \VS{41}Καὶ ἐβουλεύσαντο τῇ ἡμέρᾳ ἐκείνῃ, λέγοντες, πᾶς ἄνθρωπος ὃς ἐὰν ἔλθῃ πρὸς ἡμᾶς εἰς πόλεμον τῇ ἡμέρᾳ τῶν σαββάτων, πολεμήσωμεν κατέναντι αὐτοῦ, καὶ οὐ μὴ ἀποθάνωμεν πάντες καθὼς ἀπέθανον οἱ ἀδελφοὶ ἡμῶν ἐν τοῖς κρύφοις.
\par }{\PP \VS{42}Τότε συνήχθησαν πρὸς αὐτοὺς συναγωγὴ Ἰουδαίων, ἰσχυροὶ δυνάμει ἀπὸ Ἰσραὴλ, πᾶς ὁ ἑκουσιαζόμενος τῷ νόμῳ.
\VS{43}Καὶ πάντες οἱ φυγαδεύοντες ἀπὸ τῶν κακῶν προσετέθησαν αὐτοῖς, καὶ ἐγένοντο αὐτοῖς εἰς στήριγμα.
\VS{44}Καὶ συνεστήσαντο δύναμιν, καὶ ἐπάταξαν ἁμαρτωλοὺς ἐν ὀργῇ αὐτῶν, καὶ ἄνδρας ἀνόμους ἐν θυμῷ αὐτῶν· καὶ οἱ λοιποὶ ἔφυγον εἰς τὰ ἔθνη σωθῆναι.
\par }{\PP \VS{45}Καὶ ἐκύκλωσε Ματταθίας καὶ οἱ φίλοι αὐτοῦ, καὶ καθεῖλον τοὺς βωμοὺς.
\VS{46}Καὶ περιέτεμον τὰ παιδάρια τὰ ἀπερίτμητα ὅσα εὗρον ἐν ὁρίοις Ἰσραὴλ ἐν ἰσχύϊ.
\VS{47}Καὶ ἐδίωξαν τοὺς υἱοὺς τῆς ὑπεπηφανίας, καὶ κατευωδώθη τὸ ἔργον ἐν χειρὶ αὐτῶν.
\VS{48}Καὶ ἀντελάβοντο τοῦ νόμου ἐκ χειρὸς τῶν ἐθνῶν καὶ ἐκ χειρὸς τῶν βασιλέων· καὶ οὐκ ἔδωκαν κέρας τῷ ἁμαρτωλῷ.
\par }{\PP \VS{49}Καὶ ἤγγισαν αἱ ἡμέραι τοῦ Ματταθίου ἀποθανεῖν, καὶ εἶπε τοῖς υἱοῖς αὐτοῦ, νῦν ἐστηρίχθη ὑπερηφανία καὶ ἐλεγμὸς καὶ καιρὸς καταστροφῆς καὶ ὀργὴ θυμοῦ.
\VS{50}Καὶ νῦν, τέκνα, ζηλώσατε τῷ νόμῳ, καὶ δότε τὰς ψυχὰς ὑμῶν ὑπὲρ διαθήκης πατέρων ἡμῶν.
\VS{51}Μνήσθητε τῶν πατέρων ἡμῶν τὰ ἔργα ἃ ἐποίησαν ἐν ταῖς γενεαῖς αὐτῶν, καὶ δέξασθε δόξαν μεγάλην καὶ ὄνομα αἰώνιον.
\VS{52}Ἁβραὰμ οὐχὶ ἐν πειρασμῷ εὑρέθη πιστὸς, καὶ ἐλογίσθη αὐτῷ εἰς δικαιοσύνην;
\VS{53}Ἰωσὴφ ἐν καιρῷ στενοχωρίας αὐτοῦ ἐφύλαξεν ἐντολὴν, καὶ ἐγένετο κύριος Αἰγύπτου.
\VS{54}Φινεὲς ὁ πατὴρ ἡμῶν ἐν τῷ ζηλῶσαι ζῆλον, ἔλαβε διαθήκην ἱερωσύνης αἰωνίας.
\par }{\PP \VS{55}Ἰησοῦς ἐν τῷ πληρῶσαι λόγον, ἐγένετο κριτὴς ἐν Ἰσραήλ.
\VS{56}Χαλὲβ ἐν τῷ ἐπιμαρτύρασθαι ἐν τῇ ἐκκλησία, ἔλαβε γῆς κληρονομίαν.
\VS{57}Δαυὶδ ἐν τῷ ἐλέῳ αὐτοῦ, ἐκληρονόμησε θρόνον βασιλείας εἰς αἰῶνα αἰῶνος.
\VS{58}Ἡλίας ἐν τῷ ζηλῶσαι ζῆλον νόμου, ἀνελήφθη ἕως εἰς τὸν οὐρανόν.
\VS{59}Ἀνανίας, Ἀζαρίας, Μισαήλ, πιστεύσαντες ἐσώθησαν ἐκ φλογός.
\VS{60}Δανιὴλ ἐν τῇ ἁπλότητι αὐτοῦ ἐῤῥύσθη ἐκ στόματος λεόντων.
\VS{61}Καὶ οὕτως ἐννοήθητε κατὰ γενεὰν καὶ γενεὰν, ὅτι πάντες οἱ ἐλπίζοντες ἐπʼ αὐτὸν οὐκ ἀσθενήσουσι.
\VS{62}Καὶ ἀπὸ λόγων ἀνδρὸς ἁμαρτωλοῦ μὴ φοβηθῆτε, ὅτι ἡ δόξα αὐτοῦ εἰς κοπρίαν καὶ εἰς σκώληκας.
\VS{63}Σήμερον ἐπαρθήσεται, καὶ αὔριον οὐ μὴ εὑρεθῇ, ὅτι ἔστρεψεν εἰς τὸν χοῦν αὐτοῦ, καὶ ὁ διαλογισμὸς αὐτοῦ ἀπώλετο.
\par }{\PP \VS{64}Καὶ ὑμεῖς, τέκνα, ἰσχύσατε καὶ ἀνδρίζεσθε ἐν τῷ νόμῳ, ὅτι ἐν αὐτῷ δοξασθήσεσθε.
\VS{65}Καὶ ἰδοὺ Συμεὼν ὁ ἀδελφὸς ὑμῶν, οἶδα ὅτι ἀνὴρ βουλῆς ἐστιν, αὐτοῦ ἀκούετε πάσας τὰς ἡμέρας, αὐτὸς ὑμῖν ἔσται εἰς πατέρα.
\VS{66}Καὶ Ἰούδας Μακκαβαῖος ἰσχυρὸς δυνάμει ἐκ νεότητος αὐτοῦ, οὗτος ὑμῖν ἔσται ἄρχων στρατιᾶς, καὶ πολεμήσει πόλεμον λαῶν.
\par }{\PP \VS{67}Καὶ ὑμεῖς προσάξατε πρὸς ὑμᾶς πάντας τοὺς ποιητὰς τοῦ νόμου, καὶ ἐκδικήσατε ἐκδίκησιν τοῦ λαοῦ ὑμῶν.
\VS{68}Ἀνταπόδοτε ἀνταπόδομα τοῖς ἔθνεσι, καὶ προσέχετε εἰς τὰ προστάγματα τοῦ νόμου.
\VS{69}Καὶ εὐλόγησεν αὐτούς· καὶ προσετέθη πρὸς τοὺς πατέρας αὐτοῦ.
\VS{70}Καὶ ἀπέθανεν ἐν τῷ ἕκτῳ καὶ τεσσαρακοστῷ καὶ ἑκατοστῷ ἔτει· καὶ ἔθαψαν αὐτὸν οἱ υἱοὶ αὐτοῦ ἐν τάφοις πατέρων αὐτῶν ἐν Μωδεῒν, καὶ ἐκόψαντο αὐτὸν πᾶς Ἰσραὴλ κοπετὸν μέγαν.

\par }\Chap{3}{\PP \VerseOne{1}Καὶ ἀνέστη Ἰούδας ὁ καλούμενος Μακκαβαῖος υἱὸς αὐτοῦ ἀντʼ αὐτοῦ.
\VS{2}Καὶ ἐβοήθουν αὐτῷ πάντες οἱ ἀδελφοὶ αὐτοῦ, καὶ πάντες ὅσοι ἐκολλήθησαν τῷ πατρὶ αὐτοῦ, καὶ ἐπολέμουν τὸν πόλεμον Ἰσραὴλ μετʼ εὐφροσύνης.
\VS{3}Καὶ ἐπλάτυνε δόξαν τῷ λαῷ αὐτοῦ, καὶ ἐνεδύσατο θώρακα ὡς γίγας, καὶ συνεζώσατο τὰ σκεύη αὐτοῦ τὰ πολεμικά· καὶ συνεστήσατο πολέμους σκεπάζων παρεμβολὴν ἐν ῥομφαίᾳ.
\par }{\PP \VS{4}Καὶ ὡμοιώθη λέοντι ἐν τοῖς ἔργοις αὐτοῦ, καὶ ὡς σκύμνος ἐρευγόμενος εἰς θήραν.
\VS{5}Καὶ ἐδίωξεν ἀνόμους ἐξερευνῶν, καὶ τοὺς ταράσσοντας τὸν λαὸν αὐτοῦ ἐφλόγισε·
\VS{6}Καὶ συνεστάλησαν οἱ ἄνομοι ἀπὸ τοῦ φόβου αὐτοῦ, καὶ πάντες οἱ ἐργάται τῆς ἀνομίας συνεταράχθησαν, καὶ εὐοδώθη σωτηρία ἐν χειρὶ αὐτοῦ.
\par }{\PP \VS{7}Καὶ ἐπίκρανε βασιλεῖς πολλούς, καὶ εὔφρανε τὸν Ἰακὼβ ἐν τοῖς ἔργοις αὐτοῦ, καὶ ἕως τοῦ αἰῶνος τὸ μνημόσυνον αὐτοῦ εἰς εὐλογίαν.
\VS{8}Καὶ διῆλθεν ἐν πόλεσιν Ἰούδα, καὶ ἐξωλόθρευσεν ἀσεβεῖς ἐξ αὐτῆς, καὶ ἀπέστρεψεν ὀργὴν ἀπὸ Ἰσραήλ.
\VS{9}Καὶ ὠνομάσθη ἕως ἐσχάτου τῆς γῆς, καὶ συνήγαγεν ἀπολλυμένους.
\par }{\PP \VS{10}Καὶ συνήγαγεν Ἀπολλώνιος ἔθνη, καὶ ἀπὸ Σαμαρείας δύναμιν μεγάλην, τοῦ πολεμῆσαι πρὸς Ἰσραήλ.
\VS{11}Καὶ ἔγνω Ἰούδας, καὶ ἐξῆλθεν εἰς συνάντησιν αὐτῷ, καὶ ἐπάταξεν αὐτὸν, καὶ ἀπέκτεινεν αὐτόν· καὶ ἔπεσον τραυματίαι πολλοὶ, καὶ οἱ ἐπίλοιποι ἔφυγον.
\VS{12}Καὶ ἔλαβε τὰ σκῦλα αὐτῶν, καὶ τὴν μάχαιραν Ἀπολλωνίου ἔλαβεν Ἰούδας, καὶ ἦν πολεμῶν ἐν αὐτῇ πάσας τὰς ἡμέρας.
\par }{\PP \VS{13}Καὶ ἤκουσε Σήρων, ὁ ἄρχων τῆς δυνάμεως Συρίας, ὅτι ἤθροισεν Ἰούδας ἄθροισμα, καὶ ἐκκλησίαν πιστῶν μετʼ αὐτοῦ ἐκπορευομένων εἰς πόλεμον·
\VS{14}Καὶ εἶπε, ποιήσω ἐμαυτῷ ὄνομα καὶ δοξασθήσομαι ἐν τῇ βασιλείᾳ, καὶ πολεμήσω τὸν Ἰούδαν καὶ τοὺς σὺν αὐτῷ, τοὺς ἐξουδενοῦντας τὸν λόγον τοῦ βασιλέως.
\VS{15}Καὶ προσέθετο τοῦ ἀναβῆναι· καὶ ἀνέβη μετʼ αὐτοῦ παρεμβολὴ ἀσεβῶν ἰσχυρὰ βοηθῆσαι αὐτῷ, καὶ ποιῆσαι τὴν ἐκδίκησιν ἐν υἱοῖς Ἰσραήλ.
\par }{\PP \VS{16}Καὶ ἤγγισαν ἕως ἀναβάσεως Βαιθωρών· καὶ ἐξῆλθεν Ἰούδας εἰς συνάντησιν αὐτῷ ὀλιγοστός.
\VS{17}Ὡς δὲ ἴδον τὴν παρεμβολὴν ἐρχομένην εἰς συνάντησιν αὐτοῖς, εἶπον τῷ Ἰούδα, πῶς δυνησόμεθα ὀλιγοστοὶ ὄντες πολεμῆσαι πρὸς πλῆθος τοσοῦτον ἰσχυρόν; καὶ ἡμεῖς ἐκλελύμεθα ἀσιτοῦντες σήμερον.
\VS{18}Καὶ εἶπεν Ἰούδας, εὔκοπόν ἐστι συγκλεισθῆναι πολλοὺς ἐν χερσὶν ὀλίγων· καὶ οὐκ ἔστι διαφορὰ ἐναντίον τοῦ Θεοῦ τοῦ οὐρανοῦ σώζειν ἐν πολλοῖς ἢ ἐν ὀλίγοις.
\VS{19}Ὅτι οὐκ ἐν πλήθει δυνάμεως νίκη πολέμου ἐστὶν, ἀλλʼ ἢ ἐκ τοῦ οὐρανοῦ ἡ ἰσχύς.
\VS{20}Αὐτοὶ ἔρχονται πρὸς ἡμᾶς ἐν πλήθει ὕβρεως καὶ ἀνομίας, τοῦ ἐξᾶραι ἡμᾶς καὶ τὰς γυναῖκας ἡμῶν, καὶ τὰ τέκνα ἡμῶν, τοῦ σκυλεῦσαι ἡμᾶς.
\VS{21}Ἡμεῖς δὲ πολεμοῦμεν περὶ τῶν ψυχῶν ἡμῶν καὶ τῶν νομίμων ἡμῶν.
\VS{22}Καὶ αὐτὸς συντρίψει αὐτοὺς πρὸ προσώπου ἡμῶν· ὑμεῖς δὲ μὴ φοβηθῆτε ἀπʼ αὐτῶν.
\par }{\PP \VS{23}Ὡς δὲ ἐπαύσατο λαλῶν, ἐνήλατο εἰς αὐτοὺς ἄφνω, καὶ συνετρίβη Σήρων καὶ ἡ παρεμβολὴ αὐτοῦ ἐνώπιον αὐτοῦ.
\VS{24}Καὶ ἐδίωκον αὐτὸν ἐν τῇ καταβάσει Βαιθωρῶν ἕως τοῦ πεδίου· καὶ ἔπεσον ἀπʼ αὐτῶν εἰς ἄνδρας ὀκτακοσίους· οἱ δὲ λοιποὶ ἔφυγον εἰς γῆν Φυλιστιείμ.
\VS{25}Καὶ ἤρξατο ὁ φόβος Ἰούδα καὶ τῶν ἀδελφῶν αὐτοῦ καὶ ἡ πτόησις ἐπιπίπτειν ἐπὶ τὰ ἔθνη τὰ κύκλῳ αὐτῶν.
\VS{26}Καὶ ἤγγισεν ἕως τοῦ βασιλέως τὸ ὄνομα αὐτοῦ, καὶ ὑπὲρ τῶν παρατάξεων Ἰούδα ἐξηγεῖτο πᾶν ἔθνος.
\par }{\PP \VS{27}Ὡς δὲ ἤκουσεν Ἀντίοχος ὁ βασιλεὺς τοὺς λόγους τούτους, ὠργίσθη θυμῷ· καὶ ἀπέστειλε καὶ συνήγαγε τὰς δυνάμεις πάσας τῆς βασιλείας αὐτοῦ, παρεμβολὴν ἰσχυρὰν σφόδρα.
\VS{28}Καὶ ἤνοιξε τὸ γαζοφυλάκιον αὐτοῦ, καὶ ἔδωκεν ὀψώνια ταῖς δυνάμεσιν αὐτοῦ εἰς ἐνιαυτόν· καὶ ἐνετείλατο εἶναι αὐτοὺς ἑτοίμους εἰς πᾶσαν χρείαν.
\par }{\PP \VS{29}Καὶ εἶδεν ὅτι ἐξέλιπε τὸ ἀργύριον ἀπὸ τῶν θησαυρῶν, καὶ οἱ φορολόγοι τῆς χώρας ὀλίγοι, χάριν τῆς διχοστασίας καὶ πληγῆς ἧς κατεσκεύασεν ἐν τῇ γῇ, τοῦ ἆραι τὰ νόμιμα ἃ ἦσαν ἀφʼ ἡμερῶν τῶν πρώτων.
\VS{30}Καὶ εὐλαβήθη μὴ οὐκ ἔχῃ ὡς ἅπαξ καὶ δὶς εἰς τὰς δαπάνας καὶ τὰ δόματα ἃ ἐδίδου ἔμπροσθεν δαψιλεῖ χειρὶ, καὶ ἐπερίσσευσεν ὑπέρ τοὺς βασιλεῖς τοὺς ἔμπροσθεν.
\par }{\PP \VS{31}Καὶ ἠπορεῖτο τῇ ψυχῇ αὐτοῦ σφόδρα, καὶ ἐβουλεύσατο τοῦ πορευθῆναι εἰς τὴν Περσίδα, καὶ λαβεῖν τοὺς φόρους τῶν χωρῶν, καὶ συναγαγεῖν ἀργύριον πολύ.
\VS{32}Καὶ κατέλιπε Λυσίαν ἄνθρωπον ἔνδοξον καὶ ἀπὸ γένους τῆς βασιλείας, ἐπὶ τῶν πραγμάτων τοῦ βασιλέως ἀπὸ τοῦ ποταμοῦ Εὐφράτου ἕως τῶν ὁρίων Αἰγύπτου,
\VS{33}καὶ τρέφειν Ἀντίοχον τὸν υἱὸν αὐτοῦ ἕως τοῦ ἐπιστρέψαι αὐτόν.
\VS{34}Καὶ παρέδωκεν αὐτῷ τὰς ἡμίσεις τῶν δυνάμεων καὶ τοὺς ἐλέφαντας· καὶ ἐνετείλατο αὐτῷ περὶ πάντων ὧν ἐβούλετο, καὶ περὶ τῶν κατοικούντων τὴν Ἰουδαίαν καὶ Ἱερουσαλὴμ,
\VS{35}ἀποστεῖλαι ἐπʼ αὐτοὺς δύναμιν, τοῦ ἐκτρίψαι καὶ ἐξᾶραι τὴν ἰσχὺν Ἰσραὴλ, καὶ τὸ κατάλειμμα Ἱερουσαλὴμ, καὶ ἆραι τὸ μνημόσυνον αὐτῶν ἀπὸ τοῦ τόπου,
\VS{36}καὶ κατοικῆσαι υἱοὺς ἀλλογενεῖς ἐν πᾶσι τοῖς ὁρίοις αὐτῶν, καὶ κατακληροδοτῆσαι τὴν γῆν αὐτῶν.
\VS{37}Καὶ ὁ βασιλεὺς παρέλαβε τὰς ἡμίσεις τῶν δυνάμεων τὰς καταλειφθείσας, καὶ ἀπῇρεν ἀπὸ Ἀντιοχείας ἀπὸ πόλεως βασιλείας αὐτοῦ, ἔτους ἑβδόμου καὶ τεσσαρακοστοῦ καὶ ἑκατοστοῦ· καὶ διεπέρασε τὸν Εὐφράτην ποταμὸν, καὶ διεπορεύετο τὰς ἐπάνω χώρας.
\par }{\PP \VS{38}Καὶ ἐπέλεξε Λυσίας Πτολεμαῖον τὸν Δορυμένους, καὶ Νικάνορα, καὶ Γοργίαν, ἄνδρας δυνατοὺς τῶν φίλων τοῦ βασιλέως.
\VS{39}Καὶ ἀπέστειλε μετʼ αὐτῶν τεσσαράκοντα χιλιάδας ἀνδρῶν καὶ ἑπτακισχιλίαν ἵππον, τοῦ ἐλθεῖν εἰς γῆν Ἰούδα, καὶ καταφθεῖραι αὐτὴν, κατὰ τὸν λόγον τοῦ βασιλέως.
\VS{40}Καὶ ἀπῇραν σὺν πάσῃ τῇ δυνάμει αὐτῶν, καὶ ἦλθον, καὶ παρενέβαλον πλησίον Ἐμμαοὺμ ἐν τῇ γῇ τῇ πεδινῇ.
\par }{\PP \VS{41}Καὶ ἤκουσαν οἱ ἔμποροι τῆς χώρας τὸ ὄνομα αὐτῶν, καὶ ἔλαβον ἀργύριον καὶ χρυσίον πολὺ σφόδρα καὶ παῖδας, καὶ ἦλθον εἰς τὴν παρεμβολὴν τοῦ λαβεῖν τοὺς υἱοὺς Ἰσραὴλ εἰς παῖδας· καὶ προσετέθησαν πρὸς αὐτοὺς δύναμις Συρίας καὶ γῆς ἀλλοφύλων.
\par }{\PP \VS{42}Καὶ εἶδεν Ἰούδας καὶ οἱ ἀδελφοὶ αὐτοῦ ὅτι ἐπληθύνθη τὰ κακὰ, καὶ αἱ δυνάμεις παρεμβάλλουσιν ἐν τοῖς ὁρίοις αὐτῶν· καὶ ἐπέγνωσαν τοὺς λόγους τοῦ βασιλέως οὓς ἐνετείλατο ποιῆσαι τῷ λαῷ εἰς ἀπώλειαν καὶ συντέλειαν·
\VS{43}καὶ εἶπεν ἕκαστος πρὸς τὸν πλησίον αὐτοῦ, ἀναστήσωμεν τὴν καθαίρεσιν τοῦ λαοῦ ἡμῶν, καὶ πολεμήσωμεν περὶ τοῦ λαοῦ ἡμῶν καὶ τῶν ἁγίων.
\par }{\PP \VS{44}Καὶ συνηθροίσθη ἡ συναγωγὴ τοῦ εἶναι ἑτοίμους εἰς πόλεμον, καὶ τοῦ προσεύξασθαι, καὶ αἰτῆσαι ἔλεον καὶ οἰκτιρμούς.
\par }{\PP \VS{45}Καὶ Ἱερουσαλὴμ ἦν ἀοίκητος ὡς ἔρημος, οὐκ ἦν ὁ εἰσπορευόμενος καὶ ἐκπορευόμενος ἐκ τῶν γεννημάτων αὐτῆς· καὶ τὸ ἁγίασμα καταπατούμενον, καὶ υἱοὶ ἀλλογενῶν ἐν τῇ ἄκρᾳ, κατάλυμα τοῖς ἔθνεσι· καὶ ἐξῄρθη τέρψις ἐξ Ἰακὼβ, καὶ ἐξέλιπεν αὐλὸς καὶ κινύρα.
\VS{46}Καὶ συνήχθησαν, καὶ ἤλθοσαν εἰς Μασσηφὰ κατέναντι Ἱερουσαλὴμ, ὅτι τόπος προσευχῆς εἰς Μασσηφὰ τὸ πρότερον τῷ Ἰσραήλ.
\par }{\PP \VS{47}Καὶ ἐνήστευσαν τῇ ἡμέρᾳ ἐκείνῃ, καὶ περιεβάλοντο σάκκους καὶ σποδὸν ἐπὶ τὰς κεφαλὰς αὐτῶν, καὶ διέῤῥηξαν τὰ ἱμάτια αὐτῶν.
\VS{48}Καὶ ἐξεπέτασαν τὸ βιβλίον τοῦ νόμου, περὶ ὧν ἐξηρεύνων τὰ ἔθνη τὰ ὁμοιώματα τῶν εἰδώλων αὐτῶν.
\VS{49}Καὶ ἤνεγκαν τὰ ἱμάτια τῆς ἱερωσύνης, καὶ τὰ πρωτογεννήματα, καὶ τὰς δεκάτας· καὶ ἤγειραν τοὺς Ναζαραίους, οἳ ἐπλήρωσαν τὰς ἡμέρας.
\par }{\PP \VS{50}Καὶ ἐβόησαν φωνῇ εἰς τὸν οὐρανὸν, λέγοντες, τί ποιήσωμεν τούτοις, καὶ ποῦ αὐτοὺς ἀπαγάγωμεν;
\VS{51}Καὶ τὰ ἅγιά σου καταπεπάτηται, καὶ βεβήλωται· καὶ οἱ ἱερεῖς σου ἐν πένθει καὶ ταπεινώσει.
\VS{52}Καὶ ἰδοὺ τὰ ἔθνη συνῆκται ἐφʼ ἡμᾶς τοῦ ἐξᾶραι ἡμᾶς· σὺ οἶδας ἃ λογίζονται ἐφʼ ἡμᾶς.
\VS{53}Πῶς δυνησόμεθα ὑποστῆναι κατὰ πρόσωπον αὐτῶν, ἐὰν μὴ σὺ βοηθήσῃς ἡμῖν;
\VS{54}Καὶ ἐσάλπισαν ταῖς σάλπιγξι, καὶ ἐβόησαν φωνῇ μεγάλῃ.
\par }{\PP \VS{55}Καὶ μετὰ τοῦτο κατέστησεν Ἰούδας ἡγουμένους τοῦ λαοῦ, χιλιάρχους, καὶ ἑκατοντάρχους, καὶ πεντηκοντάρχους, καὶ δεκάρχους.
\VS{56}Καὶ εἶπον τοῖς οἰκοδομοῦσιν οἰκίας, καὶ μνηστευομένους γυναῖκας, καὶ φυτεύουσιν ἀμπελῶνας, καὶ δειλοῖς, ἀποστρέφειν ἕκαστον εἰς τὸν οἶκον αὐτοῦ, κατὰ τὸν νόμον.
\par }{\PP \VS{57}Καὶ ἀπῇρεν ἡ παρεμβολὴ, καὶ παρενέβαλε κατὰ Νότον Ἐμμαούμ.
\VS{58}Καὶ εἶπεν Ἰούδας, περιζώσασθε, καὶ γίνεσθε εἰς υἱοὺς δυνατοὺς, καὶ γίνεσθε ἕτοιμοι εἰς τοπρωῒ τοῦ πολεμῆσαι ἐν τοῖς ἔθνεσι τούτοις, τοῖς ἐπισυνηγμένοις ἐφʼ ἡμᾶς ἐξᾶραι ἡμᾶς καὶ τὰ ἅγια ἡμῶν.
\VS{59}Ὅτι κρεῖσσον ἡμᾶς ἀποθανεῖν ἐν τῷ πολέμῳ, ἢ ἐπιδεῖν ἐπὶ τὰ κακὰ τοῦ ἔθνους ἡμῶν καὶ τῶν ἁγίων·
\VS{60}Ὡς δʼ ἂν ᾖ θέλημα ἐν οὐρανῷ, οὕτω ποιήσει.

\par }\Chap{4}{\PP \VerseOne{1}Καὶ παρέλαβε Γοργίας πεντακισχιλίους ἄνδρας καὶ χιλίαν ἵππον ἐκλεκτὴν, καὶ ἀπῇρεν ἡ παρεμβολὴ νυκτὸς,
\VS{2}ὥστε ἐπιβαλεῖν ἐπὶ τὴν παρεμβολὴν τῶν Ἰουδαίων, καὶ πατάξαι αὐτοὺς ἄφνω· καὶ οἱ υἱοὶ τῆς ἄκρας ἦσαν αὐτῷ ὁδηγοί.
\VS{3}Καὶ ἤκουσεν Ἰούδας, καὶ ἀπῇρεν αὐτὸς καὶ οἱ δυνατοὶ πατάξαι τὴν δύναμιν τοῦ βασιλέως τὴν ἐν Ἐμμαοὺμ,
\VS{4}ἕως ἔτι αἱ δυνάμεις ἐσκορπισμέναι ἦσαν ἀπὸ τῆς παρεμβολῆς.
\par }{\PP \VS{5}Καὶ ἦλθε Γοργίας εἰς τὴν παρεμβολὴν Ἰούδα νυκτὸς, καὶ οὐδένα εὗρε· καὶ ἐζήτει αὐτοὺς ἐν τοῖς ὄρεσιν, ὅτι εἶπε, φεύγουσιν οὗτοι ἀφʼ ἡμῶν.
\par }{\PP \VS{6}Καὶ ἅμα τῇ ἡμέρᾳ, ὤφθη Ἰούδας ἐν τῷ πεδίῳ ἐν τρισχιλίοις ἀνδράσι· πλὴν καλύμματα καὶ μαχαίρας οὐκ εἶχον καθὼς ἠβούλοντο.
\VS{7}Καὶ εἶδον παρεμβολὴν ἐθνῶν ἰσχυρὰν, τεθωρακισμένην, καὶ ἵππον κυκλοῦσαν αὐτὴν, καὶ οὗτοι διδακτοὶ πολέμου.
\par }{\PP \VS{8}Καὶ εἶπεν Ἰούδας τοῖς ἀνδράσι τοῖς μετʼ αὐτοῦ, μὴ φοβεῖσθε τὸ πλῆθος αὐτῶν, καὶ τὸ ὅρμημα αὐτῶν μὴ δειλωθῆτε.
\VS{9}Μνήσθητε πῶς ἐσώθησαν οἱ πατέρες ἡμῶν ἐν θαλάσσῃ ἐρυθρᾷ, ὅτε ἐδίωξεν αὐτοὺς Φαραὼ ἐν δυνάμει.
\VS{10}Καὶ νῦν βοήσωμεν εἰς τὸν οὐρανὸν, εἴπως ἐλεήσει ἡμᾶς, καὶ μνησθήσεται διαθήκης πατέρων ἡμῶν, καὶ συντρίψει τὴν παρεμβολὴν ταύτην κατὰ πρόσωπον ἡμῶν σήμερον.
\VS{11}Καὶ γνώσεται πάντα τὰ ἔθνη, ὅτι ἐστὶν ὁ λυτρούμενος καὶ σώζων τὸν Ἰσραήλ.
\par }{\PP \VS{12}Καὶ ᾖραν οἱ ἀλλόφυλοι τοὺς ὀφθαλμοὺς αὐτῶν. καὶ ἴδον αὐτοὺς ἐρχομένους ἐξεναντίας,
\VS{13}καὶ ἐξῆλθον ἐκ τῆς παρεμβολῆς εἰς πόλεμον· καὶ ἐσάλπισαν οἱ μετὰ Ἰούδα.
\VS{14}Καὶ συνῆψαν, καὶ συνετρίβησαν τὰ ἔθνη, καὶ ἔφυγον εἰς τὸ πεδίον.
\VS{15}Οἱ δὲ ἔσχατοι πάντες ἔπεσον ἐν ῥομφαίᾳ· καὶ ἐδίωξαν αὐτοὺς ἕως Γαζηρὼν καὶ ἕως τῶν πεδίων τῆς Ἰδουμαίας καὶ Ἀζώτου καὶ Ἰαμνίας, καὶ ἔπεσον ἐξ αὐτῶν εἰς ἄνδρας τρισχιλίους.
\par }{\PP \VS{16}Καὶ ἐπέστρεψεν Ἰούδας καὶ ἡ δύναμις ἀπὸ τοῦ διώκειν ὄπισθεν αὐτῶν,
\VS{17}καὶ εἶπε πρὸς τὸν λαὸν, μὴ ἐπιθυμήσητε τῶν σκύλων, ὅτι πόλεμος ἐξεναντίας ἡμῶν,
\VS{18}καὶ Γοργίας καὶ ἡ δύναμις ἐν τῷ ὄρει ἐγγὺς ἡμῶν· ἀλλὰ στῆτε νῦν ἐναντίον τῶν ἐχθρῶν ἡμῶν, καὶ πολεμήσατε αὐτοὺς, καὶ μετὰ ταῦτα λήψετε τὰ σκύλα μετὰ παῤῥησίας.
\par }{\PP \VS{19}Ἔτι λαλοῦντος Ἰούδα ταῦτα, ὤφθη μέρος τι ἐκκύπτον ἐκ τοῦ ὄρους.
\VS{20}Καὶ εἶδεν ὅτι τετρόπωνται, καὶ ἐμπυρίζουσι τὴν παρεμβολὴν, ὁ γὰρ καπνὸς θεωρούμενος ἐνεφάνιζε τὸ γεγονός.
\VS{21}Οἱ δὲ ταῦτα συνιδόντες ἐδειλώθησαν σφόδρα· συνιδόντες δὲ καὶ τὴν Ἰούδα παρεμβολὴν ἐν τῷ πεδίῳ ἑτοίμην εἰς παράταξιν,
\VS{22}ἔφυγον πάντες εἰς γῆν ἀλλοφύλων.
\VS{23}Καὶ ἀνέστρεψεν Ἰούδας ἐπὶ τὴν σκυλείαν τῆς παρεμβολῆς· καὶ ἔλαβον χρυσίον πολὺ καὶ ἀργύριον καὶ ὑάκινθον καὶ πορφύραν θαλασσίαν καὶ πλοῦτον μέγαν.
\VS{24}Καὶ ἐπιστραφέντες ὕμνουν καὶ εὐλόγουν εἰς οὐρανὸν τὸν Κύριον, ὅτι καλὸν, ὅτι εἰς τὸν αἰῶνα τὸ ἔλεος αὐτοῦ.
\VS{25}Καὶ ἐγένετο σωτηρία μεγάλη τῷ Ἰσραὴλ ἐν τῇ ἡμέρᾳ ἐκείνῃ.
\par }{\PP \VS{26}Ὅσοι δὲ τῶν ἀλλοφύλων διεσώθησαν, παραγενηθέντες ἀπήγγειλαν τῷ Λυσίᾳ πάντα τὰ συμβεβηκότα.
\VS{27}Ὁ δὲ ἀκούσας συνεχύθη καὶ ἠθύμει, ὅτι οὐχ οἷα ἤθελε, τοιαῦτα γεγόνει τῷ Ἰσραὴλ, καὶ οὐχ οἷα ἐνετείλατο αὐτῷ ὁ βασιλεὺς, τοιαῦτα ἐξέβη.
\par }{\PP \VS{28}Καὶ ἐν τῷ ἐχομένῳ ἐνιαυτῷ συνελόχησεν ὁ Λυσίας ἀνδρῶν ἐπιλέκτων ἑξήκοντα χιλιάδας καὶ πεντακισχιλίαν ἵππον, ὥστε ἐκπολεμῆσαι αὐτούς.
\VS{29}Καὶ ἦλθον εἰς τὴν Ἰδουμαίαν, καὶ παρενέβαλον ἐν Βαιθσούροις, καὶ συνήντησεν αὐτοῖς Ἰούδας ἐν δέκα χιλιάσιν ἀνδρῶν.
\par }{\PP \VS{30}Καὶ εἶδε τὴν παρεμβολὴν ἰσχυρὰν, καὶ προσηύξατο, καὶ εἶπεν, εὐλογητὸς εἶ, ὁ σωτὴρ τοῦ Ἰσραὴλ, ὁ συντρίψας τὸ ὅρμημα τοῦ δυνατοῦ ἐν χειρὶ τοῦ δούλου σου Δαυὶδ, καὶ παρέδωκας τὴν παρεμβολὴν τῶν ἀλλοφύλων εἰς χεῖρας Ἰωνάθαν υἱοῦ Σαοὺλ, καὶ τοῦ αἴροντος τὰ σκεύη αὐτοῦ.
\VS{31}Σύγκλεισον τὴν παρεμβολὴν ταύτην ἐν χειρὶ λαοῦ σου Ἰσραὴλ, καὶ αἰσχυνθήτωσαν ἐπὶ τῇ δυνάμει καὶ τῇ ἵππῳ αὐτῶν.
\VS{32}Δὸς αὐτοῖς δειλίαν, καὶ τῆξον θράσος ἰσχύος αὐτῶν, καὶ σαλευθήτωσαν τῇ συντριβῇ αὐτῶν.
\VS{33}Κατάβαλε αὐτοὺς ῥομφαίᾳ ἀγαπώντων σε, καὶ αἰνεσάτωσάν σε πάντες οἱ εἰδότες τὸ ὄνομά σου ἐν ὕμνοις.
\par }{\PP \VS{34}Καὶ συνέβαλον ἀλλήλοις, καὶ ἔπεσον ἐκ τῆς παρεμβολῆς Λυσίου εἰς πεντακισχιλίους ἄνδρας, καὶ ἔπεσον ἐξ ἐναντίας αὐτῶν.
\par }{\PP \VS{35}Ἰδὼν δὲ Λυσίας τὴν γενομένην τροπὴν, τῆς αὐτοῦ συντάξεως, τῆς δὲ Ἰούδα τὸ γεγενημένον θάρσος, καὶ ὡς ἕτοιμοί εἰσιν ἢ ζῇν ἢ τεθνάναι γενναίως, ἀπῇρεν εἰς Ἀντιόχειαν, καὶ ἐξενολόγει· καὶ πλεονάσας τὸν γενηθέντα στρατὸν, ἐλογίζετο πάλιν παραγενέσθαι εἰς τὴν Ἰουδαίαν.
\par }{\PP \VS{36}Εἶπε δὲ Ἰούδας καὶ οἱ ἀδελφοὶ αὐτοῦ, Ἰδοὺ συνετρίβησαν οἱ ἐχθροὶ ἡμῶν, ἀναβῶμεν καθαρίσαι τὰ ἅγια καὶ ἐγκαινίσαι.
\VS{37}Καὶ συνήχθη ἡ παρεμβολὴ πᾶσα, καὶ ἀνέβησαν εἰς ὄρος Σιών.
\VS{38}Καὶ ἴδον τὸ ἁγίασμα ἠρημωμένον, καὶ τὸ θυσιαστήριον βεβηλωμένον, καὶ τὰς πύλας κατακεκαυμένας, καὶ ἐν ταῖς αὐλαῖς φυτὰ πεφυκότα ὡς ἐν δρυμῷ ἢ ὡς ἐν ἑνὶ τῶν ὀρέων, καὶ τὰ παστοφόρια καθῃρημένα.
\VS{39}Καὶ διέῤῥηξαν τὰ ἱμάτια αὐτῶν, καὶ ἐκόψαντο κοπετὸν μέγαν, καὶ ἐπέθεντο σποδὸν ἐπὶ τὴν κεφαλὴν αὐτῶν.
\VS{40}Καὶ ἔπεσον ἐπὶ πρόσωπον ἐπὶ τὴν γῆν, καὶ ἐσάλπισαν ταῖς σάλπιγξι τῶν σημασιῶν, καὶ ἐβόησαν εἰς τὸν οὐρανόν.
\par }{\PP \VS{41}Τότε ἐπέταξεν Ἰούδας ἄνδρας πολεμεῖν τοὺς ἐν τῇ ἄκρᾳ, ἕως ἂν καθαρίσῃ τὰ ἅγια.
\VS{42}Καὶ ἐπέλεξεν ἱερεῖς ἀμώμους, θελητὰς νόμου.
\VS{43}Καὶ ἐκαθάρισαν τὰ ἅγια, καὶ ᾖραν τοὺς λίθους τοῦ μιασμοῦ εἰς τόπον ἀκάθαρτον.
\VS{44}Καὶ ἐβουλεύσαντο περὶ τοῦ θυσιαστηρίου τῆς ὁλοκαυτώσεως τοῦ βεβηλωμένου, τί αὐτῷ ποιήσωσι.
\VS{45}Καὶ ἐπέπεσεν αὐτοῖς βουλὴ ἀγαθὴ, καθελεῖν αὐτὸ, μήποτε γένηται αὐτοῖς εἰς ὄνειδος, ὅτι ἐμίαναν τὰ ἔθνη αὐτό· καὶ καθεῖλον τὸ θυσιαστήριον,
\VS{46}καὶ ἀπέθεντο τοὺς λίθους ἐν τῷ ὄρει τοῦ οἴκου, ἐν τόπῳ ἐπιτηδείῳ, μέχρι τοῦ παραγενηθῆναι προφήτην τοῦ ἀποκριθῆναι περὶ αὐτῶν.
\par }{\PP \VS{47}Καὶ ἔλαβον λίθους ὁλοκλήρους κατὰ τὸν νόμον, καὶ ᾠκοδόμησαν τὸ θυσιαστήριον καινὸν κατὰ τὸ πρότερον.
\VS{48}Καὶ ᾠκοδόμησαν τὰ ἅγια καὶ τὰ ἐντὸς τοῦ οἴκου, καὶ τὰς αὐλὰς ἡγίασαν.
\VS{49}Καὶ ἐποίησαν σκεύη ἅγια καινὰ, καὶ εἰσήνεγκαν τὴν λυχνίαν καὶ τὸ θυσιαστήριον τῶν θυμιαμάτων καὶ τὴν τράπεζαν εἰς τὸν ναόν.
\par }{\PP \VS{50}Καὶ ἐθυμίασαν ἐπὶ τὸ θυσιαστήριον, καὶ ἐξῆψαν τοὺς λύχνους τοὺς ἐπὶ τῆς λυχνίας, καὶ ἐφαίνοσαν ἐν τῷ ναῷ.
\VS{51}Καὶ ἐπέθηκαν ἐπὶ τὴν τράπεζαν ἄρτους, καὶ ἐξεπέτασαν τὰ καταπετάσματα· καὶ ἐτέλεσαν πάντα τὰ ἔργα ἃ ἐποίησαν.
\par }{\PP \VS{52}Καὶ ὤρθρισαν τοπρωῒ τῇ πέμπτῃ καὶ εἰκάδι τοῦ μηνὸς τοῦ ἐννάτου· οὗτος ὁ μὴν Χασελεῦ τοῦ ὀγδόου καὶ τεσσαρακοστοῦ καὶ ἑκατοστοῦ ἔτους.
\VS{53}Καὶ ἀνήνεγκαν θυσίαν κατὰ τὸν νόμον ἐπὶ τὸ θυσιαστήριον τῶν ὁλοκαυτωμάτων τὸ καινὸν ὃ ἐποίησαν.
\VS{54}Κατὰ τὸν καιρὸν καὶ κατὰ τὴν ἡμέραν ἐν ᾗ ἐβεβήλωσαν αὐτὸ τὰ ἔθνη, ἐν ἐκείνῃ ἐνεκαινίσθη ἐν ᾠδαῖς καὶ κιθάραις καὶ κινύραις, καὶ ἐν κυμβάλοις.
\VS{55}Καὶ ἔπεσον πᾶς ὁ λαὸς ἐπὶ πρόσωπον, καὶ προσεκύνησαν, καὶ εὐλόγησαν εἰς οὐρανὸν τὸν εὐοδώσαντα αὐτοῖς.
\par }{\PP \VS{56}Καὶ ἐποίησαν τὸν ἐγκαινισμὸν τοῦ θυσιαστηρίου ἡμέρας ὀκτὼ, καὶ προσήνεγκαν ὁλοκαυτώματα μετʼ εὐφροσύνης, καὶ ἔθυσαν θυσίαν σωτηρίου καὶ αἰνέσεως.
\VS{57}Καὶ κατεκόσμησαν τὸ κατὰ πρόσωπον τοῦ ναοῦ στεφάνοις χρυσοῖς καὶ ἀσπιδίσκαις, καὶ ἐνεκαίνισν τὰς πύλας καὶ τὰ παστοφόρια, καὶ ἐθύρωσαν αὐτά.
\VS{58}Καὶ ἐγενήθη εὐφροσύνη μεγάλη ἐν τῷ λαῷ σφόδρα, καὶ ἀπεστράφη ὄνειδος ἐθνῶν.
\par }{\PP \VS{59}Καὶ ἔστησεν Ἰούδας καὶ οἱ ἀδελφοὶ αὐτοῦ καὶ πᾶσα ἡ ἐκκλησία Ἰσραὴλ, ἵνα ἄγωνται αἱ ἡμέραι ἐγκαινισμοῦ τοῦ θυσιαστηρίου ἐν τοῖς καιροῖς αὐτῶν ἐνιαυτὸν κατʼ ἐνιαυτὸν ἡμέρας ὀκτὼ, ἀπὸ τῆς πέμπτης καὶ εἰκάδος τοῦ μηνὸς Χασελεῦ, μετʼ εὐφροσύνης καὶ χαρᾶς.
\VS{60}Καὶ ᾠκοδόμησαν ἐν τῷ καιρῷ ἐκείνῳ τὸ ὄρος Σιὼν, κυκλόθεν τείχη ὑψηλὰ καὶ πύργους ὀχυροὺς, μήποτε παραγενηθέντα τὰ ἔθνη καταπατήσωσιν αὐτὰ, ὡς ἐποίησαν τοπρότερον.
\VS{61}Καὶ ἐπέταξεν ἐκεῖ δύναμιν τηρεῖν αὐτὸ, καὶ ὠχύρωσαν αὐτὸ τηρεῖν τὴν Βαιθσούραν, τοῦ ἔχειν τὸν λαὸν ὀχύρωμα κατὰ πρόσωπον τῆς Ἰδουμαίας.

\par }\Chap{5}{\PP \VerseOne{1}Καὶ ἐγένετο ὅτε ἤκουσαν τὰ ἔθνη κυκλόθεν ὅτι ᾠκοδομήθη τὸ θυσιαστήριον, καὶ ἐνεκαινίσθη τὸ ἁγίασμα ὡς τοπρότερον, καὶ ὠργίσθησαν σφόδρα.
\VS{2}Καὶ ἐβουλεύσαντο τοῦ ἆραι τὸ γένος Ἰακὼβ τοὺς ὄντας ἐν μέσῳ αὐτῶν, καὶ ἤρξαντο τοῦ θανατοῦν ἐν τῷ λαῷ καὶ ἐξαίρειν.
\par }{\PP \VS{3}Καὶ ἐπολέμει Ἰούδας πρὸς τοὺς υἱοὺς Ἡσαῦ ἐν τῇ Ἰδουμαίᾳ τὴν Ἀκραβαττίνην, ὅτι περιεκάθηντο τὸν Ἰσραὴλ, καὶ ἐπάταξεν αὐτοὺς πληγὴν μεγάλην, καὶ συνέστειλεν αὐτοὺς, καὶ ἔλαβε τὰ σκῦλα αὐτῶν.
\VS{4}Καὶ ἐμνήσθη τῆς κακίας υἱῶν Βαιὰν, οἳ ἦσαν τῷ λαῷ εἰς παγίδα καὶ εἰς σκάνδαλον ἐν τῷ ἐνεδρεύειν αὐτοὺς ἐν ταῖς ὁδοῖς.
\VS{5}Καὶ συνεκλείσθησαν ὑπʼ αὐτοῦ ἐν τοῖς πύργοις, καὶ παρενέβαλεν ἐπʼ αὐτοὺς, καὶ ἀνεθεμάτισεν αὐτοὺς, καὶ ἐνεπύρισε τοὺς πύργους αὐτῆς ἐν πυρὶ σὺν πᾶσι τοῖς ἐνοῦσι.
\par }{\PP \VS{6}Καὶ διεπέρασεν ἐπὶ τοὺς υἱοὺς Ἀμμὼν, καὶ εὗρε χεῖρα κραταιὰν καὶ λαὸν πολὺν, καὶ Τιμόθεον ἡγούμενον αὐτῶν.
\VS{7}καὶ συνῆψε πρὸς αὐτοὺς πολέμους πολλοὺς, καὶ συνετρίβησαν πρὸ προσώπου αὐτοῦ, καὶ ἐπάταξεν αὐτούς.
\VS{8}Καὶ προκατελάβετο τὴν Ἰαζὴρ καὶ τὰς θυγατέρας αὐτῆς, καὶ ἀνέστρεψεν εἰς τὴν Ἰούδαίαν.
\par }{\PP \VS{9}Καὶ ἐπισυνήχθησαν τὰ ἔθνη τὰ ἐν τῇ Γαλαὰδ ἐπὶ τὸν Ἰσραὴλ τοὺς ὄντας ἐπὶ τοῖς ὁρίοις αὐτῶν τοῦ ἐξᾶραι αὐτούς· καὶ ἔφυγον εἰς Δάθεμα τὸ ὀχύρωμα.
\VS{10}Καὶ ἀπέστειλαν γράμματα πρὸς Ἰούδαν καὶ τοὺς ἀδελφοὺς αὐτοῦ, λέγοντες, ἐπισυνηγμένα ἐστὶν ἐφʼ ἡμᾶς τὰ ἔθνη τὰ κύκλῳ ἡμῶν τοῦ ἐξᾶραι ἡμᾶς.
\VS{11}Καὶ ἑτοιμάζονται ἐλθεῖν καὶ προκαταλαβέσθαι τὸ ὀχύρωμα εἰς ὃ κατεφύγομεν, καὶ Τιμόθεος ἡγεῖται τῆς δυνάμεως αὐτῶν.
\par }{\PP \VS{12}Νῦν οὖν ἐλθὼν ἐξελοῦ ἡμᾶς ἐκ χειρὸς αὐτῶν, ὅτι πέπτωκεν ἐξ ἡμῶν πλῆθος.
\VS{13}Καὶ πάντες οἱ ἀδελφοὶ ἡμῶν οἱ ὄντες ἐν τοῖς Τωβίου τεθανάτωνται, καὶ ᾐχμαλωτίκασι τὰς γυναῖκας αὐτῶν καὶ τὰ τέκνα αὐτῶν καὶ τὴν ἀποσκευὴν, καὶ ἀπώλεσαν ἐκεῖ ὡς μίαν χιλιαρχίαν ἀνδρῶν.
\par }{\PP \VS{14}Ἔτι αἱ ἐπιστολαὶ ἀνεγινώσκοντο, καὶ ἰδοὺ ἄγγελοι ἕτεροι παρεγένοντο ἐκ τῆς Γαλιλαίας διεῤῥηχότες τὰ ἱμάτια, ἀπαγγέλλοντες κατὰ τὰ ῥήματα ταῦτα,
\VS{15}λέγοντες ἐπισυνῆχθαι ἐπʼ αὐτοὺς ἐκ Πτολεμαΐδος καὶ Τύρου καὶ Σιδῶνος καὶ πάσης Γαλιλαίας ἀλλοφύλων, τοῦ ἐξαναλῶσαι ἡμᾶς.
\par }{\PP \VS{16}Ὡς δὲ ἤκουσεν Ἰούδας καὶ ὁ λαὸς τοὺς λόγους τούτους, ἐπισυνήχθη ἐκκλησία μεγάλη, βουλεύσασθαι τί ποιήσωσι τοῖς ἀδελφοῖς αὐτῶν τοῖς οὖσιν ἐν θλίψει, καὶ πολεμουμένοις ὑπʼ αὐτῶν.
\VS{17}Καὶ εἶπεν Ἰούδας Σίμωνι τῷ ἀδελφῷ αὐτοῦ, ἐπίλεξον σεαυτῷ ἄνδρας, καὶ πορεύου καὶ ῥῦσαι τοὺς ἀδελφούς σου τοὺς ἐν τῇ Γαλιλαίᾳ· ἐγὼ δὲ καὶ Ἰωνάθαν ὁ ἀδελφός μου πορευσόμεθα εἰς τὴν Γαλααδίτιν.
\VS{18}Καὶ κατέλιπεν Ἰώσηφον τὸν τοῦ Ζαχαρίου, καὶ Ἀζαρίαν, ἡγουμένους τοῦ λαοῦ, μετὰ τῶν ἐπιλοίπων τῆς δυνάμεως, ἐν τῇ Ἰουδαίᾳ εἰς τήρησιν.
\VS{19}Καὶ ἐνετείλατο αὐτοῖς, λέγων, πρόστητε τοῦ λαοῦ τούτου, καὶ μὴ συνάψητε πόλεμον πρὸς τὰ ἔθνη ἕως τοῦ ἐπιστρέψαι ἡμᾶς.
\VS{20}Καὶ ἐμερίσθησαν Σίμωνι ἄνδρες τρισχίλιοι τοῦ πορευθῆναι εἰς τὴν Γαλιλαίαν, Ἰούδᾳ δὲ ἄνδρες ὀκτακισχίλιοι εἰς τὴν Γαλααδίτιν.
\par }{\PP \VS{21}Καὶ ἐπορεύθη Σίμων εἰς τὴν Γαλιλαίαν, καὶ συνῆψε πολέμους πολλοὺς πρὸς τὰ ἔθνη, καὶ συνετρίβη τὰ ἔθνη ἀπὸ προσώπου αὐτοῦ,
\VS{22}καὶ ἐδίωξεν αὐτοὺς ἕως τῆς πύλης Πτολεμαΐδος· καὶ ἔπεσον ἐκ τῶν ἐθνῶν εἰς τρισχιλίους ἄνδρας, καὶ ἔλαβε τὰ σκῦλα αὐτῶν.
\VS{23}Καὶ παρέλαβε τοὺς ἐν τῇ Γαλιλαίᾳ καὶ ἐν Ἀρβάττοις σὺν ταῖς γυναιξὶ καὶ τοῖς τέκνοις, καὶ πάντα ὅσα ἦν αὐτοῖς, καὶ ἤγαγεν εἰς τὴν Ἰουδαίαν μετʼ εὐφροσύνης μεγάλης.
\par }{\PP \VS{24}Καὶ Ἰούδας ὁ Μακκαβαῖος καὶ Ἰωνάθαν ὁ ἀδελφὸς αὐτοῦ διέβησαν τὸν Ἰορδάνην, καὶ ἐπορεύθησαν ὁδὸν τριῶν ἡμερῶν ἐν τῷ ἐρήμῳ.
\VS{25}Καὶ συνήντησαν τοῖς Ναβαταίοις, καὶ ἀπήντησαν αὐτοῖς εἰρηνικῶς, καὶ διηγήσαντο αὐτοῖς ἅπαντα τὰ συμβάντα τοῖς ἀδελφοῖς αὐτῶν ἐν τῇ Γαλααδίτιδι.
\VS{26}Καὶ ὅτι πολλοὶ ἐξ αὐτῶν συνειλημμένοι εἰσὶν εἰς Βόσσορα, καὶ Βοσὸρ, ἐν Ἀλέμοις, Χασφὼρ, Μακὲδ, καὶ Καρναΐν· πᾶσαι αἱ πόλεις αὗται ὀχυραὶ καὶ μεγάλαι·
\VS{27}καὶ ἐν ταῖς λοιπαῖς πόλεσι τῆς Γαλααδίτιδός εἰσι συνειλημμένοι, καὶ εἰς αὔριον τάσσονται παρεμβάλλειν ἐπὶ τὰ ὀχυρώματα, καὶ καταλαβέσθαι, καὶ ἐξᾶραι πάντας τούτους ἐν ἡμέρᾳ μιᾷ.
\par }{\PP \VS{28}Καὶ ἀπέστρεψεν Ἰούδας καὶ ἡ παρεμβολὴ αὐτοῦ ὁδὸν εἰς τὴν ἔρημον εἰς Βοσὸρ, ἄφνω· καὶ κατελάβετο τὴν πόλιν, καὶ ἀπέκτεινε πᾶν ἀρσενικὸν ἐν στόματι ῥομφαίας, καὶ ἔλαβε πάντα τὰ σκῦλα αὐτῶν, καὶ ἐνέπρησεν αὐτὴν πυρί.
\VS{29}Καὶ ἀπῇρεν ἐκεῖθεν νυκτὸς, καὶ ἐπορεύετο ἕως ἐπὶ τὸ ὀχύρωμα.
\par }{\PP \VS{30}Καὶ ἐγένετο ἑωθινὴ, καὶ ᾖραν τοὺς ὀφθαλμοὺς αὐτῶν, καὶ ἰδοὺ λαὸς πολὺς οὗ οὐκ ἦν ἀριθμὸς, αἴροντες κλίμακας καὶ μηχανὰς καταλαβέσθαι τὸ ὀχύρωμα, καὶ ἐπολέμουν αὐτούς.
\VS{31}Καὶ εἶδεν Ἰούδας ὅτι ἦρκται ὁ πόλεμος, καὶ ἡ κραυγὴ τῆς πόλεως ἀνέβη εἰς τὸν οὐρανὸν σάλπιγξι καὶ φωνῇ μεγάλῃ.
\VS{32}Καὶ εἶπε τοῖς ἀνδράσι τῆς δυνάμεως, πολεμήσατε σήμερον ὑπὲρ τῶν ἀδελφῶν ὑμῶν.
\VS{33}Καὶ ἐξῆλθεν ἐν τρισὶν ἀρχαῖς ἐξ ὄπισθεν αὐτῶν· καὶ ἐσάλπισαν ταῖς σάλπιγξι, καὶ ἐβόησαν ἐν προσευχῇ.
\par }{\PP \VS{34}Καὶ ἐπέγνω ἡ παρεμβολὴ Τιμοθέου ὅτι Μακκαβαῖός ἐστι, καὶ ἔφυγον ἀπὸ προσώπου αὐτοῦ, καὶ ἐπάταξεν αὐτοὺς πληγὴν μεγάλην, καὶ ἔπεσον ἐξ αὐτῶν ἐν ἐκείνῃ τῇ ἡμέρᾳ εἰς ὀκτακισχιλίους ἄνδρας.
\par }{\PP \VS{35}Καὶ ἀπέκλινεν εἰς Μασφὰ, καὶ ἐπολέμησεν αὐτὴν, καὶ προκατελάβετο αὐτὴν, καὶ ἀπέκτεινε πᾶν ἀρσενικὸν αὐτῆς, καὶ ἔλαβε τὰ σκῦλα αὐτῆς, καὶ ἐνέπρησεν αὐτὴν πυρί.
\VS{36}Ἐκεῖθεν ἀπῇρε, καὶ προκατελάβετο τὴν Χασφὼν, Μακὲδ, Βοσὸρ, καὶ τὰς λοιπὰς πόλεις τῆς Γαλααδίτιδος.
\par }{\PP \VS{37}Μετὰ δὲ τὰ ῥήματα ταῦτα συνήγαγε Τιμόθεος παρεμβολὴν ἄλλην, καὶ παρενέβαλε κατὰ πρόσωπον Ῥαφὼν ἐκ πέραν τοῦ χειμάῤῥου.
\VS{38}Καὶ ἀπέστειλεν Ἰούδας κατασκοπεῦσαι τὴν παρεμβολήν, καὶ ἀπήγγειλαν αὐτῷ, λέγοντες, ἐπισυνηγμένα εἰσὶ πρὸς αὐτοὺς πάντα τὰ ἔθνη τὰ κύκλῳ ἡμῶν, δύναμις πολλὴ σφόδρα.
\VS{39}Καὶ Ἄραβας μεμίσθωται εἰς βοήθειαν αὐτοῖς, καὶ παρενέβαλον πέραν τοῦ χειμάῤῥου ἕτοιμοι τοῦ ἐλθεῖν ἐπὶ σὲ εἰς πόλεμον· καὶ ἐπορεύθη Ἰούδας εἰς συνάντησιν αὐτῶν.
\par }{\PP \VS{40}Καὶ εἶπε Τιμόθεος τοῖς ἄρχουσι τῆς δυνάμεως αὐτοῦ, ἐν τῷ ἐγγίζειν Ἰούδαν καὶ τὴν παρεμβολὴν αὐτοῦ ἐπὶ τὸν χειμάῤῥουν τοῦ ὕδατος, ἐὰν διαβῇ πρὸς ἡμᾶς πρότερος, οὐ δυνησόμεθα ὑποστῆναι αὐτὸν, ὅτι δυνάμενος δυνήσεται πρὸς ἡμᾶς.
\VS{41}Ἐὰν δὲ δειλωθῇ, καὶ παρεμβάλῃ πέραν τοῦ ποταμοῦ, διαπεράσομεν πρὸς αὐτὸν, καὶ δυνησόμεθα πρὸς αὐτόν.
\par }{\PP \VS{42}Ὡς δὲ ἤγγισεν Ἰούδας ἐπὶ τὸν χειμάῤῥουν τοῦ ὕδατος, ἔστησε τοὺς γραμματεῖς τοῦ λαοῦ ἐπὶ τοῦ χειμάῤῥου, καὶ ἐνετείλατο αὐτοῖς, λέγων, μὴ ἀφῆτε πάντα ἄνθρωπον παρεμβαλεῖν, ἀλλʼ ἐρχέσθωσαν πάντες εἰς τὸν πόλεμον.
\VS{43}Καὶ διεπέρασεν ἐπʼ αὐτοὺς πρότερος, καὶ πᾶς ὁ λαὸς ὅπισθεν αὐτοῦ· καὶ συνετρίβησαν πρὸ προσώπου αὐτοῦ πάντα τὰ ἔθνη, καὶ ἔῤῥιψαν τὰ ὅπλα αὐτῶν, καὶ ἔφυγον εἰς τὸ τέμενος ἐν Καρναΐν.
\VS{44}Καὶ προκατελάβοντο τὴν πόλιν, καὶ τὸ τέμενος ἐνεπύρισαν ἐν πυρὶ σὺν πᾶσι τοῖς ἐν αὐτῷ· καὶ ἐτροπώθη ἡ Καρναῒν, καὶ οὐκ ἐδύναντο ἔτι ὑποστῆναι κατὰ πρόσωπον Ἰούδα.
\par }{\PP \VS{45}Καὶ συνήγαγεν Ἰούδας πάντα Ἰσραὴλ τοὺς ἐν τῇ Γαλααδίτιδι ἀπὸ μικροῦ ἕως μεγάλου, καὶ τὰς γυναῖκας αὐτῶν, καὶ τὰ τέκνα αὐτῶν, καῖ τὴν ἀποσκευὴν, παρεμβολὴν μεγάλην σφόδρα, ἐλθεῖν εἰς γῆν Ἰούδα.
\VS{46}Καὶ ἦλθον ἕως Ἐφρών· καὶ αὕτη ἡ πόλις μεγάλη ἐπὶ τῆς εἰσόδου ὀχυρὰ σφόδρα· οὐκ ἦν ἐκκλῖναι ἀπʼ αὐτῆς δεξιὰν ἢ ἀριστερὰν, ἀλλʼ ἢ διὰ μέσου αὐτῆς πορεύεσθαι.
\VS{47}Καὶ ἀπέκλεισαν αὐτοὺς οἱ ἐκ τῆς πόλεως, καὶ ἐνέφραξαν τὰς πύλας λίθοις.
\VS{48}Καὶ ἀπέστειλε πρὸς αὐτοὺς Ἰούδας λόγοις εἰρηνικοῖς, λέγων, διελευσόμεθα διὰ τῆς γῆς σου τοῦ ἀπελθεῖν εἰς τὴν γῆν ἡμῶν, καὶ οὐδεὶς κακοποιήσει ὑμᾶς, πλὴν τοῖς ποσὶ παρελευσόμεθα· καὶ οὐκ ἠβούλοντο ἀνοῖξαι αὐτῷ.
\par }{\PP \VS{49}Καὶ ἐπέταξεν Ἰούδας κηρύξαι ἐν τῇ παρεμβολῇ, τοῦ παρεμβαλεῖν ἕκαστον ἐν ᾧ ἐστι τόπῳ.
\VS{50}Καὶ παρενέβαλον οἱ ἄνδρες τῆς δυνάμεως, καὶ ἐπολέμησαν τὴν πόλιν ὅλην τὴν ἡμέραν ἐκείνην καὶ ὅλην τὴν νύκτα, καὶ παρεδόθη ἡ πόλις ἐν χερσὶν αὐτοῦ.
\VS{51}Καὶ ἀπώλεσε πᾶν ἀρσενικὸν ἐν στόματι ῥομφαίας, καὶ ἐξεῤῥίζωσεν αὐτὴν, καὶ ἔλαβε τὰ σκῦλα αὐτῆς, καὶ διῆλθε διὰ τῆς πόλεως ἐπάνω τῶν ἀπεκταμμένων.
\par }{\PP \VS{52}Καὶ διέβησαν τὸν Ἰορδάνην εἰς τὸ πεδίον τὸ μέγα κατὰ πρόσωπον Βαιθσάν.
\VS{53}Καὶ ἦν Ἰούδας ἐπισυνάγων τοὺς ἐσχατίζοντας, καὶ παρακαλῶν τὸν λαὸν κατὰ πᾶσαν τὴν ὁδὸν, ἕως οὗ ἦλθον εἰς γῆν Ἰούδα.
\VS{54}Καὶ ἀνέβησαν εἰς τὸ ὄρος Σιὼν ἐν εὐφροσύνῃ καὶ χαρᾷ· καὶ προσήγαγον ὁλοκαυτώματα, ὅτι οὐκ ἔπεσεν ἐξ αὐτῶν οὐθεὶς ἕως τοῦ ἐπιστρέψαι ἐν εἰρήνῃ.
\par }{\PP \VS{55}Καὶ ἐν ταῖς ἡμέραις αἷς ἦν Ἰούδας καὶ Ἰωνάθαν ἐν τῇ Γαλαὰδ, καὶ Σίμων ὁ ἀδελφὸς αὐτοῦ ἐν τῇ Γαλιλαίᾳ κατὰ πρόσωπον Πτολεμαΐδος,
\VS{56}ἤκουσεν Ἰωσὴφ ὁ τοῦ Ζαχαρίου, καὶ Ἀζαρίας, ἄρχοντες τῆς δυνάμεως, τῶν ἀνδραγαθιῶν καὶ τοῦ πολέμου οἷα ἐποίησαν,
\VS{57}καὶ εἶπε, ποιήσωμεν καὶ αὐτοὶ ἑαυτοῖς ὄνομα, καὶ πορευθῶμεν πολεμῆσαι πρὸς τὰ ἔθνη τὰ κύκλῳ ἡμῶν.
\par }{\PP \VS{58}Καὶ παρήγγειλαν τοῖς ἀπὸ τῆς δυνάμεως τῆς μετʼ αὐτῶν, καὶ ἐπορεύθησαν ἐπὶ Ἰάμνειαν.
\VS{59}Καὶ ἐξῆλθε Γοργίας ἐκ τῆς πόλεως, καὶ οἱ ἄνδρες αὐτοῦ, εἰς συνάντησιν αὐτοῖς εἰς πόλεμον.
\VS{60}Καὶ ἐτροπώθη Ἰώσηφος καὶ Ἀζαρίας, καὶ ἐδιώχθησαν ἕως τῶν ὁρίων τῆς Ἰουδαίας· καὶ ἔπεσον ἐν τῇ ἡμέρᾳ ἐκείνῃ ἐκ τοῦ λαοῦ τοῦ Ἰσραὴλ εἰς δισχιλίους ἄνδρας.
\VS{61}Καὶ ἐγενήθη τροπὴ μεγάλη ἐν τῷ λαῷ Ἰσραὴλ, ὅτι οὐκ ἤκουσαν Ἰούδα καὶ τῶν ἀδελφῶν αὐτοῦ, οἰόμενοι ἀνδραγαθῆσαι.
\VS{62}Αὐτοὶ δὲ οὐκ ἦσαν ἐκ τοῦ σπέρματος τῶν ἀνδρῶν ἐκείνων, οἷς ἐδόθη σωτηρία Ἰσραὴλ διὰ χειρὸς αὐτῶν.
\VS{63}Καὶ ὁ ἀνὴρ Ἰούδας καὶ οἱ ἀδελφοὶ αὐτοῦ ἐδοξάσθησαν σφόδρα ἑναντίον παντὸς Ἰσραὴλ, καὶ τῶν ἐθνῶν πάντων, οὗ ἠκούετο τὸ ὄνομα αὐτῶν.
\VS{64}Καὶ ἐπισυνήγοντο πρὸς αὐτοὺς εὐφημοῦντες.
\par }{\PP \VS{65}Καὶ ἐξῆλθεν Ἰούδας καὶ οἱ ἀδελφοὶ αὐτοῦ, καὶ ἐπολέμουν τοὺς υἱοὺς Ἡσαῦ ἐν τῇ γῇ πρὸς Νότον, καὶ ἐπάταξε τὴν Χεβρὼν καὶ τὰς θυγατέρας αὐτῆς, καὶ καθεῖλε τὸ ὀχύρωμα αὐτῆς, καὶ τοὺς πύργους αὐτῆς ἐνέπρησε κυκλόθεν.
\VS{66}Καὶ ἀπῇρε τοῦ πορευθῆναι εἰς γῆν ἀλλοφύλων, καὶ διεπορεύετο τὴν Σαμάρειαν.
\par }{\PP \VS{67}Ἐν τῇ ἡμέρᾳ ἐκείνῃ ἔπεσον ἱερεῖς ἐν πολέμῳ βουλόμενοι ἀνδραγαθῆσαι, ἐν τῷ αὐτοὺς ἐξελθεῖν εἰς πόλεμον ἀβουλεύτως.
\VS{68}Καὶ ἐξέκλινεν Ἰούδας εἰς Ἄζωτον γῆν ἀλλοφύλων, καὶ καθεῖλε τοὺς βωμοὺς αὐτῶν, καὶ τὰ γλυπτὰ τῶν θεῶν αὐτῶν κατέκαυσε πυρὶ, καὶ ἐσκύλευσε τὰ σκῦλα τῶν πόλεων, καὶ ἐπέστρεψεν εἰς τὴν γῆν Ἰούδα.

\par }\Chap{6}{\PP \VerseOne{1}Καὶ ὁ βασιλεὺς Ἀντίοχος διεπορεύετο τὰς ἐνάνω χώρας, καὶ ἤκουσεν ὅτι ἐστὶν Ἐλυμαῒς ἐν τῇ Περσίδι πόλις ἔνδοξος πλούτῳ, ἀργυρίῳ τε καὶ χρυσίῳ,
\VS{2}καὶ τὸ ἱερὸν τὸ ἐν αὐτῇ πλούσιον σφόδρα, καὶ ἐκεῖ καλύμματα χρυσᾶ, καὶ θώρακες, καὶ ὅπλα ἃ κατέλιπεν ἐκεῖ Ἀλέξανδρος ὁ Φιλίππου, βασιλεὺς ὁ Μακεδὼν, ὃς ἐβασίλευσε πρῶτος ἐν τοῖς Ἕλλησι.
\VS{3}Καὶ ἦλθε καὶ ἐζήτει καταλαβέσθαι τὴν πόλιν, καὶ προνομεῦσαι αὐτὴν, καὶ οὐκ ἠδυνάσθη, ὅτι ἐγνώσθη ὁ λόγος τοῖς ἐκ τῆς πόλεως.
\VS{4}Καὶ ἀνέστησαν αὐτῷ εἰς πόλεμον, καὶ ἔφυγε καὶ ἀπῇρεν ἐκεῖθεν μετὰ λύπης μεγάλης, ἀποστρέψαι εἰς Βαβυλῶνα.
\par }{\PP \VS{5}Καὶ ἦλθεν ἀπαγγέλλων τις αὐτῷ εἰς τὴν Περσίδα, ὅτι τετρόπωνται αἱ παρεμβολαὶ αἱ πορευθεῖσαι εἰς γῆν Ἰούδα.
\VS{6}Καὶ ἐπορεύθη Λυσίας δυνάμει ἰσχυρᾷ ἐν πρώτοις, καὶ ἀνετράπη ἀπὸ προσώπου αὐτῶν, καὶ ἐπίσχυσαν ὅπλοις καὶ δυνάμει καὶ σκύλοις πολλοῖς οἷς ἔλαβον ἀπὸ τῶν παρεμβολῶν ὧν ἐξέκοψαν.
\VS{7}Καὶ καθεῖλον τὸ βδέλυγμα ὃ ᾠκοδόμησεν ἐπὶ τὸ θυσιαστήριον τὸ ἐν Ἱερουσαλὴμ, καὶ τὸ ἁγίασμα καθὼς τὸ πρότερον ἐκύκλωσαν τείχεσιν ὑψηλοῖς, καὶ τὴν Βαιθσούραν πόλιν αὐτοῦ.
\par }{\PP \VS{8}Καὶ ἐγένετο ὡς ἤκουσεν ὁ βασιλεὺς τοὺς λόγους τούτους, ἐθαμβήθη καὶ ἐσαλεύθη σφόδρα· καὶ ἔπεσεν ἐπὶ τὴν κοίτην, καὶ ἐνέπεσεν εἰς ἀῤῥωστίαν ἀπὸ τῆς λύπης, ὅτι οὐκ ἐγένετο αὐτῷ καθὼς ἐνεθυμεῖτο.
\VS{9}Καὶ ἦν ἐκεῖ ἡμέρας πλείους, ὅτι ἀνεκαινίσθη ἐπʼ αὐτὸν λύπη μεγάλη, καὶ ἐλογίσατο ὅτι ἀποθνήσκει.
\VS{10}Καὶ ἐκάλεσε πάντας τοὺς φίλους αὐτοῦ, καὶ εἶπε πρὸς αὐτοὺς, ἀφίσταται ὁ ὕπνος ἀπὸ τῶν ὀφθαλμῶν μου, καὶ συμπέπτωκα τῇ καρδίᾳ ἀπὸ τῆς μερίμνης.
\VS{11}Καὶ εἶπα τῇ καρδίᾳ μου, ἕως τίνος θλίψεως ἦλθον καὶ κλύδωνος μεγάλου, ἐν ᾧ νῦν εἰμι; ὅτι χρηστὸς καὶ ἀγαπώμενος ἤμην ἐν τῇ ἐξουσίᾳ μου.
\VS{12}Νῦν δὲ μιμνήσκομαι τῶν κακῶν ὧν ἐποίησα ἐν Ἱερουσαλὴμ, καὶ ἔλαβον πάντα τὰ σκεύη τὰ χρυσᾶ καὶ τὰ ἀργυγᾶ τὰ ἐν αὐτῇ, καὶ ἐξαπέστειλα ἐξᾶραι τοὺς κατοικοῦντας Ἰούδα διακενῆς.
\VS{13}Ἔγνων οὖν ὅτι χάριν τούτων εὗρόν με τὰ κακὰ ταῦτα· καὶ ἰδοὺ ἀπόλλυμαι λύπῃ μεγάλῃ ἐν γῇ ἀλλοτρίᾳ.
\par }{\PP \VS{14}Καὶ ἐκάλεσε Φίλιππον ἕνα τῶν φίλων αὐτοῦ, καὶ κατέστησεν αὐτὸν ἐπὶ πάσης τῆς βασιλείας αὐτοῦ.
\VS{15}Καὶ ἔδωκεν αὐτῷ τὸ διάδημα καὶ τὴν στολὴν αὐτοῦ καὶ τὸν δακτύλιον, τοῦ ἀγαγεῖν Ἀντίοχον τὸν υἱὸν αὐτοῦ, καὶ ἐκθρέψαι αὐτὸν τοῦ βασιλεύειν.
\VS{16}Καὶ ἀπέθανεν ἐκεῖ Ἀντίοχος ὁ βασιλεὺς ἔτους ἐννάτου καὶ τεσσαρακοστοῦ καὶ ἑκατοστοῦ.
\VS{17}Καὶ ἐπέγνω Λυσίας ὅτι τέθνηκεν ὁ βασιλεύς, καὶ κατέστησε βασιλεύειν Ἀντίοχον τὸν υἱὸν αὐτοῦ ἀντʼ αὐτοῦ, ὃν ἐξέθρεψε νεώτερον, καὶ ἐκάλεσε τὸ ὄνομα αὐτοῦ Εὐπάτορα.
\par }{\PP \VS{18}Καὶ οἱ ἐκ τῆς ἄκρας ἦσαν συγκλείοντες τὸν Ἰσραὴλ κύκλῳ τῶν ἁγίων, καὶ ζητοῦντες τὰ κακὰ διʼ ὅλου, καὶ στήριγμα τοῖς ἔθνεσι
\VS{19}Καὶ ἐλογίσατο Ἰούδας ἐξᾶραι αὐτούς· καὶ ἐξεκκλησίασε πάντα τὸν λαὸν τοῦ περικαθίσαι ἐπʼ αὐτοὺς.
\VS{20}Καὶ συνήχθησαν ἅμα, καὶ περιεκάθισαν ἐπʼ αὐτοὺς ἔτους πεντηκοστοῦ καὶ ἑκατοστοῦ, καὶ ἐποίησεν ἐπʼ αὐτοὺς βελοστάσεις καὶ μηχανάς.
\par }{\PP \VS{21}Καὶ ἐξῆλθον ἐξ αὐτῶν ἐκ τοῦ συγκλεισμοῦ, καὶ ἐκολλήθησαν αὐτοῖς τινὲς τῶν ἀσεβῶν ἐξ Ἰσραήλ,
\VS{22}καὶ ἐπορεύθησαν πρὸς τὸν βασιλέα, καὶ εἶπον, ἕως πότε οὐ ποιήσῃ κρίσιν, καὶ ἐκδικήσεις τοὺς ἀδελφοὺς ἡμῶν;
\VS{23}Ἡμεῖς εὐδοκοῦμεν δουλεύειν τῷ πατρί σου, καὶ πορεύεσθαι τοῖς ὑπʼ αὐτοῦ λεγομένοις, καὶ κατακολουθεῖν τοῖς προστάγμασιν αὐτοῦ.
\VS{24}Καὶ περικάθηνται εἰς τὴν ἄκραν υἱοὶ τοῦ λαοῦ ἡμῶν, χάριν τούτου καὶ ἀλλοτριοῦνται ἀφʼ ἡμῶν· πλὴν ὅσοι εὑρίσκοντο ἀφʼ ἡμῶν ἐθανατοῦντο, καὶ αἱ κληρονομίαι ἡμῶν διηρπάζοντο.
\par }{\PP \VS{25}Καὶ οὐκ ἐφʼ ἡμᾶς μόνον ἐξέτειναν χεῖρα, ἀλλὰ καὶ ἐπὶ πάντα τὰ ὅρια αὐτῶν.
\VS{26}Καὶ ἰδοὺ παρεμβεβλήκασι σήμερον ἐπὶ τὴν ἄκραν ἐν Ἱερουσαλὴμ, τοῦ καταλαβέσθαι αὐτὴν, καὶ τὸ ἁγίασμα, καὶ τὴν Βαιθσούραν ὠχύρωσαν.
\VS{27}Καὶ ἐὰν μὴ προκαταλάβῃ αὐτοὺς διατάχους, μείζονα τούτων ποιήσουσι, καὶ οὐ δυνήσῃ τοῦ κατασχεῖν αὐτῶν.
\par }{\PP \VS{28}Καὶ ὠργίσθη ὁ βασιλεὺς ὅτε ἤκουσε, καὶ συνήγαγε πάντας τοὺς φίλους αὐτοῦ, καὶ τοὺς ἄρχοντας τῆς δυνάμεως αὐτοῦ, καὶ τοὺς ἐπὶ τῶν ἡνιῶν.
\VS{29}Καὶ ἀπὸ βασιλειῶν ἑτέρων καὶ ἀπὸ νήσων θαλασσῶν ἦλθον πρὸς αὐτὸν δυνάμεις μισθωταί.
\VS{30}Καὶ ἦν ὁ ἀριθμὸς τῶν δυνάμεων αὐτοῦ ἑκατὸν χιλιάδες τῶν πεζῶν, καὶ εἴκοσι χιλιάδες ἵππων, καὶ ἐλέφαντες δύο καὶ τριάκοντα εἰδότες πόλεμον.
\VS{31}Καὶ ἤλθοσαν διὰ τῆς Ἰδουμαίας, καὶ παρενεβάλοσαν ἐπὶ Βαιθσούραν, καὶ ἐπολέμησαν ἡμέρας πολλὰς, καὶ ἐποίησαν μηχανάς· καὶ ἐξῆλθον, καὶ ἐνεπύρισαν αὐτὰς ἐν πυρὶ, καὶ ἐπολέμησαν ἀνδρωδῶς.
\par }{\PP \VS{32}Καὶ ἀπῇρεν Ἰούδας ἀπὸ τῆς ἄκρας, καὶ παρενέβαλεν εἰς Βαιθζαχαρία ἀπέναντι τῆς παρεμβολῆς τοῦ βασιλέως.
\VS{33}Καὶ ὤρθρισεν ὁ βασιλεὺς τοπρωῒ, καὶ ἀπῇρε τὴν παρεμβολὴν ἐν ὁρμήματι αὐτῆς κατὰ τὴν ὁδὸν Βαιθζαχαρία, καὶ διεσκευάσθησαν αἱ δυνάμεις εἰς τὸν πόλεμον, καὶ ἐσάλπισαν ταῖς σάλπιγξι.
\par }{\PP \VS{34}Καὶ τοῖς ἐλέφασιν ἔδειξαν αἷμα σταφυλῆς καὶ μόρων, τοῦ παραστῆσαι αὐτοὺς εἰς τὸν πόλεμον.
\VS{35}Καὶ διεῖλον τὰ θηρία εἰς τὰς φάλαγγας, καὶ παρέστησαν ἑκάστῳ ἐλέφαντι χιλίους ἄνδρας τεθωρακισμένους ἐν ἁλυσιδωτοῖς, καὶ περικεφαλαῖαι χαλκαῖ ἐπὶ τῶν κεφαλῶν αὐτῶν, καὶ πεντακόσιοι ἵπποι διατεταγμένοι ἑκάστῳ θηρίῳ ἐκλελεγμένοι.
\VS{36}Οὗτοι πρὸ καιροῦ, οὗ ἐὰν ἦν τὸ θηρίον, ἦσαν, καὶ οὗ ἐὰν ἐπορεύετο, ἐπορεύοντο ἅμα, οὐκ ἀφίσταντο ἀπʼ αὐτοῦ.
\VS{37}Καὶ πύργοι ξύλινοι ἐπʼ αὐτοὺς ὀχυροὶ σκεπαζόμενοι ἐφʼ ἑκάστου θηρίου, ἐζωσμένοι ἐπʼ αὐτοῦ μηχαναῖς· καὶ ἐφʼ ἑκάστου ἄνδρες δυνάμεως δύο καὶ τριάκοντα οἱ πολεμοῦντες ἐπʼ αὐτοῖς, καὶ ὁ Ἰνδὸς αὐτοῦ.
\par }{\PP \VS{38}Καὶ τὴν ἐπίλοιπον ἵππον ἔνθεν καὶ ἔνθεν ἔστησαν ἐπὶ τὰ δύο μέρη τῆς παρεμβολῆς, κατασείοντες καὶ καταφρασσόμενοι ἐν ταῖς φάραγξιν.
\VS{39}Ὡς δὲ ἔστιλβεν ὁ ἥλιος ἐπὶ τὰς χρυσᾶς καὶ χαλκᾶς ἀσπίδας, ἔστιλβε τὰ ὄρη ἀπʼ αὐτῶν, καὶ κατηύγαζεν ὡς λαμπάδες πυρός.
\VS{40}Καὶ ἐξετάθη μέρος τι τῆς παρεμβολῆς τοῦ βασιλέως ἐπὶ τὰ ὑψηλὰ ὄρη, καί τινες ἐπὶ ταπεινά· καὶ ἤρχοντο ἀσφαλῶς καὶ τεταγμένως.
\VS{41}Καὶ ἐσαλεύοντο πάντες οἱ ἀκούοντες φωνῆς πλήθους αὐτῶν, καὶ ὁδοιπαρίας τοῦ πλήθους, καὶ συγκρουσμοῦ τῶν ὅπλων· ἦν γὰρ ἡ παρεμβολὴ μεγάλη σφόδρα καὶ ἰσχυρά.
\par }{\PP \VS{42}Καὶ ἤγγισεν Ἰούδας καὶ ἡ παρεμβολὴ αὐτοῦ εἰς παράταξιν· καὶ ἔπεσον ἀπὸ τῆς παρεμβολῆς τοῦ βασιλέως ἑξακόσιοι ἄνδρες.
\VS{43}Καὶ εἶδεν Ἐλεάζαρ ὁ Σαυαρὰν ἓν τῶν θηρίων τεθωρακισμένον θώρακι βασιλικῷ, καὶ ἦν ὑπεράγον πάντα τὰ θηρία, καὶ ὤφθη ὅτι ἐν αὐτῷ ἐστιν ὁ βασιλεύς.
\VS{44}Καὶ ἔδωκεν ἑαυτὸν τοῦ σῶσαι τὸν λαὸν αὐτοῦ, καὶ περιποιῆσαι ἑαυτῷ ὄνομα αἰώνιον.
\VS{45}Καὶ ἐπέδραμεν αὐτῷ θράσει εἰς μέσον τῆς φάλαγγος, καὶ ἐθανάτου δεξιὰ καὶ εὐώνυμα καὶ ἐσχίζοντο ἀπʼ αὐτοῦ ἔνθα καὶ ἔνθα.
\VS{46}Καὶ εἰσέδυ ὑπὸ τὸν ἐλέφαντα, καὶ ὑπέθηκεν αὐτῷ, καὶ ἀνεῖλεν αὐτὸν, καὶ ἔπεσεν ἐπὶ τὴν γῆν ἐπάνω αὐτοῦ, καὶ ἀπέθανεν ἐκεῖ.
\VS{47}Καὶ ἴδον τὴν ἰσχὺν τῆς βασιλείας καὶ τὸ ὅρμημα τῶν δυνάμεων, καὶ ἐξέκλιναν ἀπʼ αὐτῶν.
\par }{\PP \VS{48}Οἱ δὲ ἐκ τῆς παρεμβολῆς τοῦ βασιλέως ἀνέβαινον εἰς συνάντησιν αὐτῶν εἰς Ἱερουσαλήμ· καὶ παρενέβαλεν ὁ βασιλεὺς εἰς τὴν Ἰουδαίαν καὶ εἰς τὸ ὄρος Σιὼν,
\VS{49}καὶ ἐποίησεν εἰρήνην μετὰ τῶν ἐκ Βαιθσούρων· καὶ ἐξῆλθον ἐκ τῆς πόλεως, ὅτι οὐκ ἦν αὐτοῖς ἐκεῖ διατροφὴ τοῦ συγκεκλεῖσθαι ἐν αὐτῇ, ὅτι σάββατον ἦν τῇ γῇ.
\par }{\PP \VS{50}Καὶ κατελάβετο βασιλεὺς τὴν Βαιθσούραν, καὶ ἀπέταξεν ἐκεῖ φρουρὰν τηρεῖν αὐτὴν,
\VS{51}καὶ παρενέβαλεν ἐπὶ τὸ ἁγίασμα ἡμέρας πολλὰς, καὶ ἔστησεν ἐκεῖ βελοστάσεις καὶ μηχανὰς καὶ πυρόβολα καὶ λιθόβολα καὶ σκορπίδια εἰς τὸ βάλλεσθαι βέλη, καὶ σφενδόνας.
\VS{52}Καὶ ἐποίησαν καὶ αὐτοὶ μηχανὰς πρὸς τὰς μηχανὰς αὐτῶν, καὶ ἐπολέμησαν ἡμέρας πολλάς.
\VS{53}Βρώματα δὲ οὐκ ἦν ἐν τοῖς ἀγγείοις, διὰ τὸ ἕβδομον ἔτος εἶναι, καὶ οἱ ἀνασωζόμενοι εἰς τὴν Ἰουδαίαν ἀπὸ τῶν ἐθνῶν κατέφαγον τὸ ὑπόλειμμα τῆς παραθέσεως.
\VS{54}Καὶ ὑπελείφθησαν ἐν τοῖς ἁγίοις ἄνδρες ὀλίγοι, ὅτι κατεκράτησεν αὐτῶν ὁ λιμός· καὶ ἐσκορπίσθησαν ἕκαστος εἰς τὸν τόπον αὐτοῦ.
\par }{\PP \VS{55}Καὶ ἤκουσε Λυσίας, ὅτι Φίλιππος, ὃν κατέστησεν ὁ βασιλεὺς Ἀντίοχος ἔτι ζῶν, ἐκθρέψαι Ἀντίοχον τὸν υἱὸν αὐτοῦ εἰς τὸ βασιλεῦσαι αὐτὸν,
\VS{56}ἀπέστρεψεν ἀπὸ τῆς Περσίδος καὶ Μηδείας, καὶ αἱ δυνάμεις αἱ πορευθεῖσαι τοῦ βασιλέως μετʼ αὐτοῦ, καὶ ὅτι ζητεῖ παραλαβεῖν τὰ πράγματα.
\VS{57}Καὶ κατέσπευσε τοῦ ἀπελθεῖν, καὶ εἰπεῖν πρὸς τὸν βασιλέα καὶ τοὺς ἡγεμόνας τῆς δυνάμεως καὶ τοὺς ἄνδρας, ἐκλείπομεν καθʼ ἡμέραν, καὶ ἡ τροφὴ ἡμῖν ὀλίγη, καὶ ὁ τόπος οὗ παρεμβάλλομεν ἐστιν ὀχυρὸς, καὶ ἐπίκειται ἡμῖν τὰ τῆς βασιλείας.
\VS{58}Νῦν οὖν δῶμεν δεξιὰν τοῖς ἀνθρώποις τούτοις, καὶ ποιήσωμεν μετʼ αὐτῶν εἰρήνην καὶ μετὰ παντὸς ἔθνους αὐτῶν,
\VS{59}καὶ στήσωμεν αὐτοῖς τοῦ πορεύεσθαι τοῖς νομίμοις αὐτῶν, ὡς τοπρότερον· χάριν γὰρ τῶν νομίμων αὐτῶν ὧν διεσκεδάσαμεν, ὠργίσθησαν, καὶ ἐποίησαν ταῦτα πάντα.
\par }{\PP \VS{60}Καὶ ἤρεσεν ὁ λόγος ἐναντίον τοῦ βασιλέως καὶ τῶν ἀρχόντων, καὶ ἀπέστειλε πρὸς αὐτοὺς εἰρηνεῦσαι, καὶ ἐπεδέξαντο.
\VS{61}Καὶ ὤμοσεν αὐτοῖς ὁ βασιλεὺς καὶ οἱ ἄρχοντες· ἐπὶ τούτοις ἐξῆλθον ἐκ τοῦ ὀχυρώματος.
\VS{62}Καὶ εἰσῆλθεν ὁ βασιλεὺς εἰς τὸ ὄρος Σιὼν, καὶ εἶδε τὸ ὀχύρωμα τοῦ τόπου· καὶ ἠθέτησε τὸν ὁρκισμὸν ὃν ὤμοσε, καὶ ἐνετείλατο καθελεῖν τὸ τεῖχος κυκλόθεν.
\VS{63}Καὶ ἀπῇρε κατὰ σπουδὴν, καὶ ἀπέστρεψεν εἰς Ἀντιόχειαν, καὶ εὗρε Φίλιππον κυριεύοντα τῆς πόλεως, καὶ ἐπολέμησε πρὸς αὐτὸν, καὶ κατελάβετο τὴν πόλιν βίᾳ.

\par }\Chap{7}{\PP \VerseOne{1}Ἔτους ἑνὸς καὶ πεντηκοστοῦ καὶ ἐκατοστοῦ ἐξῆλθε Δημήτριος ὁ τοῦ Σελεύκου ἐκ Ῥώμης, καὶ ἀνέβη σὺν ἀνδράσιν ὀλίγοις εἰς πόλιν παραθαλασσίαν, καὶ ἐβασίλευσεν ἐκεῖ.
\par }{\PP \VS{2}Καὶ ἐγένετο ὡς εἰσεπορεύετο εἰς οἶκον βασιλείας πατέρων αὐτοῦ, συνέλαβον αἱ δυνάμεις τὸν Ἀντίοχον καὶ τὸν Λυσίαν ἄγειν αὐτοὺς αὐτῷ.
\VS{3}Καὶ ἐγνώσθη αὐτῷ τὸ πρᾶγμα, καὶ εἶπε, μή μοι δείξητε τὰ πρόσωπα αὐτῶν.
\VS{4}Καὶ ἀπέκτειναν αὐτοὺς αἱ δυνάμεις, καὶ ἐκάθισε Δημήτριος ἐπὶ θρόνου βασιλείας αὐτοῦ.
\VS{5}Καὶ ἦλθον πρὸς αὐτὸν πάντες ἄνδρες ἄνομοι καὶ ἀσεβεῖς ἐξ Ἰσραήλ, καὶ Ἄλκιμος ἡγεῖτο αὐτῶν, βουλόμενος ἱερατεύειν.
\VS{6}Καὶ κατηγόρησαν τοῦ λαοῦ πρὸς τὸν βασιλέα, λέγοντες, ἀπώλεσεν Ἰούδας καὶ οἱ ἀδελφοὶ αὐτοῦ τοὺς φίλους σου, καὶ ἡμᾶς ἐσκόρπισαν ἀπὸ τῆς γῆς ἡμῶν.
\VS{7}Νῦν οὖν ἀπόστειλον ἄνδρα ᾧ πιστεύεις, καὶ πορευθεὶς ἰδέτω τὴν ἐξολόθρευσιν πᾶσαν ἣν ἐποίησεν ἡμῖν καὶ τῇ χώρᾳ τοῦ βασιλέως, καὶ κολασάτω αὐτοὺς καὶ πάντας τοὺς ἐπιβοηθοῦντας αὐτοῖς.
\par }{\PP \VS{8}Καὶ ἐπέλεξεν ὁ βασιλεὺς τὸν Βακχίδην τῶν φίλων τοῦ βασιλέως, κυριεύοντα ἐν τῷ πέραν τοῦ ποταμοῦ, καὶ μέγαν ἐν τῇ βασιλείᾳ, καὶ πιστὸν τῷ βασιλεῖ.
\VS{9}Καὶ ἀπέστειλεν αὐτὸν καὶ Ἄλκιμον τὸν ἀσεβῆ, καὶ ἔστησεν αὐτῷ τὴν ἱερωσύνην, καὶ ἐνετείλατο αὐτῷ ποιῆσαι τὴν ἐκδίκησιν ἐν τοῖς υἱοῖς Ἰσραήλ.
\VS{10}Καὶ ἀπῇραν, καὶ ἦλθον μετὰ δυνάμεως πολλῆς εἰς γῆν Ἰούδα· καὶ ἀπέστειλεν ἀγγέλους πρὸς Ἰούδαν, καὶ τοὺς ἀδελφοὺς αὐτοῦ, λόγοις εἰρηνικοῖς μετὰ δόλου.
\VS{11}Καὶ οὐ προσέσχον τοῖς λόγοις αὐτῶν, ἴδον γὰρ ὅτι ἦλθον μετὰ δυνάμεως πολλῆς.
\par }{\PP \VS{12}Καὶ ἐπισυνήχθησαν πρὸς Ἄλκιμον καὶ Βακχίδην συναγωγὴ γραμματέων ἐκζητῆσαι δίκαια.
\VS{13}Καὶ πρῶτοι οἱ Ἀσιδαῖοι ἦσαν ἐν υἱοῖς Ἰσραὴλ, καὶ ἐπεζήτουν παρὰ αὐτῶν εἰρήνην·
\VS{14}Εἶπαν γὰρ, ἄνθρωπος ἱερεὺς ἐκ τοῦ σπέρματος Ἀαρὼν ἦλθεν ἐν ταῖς δυνάμεσι, καὶ οὐκ ἀδικήσει ἡμᾶς.
\VS{15}Καὶ ἐλάλησε μετʼ αὐτῶν λόγους εἰρηνικοὺς, καὶ ὤμοσεν αὐτοῖς, λέγων, οὐκ ἐκζητήσομεν ὑμῖν κακὸν, καὶ τοῦ φίλοις ὑμῶν.
\VS{16}Καὶ ἐνεπίστευσαν αὐτῷ· καὶ συνέλαβεν ἐξ αὐτῶν ἑξήκοντα ἄνδρας, καὶ ἀπέκτεινεν αὐτοὺς ἐν ἡμέρᾳ μιᾷ, κατὰ τὸν λόγον ὃν ἔγραψε,
\VS{17}σάρκας ὁσίων σου καὶ αἵματα αὐτῶν ἐξέχεαν κύκλῳ Ἱερουσαλὴμ, καὶ οὐκ ἦν αὐτοῖς ὁ θάπτων.
\VS{18}Καὶ ἐπέπεσεν αὐτῶν ὁ φόβος καὶ ὁ τρόμος ἐπὶ πάντα τὸν λαὸν, ὅτι εἶπαν, οὐκ ἔστιν ἐν αὐτοῖς ἀλήθεια καὶ κρίσις· παρέβησαν γὰρ τὴν στάσιν καὶ τὸν ὅρκον ὃν ὤμοσαν.
\par }{\PP \VS{19}Καὶ ἀπῇρε Βακχίδης ἀπὸ Ἱερουσαλὴμ, καὶ παρενέβαλεν ἐν Βηζὲθ, καὶ ἀπέστειλε καὶ συνέλαβε πολλοὺς ἀπὸ τῶν ἀπʼ αὐτοῦ αὐτομολησάντων ἀνδρῶν, καί τινας τοῦ λαοῦ, καὶ ἔθυσεν αὐτοὺς εἰς τὸ φρέαρ τὸ μέγα.
\VS{20}Καὶ κατέστησε τὴν χώραν τῷ Ἀλκίμῳ, καὶ ἀφῆκε μετʼ αὐτοῦ δύναμιν τοῦ βοηθεῖν αὐτῷ· καὶ ἀπῆλθε Βακχίδης πρὸς τὸν βασιλέα.
\VS{21}Καὶ ἠγωνίσατο Ἄλκιμος περὶ τῆς ἀρχιερωσύνης.
\VS{22}Καὶ συνήχθησαν πρὸς αὐτὸν πάντες οἱ ταράσσοντες τὸν λαὸν αὐτῶν, καὶ κατεκράτησαν γῆν Ἰούδα, καὶ ἐποίησαν πληγὴν μεγάλην ἐν Ἰσραήλ.
\par }{\PP \VS{23}Καὶ εἶδεν Ἰούδας πᾶσαν τὴν κακίαν ἣν ἐποίησεν Ἄλκιμος καὶ οἱ μετʼ αὐτοῦ ἐν υἱοῖς Ἰσραὴλ ὑπὲρ τὰ ἔθνη·
\VS{24}καὶ ἐξῆλθεν εἰς πάντα τὰ ὅρια τῆς Ἰουδαίας κυκλόθεν, καὶ ἐποίησεν ἐκδίκησιν ἐν τοῖς ἀνδράσι τοῖς αὐτομολήσασι, καὶ ἀνεστάλησαν τοῦ πορεύεσθαι εἰς τὴν χώραν.
\par }{\PP \VS{25}Ὡς δὲ εἶδεν Ἄλκιμος ὅτι ἐνίσχυσεν Ἰούδας καὶ οἱ μετʼ αὐτοῦ, καὶ ἔγνω ὅτι οὐ δύναται ὑποστῆναι αὐτούς, καὶ ἐπέστρεψε πρὸς τὸν βασιλέα, καὶ κατηγόρησεν αὐτῶν πονηρά.
\par }{\PP \VS{26}Καὶ ἀπέστειλεν ὁ βασιλεὺς Νικάνορα, ἕνα τῶν ἀρχόντων αὐτοῦ τῶν ἐνδόξων, καὶ μισοῦντα καὶ ἐχθραίνοντα τῷ Ἰσραήλ, καὶ ἐνετείλατο αὐτῷ ἐξᾶραι τὸν λαόν.
\VS{27}Καὶ ἦλθε Νικάνωρ εἰς Ἰερουσαλὴμ δυνάμει πολλῇ, καὶ ἀπέστειλε πρὸς Ἰούδαν καὶ τοὺς ἀδελφοὺς αὐτοῦ μετὰ δόλου λόγοις εἰρηνικοῖς, λέγων,
\VS{28}μὴ ἔστω μάχη ἀναμέσον ἐμοῦ καὶ ὑμῶν· ἥξω ἐν ἀνδράσιν ὀλίγοις, ἵνα ὑμῶν ἴδω τὰ πρόσωπα μετʼ εἰρήνης.
\VS{29}Καὶ ἦλθε πρὸς Ἰούδαν, καὶ ἠσπάσαντο ἀλλήλους εἰρηνικῶς· καὶ οἱ πολέμιοι ἦσαν ἕτοιμοι ἐξαρπάσαι τὸν Ἰούδαν.
\VS{30}Καὶ ἐγνώσθη ὁ λόγος τῷ Ἰούδᾳ, ὅτι μετὰ δόλου ἦλθεν ἐπʼ αὐτόν· καὶ ἐπτοήθη ἀπʼ αὐτοῦ, καὶ οὐκ ἐβουλήθη ἔτι ἰδεῖν τὸ πρόσωπον αὐτοῦ.
\par }{\PP \VS{31}Καὶ ἔγνω Νικάνωρ ὅτι ἀπεκαλύφθη ἡ βουλὴ αὐτοῦ, καὶ ἐξῆλθεν εἰς συνάντησιν τῷ Ἰούδᾳ ἐν πολέμῳ κατὰ Χαφαρσαλαμά.
\VS{32}Καὶ ἔπεσον τῶν παρὰ Νικάνορος ὡσεὶ πεντακισχίλιοι ἄνδρες, καὶ ἔφυγον εἰς τὴν πόλιν Δαυίδ.
\par }{\PP \VS{33}Καὶ μετὰ τοὺς λόγους τούτους ἀνέβη Νικάνωρ εἰς τὸ ὄρος Σιών· καὶ ἐξῆλθον ἀπὸ τῶν ἱερέων ἐκ τῶν ἁγίων καὶ ἀπὸ τῶν πρεσβυτέρων τοῦ λαοῦ ἀσπάσασθαι αὐτὸν εἰρηνικῶς, καὶ δεῖξαι αὐτῷ τὴν ὁλοκαύτωσιν τὴν προσφερομένην ὑπὲρ τοῦ βασιλέως.
\VS{34}Καὶ ἐμυκτήρισεν αὐτοὺς, καὶ κατεγέλασεν αὐτῶν, καὶ ἐμίανεν αὐτούς, καὶ ἐλάλησεν ὑπερηφάνως.
\VS{35}Καὶ ὤμοσε μετὰ θυμοῦ, λέγων, ἐὰν μὴ παραδοθῇ Ἰούδας καὶ ἡ παρεμβολὴ αὐτοῦ εἰς χεῖράς μου τὸ νῦν, καὶ ἔσται ἐὰν ἐπιστρέψω ἐν εἰρήνῃ, ἐμπυριῶ τὸν οἶκον τοῦτον· καὶ ἐξῆλθε μετὰ θυμοῦ μεγάλου.
\par }{\PP \VS{36}Καὶ εἰσῆλθον οἱ ἱερεῖς, καὶ ἔστησαν κατὰ πρόσωπον τοῦ θυσιαστηρίου καὶ τοῦ ναοῦ, καὶ ἔκλαυσαν, καὶ εἶπον,
\VS{37}σὺ, Κύριε, ἐξελέξω τὸν οἶκον τοῦτον ἐπικληθῆναι τὸ ὄνομά σου ἐπʼ αὐτῷ, εἶναι οἶκον προσευχῆς καὶ δεήσεως τῷ λαῷ σου.
\VS{38}Ποίησον ἐκδίκησιν ἐν τῷ ἀνθρώπῳ τούτῳ καὶ ἐν τῇ παρεμβολῇ αὐτοῦ, καὶ πεσέτωσαν ἐν ῥομφαίᾳ· μνήσθητι τῶν δυσφημιῶν αὐτῶν, καὶ μὴ δῷς αὐτοῖς μονήν.
\par }{\PP \VS{39}Καὶ ἐξῆλθε Νικάνωρ ἐξ Ἱερουσαλὴμ, καὶ παρενέβαλεν ἐν Βαιθωρὼν, καὶ συνήντησεν αὐτῷ δύναμις Συρίας.
\VS{40}Καὶ Ἰούδας παρενέβαλεν ἐν Ἀδασὰ ἐν τρισχιλίοις ἀνδράσι· καὶ προσηύξατο Ἰούδας, καὶ εἶπεν,
\VS{41}οἱ παρὰ τοῦ βασιλέως Ἀσσυρίων ὅτε ἐδυσφήμησαν, ἐξῆλθεν ὁ ἄγγελός σου, Κύριε, καὶ ἐπάταξεν ἐν αὐτοῖς ἑκατὸν ὀγδοηκονταπέντε χιλιάδας.
\VS{42}Οὕτω σύντριψον τὴν παρεμβολὴν ταύτην ἐνώπιον ἡμῶν σήμερον, καὶ γνώτωσαν οἱ ἐπίλοιποι, ὅτι κακῶς ἐλάλησαν ἐπὶ τὰ ἅγιά σου, καὶ κρῖνον αὐτὸν κατὰ τὴν κακίαν αὐτοῦ.
\par }{\PP \VS{43}Καὶ συνῆψαν αἱ παρεμβολαὶ εἰς πόλεμον τῇ τρισκαιδεκάτῃ τοῦ μηνὸς Ἄδαρ, καὶ συνετρίβη ἡ παρεμβολὴ Νικάνορος, καὶ ἔπεσεν αὐτὸς πρῶτος ἐν τῷ πολέμῳ.
\par }{\PP \VS{44}Ὡς δὲ εἶδεν ἡ παρεμβολὴ αὐτοῦ ὅτι ἔπεσε Νικάνωρ, ῥίψαντες τὰ ὅπλα αὐτῶν ἔφυγον.
\VS{45}Καὶ κατεδίωκον αὐτοὺς ὁδὸν ἡμέρας μιᾶς ἀπὸ Ἀδασὰ ἕως τοῦ ἐλθεῖν εἰς Γάζηρα, καὶ ἐσάλπισαν ὀπίσω αὐτῶν ταῖς σάλπιγξι τῶν σημασιῶν.
\VS{46}Καὶ ἐξῆλθον ἐκ πασῶν τῶν κωμῶν τῆς Ἰουδαίας κυκλόθεν, καὶ ὑπερεκέρων αὐτοὺς, καὶ ἀνέστρεφον οὗτοι πρὸς τούτους· καὶ ἔπεσον πάντες ῥομφαίᾳ, καὶ οὐ κατελείφθη ἐξ αὐτῶν οὐδὲ εἷς.
\par }{\PP \VS{47}Καὶ ἔλαβον τὰ σκῦλα καὶ τὴν προνομὴν, καὶ τὴν κεφαλὴν Νικάνορος ἀφεῖλον, καὶ τὴν δεξιὰν αὐτοῦ ἣν ἐξέτεινεν ὑπερηφάνως, καὶ ἤνεγκαν, καὶ ἐξέτειναν παρὰ τὴν Ἱερουσαλήμ.
\VS{48}Καὶ εὐφράνθη δ λαὸς σφόδρα, καὶ ἤγαγον τὴν ἡμέραν ἐκείνην ἡμέραν εὐφροσύνης μεγάλης.
\VS{49}Καὶ ἔστησαν τοῦ ἄγειν κατὰ ἐνιαυτὸν τὴν ἡμέραν ταύτην τὴν τρισκαιδεκάτην τοῦ Ἄδαρ.
\VS{50}Καὶ ἡσύχασεν ἡ γῆ Ἰούδα ἡμέρας ὀλίγας.

\par }\Chap{8}{\PP \VerseOne{1}Καὶ ἤκουσεν Ἰούδας τὸ ὄνομα τῶν Ῥωμαίων, ὅτι εἰσὶ δυνατοὶ ἰσχύϊ· καὶ αὐτοὶ εὐδοκοῦσιν ἐν πᾶσι τοῖς προστιθεμένοις αὐτοῖς, καὶ ὅσοι ἂν προσέλθωσιν αὐτοῖς, ἱστῶσιν αὐτοῖς φιλίαν, καὶ ὅτι εἰσὶ δυνατοὶ ἰσχύϊ·
\VS{2}καὶ διηγήσαντο αὐτῷ τοὺς πολέμους αὐτῶν, καὶ τὰς ἀνδραγαθίας ἃς ποιοῦσιν ἐν τοῖς Γαλάταις, καὶ ὅτι κατεκράτησαν αὐτῶν καὶ ἤγαγον αὐτοὺς ὑπὸ φόρον,
\VS{3}καὶ ὅσα ἐποίησαν ἐν χώρᾳ Ἱσπανιας, του κατακρατῆσαι τῶν μετάλλων τοῦ ἀργυρίου καὶ τοῦ χρυσίου τοῦ ἐκεῖ·
\VS{4}καὶ κατεκράτησαν τοῦ τόπου παντὸς τῇ βουλῇ αὐτῶν καὶ τῇ μακροθυμίᾳ, καὶ ὁ τόπος ἦν μακρὰν ἀπέχων ἀπʼ αὐτῶν σφόδρα· καὶ τῶν βασιλέων τῶν ἐπελθόντων ἐπʼ αὐτοὺς ἀπʼ ἄκρου τῆς γῆς ἕως συνέτριψαν αὐτοὺς, καὶ ἐπάταξαν ἐν αὐτοῖς πληγὴν μεγάλην, καὶ οἱ ἐπίλοιποι διδόασιν αὐτοῖς φόρον κατʼ ἐνιαυτόν·
\par }{\PP \VS{5}Καὶ τὸν Φίλιππον καὶ τὸν Περσέα Κιτιέων βασιλέα, καὶ τοὺς ἐπῃρμένους ἐπʼ αὐτοὺς, συνέτριψαν αὐτοὺς ἐν πολέμῳ, καὶ κατεκράτησαν αὐτῶν·
\VS{6}καὶ Ἀντίοχον τὸν μέγαν βασιλέα τῆς Ἀσίας, τὸν πορευθέντα ἐπʼ αὐτοὺς εἰς πόλεμον ἔχοντα ἑκατὸν εἴκοσι ἐλέφαντας καὶ ἵππον καὶ ἅρματα καὶ δύναμιν πολλὴν σφόδρα, καὶ συνετρίβη ἀπʼ αὐτῶν·
\VS{7}καὶ ἔλαβον αὐτὸν ζῶντα, καὶ ἔστησαν αὐτοῖς διδόναι αὐτόν τε καὶ τοὺς βασιλεύοντας μετʼ αὐτὸν φόρον μέγαν, διδόναι ὅμηρα καὶ διαστολήν,
\VS{8}καὶ χώραν τὴν Ἰνδικὴν, καὶ Μήδειαν, καὶ Αυδίαν, καὶ ἀπὸ τῶν καλλίστων χωρῶν αὐτῶν, καὶ λαβόντες αὐτὰς παρʼ αὐτοῦ ἔδωκαν αὐτὰς Εὐμένει τῷ βασιλεῖ.
\par }{\PP \VS{9}Καὶ ὅτι οἱ ἐκ τῆς Ἑλλάδος ἐβουλεύσαντο ἐλθεῖν καὶ ἐξᾶραι αὐτούς,
\VS{10}καὶ ἐγνώσθη ὁ λόγος αὐτοῖς, καὶ ἀπέστειλαν ἐπʼ αὐτοὺς στρατηγὸν ἕνα, καὶ ἐπολέμησαν πρὸς αὐτοὺς, καὶ ἔπεσον ἐξ αὐτῶν τραυματίαι πολλοὶ, καὶ ᾐχμαλώτευσαν τὰς γυναῖκας αὐτῶν καὶ τὰ τέκνα αὐτῶν, καὶ προενόμευσαν αὐτοὺς, καὶ κατεκράτησαν τῆς γῆς αὐτῶν, καὶ καθεῖλον τὰ ὀχυρώματα αὐτῶν, καὶ κατεδουλώσαντο αὐτοὺς ἕως τῆς ἡμέρας ταύτης.
\par }{\PP \VS{11}Καὶ τὰς ἐπιλοίπους βασιλείας, καὶ τὰς νήσους, ὅσοι ποτὲ ἀντέστησαν αὐτοῖς, κατέφθειραν, καὶ ἐδούλωσαν αὐτούς· μετὰ δὲ τῶν φίλων αὐτῶν καὶ τῶν ἐπαναπαυομένων αὐτοῖς συνετήρησαν φιλίαν,
\VS{12}καὶ κατεκράτησαν τῶν βασιλειῶν τῶν ἐγγὺς καὶ τῶν μακρὰν, καὶ ὅσοι ἤκουον τὸ ὄνομα αὐτῶν ἐφοβοῦντο ἀπʼ αὐτῶν·
\VS{13}ὅσοις δʼ ἂν βούλωνται βοηθεῖν καὶ βασιλεύειν, βασιλεύουσιν· οὓς δʼ ἂν βούλωνται, μεθιστῶσι, καὶ ὑψώθησαν σφόδρα·
\VS{14}καὶ ἐν πᾶσι τούτοις οὐκ ἐπέθετο οὐδεὶς αὐτῶν διάδημα, καὶ οὐ περιεβάλοντο πορφύραν, ὥστε ἁδρυνθῆναι ἐν αὐτῇ.
\VS{15}Καὶ βουλευτήριον ἐποίησαν ἑαυτοῖς, καὶ καθʼ ἡμέραν ἐβουλεύοντο τριακόσιοι καὶ εἴκοσι βουλευόμενοι διαπαντὸς περὶ τοῦ πλήθους, τοῦ εὐκοσμεῖν αὐτούς·
\VS{16}καὶ πιστεύουσιν ἑνὶ ἀνθρώπῳ τὴν ἀρχὴν αὐτῶν κατʼ ἐνιαυτὸν, καὶ κυριεύειν πάσης τῆς γῆς αὐτῶν, καὶ πάντες ἀκούουσι τοῦ ἑνὸς, καὶ οὐκ ἔστι φθόνος οὐδὲ ζῆλος ἐν αὐτοῖς·
\par }{\PP \VS{17}Καὶ ἐπέλεξεν Ἰούδας τὸν Εὐπόλεμον υἱὸν Ἰωάννου τοῦ Ἀκκὼς, καὶ Ἰάσονα υἱὸν Ἐλεαζάρου, καὶ ἀπέστειλεν αὐτοὺς εἰς Ῥώμην, στῆσαι αὐτοῖς φιλίαν καὶ συμμαχίαν,
\VS{18}καὶ τοῦ ἆραι τὸν ζυγὸν ἀπʼ αὐτῶν, ὅτι ἴδον τὴν βασιλείαν τῶν Ἑλλήνων καταδουλουμένους τὸν Ἰσραὴλ δουλείαν.
\par }{\PP \VS{19}Καὶ ἐπορεύθησαν εἰς Ῥώμην, καὶ ἡ ὁδὸς πολλὴ σφόδρα, καὶ εἰσῆλθον εἰς τὸ βουλευτήριον, καὶ ἀπεκρίθησαν καὶ εἶπον,
\VS{20}Ἰούδας ὁ Μακκαβαῖος καὶ οἱ ἀδελφοὶ αὐτοῦ καὶ τὸ πλῆθος τῶν Ἰουδαίων ἀπέστειλαν ἡμᾶς πρὸς ὑμᾶς, στῆσαι μεθʼ ὑμῶν συμμαχίαν καὶ εἰρήνην, καὶ γραφῆναι ἡμᾶς συμμάχους καὶ φίλους ὑμῶν.
\VS{21}Καὶ ἤρεσεν ὁ λόγος ἐνώπιον αὐτῶν.
\par }{\PP \VS{22}Καὶ τοῦτο τὸ ἀντίγραφον τῆς ἐπιστολῆς ἧς ἀντέγραψεν ἐπὶ δέλτοῖς χαλκαῖς, καὶ ἀπέστειλεν εἰς Ἱερουσαλὴμ εἶναι παρʼ αὐτοῖς ἐκεῖ μνημόσυνον εἰρήνης καὶ συμμαχίας·
\VS{23}καλῶς γένοιτο Ῥωμαίοις καὶ τῷ ἔθνει Ἰουδαίων ἐν τῇ θαλάσσῃ καὶ ἐπὶ τῆς ξηρᾶς εἰς τὸν αἰῶνα, καὶ ῥομφαία καὶ ἐχθρὸς μακρυνθείη ἀπʼ αὐτῶν.
\par }{\PP \VS{24}Ἐὰν δὲ ἐνστῇ πόλεμος ἐν Ῥώμῃ προτέρᾳ ἢ πᾶσι τοῖς συμμάχοις αὐτῶν ἐν πάσῃ κυρείᾳ αὐτῶν,
\VS{25}συμμαχήσει τὸ ἔθνος τῶν Ἰουδαίων, ὡς ἂν ὁ καιρὸς ὑπογραφῇ αὐτοῖς, καρδίᾳ πλήρει.
\VS{26}Καὶ τοῖς πολεμοῦσιν οὐ δώσουσιν οὐδὲ ἐπαρκέσουσι σῖτον, ὅπλα, ἀργύριον, πλοῖα, ὡς ἔδοξε Ῥωμαίοις· καὶ φυλάξονται τὰ φυλάγματα αὐτῶν οὐθὲν λαβόντες·
\VS{27}κατὰ τὰ αὐτὰ δὲ ἐὰν ἔθνει Ἰουδαίων συμβῇ προτέροις πόλεμος, συμμαχήσουσιν οἱ Ῥωμαῖοι ἐκ ψυχῆς, ὡς ἂν αὐτοῖς ὁ καιρὸς ὑπογραφῇ.
\VS{28}Καὶ τοῖς συμμαχοῦσιν οὐ δοθήσεται σῖτος, ὅπλα, ἀργύριον, πλοῖα, ὡς ἔδοξε Ῥώμῃ· καὶ φυλάξονται τὰ φυλάγματα αὐτῶν, καὶ οὐ μετὰ δόλου.
\par }{\PP \VS{29}Κατὰ τοὺς λόγους τούτους ἔστησαν Ῥωμαῖοι τῷ δήμῳ τῶν Ἰουδαίων.
\VS{30}Ἐὰν δὲ μετὰ τοὺς λόγους τούτους βουλεύσωνται οὗτοι καὶ οὗτοι προσθεῖναι ἢ ἀφελεῖν, ποιήσονται ἐξ αἱρέσεως αὐτῶν, καὶ ὃ ἐὰν προσθῶσιν ἢ ἀφέλωσιν, ἔσται κύρια.
\par }{\PP \VS{31}Καὶ περὶ τῶν κακῶν ὧν ὁ βασιλεὺς Δημήτριος συντελεῖται εἰς αὐτούς, ἐγράψαμεν αὐτῷ, λέγοντες, διατί ἐβάρυνας τὸν ζυγόν σου ἐπὶ τοὺς φίλους ἡμῶν τοὺς συμμάχους Ἰουδαίους;
\VS{32}Ἐὰν οὖν ἔτι ἐντύχωσιν κατὰ σοῦ, ποιήσομεν αὐτοῖς τὴν κρίσιν, καὶ πολεμήσομέν σε διὰ τῆς θαλάσσης καὶ διὰ τῆς ξηρᾶς.

\par }\Chap{9}{\PP \VerseOne{1}Καὶ ἤκουσε Δημήτριος ὅτι ἔπεσε Νικάνωρ καὶ αἱ δυνάμεις αὐτοῦ ἐν πολέμῳ, καὶ προσέθετο τὸν Βακχίδην καὶ τὸν Ἄλκιμον ἐκ δευτέρου ἀποστεῖλαι εἰς γῆν Ἰούδα, καὶ τὸ δεξιὸν κέρας μετʼ αὐτῶν.
\VS{2}Καὶ ἐπορεύθησαν ὁδὸν τὴν εἰς Γάλγαλα, καὶ παρενέβαλον ἐπὶ Μαισαλὼθ τὴν ἐν Ἀρβήλοις, καὶ προκατελάβοντο αὐτὴν, καὶ ἀπώλεσαν ψυχὰς ἀνθρώπων πολλάς.
\VS{3}Καὶ τοῦ μηνὸς τοῦ πρώτου ἔτους τοῦ δευτέρου καὶ πεντηκοστοῦ καὶ ἑκατοστοῦ παρενέβαλον ἐπὶ Ἱερουσαλήμ.
\VS{4}Καὶ ἀπῇραν καὶ ἐπορεύθησαν εἰς Βερέαν ἐν εἴκοσι χιλιάσιν ἀνδρῶν καὶ δισχιλίᾳ ἵππῳ.
\par }{\PP \VS{5}Καὶ Ἰούδας ἦν παρεμβεβληκὼς ἐν Ἐλεασὰ, καὶ τρισχίλιοι ἄνδρες ἐκλεκτοὶ μετʼ αὐτοῦ.
\VS{6}Καὶ ἴδον τὸ πλῆθος τῶν δυνάμεων ὅτι πολλοί εἰσι, καὶ ἐφοβήθησαν σφόδρα· καὶ ἐξεῤῥύησαν πολλοὶ ἀπὸ τῆς παρεμβολῆς, οὐ κατελείφθησαν ἐξ αὐτῶν ἀλλʼ ἢ ὀκτακόσιοι ἄνδρες.
\par }{\PP \VS{7}Καὶ εἶδεν Ἰούδας ὅτι ἀπεῤῥύη ἡ παρεμβολὴ αὐτοῦ, καὶ ὁ πόλεμος ἔθλιβεν αὐτόν· καὶ συνετρίβη τῇ καρδίᾳ, ὅτι οὐκ εἶχε καιρὸν συναγαγεῖν αὐτούς.
\VS{8}Καὶ ἐξελύθη, καὶ εἶπε τοῖς καταλειφθεῖσιν, ἀναστῶμεν καὶ ἀναβῶμεν ἐπὶ τοὺς ὑπεναντίους ἡμῶν, ἐὰν ἄρα δυνώμεθα πολεμῆσαι αὐτούς.
\VS{9}Καὶ ἀπέστρεψαν αὐτὸν, λέγοντες, οὐ μὴ δυνώμεθα, ἀλλʼ ἢ σώζωμεν τὰς ἑαυτῶν ψυχὰς τὸ νῦν, καὶ ἐπιστρέψωμεν μετὰ τῶν ἀδελφῶν ἡμῶν, καὶ πολεμήσωμεν πρὸς αὐτοὺς, ἡμεῖς δὲ ὀλίγοι.
\par }{\PP \VS{10}Καὶ εἶπεν Ἰούδας, μή μοι γένοιτο ποιῆσαι τὸ πρᾶγμα τοῦτο, φυγεῖν ἀπʼ αὐτῶν, καὶ εἰ ἤγγικεν ὁ καιρὸς ἡμῶν, καὶ ἀποθάνωμεν ἐν ἀνδρείᾳ χάριν τῶν ἀδελφῶν ἡμῶν, καὶ μὴ καταλίπωμεν αἰτίαν τῇ δόξῃ ἡμῶν.
\VS{11}Καὶ ἀπῇρεν ἡ δύναμις ἀπὸ τῆς παρεμβολῆς, καὶ ἔστησαν εἰς συνάντησιν αὐτοῖς, καὶ ἐμερίσθη ἡ ἵππος εἰς δύο μέρη, καὶ οἱ σφενδονηταὶ καὶ οἱ τοξόται προεπορεύοντο τῆς δυνάμεως, καὶ οἱ πρωταγωνισταὶ πάντες οἱ δυνατοί.
\par }{\PP \VS{12}Βακχίδης δὲ ἦν ἐν τῷ δεξιῷ κέρατι, καὶ ἤγγισεν ἡ φάλαγξ ἐκ τῶν δύο μερῶν, καὶ ἐφώνουν ταῖς σάλπιγξι. Καὶ ἐσάλπισαν οἱ παρὰ Ἰούδα καὶ αὐτοὶ ταῖς σάλπιγξι,
\VS{13}καὶ ἐσαλεύθη ἡ γῆ ἀπὸ τῆς φωνῆς τῶν παρεμβολῶν· καὶ ἐγένετο ὁ πόλεμος συνημένος ἀπὸ πρωΐθεν ἕως ἑσπέρας.
\par }{\PP \VS{14}Καὶ· εἶδεν Ἰούδας ὅτι Βακχίδης καὶ τὸ στερέωμα τῆς παρεμβολῆς ἐν τοῖς δεξιοῖς, καὶ συνῆλθον αὐτῷ πάντες οἱ εὔψυχοι τῇ καρδίᾳ.
\VS{15}Καὶ συνετρίβη τὸ δεξιὸν κέρας ἀπʼ αὐτῶν, καὶ ἐδίωκεν ὀπίσω αὐτῶν ἕως Ἀζώτου ὄρους.
\VS{16}Καὶ οἱ εἰς τὸ ἀριστερὸν κέρας ἴδον ὅτι συνετρίβη τὸ δεξιὸν κέρας, καὶ ἐπέστρεψαν κατὰ πόδας Ἰούδα καὶ τῶν μετʼ αὐτοῦ ἐκ τῶν ὄπισθεν.
\VS{17}Καὶ ἐβαρύνθη ὁ πόλεμος, καὶ ἔπεσον τραυματίαι πολλοὶ ἐκ τούτων καὶ ἐκ τούτων.
\VS{18}Καὶ Ἰούδας ἔπεσε, καὶ οἱ λοιποὶ ἔφυγον.
\par }{\PP \VS{19}Καὶ ᾖραν Ἰωνάθαν καὶ Σίμων Ἰούδαν τὸν ἀδελφὸν αὐτῶν, καὶ ἔθαψαν αὐτὸν ἐν τῷ τάφῳ τῶν πατέρων αὐτοῦ ἐν Μωδεεΐμ.
\VS{20}Καὶ ἔκλαυσαν αὐτὸν, καὶ ἐκόψαντο αὐτὸν πᾶς Ἰσραὴλ κοπετὸν μέγαν, καὶ ἐπένθουν ἡμέρας πολλὰς, καὶ εἶπον,
\VS{21}πῶς ἔπεσε δυνατὸς, σώζων τὸν Ἰσραήλ;
\VS{22}Καὶ τὰ περισσὰ τῶν λόγων Ἰούδα, καὶ τῶν πολέμων, καὶ τῶν ἀνδραγαθιῶν ὧν ἐποίησε, καὶ τῆς μεγαλωσύνης αὐτῶν, οὐ κατεγράφη, πολλὰ γὰρ ἦν σφόδρα.
\par }{\PP \VS{23}Καὶ ἐγένετο μετὰ τὴν τελευτὴν Ἰούδα, ἐξέκυψαν οἱ ἄνομοι ἐν πᾶσι τοῖς ὁρίοις Ἰσραὴλ, καὶ ἀνέτειλαν πάντες οἱ ἐργαζόμενοι τὴν ἀδικίαν.
\VS{24}Ἐν ταῖς ἡμέραις ἐκείναις ἐγενήθη λιμὸς μέγας σφόδρα, καὶ ηὐτομόλησεν ἡ χώρα μετʼ αὐτῶν.
\par }{\PP \VS{25}Καὶ ἐξέλεξε Βακχίδης τοὺς ἀσεβεῖς ἄνδρας, καὶ κατέστησεν αὐτοὺς κυρίους τῆς χώρας.
\VS{26}Καὶ ἐξεζήτουν καὶ ἐξηρεύνων τοὺς φίλους Ἰούδα, καὶ ἦγον αὐτοὺς πρὸς Βακχίδην· καὶ ἐξεδίκει ἐν αὐτοῖς, καὶ ἐνέπαιζεν αὐτοῖς.
\VS{27}Καὶ ἐγένετο θλίψις μεγάλη ἐν τῷ Ἰσραὴλ, ἥτις οὐκ ἐγένετο ἀφʼ ἧς ἡμέρας οὐκ ὤφθη προφήτης ἐν αὐτοῖς.
\par }{\PP \VS{28}Καὶ ἠθροίσθησαν πάντες οἱ φίλοι Ἱούδα, καὶ εἶπον τῷ Ἰωνάθαν,
\VS{29}ἀφʼ οὗ ὁ ἀδελφός σου Ἰούδας τετελεύτηκε, καὶ ἀνὴρ ὅμοιος αὐτῷ οὐκ ἔστιν ἐξελθεῖν πρὸς τοὺς ἐχθροὺς καὶ Βακχίδην, καὶ ἐν τοῖς ἐχθραίνουσι τοῦ ἔθνους ἡμῶν.
\VS{30}Νῦν οὖν σε ᾑρετισάμεθα σήμερον, τοῦ εἶναι ἀντʼ αὐτοῦ ἡμῖν εἰς ἄρχοντα καὶ ἡγούμενον, τοῦ πολεμῆσαι τὸν πόλεμον ἡμῶν.
\VS{31}Καὶ ἐπεδέξατο Ἰωνάθαν ἐν τῷ καιρῷ ἐκείνῳ τὴν ἥγησιν, καὶ ἀνέστη ἀντὶ Ἰούδα τοῦ ἀδελφοῦ αὐτοῦ.
\VS{32}Καὶ ἔγνω Βακχίδης, καὶ ἐζήτει αὐτὸν ἀποκτεῖναι.
\par }{\PP \VS{33}Καὶ ἔγνω Ἰωνάθαν, καὶ Σίμων ὁ ἀδελφὸς αὐτοῦ, καὶ πάντες οἱ μετʼ αὐτοῦ, καὶ ἔφυγον εἰς τὴν ἔρημον Θεκωὲ, καὶ παρενέβαλον ἐπὶ τὸ ὕδωρ λάκκου Ἀσφάρ.
\VS{34}Καὶ ἔγνω Βακχίδης τῇ ἡμέρᾳ τῶν σαββάτων, καὶ ἦλθεν αὐτὸς καὶ πᾶν τὸ στράτευμα αὐτοῦ πέραν τοῦ Ἰορδάνου.
\VS{35}Καὶ ἀπέστειλεν Ἰωνάθαν τὸν ἀδελφὸν αὐτοῦ ἡγούμενον τοῦ ὄχλου, καὶ παρεκάλεσε τοὺς Ναυαταίους φίλους αὐτοῦ παραθέσθαι αὐτοῖς τὴν ἀποσκευὴν αὐτῶν τὴν πολλήν.
\VS{36}Καὶ ἐξῆλθον υἱοὶ Ἰαμβρὶ ἐκ Μηδαβὰ, καὶ συνέλαβον Ἰωάννην, καὶ πάντα ὅσα εἶχε, καὶ ἀπῆλθον ἔχοντες.
\par }{\PP \VS{37}Μετὰ δὲ τοὺς λόγους τούτους ἀπήγγειλαν τῷ Ἰωνάθαν καὶ Σίμωνι τῷ ἀδελφῷ αὐτοῦ, ὅτι οἱ υἱοὶ Ἰαμβρὶ ποιοῦσι γάμον μέγαν, καὶ ἄγουσι τὴν νύμφην ἀπὸ Ναδαβὰθ, θυγατέρα ἑνὸς τῶν μεγιστάνων μεγάλων τῶν Χαναὰν, μετὰ παραπομπῆς μεγάλης.
\VS{38}Καὶ ἐμνήσθησαν Ἰωάννου τοῦ ἀδελφοῦ αὐτῶν, καὶ ἀνέβησαν, καὶ ἐκρύβησαν ὑπὸ τὴν σκέπην τοῦ ὄρους.
\VS{39}Καὶ ᾖραν τοὺς ὀφθαλμοὺς αὐτῶν, καὶ ἴδον, καὶ ἰδοὺ θροῦς, καὶ ἀποσκευὴ πολλὴ, καὶ ὁ νυμφίος ἐξῆλθε καὶ οἱ φίλοι αὐτοῦ καὶ οἱ ἀδελφοὶ αὐτοῦ εἰς συνάντησιν αὐτῶν μετὰ τυμπάνων, και μουσικῶν, καὶ ὅπλων πολλῶν.
\par }{\PP \VS{40}Καὶ ἐξανέστησαν ἐπʼ αὐτοὺς ἀπὸ τοῦ ἐνέδρου οἱ περὶ τὸν Ἰωνάθαν, καὶ ἀπέκτειναν αὐτοὺς, καὶ ἔπεσον τραυματίαι πολλοὶ, καὶ οἱ ἐπίλοιποι ἔφυγον εἰς τὸ ὄρος· καὶ ἔλαβον πάντα τὰ σκῦλα αὐτῶν.
\VS{41}Καὶ μετεστράφη ὁ γάμος εἰς πένθος, καὶ ἡ φωνὴ μουσικῶν αὐτῶν εἰς θρῆνον.
\VS{42}Καὶ ἐξεδίκησαν τὴν ἐκδίκησιν αἵματος ἀδελφοῦ αὐτῶν, καὶ ἀπέστρεψαν εἰς τὸ ἕλος τοῦ Ἰορδάνου.
\par }{\PP \VS{43}Καὶ ἤκουσε Βακχίδης, καὶ ἦλθε τῇ ἡμέρᾳ τῶν σαββάτων ἕως τῶν κρηπίδων τοῦ Ἰορδάνου ἐν δυνάμει πολλῇ.
\VS{44}Καὶ εἶπεν Ἰωνάθαν τοῖς παρʼ αὐτοῦ, ἀναστῶμεν νῦν καὶ πολεμήσωμεν ὑπὲρ τῶν ψυχῶν ἡμῶν, οὐ γὰρ ἐστι σήμερον ὡς ἐχθὲς καὶ τρίτην ἡμέραν.
\VS{45}Ἰδοὺ γὰρ ὁ πόλεμος ἐξεναντίας ἡμῶν καὶ ἐξόπισθεν ἡμῶν· τὸ δὲ ὕδωρ τοῦ Ἰορδάνου ἔνθεν καὶ ἔνθεν, καὶ ἕλος καὶ δρυμὸς, οὐκ ἔστι τόπος τοῦ ἐκκλῖναι.
\VS{46}Νῦν οὖν κεκράξατε εἰς οὐρανὸν, ὅπως διασωθῆτε ἐκ χειρὸς ἐχθρῶν ὑμῶν.
\VS{47}Καὶ συνῆψεν ὁ πόλεμος· καὶ ἐξέτεινεν Ἰωνάθαν τὴν χεῖρα αὐτοῦ πατάξαι τὸν Βακχίδην, καὶ ἐξέκλινεν ἀπʼ αὐτοῦ εἰς τὰ ὀπίσω.
\VS{48}Καὶ ἐνεπήδησεν Ἰωνάθαν καὶ οἱ μετʼ αὐτοῦ εἰς τὸν Ἰορδάνην, καὶ διεκολύμβησαν εἰς τὸ πέραν· καὶ οὐ διέβησαν ἐπʼ αὐτοὺς τὸν Ἰορδάνην.
\VS{49}Καὶ διέπεσον παρὰ Βακχίδου τῇ ἡμέρᾳ ἐκείνῃ εἰς χιλίους ἄνδρας.
\par }{\PP \VS{50}Καὶ ἐπέστρεψεν εἰς Ἱερουσαλὴμ, καὶ ᾠκοδόμησε πόλεις ὀχυρὰς ἐν τῇ Ἰουδαίᾳ, τὸ ὀχύρωμα τὸ ἐν Ἱεριχὼ, καὶ τὴν Ἐμμαοὺμ, καὶ τὴν Βαιθωρῶν, καὶ τὴν Βαιθὴλ, καὶ τὴν Θαμναθὰ, Φαραθωνὶ, καὶ τὴν Τεφὼν ἐν τείχεσιν ὑψηλοῖς καὶ πύλαις καὶ μοχλοῖς.
\VS{51}Καὶ ἔθετο φρουρὰν ἐν αὐτοῖς τοῦ ἐχθραίνειν τῷ Ἰσραήλ.
\VS{52}Καὶ ὠχύρωσε τὴν πόλιν τὴν ἐν Βαιθσούρᾳ, καὶ τὴν Γάζαρα, καὶ τὴν ἄκραν, καὶ ἔθετο ἐν αὐταῖς δυνάμεις καὶ παραθέσεις βρωμάτων.
\VS{53}Καὶ ἔλαβε τοὺς υἱοὺς τῶν ἡγουμένων τῆς χώρας ὅμηρα, καὶ ἔθετο αὐτοὺς ἐν τῇ ἄκρᾳ ἐν Ἱερουσαλὴμ ἐν φυλακῇ.
\par }{\PP \VS{54}Καὶ ἐν ἔτει τρίτῳ καὶ πεντηκοστῷ καὶ ἑκατοστῷ, μηνι τῷ δευτέρῳ, ἐπέταξεν Ἄλκιμος καθαιρεῖν τὸ τεῖχος τῆς αὐλῆς τῶν ἁγίων τῆς ἐσωτέρας, καὶ καθεῖλε τὰ ἔργα τῶν προφητῶν, καὶ ἐνήρξατο τοῦ καθαιρεῖν.
\VS{55}Ἐν τῷ καιρῷ ἐκείνῳ ἐπλήγη Ἄλκιμος, καὶ ἐνεποδίσθη τὰ ἔργα αὐτοῦ, καὶ ἀπεφράγη τὸ στόμα αὐτοῦ, καὶ παρελύθη, καὶ οὐκ ἐδύνατο ἔτι λαλῆσαι λόγον καὶ ἐντείλασθαι περὶ τοῦ οἴκου αὐτοῦ.
\VS{56}Καὶ ἀπέθανεν Ἄλκιμος ἐν τῷ καιρῷ ἐκείνῳ μετὰ βασάνου μεγάλης.
\par }{\PP \VS{57}Καὶ εἶδε Βακχίδης ὅτι ἀπέθανεν Ἄλκιμος, καὶ ἀπέστρεψε πρὸς τὸν βασιλέα· καὶ ἡσύχασεν ἡ γῆ Ἰούδα ἔτη δύο.
\VS{58}Καὶ ἐβουλεύσαντο πάντες οἱ ἄνομοι, λέγοντες, ἰδοὺ Ἰωνάθαν καὶ οἱ παρʼ αὐτοῦ ἐν ἡσυχίᾳ κατοικοῦσι πεποιθότες· νῦν οὖν ἄξομεν τὸν Βακχίδην, καὶ συλλήμψεται αὐτοὺς πάντας ἐν νυκτὶ μιᾷ.
\VS{59}Καὶ πορευθέντες συνεβουλεύσαντο αὐτῷ.
\VS{60}Καὶ ἀπῇρε τοῦ ἐλθεῖν μετὰ δυνάμεως πολλῆς, καὶ ἀπέστειλεν ἐπιστολὰς λάθρα πᾶσι τοῖς συμμάχοις αὐτοῦ τοῖς ἐν τῇ Ἰουδαίᾳ, ὅπως συλλάβωσι τὸν Ἰωνάθαν, καὶ τοὺς μετʼ αὐτοῦ· καὶ οὐκ ἐδύναντο, ὅτι ἐγνώσθη αὐτοῖς ἡ βουλὴ αὐτῶν.
\VS{61}Καὶ συνελάβοντο ἀπὸ τῶν ἀνδρῶν τῆς χώρας τῶν ἀρχηγῶν τῆς κακίας εἰς πεντήκοντα ἄνδρας, καὶ ἀπέκτειναν αὐτούς.
\par }{\PP \VS{62}Καὶ ἐξεχώρησεν Ἰωνάθαν, καὶ Σίμων, καὶ οἱ μετʼ αὐτοῦ εἰς Βαιθβασὶ τὴν ἐν τῇ ἐρήμῳ, καὶ ᾠκοδόμησε τὰ καθῃρημένα αὐτῆς, καὶ ἐστερέωσαν αὐτήν.
\VS{63}Καὶ ἔγνω Βακχίδης, καὶ συνήγαγε πᾶν τὸ πλῆθος αὐτοῦ, καὶ τοῖς ἐκ τῆς Ἰουδαίας παρήγγειλε.
\par }{\PP \VS{64}Καὶ ἐλθὼν παρενέβαλεν ἐπὶ Βαιθβασὶ, καὶ ἐπολέμησεν αὐτὴν ἡμέρας πολλὰς, καὶ ἐποίησε μηχανάς.
\VS{65}Καὶ ἀπέλιπεν Ἰωνάθαν Σίμωνα τὸν ἀδελφὸν αὐτοῦ ἐν τῇ πόλει, καὶ ἐξῆλθεν εἰς τὴν χώραν, καὶ ἐξῆλθεν ἐν ἀριθμῷ.
\VS{66}Καὶ ἐπάταξεν Ὀδοαῤῥὴν, καὶ τοὺς ἀδελφοὺς αὐτοῦ, καὶ τοὺς υἱοὺς Φασιρὼν ἐν τῷ σκηνώματι αὐτῶν, καὶ ἐξήρξατο τύπτειν, καὶ ἀναβαίνειν ἐν δυνάμεσι·
\VS{67}καὶ Σίμων, καὶ οἱ μετʼ αὐτοῦ ἐξῆλθον ἐκ τῆς πόλεως, καὶ ἐνεπύρισαν τὰς μηχανάς.
\VS{68}Καὶ ἐπολέμησαν πρὸς τὸν Βακχίδην, καὶ συνετρίβη ὑπʼ αὐτῶν, καὶ ἔθλιβον αὐτὸν σφόδρα, ὅτι ἦν ἡ βουλὴ αὐτοῦ καὶ ἡ ἔφοδος αὐτοῦ κενή.
\VS{69}Καὶ ὠργίσθηθυμῷ τοῖς ἀνδράσι τοῖς ἀνόμοις τοῖς συμβουλεύσασιν αὐτῷ ἐλθεῖν εἰς τὴν χώραν, καὶ ἀπέκτειναν ἐξ αὐτῶν πολλούς, καὶ ἐβουλεύσατο τοῦ ἀπελθεῖν εἰς τὴν γῆν αὐτοῦ.
\par }{\PP \VS{70}Καὶ ἐπέγνω Ἰωνάθαν, καὶ ἀπέστειλε πρὸς αὐτὸν πρέσβεις, τοῦ συνθέσθαι πρὸς αὐτὸν εἰρήνην, καὶ ἀποδοῦναι αὐτοῖς τὴν αἰχμαλωσίαν.
\VS{71}Καὶ ἀπεδέξατο, καὶ ἐποίησε κατὰ τοὺς λόγους αὐτοῦ, καὶ ὤμοσεν αὐτῷ μὴ ἐκζητῆσαι αὐτῷ κακὸν πάσας τὰς ἡμέρας τῆς ζωῆς αὐτοῦ.
\VS{72}Καὶ ἀπέδωκεν αὐτῷ τὴν αἰχμαλωσίαν ἣν ᾐχμαλώτευσε τοπρότερον ἐκ γῆς Ἰούδα· καὶ ἀποστρέψας ἀπῆλθεν εἰς τὴν γῆν αὐτοῦ, καὶ οὐ προσέθετο ἔτι ἐλθεῖν εἰς τὰ ὅρια αὐτῶν.
\VS{73}Καὶ κατέπαυσε ῥομφαία ἐξ Ἰσραήλ· καὶ ᾤκησεν Ἰωνάθαν ἐν Μαχμάς· καὶ ἤρξατο Ἰωνάθαν κρίνειν τὸν λαὸν καὶ ἠφάνισε τοὺς ἀσεβεῖς ἐξ Ἰσραήλ.

\par }\Chap{10}{\PP \VerseOne{1}Καὶ ἐν ἔτει ἑξηκοστῷ καὶ ἑκατοστῷ ἀνέβη Ἀλέξανδρος ὁ τοῦ Ἀντιόχου ὁ Ἐπιφανής, καὶ κατελάβετο Πτολεμαΐδα, καὶ ἐπεδέξαντο αὐτὸν, καὶ ἐβασίλευσεν ἐκεῖ.
\VS{2}Καὶ ἤκουσε Δημήτριος ὁ βασιλεύς, καὶ συνήγαγε δυνάμεις πολλὰς σφόδρα, καὶ ἐξῆλθεν εἰς συνάντησιν αὐτῷ εἰς πόλεμον.
\VS{3}Καὶ ἀπέστειλε Δημήτριος πρὸς Ἰωνάθαν ἐπιστολὰς λόγοις εἰρηνικοῖς ὥστε μεγαλῦναι αὐτόν.
\VS{4}Εἶπε γάρ, προφθάσωμεν τοῦ εἰρήνην θεῖναι μετʼ αὐτοῦ, πρινὴ θεῖναι αὐτὸν μετὰ Ἀλεξάνδρου καθʼ ἡμῶν.
\VS{5}Μνησθήσεται γὰρ πάντων τῶν κακῶν ὧν συνετελέσαμεν πρὸς αὐτόν, καὶ εἰς τοὺς ἀδελφοὺς αὐτοῦ, καὶ εἰς τὸ ἔθνος αὐτοῦ.
\VS{6}Καὶ ἔδωκεν αὐτῷ ἐξουσίαν συναγαγεῖν δυνάμεις, καὶ κατασκευάζειν ὅπλα, καὶ εἶναι αὐτὸν σύμμαχον αὐτοῦ, καὶ τὰ ὅμηρα τὰ ἐν τῇ ἄκρᾳ εἶπε παραδοῦναι αὐτῷ.
\par }{\PP \VS{7}Καὶ ἦλθεν Ἰωνάθαν εἰς Ἱερουσαλὴμ, καὶ ἀνέγνω τὰς ἐπιστολὰς εἰς τὰ ὦτα παντὸς τοῦ λαοῦ, καὶ τῶν ἐκ τῆς ἄκρας.
\VS{8}Καὶ ἐφοβήθησα φόβον μέγαν ὅτε ἤκουσαν ὅτι ἔδωκεν αὐτῷ ὁ βασιλεὺς ἐξουσίαν συναγαγεῖν δυνάμεις.
\VS{9}Καὶ παρέδωκαν οἱ ἐκ τῆς ἄκρας Ἰωνάθαν τὰ ὅμηρα, καὶ ἀπέδωκεν αὐτοὺς τοῖς γονεῦσιν αὐτῶν.
\par }{\PP \VS{10}Καὶ ᾤκησεν Ἰωνάθαν ἐν Ἱερουσαλήμ, καὶ ἤρξατο οἰκοδομεῖν καὶ καινίζειν τὴν πόλιν.
\VS{11}Καὶ εἶπε πρὸς τοὺς ποιοῦντας τὰ ἔργα, οἰκοδομεῖν τὰ τείχη, καὶ τὸ ὄρος Σιὼν κυκλόθεν ἐκ λίθων τετραγώνων εἰς ὀχύρωσιν· καὶ ἐποίησαν οὕτως.
\par }{\PP \VS{12}Καὶ ἔφυγον οἱ ἀλλογενεῖς οἱ ὄντες ἐν τοῖς ὀχυρώμασιν οἷς ᾠκοδόμησε Βακχίδης.
\VS{13}Καὶ κατέλιπεν ἕκαστος τὸν τόπον αὐτοῦ, καὶ ἀπῆλθεν εἰς τὴν γῆν αὐτοῦ.
\VS{14}Πλὴν ἐν Βαιθσούρᾳ ὑπελείφθησάν τινες τῶν καταλιπόντων τὸν νόμον καὶ τὰ προστάγματα, ἦν γὰρ αὐτοῖς φυγαδευτήριον.
\par }{\PP \VS{15}Καὶ ἤκουσεν Ἀλέξανδρος ὁ βασιλεὺς τὰς ἐπαγγελίας ὅσας ἀπέστειλε Δημήτριος τῷ Ἰωνάθαν, καὶ διηγήσαντο αὐτῷ τοὺς πολέμους καὶ τὰς ἀνδραγαθίας ἃς ἐποίησεν αὐτὸς καὶ οἱ ἀδελφοὶ αὐτοῦ, καὶ τοὺς κόπους οὓς ἔσχον,
\VS{16}καὶ εἶπε, μὴ εὑρήσομεν ἄνδρα τοιοῦτον ἕνα; καὶ νῦν ποιήσομεν αὐτὸν φίλον, καὶ σύμμαχον ἡμῶν.
\par }{\PP \VS{17}Καὶ ἔγραψεν ἐπιστολὰς, καὶ ἀπέστειλεν αὐτῷ κατὰ τοὺς λόγους τούτους, λέγων,
\VS{18}βασιλεὺς Ἀλέξανδρος τῷ ἀδελφῷ Ἰωνάθαν χαίρειν.
\VS{19}Ἀκηκόαμεν περὶ σοῦ, ὅτι ἀνὴρ δυνατὸς ἰσχύϊ, καὶ ἐπιτήδειος εἶ τοῦ εἶναι ἡμῖν φίλος.
\VS{20}Καὶ νῦν καθεστάκαμέν σε σήμερον ἀρχιερέα τοῦ ἔθνους σου, καὶ φίλον βασιλέως καλεῖσθαι· καὶ ἀπέστειλεν αὐτῷ πορφύραν καὶ στέφανον χρυσοῦν· καὶ φρονεῖν τὰ ἡμῶν, καὶ συντηρεῖν φιλίαν πρὸς ἡμᾶς.
\VS{21}Καὶ ἐνεδύσατο Ἰωνάθαν τὴν ἁγίαν στολὴν τῷ ἑβδόμῳ μηνὶ ἔτους ἑξηκοστοῦ καὶ ἑκατοστοῦ ἐν ἑορτῇ σκηνοπηγίας, καὶ συνήγαγε δυνάμεις, καὶ κατεσκεύασεν ὅπλα πολλά.
\par }{\PP \VS{22}Καὶ ἤκουσε Δημήτριος τοὺς λόγους τούτους. καὶ ἐλυπήθη, καὶ εἶπε,
\VS{23}τί τοῦτο ἐποιήσαμεν, ὅτι προέφθακεν ἡμᾶς ὁ Ἀλέξανδρος τοῦ φιλίαν καταθέσθαι τοῖς Ἰουδαίοις εἰς στήριγμα;
\VS{24}Γράψω αὐτοῖς κᾀγὼ λόγους παρακλήσεως, καὶ ὕψους, καὶ δομάτων, ὅπως ὦσι σὺν ἐμοὶ εἰς βοήθειαν.
\VS{25}Καὶ ἀπέστειλεν αὐτοῖς κατὰ τοὺς λόγους τούτους· βασιλεὺς Δημήτριος τῷ ἔθνει τῶν Ἰουδαίων χαίρειν.
\VS{26}Ἐπεὶ συνετηρήσατε τὰς πρὸς ἡμᾶς συνθήκας, καὶ ἐνεμείνατε τῇ φιλίᾳ ἡμῶν, καὶ οὐ προσεχωρήσατε τοῖς ἐχθροῖς ἡμῶν, ἠκούσαμεν, καὶ ἐχάρημεν.
\VS{27}Καὶ νῦν ἐμμείνατε ἔτι τοῦ συντηρῆσαι πρὸς ἡμᾶς πίστιν, καὶ ἀνταποδώσομεν ὑμῖν ἀγαθὰ, ἀνθʼ ὧν ποιεῖτε μεθʼ ἡμῶν.
\VS{28}Καὶ ἀφήσομεν ὑμῖν ἀφέματα πολλὰ, καὶ δώσομεν ὑμῖν δόματα.
\par }{\PP \VS{29}Καὶ νῦν ἀπολύω ὑμᾶς, καὶ ἀφίημι πάντας τοὺς Ἰουδαίους ἀπὸ τῶν φόρων, καὶ τῆς τιμῆς τοῦ ἁλὸς, καὶ ἀπὸ τῶν στεφάνων,
\VS{30}καὶ ἀντὶ τοῦ τρίτου τῆς σπορᾶς, καὶ ἀντὶ τοῦ ἡμίσους τοῦ καρποῦ τοῦ ξυλίνου τοῦ ἐπιβάλλοντός μοι λαβεῖν ἀφίημι ἀπὸ τῆς σήμερον καὶ ἐπέκεινα τοῦ λαβεῖν ἀπὸ τῆς γῆς Ἰούδα, καὶ ἀπὸ τῶν τριῶν νομῶν τῶν προστιθεμένων αὐτῇ ἀπὸ τῆς Σαμαρείτιδος καὶ Γαλιλαίας, καὶ ἀπὸ τῆς σήμερον ἡμέρας καὶ εἰς τὸν αἰῶνα χρόνον.
\par }{\PP \VS{31}Καὶ Ἱερουσαλὴμ ἤτω ἁγία καὶ ἀφειμένη, καὶ τὰ ὅρια αὐτῆς, αἱ δεκάται καὶ τὰ τέλη.
\VS{32}Ἀφίημι καὶ τὴν ἐξουσίαν τῆς ἄκρας τῆς ἐν Ἱερουσαλὴμ, καὶ δίδωμι τῷ ἀρχιερεῖ, ὅπως ἂν καταστήσῃ ἐν αὐτῇ ἄνδρας οὓς ἂν ἐκλέξηται αὐτὸς τοῦ φυλάσσειν αὐτήν.
\par }{\PP \VS{33}Καὶ πᾶσαν φυχὴν Ἰουδαίων τὴν αἰχμαλωτισθεῖσαν ἀπὸ γῆς Ἰούδα εἰς πᾶσαν βασιλείαν μου ἀφίημι ἐλευθέραν δωρέαν· καὶ πάντες ἀφιέτωσαν τοὺς φόρους καὶ τῶν κτηνῶν αὐτῶν.
\VS{34}Καὶ πᾶσαι αἱ ἑορταὶ καὶ τὰ σάββατα καὶ νουμηνίαι, καὶ ἡμέραι ἀποδεδειγμέναι, καὶ τρεῖς ἡμέραι πρὸ ἑορτῆς καὶ τρεῖς ἡμέραι μετὰ ἑορτὴν, ἔστωσαν πᾶσαι αἱ ἡμέραι ἀτελείας καὶ ἀφέσεως πᾶσι τοῖς Ἰουδαίοις τοῖς οὖσιν ἐν τῇ βασιλείᾳ μου.
\VS{35}Καὶ οὐχ ἕξει ἐξουσίαν οὐδεὶς πράσσειν καὶ παρενοχλεῖν τινα αὐτῶν περὶ παντὸς πράγματος.
\par }{\PP \VS{36}Καὶ προγραφήτωσαν τῶν Ἰουδαίων εἰς τὰς δυνάμεις τοῦ βασιλέως εἰς τριάκοντα χιλιάδας ἀνδρῶν, καὶ δοθήσεται αὐτοῖς ξένια ὡς καθήκει πάσαις ταῖς δυνάμεσι τοῦ βασιλέως.
\VS{37}Καὶ κατασταθήσεται ἐξ αὐτῶν ἐν τοῖς ὀχυρώμασι τοῦ βασιλέως τοῖς μεγάλοις, καὶ ἐκ τούτων κατασταθήσεται ἐπὶ χρειῶν τῆς βασιλείας τῶν οὐσῶν εἰς πίστιν· καὶ οἱ ἐπʼ αὐτῶν καὶ οἱ ἄρχοντες ἔστωσαν ἐξ αὐτῶν· καὶ πορευέσθωσαν τοῖς νόμοις αὐτῶν, καθὰ καὶ προσέταξεν ὁ βασιλεὺς ἐν γῇ Ἰούδα.
\par }{\PP \VS{38}Καὶ τοὺς τρεῖς νομοὺς τοὺς προστεθέντας τῇ Ἰουδαίᾳ ἀπὸ τῆς χώρας Σαμαρείας, προστεθήτω τῇ Ἰουδαίᾳ πρὸς τὸ λογισθῆναι τοῦ γενέσθαι ὑφʼ ἕνα, τοῦ μὴ ὑπακοῦσαι ἄλλης ἐξουσίας ἀλλʼ ἢ τοῦ ἀρχιερέως.
\par }{\PP \VS{39}Πτολεμαΐδα καὶ τὴν προσκυροῦσαν αὐτῇ δέδωκα δόμα τοῖς ἁγίοις τοῖς ἐν Ἰερουσαλὴμ εἰς τὴν προσήκουσαν δαπάνην τοῖς ἁγίοις.
\VS{40}Κᾀγὼ δίδωμι κατʼ ἐνιαυτὸν δεκαπέντε χιλιάδας σίκλων ἀργυρίου ἀπὸ τῶν λόγων τοῦ βασιλέως, ἀπὸ τῶν τόπων τῶν ἀνηκόντων.
\VS{41}Καὶ πᾶν τὸ πλεονάζον ὃ οὐκ ἀπεδίδοσαν οἱ ἀπὸ τῶν χρειῶν, ὡς ἐν τοῖς πρῶτοις ἔτεσιν, ἀπὸ τοῦ νῦν δώσουσιν εἰς τὰ ἔργα τοῦ οἴκου.
\par }{\PP \VS{42}Καὶ ἐπὶ τούτοις, πεντακισχιλίους σίκλους ἀργυρίου, οὓς ἐλάμβανον ἀπὸ τῶν χρειῶν τοῦ ἁγίου ἀπὸ τοῦ λόγου κατʼ ἐνιαυτὸν, καὶ ταῦτα ἀφίεται διὰ τὸ ἀνήκειν αὐτὰ τοῖς ἱερεῦσι τοῖς λειτουργοῦσι.
\VS{43}Καὶ ὅσοι ἐὰν φύγωσιν εἰς τὸ ἱερὸν τὸ ἐν Ἱεροσολύμοις καὶ ἐν πᾶσι τοῖς ὁρίοις αὐτοῦ, ὀφείλοντες βασιλικὰ καὶ πᾶν πρᾶγμα, ἀπολελύσθωσαν, καὶ πάντα ὅσα ἐστὶν αὐτοῖς ἐν τῇ βασιλείᾳ μου.
\VS{44}Καὶ τοῦ οἰκοδομηθῆναι καὶ τοῦ ἐπικαινισθῆναι τὰ ἔργα τῶν ἁγίων, καὶ ἡ δαπάνη δοθήσεται ἐκ τοῦ λόγου τοῦ βασιλέως.
\VS{45}Καὶ τοῦ οἰκοδομηθῆναι τὰ τείχη Ἱερουσαλὴμ καὶ ὀχυρῶσαι κυκλόθεν, καὶ ἡ δαπάνη δοθήσεται ἐκ τοῦ λόγου τοῦ βασιλέως, καὶ τοῦ οἰκοδομηθῆναι τὰ τείχη τὰ ἐν τῇ Ἰουδαίᾳ.
\par }{\PP \VS{46}Ὡς δὲ ἤκουσεν Ἰωνάθαν καὶ ὁ λαὸς τοὺς λόγους τούτους, οὐκ ἐπίστευσαν αὐτοῖς οὐδὲ ἐπεδέξαντο, ὅτι ἐπεμνήσθησαν τῆς κακίας τῆς μεγάλης ἧς ἐποίησεν ἐν Ἰσραὴλ, καὶ ἔθλιψεν αὐτοὺς σφόδρα.
\VS{47}Καὶ εὐδόκησαν ἐν Ἀλεξάνδρῳ, ὅτι αὐτὸς ἐγένετο αὐτοῖς ἀρχηγὸς λόγων εἰρηνικῶν, καὶ συνεμάχουν αὐτῷ πάσας τὰς ἡμέρας.
\par }{\PP \VS{48}Καὶ συνήγαγεν Ἀλέξανδρος ὁ βασιλεὺς δυνάμεις μεγάλας, καὶ παρενέβαλεν ἐξεναντίας Δημητρίου.
\VS{49}Καὶ συνῆψαν πόλεμον οἱ δύο βασιλεῖς, καὶ ἔφυγεν ἡ παρεμβολὴ Δημητρίου, καὶ ἐδίωξεν αὐτὸν ὁ Ἀλέξανδρος, καὶ ἴσχυσεν ἐπʼ αὐτούς.
\VS{50}Καὶ ἐστερέωσε τὸν πόλεμον σφόδρα ἕως ἔδυ ὁ ἥλιος, καὶ ἔπεσεν ὁ Δημήτριος ἐν τῇ ἡμέρᾳ ἐκείνῃ.
\par }{\PP \VS{51}Καὶ ἀπέστειλεν Ἀλέξανδρος πρὸς Πτολεμαῖον βασιλέα Αἰγύπτου πρέσβεις κατὰ τοὺς λόγους τούτους, λέγων,
\VS{52}ἐπεὶ ἀνέστρεψα εἰς γῆν βασιλείας μου, καὶ ἐκάθισα ἐπὶ θρόνου πατέρων μου, καὶ ἐκράτησα τῆς ἀρχῆς, καὶ συνέτριψα τὸν Δημήτριον, καὶ ἐπεκράτησα τῆς χώρας ἡμῶν·
\VS{53}καὶ συνῆψα πρὸς αὐτὸν μάχην, καὶ συνετρίβη αὐτὸς καὶ ἡ παρεμβολὴ αὐτοῦ ὑφʼ ἡμῶν, καὶ ἐκαθίσαμεν ἐπὶ θρόνου βασιλείας αὐτοῦ·
\VS{54}καὶ νῦν στήσωμεν πρὸς ἑαυτοὺς φιλίαν, καὶ νῦν δός μοι τὴν θυγατέρα σου εἰς γυναῖκα, καὶ ἐπιγαμβρεύσω σοι, καὶ δώσω σοι δόματα, καὶ αὐτῇ. ἄξιά σου.
\par }{\PP \VS{55}Καὶ ἀπεκρίθη Πτολεμαῖος ὁ βασιλεὺς, λέγων, ἀγαθὴ ἡμέρα ἐν ᾗ ἀνέστρεψας εἰς γῆν πατέρων σου, καὶ ἐκάθισας ἐπὶ θρόνου βασιλείας αὐτῶν.
\VS{56}Καὶ νῦν ποιήσω σοι ἃ ἔγραψας, ἀλλʼ ἀπάντησον εἰς Πτολεμαΐδα, ὅπως ἴδωμεν ἀλλήλους, καὶ ἐπιγαμβρεύσω σοι καθὼς εἴρηκας.
\par }{\PP \VS{57}Καὶ ἐξῆλθε Πτολεμαῖος ἐξ Αἰγύπτου αὐτὸς καὶ Κλεοπάτρα ἡ θυγάτηρ αὐτοῦ, καὶ εἰσῆλθον εἰς Πτολεμαΐδα ἔτους δευτέρου καὶ ἑξηκοστοῦ καὶ ἑκατοστοῦ.
\VS{58}Καὶ ἀπήντησεν αὐτῷ Ἀλεξανδρος ὁ βασιλεὺς, καὶ ἐξέδοτο αὐτῷ Κλεοπάτραν τὴν θυγατέρα αὐτοῦ, καὶ ἐποίησε τὸν γάμον αὐτῆς ἐν Πτολεμαΐδι, καθὼς οἱ βασιλεῖς, ἐν δόξῃ μεγάλῃ.
\par }{\PP \VS{59}Καὶ ἔγραψεν Ἀλέξανδρος ὁ βασιλεὺς τῷ Ἰωνάθαν ἐλθεῖν εἰς συνάντησιν αὐτῷ.
\VS{60}Καὶ ἐπορεύθη μετὰ δόξης εἰς Πτολεμαΐδα, καὶ ἀπήντησε τοῖς δυσὶ βασιλεῦσι· καὶ ἔδωκεν αὐτοῖς ἀργύριον καὶ χρυσίον, καὶ τοῖς φίλοις αὐτῶν, καὶ δόματα πολλὰ, καὶ εὗρε χάριν ἐναντίον αὐτῶν.
\par }{\PP \VS{61}Καὶ ἐπισυνήχθησαν πρὸς αὐτὸν ἄνδρες λοιμοὶ ἐξ Ἰσραὴλ, ἄνδρες παράνομοι, ἐντυχεῖν κατʼ αὐτοῦ, καὶ οὐ προσέσχεν αὐτοῖς ὁ βασιλεύς.
\VS{62}Καὶ προσέταξεν ὁ βασιλεὺς, καὶ ἐξέδυσαν Ἰωνάθαν τὰ ἱμάτια αὐτοῦ, καὶ ἐνέδυσαν αὐτὸν πορφύραν, καὶ ἐποίησαν οὕτως.
\VS{63}Καὶ ἐκάθισεν αὐτὸν ὁ βασιλεὺς μετʼ αὐτοῦ, καὶ εἶπε τοῖς ἄρχουσιν αὐτοῦ, ἐξέλθετε μετʼ αὐτοῦ εἰς μέσον τῆς πόλεως, καὶ κηρύξατε τοῦ μηδένα ἐντυγχάνειν κατʼ αὐτοῦ περὶ μηδενὸς πράγματος, καὶ μηδεὶς αὐτῷ παρενοχλείτω περὶ παντὸς λόγου.
\par }{\PP \VS{64}Καὶ ἐγένετο ὡς ἴδον οἱ ἐντυγχάνοντες τὴν δόξαν αὐτοῦ καθὼς ἐκήρυξαν, καὶ περιβεβλημένον αὐτὸν πορφύραν, καὶ ἔφυγον πάντες.
\VS{65}Καὶ ἐδόξασεν αὐτὸν ὁ βασιλεύς, καὶ ἔγραψεν αὐτὸν τῶν πρώτων φίλων, καὶ ἔθετο αὐτὸν στρατηγὸν καὶ μεριδάρχην.
\VS{66}Καὶ ἐπέστρεψεν Ἰωνάθαν εἰς Ἱερουσαλὴμ μετʼ εἰρήνης καὶ εὐφροσύνης.
\par }{\PP \VS{67}Καί ἐν ἔτει πέμπτῳ καὶ ἑξηκοστῷ καὶ ἑκατοστῷ ἦλθε Δημήτριος υἱὸς Δημητρίου ἐκ Κρήτης εἰς τὴν γῆν τῶν πατέρων αὐτοῦ.
\VS{68}Καὶ ἤκουσεν Ἀλέξανδρος ὁ βασιλεὺς, καὶ ἐλυπήθη σφόδρα, καὶ ἀπέστρεψεν εἰς Ἀντιόχειαν.
\par }{\PP \VS{69}Καὶ κατέστησε Δημήτριος Ἀπολλώνιον τὸν ὄντα ἐπὶ κοίλης Συρίας, καὶ συνήγαγε δύναμιν μεγάλην, καὶ παρενέβαλεν ἐν Ἰαμνείᾳ·
\VS{70}καὶ ἀπέστειλε πρὸς Ἰωνάθαν τὸν ἀρχιερέα, λέγων, σὺ μονώτατος ἐπαίρῃ ἐφʼ ἡμᾶς, ἐγὼ δὲ ἐγενήθην εἰς καταγέλωτα καὶ ὀνειδισμὸν διὰ σέ· καὶ διατί σὺ ἐξουσιάζῃ ἐφʼ ἡμᾶς ἐν τοῖς ὄρεσι;
\par }{\PP \VS{71}Νῦν οὖν εἰ πέποιθας ἐπὶ ταῖς δυνάμεσί σου, κατάβηθι πρὸς ἡμᾶς εἰς τὸ πεδίον, καὶ συγκριθῶμεν ἑαυτοῖς ἐκεῖ, ὅτι μετʼ ἐμοῦ ἐστι δύναμις τῶν πόλεων.
\VS{72}Ἐρώτησον καὶ μάθε τίς εἰμι καὶ οἱ λοιποὶ οἱ βοηθοῦντες ἡμῖν, καὶ λέγουσιν, οὐκ ἔστιν ὑμῖν στάσις ποδὸς κατὰ πρόσωπον ἡμῶν; ὅτι δὶς ἐτροπώθησαν οἱ πατέρες σου ἐν τῇ γῇ αὐτῶν.
\VS{73}Καὶ νῦν οὐ δυνήσῃ ὑποστῆναι τὴν ἵππον καὶ δύναμιν τοιαύτην ἐν τῷ πεδίῳ, ὅπου οὐκ ἔστι λίθος οὐδὲ κόχλαξ οὐδὲ τόπος τοῦ φυγεῖν.
\par }{\PP \VS{74}Ὡς δὲ ἤκουσεν Ἰωνάθαν τῶν λόγων Ἀπολλωνίου, ἐκινήθη τῇ διανοίᾳ, καὶ ἐπέλεξε δέκα χιλιάδας ἀνδρῶν, καὶ ἐξῆλθεν ἐξ Ἱερουσαλὴμ, καὶ συνήντησεν αὐτῷ Σίμων ὁ ἀδελφὸς αὐτοῦ ἐπὶ βοήθειαν αὐτοῦ.
\VS{75}Καὶ παρενέβαλεν ἐπὶ Ἰόππην, καὶ ἀπέκλεισαν αὐτὸν ἐκ τῆς πόλεως, ὅτι φρουρὰ Ἀπολλωνίου ἐν Ἰόππῃ, καὶ ἐπολέμησαν αὐτήν.
\par }{\PP \VS{76}Καὶ φοβηθέντες ἤνοιξαν οἱ ἐκ τῆς πόλεως, καὶ ἐκυρίευσεν Ἰωνάθαν Ἰόππης.
\VS{77}Καὶ ἤκουσεν Ἀπολλώνιος, καὶ παρενέβαλε τρισχιλίαν ἵππον, καὶ δύναμιν πολλήν· καὶ ἐπορεύθη εἰς Ἄζωτον ὡς διοδεύων, καὶ ἅμα προῆγεν εἰς τὸ πεδίον, διὰ τὸ ἔχειν αὐτὸν πλῆθος ἵππου καὶ πεποιθέναι ἐπʼ αὐτῇ.
\par }{\PP \VS{78}Καὶ κατεδίωξεν Ἰωνάθαν ὀπίσω αὐτοῦ εἰς Ἄζωτον, καὶ συνῆψαν αἱ παρεμβολαὶ εἰς πόλεμον.
\VS{79}Καὶ ἀπέλιπεν Ἀπολλώνιος χιλίαν ἵππον ἐν κρυπτῷ κατόπισθεν αὐτῶν.
\VS{80}Καὶ ἔγνω Ἰωνάθαν ὅτι ἐστὶν ἔνεδρον κατόπισθεν αὐτοῦ, καὶ ἐκύκλωσαν αὐτοῦ τὴν παρεμβολὴν, καὶ ἐξετίναξαν τὰς σχίζας εἰς τὸν λαὸν ἐκ πρωΐθεν ἕως ἑσπέρας.
\par }{\PP \VS{81}Ὁ δὲ λαὸς εἱστήκει, καθὼς ἐπέταξεν Ἰωνάθαν, καὶ ἐκοπίασαν οἱ ἵπποι αὐτῶν.
\VS{82}Καὶ εἵλκυσε Σίμων τὴν δύναμιν αὐτοῦ, καὶ συνῆψε πρὸς τὴν φάραγγα· ἡ γὰρ ἵππος ἐξελύθη· καὶ συνετρίβησαν ὑπʼ αὐτοῦ, καὶ ἔφυγον.
\VS{83}Καὶ ἡ ἵππος ἐσκορπίσθη ἐν τῷ πεδίῳ, καὶ ἔφυγον εἰς Ἄζωτον, καὶ εἰσῆλθον εἰς Βηθδαγὼν τὸ εἰδωλεῖον αὐτῶν, τοῦ σωθῆναι
\par }{\PP \VS{84}Καὶ ἐνεπύρισεν Ἰωνάθαν τὴν Ἄζωτον καὶ τὰς πόλεις τὰς κύκλῳ αὐτῆς, καὶ ἔλαβε τὰ σκῦλα αὐτῶν, καὶ τὸ ἱερὸν Δαγὼν καὶ τοὺς συμφυγόντας εἰς αὐτὸ ἐνεπύρισε πυρί.
\VS{85}Καὶ ἐγένοντο οἱ πεπτωκότες μαχαίρᾳ σὺν τοῖς ἐμπυρισθεῖσιν εἰς ἄνδρας ὀκτακισχιλίους.
\VS{86}Καὶ ἀπῇρεν ἐκεῖθεν Ἰωνάθαν, καὶ παρενέβαλεν ἐπὶ Ἀσκάλωνα, καὶ ἐξῆλθον οἱ ἐκ τῆς πόλεως εἰς συνάντησιν αὐτῷ ἐν δόξῃ μεγάλῃ.
\VS{87}Καὶ ἐπέστρεψεν Ἰωνάθαν εἰς Ἱερουσαλὴμ σὺν τοῖς παρʼ αὐτοῦ, ἔχοντες σκῦλα πολλά.
\par }{\PP \VS{88}Καὶ ἐγένετο ὡς ἤκουσεν Ἀλέξανδρος ὁ βασιλεὺς τοὺς λόγους τούτους, καὶ προσέθετο δοξάσαι τὸν Ἰωνάθαν.
\VS{89}Καὶ ἀπέστειλεν αὐτῷ πόρπην χρυσῆν, ὡς ἔθος ἐστὶ δίδοσθαι τοῖς συγγενέσι τῶν βασιλέων· καὶ ἔδωκεν αὐτῷ τὴν Ἀκκαρὼν καὶ πάντα τὰ ὅρια αὐτῆς εἰς κληροδοσίαν.

\par }\Chap{11}{\PP \VerseOne{1}Καὶ ὁ βασιλεὺς Αἰγύπτου ἤθροισε δυνάμεις πολλὰς, ὡς τὴν ἄμμον τὴν περὶ τὸ χεῖλος τῆς θαλάσσης, καὶ πλοῖα πολλά· καὶ ἐζήτησε κατακρατῆσαι τῆς βασιλείας Ἀλεξάνδρου δόλῳ, καὶ προσθεῖναι αὐτὴν τῇ βασιλείᾳ αὑτοῦ.
\VS{2}Καὶ ἐξῆλθεν εἰς Συρίαν λόγοις εἰρηνικοῖς, καὶ ἤνοιγον αὐτῷ οἱ ἀπὸ τῶν πόλεων, καὶ συνήντων αὐτῷ, ὅτι ἐντολὴ ἦν Ἀλεξάνδρου τοῦ βασιλέως συναντᾷν αὐτῷ, διὰ τὸ πενθερὸν αὐτοῦ εἶναι,
\par }{\PP \VS{3}Ὡς δὲ εἰσεπορεύετο εἰς τὰς πόλεις Πτολεμαῖος, ἀπέτασσε τὰς δυνάμεις φρουρὰν ἐν ἑκάστῃ πόλει.
\VS{4}Ὡς δὲ ἤγγισεν Ἀζώτου, ἔδειξαν αὐτῷ τὸ ἱερὸν Δαγὼν ἐμπεπυρισμένον, καὶ Ἄζωτον, καὶ τὰ περιπόλια αὐτῆς καθῃρημένα, καὶ τὰ σώματα ἐῤῥιμμένα, καὶ τοὺς ἐμπεπυρισμένους οὓς ἐνεπύρισεν ἐν τῷ πολέμῳ· ἐποίησαν γὰρ θημωνίας αὐτῶν ἐν τῇ ὁδῷ αὐτοῦ.
\VS{5}Καὶ διηγήσαντο τῷ βασιλεῖ ἃ ἐποίησεν Ἰωνάθαν, εἰς τὸ ψογῆσαι αὐτόν· καὶ ἐσίγησεν ὁ βασιλεύς.
\par }{\PP \VS{6}Καὶ συνήντησεν Ἰωνάθαν τῷ βασιλεῖ εἰς Ἰόππην μετὰ δόξης, καὶ ἠσπάσαντο ἀλλήλους, καὶ ἐκοιμήθησαν ἐκεῖ.
\VS{7}Καὶ ἐπορεύθη Ἰωνάθαν μετὰ τοῦ βασιλέως ἕως τοῦ ποταμοῦ τοῦ καλουμένου Ἐλευθέρου, καὶ ἐπέστρεψεν εἰς Ἱερουσαλήμ.
\par }{\PP \VS{8}Ὁ δὲ βασιλεὺς Πτολεμαῖος ἐκυρίευσε τῶν πόλεων τῆς παραλίας ἕως Σελευκίας τῆς παραθαλασσίας, καὶ διελογίζετο περὶ Ἀλεξάνδρου λογισμοὺς πονηρούς.
\VS{9}Καὶ ἀπέστειλε πρέσβεις πρὸς Δημήτριον τὸν βασιλέα, λέγων, δεῦρο συνθώμεθα πρὸς ἑαυτοὺς διαθήκην, καὶ δώσω σοι τὴν θυγατέρα μου ἣν ἔχει Ἀλέξανδρος, καὶ βασιλεύσεις τῆς βασιλείας τοῦ πατρός σου.
\VS{10}Μεταμεμέλημαι γὰρ δοὺς αὐτῷ τὴν θυγατέρα μου, ἐζήτησε γὰρ ἀποκτεῖναί με.
\VS{11}Καὶ ἐψόγησεν αὐτὸν χάριν τοῦ ἐπιθυμῆσαι αὐτὸν τῆς βασιλείας αὐτοῦ·
\par }{\PP \VS{12}Καὶ ἀφελόμενος αὐτοῦ τὴν θυγατέρα, ἔδωκεν αὐτὴν τῷ Δημητρίῳ, καὶ ἠλλοιώθη τοῦ Ἀλεξάνδρου, καὶ ἐφάνη ἡ ἔχθρα αὐτῶν.
\VS{13}Καὶ εἰσῆλθε Πτολεμαῖος εἰς Ἀντιόχειαν, καὶ περιέθετο δύο διαδήματα περὶ τὴν κεφαλὴν αὐτοῦ, τὸ τῆς Ἀσίας καὶ Αἰγύπτου.
\par }{\PP \VS{14}Ἀλέξανδρος δὲ ὁ βασιλεὺς ἦν ἐν Κιλικίᾳ κατὰ τοὺς καιροὺς ἐκείνους, ὅτι ἀπεστάτουν οἱ ἀπὸ τῶν τόπων ἐκείνων.
\VS{15}Καὶ ἤκουσεν Ἀλέξανδρος, καὶ ἦλθεν ἐπʼ αὐτὸν πολέμῳ· καὶ ἐξήγαγε Πτολεμαῖος τὴν δύναμιν, καὶ ἀπήντησεν αὐτῷ ἐν χειρὶ ἰσχυρᾷ, καὶ ἐτροπώσατο αὐτόν.
\par }{\PP \VS{16}Καὶ ἔφυγεν Ἀλέξανδρος εἰς τὴν Ἀραβίαν, τοῦ σκεπασθῆναι αὐτὸν ἐκεῖ· ὁ δὲ βασιλεὺς Πτολεμαῖος ὑψώθη.
\VS{17}Καὶ ἀφειλε Ζαβδιὴλ ὁ Ἄραψ τὴν κεφαλὴν Ἀλεξάνδρου, καὶ ἀπέστειλε τῷ Πτολεμαίῳ.
\par }{\PP \VS{18}Καὶ ὁ βασιλεὺς Πτολεμαῖος ἀπέθανεν ἐν τῇ ἡμέρᾳ τῇ τρίτῃ, καὶ οἱ ὄντες ἐν τοῖς ὀχυρώμασιν ἀπώλοντο ὑπὸ τῶν ἐν τοῖς ὀχυρώμασι.
\VS{19}Καὶ ἐβασίλευσε Δημήτριος ἔτους ἑβδόμου καὶ ἑξηκοστοῦ καὶ ἑκατοστοῦ.
\par }{\PP \VS{20}Ἐν ταῖς ἡμέραις ἐκείναις συνήγαγεν Ἰωνάθαν τοὺς ἐκ τῆς Ἰουδαίας, τοῦ ἐκπολεμῆσαι τὴν ἄκραν τὴν ἐν Ἱερουσαλήμ, καὶ ἐποίησεν ἐπʼ αὐτὴν μηχανὰς πολλάς.
\VS{21}Καὶ ἐπορεύθησάν τινες μισοῦντες τὸ ἔθνος αὐτῶν, ἄνδρες παράνομοι, πρὸς τὸν βασιλέα, καὶ ἀπήγγειλαν αὐτῷ ὅτι Ἰωνάθαν περικάθηται τὴν ἄκραν.
\VS{22}Καὶ ἀκούσας ὠργίσθη· ὡς δὲ ἤκουσεν, εὐθέως ἀναζεύξας ἦλθεν εἰς Πτολεμαΐδα, καὶ ἔγραψεν Ἰωνάθαν, τοῦ μὴ περικαθῆσθαι τῇ ἄκρᾳ, καὶ τοῦ ἀπαντῆσαι αὐτὸν αὐτῷ συνμίσγειν εἰς Πτολεμαΐδα τὴν ταχίστην.
\par }{\PP \VS{23}Ὡς δὲ ἤκουσεν Ἰωνάθαν, ἐκέλευσε περικαθῆσθαι, καὶ ἐπέλεξε τῶν πρεσβυτέρων Ἰσραὴλ καὶ τῶν ἱερέων, καὶ ἔδωκεν ἑαυτὸν τῷ κινδύνῳ.
\VS{24}Καὶ λαβὼν ἀργύριον, καὶ χρυσίον, καὶ ἱματισμὸν, καὶ ἕτερα ξένια πλείονα, ἐπορεύθη πρὸς τὸν βασιλέα εἰς Πτολεμαΐδα, καὶ εὗρε χάριν ἑνώτιον αὐτοῦ.
\par }{\PP \VS{25}Καὶ ἐνετύγχανον κατʼ αὐτοῦ τινὲς ἄνομοι τῆς ἐκ τοῦ ἔθνους.
\VS{26}Καὶ ἐποίησεν αὐτῷ ὁ βασιλεὺς καθὼς ἐποίησαν αὐτῷ οἱ πρὸ αὐτοῦ, καὶ ὕψωσεν αὐτὸν ἐναντίον πάντων τῶν φίλων αὐτοῦ.
\VS{27}Καὶ ἔστησεν αὐτῷ τὴν ἀρχιερωσύνην, καὶ ὅσα ἄλλα εἶχε τίμια τοπρότερον, καὶ ἐποίησεν αὐτὸν τῶν πρώτων φίλων ἡγεῖσθαι.
\par }{\PP \VS{28}Καὶ ἠξίωσεν Ἰωνάθαν τὸν βασιλέα ποιῆσαι τὴν Ἰουδαίαν ἀφορολόγητον, καὶ τὰς τρεῖς τοπαρχίας, καὶ τὴν Σαμαρεῖτιν, καὶ ἐπηγγείλατο αὐτῷ τάλαντα τριακόσια.
\VS{29}Καὶ εὐδόκησεν ὁ βασιλεύς, καὶ ἔγραψε τῷ Ἰωνάθαν ἐπιστολὰς περὶ πάντων τούτων ἐχούσας τὸν τρόπον τοῦτον·
\par }{\PP \VS{30}Βασιλεὺς Δημήτριος Ἰωνάθαν τῷ ἀδελφῷ χαίρειν, καὶ ἔθνει Ἰουδαίων.
\VS{31}Τὸ ἀντίγραφον τῆς ἐπιστολῆς ἧς ἐγράψαμεν Λασθένει τῷ συγγενεῖ ἡμῶν περὶ ὑμῶν, γεγράφαμεν καὶ πρὸς ὑμᾶς. ὅπως εἰδῆτε.
\par }{\PP \VS{32}Βασιλεὺς Δημήτριος Λασθένει τῷ πατρὶ χαίρειν.
\VS{33}Τῷ ἔθνει τῶν Ἰουδαίων φίλοις ἡμῶν καὶ συντηροῦσι τὰ πρὸς ἡμᾶς δίκαια ἐκρίναμεν ἀγαθοποιῆσαι, χάριν τῆς ἑξ αὐτῶν εὐνοίας πρὸς ἡμᾶς.
\VS{34}Ἑστάκαμεν οὖν αὐτοῖς τὰ τε ὅρια τῆς Ἰουδαίας, καὶ τοὺς τρεῖς νομοὺς, Ἀφαίρεμα, καὶ Λύδδαν, καὶ Ῥαμαθὲμ, αἵτινες προσετέθησαν τῇ Ἰουδαίᾳ ἀπὸ τῆς Σαμαρείτιδος, καὶ πάντα τὰ συγκυροῦντα αὐτοῖς πᾶσι τοῖς θυσιάζουσιν εἰς Ἱεροσόλυμα, ἀντὶ τῶν βασιλικῶν ὧν ἐλάμβανεν ὁ βασιλεὺς παρʼ αὐτῶν τοπρότερον κατʼ ἐνιαυτὸν ἀπὸ τῶν γεννημάτων τῆς γῆς, καὶ ἀπὸ τῶν ἀκροδρύων.
\par }{\PP \VS{35}Καὶ τὰ ἄλλα τὰ ἀνήκοντα ἡμῖν ἀπὸ τοῦ νῦν τῶν δεκατῶν, καὶ τῶν τελῶν τῶν ἀνηκόντων ἡμῖν, καὶ τὰς τοῦ ἁλὸς λίμνας, καὶ τοὺς ἀνήκοντας ἡμῖν στεφάνους, πάντα ἐπαρκῶς παρίεμεν αὐτοῖς.
\VS{36}Καὶ οὐκ ἀθετηθήσεται οὐδὲ ἓν τούτων ἀπὸ τοῦ νῦν καὶ εἰς τὸν ἅπαντα χρόνον.
\par }{\PP \VS{37}Νῦν οὖν ἐπιμέλεσθε τοῦ ποιῆσαι τούτων ἀντίγραφον· καὶ δοθήτω Ἰωνάθαν, καὶ τεθήτω ἐν τῷ ὄρει τῷ ἁγίῳ ἐν τόπῳ ἐπισήμῳ.
\par }{\PP \VS{38}Καὶ εἶδε Δημήτριος ὁ βασιλεὺς ὅτι ἡσύχασεν ἡ γῆ ἐνώπιον αὐτοῦ, καὶ οὐδὲν αὐτῷ ἀνθειστήκει, καὶ ἀπέλυσε πάσας τὰς δύναμεις αὐτοῦ ἕκαστον εἰς τὸν ἴδιον τόπον, πλὴν τῶν ξένων δυνάμεων ὧν ἐξενολόγησεν ἀπὸ τῶν νήσων τῶν ἐθνῶν· καὶ ἤχθραναν αὐτῷ πᾶσαι αἱ δυνάμεις τῶν πατέρων αὐτοῦ.
\par }{\PP \VS{39}Τρύφων δὲ ἦν τῶν παρὰ Ἀλεξάνδρου τοπρότερον, καὶ εἶδεν ὅτι πᾶσαι αἱ δυνάμεις καταγογγύζουσι τοῦ Δημητρίου, καὶ ἐπορεύθη πρὸς Εἰμαλκουαὶ τὸν Ἄραβα, ὃς ἔτρεφε τὸν Ἀντίοχον τὸ παιδάριον τὸ τοῦ Ἀλεξάνδρου·
\VS{40}καὶ προσήδρευεν αὐτῷ, ὅπως παραδοῖ αὐτὸν αὐτῷ, ὅπως βασιλεύσῃ ἀντὶ τοῦ πατρὸς αὐτοῦ· καὶ ἀπήγγειλεν αὐτῷ ὅσα συνετέλεσε Δημήτριος, καὶ τὴν ἔχθραν ἣν ἐχθραίνουσιν αὐτῷ αἱ δυνάμεις αὐτοῦ· καὶ ἔμεινεν ἐκεῖ ἡμέρας πολλάς.
\par }{\PP \VS{41}Καὶ ἀπέστειλεν Ἰωνάθαν πρὸς Δημήτριον τὸν βασιλέα, ἵνα ἐκβάλῃ τοὺς ἐκ τῆς ἄκρας ἐξ Ἱερουσαλὴμ, καὶ τοὺς ἐν τοῖς ὀχυρώμασιν, ἦσαν γὰρ πολεμοῦντες τὸν Ἰσραήλ.
\VS{42}Καὶ ἀπέστειλε Δημήτριος πρὸς Ἰωνάθαν, λέγων, οὐ μόνον ταῦτα ποιήσω σοι καὶ τῷ ἔθνει σου, ἀλλὰ δόξῃ δοξάσω σε καὶ τὸ ἔθνος σου, ἐὰν εὐκαιρίας τύχω.
\VS{43}Νῦν οὖν ὀρθῶς ποιήσεις ἀποστείλας μοι ἄνδρας οἳ συμμαχήσουσιν, ὅτι ἀπέστησαν πᾶσαι αἱ δυνάμεις μου.
\par }{\PP \VS{44}Καὶ ἀπέστειλεν Ἰωνάθαν ἄνδρας τρισχιλίους δυνατοὺς ἰσχύϊ αὐτῷ εἰς Ἀντιόχειαν, καὶ ἤλθοσαν πρὸς τὸν βασιλέα, καὶ εὐφράνθη ὁ βασιλεὺς ἐπὶ τῇ ἐφόδῳ αὐτῶν.
\VS{45}Καὶ ἐπισυνήχθησαν οἱ ἐκ τῆς πόλεως εἰς μέσον τῆς πόλεως εἰς ἀνδρῶν δώδεκα μυριάδας, καὶ ἠβούλοντο ἀνελεῖν τὸν βασιλέα.
\VS{46}Καὶ ἔφυγεν ὁ βασιλεὺς εἰς τὴν αὐλὴν, καὶ κατελάβοντο οἱ ἐκ τῆς πόλεως τὰς διόδους τῆς πόλεως, καὶ ἤρξαντο πολεμεῖν.
\par }{\PP \VS{47}Καὶ ἐκάλεσεν ὁ βασιλεὺς τοὺς Ἰουδαίους ἐπὶ βοήθειαν, καὶ ἐπισυνήχθησαν πρὸς αὐτὸν πάντες ἅμα· καὶ διεσπάρησαν ἐν τῇ πόλει πάντες ἅμα· καὶ ἀπέκτειναν ἐν τῇ πόλει τῇ ἡμέρᾳ ἐκείνῃ εἰς μυριάδας δέκα.
\VS{48}Καὶ ἐνεπύρισαν τὴν πόλιν, καὶ ἐλάβοσαν σκῦλα πολλὰ ἐν ἐκείνῃ τῇ ἡμέρᾳ, καὶ ἔσωσαν τὸν βασιλέα.
\par }{\PP \VS{49}Καὶ ἴδον οἱ ἀπὸ τῆς πόλεως ὅτι κατεκράτησαν οἱ Ἰουδαῖοι τῆς πόλεως, ὡς ἠβούλοντο, καὶ ἠσθένησαν ταῖς διανοίαις αὐτῶν, καὶ ἐκέκραξαν πρὸς τὸν βασιλέα μετὰ δεήσεως, λέγοντες,
\VS{50}δὸς ἡμῖν δεξιὰς, καὶ παυσάσθωσαν οἱ Ἰουδαῖοι πολεμοῦντες ἡμᾶς καὶ τὴν πόλιν.
\VS{51}Καὶ ἔῤῥιψαν τὰ ὅπλα, καὶ ἐποίησαν εἰρήνην· καὶ ἐδοξάσθησαν οἱ Ἰουδαῖοι ἐναντίον τοῦ βασιλέως, καὶ ἐνώπιον πάντων τῶν ἐν τῇ βασιλείᾳ αὐτοῦ, καὶ ἐπέστρεψαν εἰς Ἱερουσαλὴμ ἔχοντες σκῦλα πολλά.
\par }{\PP \VS{52}Καὶ ἐκάθισε Δημήτριος ὁ βασιλεὺς ἐπὶ θρόνου τῆς βασιλείας αὐτοῦ, καὶ ἡσύχασεν ἡ γῆ ἐνώπιον αὐτοῦ.
\VS{53}Καὶ ἐψεύσατο πάντα ὅσα εἶπε, καὶ ἠλλοτριώθη τῷ Ἰωνάθαν, καὶ οὐκ ἀνταπέδωκε κατὰ τὰς εὐνοίας ἃς ἀνταπέδωκεν αὐτῷ, καὶ ἔθλιβεν αὐτὸν σφόδρα.
\par }{\PP \VS{54}Μετὰ δὲ ταῦτα ἀπέστρεψε Τρύφων καὶ Ἀντίοχος μετʼ αὐτοῦ παιδάριον νεώτερον· καὶ ἐβασίλευσε καὶ ἐπέθετο διάδημα.
\VS{55}Καὶ ἐπισυνήχθησαν πρὸς αὐτὸν πᾶσαι αἱ δυνάμεις ἃς ἀπεσκόρπισε Δημήτριος, καὶ ἐπολέμησαν πρὸς αὐτὸν, καὶ ἔφυγε καὶ ἐτροπώθη.
\VS{56}Καὶ ἔλαβε Τρύφων τὰ θηρία, καὶ κατεκράτησεν Ἀντιοχείας
\par }{\PP \VS{57}Καὶ ἔγραψεν Ἀντίοχος ὁ νεώτερος τῷ Ἰωνάθαν, λέγων. ἵστημί σοι τὴν ἀρχιερωσύνην, καὶ καθίστημί σε ἐπὶ τῶν τεσσάρων νομῶν, καὶ εἶναί σε τῶν φίλων τοῦ βασιλέως.
\VS{58}Καὶ ἀπέστειλεν αὐτῷ χρυσώματα καὶ διακονίαν, καὶ ἔδωκεν αὐτῷ ἐξουσίαν πίνειν ἐν χρυσώμασι, καὶ εἶναι ἐν πορφύρᾳ, καὶ ἔχειν πόρπην χρυσῆν.
\VS{59}Καὶ Σίμωνα τὸν ἀδελφὸν αὐτοῦ κατέστησε στρατηγὸν ἀπὸ τῆς κλίμακος Τύρου ἕως τῶν ὁρίων Αἰγύπτου.
\par }{\PP \VS{60}Καὶ ἐξῆλθεν Ἰωνάθαν, καὶ διεπορεύετο πέραν τοῦ ποταμοῦ, καὶ ἐν ταῖς πόλεσι, καὶ ἠθροίσθησαν πρὸς αὐτὸν πᾶσαι αἱ δυνάμεις Συρίας εἰς συμμαχίαν, καὶ ἦλθεν εἰς Ἀσκάλωνα, καὶ ἀπήντησαν αὐτῷ οἱ ἐκ τῆς πόλεως ἐνδόξως.
\par }{\PP \VS{61}Καὶ ἀπῆλθεν ἐκεῖθεν εἰς Γάζαν, καὶ ἀπέκλεισαν οἱ ἀπὸ Γάζης, καὶ περιεκάθισε περὶ αὐτὴν, καὶ ἐνεπύρισε τὰ περιπόλια αὐτῆς πυρὶ, καὶ ἐσκύλευσεν αὐτά.
\VS{62}Καὶ ἠξίωσαν οἱ ἀπὸ Γάζης τὸν Ἰωνάθαν, καὶ ἔδωκεν αὐτοῖς δεξιάς, καὶ ἔλαβε τοὺς υἱοὺς ἀρχόντων αὐτῶν εἰς ὅμηρα, καὶ ἐξαπέστειλεν αὐτοὺς εἰς Ἱερουσαλὴμ, καὶ διῆλθε τὴν χώραν ἕως Δαμασκοῦ.
\par }{\PP \VS{63}Καὶ ἤκουσεν Ἰωνάθαν ὅτι παρῆσαν οἱ ἄρχοντες Δημητρίου εἰς Κάδης τὴν ἐν τῇ Γαλιλαίᾳ, μετὰ δυνάμεως πολλῆς, βουλόμενοι μεταστῆσαι αὐτὸν τῆς χρείας.
\VS{64}Καὶ συνήντησεν αὐτοῖς, τὸν δὲ ἀδελφὸν αὐτοῦ Σίμωνα κατέλιπεν ἐν τῇ χώρᾳ.
\VS{65}Καὶ παρενέβαλε Σίμων ἐπὶ Βαιθσούρα, καὶ ἐπολέμει αὐτὴν ἡμέρας πολλὰς, καὶ συνέκλεισεν αὐτήν.
\VS{66}Καὶ ἠξίωσαν αὐτὸν τοῦ δεξιὰς λαβεῖν, καὶ ἔδωκεν αὐτοῖς, καὶ ἐξέβαλεν αὐτοὺς ἐκεῖθεν, καὶ κατελάβετο τὴν πόλιν, καὶ ἔθετο ἐπʼ αὐτῇ φρουράν.
\par }{\PP \VS{67}Καὶ Ἰωνάθαν καὶ ἡ παρεμβολὴ αὐτοῦ παρενέβαλον ἐπὶ τὸ ὕδωρ Γεννησάρ, καὶ ὤρθρισαν τοπρωῒ εἰς τὸ πεδίον Νασώρ.
\VS{68}Καὶ ἰδοὺ παρεμβολὴ ἀλλοφύλων ἀπήντα αὐτῷ ἐν τῷ πεδίῳ, καὶ ἐξέβαλον ἔνεδρον ἐπʼ αὐτὸν ἐν τοῖς ὄρεσιν, αὐτοὶ δὲ ἀπήντησαν ἐξεναντίας.
\par }{\PP \VS{69}Τὰ δὲ ἔνεδρα ἐξανέστησαν ἐκ τῶν τόπων αὐτῶν, καὶ συνῆψαν πόλεμον· καὶ ἔφυγον οἱ παρὰ Ἰωνάθαν πάντες,
\VS{70}οὐδὲ εἷς κατελείφθη ἀπʼ αὐτῶν, πλὴν Ματταθίας ὁ τοῦ Ἀβεσσαλώμου, καὶ Ἰούδας ὁ τοῦ Χαλφί, ἄρχοντες τῆς στρατιᾶς τῶν δυνάμεων.
\par }{\PP \VS{71}Καὶ διέῤῥηξεν Ἰωνάθαν τὰ ἱμάτια αὐτοῦ, καὶ ἐπέθηκε γῆν ἐπὶ τὴν κεφαλὴν αὐτοῦ, καὶ προσηύξατο.
\VS{72}Καὶ ὑπέστρεψε πρὸς αὐτοὺς πολέμῳ, καὶ ἐτροπώσατο αὐτούς, καὶ ἔφυγον.
\VS{73}Καὶ ἴδον οἱ φεύγοντες οἱ παρʼ αὐτοῦ, καὶ ἐπέστρεψαν πρὸς αὐτόν, καὶ ἐδίωκον μετʼ αὐτοῦ ἕως Κάδης ἕως της παρεμβολῆς αὐτῶν, καὶ παρενέβαλον ἐκεῖ.
\par }{\PP \VS{74}Καὶ ἔπεσον ἐκ τῶν ἀλλοφύλων ἐν τῇ ἡμέρα ἐκείνῃ εἰς ἄνδρας τρισχιλίους· καὶ ἐπέστρεψεν Ἰωνάθαν εἰς Ἱερουσαλήμ.

\par }\Chap{12}{\PP \VerseOne{1}Καὶ εἶδεν Ἰωνάθαν ὅτι ὁ καιρὸς αὐτῷ συνεργεῖ, καὶ ἐπέλεξεν ἄνδρας, καὶ ἀπέστειλεν εἰς Ῥώμην, στῆσαι καὶ ἀνανεώσασθαι τὴν πρὸς αὐτοὺς φιλίαν.
\VS{2}Καὶ πρὸς Σπαρτιάτας, καὶ τόπους ἑτέρους ἀπέστειλεν ἐπιστολὰς κατὰ τὰ αὐτά.
\par }{\PP \VS{3}Καὶ ἐπορεύθησαν εἰς Ῥώμην, καὶ εἰσῆλθον εἰς τὸ βουλευτήριον, καὶ εἶπον, Ἰωνάθαν ὁ ἀρχιερεὺς καὶ τὸ ἔθνος τῶν Ἰουδαίων ἀπέστειλεν ἡμᾶς ἀνανεώσασθαι τὴν φιλίαν αὐτοῖς, καὶ τὴν συμμαχίαν κατὰ τὸ πρότερον.
\VS{4}Καὶ ἔδωκαν ἐπιστολὰς αὐτοῖς πρὸς αὐτοὺς κατὰ τόπον, ὅπως προπέμπωσιν αὐτοὺς εἰς γῆν Ἰούδα μετʼ εἰρήνης.
\VS{5}Καὶ τοῦτο τὸ ἀντίγραφον τῶν ἐπιστολῶν ὧν ἔγραψεν Ἰωνάθαν τοῖς Σπαρτιάταις·
\par }{\PP \VS{6}Ἰωνάθαν ἀρχιερεὺς, καὶ ἡ γερουσία τοῦ ἔθνους, καὶ οἱ ἱερεῖς, καὶ ὁ λοιπὸς δῆμος τῶν Ἰουδαίων, Σπαρτιάταις τοῖς ἀδελφοῖς χαίρειν.
\par }{\PP \VS{7}Ἔτι πρότερον ἀπεστάλησαν ἐπιστολαὶ πρὸς Ὀνίαν τὸν ἀρχιερέα παρὰ Δαρείου τοῦ βασιλεύοντος ἐν ὑμῖν, ὅτι ἐστὲ ἀδελφοὶ ἡμῶν, ὡς τὸ ἀντίγραφον ὑπόκειται.
\VS{8}Καὶ ἐπεδέξατο Ὀνίας τὸν ἄνδρα τὸν ἀπεσταλμένον ἐνδόξως, καὶ ἔλαβε τὰς ἐπιστολάς ἐν αἷς διεσαφεῖτο περὶ συμμαχίας καὶ φιλίας.
\par }{\PP \VS{9}Καὶ ἡμεῖς οὖν ἀπροσδεεῖς τούτων ὄντες, παράκλησιν ἔχοντες τὰ βιβλία τὰ ἅγια τὰ ἐν ταῖς χερσὶν ἡμῶν,
\VS{10}ἐπειράθημεν ἀποστεῖλαι τὴν πρὸς ὑμᾶς ἀδελφότητα καὶ φιλίαν ἀνανεώσασθαι, πρὸς τὸ μὴ ἐξαλλοτριωθῆναι ὑμῶν· πολλοὶ γὰρ καιροὶ διῆλθον ἀφʼ οὗ ἀπεστείλατε πρὸς ἡμᾶς.
\par }{\PP \VS{11}Ἡμεῖς οὖν ἐν παντὶ καιρῷ ἀδιαλείπτως ἔν τε ταῖς ἑορταῖς καὶ ταῖς λοιπαῖς καθηκούσαις ἡμέραις μιμνησκόμεθα ὑμων, ἐφʼ ὧν προσφέρομεν θυσιῶν, καὶ ἐν ταῖς προσευχαῖς, ὡς δέον ἐστὶ καὶ πρέπον μνημονεύειν ἀδελφῶν.
\VS{12}Εὐφραινόμεθα δὲ ἐπὶ τῇ δόξῃ ὑμῶν.
\par }{\PP \VS{13}Ἡμᾶς δὲ ἐκύκλωσαν πολλαὶ θλίψεις, καὶ πόλεμοι πολλοὶ, καὶ ἐπολέμησαν ἡμᾶς οἱ βασιλεῖς οἱ κύκλῳ ἡμῶν.
\VS{14}Καὶ οὐκ ἠβουλόμεθα οὖν παρενοχλεῖν ὑμῖν, καὶ τοῖς λοιποῖς συμμάχοις, καὶ φίλοις ἡμῶν, ἐν τοῖς πολέμοις τούτοις.
\VS{15}Ἔχομεν γὰρ τὴν ἐξ οὐρανοῦ βοήθειαν βοηθοῦσαν ἡμῖν, καὶ ἐῤῥύσθημεν ἀπὸ τῶν ἐχθρῶν ἡμῶν, καὶ ἐταπεινώθησαν οἱ ἐχθροὶ ἡμῶν.
\VS{16}Ἐπελέξαμεν οὖν Νουμήνιον Ἀντιόχου καὶ Ἀντίπατρον Ἰάσωνος, καὶ ἀπεστάλκαμεν πρὸς Ῥωμαίους ἀνενεώσασθαι τὴν πρὸς αὐτοὺς φιλίαν καὶ συμμαχίαν τὴν προτέραν.
\VS{17}Ἐνετειλάμεθα οὖν αὐτοῖς καὶ πρὸς ὑμᾶς πορευθῆναι, καὶ ἀσπάσασθαι ὑμᾶς, καὶ ἀποδοῦναι ὑμῖν τὰς παρʼ ἡμῶν ἐπιστολὰς περὶ τῆς ἀνανεώσεως καὶ τῆς ἀδελφότητος ἡμῶν.
\VS{18}Καὶ νῦν καλῶς ποιήσετε ἀντιφωνήσαντες ἡμῖν πρὸς ταῦτα.
\par }{\PP \VS{19}Καὶ τοῦτο τὸ ἀντίγραφον τῶν ἐπιστολῶν ὧν ἀπέστειλεν. Ὀνιάρης
\VS{20}βασιλεὺς Σπαρτιατῶν Ὀνίᾳ ἱερεῖ μεγάλῳ χαίρειν.
\par }{\PP \VS{21}Εὑρέθη ἐν γραφῇ περί τε τῶν Σπαρτιατῶν καὶ Ἰουδαίων ὅτι εἰσὶν ἀδελφοὶ, καὶ ὅτι εἰσὶν ἐκ γένους Ἁβραάμ.
\VS{22}Καὶ νῦν ἀφʼ οὗ ἔγνωμεν ταῦτα, καλῶς ποιήσετε γράφοντες ἡμῖν περὶ τῆς εἰρήνης ὑμῶν.
\VS{23}Καὶ ἡμεῖς δὲ ἁντιγράφομεν ὑμῖν, τὰ κτήνη ὑμῶν καὶ ἡ ὕπαρξις ὑμῶν ἡμῖν ἐστι, καὶ τὰ ἡμῶν ὑμῖν ἐστιν· ἐντελλόμεθα οὖν ὅπως ἀπαγγείλωσιν ὑμῖν κατὰ ταῦτα.
\par }{\PP \VS{24}Καὶ ἤκουσεν Ἰωνάθαν ὅτι ἐπέστρεψαν οἱ ἄρχοντες Δημητρίου μετὰ δυνάμεως πολλῆς ὑπὲρ τὸ πρότερον τοῦ πολεμῆσαι πρὸς αὐτόν.
\VS{25}Καὶ ἀπῇρεν ἐξ Ἱερουσαλὴμ, καὶ ἀπήντησεν αὐτοῖς εἰς τὴν Ἀμαθῖτιν χώραν· οὐ γὰρ ἔδωκεν αὐτοῖς ἀνοχὴν ἐμβατεῦσαι εἰς τὴν χώραν αὐτοῦ.
\par }{\PP \VS{26}Καὶ ἀπέστειλε κατασκόπους εἰς τὴς παρεμβολὴν αὐτῶν, καὶ ἀπέστρεψαν, καὶ ἀπήγγειλαν αὐτῷ, ὅτι οὕτω τάσσονται ἐπιπεσεῖν ἐπʼ αὐτοὺς τὴν νύκτα.
\VS{27}Ὡς δὲ ἔδυ ὁ ἥλιος, ἐπέταζεν Ἰωνάθαν τοῖς παρʼ αὐτοῦ γρηγορεῖν, καὶ εἶναι ἐπὶ τοῖς ὅπλοις, καὶ ἑτοιμάζεσθαι εἰς πόλεμον διʼ ὅλης τῆς νυκτὸς, καὶ ἐξέβαλε προφύλακας κύκλῳ τῆς παρεμβολῆς.
\par }{\PP \VS{28}Καὶ ἤκουσαν οἱ ὑπεναντίοι ὅτι ἡτοίμασται Ἰωνάθαν καὶ οἱ παρʼ αὐτοῦ εἰς πόλεμον, καὶ ἐφοβήθησαν καὶ ἔπτηξαν τῇ καρδίᾳ αὐτῶν, καὶ ἀνέκαυσαν πυρὰς ἐν τῇ παρεμβολῇ αὐτῶν.
\VS{29}Ἰωνάθαν δὲ καὶ οἱ παρʼ αὐτοῦ οὐκ ἔγνωσαν ἕως πρωῒ, ἔβλεπον γὰρ τὰ φῶτα καιόμενα.
\VS{30}Καὶ κατεδίωξεν Ἰωνάθαν ὀπίσω αὐτῶν, καὶ οὐ κατέλαβεν αὐτούς, διέβησαν γὰρ τὸν Ἐλεύθερον ποταμόν.
\VS{31}Καὶ ἐξέκλινεν Ἰωνάθαν ἐπὶ τοὺς Ἄραβας τοὺς καλουμένους Ζαβεδαίους, καὶ ἐπάταξεν αὐτούς, καὶ ἔλαβε τὰ σκῦλα αὐτῶν.
\VS{32}Καὶ ἀναζεύξας ἦλθεν εἰς Δαμασκόν, καὶ διώδευσεν ἐν πάσῃ τῇ χώρᾳ.
\par }{\PP \VS{33}Καὶ Σίμων ἐξῆλθε, καὶ διώδευσεν ἕως Ἀσκάλωνος, καὶ τῶν πλησίων ὀχυρωμάτων, καὶ ἐξέκλινεν εἰς Ἰόππην, καὶ προκατελάβετο αὐτήν.
\VS{34}Ἤκουσε γὰρ ὅτι βούλονται τὸ ὀχύρωμα παραδοῦναι τοῖς παρὰ Δημητρίου, καὶ ἔθετο ἐκεῖ φρουρὰν, ὅπως φυλάσσωσιν αὐτήν.
\par }{\PP \VS{35}Καὶ ἐπέστρεψεν Ἰωνάθαν, καὶ ἐξεκκλησίασε τοὺς πρεσβυτέρους τοῦ λαοῦ, καὶ ἐβουλεύσατο μετʼ αὐτῶν τοῦ οἰκοδομῆσαι ὀξυρώματα ἐν τῇ Ἰουδαίᾳ,
\VS{36}καὶ προσυψῶσαι τὰ τείχη Ἱερουσαλὴμ, καὶ ὑψῶσαι ὕψος μέγα ἀναμέσον τῆς ἄκρας καὶ τῆς πόλεως, εἰς τὸ διαχωρίζειν αὐτὴν τῆς πόλεως, ἵνα ᾖ αὕτη κατὰ μόνας, ὅπως μήτε ἀγοράζωσι μήτε πωλῶσι.
\VS{37}Καὶ συνήχθησαν τοῦ οἰκοδομεῖν τὴν πόλιν, καὶ ἤγγισε τοῦ τείχους τοῦ χειμάῤῥου τοῦ ἐξ ἀπηλιώτου, καὶ ἐπεσκεύασαν τὸ καλούμενον Χαφεναθά.
\VS{38}καὶ Σίμων ᾠκοδόμησε τὴν Ἀδιδὰ ἐν τῇ Σεφήλᾳ, καὶ ὠχύρωσε θύρας καὶ μοχλούς.
\par }{\PP \VS{39}Καὶ ἐζήτησε Τρύφων βασιλεῦσαι τῆς Ἀσίας, καὶ περιθέσθαι τὸ διάδημα, καὶ ἐκτεῖναι χεῖρα ἐπὶ Ἀντίοχον τὸν βασιλέα.
\VS{40}Καὶ ἐφοβήθη μήποτε οὐκ ἐάσῃ αὐτὸν Ἰωνάθαν, καὶ μήποτε πολεμήσῃ πρὸς αὐτὸν, καὶ ἐζήτει πόρον τοῦ συλλαβεῖν τὸν Ἰωνάθαν τοῦ ἀπολυσαι αὐτὸν, καὶ ἀπάρας ἦλθεν εἰς Βαιθσάν.
\par }{\PP \VS{41}Καὶ ἐξῆλθεν Ἰωνάθαν εἰς ἀπάντησιν αὐτῷ ἐν τεσσαράκοντα χιλιάσιν ἀνδρῶν ἐπιλελεγμέναις εἰς παράταξιν, καὶ ἦλθεν εἰς Βαιθσάν.
\par }{\PP \VS{42}Καὶ εἶδε Τρύφων ὅτι πάρεστιν Ἰωνάθαν μετὰ δυνάμεως πολλῆς, καὶ ἐκτεῖναι χεῖρας ἐπʼ αὐτὸν εὐλαβήθη.
\VS{43}Καὶ ἐπεδέξατο αὐτὸν ἐνδόξως, καὶ συνέστησεν αὐτὸν πᾶσι τοῖς φίλοις αὐτοῦ, καὶ ἔδωκεν αὐτῷ δόματα, καὶ ἐπέταξε ταῖς δυνάμεσιν αὐτοῦ ὑπακούειν αὐτῷ ὡς ἑαυτῷ.
\par }{\PP \VS{44}Καὶ εἶπε τῷ Ἰωνάθαν, ἱνατί ἔκοψας πάντα τὸν λαὸν τοῦτον, πολέμου μὴ ἐνεστηκότος ἡμῖν;
\VS{45}Καὶ νῦν ἀπόστειλον αὐτοὺς εἰς τοὺς οἴκους αὐτῶν, ἐπίλεξαι δὲ σεαυτῷ ἄνδρας ὀλίγους οἵτινες ἔσονται μετὰ σοῦ, καὶ δεῦρο μετʼ ἐμοῦ εἰς Πτολεμαΐδα, καὶ παραδώσω σοι αὐτὴν καὶ τὰ λοιπὰ ὀχυρώματα καὶ τὰς δυνάμεις τὰς λοιπὰς καὶ πάντας τοὺς ἐπὶ τῶν χειρῶν, καὶ ἐπιστρέψας ἀπελεύσομαι, τούτου γὰρ χάριν πάρειμι.
\par }{\PP \VS{46}Καὶ ἐμπιστεύσας αὐτῷ ἐποίησε καθὼς εἶπε, καὶ ἐξαπέστειλε τὰς δυνάμεις, καὶ ἀπῆλθον εἰς γῆν Ἰούδα.
\VS{47}Κατέλιπε δὲ μεθʼ ἑαυτοῦ ἄνδρας τρισχιλίους, ὧν δισχιλίους ἀφῆκεν ἐν τῇ Γαλιλαίᾳ, χίλιοι δὲ συνῆλθον αὐτῷ.
\par }{\PP \VS{48}Ὡς δὲ εἰσῆλθεν Ἰωνάθαν εἰς Πτολεμαΐδα, ἀπέκλεισαν οἱ Πτολεμαεῖς τὰς πύλας, καὶ συνέλαβον αὐτὸν, καὶ πάντας τοὺς εἰσελθόντας μετʼ αὐτοῦ ἀπέκτειναν ἐν ῥομφαίᾳ.
\VS{49}Καὶ ἀπέστειλε Τρύφων δυνάμεις, καὶ ἵππον εἰς τὴν Γαλιλαίαν, καὶ τὸ πεδίον τὸ μέγα, τοῦ ἀπολέσαι πάντας τοὺς παρὰ Ἰωνάθαν.
\VS{50}Καὶ ἐπέγνωσαν ὅτι συνελήφθη Ἰωνάθαν καὶ ἀπόλωλε, καὶ οἱ μετʼ αὐτοῦ, καὶ παρεκάλεσαν ἑαυτοὺς, καὶ ἐπορεύοντο συνεστραμμένοι ἕτοιμοι εἰς πόλεμον.
\par }{\PP \VS{51}Καὶ ἴδον οἱ διώκοντες ὅτι περὶ ψυχῆς αὐτοῖς ἐστι, καὶ ἐπέστρεψαν.
\VS{52}Καὶ ἦλθον πάντες μετʼ εἰρήνης εἰς γῆν Ἰούδα, καὶ ἐπένθησαν τὸν Ἰωνάθαν, καὶ τοὺς μετʼ αὐτοῦ, καὶ ἐφοβήθησαν σφόδρα, καὶ ἐπένθησε τᾶς Ἰσραὴλ πένθος μέγα.
\par }{\PP \VS{53}Καὶ ἐζήτησαν πάντα τὰ ἔθνη τὰ κύκλῳ αὐτῶν ἐκτρίψαι αὐτούς· εἶπαν γάρ, οὐκ ἔχουσιν ἄρχοντα καὶ βοηθοῦντα· νῦν οὖν πολεμήσωμεν αὐτοὺς, καὶ ἐξάρωμεν ἐξ ἀνθρώπων τὸ μνημόσυνον αὐτῶν.

\par }\Chap{13}{\PP \VerseOne{1}Καὶ ἤκουσε Σίμων ὅτι συνήγαγε Τρύφων δύναμιν πολλὴν τοῦ ἐλθεῖν εἰς γῆν Ἰούδα, καὶ ἐκ ρίψαι αὐτήν.
\VS{2}Καὶ εἶδε τὸν λαὸν ὅτι ἐστὶν ἔντρομος, καὶ ἔμφοβος, καὶ ἀνέβη εἰς Ἱερουσαλὴμ, καὶ ἤθροισε τὸν λαόν.
\VS{3}Καὶ παρεκάλεσεν αὐτοὺς, καὶ εἶπεν αὐτοῖς, αὐτοὶ οἴδατε ὅσα ἐγὼ, καὶ οἱ ἀδελφοί μου, ὁ καὶ οἶκος τοῦ πατρός μου, ἐποιήσαμεν περὶ τῶν νόμων, καὶ τῶν ἁγίων, καὶ τοὺς πολέμους, καὶ τὰς στενοχωρίας ἃς εἴδομεν.
\VS{4}Τούτου χάριν ἀπώλοντο οἱ ἀδελφοί μου πάντες χάριν τοῦ Ἰσραὴλ, καὶ κατελείφθην ἐγὼ μόνος.
\par }{\PP \VS{5}Καὶ νῦν μή μοι γένοιτο φείσασθαί μου τῆς ψυχῆς ἐν παντὶ καιρῷ θλίψεως, οὐ γάρ εἰμι κρείσσων τῶν ἀδελφῶν μου.
\VS{6}Πλὴν ἐκδικήσω περὶ τοῦ ἔθνους μου, καὶ περὶ τῶν ἁγίων, καὶ περὶ τῶν γυναικῶν καὶ τῶν τέκνων ἡμῶν, ὅτι συνήχθησαν πάντα τὰ ἔθνη ἐκτρίψαι ἡμᾶς ἔχθρας χάριν.
\par }{\PP \VS{7}Καὶ ἀνεζωοπύρησε τὸ πνεῦμα τοῦ λαοῦ ἅμα τῷ ἀκοῦσαι τῶν λόγων τούτων,
\VS{8}καὶ ἀπεκρίθησαν φωνῇ μεγάλῃ, λέγοντες, σὺ εἶ ἡμῶν ἡγούμενος ἀντὶ Ἰούδα, καὶ Ἰωνάθαν τοῦ ἀδελφοῦ σου.
\VS{9}Πολέμησον τὸν πόλεμον ἡμῶν, καὶ πάντα ὅσα ἂν εἴπῃς ἡμῖν, ποιήσομεν.
\par }{\PP \VS{10}Καὶ συνήγαγε πάντας τοὺς ἄνδρας τοὺς πολεμιστὰς, καὶ ἐτάχυνε τοῦ τελέσαι τὰ τείχη Ἱερουσαλήμ, καὶ ὠχύρωσεν αὐτὴν κυκλόθεν.
\VS{11}Καὶ ἀπέστειλεν Ἰωνάθαν τὸν τοῦ Ἀβεσσαλώμου καὶ μετʼ αὐτοῦ δύναμιν ἱκανὴν εἰς Ἰόππην, καὶ ἐξέβαλε τοὺς ὄντας ἐν αὐτῇ, καὶ ἔμεινεν ἐκεῖ ἐν αὐτῇ.
\par }{\PP \VS{12}Καὶ ἀπῇρε Τρύφων ἀπὸ Πτολεμαΐδος μετὰ δυνάμεως πολλῆς εἰσελθεῖν εἰς γῆν Ἰούδα, καὶ Ἰωνάθαν μετʼ αὐτοῦ ἐν φυλακῇ.
\VS{13}Σίμων δὲ παρενέβαλεν ἐν Ἀδιδὰ κατὰ πρόσωπον τοῦ πεδίου.
\par }{\PP \VS{14}Καὶ ἐπέγνω Τρύφων ὅτι ἀνέστη Σίμων ἀντὶ Ἰωνάθου τοῦ ἀδελφοῦ αὐτοῦ, καὶ ὅτι συνάπτειν αὐτῷ μέλλει πόλεμον, καὶ ἀπέστειλε πρὸς αὐτὸν πρέσβεις, λέγων,
\VS{15}περὶ ἀργυρίου οὗ ὤφειλεν Ἰωνάθαν ὁ ἀδελφός σου εἰς τὸ βασιλικὸν διʼ ἃς εἶχε χρείας συνέχομεν αὐτόν.
\VS{16}Καὶ νῦν ἀπόστειλον ἀργυρίου τάλαντα ἑκατὸν, καὶ δύο τῶν υἱῶν αὐτοῦ ὅμηρα, ὅπως μὴ ἀφεθεὶς ἀποστατήσῃ ἀφʼ ἡμῶν, καὶ ἀφήσομεν αὐτόν.
\par }{\PP \VS{17}Καὶ ἔγνω Σίμων ὅτι δόλῳ λαλοῦσι πρὸς αὐτὸν, καὶ πέμπει τὸ ἀργύριον, καὶ τὰ παιδάρια, μήποτε ἔχθραν ἄρῃ μεγάλην πρός τὸν λαόν,
\VS{18}λέγων, ὅτι οὐκ ἀπέστειλα αὐτῷ τὸ ἀργύριον καὶ τὰ παιδάρια, καὶ ἀπώλετο.
\VS{19}Καὶ ἀπέστειλε τὰ παιδάρια, καὶ τὰ ἑκατὸν τάλαντα· καὶ διεψεύσατο, καὶ οὐκ ἀφῆκε τὸν Ἰωνάθαν.
\par }{\PP \VS{20}Καὶ μετὰ ταῦτα ἦλθε Τρύφων τοῦ ἐμβατεῦσααι εἰς τὴν χώραν, καὶ ἐκτρίψαι αὐτὴν, καὶ ἐκύκλωσεν ὁδὸν τὴν εἰς Ἄδωρα· καὶ Σίμων καὶ ἡ παρεμβολὴ αὐτοῦ ἀντιπαρῆγεν αὐτῷ εἰς πάντα τόπον οὗ ἂν ἐπορεύετο.
\par }{\PP \VS{21}Οἱ δὲ ἐκ τῆς ἄκρας ἀπέστελλον πρὸς Τρύφωνα πρεσβευτὰς κατασπεύδοντας αὐτὸν τοῦ ἐλθεῖν προὺς αὐτὸν διὰ τῆς ἐρήμου, καὶ· ἀποστεῖλαι αὐτοῖς τροφάς.
\VS{22}Καὶ ἡτοίμασε Τρύφων πᾶσαν τὴν ἵππον αὐτοῦ ἐλθεῖν ἐν τῇ νυκτὶ ἐκείνῃ· καὶ ἦν χιὼν πολλὴ σφόδρα, καὶ οὐκ ἦλθε διὰ τὴν χιόνα, καὶ ἀπῇρε, καὶ ἦλθεν εἰς τὴν Γαλααδίτιν.
\VS{23}Ὡς δὲ ἤγγισε τῇ Βασκαμᾷ, ἀπέκτεινε τὸν Ἰωνάθαν, καὶ ἐτάφη ἐκεῖ.
\VS{24}Καὶ ἐπέστρεψε Τρύφων, καὶ ἀπῆλθεν εἰς τὴν γῆν αὐτοῦ.
\par }{\PP \VS{25}Καὶ ἀπέστειλε Σίμων, καὶ ἔλαβε τὰ ὀστᾶ Ἰωνάθαν τοῦ ἀδελφοῦ αὐτοῦ, καὶ ἔθαψεν αὐτὰ ἐν Μωδεῒν πόλει τῶν πατέρων αὐτοῦ.
\VS{26}Καὶ ἐκόψαντο αὐτὸν πᾶς Ἰσραὴλ κοπετὸν μέγαν, καὶ ἐπένθησαν αὐτὸν ἡμέρας πολλάς.
\par }{\PP \VS{27}Καὶ ᾠκοδόμησε Σίμων ἐπὶ τὸν τάφον τοῦ πατρὸς αὐτοῦ καὶ τῶν ἀδελφῶν αὐτοῦ, καὶ ὕψωσεν αὐτὸν τῇ ὁράσει λίθῳ ξεστῷ ἐκ τῶν ὄπισθεν καὶ ἐκ τῶν ἔμπροσθεν.
\VS{28}Καὶ ἔστησεν ἐπʼ αὐτὰ ἑπτὰ πυραμίδας, μίαν κατέναντι τῆς μιᾶς, τῷ πατρὶ καὶ τῇ μητρὶ καὶ τῇ μητρὶ καὶ τοῖς τέσσαρσιν ἀδελφοῖς.
\VS{29}Καὶ ταύταις ἐποίησε μηχανήματα, περιθεὶς στύλους μεγάλους, καὶ ἐποίησεν ἐπὶ τοῖς στύλοις πανοπλίας εἰς ὄνομα αἰώνιον, καὶ παρὰ ταῖς πανοπλίαις πλοῖα ἐπιγεγλυμμένα, εἰς τὸ θεωρεῖσθαι ὑπὸ πάντων τῶν πλεόντων τὴν θάλασσαν.
\VS{30}Οὗτος ὁ τάφος ὃν ἐποίησεν ἐν Μωδεῒν, ἕως τῆς ἡμέρας ταύτης.
\par }{\PP \VS{31}Ὁ δὲ Τρύφων ἐπορεύετο δόλῳ μετὰ Ἀντιόχου τοῦ βασιλέως τοῦ νεωτέρου, καὶ ἀπέκτεινεν αὐτὸν,
\VS{32}καὶ ἐβασίλευσεν ἀντʼ αὐτοῦ, καὶ περιέθετο διάδημα τῆς Ἀσίας, καὶ ἐποίησε πληγὴν μεγάλην ἐπὶ τῆς γῆς.
\par }{\PP \VS{33}Καὶ ᾠκοδόμησε Σίμων τὰ ὀχυρώματα τῆς Ἰουδαίας, καὶ περιετείχισε πύργοις ὑψηλοῖς, καὶ τείχεσι μεγάλοις, καὶ πύλαις, καὶ μοχλοῖς, καὶ ἔθετο βρώματα ἐν τοῖς ὀχυρώμασι.
\VS{34}Καὶ ἐπέλεξε Σίμων ἄνδρας, καὶ ἀπέστειλε πρὸς Δημήτριον τὸν βασιλέα τοῦ ποιῆσαι ἄφεσιν τῇ χώρᾳ, ὅτι πᾶσαι αἱ πράξεις Τρύφωνος ἦσαν ἁρπαγαί.
\par }{\PP \VS{35}Καὶ ἀπέστειλεν αὐτῷ Δημήτριος ὁ βασιλεὺς κατὰ τοὺς λόγους τούτους· καὶ ἀπεκρίθη αὐτῷ, καὶ ἔγραψεν αὐτῷ ἐπιστολὴν τοιαύτην·
\VS{36}Βασιλεὺς Δημήτριος Σίμονι ἀρχιερεῖ καὶ φίλῳ βασιλέων, καὶ πρεσβυτέροις, καὶ ἔθνει Ἰουδαίων χαίρειν.
\VS{37}Τὸν στέφανον τὸν χρυσοῦν, καὶ τὴν βαΐνην ἣν ἀπεστείλατε, κεκομίσμεθα, καὶ ἕτοιμοί ἐσμεν τοῦ ποιεῖν ὑμῖν εἰρήνην μεγάλην, καὶ γράφειν τοῖς ἐπὶ τῶν χρειῶν τοῦ ἀφιέναι ὑμῖν ἀφέματα.
\VS{38}Καὶ ὅσα ἑστήκαμεν πρὸς ὑμᾶς ἔστηκε, καὶ τὰ ὀχυρώματα ἃ ᾠκοδομήκατε ὑπαρχέτω ὑμῖν.
\VS{39}Ἀφίεμεν δὲ ἀγνοήματα καὶ τὰ ἁμαρτήματα ἕως τῆς σήμερον ἡμέρας, καὶ τὸν στέφανον ὃν ὠφείλετε, καὶ εἴ τι ἄλλο ἐτελωνεῖτο ἐν Ἱερουσαλήμ, μηκέτι τελωνείσθω.
\VS{40}Καὶ εἴ τινες ἐπιτήδειοι ὑμῶν γραφῆναι εἰς τοὺς περὶ ἡμᾶς, ἐγγραφέσθωσαν, καὶ γινέσθω ἀναμέσον ἡμῶν εἰρήνη.
\par }{\PP \VS{41}Ἔτους ἑβδομηκοστοῦ καὶ ἑκατοστοῦ ᾔρθη ὁ ζυγὸς τῶν ἐθνῶν ἀπὸ τοῦ Ἰσραήλ.
\VS{42}Καὶ ἤρξατο ὁ λαὸς Ἰσραὴλ γράφειν ἐν ταῖς συγγραφαῖς καὶ συναλλάγμασιν, ἔτους πρώτου ἐπὶ Σίμωνος ἀρχιερέως μεγάλου καὶ στρατηγοῦ καὶ ἡγουμένου Ἰουδαίων.
\par }{\PP \VS{43}Ἐν ταῖς ἡμέραις ἐκείναις παρενέβαλε Σίμων ἐπὶ Γάζαν, καὶ ἐκύκλωσεν αὐτὴν παρεμβολαῖς, καὶ ἐποίησεν ἑλεπόλεις καὶ προσήγαγε τῇ πόλει, καὶ ἐπάταξε πύργον ἕνα καὶ κατελάβετο.
\VS{44}Καὶ ἐξήλλοντο οἱ ἐν τῇ ἑλεπόλει εἰς τὴν πόλιν, καὶ ἐγένετο κίνημα μέγα ἐν τῇ πόλει.
\VS{45}Καὶ ἀνέβησαν οἱ ἐν τῇ πόλει σὺν ταῖς γυναιξὶ καὶ τοῖς τέκνοις ἐπὶ τὸ τεῖχος διεῤῥηχότες τὰ ἱμάτια αὐτῶν, καὶ ἐβόησαν φωνῇ μεγάλῃ ἀξιοῦντες Σίμωνα δεξιὰς αὐτοῖς δοῦναι,
\VS{46}καὶ εἶπον, μὴ ἡμῖν χρήσῃ κατὰ τὰς πονηρίας ἡμῶν, ἀλλὰ κατὰ τὸ ἔλεός σου.
\par }{\PP \VS{47}Καὶ συνελύθη Σίμων αὐτοῖς, καὶ οὐκ ἐπολέμησεν αὐτούς· καὶ ἐξέβαλεν αὐτοὺς ἐκ τῆς πόλεως, καὶ ἐκαθάρισε τὰς οἰκίας ἐν αἷς ἦν τὰ εἴδωλα, καὶ οὕτως εἰσῆλθεν εἰς αὐτὴς ὑμνῶν καὶ εὐλογῶν.
\VS{48}Καὶ ἐξέβαλεν ἐξ αὐτῆς πᾶσαν ἀκαθαρσίαν, καὶ κατῴκισεν ἐκεῖ ἄνδρας οἵτινες τὸν νόμον ποιοῦσι, καὶ προσωχύρωσεν αὐτὴν, καὶ ᾠκοδόμησεν ἑαυτῷ ἑν αὐτῇ οἴκησιν.
\par }{\PP \VS{49}Οἱ δὲ ἐκ τῆς ἄκρας ἐν Ἱερουσαλὴμ ἐκωλύοντο ἐκπορεύεσθαι καὶ εἰσπορεύεσθαι εἰς τὴν χώραν, καὶ ἀγοράζειν καὶ πωλεῖν, καὶ ἐπείνασαν σφόδρα, καὶ ἀπώλοντο ἐξ αὐτῶν ἱκανοὶ τῇ λιμῷ.
\VS{50}Καὶ ἐβόησαν πρὸς Σίμωνα δεξιὰς λαβεῖν, καὶ ἔδωκεν αὐτοῖς, καὶ ἐξέβαλεν αὐτοὺς ἐκεῖθεν, καὶ ἐκαθάρισε τὴν ἄκραν ἀπὸ τῶν μιασμάτων.
\VS{51}Καὶ εἰσῆλθεν εἰς αὐτὴν τῇ τρίτῃ καὶ εἰκάδι τοῦ δευτέρου μηνὸς ἔτους ἑνὸς καὶ ἑβδομηκοστοῦ καὶ ἑκατοστοῦ μετὰ αἰνέσεως καὶ βαΐων, καὶ ἐν κινύραις, καὶ ἐν κυμβάλοις, καὶ ἐν νάβλαις, καὶ ἐν ὕμνοις, καὶ ἐν ᾠδαῖς, ὅτι συνετρίβη ἐχθρὸς μέγας ἐξ Ἰσραήλ.
\par }{\PP \VS{52}Καὶ ἔστησε κατʼ ἐνιαυτὸν τοῦ ἄγειν τὴν ἡμέραν ταύτην μετʼ εὐφροσύνης· καὶ προσωχύρωσε τὸ ὄρος τοῦ ἱεροῦ τὸ παρὰ τὴν ἄκραν, καὶ ᾤκει ἐκεῖ αὐτὸς καὶ οἱ παρʼ αὐτοῦ.
\VS{53}Καὶ εἶδε Σίμων τὸν Ἰωάννην υἱὸν αὐτοῦ, ὅτι ἀνήρ ἐστι, καὶ ἔθετο αὐτὸν ἡγούμενον τῶν δυνάμεων πασῶν, καὶ ᾤκει ἐν Γαζάροις.

\par }\Chap{14}{\PP \VerseOne{1}Καὶ ἐν ἔτει δευτέρῳ καὶ ἑβδομηκοστῷ καὶ ἑκατοστῷ συνήγαγε Δημήτριος ὁ βασιλεὺς τὰς δυνάμεις αὐτοῦ· καὶ ἐπορεύθη εἰς Μήδειαν τοῦ ἐπισπάσασθαι βοήθειαν αὐτῷ, ὅπως πολεμήσῃ τὸν Τρύφωνα.
\par }{\PP \VS{2}Καὶ ἤκουσεν Ἀρσάκης ὁ βασιλεὺς τῆς Περσίδος καὶ Μηδειας ὅτι ἦλθε Δημήτριος εἰς τὰ ὅρια αὐτοῦ, καὶ ἀπέστειλεν ἕνα τῶν ἀρχόντων αὐτοῦ συλλάβεῖν αὐτὸν ζῶντα.
\VS{3}Καὶ ἐπορεύθη καὶ ἐπάταξε τὴν παρεμβολὴν Δημητρίου, καὶ συνέλαβεν αὐτὸν, καὶ ἤγαγεν αὐτὸν πρὸς Ἀρσάκην, καὶ ἔθετο αὐτὸν ἐν φυλακῇ.
\par }{\PP \VS{4}Καὶ ἡσύχασεν ἡ γῆ Ἰούδα πάσας τὰς ἡμέρας Σίμωνος· καὶ ἐζήτησεν ἀγαθὰ τῷ ἔθνει αὐτοῦ, καὶ ἤρεσεν αὐτοῖς ἡ ἐξουσία αὐτοῦ καὶ ἡ δόξα αὐτοῦ πάσας τὰς ἡμέρας.
\VS{5}Καὶ μετὰ πάσης τῆς δόξης αὐτοῦ ἔλαβε τὴν Ἰόππην εἰς λιμένα, καὶ ἐποιήσεν εἴσοδον ταῖς νήσοις τῆς θαλάσσης.
\VS{6}Καὶ ἐπλάτυνε τὰ ὅρια τῷ ἔθνει αὐτοῦ, καὶ ἐκράτησε τῆς χώρας.
\VS{7}Καὶ συνήγαγεν αἰχμαλωσίαν πολλὴν, καὶ ἐκυρίευσε Γαζαρῶν καὶ Βαιθσούρων καὶ τῆς ἄκρας· καὶ ἐξῇρε τὰς ἀκαθαρσίας ἐξ αὐτῆς, καὶ οὐκ ἦν ὁ ἀντικείμενος αὐτῷ.
\par }{\PP \VS{8}Καὶ ἦσαν γεωργοῦντες τὴν γῆν αὐτῶν μετʼ εἰρήνης, καὶ ἡ γῆ ἐδίδου τὰ γεννήματα αὐτῆς, καὶ τὰ ξύλα τῶν πεδίων τὸν καρπὸν αὐτῶν.
\VS{9}Πρεσβύτεροι ἐν ταῖς πλατείαις ἐκάθηντο, πάντες περὶ ἀγαθῶν ἐκοινολογοῦντο, καὶ οἱ νεανίσκοι ἐνεδύσαντο δόξας καὶ στολὰς πολέμου.
\VS{10}Ταῖς πόλεσιν ἐχορήγησε βρώματα, καὶ ἔταξεν αὐτὰς ἐν σκεύεσιν ὀχυρώσεως, ἕως ὅτου ὠνομάσθη τὸ ὄνομα τῆς δόξης αὐτοῦ ἕως ἄκρου τῆς γῆς.
\par }{\PP \VS{11}Ἐποίησε τὴν εἰρήνην ἐπὶ τῆς γῆς, καὶ εὐφράνθη Ἰσραὴλ εὐφροσύνην μεγάλην.
\VS{12}Καὶ ἐκάθισεν ἕκαστος ὑπὸ τὴν ἄμπελον αὐτοῦ καὶ τὴν συκῆν αὐτοῦ, καὶ οὐκ ἦν ὁ ἐκφοβῶν αὐτούς.
\VS{13}Καὶ ἐξέλιπεν ὁ πολεμῶν αὐτοὺς ἐπὶ τῆς γῆς, καὶ οἱ βασιλεῖς συνετρίβησαν ἐν ταῖς ἡμέραις ἐκείναις.
\VS{14}Καὶ ἐστήρισε πάντας τοὺς ταπεινοὺς τοῦ λαοῦ αὐτοῦ· τὸν νόμον ἐξεζήτησε, καὶ ἐξῇρε πάντα ἄνομον καὶ πονηρόν.
\VS{15}Τὰ ἅγια ἐδόξασε, καὶ ἐπλήθυνε τὰ σκεύη τῶν ἁγίων.
\par }{\PP \VS{16}Καὶ ἠκούσθη ἐν Ῥώμῃ ὅτι ἀπέθανεν Ἰωνάθαν, καὶ ἕως Σπάρτης, καὶ ἐλυπήθησαν σφόδρα.
\VS{17}Ὡς δὲ ἤκουσαν ὅτι Σίμων ὁ ἀδελφὸς αὐτοῦ γέγονεν ἀντʼ αὐτοῦ ἀρχιερεὺς, καὶ ἐπικρατεῖ τῆς χώρας καὶ τῶν πόλεων τῶν ἐν αὐτῇ.
\VS{18}Ἔγραψαν πρὸς αὐτὸν δέλτοις χαλκαῖς, τοῦ ἀνανεώσασθαι πρὸς αὐτὸν φιλίαν καὶ τὴν συμμαχίαν ἣν ἔστησαν πρὸς Ἰούδαν καὶ Ἰωνάθαν τοὺς ἀδελφοὺς αὐτοῦ.
\VS{19}Καὶ ἀνεγνώσθησαν ἐνώπιον τῆς ἐκκλησίας ἐν Ἱερουσαλήμ.
\par }{\PP \VS{20}Καὶ τοῦτο τὸ ἀντίγραφον τῶν ἐπιστολῶν ὧν ἀπέστειλαν οἱ Σπαρτιάται· Σπαρτιατῶν ἄρχοντες καὶ ἡ πόλις Σίμωνι ἱερεῖ μεγάλῳ, καὶ τοῖς πρεσβυτέροις, καὶ τοῖς ἱερεῦσι, καὶ τῷ λοιπῷ δήμῳ τῶν Ἰουδαίων ἀδελφοῖς χαίρειν.
\VS{21}Οἱ πρεσβεύται οἱ ἀποσταλέντες πρὸς τὸν δῆμον ἡμῶν ἀπήγγειλαν ἡμῖν περὶ τῆς δόξης ὑμῶν καὶ τιμῆς, καὶ εὐφράνθημεν ἐπὶ τῆ ἐφόδῳ αὐτῶν.
\VS{22}Καὶ ἀνεγράψαμεν τὰ ὑπʼ αὐτῶν εἰρημένα ἐν ταῖς βουλαῖς τοῦ δήμου οὕτως, Νουμήνιος Ἀντιόχου καὶ Ἀντίπατρος Ἰάσωνος πρεσβευταὶ Ἰουδαίων ἤλθοσαν πρὸς ἡμᾶς ἀνανεούμενοι τὴν πρὸς ἡμᾶς φιλίαν.
\par }{\PP \VS{23}Καὶ ἤρεσε τῷ δήμῳ ἐπιδέξασθαι τοὺς ἄνδρας ἐνδόξως, καὶ τοῦ θέσθαι τὸ ἀντίγραφον τῶν λόγων αὐτῶν ἐν τοῖς ἀποδεδειγμένοις τοῦ δήμου βιβλίοις, τοῦ ἔχειν μνημόσυνον τὸν δῆμον τῶν Σπαρτιατῶν· τὸ δὲ ἀντίγραφον τούτων ἐγράψαμεν Σίμωνι τῷ ἀρχιερεῖ.
\par }{\PP \VS{24}Μετὰ ταῦτα ἀπέστειλε Σίμων τὸν Νουμήνιον εἰς Ῥώμην ἔχοντα ἀσπίδα χρυσῆν μεγάλην ὁλκῆς μνῶν χιλιων, εἰς τὸ στῆσαι πρὸς αὐτοὺς τὴν συμμαχίαν.
\VS{25}Ὡς δὲ ἤκουσεν ὁ δῆμος τῶν λόγων τούτων, εἶπον, τίνα χάριν ἀποδώσομεν Σίμωνι καὶ τοῖς υἱοῖς αὐτοῦ;
\VS{26}Ἐστήρισε γὰρ αὐτὸς καὶ οἱ ἀδελφοὶ αὐτοῦ, καὶ ὁ οἶκος τοῦ πατρὸς αὐτοῦ, καὶ ἐπολέμησαν τοὺς ἐχθροὺς Ἰσραὴλ ἀπʼ αὐτῶν, καὶ ἔστησαν αὐτῷ ἐλευθερίαν.
\par }{\PP \VS{27}Καὶ κατέγραψαν ἐν δέλτοις χαλκαῖς, καὶ ἔθεντο ἐν στήλαις ἐν ὄρει Σιών· καὶ τοῦτο τὸ ἀντίγραφον τῆς γραφῆς· ὀκτωκαιδεκάτῃ Ἐλούλ, ἔτους δευτέρου καὶ ἑβδομηκοστοῦ καὶ ἑκατοστοῦ· καὶ τοῦτο τρίτον ἔτος ἐπὶ Σίμωνος ἀρχιερέως·
\VS{28}ἐν Σαραμὲλ, ἐπὶ συναγωγῆς μεγάλης ἱερέων, καὶ λαοῦ, καὶ ἀρχόντων ἔθνους, καὶ τῶν πρεσβυτέρων τῆς χώρας ἐγνώρισεν ἡμῖν.
\par }{\PP \VS{29}Ἐπεὶ πολλάκις ἐγενήθησαν πόλεμοι ἐν τῇ χώρᾳ· Σίμων δὲ ὁ υἱὸς Ματταθίου ὁ υἱὸς τῶν υἱῶν Ἰαρὶβ καὶ οἱ ἀδελφοὶ αὐτοῦ ἔδωκαν ἑαυτοὺς τῷ κινδύνῳ, καὶ ἀντέστησαν τοῖς ὑπεναντίοις τοῦ ἔθνους αὐτῶν, ὅπως σταθῇ τὰ ἅγια αὐτῶν καὶ ὁ νόμος, καὶ δόξῃ μεγάλῃ ἐδόξασαν τὸ ἔθνος αὐτῶν·
\par }{\PP \VS{30}Καὶ ἤθροισεν Ἰωνάθαν τὸ ἔθνος αὐτῶν, καὶ ἐγενήθη αὐτοῖς ἀρχιερεὺς, καὶ προσετέθη πρὸς τὸν λαὸν αὐτοῦ.
\VS{31}Καὶ ἐβουλήθησαν οἱ ἐχθροὶ αὐτῶν ἐμβατεῦσαι εἰς τὴν χώραν αὐτῶν, τοῦ ἐκτρίψαι τὴν χώραν αὐτῶν, καὶ ἐκτεῖναι χεῖρας ἐπὶ τὰ ἅγια αὐτῶν·
\VS{32}τότε ἀνέστη Σίμων, καὶ ἐπολέμησε περὶ τοῦ ἔθνους αὐτοῦ, καὶ ἐδαπάνησε χρήματα πολλὰ τῶν ἑαυτοῦ, καὶ ὡπλοδότησε τοὺς ἄνδρας τῆς δυνάμεως τοῦ ἔθνους αὐτοῦ, καὶ ἔδωκεν αὐτοῖς ὀψώνια,
\VS{33}καὶ ὠχύρωσε τὰς πόλεις τῆς Ἰουδαίας, καὶ τὴν Βαιθσούραν τὴν ἐπὶ τῶν ὁρίων τῆς Ἰουδαίας, οὗ ἦν τὰ ὅπλα τῶν πολεμίων τοπρότερον, καὶ ἔθετο ἐκεῖ φρουρὰν ἄνδρας Ἰουδαίους.
\VS{34}Καὶ Ἰόππην ὠχύρωσε τὴν ἐπὶ τῆς θαλάσσης, καὶ τὴν Γάζαρα τὴν ἐπὶ τῶν ὁρίων Ἀζώτου, ἐν ᾗ ᾤκουν οἱ πολέμιοι τοπρότερον ἐκεῖ, καὶ κατῴκισεν ἐκεῖ Ἰουδαίους, καὶ ὅσα ἐπιτήδεια ἦν πρὸς τὴν τούτων ἐπανόρθωσιν ἔθετο ἐν αὐτοῖς.
\par }{\PP \VS{35}Καὶ εἶδεν ὁ λαὸς τὴς πρᾶξιν τοῦ Σίμωνος, καὶ τὴν δόξαν ἣν ἐβουλεύσατο ποιῆσαι τῷ ἔθνει αὐτοῦ, καὶ ἔθεντο αὐτὸν ἡγοὺμενον αὐτῶν καὶ ἀρχιερέα, διὰ τὸ αὐτὸν πεποιηκέναι πάντα ταῦτα, καὶ τὴν δικαιοσύνην, καὶ τὴν πίστιν ἣν συνετήρησε τῷ ἔθνει αὐτοῦ, καὶ ἐζήτησε παντὶ τρόπῳ ὑψῶσαι τὸν λαὸν αὐτοῦ.
\par }{\PP \VS{36}Καὶ ἐν ταῖς ἡμέραις αὐτοῦ εὐωδώθη ἐν ταῖς χερσὶν αὐτοῦ, τοῦ ἐξαρθῆναι τὰ ἔθνη ἐκ τῆς χώρας αὐτῶν, καὶ τοὺς ἐν τῇ πόλει Δαυὶδ τοὺς ἐν Ἱερουσαλὴμ, οἳ ἐποίησαν ἑαυτοῖς ἄκραν, ἐξ ἧς ἐξεπορεύοντο καὶ ἐμίαινον κύκλῳ τῶν ἁγίων, καὶ ἐποίουν πληγὴν μεγάλην ἐν τῇ ἁγνείᾳ.
\VS{37}Καὶ κατῴκισεν ἐν αὐτῇ ἄνδρας Ἰουδαίους, καὶ ὠχύρωσεν αὐτὴν πρὸς ἀσφάλειαν τῆς χώρας καὶ τῆς πόλεως, καὶ ὕψωσε τὰ τείχη Ἱερουσαλήμ.
\par }{\PP \VS{38}Καὶ ὁ βασιλεὺς Δημήτριος ἔστησεν αὐτῷ τὴν ἀρχιερωσύνην κατὰ ταῦτα,
\VS{39}καὶ ἐποίησεν αὐτὸν τῶν φίλων αὐτοῦ, καὶ ἐδόξασεν αὐτὸν δόξῃ μεγάλῃ.
\par }{\PP \VS{40}Ἤκουσε γὰρ ὅτι προσηγόρευνται οἱ Ἰουδαῖοι ὑπὸ Ῥωμαίων φίλοι καὶ σύμμαχοι καὶ ἀδελφοὶ, καὶ ὅτι ἀπήντησαν τοῖς πρεσβευταῖς Σίμωνος ἐνδόξως·
\VS{41}καὶ ὅτι εὐδόκησαν οἱ Ἰουδαῖοι, καὶ οἱ ἱερεῖς, τοῦ εἶναι Σίμωνα ἡγουμένον καὶ ἀρχιερέα εἰς τὸν αἰῶνα, ἕως τοῦ ἀναστῆναι προφήτην πιστόν·
\VS{42}καὶ τοῦ εἶναι ἐπʼ αὐτῶν στρατηγόν, καὶ ὅπως μέλοι αὐτῷ περὶ τῶν ἁγίων καθιστάναι αὐτοὺς ἐπὶ τῶν ἔργων αὐτῶν καὶ ἐπὶ τῆς χώρας, καὶ ἐπὶ τῶν ὅπλων, καὶ ἐπὶ τῶν ὀχυρωμάτων·
\VS{43}καὶ ὅπως μέλοι αὐτῷ περὶ τῶν ἁγίων, καὶ ὅπως ἀκούηται ὑπὸ πάντων, καὶ ὅπως γράφωνται ἐπὶ τῷ ὀνόματι αὐτοῦ πᾶσαι συγγραφαὶ ἐν τῇ χώρᾳ, καὶ ὅπως περιβάληται πορφύραν, καὶ χρυσοφορῇ.
\par }{\PP \VS{44}Καὶ οὐκ ἐξέσται οὐδενὶ τοῦ λαοῦ καἰ τῶν ἱερέων ἀθετῆσαί τι τούτων, καὶ ἀντειπεῖν τοῖς ὑπʼ αὐτοῦ ῥηθησομένοις, καὶ ἐπισυστρέψαι συστροφὴν ἐν τῇ χώρᾳ ἄνευ αὐτοῦ, καὶ περιβάλλεσθαι πορφύραν, καὶ ἐμπορποῦσθαι πόρπην χρυσῆν·
\VS{45}Ὃς δʼ ἂν παρὰ ταῦτα ποιήσῃ ἢ ἀθετήσῃ τι τούτων, ἔνοχος ἔσται.
\VS{46}Καὶ εὐδόκησε πᾶς ὁ λαὸς θέσθαι Σίμωνι, καὶ ποιῆσαι κατὰ τοὺς λόγους τούτους.
\VS{47}Καὶ ἐπεδέξατο Σίμων, καὶ εὐδόκησεν ἀρχιερατεύειν, καὶ εἶναι στρατηγὸς καὶ ἐθνάρχης τῶν Ἰουδαίων, καὶ ἱερέων, καὶ τοῦ προστατῆσαι πάντων.
\par }{\PP \VS{48}Καὶ τὴν γραφὴν ταύτην εἶπον θέσθαι ἐν δέλτοις χαλκαῖς, καὶ στῆσαι αὐτὰς ἐν περιβόλῳ τῶν ἁγίων ἐν τόπῳ ἐπισήμῳ,
\VS{49}τὰ δὲ ἀντίγραφα αὐτῶν θέσθαι ἐν τῷ γαζοφυλακίῳ, ὅπως ἕχῃ Σίμων, καὶ οἱ υἱοὶ αὐτοῦ.

\par }\Chap{15}{\PP \VerseOne{1}Καὶ ἀπέστειλεν ὁ Ἀντίοχος υἱὸς Δημητρίου τοῦ βασιλέως ἐπιστολὰς ἀπὸ τῶν νήσων τῆς θαλάσσης Σίμωνι ἱερεῖ καὶ ἐθνάρχῃ τῶν Ἰουδαίων, καὶ παντὶ τῷ ἔθνει.
\VS{2}Καὶ ἦσαν περιέχουσαι τὸν τρόπον τοῦτον· βασιλεὺς Ἀντίοχος Σίμωνι ἱερεῖ μεγάλῳ, καὶ ἐθνάρχῃ, καὶ ἔθνει Ἰουδαίων χαίρειν.
\par }{\PP \VS{3}Ἐπειδὴ ἄνδρες λοιμοὶ κατεκράτησαν τῆς βασιλείας τῶν πατέρων ἡμῶν, βούλομαι δὲ ἀντιποιήσασθαι τῆς βασιλείας, ὅπως ἀποκαταστήσω αὐτὴν ὡς ἦν πρότερον, ἐξενολόγησα δὲ πλῆθος δυνάμεων, καὶ κατεσκεύασα πλοῖα πολεμικά,
\VS{4}βούλομαι δὲ ἐκβῆναι κατὰ τὴν χώραν, ὅπως μετέλθω τοὺς κατεφθαρκότας τὴν χώραν ἡμῶν, καὶ τοὺς ἠρημωκότας πόλεις πολλὰς ἐν τῇ βασιλείᾳ·
\VS{5}νῦν οὗν ἵστημί σοι πάντα τὰ ἀφαιρέματα ἃ ἀφῆκάν σοι οἱ πρὸ ἐμοῦ βασιλεῖς, καὶ ὅσα ἄλλα δόματα ἀφῆκάν σοι.
\par }{\PP \VS{6}Καὶ ἐπέτρεψά σοι ποιῆσαι κόμμα ἴδιον νόμισμα τῇ χώρᾳ σου,
\VS{7}Ἱερουσαλὴμ δὲ καὶ τὰ ἅγια εἶναι ἐλεύθερα· καὶ πάντα τὰ ὅπλα ὅσα κατεσκεύασας, καὶ τὰ ὀχυρώματα ἃ ᾠκοδόμησας, ὧν κρατεῖς, μενέτω σοι.
\VS{8}Καὶ πᾶν ὀφείλημα βασιλικὸν, καὶ τὰ ἐσόμενα βασιλικὰ, ἀπὸ τοῦ νῦν καὶ εἰς τὸν ἅπαντα χρόνον ἀφιέσθω σοι·
\VS{9}ὡς δʼ ἂν κρατήσωμεν τῆς βασιλείας ἡμῶν, δοξάσομέν σε, καὶ τὸ ἔθνος σου, καὶ τὸ ἱερὸν δόξῃ μεγάλῃ, ὥστε φανερὰν γενέσθαι τὴν δόξαν ὑμῶν ἐν πάσῃ τῇ γῇ.
\par }{\PP \VS{10}Ἔτους τετάρτου καὶ ἑβδομηκοστοῦ καὶ ἑκατοστοῦ ἐξῆλθεν Ἀντίοχος εἰς τὴν γῆν πατέρων αὐτοῦ, καὶ συνῆλθον πρὸς αὐτὸν πᾶσαι αἱ δυνάσεις, ὥστε ὀλίγους εἶναι τσὺς καταλειφθέντας σὺν Τρύφωνι.
\par }{\PP \VS{11}Καὶ ἐδίωξεν αὐτὸν Ἀντίοχος ὁ βασιλεύς, καὶ ἦλθε φεύγων εἰς Δωρᾶ τὴν ἐπὶ τῆς θαλάσσης.
\VS{12}Εἶδε γὰρ ὅτι συνῆκται ἐπʼ αὐτὸν τὰ κακὰ, καὶ ἀφῆκαν αὐτὸν αἱ δυνάμεις.
\par }{\PP \VS{13}Καὶ παρενέβαλεν Ἀντιοχος ἐπὶ Δωρᾶ, καὶ σύν αὐτῷ δώδεκα μυριάδες ἀνδρῶν πολεμιστῶν, καὶ ὀκτακισχιλία ἵππος.
\VS{14}Καὶ ἐκύκλωσεν τὴν πόλιν, καὶ τὰ πλοῖα ἀπὸ θαλάσσης συνῆψαν, καὶ ἔθλιβε τὴν πόλιν ἀπὸ τῆς γῆς, καὶ τῆς θαλάσσης, καὶ οὐκ εἴασεν οὐδένα ἐκπορεύεσθαι καὶ εἰσπορεύεσθαι.
\par }{\PP \VS{15}Καὶ ἦλθε Νουμήνιος, καὶ οἱ παρʼ αὐτοῦ, ἐκ Ῥώμης, ἔχοντες ἐπιστολὰς τοῖς βασιλεῦσι, καὶ ταῖς χώραις ἐν αἷς ἐγέγραπτο τάδε·
\par }{\PP \VS{16}Λεύκιος ὕπατος Ῥωμαίων Πτολεμαίῳ βασιλεῖ χαίρειν.
\VS{17}Οἱ πρεσβευταὶ τῶν Ἰουδαίων ἦλθον πρὸς ἡμᾶς φίλοι ἡμῶν, καὶ σύμμαχοι, ἀνανεούμενοι τὴν ἐξ ἀρχῆς φιλίαν καὶ συμμαχίαν, ἀπεσταλμένοι ἀπὸ Σίμωνος τοῦ ἀρχιερέως, καὶ τοῦ δήμου τῶν Ἰουδαίων.
\VS{18}Ἤνεγκαν δὲ ἀσπίδα χρυσῆν ἀπὸ μνῶν χιλίων.
\VS{19}Ἤρεσεν οὖν ἡμῖν γράψαι τοῖς βασιλεῦσι, καὶ ταῖς χώραις, ὅπως μὴ ἐκζητήσωσιν αὐτοῖς, κακὰ καὶ μὴ πολεμήσωσιν αὐτοὺς, καὶ τὰς πόλεις αὐτῶν, καὶ τὴν χώραν αὐτῶν, καὶ ἵνα μὴ συμμαχήσωσι τοῖς πολεμοῦσιν αὐτούς.
\VS{20}Ἔδοξε δὲ ἡμῖν δέξασθαι τὴν ἀσπίδα παρʼ αὐτῶν.
\VS{21}Εἴ τινες οὖν λοιμοὶ διαπεφεύγασιν ἐκ τῆς χώρας αὐτῶν πρὸς ὑμᾶς, παράδοτε αὐτοὺς Σίμωνι τῷ ἀρχιερεῖ, ὅπως ἐκδικήσῃ ἐν αὐτοῖς κατὰ τὸν νόμον αὐτῶν.
\par }{\PP \VS{22}Καὶ τὰ αὐτὰ ἔγραψε Δημητρίῳ τῷ βασιλεῖ, καὶ Ἀττάλῳ, Ἀριαράθῃ, καὶ Ἀρσάκῃ.
\VS{23}Καὶ εἰς πάσας τὰς χώρας, καὶ Σαμψάκῃ, καὶ Σπαρτιάταις, καὶ εἰς Δῆλον, καὶ εἰς Μύνδον, καὶ εἰς Σικυῶνα, καὶ εἰς τὴν Καρίαν, καὶ εἰς Σάμον, καὶ εἰς τὴν Παμφυλίαν, καὶ εἰς τὴν Λυκίαν, καὶ εἰς Ἀλικαρνασσὸν, καὶ εἰς Ῥόδον, καὶ εἰς Φασηλίδα, καὶ εἰς Κῶ, καὶ εἰς Σίδην, καὶ εἰς Ἄραδον, καὶ εἰς Γόρτυναν, καὶ Κνίδον, καὶ Κύπρον, καὶ Κυρήνην.
\VS{24}Τὸ δὲ ἀντίγραφον αὐτῶν ἔγραψαν Σίμωνι τῷ ἀρχιερεῖ.
\par }{\PP \VS{25}Αντιόχος δὲ ὁ βασιλεὺς παρενέβαλεν ἐπὶ Δωρᾶ ἐν τῇ δευτέρᾳ, προσάγων διαπαντὸς αὐτῇ τὰς χεῖρας, καὶ μηχανὰς ποιούμενος, καὶ συνέκλεισε τὸν Τρύφωνα τοῦ μὴ εἰσπορεύεσθαι καὶ ἐκπορεύεσθαι.
\par }{\PP \VS{26}Καὶ ἀπέστειλεν αὐτῷ Σίμων δισχιλίους ἄνδρας ἐκλεκτοὺς συμμαχῆσαι αὐτῷ, καὶ ἀργύριον καὶ χρυσίον, καὶ σκεύη ἱκανά.
\VS{27}Καὶ οὐκ ἠβούλετο αὐτὰ δέξασθαι, ἀλλʼ ἠθέτησε πάντα ὅσα συνέθετο αὐτῷ τοπρότερον, καὶ ἠλλοτριοῦτο αὐτῷ.
\par }{\PP \VS{28}Καὶ ἀπέστειλε πρὸς αὐτὸν Ἀθηνόβιον ἕνα τῶν φίλων αὐτοῦ κοινολογησόμενον αὐτῷ λέγων, ὑμεῖς κατακρατεῖτε τῆς Ἰόππης καὶ Γαζάρων καὶ τῆς ἄκρας τῆς ἐν Ἱερουσαλὴμ, πόλεις τῆς βασιλείας μου.
\VS{29}Τὰ ὅρια αὐτῶν ἠρημώσατε, καὶ ἐποιήσατε πληγὴν μεγάλην ἐπὶ τῆς γῆς, καὶ ἐκυριεύσατε τόπων πολλῶν ἐν τῇ βασιλείᾳ μου.
\VS{30}Νῦν οὖν παράδοτε τὰς πόλεις ἃς κατελάβεσθε, καὶ τοὺς φόρους τῶν τόπων ὧν κατεκυριεύσατε ἐκτὸς τῶν ὁρίων τῆς Ἰουδαίας.
\VS{31}Εἰ δὲ μὴ, δότε ἀντʼ αὐτῶν πεντακόσια τάλαντα ἀργυρίου, καὶ τῆς καταφθορᾶς ἧς κατεφθάρκατε, καὶ τῶν φόρων τῶν πόλεων ἄλλα τάλαντα πεντακόσια· εἰ δὲ δὴ, παραγενόμενοι ἐκπολεμήσομεν ὑμᾶς.
\par }{\PP \VS{32}Καὶ ἦλθεν Ἀθηνόβιος φίλος τοῦ βασιλέως εἰς Ἱερουσαλήμ, καὶ εἶδε τὴν δόξαν Σίμωνος, καὶ κυλικεῖον μετὰ χρυσωμάτων, καὶ ἀργυρωμάτων, καὶ παράστασιν ἱκανὴν, καὶ ἐξίστατο, καὶ ἀπήγγειλεν αὐτῷ τοὺς λόγους τοῦ βασιλέως.
\par }{\PP \VS{33}Καὶ ἀποκριθεὶς Σίμων εἶπεν αὐτῷ, οὔτε γῆν ἀλλοτρίαν εἰλήφαμεν, οὔτε ἀλλοτρίων κεκρατήκαμεν, ἀλλὰ τῆς κληρονομίας τῶν πατέρων ἡμῶν, ὑπὸ δὲ ἐχθρῶν ἡμῶν ἔν τινι καιρῷ ἀκρίτως κατεκρατήθη.
\VS{34}Ἡμεῖς δὲ καιρὸν ἔχοντες ἀντεχόμεθα τῆς κληρονομίας τῶν πατέρων ἡμῶν.
\VS{35}Περὶ δὲ Ἰόππης καὶ Γαζάρων ὧν αἰτεῖς, αὗται ἐποίουν ἐν τῷ λαῷ πληγὴν μεγάλην κατὰ τὴν χώραν ἡμῶν, τούτων δώσομεν τάλαντα ἑκατόν·
\par }{\PP \VS{36}Καὶ οὐκ ἀπεκρίθη αὐτῷ Ἀθηνόβιος λόγον. Ἀπέστρεψε δὲ μετὰ θυμοῦ πρὸς τὸν βασιλέα, καὶ ἀπήγγειλεν αὐτῷ τοὺς λόγους τούτους, καὶ τὴν δόξαν Σίμωνος, καὶ πάντα ὅσα εἶδε· καὶ ὠργίσθη ὁ βασιλεὺς ὀργῆν μεγάλην.
\VS{37}Τρύφων δὲ ἐμβὰς εἰς πλοῖον ἔφυγεν εἰς Ὀρθωσιάδα.
\par }{\PP \VS{38}Καὶ κατέστησεν ὁ βασιλεὺς τὸν Κενδεβαῖον στρατηγὸν τῆς παραλίας, καὶ δυνάμεις πεζικὰς καὶ ἱππικὰς ἔδωκεν αὐτῷ.
\VS{39}Καὶ ἐνετείλατο αὐτῷ παρεμβαλεῖν κατὰ πρόσωπον τῆς Ἰουδαίας· καὶ ἐνετείλατο αὐτῷ οἰκοδομῆσαι τὴν Κεδρὼν, καὶ ὀχυρῶσαι τὰς πύλας, καὶ ὅπως πολεμήσῃ τὸν λαόν· ὁ δὲ βασιλεὺς ἐδίωκε τὸν Τρύφωνα.
\par }{\PP \VS{40}Καὶ παρεγενήθη Κενδεβαῖος εἰς Ἰάμνειαν, καὶ ἤρξατο τοῦ ἐρεθίζειν τὸν λαὸν, καὶ ἐμβατεύειν εἰς τὴν Ἰουδαίαν, καὶ αἰχμαλωτίζειν τὸν λαὸν καὶ φονεύειν.
\VS{41}Καὶ ᾠκοδόμησε τὴν Κεδρών· καὶ ἔταξεν ἐκεῖ ἱππεῖς καὶ δυνάμεις, ὅπως ἐκπορευόμενοι ἐξοδεύωσι τὰς ὁδοὺς τῆς Ἰουδαίας, καθὰ συνεταξεν αὐτῷ ὁ βασιλεύς.

\par }\Chap{16}{\PP \VerseOne{1}Καὶ ἀνέβη Ἰωάννης ἐκ Γαζάρων, καὶ ἀπήγγειλε Σιμωνι τῷ πατρὶ αὐτοῦ ἃ συνετέλει Κενδεβαῖος.
\par }{\PP \VS{2}Καὶ ἐκάλεσε Σίμων τοὺς δύο υἱοὺς αὐτοῦ τοὺς πρεσβυτέρους Ἰούδαν καὶ Ἰωάννην, καὶ εἶπεν αὐτοῖς, ἐγὼ καὶ οἱ ἀδελφοί μου, καὶ ὁ οἶκος τοῦ πατρός μου, ἐπολεμήσαμεν τοὺς πολεμίους Ἰσραὴλ ἀπὸ νεότητος ἕως τῆς σήμερον ἡμέρας, καὶ εὐωδώθη ἐν ταῖς χερσὶν ἡμῶν ῥύσασθαι τὸν Ἰσραὴλ πλεονάκις.
\VS{3}Νῦν δὲ γεγήρακα, καὶ ὑμεῖς δὲ ἐν τῷ ἐλέει ἱκανοί ἐστε ἐν τοῖς ἔτεσι· γίνεσθε ἀντʼ ἐμοῦ, καὶ τοῦ ἀδελφοῦ μου, καὶ ἐξελθόντες ὑπερμαχεῖτε ὑπὲρ τοῦ ἔθνους ἡμῶν, ἡ δὲ ἐκ τοῦ οὐρανοῦ βοήθεια ἔστω μεθʼ ὑμῶν.
\par }{\PP \VS{4}Καὶ ἐπέλεξεν ἐκ τῆς χώρας εἴκοσι χιλιάδας ἀνδρῶν πολεμιστῶν, καὶ ἱππεῖς, καὶ ἐπορεύθωσαν ἐπὶ τὸν Κενδεβαῖον, καὶ ἐκοιμήθησαν ἐν Μωδεΐν.
\par }{\PP \VS{5}Καὶ ἀναστάντες τοπρωῒ ἐπορεύοντο εἰς τὸ πεδίον, καὶ ἰδοὺ δύναμις πολλὴ εἰς συνάντησιν αὐτοῖς πεζικὴ, καὶ ἱππεῖς, καὶ ἦν χειμάῤῥους ἀναμέσον αὐτῶν.
\VS{6}Καὶ παρενέβαλε κατὰ πρόσωπον αὐτῶν αὐτὸς καὶ ὁ λαὸς αὐτοῦ· καὶ εἶδε τὸν λαὸν δειλούμενον διαπερᾶσαι τὸν χειμάῤῥουν, καὶ διεπέρασε πρῶτος, καὶ ἴδον αὐτὸν οἱ ἄνδρες, καὶ διεπέρασαν κατόπισθεν αὐτοῦ·
\VS{7}Καὶ διεῖλε τὸν λαὸν, καὶ τοὺς ἱππεῖς ἐν μέσῳ τῶν πέζῶν· ἡ δὲ ἵππος τῶν ὑπεναντίων πολλὴ σφόδρα.
\par }{\PP \VS{8}Καὶ ἐσάλπισαν ταῖς ἱεραῖς σάλπιγξι, καὶ ἐτροπώθη Κενδεβαῖος καὶ ἡ παρεμβολὴ αὐτοῦ, καὶ ἔπεσον ἐξ αὐτῶν τραυματίαι πολλοί· οἱ δὲ καταλειφθέντες ἔφυγον εἰς τὸ ὀχύρωμα.
\par }{\PP \VS{9}Τότε ἐτραυματίσθη Ἰούδας ὁ ἀδελφὸς Ἰωάννου· Ἰωάννης δὲ κατεδίωξεν αὐτοὺς ἕως ἦλθεν εἰς Κεδρών, ἣν ᾠκοδόμησεν
\VS{10}Καὶ ἔφυγον ἕως εἰς τοὺς πύργους τοὺς ἐν τοῖς ἀγροῖς Ἀζώτου, καὶ ἐνεπύρισεν αὐτὴν ἐν πυρί, καὶ ἔπεσον ἐξ αὐτῶν εἰς ἄνδρας δισχιλίους· καὶ ἀπέστρεψεν εἰς γῆν Ἰούδα μετʼ εἰρήνης.
\par }{\PP \VS{11}Καὶ Πτολεμαῖος ὁ τοῦ Ἀβούβου ἦν καθεσταμένος στρατηγὸς εἰς τὸ πεδίον Ἱεριχὼ, καὶ ἔσχεν ἀργύριον καὶ χρυσίον πολύ·
\VS{12}ἦν γὰρ γαμβρὸς τοῦ ἀρχιερέως.
\VS{13}Καὶ ὑψώθη ἡ καρδία αὐτοῦ, καὶ ἠβουλήθη κατακρατῆσαι τῆς χώρας, καὶ ἐβουλεύετο δόλῳ κατὰ Σίμωνος, καὶ τῶν υἱῶν αὐτοῦ, ἆραι αὐτούς.
\par }{\PP \VS{14}Σίμων δὲ ἦν ἐφοδεύων τὰς πόλεις τὰς ἐν τῇ χώρᾳ, καὶ φροντίζων τῆς ἐπιμελείας αὐτῶν, καὶ κατέβη εἰς Ἱεριχὼ αὐτὸς, καὶ Ματταθίας καὶ Ἰούδας οἱ υἱοὶ αὐτοῦ, ἐτους ἑβδόμου καὶ ἑβδομηκοστοῦ καὶ ἑκατοστοῦ, ἐν μηνὶ ἑνδεκάτῳ, οὗτος ὁ μὴν Σαβάτ.
\VS{15}Καὶ ὑπεδέξατο αὐτοὺς ὁ τοῦ Ἀβούβου εἰς τὸ ὀχυρωμάτιον τὸ καλούμενον Δὼκ, μετὰ δόλου, ὃ ᾠκοδόμησε, καὶ ἐποίησεν αὐτοῖς πότον μέγαν, καὶ ἐνέκρυψεν ἐκεῖ ἄνδρας.
\par }{\PP \VS{16}Καὶ ὅτε ἐμεθύσθη Σίμων καὶ οἱ υἱοὶ αὐτοῦ, ἐξανέστη Πτολεμαῖος καὶ οἱ παρʼ αὐτοῦ, καὶ ἐλάβοσαν τὰ ὅπλα αὐτῶν, καὶ ἐπεισήλθοσαν τῷ Σίμωνι εἰς τὸ συμπόσιον, καὶ ἀπέκτειναν αὐτὸν καὶ τοὺς δύο υἱοὺς αὐτοῦ, καί τινας τῶν παιδαρίων αὐτοῦ.
\VS{17}Καὶ ἐποίησεν ἀθεσίαν μεγάλην, καὶ ἀπέδωκε κατὰ ἀντὶ ἀγαθῶν·
\par }{\PP \VS{18}Καὶ ἔγραψε ταῦτα Πτολεμαῖος, καὶ ἀπέστειλε τῷ βασιλεῖ ὅπως ἀποστείλῃ αὐτῷ δυνάμεις εἰς βοήθειαν, καὶ παραδῷ αὐτῷ τὴν χώραν αὐτῶν, καὶ τὰς πόλεις.
\par }{\PP \VS{19}Καὶ ἀπέστειλεν ἑτέρους εἰς Γάζαρα ἆραι τὸν Ἰωάννην, καὶ τοῖς χιλιάρχοις ἀπέστειλεν ἐπιστολὰς παραγενέσθαι πρὸς αὐτόν, ὅπως δῷ αὐτοῖς ἀργύριον καὶ χρυσίον καὶ δόματα.
\VS{20}Καὶ ἑτέρους ἀπέστειλε καταλαβέσθαι τὴν Ἱερουσαλὴμ, καὶ τὸ ὄρος τοῦ ἱεροῦ.
\par }{\PP \VS{21}Καὶ προδραμών τις ἀπήγγειλεν Ἰωάννῃ εἰς Γάζαρα, ὅτι ἀπώλετο ὁ πατὴρ αὐτοῦ καὶ οἱ ἀδελφοὶ αὐτοῦ, καὶ ὅτι ἀπέσταλκε καὶ σὲ ἀποκτεῖναι.
\VS{22}Καὶ ἀκούσας ἐξέστη σφόδρα· καὶ συνέλαβε τοὺς ἄνδρας τοὺς ἐλθόντας ἀπολέσαι αὐτόν, καὶ ἀπέκτεινεν αὐτοὺς, ἐπέγνω γὰρ ὅτι ἐζήτουν αὐτὸν ἀπολέσαι.
\par }{\PP \VS{23}Καὶ τὰ λοιπὰ τῶν λόγων Ἰωάννου, καὶ τῶν πολέμων αὐτοῦ, καὶ τῶν ἀνδραγαθιῶν αὐτοῦ ὧν ἠνδραγάθησε, καὶ τῆς οἰκοδομῆς τῶν τειχέων ὧν ᾠκοδόμησε, καὶ τῶν πράξεων αὐτοῦ,
\VS{24}ἰδοὺ ταῦτα γέγραπται ἐπὶ βιβλίῳ ἡμερῶν ἀρχιερωσύνης αὐτοῦ, ἀφʼ οὗ ἐγενήθη ἀρχιερεὺς μετὰ τὸν πατέρα αὐτοῦ.
\par }