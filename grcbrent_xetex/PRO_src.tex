\NormalFont\ShortTitle{ΠΑΡΟΙΜΙΑΙ}
{\MT ΠΑΡΟΙΜΙΑΙ ΕΑΛΩΜΩΝΤΟΣ

\par }\ChapOne{1}{\PP \VerseOne{1}ΠΑΡΟΙΜΙΑΙ Σαλωμῶντος υἱοῦ Δαυὶδ, ὃς ἐβασίλευσεν ἐν Ἰσραήλ·
\VS{2}γνῶναι σοφίαν καὶ παιδείαν, νοῆσαί τε λόγους φρονήσεως,
\VS{3}δέξασθαί τε στροφὰς λόγων, νοῆσαί τε δικαιοσύνην ἀληθῆ, καὶ κρίμα κατευθύνειν·
\VS{4}Ἵνα δῷ ἀκάκοις πανουργίαν, παιδὶ δὲ νέῳ αἴσθησίν τε καὶ ἔννοιαν.
\VS{5}Τῶν δὲ γὰρ ἀκούσας σοφὸς σοφώτερος ἔσται, ὁ δὲ νοήμων κυβέρνησιν κτήσεται·
\VS{6}Νοήσει τε παραβολὴν καὶ σκοτεινὸν λόγον, ῥήσεις τε σοφῶν καὶ αἰνίγματα.
\par }{\PP \VS{7}Ἀρχὴ σοφίας φόβος Κυριου, σύνεσις δὲ ἀγαθὴ πᾶσι τοῖς ποιοῦσιν αὐτήν· εὐσέβεια δὲ εἰς Θεὸν ἀρχὴ αἰσθήσεως, σοφίαν δὲ καὶ παιδείαν ἀσεβεῖς ἐξουθενήσουσιν.
\VS{8}Ἄκουε υἱὲ παιδείαν πατρός σου, καὶ μὴ ἀπώσῃ θεσμοὺς μητρός σου.
\VS{9}Στέφανον γὰρ χαρίτων δέξῃ σῇ κορυφῇ, καὶ κλοιὸν χρύσεον περὶ σῷ τραχήλῳ.
\par }{\PP \VS{10}Υἱὲ μή σε πλανήσωσιν ἄνδρες ἀσεβεῖς, μηδὲ βουληθῇς.
\VS{11}Ἐὰν παρακαλέσωσί σε, λέγοντες, ἐλθὲ μεθʼ ἡμῶν, κοινώνησον αἵματος, κρύψωμεν δὲ εἰς γῆν ἄνδρα δίκαιον ἀδίκως,
\VS{12}καταπίωμεν δὲ αὐτὸν ὥσπερ ᾅδης ζῶντα, καὶ ἄρωμεν αὐτοῦ τὴν μνήμην ἐκ γῆς,
\VS{13}τὴν κτῆσιν αὐτοῦ τὴν πολυτελῆ καταλαβώμεθα, πλήσωμεν δὲ οἴκους ἡμετέρους σκύλων·
\VS{14}Τὸν δὲ σὸν κλῆρον βάλε ἐν ἡμῖν, κοινὸν δὲ βαλάντιον κτησώμεθα πάντες, καὶ μαρσίππιον ἓν γενηθήτω ἡμῖν·
\VS{15}Μὴ πορευθῇς ἐν ὁδῷ μετʼ αὐτῶν, ἔκκλινον δὲ τὸν πόδα σου ἐκ τῶν τρίβων αὐτῶν.
\VS{17}Οὐ γὰρ ἀδίκως ἐκτείνεται δίκτυα πτερωτοῖς.
\VS{18}Αὐτοὶ γὰρ οἱ φόνου μετέχοντες, θησαυρίζουσιν ἑαυτοῖς κακά· ἡ δὲ καταστροφὴ ἀνδρῶν παρανόμων κακή.
\VS{19}Αὗται αἱ ὁδοί εἰσι πάντων τῶν συντελούντων τὰ ἄνομα· τῇ γὰρ ἀσεβείᾳ τὴν ἑαυτῶν ψυχὴν ἀφαιροῦνται.
\par }{\PP \VS{20}Σοφία ἐν ἐξόδοις ὑμνεῖται, ἐν δὲ πλατείαις παῤῥησίαν ἄγει.
\VS{21}Ἐπʼ ἄκρων δὲ τειχέων κηρύσσεται, ἐπὶ δὲ πύλαις δυναστῶν παρεδρεύει, ἐπὶ δὲ πύλαις πόλεως θαῤῥοῦσα λέγει,
\VS{22}ὅσον ἂν χρόνον ἄκακοι ἔχονται τῆς δικαιοσύνης, οὐκ αἰσχυνθήσονται· οἱ δὲ ἄφρονες τῆς ὕβρεως ὄντες ἐπιθυμηταί, ἀσεβεῖς γενόμενοι ἐμίσησαν αἴσθησιν,
\VS{23}καὶ ὑπεύθυνοι ἐγένοντο ἐλέγχοις· ἰδοὺ προήσομαι ὑμῖν ἐμῆς πνοῆς ῥῆσιν· διδάξω δὲ ὑμᾶς τὸν ἐμὸν λόγον.
\par }{\PP \VS{24}Ἐπειδὴ ἐκάλουνμ, καὶ οὐχ ὑπηκούσατε· καὶ ἐξέτεινον λόγους, καὶ οὐ προσείχετε·
\VS{25}ἀλλὰ ἀκύρους ἐποιεῖτε ἐμὰς βουλὰς, τοῖς δὲ ἐμοῖς ἐλέγχοις ἠπειθήσατε·
\VS{26}Τοιγαροῦν κᾀγὼ τῇ ὑμετέρᾳ ἀπωλείᾳ ἐπιγελάσομαι, καταχαροῦμαι δὲ ἡνίκα ἔρχηται ὑμῖν ὄλεθρος·
\VS{27}Καὶ ὡς ἂν ἀφίκηται ὑμῖν ἄφνω θόρυβος, ἡ δὲ καταστροφὴ ὁμοίως καταιγίδι παρῇ, καὶ ὅταν ἔρχηται ὑμῖν θλίψις καὶ πολιορκία, ἢ ὅταν ἔρχηται ὑμῖν ὄλεθρος.
\VS{28}Ἔσται γὰρ ὅταν ἐπικαλέσησθέ με, ἐγὼ δὲ οὐκ εἰσακούσομαι ὑμῶν· ζητήσουσί με κακοὶ, καὶ οὐχ εὑρήσουσιν.
\VS{29}Ἐμίσησαν γὰρ σοφίαν, τὸν δὲ λόγον τοῦ Κυρίου οὐ προείλαντο,
\VS{30}οὐδὲ ἤθελον ἐμαῖς προσέχειν βουλαῖς, ἐμυκτήριζον δὲ ἐμοὺς ἐλέγχους·
\VS{31}Τοιγαροῦν ἔδονται τῆς ἑαυτῶν ὁδοῦ τοὺς καρποὺς, καὶ τῆς ἐαυτῶν ἀσεβείας πλησθήσονται.
\VS{32}Ἀνθʼ ὧν γὰρ ἠδίκουν νηπίους, φονευθήσονται, καὶ ἐξετασμὸς ἀσεβεῖς ὀλεῖ.
\VS{33}Ὁ δὲ ἐμοῦ ἀκούων κατασκηνώσει ἐπʼ ἐλπίδι, καὶ ἡσυχάσει ἀφόβως ἀπὸ παντὸς κακοῦ.

\par }\Chap{2}{\PP \VerseOne{1}Υἱὲ, ἐὰν δεξάμενος ῥῆσιν ἐμῆς ἐντολῆς κρύψῃς παρὰ σεαυτῷ,
\VS{2}ὑπακούσεται σοφίας τὸ οὖς σου, καὶ παραβαλεῖς καρδίαν σου εἰς σύνεσιν, παραβαλεῖς δὲ αὐτὴν ἐπὶ νουθέτησιν τῷ υἱῷ σου·
\par }{\PP \VS{3}Ἐὰν γὰρ τὴν σοφίαν ἐπικαλέσῃ, καὶ τῇ συνέσει δῷς φωνήν σου,
\VS{4}καὶ ἐὰν ζητήσῃς αὐτὴν ὡς ἀργύριον, καὶ ὡς θησευροὺς ἐξεραυνήσῃς αὐτήν·
\VS{5}Τότε συνήσεις φόβον Κυρίου, καὶ ἐπίγνωσιν Θεοῦ εὑρήσεις.
\par }{\PP \VS{6}Ὅτι Κύριος δίδωσι σοφίαν, καὶ ἀπὸ προσώπου αὐτοῦ γνῶσις καὶ σύνεσις.
\VS{7}Καὶ θησαυρίζει τοῖς κατορθοῦσι σωτηρίαν, ὑπερασπιεῖ τὴν πορείαν αὐτῶν,
\VS{8}τοῦ φυλάξαι ὁδοὺς δικαιωμάτων, καὶ ὁδὸν εὐλαβουμένων αὐτὸν διαφυλάξει.
\VS{9}Τότε συνήσεις δικαιοσύνην καὶ κρίμα, καὶ κατορθώσεις πάντας ἄξονας ἀγαθούς.
\par }{\PP \VS{10}Ἐὰν γὰρ ἔλθῃ ἡ σοφία εἰς σὴν διάνοιαν, ἡ δὲ αἴσθησις τῇ σῇ ψυχῇ καλὴ εἶναι δόξῃ,
\VS{11}βουλὴ καλὴ φυλάξει σε, ἔννοια δὲ ὁσία τηρήσει σε·
\VS{12}Ἵνα ῥύσηταί σε ἀπὸ ὁδοῦ κακῆς, καὶ ἀπὸ ἀνδρὸς λαλοῦντος μηδὲν πιστόν.
\par }{\PP \VS{13}Ὦ οἱ ἐγκαταλείποντες ὁδοὺς εὐθείας τοῦ πορεύεσθαι ἐν ὁδοῖς σκότους·
\VS{14}Οἱ εὐφραινόμενοι ἐπὶ κακοῖς καὶ χαίροντες ἐπὶ διαστροφῇ κακῇ·
\VS{15}Ὧν αἱ τρίβοι σκολιαὶ, καὶ καμπύλαι αἱ τροχιαὶ αὐτῶν,
\VS{16}τοῦ μακράν σε ποιῆσαι ἀπὸ ὁδοῦ εὐθείας, καὶ ἀλλότριον τῆς δικαίας γνώμης· υἱὲ, μή σε καταλάβῃ κακὴ βουλή·
\VS{17}Ἡ ἀπολιποῦσα διδασκαλίαν νεότητος, καὶ διαθήκην θείαν ἐπιλελησμένη.
\VS{18}Ἔθετο γὰρ παρὰ τῷ θανάτῳ τὸν οἶκον αὐτῆς, καὶ παρὰ τῷ ᾅδῃ μετὰ τῶν γηγενῶν τοὺς ἄξονας αὐτῆς.
\VS{19}Πάντες οἱ πορευόμενοι ἐν αὐτῇ οὐκ ἀναστρέψουσιν, οὐδὲ μὴ καταλάβωσι τρίβους εὐθείας· οὐ γὰρ καταλαμβάνονται ὑπὸ ἐνιαυτῶν ζωῆς.
\VS{20}Εἰ γὰρ ἐπορεύοντο τρίβους ἀγαθὰς, εὕροσαν ἂν τρίβους δικαιοσύνης λείας.
\VS{21}Ὅτι εὐθεῖς κατασκηνώσουσι γῆν, καὶ ὅσιοι ὑπολειφθήσονται ἐν αὐτῇ.
\VS{22}Ὁδοὶ ἀσεβῶν ἐκ γῆς ὀλοῦνται, οἱ δὲ παράνομοι ἐξωσθήσονται ἀπʼ αὐτῆς.

\par }\Chap{3}{\PP \VerseOne{1}Υἱὲ, ἐμῶννομίμων μὴ ἐπιλανθάνου, τὰ δὲ ῥήματά μου τηρείτω σὴ καρδία·
\VS{2}Μῆκος γὰρ βίου, καὶ ἔτη ζωῆς, καὶ εἰρήνην προσθήσουσί σοι.
\VS{3}Ἐλεημοσύναι καὶ πίστεις μὴ ἐκλειπέτωσάν σε· ἄφαψαι δὲ αὐτὰς ἐπὶ σῷ τραχήλῳ, καὶ εὑρήσεις χάριν·
\VS{4}καὶ προνοοῦ καλὰ ἐνώπιον Κυρίου καὶ ἀνθρώπων.
\par }{\PP \VS{5}Ἴσθι πεποιθὼς ἐν ὅλῃ τῇ καρδίᾳ ἐπὶ Θεῷ, ἐπὶ δὲ σῇ σοφίᾳ μὴ ἐπαίρου.
\VS{6}Πάσαις ὁδοῖς σου γνώριζε αὐτὴν, ἵνα ὀρθοτομῇ τὰς ὁδούς σου.
\VS{7}Μὴ ἴσθι φρόνιμος παρὰ σεαυτῷ, φοβοῦ δὲ τὸν Θεὸν, καὶ ἔκκλινε ἀπὸ παντὸς κακοῦ.
\VS{8}Τότε ἴασις ἔσται τῷ σώματί σου, καὶ ἐπιμέλεια τοῖς ὀστέοις σου.
\par }{\PP \VS{9}Τίμα τὸν Κύριον ἀπὸ σῶν δικαίων πόνων, καὶ ἀπάρχου αὐτῷ ἀπὸ σῶν καρπῶν δικαιοσύνης·
\VS{10}Ἵνα πίμπληται τὰ ταμιεῖά σου πλησμονῆς σίτῳ, οἴνῳ δὲ αἱ ληνοί σου ἐκβλύζωσιν.
\par }{\PP \VS{11}Υἱὲ, μὴ ὀλιγώρει παιδείας Κυρίου, μηδὲ ἐκλύου ὑπʼ αὐτοῦ ἐλεγχόμενος.
\VS{12}Ὃν γὰρ ἀγαπᾷ Κύριος, ἐλέγχει, μαστιγοῖ δὲ πάντα υἱὸν ὃν παραδέχεται.
\par }{\PP \VS{13}Μακάριος ἄνθρωπος ὃς εὗρε σοφίαν, καὶ θνητὸς ὃς εἶδε φρόνησιν.
\VS{14}Κρεῖσσον γὰρ αὐτὴν ἐμπορεύεσθαι, ἢ χρυσίου καὶ ἀργυρίου θησαυρούς.
\VS{15}Τιμιωτέρα δέ ἐστι λίθων πολυτελῶν, οὐκ ἀντιτάξεται αὐτῇ οὐδὲν πονηρόν· εὔγνωστός ἐστι πᾶσι τοῖς ἐγγίζουσιν αὐτῇ, πᾶν δὲ τίμιον οὐκ ἄξιον αὐτῆς ἐστι.
\VS{16}Μῆκος γὰρ βίου καὶ ἔτη ζωῆς ἐν τῇ δεξιᾷ αὐτῆς, ἐν δὲ τῇ ἀριστερᾷ αὐτῆς πλοῦτος καὶ δόξα·
\VS{16a}ἐκ τοῦ στόματος αὐτῆς ἐκπορεύεται δικαιοσύνη, νόμον δὲ καὶ ἔλεον ἐπὶ γλώσσης φορεῖ.
\VS{17}Αἱ ὁδοὶ αὐτῆς ὁδοὶ καλαί, καὶ πάσαι αἱ τρίβοι αὐτῆς ἐν εἰρήνῃ.
\VS{18}Ξύλον ζωῆς ἐστι πᾶσι τοῖς ἀντεχομένοις αὐτῆς, καὶ τοῖς ἐπερειδομένοις ἐπʼ αὐτὴν ὡς ἐπὶ Κύριον ἀσφαλής.
\par }{\PP \VS{19}Ὁ Θεὸς τῇ σοφίᾳ ἐθεμελίωσε τὴν γῆν, ἡτοίμασε δὲ οὐρανοὺς φρονήσει.
\VS{20}Ἐν αἰσθήσει ἄβυσσοι ἐῤῥάγησαν, νέφη δὲ ἐῤῥύησαν δρόσους.
\par }{\PP \VS{21}Υἱὲ, μὴ παραῤῥυῇς, τήρησον δὲ ἐμὴν βουλὴν καὶ ἔννοιαν·
\VS{22}ἵνα ζήσῃ ἡ ψυχή σου, καὶ χάρις ᾖ περὶ σῷ τραχήλῳ·
\VS{22a}ἔσται δὲ ἴασις ταῖς σαρξί σου, καὶ ἐπιμέλεια τοῖς σοῖς ὀστέοις·
\VS{23}ἵνα πορεύῃ πεποιθὼς ἐν εἰρήνῃ πάσας τὰς ὁδούς σου, ὁ δὲ πούς σου οὐ μὴ προσκόψῃ.
\VS{24}Ἐὰν γὰρ κάθῃ, ἄφοβος ἔσῃ· ἐὰν δὲ καθεύδῃς, ἡδέως ὑπνώσεις.
\VS{25}Καὶ οὐ φοβηθήσῃ πτόησιν ἐπελθοῦσαν, οὐδὲ ὁρμὰς ἀσεβῶν ἐπερχομένας.
\VS{26}Ὁ γὰρ Κύριος ἔσται ἐπὶ πασῶν ὁδῶν σου, καὶ ἐρείσει σὸν πόδα ἵνα μὴ σαλευθῇς.
\par }{\PP \VS{27}Μὴ ἀπόσχῃ εὖ ποιεῖν ἐνδεῆ, ἡνίκα ἂν ἔχῃ ἡ χείρ σου βοηθεῖν.
\VS{28}Μὴ εἴπῃς, ἐπανελθὼν ἐπάνηκε, αὔριον δώσω, δυνατοῦ σου ὄντος εὖ ποιεῖν· οὐ γὰρ οἶδας τί τέξεται ἡ ἐπιοῦσα.
\VS{29}Μὴ τεκτῄνῃ ἐπὶ σὸν φίλον κακὰ παροικοῦντα καὶ πεποιθότα ἐπὶ σοί.
\par }{\PP \VS{30}Μὴ φιλεχθρήσῃς πρὸς ἄνθρωπον μάτην, μήτί σε ἐργάσηται κακόν.
\par }{\PP \VS{31}Μὴ κτήσῃ κακῶν ἀνδρῶν ὀνείδη, μηδὲ ζηλώσῃς τὰς ὁδοὺς αὐτῶν.
\VS{32}Ἀκάθαρτος γὰρ ἔναντι Κυρίου πᾶς παράνομος, ἐν δὲ δικαίοις οὐ συνεδριάζει.
\VS{33}Κατάρα Θεοῦ ἐν οἴκοις ἀσεβῶν, ἐπαύλεις δὲ δικαίων εὐλογοῦνται.
\VS{34}Κύριος ὑπερηφάνοις ἀντιτάσσεται, ταπεινοῖς δὲ δίδωσι χάριν.
\VS{35}Δόξαν σοφοὶ κληρονομήσουσιν, οἱ δὲ ἀσεβεῖς ὕψωσαν ἀτιμίαν.

\par }\Chap{4}{\PP \VerseOne{1}Ἀκούσατε, παῖδες, παιδείαν πατρὸς, καὶ προσέχετε γνῶναι ἔννοιαν.
\VS{2}Δῶρον γὰρ ἀγαθὸν δωροῦμαι ὑμῖν, τὸν ἐμὸν νόμον μὴ ἐγκαταλίπητε.
\VS{3}Υἱὸς γὰρ ἐγενόμην κᾀγὼ πατρὶ ὑπήκοος, καὶ ἀγαπώμενος ἐν προσώπῳ μητρός.
\VS{4}Οἳ ἔλεγον καὶ ἐδίδασκόν με, ἐρειδέτω ὁ ἡμέτερος λόγος εἰς σὴν καρδίαν· φύλασσε ἐντολὰς,
\VS{5}μὴ ἐπιλάθῃ· Μηδὲ παρίδῃς ῥῆσιν ἐμοῦ στόματος,
\VS{6}μηδὲ ἐγκαταλίπῃς αὐτὴν, καὶ ἀνθέξεταί σου· ἐράσθητι αὐτῆς, καὶ τηρήσει σε.
\VS{8}Περιχαράκωσον αὐτὴν, καὶ ὑψώσει σε· τίμησον αὐτὴν, ἵνα σε περιλάβῃ·
\VS{9}Ἵνα δῷ τῇ σῇ κεφαλῇ στέφανον χαρίτων, στεφάνῳ δὲ τρυφῆς ὑπερασπίσῃ σου.
\par }{\PP \VS{10}Ἄκουε υἱὲ καὶ δέξαι ἐμοὺς λόγους, καὶ πληθυνθήσεται ἔτη ζωῆς σου, ἵνα σοι γένωνται πολλαὶ ὁδοὶ βίου.
\VS{11}Ὁδοὺς γὰρ σοφίας διδάσκω σε, ἐμβιβάζω δέ σε τροχιαῖς ὀρθαῖς.
\VS{12}Ἐὰν γὰρ πορεύῃ, οὐ συγκλεισθήσεταί σου τὰ διαβήματα· ἐὰν δὲ τρέχῃς, οὐ κοπιάσεις.
\VS{13}Ἐπιλαβοῦ ἐμῆς παιδείας, μὴ ἀφῇς, ἀλλὰ φύλαξον αὐτὴν σεαυτῷ εἰς ζωήν σου.
\par }{\PP \VS{14}Ὁδοὺς ἀσεβῶν μὴ ἐπέλθῃς, μηδὲ ζηλώσῃς ὁδοὺς παρανόμων.
\VS{15}Ἐν ᾧ ἂν τόπῳ στρατοπεδεύσωσι, μὴ ἐπέλθῃς ἐκεὶ, ἔκκλινον δὲ ἀπʼ αὐτῶν καὶ παράλλαξον.
\VS{16}Οὐ γὰρ μὴ ὑπνώσωσιν, ἐὰν μὴ κακοποιήσωσιν· ἀφῄρηται ὁ ὕπνος αὐτῶν, καὶ οὐ κοιμῶνται.
\VS{17}Οἵδε γὰρ σιτοῦνται σῖτα ἀσεβείας, οἴνῳ δὲ παρανόμῳ μεθύσκονται.
\VS{18}Αἱ δὲ ὁδοὶ τῶν δικαίων ὁμοίως φωτὶ λάμπουσι, προπορεύονται καὶ φωτίζουσιν, ἕως κατορθώσῃ ἡ ἡμέρα.
\VS{19}Αἱ δὲ ὁδοὶ τῶν ἀσεβῶν σκοτειναὶ, οὐκ οἴδασι πῶς προσκόπτουσιν.
\par }{\PP \VS{20}Υἱὲ ἐμῇ ῥήσει πρόσεχε, τοῖς δὲ ἐμοῖς λόγοις παράβαλλε σὸν οὖς.
\VS{21}Ὅπως μὴ ἐκλίπωσί σε αἱ πηγαί σου, φύλασσε αὐτὰς ἐν καρδίᾳ.
\VS{22}Ζωὴ γάρ ἐστι τοῖς εὑρίσκουσιν αὐτὰς, καὶ πάσῃ σαρκὶ ἴασις.
\VS{23}Πάσῃ φυλακῇ τήρει σὴν καρδίαν, ἐκ γὰρ τούτων ἔξοδοι ζωῆς.
\VS{24}Περίελε σεαυτοῦ σκολιὸν στόμα, καὶ ἄδικα χείλη μακρὰν ἀπὸ σοῦ ἄπωσαι.
\VS{25}Οἱ ὀφθαλμοί σου ὀρθὰ βλεπέτωσαν, τὰ δὲ βλέφαρά σου νευέτω δίκαια.
\VS{26}Ὀρθὰς τροχιὰς ποίει σοῖς ποσί, καὶ τὰς ὁδούς σου κατεύθυνε.
\VS{27}Μὴ ἐκκλίνῃς εἰς τὰ δεξιὰ, μηδὲ εἰς τὰ ἀριστερά, ἀπόστρεψον δὲ σὸν πόδα ἀπὸ ὁδοῦ κακῆς·
\VS{27a}ὁδοὺς γὰρ τὰς ἐκ δεξιῶν οἶδεν ὁ Θεὸς, διεστραμμέναι δέ εἰσιν αἱ ἐξ ἀριστερῶν·
\VS{27b}αὐτὸς δὲ ὀρθὰς ποιήσει τὰς τροχιάς σου, τὰς δὲ πορείας σου ἐν εἰρήνῃ προάξει.

\par }\Chap{5}{\PP \VerseOne{1}Υἱὲ, ἐμῇ σοφίᾳ πρόσεχε, ἐμοῖς δὲ λόγοις παράβαλλε σὸν οὖς,
\VS{2}ἵνα φυλάξῃς ἔννοιαν ἀγαθήν· αἴσθησις δὲ ἐμῶν χειλέων ἐντέλλεταί σοι·
\par }{\PP \VS{3}Μὴ πρόσεχε φαύλῃ γυναικί. Μέλι γὰρ ἀποστάζει ἀπὸ χειλέων γυναικὸς πόρνης, ἣ πρὸς καιρὸν λιπαίνει σὸν φάρυγγα,
\VS{4}ὕστερον μέντοι πικρότερον χολῆς εὑρήσεις, καὶ ἠκονημένον μᾶλλον μαχαίρας διστόμου.
\VS{5}Τῆς γὰρ ἀφροσύνης οἱ πόδες κατάγουσι τοὺς χρωμένους αὐτῇ μετὰ θανάτου εἰς τὸν ᾅδην, τὰ δὲ ἴχνη αὐτῆς οὐκ ἐρείδεται.
\VS{6}Ὁδοὺς γὰρ ζωῆς οὐκ ἐπέρχεται, σφαλεραὶ δὲ αἱ τροχιαὶ αὐτῆς, καὶ οὐκ εὔγνωστοι.
\par }{\PP \VS{7}Νῦν οὖν υἱὲ ἄκουέ μου, καὶ μὴ ἀκύρους ποιήσεις ἐμοὺς λόγους.
\VS{8}Μακρὰν ποίησον ἀπʼ αὐτῆς σὴν ὁδόν· μὴ ἐγγίσῃς πρὸς θύραις οἴκων αὐτῆς,
\VS{9}ἵνα μὴ πρόῃ ἄλλοις ζωήν σου, καὶ σὸν βίον ἀνελεήμοσιν·
\VS{10}Ἵνα μὴ πλησθῶσιν ἀλλότριοι σῆς ἰσχύος, οἱ δὲ σοὶ πόνοι εἰς οἴκους ἀλλοτρίων ἔλθεσι·
\VS{11}Καὶ μεταμεληθήσῃ ἐπʼ ἐσχάτων, ἡνίκα ἂν κατατριβῶσι σάρκες σώματός σου,
\VS{12}καὶ ἐρεῖς, πῶς ἐμίσησα παιδείαν, καὶ ἐλέγχους ἐξέκλινεν ἡ καρδία μου;
\VS{13}Οὐκ ἤκουον φωνὴν παιδεύοντός με καὶ διδάσκοντός με, οὐδὲ παρέβαλλον τὸ οὖς μου.
\VS{14}Παρʼ ὀλίγον ἐγενόμην ἐν παντὶ κακῷ, ἐν μέσῳ ἐκκλησίας καὶ συναγωγῆς.
\par }{\PP \VS{15}Πίνε ὕδατα ἀπὸ σῶν ἀγγείων, καὶ ἀπὸ σῶν φρεάτων πηγῆς.
\VS{16}Μὴ ὑπερεκχείσθω σοι ὕδατα ἐκ τῆς σῆς πηγῆς, εἰς δὲ σὰς πλατείας διαπορευέσθω τὰ σὰ ὕδατα.
\VS{17}Ἔστω σοι μόνῳ ὑπάρχοντα, καὶ μηδεὶς ἀλλότριος μετασχέτω σοι.
\VS{18}Ἡ πηγή σου τοῦ ὕδατος ἔστω σοι ἰδία, καὶ συνευφραίνου μετὰ γυναικὸς τῆς ἐκ νεότητός σου.
\VS{19}Ἔλαφος φιλίας καὶ πῶλος σῶν χαρίτων ὁμιλείτω σοι, ἡ δὲ ἰδία ἡγείσθω σου καὶ συνέστω σοι ἐν παντὶ καιρῷ· ἐν γὰρ τῇ ταύτης φιλίᾳ συμπεριφερόμενος, πολλοστὸς ἔσῃ.
\VS{20}Μὴ πολὺς ἴσθι πρὸς ἀλλοτρίαν, μηδὲ συνέχου ἀγκάλαις τῆς μὴ ἰδίας.
\VS{21}Ἐνώπιον γάρ εἰσι τῶν τοῦ Θεοῦ ὀφθαλμῶν ὁδοὶ ἀνδρὸς, εἰς δὲ πάσας τὰς τροχιὰς αὐτοῦ σκοπεύει.
\VS{22}Παρανομίαι ἄνδρα ἀγρεύουσι, σειραῖς δὲ τῶν ἑαυτοῦ ἁμαρτιῶν ἕκαστος σφίγγεται.
\VS{23}Οὗτος τελευτᾷ μετὰ ἀπαιδεύτων, ἐκ δὲ πλήθους τῆς ἑαυτοῦ βιότητος ἐξεῤῥίφη, καὶ ἀπώλετο διʼ ἀφροσύνην.

\par }\Chap{6}{\PP \VerseOne{1}Υἱὲ, ἐὰν ἐγγυήσῃ σὸν φίλον, παραδώσεις σὴν χεῖρα ἐχθρῷ.
\VS{2}Παγὶς γὰρ ἰσχυρὰ ἀνδρὶ τὰ ἴδια χείλη, καὶ ἁλίσκεται χείλεσιν ἰδίου στόματος.
\VS{3}Ποίει υἱὲ ἃ ἐγώ σοι ἐντέλλομαι, καὶ σώζου· ἥκεις γὰρ εἰς χεῖρας κακῶν διὰ σὸν φίλον· ἴσθι μὴ ἐκλυόμενος, παρόξυνε δὲ καὶ τὸν φίλον σου ὃν ἐνεγγυήσω.
\VS{4}Μὴ δῷς ὕπνον σοῖς ὄμμασι, μηδὲ ἐπινυστάξῃς σοῖς βλεφάροις,
\VS{5}ἵνα σώζῃ ὥσπερ δορκὰς ἐκ βρόχων, καὶ ὥσπερ ὄρνεον ἐκ παγίδος.
\par }{\PP \VS{6}Ἴθι πρὸς τὸν μύρμηκα ὦ ὀκνηρὲ, καὶ ζήλωσον ἰδὼν τὰς ὁδοὺς αὐτοῦ, καὶ γενοῦ ἐκείνου σοφώτερος.
\par }{\PP \VS{7}Ἐκείνῳ γὰρ γεωργίου μὴ ὑπάρχοντος, μηδὲ τὸν ἀναγκάζοντα ἔχων, μηδὲ ὑπὸ δεσπότην ὢν,
\VS{8}ἐτοιμάζεται θέρους τὴν τροφὴν, πολλήν τε ἐν τῷ ἀμητῷ ποιεῖται τὴν παράθεσιν·
\VS{8a}ἢ πορεύθητι πρὸς τὴν μέλισσαν, καὶ μάθε ὡς ἐργάτις ἐστὶ, τήν τε ἐργασίαν ὡς σεμνὴν ποιεῖται·
\VS{8b}ἧς τοὺς πόνους βασιλεῖς καὶ ἰδιῶται πρὸς ὑγίειαν προσφέρονται· ποθεινὴ δέ ἐστι πᾶσι καὶ ἐπίδοξος,
\VS{8c}καίπερ οὖσα τῇ ῥώμῃ ἀσθενὴς, τὴν σοφίαν τιμήσασα προήχθη.
\VS{9}Ἕως τίνος ὀκνηρὲ κατάκεισαι; πότε δὲ ἐξ ὕπνου ἐγερθήσῃ;
\VS{10}ὀλίγον μὲν ὑπνοῖς, ὀλίγον δὲ κάθησαι, μικρὸν δὲ νυστάζεις, ὀλίγον δὲ ἐναγκαλίζῃ χερσὶ στήθη.
\VS{11}Εἶτʼ ἐνπαραγίνεταί σοι ὥσπερ κακὸς ὁδοιπόρος ἡ πενία, καὶ ἡ ἔνδεια ὥσπερ ἀγαθὸς δρομεύς·
\VS{11a}ἐὰν δὲ ἄοκνος ᾖς, ἥξει ὥσπερ πηγὴ ὁ ἀμητός σου· ἡ δὲ ἔνδεια, ὥσπερ κακὸς δρομεὺς ἀπαυτομολήσει.
\par }{\PP \VS{12}Ἀνὴρ ἄφρων καὶ παράνομος πορεύεται ὁδοὺς οὐκ ἀγαθάς.
\VS{13}Ὁ δʼ αὐτὸς ἐννεύει ὀφθαλμῷ, σημαίνει δὲ ποδὶ, διδάσκει δὲ ἐννεύμασι δακτύλων.
\VS{14}Διεστραμμένη καρδία τεκταίνεται κακὰ, ἐν παντὶ καιρῷ ὁ τοιοῦτος ταραχὰς συνίστησιν πόλει.
\VS{15}Διὰ τοῦτο ἐξαπίνης ἔρχεται ἡ ἀπώλεια αὐτοῦ, διακοπὴ καὶ συντριβὴ ἀνίατος.
\par }{\PP \VS{16}Ὅτι χαίρει πᾶσιν οἷς μισεῖ ὁ Θεὸς, συντρίβεται δὲ διʼ ἀκαθαρσίαν ψυχῆς.
\VS{17}Ὀφθαλμὸς ὑβριστοῦ, γλῶσσα ἄδικος· χεῖρες ἐκχέουσαι αἷμα δικαίου,
\VS{18}καὶ καρδία τεκταινομένη λογισμοὺς κακοὺς, καὶ πόδες ἐπισπεύδοντες κακοποιεῖν.
\VS{19}Ἐκκαίει ψευδῆ μάρτυς ἄδικος, καὶ ἐπιπέμπει κρίσεις ἀναμέσον ἀδελφῶν.
\par }{\PP \VS{20}Υἱὲ, φύλασσε νόμους πατρός σου, καὶ μὴ ἀπώσῃ θεσμοὺς μητρός σου·
\VS{21}Ἄφαψαι δὲ αὐτοὺς ἐπὶ σῇ ψυχῇ διαπαντὸς, καὶ ἐλκλοίωσαι περὶ σῷ τραχήλῳ·
\VS{22}Ἡνίκα ἂν περιπατῇς, ἐπάγου αὐτὴν καὶ μετὰ σοῦ ἔστω, ὡς δʼ ἂν καθεύδῃς φυλασσέτω σε, ἵνα ἐγειρομένῳ συλλαλῇ σοι.
\VS{23}Ὅτι λύχνος ἐντολὴ νόμου καὶ φῶς, ὁδὸς ζωῆς, καὶ ἔλεγχος καὶ παιδεία,
\VS{24}τοῦ διαφυλάσσειν σε ἀπὸ γυναικὸς ὑπάνδρου, καὶ ἀπὸ διαβολῆς γλώσσης ἀλλοτρίας.
\par }{\PP \VS{25}Μή σε νικήσῃ κάλλους ἐπιθυμία, μηδὲ ἀγρευθῇς σοῖς ὀφθαλμοῖς, μηδὲ συναρπασθῇς ἀπὸ τῶν αὐτῆς βλεφάρων.
\VS{26}Τιμὴ γὰρ πόρνης ὅση καὶ ἑνὸς ἄρτου, γυνὴ δὲ ἀνδρῶν τιμίας ψυχὰς ἀγρεύει.
\VS{27}Ἀποδήσει τις πῦρ ἐν κόλπῳ, τὰ δὲ ἱμάτια οὐ κατακαύσει;
\VS{28}ἢ περιπατήσει τις ἐπʼ ἀνθράκων πυρὸς, τοὺς δὲ πόδας οὐ κατακαύσει;
\VS{29}Οὕτως ὁ εἰσελθὼν πρὸς γυναῖκα ὕπανδρον, οὐκ ἀθωωθήσεται, οὐδὲ πᾶς ὁ ἁπτόμενος αὐτῆς.
\VS{30}Οὐ θαυμαστὸν ἐὰν ἁλῷ τις κλέπτων, κλέπτει γὰρ ἵνα ἐμπλήσῃ τὴν ψυχὴν πεινῶν.
\VS{31}Ἐὰν δὲ ἁλῷ, ἀποτίσει ἑπταπλάσια, καὶ πάντα τὰ ὑπάρχοντα αὐτοῦ δοὺς ῥύσεται ἑσυτόν.
\VS{32}Ὁ δὲ μοιχὸς διʼ ἔνδειαν φρενῶν ἀπώλειαν τῇ ψυχῇ αὐτοῦ περιποιεῖται,
\VS{33}ὀδύνας τε καὶ ἀτιμίας ὑποφέρει, τὸ δὲ ὄνειδος αὐτοῦ οὐκ ἐξαλειφθήσεται εἰς τὸν αἰῶνα.
\VS{34}Μεστὸς γὰρ ζήλου θυμὸς ἀνδρὸς αὐτῆς, οὐ φείσεται ἐν ἡμέρᾳ κρίσεως.
\VS{35}Οὐκ ἀνταλλάξεται οὐδενὸς λύτρου τὴν ἔχθραν, οὐδὲ μὴ διαλυθῇ πολλῶν δώρων.

\par }\Chap{7}{\PP \VerseOne{1}Υἱὲ φύλασσε ἐμοὺς λόγους, τὰς δὲ ἐμὰς ἐντολὰς κρύψον παρὰ σεαυτῷ·
\VS{1a}Υἱὲ τίμα τὸν Κύριον καὶ ἰσχύσεις, πλὴν δὲ αὐτοῦ μὴ φοβοῦ ἄλλον·
\VS{2}φύλαξον ἐμὰς ἐντολὰς καὶ βιώσεις, τοὺς δὲ ἐμοὺς λόγους ὥσπερ κόρας ὀμμάτων.
\VS{3}Περίθου δὲ αὐτοὺς σοῖς δακτύλοις, ἐπίγραψον δὲ ἐπὶ τὸ πλάτος τῆς καρδίας σου.
\par }{\PP \VS{4}Εἰπὸν τὴν σοφίαν σὴν ἀδελφὴν εἶναι, τὴν δὲ φρόνησιν γνώριμον περιποίησαι σεαυτῷ.
\VS{5}Ἵνα σε τηρήσῃ ἀπὸ γυναικὸς ἀλλοτρίας καὶ πονηρᾶς, ἐάν σε λόγοις τοῖς πρὸς χάριν ἐμβάληται.
\par }{\PP \VS{6}Ἀπὸ γὰρ θυρίδος ἐκ τοῦ οἴκου αὐτῆς εἰς τὰς πλατείας παρακύπτουσα,
\VS{7}ὃν ἂν ἴδῃ τῶν ἀφρόνων τέκνων νεανίαν ἐνδεῆ φρενῶν,
\VS{8}παραπορευόμενον παρὰ γωνίαν ἐν διόδοις οἴκων αὐτῆς, καὶ λαλοῦντα
\VS{9}ἐν σκότει ἑσπερινῷ, ἡνίκα ἂν ἡσυχία νυκτερινὴ καὶ γνοφώδης,
\VS{10}ἡ δὲ γυνὴ συναντᾷ αὐτῷ, εἶδος ἔχουσα πορνικὸν, ἣ ποιεῖ νέων ἐξίπτασθαι καρδίας.
\VS{11}Ἀνεπτερωμένη δέ ἐστι καὶ ἄσωτος, ἐν οἴκῳ δὲ οὐχ ἡσυχάζουσιν οἱ πόδες αὐτῆς.
\VS{12}Χρόνον γάρ τινα ἔξω ῥέμβεται, χρόνον δὲ ἐν πλατείαις παρὰ πᾶσαν γωνίαν ἐνεδρεύει.
\VS{13}Εἶτα ἐπιλαβομένη ἐφίλησεν αὐτὸν, ἀναιδεῖ δὲ προσώπῳ προσεῖπεν αὐτῷ,
\VS{14}θυσία εἰρηνική μοι ἐστὶ, σήμερον ἀποδίδωμι τὰς εὐχάς μου.
\VS{15}Ἕνεκα τούτου ἐξῆλθον εἰς συνάντησίν σοι, ποθοῦσα τὸ σὸν πρόσωπον, εὕρηκά σε.
\VS{16}κειρίαις τέτακα τὴν κλίνην μου, ἀμφιτάποις δὲ ἔστρωκα τοῖς ἀπʼ Αἰγύπτου.
\VS{17}Διέῤῥαγκα τὴν κοίτην μου κροκίνῳ, τὸν δὲ οἶκόν μου κινναμώμῳ·
\VS{18}Ἐλθὲ καὶ ἀπολαύσωμεν φιλίας ἕως ὄρθρου, δεῦρο καὶ ἐλκυλισθῶμεν ἔρωτι.
\VS{19}Οὐ γὰρ πάρεστιν ὁ ἀνήρ μου ἐν οἴκω, πεπόρευται δὲ ὁδὸν μακράν,
\VS{20}ἔνδεσμον ἀργυρίου λαβὼν ἐν χειρὶ αὐτοῦ, διʼ ἡμερῶν πολλῶν ἐπανήξει εἰς τὸν οἶκον αὐτοῦ.
\par }{\PP \VS{21}Ἀπεπλάνησε δὲ αὐτὸν πολλῇ ὁμιλίᾳ, βρόχοις τε τοῖς ἀπὸ χειλέων ἐξώκειλεν αὐτόν.
\VS{22}Ὁ δὲ ἐπηκολούθησεν αὐτῇ κεπφωθείς· ὥσπερ δὲ βοῦς ἐπὶ σφαγὴν ἄγεται, καὶ ὥσπερ κύων ἐπὶ δεσμοὺς,
\VS{23}ἢ ὡς ἔλαφος τοξεύματι πεπληγὼς εἰς τὸ ἧπαρ· σπεύδει δὲ ὥσπερ ὄρνεον εἰς παγίδα, οὐκ εἰδὼς ὅτι περὶ ψυχῆς τρέχει.
\par }{\PP \VS{24}Νῦν οὖν υἱὲ ἄκουέ μου, καὶ πρόσεχε ῥήμασι στόματός μου.
\VS{25}Μὴ ἐκκλινάτω εἰς τὰς ὁδοὺς αὐτῆς ἡ καρδία σου,
\VS{26}πολλοὺς γὰρ τρώσασα καταβέβληκε, καὶ ἀναρίθμητοί εἰσιν οὓς πεφόνευκεν.
\VS{27}Ὁδοὶ ᾅδου ὁ οἶκος αὐτῆς, κατάγουσαι εἰς τὰ ταμιεῖα τοῦ θανάτου.

\par }\Chap{8}{\PP \VerseOne{1}Σὺ τὴν σοφίαν κηρύξεις, ἵνα φρόνησίς σοι ὑπακούσῃ.
\VS{2}Ἐπὶ γὰρ τῶν ὑψηλῶν ἄκρων ἐστὶν, ἀναμέσον δὲ τῶν τρίβων ἕστηκε.
\VS{3}Παρὰ γὰρ πύλαις δυναστῶν παρεδρεύει, ἐν δὲ εἰσόδοις ὑμνεῖται.
\VS{4}Ὑμᾶς ὦ ἄνθρωποι παρακαλῶ, καὶ προΐεμαι ἐμὴν φωνὴν υἱοῖς ἀνθρώπων.
\VS{5}Νοήσατε ἄκακοι πανουργίαν, οἱ δὲ ἀπαίδευτοι ἔνθεσθε καρδίαν.
\VS{6}Εἰσακούσατέ μου, σεμνὰ γὰρ ἐρῶ, καὶ ἀνοίσω ἀπὸ χειλέων ὀρθά.
\VS{7}Ὅτι ἀλήθειαν μελετήσει ὁ φάρυγξ μου, ἐβδελυγμένα δὲ ἐναντίον ἐμοῦ χείλη ψευδῆ.
\VS{8}Μετὰ δικαιοσύνης πάντα τὰ ῥήματα τοῦ στόματός μου, οὐδὲν ἐαυτοῖς σκολιὸν οὐδὲ στραγγαλιῶδες.
\VS{9}Πάντα ἐνώπια τοῖς συνιοῦσι, καὶ ὀρθὰ τοῖς εὑρίσκουσι γνῶσιν.
\VS{10}Λάβετε παιδείαν καὶ μὴ ἀργύριον, καὶ γνῶσιν ὑπὲρ χρυσίον δεδοκιμασμένον·
\VS{11}Κρείσσων γὰρ σοφία λίθων πολυτελῶν, πᾶν δὲ τίμιον οὐκ ἄξιον αὐτῆς ἐστιν.
\par }{\PP \VS{12}Ἐγὼ ἡ σοφία κατεσκήνωσα βουλὴν καὶ γνῶσιν, καὶ ἔννοιαν ἐγὼ ἐπεκαλεσάμην.
\VS{13}Φόβος Κυρίου μισεῖ ἀδικίαν, ὕβριν τε καὶ ὑπερηφανίαν καὶ ὁδοὺς πονηρῶν· μεμίσηκα δὲ ἐγὼ διεστραμμένας ὁδοὺς κακῶν.
\VS{14}Ἐμὴ βουλὴ καὶ ἀσφάλεια, ἐμὴ φρόνησις, ἐμὴ δὲ ἰσχύς.
\VS{15}Διʼ ἐμοῦ βασιλεῖς βασιλεύουσι, καὶ οἱ δυνάσται γράφουσιν δικαιοσύνην.
\VS{16}Διʼ ἐμοῦ μεγιστᾶνες μεγαλύνονται, καὶ τύραννοι διʼ ἐμοῦ κρατοῦσι γῆς.
\VS{17}Ἐγὼ τοὺς ἐμὲ φιλοῦντας ἀγαπῶ, οἱ δὲ ἐμὲ ζητοῦντες εὑρήσουσιν.
\par }{\PP \VS{18}Πλοῦτος καὶ δόξα ἐμοὶ ὑπάρχει, καὶ κτῆσις πολλῶν καὶ δικαιοσύνη.
\VS{19}Βέλτιον ἐμὲ καρπίζεσθαι ὑπὲρ χρυσίον καὶ λίθον τίμιον, τὰ δὲ ἐμὰ γεννήματα κρείσσω ἀργυρίου ἐκλεκτοῦ.
\VS{20}Ἐν ὁδοῖς δικαιοσύνης περιπατῶ, καὶ ἀναμέσον τρίβων δικαιώματος ἀναστρέφομαι·
\VS{21}ἵνα μερίσω τοῖς ἐμὲ ἀγαπῶσιν ὕπαρξιν, καὶ τοὺς θησαυροὺς αὐτῶν ἐμπλήσω ἀγαθῶν·
\VS{21a}ἐὰν ἀναγγείλω ὑμῖν τὰ καθʼ ἡμέραν γινόμενα, μνημονεύσω τὰ ἐξ αἰῶνος ἀριθμῆσαι.
\par }{\PP \VS{22}Κύριος ἔκτισέ με ἀρχὴν ὁδῶν αὐτοῦ εἰς ἔργα αὐτοῦ,
\VS{23}πρὸ τοῦ αἰῶνος ἐθεμελίωσέ με, ἐν ἀρχῇ πρὸ τοῦ τὴν γῆν ποιῆσαι,
\VS{24}καὶ πρὸ τοῦ τὰς ἀβύσσους ποιῆσαι, πρὸ τοῦ προελθεῖν τὰς πηγὰς τῶν ὑδάτων·
\VS{25}Πρὸ τοῦ ὄρη ἑδρασθῆναι, πρὸ δὲ πάντων βουνῶν, γεννᾷ με.
\VS{26}Κύριος ἐποίησε χώρας καὶ ἀοικήτους, καὶ ἄκρα οἰκούμενα τῆς ὑπʼ οὐρανῶν.
\VS{27}Ἡνίκα ἡτοίμαζε τὸν οὐρανὸν, συμπαρήμην αὐτῷ, καὶ ὅτε ἀφώριζε τὸν ἑαυτοῦ θρόνον ἐπʼ ἀνέμων,
\VS{28}καὶ ὡς ἰσχυρὰ ἐποίει τὰ ἄνω νέφη, καὶ ὡς ἀσφαλεῖς ἐτίθει πηγὰς τῆς ὑπʼ οὐρανὸν,
\VS{29}καὶ ὡς ἰσχυρὰ ἐποίει τὰ θεμέλια τῆς γῆς,
\VS{30}ἤμην παρʼ αὐτῷ ἁρμόζουσα· ἐγὼ ἤμην ᾗ προσέχαιρε· καθʼ ἡμέραν δὲ εὐφραινόμην ἐν προσώπῳ αὐτοῦ ἐν παντὶ καιρῷ,
\VS{31}ὅτε ἐνευφραίνετο τὴν οἰκουμένην συντελέσας, καὶ ἐνευφραίνετο ἐν υἱοῖς ἀνθρώπων.
\par }{\PP \VS{32}Νῦν οὖν υἱὲ ἄκουέ μου,
\VS{34}μακάριος ἀνὴρ ὃς εἰσακούσεταί μου, καὶ ἄνθρωπος ὃς τὰς ἐμὰς ὁδοὺς φυλάξει, ἀγρυπνῶν ἐπʼ ἐμαῖς θύραις καθʼ ἡμέραν, τηρῶν σταθμοὺς ἐμῶν εἰσόδων.
\VS{35}Αἱ γὰρ ἔξοδοί μου, ἔξοδοι ζωῆς, καὶ ἑτοιμάζεται θέλησις παρὰ Κυρίου.
\VS{36}Οἱ δὲ ἁμαρτάνοντες εἰς ἐμὲ, ἀσεβοῦσιν εἰς τὰς ἑαυτῶν ψυχὰς, καὶ οἱ μισοῦντές με ἀγαπῶσι θάνατον.

\par }\Chap{9}{\PP \VerseOne{1}Ἡ σοφία ᾠκοδόμησεν ἑαυτῇ οἶκον, καὶ ὑπήρεισε στύλους ἑπτά.
\VS{2}Ἔσφαξε τὰ ἑαυτῆς θύματα, ἐκέρασεν εἰς κρατῆρα τὸν ἑαυτῆς οἶνον, καὶ ἡτοιμάσατο τὴν ἑαυτῆς τράπεζαν.
\VS{3}Ἀπέστειλε τοὺς ἑαυτῆς δούλους, συγκαλοῦσα μετὰ ὑψηλοῦ κηρύγματος ἐπὶ κρατῆρα, λέγουσα,
\VS{4}Ὅς ἐστιν ἄφρων, ἐκκλινάτω πρὸς μέ· καὶ τοῖς ἐνδεέσι φρενῶν εἶπεν,
\VS{5}ἔλθατε, φάγετε τῶν ἐμῶν ἄρτων, καὶ πίετε οἶνον ὃν ἐκέρασα ὑμῖν.
\par }{\PP \VS{6}Ἀπολείπετε ἀφροσύνην, ἵνα εἰς τὸν αἰῶνα βασιλεύσητε· καὶ ζητήσατε φρόνησιν, καὶ κατορθώσατε ἐν γνώσει σύνεσιν.
\VS{7}Ὁ παιδεύων κακοὺς λήψεται ἑαυτῷ ἀτιμίαν· ἐλέγχων δὲ τὸν ἀσεβῆ μωμήσεται ἑαυτόν.
\VS{8}Μὴ ἔλεγχε κακοὺς, ἵνα μὴ μισήσωσί σε· ἔλεγχε σοφὸν, καὶ ἀγαπήσει σε.
\VS{9}Δίδου σοφῷ ἀφορμὴν, καὶ σοφώτερος ἔσται· γνώριζε δικαίῳ, καὶ προσθήσει τοῦ δέχεσθαι.
\VS{10}Ἀρχὴ σοφίας φόβος Κυρίου, καὶ βουλὴ ἁγίων σύνεσις·
\VS{10a}τὸ γὰρ γνῶναι νόμον, διανοίας ἐστὶν ἀγαθῆς.
\VS{11}Τούτῳ γὰρ τῷ τρόπῳ πολὺν ζήσεις χρόνον, καὶ προστεθήσεταί σοι ἔτη ζωῆς σου.
\par }{\PP \VS{12}Υἱὲ ἐὰν σοφὸς γένῃ σεαυτῷ, σοφὸς ἔσῃ καὶ τοῖς πλησίον· ἐὰν δὲ κακὸς ἀποβῇς, μόνος ἂν ἀντλήσεις κακά·
\VS{12a}ὃς ἐρείδεται ἐπὶ ψεύδεσιν, οὗτος ποιμαίνει ἀνέμους, ὁ δʼ αὐτὸς διώξεται ὄρνεα πετόμενα·
\VS{12b}ἀπέλιπε γὰρ ὁδοὺς τοῦ ἑαυτοῦ ἀμπελῶνος, τοὺς δὲ ἄξονας τοῦ ἰδίου γεωργίου πεπλάνηται·
\VS{12c}διαπορεύεται δὲ διʼ ἀνύδρου ἐρήμου, καὶ γῆν διατεταγμένην ἐν διψώδεσι, συνάγει δὲ χερσὶν ἀκαρπίαν.
\par }{\PP \VS{13}Γυνὴ ἄφρων καὶ θρασεῖα ἐνδεὴς ψωμοῦ γίνεται, ἣ οὐκ ἐπίσταται αἰσχύνην.
\VS{14}Ἐκάθισεν ἐπὶ θύραις τοῦ ἑαυτῆς οἴκου, ἐπὶ δίφρου ἐμφανῶς ἐν πλατείαις,
\VS{15}προσκαλουμένη τοὺς παριόντας καὶ κατευθύνοντας ἐν ταῖς ὁδοῖς αὐτῶν·
\VS{16}Ὅς ἐστιν ὑμῶν ἀφρονέστατος, ἐκκλινάτω πρὸς μέ· καὶ τοῖς ἐνδεέσι φρονήσεως παρακελεύομαι, λέγουσα,
\VS{17}ἄρτων κρυφίων ἡδέως ἅψασθε, καὶ ὕδατος κλοπῆς γλυκεροῦ.
\par }{\PP \VS{18}Ὁ δὲ οὐκ οἶδεν ὅτι γηγενεῖς παρʼ αὐτῇ ὄλλυνται, καὶ ἐπὶ πέταυρον ᾅδου συναντᾷ·
\VS{18a}ἀλλὰ ἀποπήδησον, μὴ χρονίσῃς ἐν τῷ τόπῳ, μηδὲ ἐπιστήσῃς τὸ σὸν ὄμμα πρὸς αὐτὴν,
\VS{18b}οὕτως γὰρ διαβήσῃ ὕδωρ ἀλλότριον·
\VS{18c}ἀπὸ δὲ ὕδατος ἀλλοτρίου ἀπόσχου, καὶ ἀπὸ πηγῆς ἀλλοτρίας μὴ πίῃς
\VS{18d}ἵνα πολὺν ζησῃς χρόνον, προστεθῇ δέ σοι ἔτη ζωῆς.

\par }\Chap{10}{\PP \VerseOne{1}Υἱὸς σοφὸς εὐφραίνει πατέρα, υἱὸς δὲ ἄφρων λύπη τῇ μητρί.
\VS{2}Οὐκ ὠφελήσουσι θησαυροὶ ἀνόμους, δικαιοσύνη δὲ ῥύσεται ἐκ θανάτου.
\VS{3}Οὐ λιμοκτονήσει Κύριος ψυχὴν δικαίαν, ζωὴν δὲ ἀσεβῶν ἀνατρέψει.
\par }{\PP \VS{4}Πενία ἄνδρα ταπεινοῖ, χεῖρες δὲ ἀνδρείων πλουτίζουσιν·
\VS{4a}υἱὸς πεπαιδευμένος σοφὸς ἔσται, τῷ δὲ ἄφρονι διακόνῳ χρήσεται.
\VS{5}Διεσώθη ἀπὸ καύματος υἱὸς νοήμων, ἀνεμόφθορος δὲ γίνεται ἐν ἀμητῷ υἱὸς παράνομος.
\par }{\PP \VS{6}Εὐλογία Κυρίου ἐπὶ κεφαλὴν δικαίου, στόμα δὲ ἀσεβῶν καλύψει πένθος ἄωρον.
\VS{7}Μνήμη δικαίων μετʼ ἐγκωμίων, ὄνομα δὲ ἀσεβοῦς σβέννυται.
\VS{8}Σοφὸς καρδίᾳ δέξεται ἐντολὰς, ὁ δὲ ἄστεγος χείλες σκολιάζων ὑποσκελισθήσεται.
\VS{9}Ὃς πορεύεται ἁπλῶς, πορεύεται πεποιθώς· ὁ δὲ διαστρέφων τὰς ὁδοὺς αὐτοῦ, γνωσθήσεται.
\VS{10}Ὁ ἐννεύων ὀφθαλμοῖς μετὰ δόλου, συνάγει ἀνδράσι λύπας· ὁ δὲ ἐλέγχων μετὰ παῤῥησίας, εἰρηνοποιεῖ.
\VS{11}Πηγὴ ζωῆς ἐν χειρὶ δικαίου, στόμα δὲ ἀσεβοῦς καλύψει ἀπώλεια.
\par }{\PP \VS{12}Μῖσος ἐγείρει νεῖκος, πάντας δὲ τοὺς μὴ φιλονεικοῦντας καλύπτει φιλία.
\VS{13}Ὃ ἐκ χειλέων προφέρει σοφίαν, ῥάβδῳ τύπτει ἄνδρα ἀκάρδιον.
\VS{14}Σοφοὶ κρύψουσιν αἴσθησιν, στόμα δὲ προπετοῦς ἐγγίζει συντριβῇ.
\VS{15}Κτῆσις πλουσίων πόλις ὀχυρὰ, συντριβὴ δὲ ἀσεβῶν πενία.
\VS{16}Ἔργα δικαίων ζωὴν ποιεῖ, καρποὶ δὲ ἀσεβῶν ἁμαρτίας.
\VS{17}Ὁδοὺς δικαίας ζωῆς φυλάσσει παιδεία, παιδεία δὲ ἀνεξέλεγκτος πλανᾶται.
\par }{\PP \VS{18}Καλύπτουσιν ἔχθραν χείλη δίκαια, οἱ δὲ ἐκφέροντες λοιδορίας ἀφρονέστατοί εἰσιν.
\VS{19}Ἐκ πολυλογίας οὐκ ἐκφεύξῃ ἁμαρτίαν, φειδόμενος δὲ χειλέων νοήμων ἔσῃ.
\VS{20}Ἄργυρος πεπυρωμένος γλῶσσα δικαίου, καρδία δὲ ἀσεβοῦς ἐκλείψει.
\VS{21}Χείλη δικαίων ἐπίσταται ὑψηλὰ, οἱ δὲ ἄφρονες ἐν ἐνδείᾳ τελευτῶσιν.
\VS{22}Εὐλογία Κυρίου ἐπὶ κεφαλὴν δικαίου, αὕτη πλουτίζει, καὶ οὐ μὴ προστεθῇ αὐτῇ λύπη ἐν καρδίᾳ.
\par }{\PP \VS{23}Ἐν γέλωτι ἄφρων πράσσει κακὰ, ἡ δὲ σοφία ἀνδρὶ τίκτει φρόνησιν.
\par }{\PP \VS{24}Ἐν ἀπωλείᾳ ἀσεβὴς περιφέρεται, ἐπιθυμία δὲ δικαίου δεκτή.
\VS{25}Παραπορευομένης καταιγίδος ἀφανίζεται ἀσεβὴς, δίκαιος δὲ ἐκκλίνας σώζεται εἰς τὸν αἰῶνα.
\VS{26}Ὥσπερ ὄμφαξ ὀδοῦσι βλαβερὸν, καὶ καπνὸς ὄμμασιν, οὕτως παρανομία τοῖς χρωμένοις αὐτῇ.
\VS{27}Φόβος Κυρίου προστίθησιν ἡμέρας, ἔτη δὲ ἀσεβῶν ὀλιγωθήσεται.
\VS{28}Ἐγχρονίζει δικαίοις εὐφροσύνη, ἐλπὶς δὲ ἀσεβῶν ἀπολεῖται.
\VS{29}Ὀχύρωμα ὁσίου φόβος Κυρίου, συντριβὴ δὲ τοῖς ἐργαζομένοις κακά.
\par }{\PP \VS{30}Δίκαιος εἰς τὸν αἰῶνα οὐκ ἐνδώσει, ἀσεβεῖς δὲ οὐκ οἰκήσουσι γῆν.
\VS{31}Στόμα δικαίου ἀποστάζει σοφίαν, γλῶσσα δὲ ἀδίκου ἐξολεῖται.
\VS{32}Χείλη ἀνδρῶν δικαίων ἀποστάζει χάριτας, στόμα δὲ ἀσεβῶν ἀποστρέφεται.

\par }\Chap{11}{\PP \VerseOne{1}Ζυγοί δόλιοι βδέλυγμα ἐνώπιον Κυρίου, στάθμιον δὲ δίκαιον δεκτὸν αὐτῷ.
\VS{2}Οὗ ἐὰν εἰσέλθῃ ὕβρις, ἐκεῖ καὶ ἀτιμία· στόμα δὲ ταπεινῶν μελετᾷ σοφίαν.
\VS{3}Ἀποθανὼν δίκαιος ἔλιπε μετάμελον, πρόχειρος δὲ γίνεται καὶ ἐπίχαρτος ἀσεβῶν ἀπώλεια.
\VS{5}Δικαιοσύνη ἀμώμους ὀρθοτομεῖ ὁδοὺς, ἀσέβεια δὲ περιπίπτει ἀδικίᾳ.
\par }{\PP \VS{6}Δικαιοσύνη ἀνδρῶν ὀρθῶν ῥύεται αὐτούς, τῇ δὲ ἀπωλείᾳ αὐτῶν ἁλίσκονται παράνομοι.
\VS{7}Τελευτήσαντος ἀνδρὸς δικαίου, οὐκ ὄλλυται ἐλπίς, τὸ δὲ καύχημα τῶν ἀσεβῶν ὄλλυται.
\VS{8}Δίκαιος ἐκ θήρας ἐκδύνει, ἀντʼ αὐτοῦ δὲ παραδίδοται ὁ ἀσεβής.
\VS{9}Ἐν στόματι ἀσεβῶν παγὶς πολίταις, αἴσθησις δὲ δικαίων εὔοδος.
\VS{10}Ἐν ἀγαθοῖς δικαίων κατώρθωσε πόλις,
\VS{11}στόμασι δὲ ἀσεβῶν κατεσκάφη.
\par }{\PP \VS{12}Μυκτηρίζει πολίτας ἐνδεὴς φρενῶν, ἀνὴρ δὲ φρόνιμος ἡσυχίαν ἄγει.
\VS{13}Ἀνὴρ δίγλωσσος ἀποκαλύπτει βουλὰς ἐν συνεδρίῳ, πιστὸς δὲ πνοῇ κρύπτει πράγματα.
\VS{14}Οἷς μὴ ὑπάρχει κυβέρνησις, πίπτουσιν ὥσπερ φύλλα, σωτηρία δὲ ὑπάρχει ἐν πολλῇ βουλῇ.
\par }{\PP \VS{15}Πονηρὸς κακοποιεῖ ὅταν συνμίξῃ δικαίῳ, μισεῖ δὲ ἦχον ἀσφαλείας.
\VS{16}Γυνὴ εὐχάριστος ἐγείρει ἀνδρὶ δόξαν, θρόνος δὲ ἀτιμίας γυνὴ μισοῦσα δίκαια· πλούτου ὀκνηροὶ ἐνδεεῖς γίνονται, οἱ δὲ ἀνδρεῖοι ἐρείδονται πλούτῳ.
\VS{17}Τῇ ψυχῇ αὐτοῦ ἀγαθὸν ποιεῖ ἀνὴρ ἐλεήμων, ἐξολλύει δὲ αὐτοῦ σῶμα ὁ ἀνελεήμων.
\par }{\PP \VS{18}Ἀσεβὴς ποιεῖ ἔργα ἄδικα, σπέρμα δὲ δικαίων μισθὸς ἀληθείας.
\VS{19}Υἱὸς δίκαιος γεννᾶται εἰς ζωὴν, διωγμὸς δὲ ἀσεβοῦς εἰς θάνατον.
\VS{20}Βδέλυγμα Κυρίῳ διεστραμμέναι ὁδοὶ, προσδεκτοὶ δὲ αὐτῷ πάντες ἄμωμοι ἐν ταῖς ὁδοῖς αὐτῶν.
\VS{21}Χειρὶ χεῖρας ἐμβαλὼν ἀδίκως οὐκ ἀτιμώρητος ἔσται, ὁ δὲ σπείρων δικαιοσύνην λήψεται μισθὸν πιστόν.
\VS{22}Ὥσπερ ἐνώτιον ἐν ῥινὶ ὑός, οὕτως γυναικὶ κακόφρονι κάλλος.
\VS{23}Ἐπιθυμία δικαίων πᾶσα ἀγαθὴ, ἐλπὶς δὲ ἀσεβῶν ἀπολεῖται.
\par }{\PP \VS{24}Εἰσὶν, οἳ τὰ ἴδια σπείροντες πλείονα ποιοῦσιν· εἰσὶδέ καὶ οἳ συνάγοντες ἐλαττονοῦνται.
\VS{25}Ψυχὴ εὐλογουμένη πᾶσα ἁπλῇ, ἀνὴρ δὲ θυμώδης οὐκ εὐσχήμων.
\VS{26}Ὁ συνέχων σῖτον ὑπολείποιτο αὐτὸν τοῖς ἔθνεσιν· εὐλογία δὲ εἰς κεφαλὴν τοῦ μεταδιδόντος.
\VS{27}Τεκταινόμενος ἀγαθὰ ζητεῖ χάριν ἀγαθὴν, ἐκζητοῦντα δὲ κακὰ καταλήψεται αὐτόν.
\VS{28}Ὁ πεποιθὼς ἐπὶ πλούτῳ οὗτος πεσεῖται, ὁ δὲ ἀντιλαμβανόμενος δικαίων οὗτος ἀνατελεῖ.
\VS{29}Ὁ μὴ συμπεριφερόμενος τῷ ἑαυτοῦ οἴκῳ, κληρονομήσει ἄνεμον, δουλεύσει δὲ ἄφρων φρονίμῳ.
\VS{30}Ἐκ καρποῦ δικαιοσύνης φύεται δένδρὅν ζωῆς, ἀφαιροῦνται δὲ ἄωροι ψυχαὶ παρανόμων.
\VS{31}Εἰ ὁ μὲν δίκαιος μόλις σώζεται, ὁ ἀσεβὴς καὶ ἁμαρτωλὸς ποῦ φανεῖται;

\par }\Chap{12}{\PP \VerseOne{1}Ὁ ἀγαπῶν παιδείαν, ἀγαπᾷ αἴσθησιν· ὁ δὲ μισῶν ἐλέγχους, ἄφρων.
\VS{2}Κρείσσων ὁ εὑρὼν χάριν παρὰ Κυρίῳ, ἀνὴρ δὲ παράνομος παρασιωπηθήσεται.
\VS{3}Οὐ κατορθώσει ἄνθρωπος ἐξ ἀνόμου, αἱ δὲ ῥίζαι τῶν δικαίων οὐκ ἐξαρθήσονται.
\VS{4}Γυνὴ ἀνδρεία στέφανος τῷ ἀνδρὶ αὐτῆς· ὥσπερ δὲ ἐν ξύλῳ σκώληξ, οὕτως ἄνδρα ἀπόλλυσι γυνὴ κακοποιός.
\par }{\PP \VS{5}Λογισμοὶ δικαίων κρίματα, κυβερνῶσι δὲ ἀσεβεῖς δόλους.
\par }{\PP \VS{6}Λόγοι ἀσεβῶν δόλιοι, στόμα δὲ ὀρθῶν ῥύσεται αὐτούς.
\VS{7}Οὗ ἐὰν στραφῇ ὁ ἀσεβὴς, ἀφανίζεται, οἶκοι δὲ δικαίων παραμένουσι·
\VS{8}Στόμα συνετοῦ ἐγκωμιάζεται ὑπὸ ἀνδρὸς, νωθροκάρδιος δὲ μυκτηρίζεται.
\VS{9}Κρείσσων ἀνὴρ ἐν ἀτιμίᾳ δουλεύων ἑαυτῷ, ἢ τιμὴν ἑαυτῷ περιτιθεὶς καὶ προσδεόμενος ἄρτου.
\par }{\PP \VS{10}Δίκαιος οἰκτείρει ψυχὰς κτηνῶν αὐτοῦ, τὰ δὲ σπλάγχνα τῶν ἀσεβῶν ἀνελεήμονα.
\VS{11}Ὁ ἐργαζόμενος τὴν ἑαυτοῦ γῆν, ἐμπλησθήσεται ἄρτων, οἱ δὲ διώκοντες μάταια, ἐνδεεῖς φρενῶν·
\VS{11a}ὅς ἐστιν ἡδὺς ἐν οἴνων διατριβαῖς, ἐν τοῖς ἑαυτοῦ ὀχυρώμασι καταλείψει ἀτιμίαν.
\par }{\PP \VS{12}Ἐπιθυμίαι ἀσεβῶν κακαὶ, αἱ δὲ ῥίζαι τῶν εὐσεβῶν ἐν ὀχυρώμασι.
\VS{13}Διʼ ἁμαρτίαν χειλέων ἐμπίπτει εἰς παγίδας ἁμαρτωλὸς, ἐκφεύγει δὲ ἐξ αὐτῶν δίκαιος·
\VS{13a}ὁ βλέπων λεῖα ἐλεηθήσεται, ὁ δὲ συναντῶν ἐν πύλαις ἐκθλίψει ψυχάς.
\VS{14}Ἀπὸ καρπῶν στόματος ψυχὴ ἀνδρὸς πλησθήσεται ἀγαθῶν, ἀνταπόδομα δὲ χειλέων αὐτοῦ δοθήσεται αὐτῷ.
\VS{15}Ὁδοὶ ἀφρόνων ὀρθαὶ ἐνώπιον αὐτῶν, εἰσακούει δὲ συμβουλίας σοφός.
\VS{16}Ἄφρων αὐθημερὸν ἐξαγγέλλει ὀργὴν αὐτοῦ, κρύπτει δὲ τὴν ἑαυτοῦ ἀτιμίαν ἀνὴρ πανοῦργος.
\VS{17}Ἐπιδεικνυμένην πίστιν ἀπαγγέλλει δίκαιος, ὁ δὲ μάρτυς τῶν ἀδίκων δόλιος.
\par }{\PP \VS{18}Εἰσὶν οἳ λέγοντες τιτρώσκουσι, μάχαιραι· γλῶσσαι δὲ σοφῶν ἰῶνται.
\VS{19}Χείλη ἀληθινὰ κατορθοῖ μαρτυρίαν, μάρτυς δὲ ταχὺς γλῶσσαν ἔχει ἄδικον.
\VS{20}Δόλος ἐν καρδίᾳ τεκταινομένου κακὰ, οἱ δὲ βουλόμενοι εἰρήνην εὐφρανθήσονται.
\VS{21}Οὐκ ἀρέσει τῷ δικαίῳ οὐδὲν ἄδικον, οἱ δὲ ἀσεβεῖς πλησθήσονται κακῶν.
\VS{22}Βδέλυγμα Κυρίῳ χείλη ψευδῆ, ὁ δὲ ποιῶν πίστεις δεκτὸς παρʼ αὐτῷ.
\VS{23}Ἀνὴρ συνετὸς θρόνος αἰσθήσεως, καρδία δὲ ἀφρόνων συναντήσεται ἀραῖς.
\par }{\PP \VS{24}Χεὶρ ἐκλεκτῶν κρατήσει εὐχερῶς, δόλιοι δὲ ἔσονται ἐν προνομῇ.
\VS{25}Φοβερὸς λόγος καρδίαν ταράσσει ἀνδρὸς δικαίου, ἀγγελία δὲ ἀγαθὴ εὐφραίνει αὐτόν.
\VS{26}Ἐπιγνώμων δίκαιος ἑαυτοῦ φίλος ἔσται, ἁμαρτάνοντας δὲ καταδιώξεται κακὰ, ἡ δὲ ὁδὸς τῶν ἀσεβῶν πλανήσει αὐτούς.
\VS{27}Οὐκ ἐπιτεύξεται δόλιος θήρας, κτῆμα δὲ τίμιον ἀνὴρ καθαρός.
\VS{28}Ἐν ὁδοῖς δικαιοσύνης ζωὴ, ὁδοὶ δὲ μνησικάκων εἰς θάνατον.

\par }\Chap{13}{\PP \VerseOne{1}Υἱὸς πανοῦργος ὑπήκοος πατρὶ, υἱὸς δὲ ἀνήκοος ἐν ἀπωλείᾳ.
\VS{2}Ἀπὸ καρπῶν δικαιοσύνης φάγεται ἀγαθὸς, ψυχαὶ δὲ παρανόμων ὀλοῦνται ἄωροι.
\VS{3}Ὃς φυλάσσει τὸ ἑαυτοῦ στόμα τηρεῖ τὴν ἑαυτοῦ ψυχὴν, ὁ δὲ προπετὴς χείλεσι πτοήσει ἑαυτόν.
\VS{4}Ἐν ἐπιθυμίαις ἐστὶ πᾶς ἀεργὸς, χεῖρες δὲ ἀνδρείων ἐν ἐπιμελείᾳ.
\VS{5}Λόγον ἄδικον μισεῖ δίκαιος, ἀσεβὴς δὲ αἰσχύνεται, καὶ οὐκ ἕξει παῤῥησίαν.
\VS{7}Εἰσὶν οἱ πλουτίζοντες ἑαυτοὺς μηδὲν ἔχοντες, καὶ εἰσὶν οἱ ταπεινοῦντες ἑαυτοὺς ἐν πολλῷ πλούτῳ.
\par }{\PP \VS{8}Λύτρον ἀνδρὸς ψυχῆς ὁ ἴδιος πλοῦτος, πτωχὸς δὲ οὐχ ὑφίσταται ἀπειλήν.
\VS{9}Φῶς δικαίοις διαπαντὸς, φῶς δὲ ἀσεβῶν σβέννυται·
\VS{9a}ψυχαὶ δόλιαι πλανῶνται ἐν ἁμαρτίαις, δίκαιοι δὲ οἰκτείρουσι καὶ ἐλεοῦσι.
\VS{10}Κακὸς μεθʼ ὕβρεως πράσσει κακὰ, οἱ δʼ ἑαυτῶν ἐπιγνώμονες σοφοί.
\VS{11}Ὕπαρξις ἐπισπουδαζομένη μετὰ ἀνομίας, ἐλάσσων γίνεται, ὁ δὲ συνάγων ἑαυτῷ μετʼ εὐσεβείας πληθυνθήσεται· δίκαιος οἰκτείρει καὶ κιχρᾷ.
\VS{12}Κρείσσων ἐναρχόμενος βοηθῶν καρδίᾳ, τοῦ ἐπαγγελλομένου καὶ εἰς ἐλπίδα ἄγοντος· δένδρον γὰρ ζωῆς, ἐπιθυμία ἀγαθή.
\VS{13}Ὃς καταφρονεῖ πράγματος, καταφρονηθήσεται ὑπʼ αὐτοῦ· ὁ δὲ φοβούμενος ἐντολὴν, οὗτος ὑγιαίνει·
\VS{13a}υἱῷ δολίῳ οὐδὲν ἔσται ἀγαθὸν, οἰκέτῃ δὲ σοφῷ εὔοδοι ἔσονται πράξεις, καὶ κατευθυνθήσεται ἡ ὁδὸς αὐτοῦ.
\par }{\PP \VS{14}Νόμος σοφοῦ πηγὴ ζωῆς, ὁ δὲ ἄνους ὑπὸ παγίδος θανεῖται.
\VS{15}Σύνεσις ἀγαθὴ δίδωσι χάριν, τὸ δὲ γνῶναι νόμον διανοίας ἐστὶν ἀγαθῆς, ὁδοὶ δὲ καταφρονούντων ἐν ἀπωλείᾳ.
\par }{\PP \VS{16}Πᾶς πανοῦργος πράσσει μετὰ γνώσεως, ὁ δὲ ἄφρων ἐξεπέτασεν ἑαυτοῦ κακίαν.
\VS{17}Βασιλεὺς θρασὺς ἐμπεσεῖται εἰς κακὰ, ἄγγελος δὲ σοφὸς ῥύσεται αὐτόν.
\VS{18}Πενίαν καὶ ἀτιμίαν ἀφαιρεῖται παιδεία, ὁ δὲ φυλάσσων ἐλέγχους δοξασθήσεται.
\VS{19}Ἐπιθυμίαι εὐσεβῶν ἡδύνουσι ψυχὴν, ἔργα δὲ ἀσεβῶν μακρὰν ἀπὸ γνώσεως.
\VS{20}Συμπορευόμενος σοφοῖς σοφὸς ἔσῃ, ὁ δὲ συμπορευόμενος ἄφροσι γνωσθήσεται.
\VS{21}Ἁμαρτάνοντας καταδιώξεται κακὰ, τοὺς δὲ δικαίους καταλήψεται ἀγαθά.
\VS{22}Ἀγαθὸς ἀνὴρ κληρονομήσει υἱοὺς υἱῶν, θησαυρίζεται δὲ δικαίοις πλοῦτος ἀσεβῶν.
\VS{23}Δίκαιοι ποιήσουσιν ἐν πλούτῳ ἔτη πολλὰ, ἄδικοι δὲ ἀπολοῦνται συντόμως.
\par }{\PP \VS{24}Ὃς φείδεται τῆς βακτηρίας, μισεῖ τὸν υἱὸν αὐτοῦ ὁ δὲ ἀγαπῶν, ἐπιμελῶς παιδεύει.
\VS{25}Δίκαιος ἔσθων ἐμπιπλᾷ τὴν ψυχὴν αὐτοῦ, ψυχαὶ δὲ ἀσεβῶν ἐνδεεῖς.

\par }\Chap{14}{\PP \VerseOne{1}Σοφαὶ γυναῖκες ᾠκοδόμησαν οἴκους, ἡ δὲ ἄφρων κατέσκαψε ταῖς χερσὶν αὐτῆς.
\VS{2}Ὁ πορευόμενος ὀρθῶς φοβεῖται τὸν Κύριον, ὁ δὲ σκολιάζων ταῖς ὁδοῖς αὐτοῦ ἀτιμασθήσεται.
\VS{3}Ἐκ στόματος ἀφρόνων βακτηρία ὕβρεως, χείλη δὲ σοφῶν φυλάσσει αὐτούς.
\VS{4}Οὗ μή εἰσι βόες, φάτναι καθαραί· οὗ δὲ πολλὰ γεννήματα, φανερὰ βοὸς ἰσχύς.
\VS{5}Μάρτυς πιστὸς οὐ ψεύδεται, ἐκκαίει δὲ ψευδῆ μάρτυς ἄδικος.
\VS{6}Ζητήσεις σοφίαν παρὰ κακοῖς καὶ οὐχ εὑρήσεις, αἴσθησις δὲ παρὰ φρονίμοις εὐχερής.
\par }{\PP \VS{7}Πάντα ἐναντία ἀνδρὶ ἄφρονι, ὅπλα δὲ αἰσθήσεως χείλη σοφά.
\VS{8}Σοφία πανούργων ἐπιγνώσεται τὰς ὁδοὺς αὐτῶν, ἄνοια δὲ ἀφρόνων ἐν πλάνῃ.
\VS{9}Οἰκίαι παρανόμων ὀφειλήσουσι καθαρισμὸν, οἰκίαι δὲ δικαίων δεκταί.
\par }{\PP \VS{10}Καρδία ἀνδρὸς αἰσθητικὴ, λυπηρὰ ψυχὴ αὐτοῦ, ὅταν δὲ εὐφραίνηται οὐκ ἐπιμίγνυται ὕβρει.
\VS{11}Οἰκίαι ἀσεβῶν ἀφανισθήσονται, σκηναὶ δὲ κατορθούντων στήσονται.
\VS{12}Ἔστιν ὁδὸς ἣ δοκεῖ παρὰ ἀνθρώποις ὀρθὴ εἶναι, τὰ δὲ τελευταῖα αὐτῆς ἔρχεται εἰς πυθμένα ᾅδου.
\VS{13}Ἐν εὐφροσύναις οὐ προσμίγνυται λύπη, τελευταῖα δὲ χαρὰ εἰς πένθος ἔρχεται.
\VS{14}Τῶν ἑαυτοῦ ὁδῶν πλησθήσεται θρασυκάρδιος, ἀπὸ δὲ τῶν διανοημάτων αὐτοῦ ἀνὴρ ἀγαθός.
\VS{15}Ἄκακος πιστεύει παντὶ λόγῳ, πανοῦργος δὲ ἔρχεται εἰς μετάνοιαν.
\VS{16}Σοφὸς φοβηθεὶς ἐξέκλινεν ἀπὸ κακοῦ, ὁ δὲ ἄφρων ἑαυτῷ πεποιθὼς μίγνυται ἀνόμῳ.
\VS{17}Ὀξύθυμος πράσσει μετὰ ἀβουλίας, ἀνὴρ δὲ φρόνιμος πολλὰ ὑποφέρει.
\par }{\PP \VS{18}Μεριοῦνται ἄφρονες κακίαν, οἱ δὲ πανοῦργοι κρατήσουσιν αἰσθήσεως.
\VS{19}Ὀλισθήσουσι κακοὶ ἔναντι ἀγαθῶν, καὶ ἀσεβεῖς θεραπεύσουσι θύρας δικαίων.
\VS{20}Φίλοι μισήσουσι φίλους πτωχούς, φίλοι δὲ πλουσίων πολλοί.
\VS{21}Ὁ ἀτιμάζων πένητας ἁμαρτάνει, ἐλεῶν δὲ πτωχοὺς μακαριστός.
\VS{22}Πλανώμενοι τεκταίνουσι κακά, ἔλεον δὲ καὶ ἀλήθειαν τεκταίνουσιν ἀγαθοί· οὐκ ἐπίστανται ἔλεον καὶ πίστιν τέκτονες κακῶν, ἐλεημοσύναι δὲ καὶ πίστεις παρὰ τέκτοσιν ἀγαθοῖς.
\VS{23}Ἐν παντὶ μεριμνῶντι ἔνεστι περισσόν, ὁ δὲ ἡδὺς καὶ ἀνάλγητος ἐν ἐνδείᾳ ἔσται.
\VS{24}Στέφανος σοφῶν πανοῦργος, ἡ δὲ διατριβὴ ἀφρόνων κακή.
\par }{\PP \VS{25}Ῥύσεται ἐκ κακῶν ψυχὴν μάρτυς πιστὸς, ἐκκαίει δὲ ψευδῆ δόλιος.
\VS{26}Ἐν φόβῳ Κυρίου ἐλπὶς ἰσχύος, τοῖς δὲ τέκνοις αὐτοῦ καταλείπει ἔρεισμα.
\VS{27}Πρόσταγμα Κυρίου πηγὴ ζωῆς, ποιεῖ δὲ ἐκκλίνειν ἐκ παγίδος θανάτου.
\par }{\PP \VS{28}Ἐν πολλῷ ἔθνει δόξα βασιλέως, ἐν δὲ ἐκλείψει λαοῦ συντριβὴ δυνάστου.
\VS{29}Μακρόθυμος ἀνὴρ πολὺς ἐν φρονήσει, ὁ δὲ ὀλιγόψυχος ἰσχυρῶς ἄφρων.
\VS{30}Πρᾳΰθυμος ἀνὴρ καρδίας ἰατρὸς, σὴς δὲ ὀστέων καρδία αἰσθητική·
\VS{31}Ὁ συκοφαντῶν πένητα παροξύνει τὸν ποιήσαντα αὐτὸν, ὁ δὲ τιμῶν αὐτὸν ἐλεεῖ πτωχόν.
\VS{32}Ἐν κακίᾳ αὐτοῦ ἀπωσθήσεται ἀσεβής, ὁ δὲ πεποιθὼς τῇ ἑαυτοῦ ὁσιότητι δίκαιος.
\par }{\PP \VS{33}Ἐν καρδίᾳ ἀγαθῇ ἀνδρὸς σοφία, ἐν δὲ καρδίᾳ ἀφρόνων οὐ διαγινώσκεται.
\VS{34}Δικαιοσύνη ὑψοῖ ἔθνος, ἐλασσονοῦσι δὲ φυλὰς ἁμαρτίαι.
\VS{35}Δεκτὸς βασιλεῖ ὑπηρέτης νοήμων, τῇ δὲ ἑαυτοῦ εὐστροφίᾳ ἀφαιρεῖται ἀτιμίαν.

\par }\Chap{15}{\PP \VerseOne{1}Ὀργὴ ἀπόλλυσι καὶ φρονίμους, ἀπόκρισις δὲ ὑποπίπτουσα ἀποστρέφει θυμὸν, λόγος δὲ λυπηρὸς ἐγείρει ὀργάς.
\VS{2}Γλῶσσα σοφῶν καλὰ ἐπίσταται, στόμα δὲ ἀφρόνων ἀναγγέλλει κακά.
\par }{\PP \VS{3}Ἐν παντὶ τόπῳ ὀφθαλμοὶ Κυρίου σκοπεύουσι κακούς τε καὶ ἀγαθούς.
\VS{4}Ἴασις γλώσσης δένδρον ζωῆς, ὁ δὲ συντηρῶν αὐτὴν πλησθήσεται πνεύματος.
\VS{5}Ἄφρων μυκτηρίζει παιδείαν πατρὸς, ὁ δὲ φυλάσσων ἐντολὰς, πανουργότερος· ἐν πλεοναζούσῃ δικαιοσύνῃ ἰσχὺς πολλὴ, οἱ δὲ ἀσεβεῖς ὁλόῤῥιζοι ἐκ γῆς ἀπολοῦνται.
\par }{\PP \VS{6}Οἴκοις δικαίων ἰσχὺς πολλή, καρποὶ δὲ ἀσεβῶν ἀπολοῦνται.
\VS{7}Χείλη σοφῶν δέδεται αἰσθήσει, καρδίαι δὲ ἀφρόνων οὐκ ἀσφαλεῖς.
\VS{8}Θυσίαι ἀσεβῶν βδέλυγμα Κυρίῳ, εὐχαὶ δὲ κατευθυνόντων δεκταὶ παρʼ αὐτῷ.
\VS{9}Βδέλυγμα Κυρίῳ ὁδοὶ ἀσεβοῦς, διώκοντας δὲ δικαιοσύνην ἀγαπᾷ.
\VS{10}Παιδεία ἀκάκου γνωρίζεται ὑπὸ τῶν παριόντων, οἱ δὲ μισοῦντες ἐλέγχους τελευτῶσιν αἰσχρῶς.
\par }{\PP \VS{11}Ἅδης καὶ ἀπώλεια φανερὰ παρὰ τῷ Κυρίῳ· πῶς οὐχὶ καὶ αἱ καρδίαι τῶν ἀνθρώπων;
\VS{12}Οὐκ ἀγαπήσει ἀπαίδευτος τοὺς ἐλέγχοντας αὐτόν, μετὰ δὲ σοφῶν οὐχ ὁμιλήσει.
\VS{13}Καρδίας εὐφραινομένης πρόσωπον θάλλει, ἐν δὲ λύπαις οὔσης σκυθρωπάζει.
\VS{14}Καρδία ὀρθὴ ζητεῖ αἴσθησιν, στόμα δὲ ἀπαιδεύτων γνώσεται κακά.
\par }{\PP \VS{15}Πάντα τὸν χρόνον οἱ ὀφθαλμοὶ τῶν κακῶν προσδέχονται κακὰ, οἱ δὲ ἀγαθοὶ ἡσυχάζουσι διαπαντός.
\VS{16}Κρεῖσσον μικρὰ μερὶς μετὰ φόβου Κυρίου, ἢ θησαυροὶ μεγάλοι μετὰ ἀφοβίας.
\VS{17}Κρείσσων ξενισμὸς μετὰ λαχάνων πρὸς φιλίαν καὶ χάριν, ἢ παράθεσις μόσχων μετὰ ἔχθρας.
\VS{18}Ἀνὴρ θυμώδης παρασκευάζει μάχας· μακρόθυμος δὲ καὶ τὴν μέλλουσαν καταπρᾳΰνει·
\VS{18a}μακρόθυμος ἀνὴρ κατασβέσει κρίσεις, ὁ δὲ ἀσεβὴς ἐγείρει μᾶλλον.
\VS{19}Ὁδοὶ ἀεργῶν ἐστρωμέναι ἀκάνθαις, αἱ δὲ τῶν ἀνδρείων τετριμμέναι.
\VS{20}Υἱὸς σοφὸς εὐφραίνει πατέρα, υἱὸς δὲ ἄφρων μυκτηρίζει μητέρα αὐτοῦ.
\VS{21}Ἀνοήτου τρίβοι ἐνδεεῖς φρενῶν, ἀνὴρ δὲ φρόνιμος κατευθύνων πορεύεται.
\VS{22}Ὑπερτίθενται λογισμοὺς οἱ μὴ τιμῶντες συνέδρια, ἐν δὲ καρδίαις βουλευομένων μένει βουλή.
\par }{\PP \VS{23}Οὐ μὴ ὑπακούσει ὁ κακὸς αὐτῇ, οὐδὲ μὴ εἴπῃ καίριόν τι καὶ καλὸν τῷ κοινῷ.
\par }{\PP \VS{24}Ὁδοὶ ζωῆς διανοήματα συνετοῦ, ἵνα ἐκκλίνας ἐκ τοῦ ᾅδου σωθῇ.
\VS{25}Οἴκους ὑβριστῶν κατασπᾷ Κύριος, ἐστήρισε δὲ ὅριον χήρας.
\VS{26}Βδέλυγμα Κυρίῳ λογισμὸς ἄδικος, ἁγνῶν δὲ ῥήσεις σεμναί.
\VS{27}Ἐξόλλυσιν ἑαυτὸν ὁ δωρολήπτης, ὁ δὲ μισῶν δώρων λήψεις σώζεται·
\VS{27a}ἐλεημοσύναις καὶ πίστεσιν ἀποκαθαίρονται ἁμαρτίαι, τῷ δὲ φόβῳ Κυρίου ἐκκλίνει πᾶς ἀπὸ κακοῦ.
\par }{\PP \VS{28}Καρδίαι δικαίων μελετῶσι πίστεις, στόμα δὲ ἀσεβῶν ἀποκρίνεται κακά·
\VS{28a}δεκταὶ παρὰ Κυρίῳ ὁδοὶ ἀνθρώπων δικαίων, διὰ δὲ αὐτῶν καὶ οἱ ἐχθροὶ φίλοι γίνονται.
\VS{29}Μακρὰν ἀπέχει ὁ Θεὸς ἀπὸ ἀσεβῶν, εὐχαῖς δὲ δικαίων ἐπακούει·
\VS{29a}κρείσσων ὀλίγη λῆψις μετὰ δικαιοσύνης, ἢ πολλὰ γεννήματα μετὰ ἀδικίας.
\par }{\PP \VS{29b}Καρδία ἀνδρὸς λογιζέσθω δίκαια, ἵνα ὑπὸ τοῦ Θεοῦ διορθωθῇ τὰ διαβήματα αὐτοῦ.
\VS{30}Θεωρῶν ὀφθαλμὸς καλὰ εὐφραίνει καρδίαν, φημη δὲ ἀγαθὴ πιαίνει ὀστᾶ.
\VS{32}Ὃς ἀπωθεῖται παιδείαν, μισεῖ ἑαυτὸν· ὁ δὲ τηρῶν ἐλέγχους, ἀγαπᾷ ψυχὴν αὐτοῦ.
\VS{33}Φόβος Κυρίου παιδεία καὶ σοφία, καὶ ἀρχὴ δόξης ἀποκριθήσεται αὐτῇ.

\Chap{16}\VS{2}Πάντα τὰ ἔργα τοῦ ταπεινοῦ φανερὰ παρὰ τῷ Θεῷ, οἱ δὲ ἀσεβεῖς ἐν ἡμέρᾳ κακῇ ὀλοῦνται.
\VS{5}Ἀκάθαρτος παρὰ Θεῷ πᾶς ὑψηλοκάρδιος, χειρὶ δὲ χεῖρας ἐμβαλὼν ἀδίκως οὐκ ἀθωωθήσεται·
\VS{7}ἀρχὴ ὁδοῦ ἀγαθῆς τὸ ποιεῖν τὰ δίκαια, δεκτὰ δὲ παρὰ Θεῷ μᾶλλον ἢ θύειν θυσίας·
\VS{8}ὁ ζητῶν τὸν Κύριον εὑρήσει γνῶσιν μετὰ δικαιοσύνης, οἱ δὲ ὀρθῶς ζητοῦντες αὐτὸν εὑρήσουσιν εἰρήνην.
\VS{9}Πάντα τὰ ἔργα τοῦ Κυρίου μετὰ δικαιοσύνης, φυλάσσεται δὲ ὁ ἀσεβὴς εἰς ἡμέραν κακήν.
\par }{\PP \VS{10}Μαντεῖον ἐπὶ χείλεσι βασιλέως, ἐν δὲ κρίσει οὐ μὴ πλανηθῇ τὸ στόμα αὐτοῦ.
\VS{11}Ῥοπὴ ζυγοῦ δικαιοσύνη παρὰ Κυρίῳ, τὰ δὲ ἔργα αὐτοῦ στάθμια δίκαια.
\VS{12}Βδέλυγμα βασιλεῖ ὁ ποιῶν κακὰ, μετὰ γὰρ δικαιοσύνης ἑτοιμάζεται θρόνος ἀρχῆς.
\VS{13}Δεκτὰ βασιλεῖ χείλη δίκαια, λόγους δέ ὀρθοὺς ἀγαπᾷ.
\VS{14}Θυμὸ βασιλέως ἄγγελος θανάτου, ἀνὴρ δὲ σοφὸς ἐξιλάσεται αὐτόν.
\VS{15}Ἐν φωτὶ ζωῆς υἱὸς βασιλέως, οἱ δὲ προσδεκτοὶ αὐτῷ ὥσπερ νέφος ὄψιμον.
\VS{16}Νοσσιαὶ σοφίας αἱρετώτεραι χρυσίου, νοσσιαὶ δὲ φρονήσεως αἱρετώτεραι ὑπὲρ ἀργύριον.
\VS{17}Τρίβοι ζωῆς ἐκκλίνουσιν ἀπὸ κακῶν, μῆκος δὲ βίου ὁδοὶ δικαιοσύνης. Ὁ δεχόμενος παιδείαν ἐν ἀγαθοῖς ἔσται, ὁ δὲ φυλάσσων ἐλέγχους σοφισθήσεται· ὃς φυλάσσει τὰς ἑαυτοῦ ὁδοὺς, τηρεῖ τὴν ἑαυτοῦ ψυχήν· ἀγαπῶν δὲ ζωὴν αὐτοῦ, φείσεται στόματος αὐτοῦ.
\par }{\PP \VS{18}Πρὸ συντριβῆς ἡγεῖται ὕβρις, πρὸ δὲ πτώματος κακοφροσύνη.
\VS{19}Κρείσσων πρᾳΰθυμος μετὰ ταπεινώσεως, ἢ ὃς διαιρεῖται σκῦλα μετὰ ὑβριστῶν.
\VS{20}Συνετὸς ἐν πράγμασιν εὑρετὴς ἀγαθῶν, πεποιθὼς δὲ ἐπὶ Θεῷ μακαριστός.
\VS{21}Τοὺς σοφοὺς καὶ συνετοὺς φαύλους καλοῦσιν, οἱ δὲ γλυκεῖς ἐν λόγῳ πλείονα ἀκούσονται.
\VS{22}Πηγὴ ζωῆς ἔννοια τοῖς κεκτημένοις, παιδεία δὲ ἀφρόνων κακή.
\VS{23}Καρδία σοφοῦ νοήσει τὰ ἀπὸ τοῦ ἰδίου στόματος, ἐπὶ δὲ χείλεσι φορέσει ἐπιγνωμοσύνην·
\VS{24}Κηρία μέλιτος λόγοι καλοί, γλύκασμα δὲ αὐτοῦ ἴασις ψυχῆς.
\par }{\PP \VS{25}Εἰσὶν ὁδοὶ δοκοῦσαι εἶναι ὀρθαὶ ἀνδρὶ, τὰ μέντοι τελευταῖα αὐτῶν βλέπει εἰς πυθμένα ᾅδου.
\VS{26}Ἀνὴρ ἐν πόνοις πονεῖ ἑαυτῷ, καὶ ἐκβιάζεται τὴν ἀπώλειαν ἑαυτοῦ. Ὁ μέντοι σκολιὸς ἐπὶ τῷ ἑαυτοῦ στόματι φορεῖ τὴν ἀπώλειαν·
\VS{27}ἀνὴρ ἄφρων ὀρύσσει ἑαυτῷ κακὰ, ἐπὶ δὲ τῶν ἑαυτοῦ χειλέων θησαυρίζει πῦρ.
\VS{28}Ἀνὴρ σκολιὸς διαπέμπεται κακὰ, καὶ λαμπτῆρα δόλου πυρσεύσει κακοῖς, καὶ διαχωρίζει φίλους.
\VS{29}Ἀνὴρ παράνομος ἀποπειρᾶται φίλων, καὶ ἀπάγει αὐτοὺς ὁδοὺς οὐκ ἀγαθάς.
\par }{\PP \VS{30}Στηρίζων δὲ ὀφθαλμοὺς αὐτοῦ διαλογίζεται διεστραμμένα, ὁρίζει δὲ τοῖς χείλεσιν αὐτοῦ πάντα τὰ κακά· οὗτος κάμινός ἐστι κακίας.
\VS{31}Στέφανος καυχήσεως γῆρας, ἐν δὲ ὁδοῖς δικαιοσύνης εὑρίσκεται.
\VS{32}Κρείσσων ἀνὴρ μακρόθυμος ἰσχυροῦ, ὁ δὲ κρατῶν ὀργῆς κρείσσων καταλαμβανομένου πόλιν.
\VS{33}Εἰς κόλπους ἐπέρχεται πάντα τοῖς ἀδίκοις, παρὰ δὲ Κυρίου πάντα τὰ δίκαια.

\par }\Chap{17}{\PP \VerseOne{1}Κρείσσων ψωμὸς μεθʼ ἡδονῆς ἐν εἰρήνῃ, ἢ οἶκος πολλῶν ἀγαθῶν καὶ ἀδίκων θυμάτων μετὰ μάχης.
\VS{2}Οἰκέτης νοήμων κρατήσει δεσποτῶν ἀφρόνων, ἐν δὲ ἀδελφοῖς διελεῖται μέρη.
\VS{3}Ὥσπερ δοκιμάζεται ἐν καμίνῳ ἄργυρος καὶ χρυσὸς, οὕτως ἐκλεκταὶ καρδίαι παρὰ Κυρίῳ.
\VS{4}Κακὸς ὑπακούει γλώσσης παρανόμων, δίκαιος δὲ οὐ προσέχει χείλεσι ψευδέσιν.
\VS{5}Ὁ καταγελῶν πτωχοῦ παροξύνει τὸν ποιήσαντα αὐτὸν, ὁ δὲ ἐπιχαίρων ἀπολλυμένῳ οὐκ ἀθωωθήσεται, ὁ δὲ ἐπισπλαγχνιζόμενος ἐλεηθήσεται.
\par }{\PP \VS{6}Στέφανος γερόντων τέκνα τέκνων, καύχημα δὲ τέκνων πατέρες αὐτῶν·
\VS{6a}τοῦ πιστοῦ ὅλος ὁ κόσμος τῶν χρημάτων, τοῦ δὲ ἀπίστου οὐδὲ ὀβολός.
\VS{7}Οὐχ ἁρμόσει ἄφρονι χείλη πιστὰ, οὐδὲ δικαίῳ χείλη ψευδῆ.
\VS{8}Μισθὸς χαρίτων παιδεία τοῖς χρωμένοις, οὗ δʼ ἂν ἐπιστρέψῃ εὐοδωθήσεται.
\VS{9}Ὃς κρύπτει ἀδικήματα, ζητεῖ φιλίαν· ὃς δὲ μισεῖ κρύπτειν, διΐστησι φίλους καὶ οἰκείους.
\VS{10}Συντρίβει ἀπειλὴ καρδίαν φρονίμου, ἄφρων δὲ μαστιγωθεὶς οὐκ αἰσθάνεται.
\VS{11}Ἀντιλογίας ἐγείρει πᾶς κακὸς, ὁ δὲ Κύριος ἄγγελον ἀνελεήμονα ἐκπέμψει αὐτῷ.
\par }{\PP \VS{12}Ἐμπεσεῖται μέριμνα ἀνδρὶ νοήμονι, οἱ δὲ ἄφρονες διαλογιοῦνται κακά.
\VS{13}Ὃς ἀποδίδωσι κακὰ ἀντὶ ἀγαθῶν, οὐ κινηθήσεται κακὰ ἐκ τοῦ οἴκου αὐτοῦ.
\VS{14}Ἐξουσίαν δίδωσι λόγοις ἀρχὴ δικαιοσύνης, προηγεῖται δὲ τῆς ἐνδείας στάσις καὶ μάχη.
\VS{15}Ὃς δίκαιον κρίνει τὸν ἄδικον, ἄδικον δὲ τὸν δίκαιον, ἀκάθαρτος καὶ βδελυκτὸς παρὰ Θεῷ.
\VS{16}Ἱνατί ὑπῆρξε χρήματα ἄφρονι; κτήσασθαι γὰρ σοφίαν ἀκάρδιος οὐ δυνήσεται·
\VS{16a}ὃς ὑψηλὸν ποιεῖ τὸν ἑαυτοῦ οἶκον, ζητεῖ συντριβήν· ὁ δὲ σκολιάζων τοῦ μαθεῖν, ἐμπεσεῖται εἰς κακά.
\VS{17}Εἰς πάντα καιρὸν φίλος ὑπαρχέτω σοι, ἀδελφοὶ δὲ ἐν ἀνάγκαις χρήσιμοι ἔστωσαν, τούτου γὰρ χάριν γεννῶνται.
\VS{18}Ἀνὴρ ἄφρων ἐπικροτεῖ καὶ ἐπιχαίρει ἑαυτῷ, ὡς καὶ ὁ ἐγγυώμενος ἐγγύῃ τῶν ἑαυτοῦ φίλων.
\par }{\PP \VS{19}Φιλαμαρτήμων χαίρει μάχαις,
\VS{20}ὁ δὲ σκληροκάρδιος οὐ συναντᾷ ἀγαθοῖς· ἀνὴρ εὐμετάβολος γλώσσῃ ἐμπεσεῖται εἰς κακὰ,
\VS{21}καρδία δὲ ἄφρονος ὀδύνη τῷ κεκτημένῳ αὐτήν· οὐκ εὐφραίνεται πατὴρ ἐφʼ υἱῷ ἀπαιδεύτῳ, υἱὸς δὲ φρόνιμος εὐφραίνει μητέρα αὐτοῦ.
\VS{22}Καρδία εὐφραινομένη εὐεκτεῖν ποιεῖ, ἀνδρὸς δὲ λυπηροῦ ξηραίνεται τὰ ὀστᾶ.
\VS{23}Λαμβάνοντος δῶρα ἀδίκως ἐν κόλποις οὐ κατευοδοῦνται ὁδοὶ, ἀσεβὴς δὲ ἐκκλίνει ὁδοὺς δικαιοσύνης.
\VS{24}Πρόσωπον συνετὸν ἀνδρὸς σοφοῦ, οἱ δὲ ὀφθαλμοὶ τοῦ ἄφρονος ἐπʼ ἄκρα γῆς.
\VS{25}Ὀργὴ πατρὶ υἱὸς ἄφρων, καὶ ὀδύνη τῇ τεκούσῃ αὐτόν.
\par }{\PP \VS{26}Ζημιοῦν ἄνδρα δίκαιον οὐ καλὸν, οὐδὲ ὅσιον ἐπιβουλεύειν δυνάσταις δικαίοις.
\VS{27}Ὃς φείδεται ῥῆμα προέσθαι σκληρὸν, ἐπιγνώμων· μακρόθυμος δὲ ἀνὴρ φρόνιμος.
\VS{28}Ἀνοήτῳ ἐπερωτήσαντι σοφίαν σοφία λογισθήσεται, ἐνεὸν δέ τις ἑαυτὸν ποιήσας, δόξει φρόνιμος εἶναι.

\par }\Chap{18}{\PP \VerseOne{1}Προφάσεις ζητεῖ ἀνὴρ βουλόμενος χωρίζεσθαι ἀπὸ φίλων, ἐν παντὶ δὲ καιρῷ ἐπονείδιστος ἔσται.
\VS{2}Οὐ χρείαν ἔχει σοφίας ἐνδεὴς φρενῶν, μᾶλλον γᾶρ ἄγεται ἀφροσύνῃ.
\VS{3}Ὅταν ἔλθῃ ἀσεβὴς εἰς βάθος κακῶν, καταφρονεῖ, ἐπέρχεται δὲ αὐτῷ ἀτιμία καὶ ὄνειδος.
\VS{4}Ὕδωρ βαθὺ λόγος ἐν καρδίᾳ ἀνδρὸς, ποταμὸς δὲ ἀναπηδύει καὶ πηγὴ ζωῆς.
\VS{5}Θαυμάσαι πρόσωπον ἀσεβοῦς οὐ καλὸν, οὐδὲ ὅσιον ἐκκλίνειν τὸ δίκαιον ἐν κρίσει.
\par }{\PP \VS{6}Χείλη ἄφρονος ἄγουσιν αὐτὸν εἰς κακὰ, τὸ δὲ στόμα αὐτοῦ τὸ θρασὺ θάνατον ἐπικαλεῖται.
\VS{7}Στόμα ἄφρονος συντριβὴ αὐτῷ, τὰ δὲ χείλη αὐτοῦ παγὶς τῇ ψυχῇ αὐτοῦ.
\VS{8}Ὀκνηροὺς καταβάλλει φόβος, ψυχαὶ δὲ ἀνδρογύνων πεινάσουσιν.
\VS{9}Ὁ μὴ ἰώμενος αὐτὸν ἐν τοῖς ἔργοις αὐτοῦ, ἀδελφός ἐστι τοῦ λυμαινομένου ἑαυτόν.
\VS{10}Ἐκ μεγαλωσύνης ἰσχύος ὄνομα Κυρίου, αὐτῷ δὲ προσδραμόντες δίκαιοι ὑψοῦνται.
\VS{11}Ὕπαρξις πλουσίου ἀνδρὸς πόλις ὀχυρὰ, ἡ δὲ δόξα αὐτῆς μέγα ἐπισκιάζει.
\VS{12}Πρὸ συντριβῆς ὑψοῦται καρδία ἀνδρὸς, καὶ πρὸ δόξης ταπεινοῦνται.
\VS{13}Ὃς ἀποκρίνεται λόγον πρὶν ἀκοῦσαι, ἀφροσύνη αὐτῷ ἐστι καὶ ὄνειδος.
\VS{14}Θυμὸν ἀνδρὸς πρᾳΰνει θεράπων φρόνιμος, ὀλιγόψυχον δὲ ἄνδρα τίς ὑποίσει;
\VS{15}Καρδία φρονίμου κτᾶται αἴσθησιν, ὦτα δὲ σοφῶν ζητεῖ ἔννοιαν.
\VS{16}Δόμα ἀνθρώπου ἐμπλατύνει αὐτὸν, καὶ παρὰ δυνάσταις καθιζάνει αὐτόν.
\VS{17}Δίκαιος ἑαυτοῦ κατήγορος ἐν πρωτολογίᾳ, ὡς δʼ ἂν ἐπιβάλῃ ὁ ἀντίδικος ἐλέγχεται.
\par }{\PP \VS{18}Ἀντιλογίας παύει σιγηρὸς, ἐν δὲ δυναστείαις ὁρίζει.
\VS{19}Ἀδελφὸς ὑπὸ ἀδελφοῦ βοηθούμενος, ὡς πόλις ὀχυρὰ καὶ ὑψηλὴ, ἰσχύει δὲ ὥσπερ τεθεμελιωμένον βασίλειον.
\VS{20}Ἀπὸ καρπῶν στόματος ἀνὴρ πίμπλησι κοιλίαν αὐτοῦ, ἀπὸ δὲ καρπῶν χειλέων αὐτοῦ ἐμπλησθήσεται.
\VS{21}Θάνατος καὶ ζωὴ ἐν χειρὶ γλώσσης, οἱ δὲ κρατοῦντες αὐτῆς ἔδονται τοὺς καρποὺς αὐτῆς.
\VS{22}Ὃς εὗρε γυναῖκα ἀγαθὴν, εὗρε χάριτας, ἔλαβε δὲ παρὰ Θεοῦ ἱλαρότητα·
\VS{22a}ὃς ἐκβάλλει γυναῖκα ἀγαθὴν, ἐκβάλλει τὰ ἀγαθὰ, ὁ δὲ κατέχων μοιχαλίδα, ἄφρων καὶ ἀσεβής.

\par }\Chap{19}{\PP \VS{3}Ἀφροσύνη ἀνδρὸς λυμαίνεται τὰς ὁδοὺς αὐτοῦ, τὸν δὲ Θεὸν αἰτιᾶται τῇ καρδίᾳ αὐτοῦ.
\par }{\PP \VS{4}Πλοῦτος προστίθησι φίλους πολλοὺς, ὁ δὲ πτωχὸς καὶ ἀπὸ τοῦ ὑπάρχοντος φίλου λείπεται.
\VS{5}Μάρτυς ψευδὴς οὐκ ἀτιμώρητος ἔσται, ὁ δὲ ἐγκαλῶν ἀδίκως οὐ διαφεύξεται.
\VS{6}Πολλοὶ θεραπεύουσι πρόσωπα βασιλέων, πᾶς δὲ ὁ κακὸς γίνεται ὄνειδος ἀνδρί.
\VS{7}Πᾶς ὃς ἀδελφὸν πτωχὸν μισεῖ, καὶ φιλίας μακρὰν ἔσται· ἔννοια ἀγαθὴ τοῖς εἰδόσιν αὐτὴν ἐγγιεῖ, ἀνὴρ δὲ φρόνιμος εὑρήσει αὐτήν· ὁ πολλὰ κακοποιῶν τελεσιουργεῖ κακίαν, ὃς δὲ ἐρεθίζει λόγους, οὐ σωθήσεται.
\par }{\PP \VS{8}Ὁ κτώμενος φρόνησιν ἀγαπᾷ ἑαυτὸν, ὃς δὲ φυλάσσει φρόνησιν, εὑρήσει ἀγαθά.
\VS{9}Μάρτυς ψευδὴς οὐκ ἀτιμώρητος ἔσται, ὃς δʼ ἂν ἐκκαύσῃ κακίαν, ἀπολεῖται ὑπʼ αὐτῆς.
\VS{10}Οὐ συμφέρει ἄφρονι τρυφὴ, καὶ ἐὰν οἰκέτης ἄρξηται μεθʼ ὕβρεως δυναστεύειν.
\VS{11}Ἐλεήμων ἀνὴρ μακροθυμεῖ, τὸ δὲ καύχημα αὐτοῦ ἐπέρχεται παρανόμοις.
\VS{12}Βασιλέως ἀπειλὴ ὁμοία βρυγμῷ λέοντος· ὥσπερ δὲ δρόσος ἐπὶ χόρτῳ, οὕτως τὸ ἱλαρὸν αὐτοῦ.
\par }{\PP \VS{13}Αἰσχύνη πατρὶ υἱὸς ἄφρων, οὐχ ἁγναὶ εὐχαὶ ἀπὸ μισθώματος ἑταίρας.
\VS{14}Οἶκον καὶ ὕπαρξιν μερίζουσι πατέρες παισὶ, παρὰ δὲ Κυρίου ἁρμόζεται γυνὴ ἀνδρί.
\VS{15}Δειλία κατέχει ἀνδρόγυνον, ψυχὴ δὲ ἀεργοῦ πεινάσει.
\VS{16}Ὃς φυλάσσει ἐντολὴν, τηρεῖ τὴν ἑαυτοῦ ψυχήν· ὁ δὲ καταφρονῶν τῶν ἑαυτοῦ ὁδῶν, ἀπολεῖται.
\VS{17}Δανείζει Θεῷ ὁ ἐλεῶν πτωχὸν, κατὰ δὲ τὸ δόμα αὐτοῦ ἀνταποδώσει αὐτῷ.
\VS{18}Παίδευε υἱόν σου, οὕτως γὰρ ἔσται εὔελπις, εἰς δὲ ὕβριν μὴ ἐπαίρου τῇ ψυχῇ σου.
\VS{19}Κακόφρων ἀνὴρ πολλὰ ζημιωθήσεται· ἐὰν δὲ λοιμεύηται, καὶ τὴν ψυχὴν αὐτοῦ προσθήσει.
\par }{\PP \VS{20}Ἄκουε, υἱὲ, παιδείαν πατρός σου, ἵνα σοφὸς γένῃ ἐπʼ ἐσχάτων σου.
\VS{21}Πολλοὶ λογισμοὶ ἐν καρδίᾳ ἀνδρὸς, ἡ δὲ βουλὴ τοῦ Κυρίου εἰς τὸν αἰῶνα μένει.
\VS{22}Καρπὸς ἀνδρὶ ἐλεημοσύνη, κρείσσων δὲ πτωχὸς δίκαιος ἢ πλούσιος ψευδής.
\VS{23}Φόβος Κυρίου εἰς ζωὴν ἀνδρὶ· ὁ δὲ ἄφοβος αὐλισθήσεται ἐν τόποις οὗ οὐκ ἐπισκοπεῖται γνῶσις.
\VS{24}Ὁ ἐγκρύπτων εἰς τὸν κόλπον αὐτοῦ χεῖρας ἀδίκως, οὐδὲ τῷ στόματι οὐ μὴ προσενείκῃ αὐτάς.
\VS{25}Λοιμοῦ μαστιγουμένου, ἄφρων πανουργότερος γίνεται· ἐὰν δὲ ἐλέγχῃς ἄνδρα φρόνιμον, νοήσει αἴσθησιν.
\par }{\PP \VS{26}Ὁ ἀτιμάζων πατέρα καὶ ἀπωθούμενος μητέρα αὐτοῦ, καταισχυνθήσεται καὶ ἐπονείδιστος ἔσται.
\VS{27}Υἱὸς ἀπολειπόμενος φυλάξαι παιδείαν πατρὸς, μελετήσει ῥήσεις κακάς.
\VS{28}Ὁ ἐγγυώμενος παῖδα ἄφρονα, καθυβρίσει δικαίωμα· στόμα δὲ ἀσεβῶν καταπίεται κρίσεις.
\VS{29}Ἑτοιμάζονται ἀκολάστοις μάστιγες, καὶ τιμωρίαι ὁμοίως ἄφροσιν.

\par }\Chap{20}{\PP \VerseOne{1}Ἀκολάστον οἶνος, καὶ ὑβριστικὸν μέθη, πᾶς δὲ ἄφρων τοιούτοις συμπλέκεται.
\VS{2}Οὐ διαφέρει ἀπειλὴ βασιλέως θυμοῦ λέοντος, ὁ δὲ παροξύνων αὐτὸν ἁμαρτάνει εἰς τὴν ἑαυτοῦ ψυχήν.
\VS{3}Δόξα ἀνδρὶ ἀποστρέφεσθαι λοιδορίας, πᾶς δὲ ἄφρων τοιούτοις συμπλέκεται.
\VS{4}Ὀνειδιζόμενος ὀκνηρὸς οὐκ αἰσχύνεται, ὡσαύτως καὶ ὁ δανειζόμενος σῖτον ἐν ἀμητῷ.
\par }{\PP \VS{5}Ὕδωρ βαθὺ βουλὴ ἐν καρδίᾳ ἀνδρὸς, ἀνὴρ δὲ φρόνιμος ἐξαντλήσει αὐτήν.
\VS{6}Μέγα ἄνθρωπος, καὶ τίμιον ἀνὴρ ἐλεήμων, ἄνδρα δὲ πιστὸν ἔργον εὑρεῖν.
\VS{7}Ὃς ἀναστρέφεται ἄμωμος ἐν δικαιοσύνῃ, μακαρίους τοὺς παῖδας αὐτοῦ καταλείψει.
\VS{8}Ὅταν βασιλεὺς δίκαιος καθίσῃ ἐπὶ θρόνου, οὐκ ἐναντιοῦται ἐν ὀφθαλμοῖς αὐτοῦ πᾶν πονηρόν.
\VS{9}Τίς καυχήσεται ἁγνὴν ἔχειν τὴν καρδίαν; ἢ τίς παῤῥησιάσεται καθαρὸς εἶναι ἀπὸ ἁμαρτιῶν;
\VS{9a}Κακολογοῦντος πατέρα ἢ μητέρα σβεσθήσεται λαμπτὴρ, αἱ δὲ κόραι τῶν ὀφθαλμῶν αὐτοῦ ὄψονται σκότος.
\par }{\PP \VS{9b}Μερὶς ἐπισπουδαζομένη ἐν πρώτοις, ἐν τοῖς τελευταίοις οὐκ εὐλογηθήσεται.
\VS{9c}Μὴ εἴπῃς, τίσομαι τὸν ἐχθρὸν, ἀλλʼ ὑπόμεινον τὸν Κύριον, ἵνα σοι βοηθήσῃ.
\par }{\PP \VS{10}Στάθμιον μέγα καὶ μικρὸν, καὶ μέτρα δισσὰ, ἀκάθαρτα ἐνώπιον Κυρίου καὶ ἀμφότερα, καὶ ὁ ποιῶν αὐτά.
\VS{11}Ἐν τοῖς ἐπιτηδεύμασιν αὐτοῦ συμποδισθήσεται νεανίσκος μετὰ ὁσίου, καὶ εὐθεῖα ἡ ὁδὸς αὐτοῦ.
\VS{12}Οὖς ἀκούει, καὶ ὀφθαλμὸς ὁρᾷ, Κυρίου ἔργα καὶ ἀμφότερα.
\VS{13}Μὴ ἀγάπα καταλαλεῖν, ἵνα μὴ ἐξαρθῇς· διάνοιξον τοὺς ὀφθαλμούς σου, καὶ ἐμπλήσθητι ἄρτων.
\par }{\PP \VS{23}Βδέλυγμα Κυρίῳ δισσὸν στάθμιον, καὶ ζυγὸς δόλιος οὐ καλὸν ἐνώπιον αὐτοῦ.
\VS{24}Παρὰ Κυρίου εὐθύνεται τὰ διαβήματα ἀνδρὶ, θνητὸς δὲ πῶς ἂν νοήσαι τὰς ὁδοὺς αὐτοῦ;
\VS{25}Παγὶς ἀνδρὶ ταχύ τι τῶν ἰδίων ἁγιάσαι, μετὰ γὰρ τὸ εὔξασθαι μετανοεῖν γίνεται.
\VS{26}Λικμήτωρ ἀσεβῶν βασιλεὺς σοφὸς, καὶ ἐπιβαλεῖ αὐτοῖς τροχόν.
\par }{\PP \VS{27}Φῶς Κυρίου πνοὴ ἀνθρώπων, ὃς ἐρευνᾷ ταμιεῖα κοιλίας.
\VS{28}Ἐλεημοσύνη καὶ ἀλήθεια φυλακὴ βασιλεῖ, καὶ περικυκλώσουσιν ἐν δικαιοσύνῃ τὸν θρόνον αὐτοῦ.
\VS{29}Κόσμος νεανίαις σοφία, δόξα δὲ πρεσβυτέρων πολιαί.
\VS{30}Ὑπώπια καὶ συντρίμματα συναντᾷ κακοῖς, πληγαὶ δὲ εἰς ταμιεῖα κοιλίας.

\par }\Chap{21}{\PP \VerseOne{1}Ὥσπερ ὁρμὴ ὕδατος, οὕτως καρδία βασιλέως ἐν χειρὶ Θεοῦ, οὗ ἐὰν θέλων νεῦσαι ἐκεῖ ἔκλινεν αὐτήν.
\VS{2}Πᾶς ἀνὴρ φαίνεται ἑαυτῷ δίκαιος, κατευθύνει δὲ καρδίας Κύριος.
\VS{3}Ποιεῖν δίκαια καὶ ἀληθεύειν, ἀρεστὰ παρὰ Θεῷ μᾶλλον ἢ θυσιῶν αἷμα.
\VS{4}Μεγαλόφρων ἐν ὕβρει θρασυκάρδιος, λαμπτὴρ δὲ ἀσεβῶν ἁμαρτία.
\VS{6}Ὁ ἐνεργῶν θησαυρίσματα γλώσσῃ ψευδεῖ, μάταια διώκει ἐπὶ παγίδας θανάτου.
\VS{7}Ὄλεθρος ἀσεβέσιν ἐπιξενωθήσεται, οὐ γὰρ βούλονται πράσσειν τὰ δίκαια.
\VS{8}Πρὸς τοὺς σκολιοὺς σκολιὰς ὁδοὺς ἀποστέλλει ὁ Θεὸς, ἁγνὰ γὰρ καὶ ὀρθὰ τὰ ἔργα αὐτοῦ.
\VS{9}Κρεῖσσον οἰκεῖν ἐπὶ γωνίας ὑπαίθρου, ἢ ἐν κεκονιαμένοις μετὰ ἀδικίας καὶ ἐν οἴκῳ κοινῷ.
\VS{10}Ψυχὴ ἀσεβοῦς οὐκ ἐλεηθήσεται ὑπʼ οὐδενὸς τῶν ἀνθρώπων.
\VS{11}Ζημιουμένου ἀκολάστου πανουργότερος γίνεται ὁ ἄκακος, συνιῶν δὲ σοφὸς δέξεται γνῶσιν.
\VS{12}Συνιεῖ δίκαιος καρδίας ἀσεβῶν, καὶ φαυλίζει ἀσεβεῖς ἐν κακοῖς.
\par }{\PP \VS{13}Ὃς φράσσει τὰ ὦτα αὐτοῦ τοῦ μὴ ἐπακοῦσαι ἀσθενοῦς, καὶ αὐτὸς ἐπικαλέσεται καὶ οὐκ ἔσται ὁ εἰσακούων.
\VS{14}Δόσις λάθριος ἀνατρέπει ὀργάς, δώρων δὲ ὁ φειδόμενος θυμὸν ἐγείρει ἰσχυρόν.
\VS{15}Εὐφροσύνη δικαίων ποιεῖν κρίμα, ὅσιος δὲ ἀκάθαρτος παρὰ κακούργοις.
\VS{16}Ἀνὴρ πλανώμενος ἐξ ὁδοῦ δικαιοσύνης, ἐν συναγωγῇ γιγάντων ἀναπαύσεται.
\VS{17}Ἀνὴρ ἐνδεὴς ἀγαπᾷ εὐφροσύνην, φιλῶν οἶνον καὶ ἔλαιον εἰς πλοῦτον·
\VS{18}Περικάθαρμα δὲ δικαίου ἄνομος.
\VS{19}Κρεῖσσον οἰκεῖν ἐν τῇ ἐρήμῳ, ἢ μετὰ γυναικὸς μαχίμου καὶ γλωσσώδους καὶ καὶ ὀργίλου.
\VS{20}Θησαυρὸς ἐπιθυμητὸς ἀναπαύσεται ἐπὶ στόματος σοφοῦ, ἄφρονες δὲ ἄνδρες καταπίονται αὐτόν.
\VS{21}Ὁδὸς δικαιοσύνης καὶ ἐλεημοσύνης εὑρήσει ζωὴν καὶ δόξαν.
\VS{22}Πόλεις ὀχυρὰς ἐπέβη σοφὸς, καὶ καθεῖλε τὸ ὀχύρωμα ἐφʼ ᾧ ἐπεποίθεισαν οἱ ἀσεβεῖς.
\VS{23}Ὃς φυλάσσει τὸ στόμα αὐτοῦ καὶ τὴν γλῶσσαν, διατηρεῖ ἐκ θλίψεως τὴν ψυχὴν αὐτοῦ.
\par }{\PP \VS{24}Θρασὺς καὶ αὐθάδης καὶ ἀλαζὼν λοιμὸς καλεῖται, ὃς δὲ μνησικακεῖ παράνομος.
\VS{25}Ἐπιθυμίαι ὀκνηρὸν ἀποκτείνουσιν, οὐ γὰρ προαιροῦνται αἱ χεῖρες αὐτοῦ ποιεῖν τι.
\VS{26}Ἀσεβὴς ἐπιθυμεῖ ὅλην τὴν ἡμέραν ἐπιθυμίας κακὰς, ὁ δὲ δίκαιος ἐλεᾷ καὶ οἰκτείρει ἀφειδῶς.
\VS{27}Θυσίαι ἀσεβῶν βδέλυγμα Κυρίῳ, καὶ γὰρ παρανόμως προσφέρουσιν αὐτάς.
\VS{28}Μάρτυς ψευδὴς ἀπολεῖται, ἀνὴρ δὲ ὑπήκοος φυλασσόμενος λαλήσει.
\VS{29}Ἀσεβὴς ἀνὴρ ἀναιδῶς ὑφίσταται προσώπῳ, ὁ δὲ εὐθὺς αὐτὸς συνιεῖ τὰς ὁδοὺς αὐτοῦ.
\VS{30}Οὐκ ἔστι σοφία, οὐκ ἔστιν ἀνδρεία, οὐκ ἔστι βουλὴ πρὸς τὸν ἀσεβῆ.
\VS{31}Ἵππος ἑτοιμάζεται εἰς ἡμέραν πολέμου, παρὰ δὲ Κυρίου ἡ βοήθεια.

\par }\Chap{22}{\PP \VerseOne{1}Αἱρετώτερον ὄνομα καλὸν ἢ πλοῦτος πολύς, ὑπὲρ δὲ ἀργύριον καὶ χρυσίον χάρις ἀγαθή.
\VS{2}Πλούσιος καὶ πτωχὸς συνήντησαν ἀλλήλοις, ἀμφοτέρους δὲ ὁ κύριος ἐποίησε.
\VS{3}Πανοῦργος ἰδὼν πονηρὸν τιμωρούμενον κραταιῶς αὐτὸς παιδεύεται, οἱ δὲ ἄφρονες παρελθόντες ἐζημιώθησαν.
\VS{4}Γενεὰ σοφίας φόβος Κυρίου, καὶ πλοῦτος, καὶ δόξα, καὶ ζωή.
\VS{5}Τρίβολοι καὶ παγίδες ἐν ὁδοῖς σκολιαῖς, ὁ δὲ φυλάσσων τὴν ἑαυτοῦ ψυχὴν ἀφέξεται αὐτῶν.
\VS{7}Πλούσιοι πτωχῶν ἄρξουσι, καὶ οἰκέται ἰδίοις δεσπόταις δανειοῦσιν.
\par }{\PP \VS{8}Ὁ σπείρων φαῦλα θερίσει κακὰ, πληγὴν δὲ ἔργων αὐτοῦ συντελέσει·
\VS{8a}ἄνδρα ἱλαρὸν καὶ δότην εὐλογεῖ ὁ Θεὸς, ματαιότητα δὲ ἔργων αὐτοῦ συντελέσει.
\VS{9}Ὁ ἐλεῶν πτωχὸν αὐτὸς διατραφήσεται, τῶν γὰρ ἑαυτοῦ ἄρτων ἔδωκε τῷ πτωχῷ·
\VS{9a}νίκην καὶ τιμὴν περιποιεῖται ὁ δῶρα δοὺς, τὴν μέντοι ψυχὴν ἀφαιρεῖται τῶν κεκτημένων.
\VS{10}Ἔκβαλε ἐκ συνεδρίου λοιμὸν, καὶ συνεξελεύσεται αὐτῷ νεῖκος, ὅταν γὰρ καθίσῃ ἐν συνεδρίῳ πάντας ἀτιμάζει.
\par }{\PP \VS{11}Ἀγαπᾷ Κύριος ὁσίας καρδίας, δεκτοὶ δὲ αὐτῷ πάντες ἄμωμοι· χείλεσι ποιμαίνει βασιλεύς.
\VS{12}Οἱ δὲ ὀφθαλμοὶ Κυρίου διατηροῦσιν αἴσθησιν, φαυλίζει δὲ λόγους παράνομος.
\VS{13}Προφασίζεται, καὶ λέγει ὀκνηρὸς, λέων ἐν ταῖς ὁδοῖς, ἐν δὲ ταῖς πλατείαις φονευταί.
\VS{14}Βόθρος βαθὺς στόμα παρανόμου, ὁ δὲ μισηθεὶς ὑπὸ Κυρίου ἐμπεσεῖται εἰς αὐτόν.
\VS{14a}εἰσὶν ὁδοὶ κακαὶ ἐνώπιον ἀνδρὸς, καὶ οὐκ ἀγαπᾷ τοῦ ἀποστρέψαι ἀπʼ αὐτῶν, ἀποστρέφειν δὲ δεῖ ἀπὸ ὁδοῦ σκολιᾶς καὶ κακῆς.
\VS{15}Ἄνοια ἐξῆπται καρδίας νέου, ῥάβδος δὲ καὶ παιδεία μακρὰν ἀπʼ αὐτοῦ.
\par }{\PP \VS{16}Ὁ συκοφαντῶν πένητα, πολλὰ ποιεῖ τὰ ἑαυτοῦ, δίδωσι δὲ πλουσίῳ ἐπʼ ἐλάσσονι.
\par }{\PP \VS{17}Λόγοις σοφῶν παράβαλλε σὸν οὖς, καὶ ἄκουε ἐμὸν λόγον, τὴν δὲ σὴν καρδίαν ἐπίστησον, ἵνα γνῷς ὅτι καλοί εἰσι·
\VS{18}καὶ ἐὰν ἐμβάλῃς αὐτοὺς εἰς τὴν καρδίαν σου, εὐφρανοῦσί σε ἅμα ἐπὶ σοῖς χείλεσιν·
\VS{19}Ἵνα σου γένηται ἐπὶ Κύριον ἡ ἐλπὶς, καὶ γνωρίσῃ σοι τὴν ὁδόν σου.
\VS{20}Καὶ σὺ δὲ ἀπόγραψαι αὐτὰ σεαυτῷ τρισσῶς, εἰς βουλὴν καὶ γνῶσιν ἐπὶ τὸ πλάτος τῆς καρδίας σου.
\VS{21}Διδάσκω οὖν σε ἀληθῆ λόγον, καὶ γνῶσιν ἀγαθὴν ὑπακούειν, τοῦ ἀποκρίνεσθαί σε λόγους ἀληθείας τοῖς προβαλλομένοις σοι.
\par }{\PP \VS{22}Μὴ ἀποβιάζου πένητα, πτωχὸς γὰρ ἐστι, καὶ μὴ ἀτιμάσῃς ἀσθενῆ ἐν πύλαις.
\VS{23}Ὁ γὰρ Κύριος κρινεῖ αὐτοῦ τὴν κρίσιν, καὶ ῥύσῃ σὴν ἄσυλον ψυχήν.
\par }{\PP \VS{24}Μὴ ἴσθι ἑταῖρος ἀνδρὶ θυμώδει, φίλῳ δὲ ὀργίλῳ μὴ συναυλίζου·
\VS{25}μήποτε μάθῃς τῶν ὁδῶν αὐτοῦ, καὶ λάβῃς βρόχους τῇ σῇ ψυχῇ.
\par }{\PP \VS{26}Μὴ δίδου σεαυτὸν εἰς ἐγγύην αἰσχυνόμενος πρόσωπον·
\VS{27}Ἐὰν γὰρ μὴ ἔχῃ πόθεν ἀποτίσῃς, λήψονται τὸ στρῶμα τὸ ὑπὸ τὰς πλευράς σου.
\VS{28}Μὴ μέταιρε ὅρια αἰώνια, ἃ ἔθεντο οἱ πατέρες σου.
\par }{\PP \VS{29}Ὁρατικὸν ἄνδρα καὶ ὀξὺν ἐν τοῖς ἔργοις αὐτοῦ βασιλεῦσι δεῖ παρεστάναι, καὶ μὴ παρεστάναι ἀνδράσι νωθροῖς.

\par }\Chap{23}{\PP \VerseOne{1}Ἐὰν καθίσῃς δειπνεῖν ἐπὶ τραπέζης δυνάστου, νοητῶς νόει τὰ παρατιθέμενά σοι.
\VS{2}Καὶ ἐπίβαλλε τὴν χεῖρά σου, εἰδὼς ὅτι τοιαῦτά σε δεῖ παρασκευάσαι· εἰ δὲ ἀπληστότερος εἶ,
\VS{3}μὴ ἐπιθύμει τῶν ἐδεσμάτων αὐτοῦ, ταῦτα γὰρ ἔχεται ζωῆς ψευδοῦς.
\par }{\PP \VS{4}Μὴ παρεκτείνου πένης ὢν πλουσίῳ, τῇ δὲ σῇ ἐννοίᾳ ἀπόσχου.
\VS{5}Ἐὰν ἐπιστήσῃς τὸ σὸν ὄμμα πρὸς αὐτὸν, οὐδαμοῦ φανεῖται· κατεσκεύασται γὰρ αὐτῷ πτέρυγες ὥσπερ ἀετοῦ, καὶ ὑποστρέφει εἰς τὸν οἶκον τοῦ προεστηκότος αὐτοῦ.
\VS{6}Μὴ συνδείπνει ἀνδρὶ βασκάνῳ, μηδὲ ἐπιθύμει τῶν βρωμάτων αὐτοῦ,
\VS{7}ὃν τρόπον γὰρ εἴ τις καταπίοι τρίχα, οὕτως ἐσθίει καὶ πίνει· μηδὲ πρὸς σὲ εἰσαγάγῃς αὐτὸν, καὶ φάγῃς τὸν ψωμόν σου μετʼ αὐτοῦ,
\VS{8}ἐξεμέσει γὰρ αὐτὸν, καὶ λυμανεῖται τοὺς λόγους σου τοὺς καλούς.
\par }{\PP \VS{9}Εἰς ὦτα ἄφρονος μηδὲν λέγε, μήποτε μυκτηρίσῃ τοὺς συνετοὺς λόγους σου.
\VS{10}Μὴ μεταθῇς ὅρια αἰώνια, εἰς δὲ κτῆμα ὀρφανῶν μὴ εἰσέλθῃς·
\VS{11}Ὁ γὰρ λυτρούμενος αὐτοὺς Κύριος, κραταιός ἐστι, καὶ κρινεῖ τὴν κρίσιν αὐτῶν μετὰ σοῦ.
\VS{12}Δὸς εἰς παιδείαν τὴν καρδίαν σου, τὰ δὲ ὦτά σου ἑτοίμασον λόγοις αἰσθήσεως.
\par }{\PP \VS{13}Μὴ ἀπόσχῃ νήπιον παιδεύειν, ὅτι ἐὰν πατάξῃς αὐτὸν ῥάβδῳ, οὐ μὴ ἀποθάνῃ.
\VS{14}Συ μὲν γὰρ πατάξεις αὐτὸν ῥάβδῳ, τὴν δὲ ψυχὴν αὐτοῦ ἐκ θανάτου ῥύσῃ.
\par }{\PP \VS{15}Υἱὲ, ἐὰν σοφὴ γένηταί σου ἡ καρδία, εὐφρανεῖς καὶ τὴν ἐμὴν καρδίαν,
\VS{16}καὶ ἐνδιατρίψει λόγοις τὰ σὰ χείλη πρὸς τὰ ἐμὰ χείλη ἐὰν ὀρθὰ ὦσι.
\VS{17}Μὴ ζηλούτω ἡ καρδία σου ἁμαρτωλοὺς, ἀλλὰ ἐν φόβῳ Κυρίου ἴσθι ὅλην τὴν ἡμέραν.
\VS{18}Ἐὰν γὰρ τηρήσῃς αὐτὰ, ἔσται σοι ἔκγονα, ἡ δὲ ἐλπίς σου οὐκ ἀποστήσεται.
\par }{\PP \VS{19}Ἄκουε υἱὲ, καὶ σοφὸς γίνου, καὶ κατεύθυνε ἐννοίας σῆς καρδίας.
\VS{20}Μὴ ἴσθι οἰνοπότης, μηδὲ ἐκτείνου συμβολαῖς, κρεῶν τε ἀγορασμοῖς.
\VS{21}Πᾶς γὰρ μέθυσος καὶ πορνοκόπος πτωχεύσει, καὶ ἐνδύσεται διεῤῥηγμένα καὶ ῥακώδη πᾶς ὑπνώδης.
\par }{\PP \VS{22}Ἄκουε, υἱὲ, πατρὸς τοῦ γεννήσαντός σε, καὶ μὴ καταφρόνει ὅτι γεγήρακέ σου ἡ μήτηρ.
\VS{24}Καλῶς ἐκτρέφει πατὴρ δίκαιος, ἐπὶ δὲ υἱῷ σοφῷ εὐφραίνεται ἡ ψυχὴ αὐτοῦ.
\VS{25}Εὐφραινέσθω ὁ πατὴρ καὶ ἡ μήτηρ ἐπὶ σοὶ, καὶ χαιρέτω ἡ τεκοῦσά σε.
\par }{\PP \VS{26}Δός μοι υἱὲ σὴν καρδίαν, οἱ δὲ σοὶ ὀφθαλμοὶ ἐμὰς ὁδοὺς τηρείτωσαν.
\VS{27}Πίθος γὰρ τετρημένος ἐστὶν ἀλλότριος οἶκος, καὶ φρέαρ στενὸν ἀλλότριον.
\VS{28}Οὗτος γὰρ συντόμως ἀπολεῖται, καὶ πᾶς παράνομος ἀναλωθήσεται.
\par }{\PP \VS{29}Τίνι οὐαί; τίνι θόρυβος; τίνι κρίσεις; τίνι δὲ ἀηδίαι καὶ λέσχαι; τίνι συντρίμματα διακενῆς; τίνος πελιδνοὶ οἱ ὀφθαλμοί;
\VS{30}Οὐ τῶν ἐγχρονιζόντων ἐν οἴνοις; οὐ τῶν ἰχνευόντων ποῦ πότοι γίνονται; μὴ μεθύσκεσθε ἐν οἴνοις, ἀλλὰ ὁμιλεῖτε ἀνθρώποις δικαίοις καὶ ὁμιλεῖτε ἐν περιπάτοις.
\VS{31}Ἐὰν γὰρ εἰς τὰς φιάλας καὶ τὰ ποτήρια δῷς τοὺς ὀφθαλμούς σου, ὕστερον περιπατήσεις γυμνότερος ὑπέρου.
\VS{32}Τὸ δὲ ἔσχατον ὥσπερ ὑπὸ ὄφεως πεπληγὼς ἐκτείνεται, καὶ ὥσπερ ὑπὸ κεράστου διαχεῖται αὐτῷ ὁ ἰός.
\par }{\PP \VS{33}Οἱ ὀφθαλμοί σου ὅταν ἴδωσιν ἀλλοτρίαν, τὸ στόμα σου τότε λαλήσει σκολιά.
\VS{34}Καὶ κατακείσῃ ὥσπερ ἐν καρδίᾳ θαλάσσης, καὶ ὥσπερ κυβερνήτης ἐν πολλῷ κλύδωνι.
\VS{35}Ἐρεῖς δὲ, τύπτουσί με καὶ οὐκ ἐπόνεσα, καὶ ἐνέπαιξάν μοι, ἐγὼ δὲ οὐκ ᾔδειν· πότε ὄρθρος ἔσται, ἵνα ἐλθὼν ζητήσω μεθʼ ὧν συνελεύσομαι;

\par }\Chap{24}{\PP \VerseOne{1}Υἱέ, μὴ ζηλώσῃς κακοὺς ἄνδρας, μηδὲ ἐπιθυμήσῃς εἶναι μετʼ αὐτῶν.
\VS{2}Ψευδῆ γὰρ μελετᾷ ἡ καρδία αὐτῶν, καὶ πόνους τὰ χείλη αὐτῶν λαλεῖ.
\VS{3}Μετὰ σοφίας οἰκοδομεῖται οἶκος, καὶ μετὰ συνέσεως ἀνορθοῦται.
\VS{4}Μετὰ αἰσθήσεως ἐμπίμπλανται ταμιεῖα ἐκ παντὸς πλούτου τιμίου καὶ καλοῦ.
\VS{5}Κρείσσων σοφὸς ἰσχυροῦ, καὶ ἀνὴρ φρόνησιν ἔχων γεωργίου μεγάλου.
\VS{6}Μετὰ κυβερνήσεως γίνεται πόλεμος, βοήθεια δὲ μετὰ καρδίας βουλευτικῆς.
\par }{\PP \VS{7}Σοφία καὶ ἔννοια ἀγαθὴ ἐν πύλαις σοφῶν· σοφοὶ οὐκ ἐκκλίνουσιν ἐκ στόματος Κυρίου,
\VS{8}ἀλλὰ λογίζονται ἐν συνεδρίοις· ἀπαιδεύτοις συναντᾷ θάνατος,
\VS{9}ἀποθνήσκει δὲ ἄφρων ἐν ἁμαρτίαις· ἀκαθαρσία δὲ ἀνδρὶ λοιμῷ,
\VS{10}ἐμμολυνθήσεται ἐν ἡμέρᾳ κακῇ, καὶ ἐν ἡμέρᾳ θλίψεως ἕως ἂν ἐκλίπῃ.
\par }{\PP \VS{11}Ῥῦσαι ἀγομένους εἰς θάνατον, καὶ ἐκπρίου κτεινομένους, μὴ φείσῃ.
\VS{12}Ἐὰν δὲ εἴπῃς, οὐκ οἶδα τοῦτον, γίνωσκε, ὅτι Κύριος καρδίας πάντων γινώσκει· καὶ ὁ πλάσας πνοὴν πᾶσιν, αὐτὸς οἶδε πάντα, ὃς ἀποδίδωσιν ἑκάστῳ κατὰ τὰ ἔργα αὐτοῦ.
\VS{13}Φάγε μέλι υἱὲ, ἀγαθὸν γὰρ κηρίον, ἵνα γλυκανθῇ σου ὁ φάρυγξ.
\VS{14}Οὕτως αἰσθητήσῃ σοφίαν τῇ σῇ ψυχῇ· ἐὰν γὰρ εὕρῃς, ἔσται καλὴ ἡ τελευτή σου, καὶ ἐλπίς σε οὐκ ἐγκαταλείψει.
\par }{\PP \VS{15}Μὴ προσαγάγῃς ἀσεβῆ νομῇ δικαίων, μηδὲ ἀπατηθῇς χορτασίᾳ κοιλίας.
\VS{16}Ἑπτάκις γὰρ πεσεῖται δίκαιος καὶ ἀναστήσεται, οἱ δὲ ἀσεβεῖς ἀσθενήσουσιν ἐν κακοῖς.
\VS{17}Ἐὰν πέσῃ ὁ ἐχθρός σου, μὴ ἐπιχαρῇς αὐτῷ, ἐν δὲ τῷ ὑποσκελίσματι αὐτοῦ μὴ ἐπαίρου.
\VS{18}Ὅτι ὄψεται Κύριος καὶ οὐκ ἀρέσει αὐτῷ, καὶ ἀποστρέψει τὸν θυμὸν αὐτοῦ ἀπʼ αὐτοῦ.
\VS{19}Μὴ χαῖρε ἐπὶ κακοποιοῖς, μηδὲ ζήλου ἁμαρτωλούς.
\VS{20}Οὐ γὰρ μὴ γένηται ἔκγονα πονηρῷ, λαμπτὴρ δὲ ἀσεβῶν σβεσθήσεται.
\par }{\PP \VS{21}Φοβοῦ τὸν Θεὸν υἱὲ, καὶ βασιλέα, καὶ μηθʼ ἑτέρῳ αὐτῶν ἀπειθήσῃς.
\VS{22}Ἐξαίφνης γὰρ τίσονται τοὺς ἀσεβεῖς, τὰς δὲ τιμωρίας ἀμφοτέρων τίς γνώσεται;
\par }{\PP \VS{22a}Λόγον φυλασσόμενος υἱὸς ἀπωλείας ἐκτὸς ἔσται, [δεχόμενος δὲ ἐδέξατο αὐτόν·
\VS{22b}μηδὲν ψεῦδος ἀπὸ γλώσσης βασιλεῖ λεγέσθω, καὶ οὐδὲν ψεῦδος ἀπὸ γλώσσης αὐτοῦ οὐ μή ἐξέλθῃ·
\VS{22c}μάχαιρα γλῶσσα βασιλέως καὶ οὐ σαρκίνη, ὃς δʼ ἂν παραδοθῇ συντριβήσεται·
\VS{22d}ἐὰν γὰρ ὀξυνθῇ ὁ θυμὸς αὐτοῦ, σὺν νεύροις ἀνθρώπους ἀναλίσκει,
\VS{22e}καὶ ὀστᾶ ἀνθρώπων κατατρώγει, καὶ συγκαίει ὥσπερ φλὸξ, ὥστε ἄβρωτα εἶναι νεοσσοῖς ἀετῶν·
\VS{22f}τοὺς ἐμοὺς λόγους υἱὲ φοβήθητι, καὶ δεξάμενος αὐτοὺς μετανόει.]
\par }{\PP Τάδε λέγει ὁ ἀνὴρ τοῖς πιστεύουσι Θεῷ, καὶ παύομαι.
\par }{\PP \VS{22g}Ἀφρονέστατος γάρ εἰμι ἁπάντων ἀνθρώπων, καὶ φρόνησις ἀνθρώπων οὐκ ἔστιν ἐν ἐμοί.
\VS{22h}Θεὸς δεδίδαχέ με σοφίαν, καὶ γνῶσιν ἁγίων ἔγνωκα.
\VS{22i}Τίς ἀνέβη εἰς τὸν οὐρανὸν καὶ κατέβη; τίς συνήγαγεν ἀνέμους ἐν κόλπῳ; τίς συνέστρεψεν ὕδωρ ἐν ἱματίῳ; τίς ἐκράτησε πάντων τῶν ἄκρων τῆς γῆς; τί ὄνομα αὐτῷ; ἢ τί ὄνομα τοῖς τέκνοις αὐτοῦ;
\VS{22k}Πάντες γὰρ λόγοι Θεοῦ πεπυρωμένοι, ὑπερασπίζει δὲ αὐτὸς τῶν εὐλαβουμένων αὐτόν.
\VS{22l}Μὴ προσθῇς τοῖς λόγοις αὐτοῦ, ἵνα μὴ ἐλέγξῃ σε, καὶ ψευδὴς γένῃ.
\par }{\PP \VS{22m}Δύο αἰτοῦμαι παρὰ σοῦ, μὴ ἀφέλῃς μου χάριν πρὸ τοῦ ἀποθανεῖν με.
\VS{22n}Μάταιον λόγον καὶ ψευδῆ μακράν μου ποίησον, πλοῦτον δὲ καὶ πενίαν μή μοι δῷς, σύνταξον δέ μοι τὰ δέοντα καὶ τὰ αὐτάρκη·
\VS{22o}Ἵνα μὴ πλησθεὶς ψευδὴς γένωμαι, καὶ εἴπω, τίς με ὁρᾷ; ἢ πενηθεὶς κλέψω, καὶ ὀμόσω τὸ ὄνομα τοῦ Θεοῦ.
\par }{\PP \VS{22p}Μὴ παραδῷς οἰκέτην εἰς χεῖρας δεσπότου, μήποτε καταράσηταί σε καὶ ἀφανισθῇς.
\VS{22q}Ἔκγονον κακὸν πατέρα καταρᾶται, τὴς δὲ μητέρα οὐκ εὐλογεῖ.
\VS{22r}Ἔκγονον κακὸν δίκαιον ἑαυτὸν κρίνει, τὴν δʼ ἔξοδον αὐτοῦ οὐκ ἀπένιψεν.
\VS{22s}Ἔκγονον κακὸν ὑψηλοὺς ὀφθαλμοὺς ἔχει, τοῖς δὲ βλεφάροις αὐτοῦ ἐπαίρεται.
\VS{22t}Ἔκγονον κακὸν μαχαίρας τοὺς ὀδόντας ἔχει, καὶ τὰς μύλας, τομίδας, ὥστε ἀναλίσκειν καὶ κατεσθίειν τοὺς ταπεινοὺς ἀπὸ τῆς γῆς, καὶ τοὺς πένητας αὐτῶν ἐξ ἀνθρώπων.
\par }{\PP \VS{23}Ταῦτα δὲ λέγω ὑμῖν τοῖς σοφοῖς ἐπιγινώσκειν· αἰδεῖσθαι πρόσωπον ἐν κρίσει οὐ καλόν.
\VS{24}Ὁ εἰπὼν τὸν ἀσεβῆ, δίκαιός ἐστιν, ἐπικατάρατος λαοῖς ἔσται καὶ μισητὸς εἰς ἔθνη.
\VS{25}Οἱ δὲ ἐλέγχοντες βελτίους φανοῦνται, ἐπʼ αὐτοὺς δὲ ἥξει εὐλογία·
\VS{26}χείλη δὲ φιλήσουσιν ἀποκρινόμενα λόγους ἀγαθούς.
\VS{27}Ἑτοίμαζε εἰς τὴν ἔξοδον τὰ ἔργα σου, καὶ παρασκευάζου εἰς τὸν ἀγρὸν, καὶ πορεύου κατόπισθέν μου, καὶ ἀνοικοδομήσεις τὸν οἶκόν σου.
\VS{28}Μὴ ἴσθι ψευδὴς μάρτυς ἐπὶ σὸν πολίτην, μηδὲ πλατύνου σοῖς χείλεσι.
\VS{29}Μὴ εἴπῃς, ὃν τρόπον ἐχρήσατό μοι, χρήσομαι αὐτῷ, τίσομαι δὲ αὐτὸν ἅ με ἠδίκησεν.
\VS{30}Ὥσπερ γεώργιον ἀνὴρ ἄφρων, καὶ ὥσπερ ἀμπελὼν ἄνθρωπος ἐνδεὴς φρενῶν.
\VS{31}Ἐὰν ἀφῇς αὐτὸν, χερσωθήσεται καὶ χορτομανήσει ὅλος, καὶ γίνεται ἐκλελειμμένος, οἱ δὲ φραγμοὶ τῶν λίθων αὐτοῦ κατασκάπτονται.
\VS{32}Ὕστερον ἐγὼ μετενόησα, ἐπέβλεψα τοῦ ἐκλέξασθαι παιδείαν.
\VS{33}Ὀλίγον νυστάζω, ὀλίγον δὲ καθυπνῶ, ὀλίγον δὲ ἐναγκαλίζομαι χερσὶ στήθη.
\VS{34}Ἐὰν δὲ τοῦτο ποιῇς, ἥξει προπορευομένη ἡ πενία σου, καὶ ἡ ἔνδειά σου ὥσπερ ἀγαθὸς δρομεύς.
\par }{\PP \VS{35}Τῇ βδέλλῃ τρεῖς θυγατέρες ἦσαν ἀγαπήσει ἀγαπώμεναι, καὶ αἱ τρεῖς αὗται οὐκ ἐνεπίμπλασαν αὐτὴν, καὶ ἡ τετάρτη οὐκ ἠρκέσθη εἰπεῖν, ἱκανόν.
\VS{36}Ἄδης καὶ ἔρως γυναικὸς, καὶ γῆ οὐκ ἐμπιπλαμένη ὕδατος, καὶ ὕδωρ καὶ πῦρ οὐ μὴ εἴπωσιν, ἀρκεῖ.
\par }{\PP \VS{37}Ὀφθαλμὸν καταγελῶντα πατρὸς, καὶ ἀτιμάζοντα γῆρας μητρὸς, ἐκκόψαισαν αὐτὸν κόρακες ἐκ τῶν φαράγγων, καὶ καταφάγοισαν αὐτὸν νεοσσοὶ ἀετῶν.
\VS{38}Τρία δέ ἐστιν ἀδύνατά μοι νοῆσαι, καὶ τὸ τέταρτον οὐκ ἐπιγινώσκω·
\VS{39}Ἴχνη ἀετοῦ πετομένου, καὶ ὁδοὺς ὄφεως ἐπὶ πέτρας, καὶ τρίβους νηὸς ποντοπορούσης, καὶ ὁδοὺς ἀνδρὸς ἐν νεότητι.
\VS{40}Τοιαύτη ὁδὸς γυναικὸς μοιχαλίδος, ἣ ὅτʼ ἂν πράξῃ ἀπονιψαμένη, οὐδέν φησι πεπραχέναι ἄτοπον.
\par }{\PP \VS{41}Διὰ τριῶν σείεται ἡ γῆ, τὸ δὲ τέταρτον οὐ δύναται φέρειν·
\VS{42}Ἐὰν οἰκέτης βασιλεύσῃ, καὶ ἄφρων πλησθῇ σιτίων,
\VS{43}καὶ οἰκέτις ἐὰν ἐκβάλῃ τὴν ἑαυτῆς κυρίαν, καὶ μισητὴ γυνὴ ἐὰν τύχῃ ἀνδρὸς ἀγαθοῦ.
\par }{\PP \VS{44}Τέσσαρα δὲ ἐλάχιστα ἐπὶ τῆς γῆς, ταῦτα δέ ἐστι σοφώτερα τῶν σοφῶν·
\VS{45}Οἱ μύρμηκες οἷς μή ἐστιν ἰσχὺς, καὶ ἑτοιμάζονται θέρους τὴν τροφήν·
\VS{46}Καὶ οἱ χοιρογρύλλιοι ἔθνος οὐκ ἰσχυρὸν, οἳ ἐποιήσαντο ἐν πέτραις τοὺς ἑαυτῶν οἴκους·
\VS{47}Ἀβασίλευτόν ἐστιν ἡ ἀκρὶς, καὶ στρατεύει ἀφʼ ἑνὸς κελεύσματος εὐτάκτως·
\VS{48}Καὶ καλαβώτης χερσὶν ἐρειδόμενος, καὶ εὐάλωτος ὢν, κατοικεῖ ἐν ὀχυρώμασι βασιλέων.
\par }{\PP \VS{49}Τρία δέ ἐστιν ἃ εὐόδως πορεύεται, καὶ τέταρτον ὃ καλῶς διαβαίνει·
\VS{50}Σκύμνος λέοντος ἰσχυρότερος κτηνῶν, ὃς οὐκ ἀποστρέφεται, οὐδὲ καταπτήσσει κτῆνος·
\VS{51}Καὶ ἀλέκτωρ ἐμπεριπατῶν θηλείαις εὔψυχος, καὶ τράγος ἡγούμενος αἰπολίου, καὶ βασιλεὺς δημηγορῶν ἐν ἔθνει.
\par }{\PP \VS{52}Ἐὰν πρόῃ σεαυτὸν ἐν εὐφροσύνῃ, καὶ ἐκτείνῃς τὴν χεῖρά σου μετὰ μάχης, ἀτιμασθήσῃ.
\VS{53}Ἄμελγε γάλα, καὶ ἔσται βούτυρον· ἐὰν δὲ ἐκπιέζῃς μυκτῆρας ἐξελεύσεται αἷμα, ἐὰν δὲ ἐξέλκῃς λόγους, ἐξελεύσονται κρίσεις καὶ μάχαι.
\par }{\PP \VS{54}Οἱ ἐμοὶ λόγοι εἴρηνται ὑπὸ Θεοῦ, βασιλέως χρηματισμὸς, ὃν ἐπαίδευσεν ἡ μήτηρ αὐτοῦ.
\par }{\PP \VS{55}Τί τέκνον τηρήσεις; τί; ῥήσεις Θεοῦ· πρωτογενὲς σοὶ λέγω υἱέ· τί τέκνον ἐμῆς κοιλίας; τί τέκνον ἐμῶν εὐχῶν;
\VS{56}Μὴ δῷς γυναιξὶ σὸν πλοῦτον, καὶ τὸν σὸν νοῦν καὶ βίον εἰς ὑστεροβουλίαν·
\VS{57}μετὰ βουλῆς πάντα ποίει, μετὰ βουλῆς οἰνοπότει. Οἱ δυνάσται θυμώδεις εἰσὶν, οἶνον δὲ μὴ πινέτωσαν,
\VS{58}ἵνα μὴ πιόντες ἐπιλάθωνται τῆς σοφίας, καὶ ὀρθὰ κρῖναι οὐ μὴ δύνωνται τοὺς ἀσθενεῖς.
\VS{59}Δίδοτε μέθην τοῖς ἐν λύπαις, καὶ οἶνον πίνειν τοῖς ἐν ὀδύναις,
\VS{60}ἵνα ἐπιλάθωνται τῆς πενίας, καὶ τῶν πόνων μὴ μνησθῶσιν ἔτι.
\VS{61}Ἄνοιγε σὸν στόμα λόγῳ Θεοῦ, καὶ κρίνε πάντας ὑγιῶς.
\VS{62}Ἄνοιγε σὸν στόμα καὶ κρίνε δικαίως, διάκρινε δὲ πένητα καὶ ἀσθενῆ.

\par }\Chap{25}{\PP \VerseOne{1}Αὗται αἱ παιδεῖαι Σαλωμῶντος αἱ ἀδιάκριτοι, ἃς ἐξεγράψαντο οἱ φίλοι Ἐζεκίου τοῦ βασιλέως τῆς Ἰουδαίας.
\par }{\PP \VS{2}Δόξα Θεοῦ κρύπτει λόγον, δόξα δὲ βασιλέως τιμᾷ πράγματα.
\VS{3}Οὐρανὸς ὑψηλὸς, γῆ δὲ βαθεῖα, καρδία δὲ βασιλέως ἀνεξέλεγκτος.
\VS{4}Τύπτε ἀδόκιμον ἀργύριον, καὶ καθαρισθήσεται καθαρὸν ἅπαν.
\VS{5}Κτεῖνε ἀσεβεῖς ἐκ προσώπου βασιλέως, καὶ κατορθώσει ἐν δικαιοσύνῃ ὁ θρόνος αὐτοῦ.
\par }{\PP \VS{6}Μὴ ἀλαζονεύου ἐνώπιον βασιλέως, μηδὲ ἐν τόποις δυναστῶν ὑφίστασο·
\VS{7}Κρεῖσσον γάρ σοι τὸ ῥηθῆναι, ἀνάβαινε πρὸς μὲ, ἢ ταπεινῶσαί σε ἐν προσώπῳ δυνάστου· ἃ εἶδον οἱ ὀφθαλμοί σου λέγε.
\par }{\PP \VS{8}Μὴ πρόσπιπτε εἰς μάχην ταχέως, ἵνα μὴ μεταμεληθῇς ἐπʼ ἐσχάτων· ἡνίκα ἄν σε ὀνειδίσῃ ὁ σὸς φίλος,
\VS{9}ἀναχώρει εἰς τὰ ὀπίσω· μὴ καταφρόνει,
\VS{10}μή σε ὀνειδίσῃ μὲν ὁ φίλος, ἡ δὲ μάχη σου καὶ ἡ ἔχθρα οὐκ ἀπέσται, ἀλλὰ ἔσται σοι ἴση θανάτῳ·
\VS{10a}χάρις καὶ φιλία ἐλευθεροῖ, ἃς τήρησον σεαυτῷ, ἵνα μὴ ἐπονείδιστος γένῃ, ἀλλὰ φύλαξον τὰς ὁδούς σου εὐσυναλλάκτως.
\par }{\PP \VS{11}Μῆλον χρυσοῦν ἐν ὁρμίσκῳ σαρδίου, οὕτως εἰπεῖν λόγον.
\VS{12}Εἰς ἐνώτιον χρυσοῦν καὶ σάρδιον πολυτελὲς δέδεται, λόγος σοφὸς εἰς εὐήκοον οὖς.
\VS{13}Ὥσπερ ἔξοδος χιόνος ἐν ἀμητῷ κατὰ καῦμα ὠφελεῖ, οὕτως ἄγγελος πιστὸς τοὺς ἀποστείλαντας αὐτόν· ψυχὰς γὰρ τῶν αὐτῷ χρωμένων ὠφελεῖ.
\par }{\PP \VS{14}Ὥσπερ ἄνεμοι καὶ νέφη καὶ ὑετοὶ, ἐπιφανέστατα, οὕτως ὁ καυχώμενος ἐπὶ δόσει ψευδεῖ.
\VS{15}Ἐν μακροθυμίᾳ εὐοδία βασιλεῦσι, γλῶσσα δὲ μαλακὴ συντρίβει ὀστᾶ.
\VS{16}Μέλι εὑρὼν φάγε τὸ ἱκανὸν, μή ποτε πλησθεὶς ἐξεμέσῃς.
\VS{17}Σπάνιον εἴσαγε σὸν πόδα πρὸς σεαυτοῦ φίλον, μή ποτε πλησθείς σου μισήσῃ σε.
\VS{18}Ῥόπαλον καὶ μάχαιρα καὶ τόξευμα ἀκιδωτὸν, οὕτως καὶ ἀνὴρ ὁ καταμαρτυρῶν τοῦ φίλου αὐτοῦ μαρτυρίαν ψευδῆ.
\VS{19}Οδὸς κακοῦ καὶ ποὺς παρανόμου ὀλεῖται ἐν ἡμέρᾳ κακῇ.
\par }{\PP \VS{20}Ὥσπερ ὄξος ἕλκει ἀσύμφορον, οὕτως προσπεσὸν πάθος ἐν σώματι καρδίαν λυπεῖ·
\VS{20a}ὥσπερ σὴς ἐν ἱματίῳ καὶ σκώληξ ξύλῳ, οὕτως λύπη ἀνδρὸς βλάπτει καρδίαν.
\par }{\PP \VS{21}Ἐὰν πεινᾷ ὁ ἐχθρός σου, ψώμιζε αὐτὸν, ἐὰν διψᾷ, πότιζε αὐτόν·
\VS{22}Τοῦτο γὰρ ποιῶν ἄνθρακας πυρὸς σωρεύσεις ἐπὶ τὴν κεφαλὴν αὐτοῦ, ὁ δὲ Κύριος ἀνταποδώσει σοι ἀγαθά.
\VS{23}Ἄνεμος Βορέας ἐξεγείρει νέφη, πρόσωπον δὲ ἀναιδὲς γλῶσσαν ἐρεθίζει·
\VS{24}Κρεῖσσον οἰκεῖν ἐπὶ γωνίας δώματος, ἢ μετὰ γυναικὸς λοιδόρου ἐν οἰκίᾳ κοινῇ.
\VS{25}Ὥσπερ ὕδωρ ψυχρὸν ψυχῇ διψώσῃ προσηνὲς, οὕτως ἀγγελία ἀγαθὴ ἐκ γῆς μακρόθεν.
\VS{26}Ὥσπερ εἴ τις πηγὴν φράσσοι καὶ ὕδατος ἔξοδον λυμαίνοιτο, οὕτως ἄκοσμον δίκαιον πεπτωκέναι ἐνώπιον ἀσεβοῦς.
\VS{27}Ἐσθίειν μέλι πολὺ οὐ καλὸν, τιμᾷν δὲ χρὴ λόγους ἐνδόξους.
\VS{28}Ὥσπερ πόλις τὰ τείχη καταβεβλημένη καὶ ἀτείχιστος, οὕτως ἀνὴρ ὃς οὐ μετὰ βουλῆς τι πράσσει.

\par }\Chap{26}{\PP \VerseOne{1}Ὥσπερ δρόσος ἐν ἀμητῷ, καὶ ὥσπερ ὑετὸς ἐν θέρει, οὕτως οὐκ ἔστιν ἄφρουι τιμή.
\VS{2}Ὥσπερ ὄρνεα πέταται καὶ στρουθοί, οὕτως ἀρὰ ματαία οὐκ ἐπελεύσεται οὐδενί.
\VS{3}Ὥσπερ μάστιξ ἵππῳ καὶ κέντρον ὄνῳ, οὕτως ῥάβδος ἔθνει παρανόμῳ.
\VS{4}Μὴ ἀποκρίνου ἄφρονι πρὸς τὴν ἐκείνου ἀφροσύνην, ἵνα μὴ ὅμοιος γένῃ αὐτῷ.
\VS{5}Ἀλλὰ ἀποκρίνου ἄφρονι κατὰ τὴν ἀφροσύνην αὐτοῦ, ἵνα μὴ φαίνηται σοφὸς παρʼ ἑαυτῷ.
\VS{6}Ἐκ τῶν ὁδῶν ἑαυτοῦ ὄνειδος ποιεῖται ὁ ἀποστείλας διʼ ἀγγέλου ἄφρονος λόγον.
\VS{7}Ἄφελοῦ πορείαν σκελῶν, καὶ παρανομίαν ἐκ στόματος ἀφρόνων.
\VS{8}Ὃς ἀποδεσμεύει λίθον ἐν σφενδόνῃ, ὅμοιός ἐστι τῷ διδόντι ἄφρονι δόξαν.
\VS{9}Ἄκανθαι φύονται ἐν χειρὶ μεθύσου, δουλεία δὲ ἐν χειρὶ τῶν ἀφρόνων.
\VS{10}Πολλὰ χειμάζεται πᾶσα σὰρξ ἀφρόνων, συντρίβεται γὰρ ἡ ἔκστασις αὐτῶν.
\VS{11}Ὥσπερ κύων ὅταν ἐπέλθῃ ἐπὶ τὸν ἑαυτοῦ ἔμετον καὶ μισητὸς γένηται, οὕτως ἄφρων τῇ ἑαυτοῦ κακίᾳ ἀναστρέψας ἐπὶ τὴν ἑαυτοῦ ἁμαρτίαν·
\VS{11a}ἔστιν αἰσχύνη ἐπάγουσα ἁμαρτίαν, καὶ ἐστιν αἰσχύνη δόξα καὶ χάρις.
\VS{12}Εἶδον ἄνδρα δόξαντα παρʼ αὐτῷ σοφὸν εἶναι, ἐλπίδα μέντοι ἔσχε μᾶλλον ἄφρων αὐτοῦ.
\VS{13}Λέγει ὀκνηρὸς ἀποστελλόμενος εἰς ὁδὸν, λέων ἐν ταῖς ἐν δὲ ταῖς πλατείαις φονευταί.
\par }{\PP \VS{14}Ὥσπερ θύρα στρέφεται ἐπὶ τοῦ στρόφιγγος, οὕτως ὀκνηρὸς ἐπὶ τῆς κλίνης αὐτοῦ.
\VS{15}Κρύψας ὀκνηρὸς τὴν χεῖρα ἐν τῷ κόλπῳ αὐτοῦ, οὐ δυνήσεται ἐπενεγκεῖν ἐπὶ στόμα.
\VS{16}Σοφώτερος ἑαυτῷ ὀκνηρὸς φαίνεται, τοῦ ἐν πλησμονῇ ἀποκομίζοντος ἀγγελίαν.
\par }{\PP \VS{17}Ὥσπερ ὁ κρατῶν κέρκου κυνὸς, οὕτως ὁ προεστὼς ἀλλοτρίας κρίσεως.
\VS{18}Ὥσπερ οἱ ἰώμενοι προβάλλουσι λόγους εἰς ἀνθρώπους, ὁ δὲ ἀπαντήσας τῷ λόγῳ πρῶτος ὑποσκελισθήσεται·
\VS{19}Οὕτως πάντες οἱ ἐνεδρεύοντες τοὺς ἑαυτῶν φίλους, ὅταν δὲ ὁραθῶσι, λέγουσιν, ὅτι παίζων ἔπραξα.
\VS{20}Ἐν πολλοῖς ξύλοις θάλλει πῦρ, ὅπου δὲ οὐκ ἔστι δίθυμος, ἡσυχάζει μάχη.
\VS{21}Ἐσχάρα ἄνθραξι καὶ ξύλα πυρὶ, ἀνὴρ δὲ λοίδορος εἰς ταραχὴν μάχης.
\VS{22}Λόγοι κερκώπων μαλακοὶ, οὗτοι δὲ τύπτουσιν εἰς ταμιεῖα σπλάγχνων.
\par }{\PP \VS{23}Ἀργύριον διδόμενον μετὰ δόλου, ὥσπερ ὄστρακον ἡγητέον· χείλη λεῖα καρδίαν καλύπτει λυπηράν.
\VS{24}Χείλεσι πάντα ἐπινεύει ἀποκλαιόμενος ἐχθρὸς, ἐν δὲ τῇ καρδίᾳ τεκταίνεται δόλους.
\VS{25}Ἐάν σου δέηται ὁ ἐχθρὸς μεγάλῃ τῇ φωνῇ, μὴ πεισθῇς, ἑπτὰ γάρ πονηρίαι ἐν τῇ ψυχῇ αὐτοῦ.
\VS{26}Ὁ κρύπτων ἔχθραν συνίστησι δόλον, ἐκκαλύπτει δὲ τὰς ἑαυτοῦ ἁμαρτίας εὔγνωστος ἐν συνεδρίοις.
\VS{27}Ὁ ὀρύσσων βόθρον τῷ πλησίον, ἐμπεσεῖται εἰς αὐτόν· ὁ δὲ κυλίων λίθον, ἐφʼ ἑαυτὸν κυλίει.
\VS{28}Γλῶσσα ψευδὴς μισεῖ ἀλήθειαν, στόμα δὲ ἄστεγον ποιεῖ ἀκαταστασίας.

\par }\Chap{27}{\PP \VerseOne{1}Μὴ καυχῶ τὰ εἰς αὔριον, οὐ γὰρ γινώσκεις τί τέξεται ἡ ἐπιοῦσα.
\VS{2}Ἐγκωμιαζέτω σε ὁ πέλας καὶ μὴ τὸ σὸν στόμα, ἀλλότριος καὶ μὴ τὰ σὰ χείλη.
\VS{3}Βαρὺ λίθος καὶ δυσβάστακτον ἄμμος, ὀργὴ δὲ ἄφρονος βαρυτέρα ἀμφοτέρων.
\VS{4}Ἀνελεήμων θυμὸς καὶ ὀξεῖα ὀργὴ, ἀλλʼ οὐδὲν ὑφίσταται ζῆλος.
\VS{5}Κρείσσους ἔλεγχοι ἀποκεκαλυμμένοι κρυπτομένης φιλίας.
\VS{6}Ἀξιοπιστότερά ἐστι τραύματα φίλου, ἢ ἑκούσια φιλήματα ἐχθροῦ.
\par }{\PP \VS{7}Ψυχὴ ἐν πλησμονῇ οὖσα κηρίοις ἐμπαίζει, ψυχῇ δὲ ἐνδεεῖ καὶ τὰ πικρὰ γλυκέα φαίνεται.
\VS{8}Ὥσπερ ὅταν ὄρνεον καταπετασθῇ ἐκ τῆς ἰδίας νοσσιᾶς, οὕτως ἄνθρωπος δουλοῦται ὅταν ἀποξενωθῇ ἐκ τῶν ἰδίων τόπων.
\VS{9}Μύροις καὶ οἴνοις καὶ θυμιάμασι τέρπεται καρδία, καταῤῥήγνυται δὲ ὑπὸ συμπτωμάτων ψυχή.
\par }{\PP \VS{10}Φίλον σὸν ἢ φίλον πατρῷον μὴ ἐγκαταλίπῃς, εἰς δὲ τὸν οἶκον τοῦ ἀδελφοῦ σου μὴ εἰσέλθῃς ἀτυχῶν· κρείσσων φίλος ἐγγὺς, ἢ ἀδελφὸς μακρὰν οἰκῶν.
\VS{11}Σοφὸς γίνου υἱὲ, ἵνα σου εὐφραίνηται ἡ καρδία, καὶ ἀπόστρεψον ἀπὸ σοῦ ἐπονειδίστους λόγους.
\VS{12}Πανοῦργος κακῶν ἐπερχομένων ἀπεκρύβη, ἄφρονες δὲ ἐπελθόντες ζημίαν τίσουσιν.
\VS{13}Ἀφελοῦ τὸ ἱμάτιον αὐτοῦ, παρῆλθε γὰρ ὑβριστὴς, ὅστις τὰ ἀλλότρια λυμαίνεται.
\VS{14}Ὃς ἂν εὐλογῇ θίλον τοπρωῒ μεγάλῃ τῇ φωνῇ, καταρωμένου οὐδὲν διαφέρειν δόξει.
\par }{\PP \VS{15}Σταγόνες ἐκβάλλουσιν ἄνθρωπον ἐν ἡμέρᾳ χειμερινῇ ἐκ τοῦ οἴκου αὐτοῦ, ὡσαύτως καὶ γυνὴ λοίδορος ἐκ τοῦ ἰδίου οἴκου.
\VS{16}Βορέας σκληρὸς ἄνεμος, ὀνόματι δὲ ἐπιδέξιος καλεῖται.
\VS{17}Σίδηρος σίδηρον ὀξύνει, ἀνὴρ δὲ παροξύνει πρόσωπον ἑταίρου.
\VS{18}Ὃς φυτεύει συκὴν φάγεται τοὺς καρποὺς αὐτῆς, ὃς δὲ φυλάσσει τὸν ἑαυτοῦ κύριον τιμηθήσεται.
\VS{19}Ὥσπερ οὐχ ὅμοια πρόσωπα προσώποις, οὕτως οὐδὲ αἱ διάνοιαι τῶν ἀνθρώπων.
\VS{20}Ἅδης καὶ ἀπώλεια οὐκ ἐμπίμπλανται, ὡσαύτως καὶ οἱ ὀφθαλμοὶ τῶν ἀνθρώπων ἄπληστοι·
\VS{20a}βδέλυγμα Κυρίῳ στηρίζων ὀφθαλμὸν, καὶ οἱ ἀπαίδευτοι ἀκρατεῖς γλώσσῃ.
\VS{21}Δοκίμιον ἀργυρίῳ καὶ χρυσῷ πύρωσις, ἀνὴρ δὲ δοκιμάζεται διὰ στόματος ἐγκωμιαζόντων αὐτόν.
\VS{21a}καρδία ἀνόμου ἐκζητεῖ κακὰ, καρδία δὲ εὐθὴς ζητεῖ γνῶσιν.
\VS{22}Ἐὰν μαστιγοῖς ἄφρονα ἐν μέσῳ συνεδρίου ἀτιμάζων, οὐ μὴ περιέλῃς τὴν ἀφροσύνην αὐτοῦ.
\par }{\PP \VS{23}Γνωστῶς ἐπιγνώσῃ ψυχὰς ποιμνίου σου, καὶ ἐπιστήσεις καρδίαν σου σαῖς ἀγέλαις.
\VS{24}Ὅτι οὐκ εἰς τὸν αἰῶνα ἀνδρὶ κράτος καὶ ἰσχὺς, οὐδὲ παραδίδωσιν ἐκ γενεᾶς εἰς γενεάν.
\VS{25}Ἐπιμελοῦ τῶν ἐν τῷ πεδίῳ χλωρῶν, καὶ κερεῖς πόαν, καὶ σύναγε χόρτον ὀρεινὸν,
\VS{26}ἵνα ἔχῃς πρόβατα εἰς ἱματισμόν· τίμα πεδίον, ἵνα ὠσί σοι ἄρνες.
\VS{27}Υἱὲ, παρʼ ἐμοῦ ἔχεις ῥήσεις ἰσχυρὰς εἰς τὴν ζωήν σου, καὶ εἰς τὴν ζωὴν σῶν θεραπόντων.

\par }\Chap{28}{\PP \VerseOne{1}Φεύγει ἀσεβὴς μηδενὸς διώκοντος, δίκαιος δὲ ὥσπερ λέων πέποιθε.
\VS{2}Διʼ ἁμαρτίας ἀσεβῶν κρίσεις ἐγείρονται, ἀνὴρ δὲ πανοῦργος κατασβέσει αὐτάς.
\VS{3}Ἀνδρεῖος ἐν ἀσεβείαις συκοφαντεῖ πτωχούς· ὥσπερ ὑετὸς λάβρος καὶ ἀνωφελὴς,
\VS{4}οὕτως οἱ ἐγκαταλείποντες τὸν νόμον ἐγκωμιάζουσιν ἀσέβειαν· οἱ δὲ ἀγαπῶντες τὸν νόμον, περιβάλλουσιν ἑαυτοῖς τεῖχος.
\VS{5}Ἄνδρες κακοὶ οὐ συνήσουσι κρίμα, οἱ δὲ ζητοῦντες τὸν Κύριον συνήσουσιν ἐν παντί.
\par }{\PP \VS{6}Κρείσσων πτωχὸς πορευόμενος ἐν ἀληθείᾳ, πλουσίου ψευδοῦς.
\VS{7}Φυλάσσει νόμον υἱὸς συνετὸς, ὃς δὲ ποιμαίνει ἀσωτίαν ἀτιμάζει πατέρα.
\VS{8}Ὁ πληθύνων τὸν πλοῦτον αὐτοῦ μετὰ τόκων καὶ πλεονασμῶν, τῷ ἐλεῶντι πτωχοὺς συνάγει αὐτόν.
\VS{9}Ὁ ἐκκλίνων τὸ οὖς αὐτοῦ μὴ εἰσακοῦσαι νόμου, καὶ αὐτὸς τὴν προσευχὴν αὐτοῦ ἐβδέλυκται.
\par }{\PP \VS{10}Ὃς πλανᾷ εὐθεῖς ἐν ὁδῷ κακῇ, εἰς διαφθορὰν αὐτὸς ἐμπεσεῖται· οἱ δὲ ἄνομοι διελεύσονται ἀγαθὰ, καὶ οὐκ εἰσελεύσονται εἰς αὐτά.
\VS{11}Σοφὸς παρʼ ἑαυτῷ ἀνὴρ πλούσιος, πένης δὲ νοήμων καταγνώσεται αὐτοῦ.
\VS{12}Διὰ βοήθειαν δικαίων πολλὴ γίνεται δόξα, ἐν δὲ τόποις ἀσεβῶν ἁλίσκονται ἄνθρωποι.
\par }{\PP \VS{13}Ὁ ἐπικαλύπτων ἀσέβειαν ἑαυτοῦ οὐκ εὐοδωθήσεται, ὁ δὲ ἐξηγούμενος ἐλέγχους ἀγαπηθήσεται.
\VS{14}Μακάριος ἀνὴρ ὃς καταπτήσσει πάντα διʼ εὐλάβειαν, ὁ δὲ σκληρὸς τὴν καρδίαν ἐμπεσεῖται κακοῖς.
\VS{15}Λέων πεινῶν καὶ λύκος διψῶν, ὃς τυραννεῖ, πτωχὸς ὢν, ἔθνους πενιχροῦ.
\VS{16}Βασιλεὺς ἐνδεὴς προσόδων μέγας συκοφάντης, ὁ δὲ μισῶν ἀδικίαν μακρὸν χρόνον ζήσεται.
\par }{\PP \VS{17}Ἄνδρα τὸν ἐν αἰτίᾳ φόνου ὁ ἐγγυώμενος, φυγὰς ἔσται καὶ οὐκ ἐν ἀσφαλείᾳ·
\VS{17a}παίδευε υἱὸν καὶ ἀγαπήσει σε, καὶ δώσει κόσμον τῇ σῇ ψυχῇ, οὐ μὴ ὑπακούσει ἔθνει παρανόμῳ.
\VS{18}Ὁ πορευόμενος δικαίως βεβοήθηται, ὁ δὲ σκολιαῖς ὁδοῖς πορευόμενος ἐμπλακήσεται.
\VS{19}Ὁ ἐργαζόμενος τὴν ἑαυτοῦ γῆν πλησθήσεται ἄρτων, ὁ δὲ διώκων σχολὴν πλησθήσεται πενίας.
\par }{\PP \VS{20}Ἀνὴρ ἀξιόπιστος πολλὰ εὐλογηθήσεται, ὁ δὲ κακὸς οὐκ ἀτιμώρητος ἔσται.
\VS{21}Ὃς οὐκ αἰσχύνεται πρόσωπα δικαίων, οὐκ ἀγαθὸς, ὁ τοιοῦτος ψωμοῦ ἄρτου ἀποδώσεται ἄνδρα.
\VS{22}Σπεύδει πλουτεῖν ἀνὴρ βάσκανος, καὶ οὐκ οἶδεν ὅτι ἐλεήμων κρατήσει αὐτοῦ.
\par }{\PP \VS{23}Ὁ ἐλέγχων ἀνθρώπου ὁδοὺς, χάριτας ἕξει μᾶλλον τοῦ γλωσσοχαριτοῦντος.
\VS{24}Ὃς ἀποβάλλεται πατέρα ἢ μητέρα, καὶ δοκεῖ μὴ ἁμαρτάνειν, οὗτος κοινωνός ἐστιν ἀνδρὸς ἀσεβοῦς.
\VS{25}Ἄπιστος ἀνὴρ κρίνει εἰκῆ, ὃς δὲ πέποιθεν ἐπὶ Κύριον ἐν ἐπιμελείᾳ ἔσται.
\VS{26}Ὃς πέποιθε θρασείᾳ καρδίᾳ, ὁ τοιοῦτος ἄφρων, ὃς δὲ πορεύεται σοφίᾳ σωθήσεται.
\VS{27}Ὃς δίδωσι πτωχοῖς οὐκ ἐνδεηθήσεται, ὃς δὲ ἀποστρέφει τὸν ὀφθαλμὸν αὐτοῦ ἐν πολλῇ ἀπορίᾳ ἔσται.
\VS{28}Ἐν τόποις ἀσεβῶν στένουσι δίκαιοι, ἐν δὲ τῇ ἐκείνων ἀπωλείᾳ πληθυνθήσονται δίκαιοι.

\par }\Chap{29}{\PP \VerseOne{1}Κρείσσων ἀνὴρ ἐλέγχων ἀνδρὸς σκληροτραχήλου, ἐξαπίνης γὰρ φλεγομένου αὐτοῦ οὐκ ἔστιν ἴασις.
\VS{2}Ἐγκωμιαζομένων δικαίων εὐφρανθήσονται λαοὶ, ἀρχόντων δὲ ἀσεβῶν στένουσιν ἄνδρες.
\VS{3}Ἀνδρὸς φιλοῦντος σοφίαν εὐφραίνεται πατὴρ αὐτοῦ, ὃς δὲ ποιμαίνει πόρνας ἀπολεῖ πλοῦτον.
\VS{4}Βασιλεὺς δίκαιος ἀνίστησι χώραν, ἀνὴρ δὲ παράνομος κατασκάπτει.
\VS{5}Ὃς παρασκευάζεται ἐπὶ πρόσωπον τοῦ ἑαυτοῦ φιλου δίκτυον, περιβάλλει αὐτὸ τοῖς ἑαυτοῦ ποσίν.
\VS{6}Ἁμαρτάνοντι ἀνδρὶ μεγάλη παγὶς, δίκαιος δὲ ἐν χαρᾷ καὶ ἐν εὐφροσύνῃ ἔσται.
\VS{7}Ἐπίσταται δίκαιος κρίνειν πενιχροῖς, ὁ δὲ ἀσεβὴς οὐ νοεῖ γνῶσιν, καὶ πτωχῷ οὐχ ὑπάρχει νοῦς ἐπιγνώμων.
\par }{\PP \VS{8}Ἄνδρες ἄνομοι ἐξέκαυσαν πόλιν, σοφοὶ δὲ ἀπέστρεψαν ὀργήν.
\VS{9}Ἀνὴρ σοφὸς κρινεῖ ἔθνη, ἀνὴρ δὲ φαῦλος ὀργιζόμενος καταγελᾶται καὶ οὐ καταπτήσσει.
\VS{10}Ἄνδρες αἱμάτων μέτοχοι μισοῦσιν ὅσιον, οἱ δὲ εὐθεῖς ἐκζητήσουσι ψυχὴν αὐτοῦ.
\VS{11}Ὅλον τὸν θυμὸν αὐτοῦ ἐκφέρει ἄφρων, σοφὸς δὲ ταμιεύεται κατὰ μέρος.
\VS{12}Βασιλέως ὑπακούοντος λόγον ἄδικον, πάντες οἱ ὑπʼ αὐτὸν παράνομοι.
\VS{13}Δανιστοῦ καὶ χρεωφειλέτου ἀλλήλοις συνελθόντων, ἐπισκοπὴν ἀμφοτέρων ποιεῖται ὁ Κύριος.
\VS{14}Βασιλέως ἐν ἀληθείᾳ κρίνοντος πτωχοὺς, ὁ θρόνος αὐτοῦ εἰς μαρτύριον κατασταθήσεται.
\VS{15}Πληγαὶ καὶ ἔλεγχοι διδόασι σοφίαν, παῖς δὲ πλανώμενος αἰσχύνει γονεῖς αὐτοῦ.
\VS{16}Πολλῶν ὄντων ἀσεβῶν πολλαὶγίνονται ἁμαρτίαι, οἱ δὲ δίκαιοι ἐκείνων πιπτόντων κατάφοβοι γίνονται.
\par }{\PP \VS{17}Παίδευε υἱόν σου, καὶ ἀναπαύσει σε, καὶ δώσει κόσμον τῇ ψυχῇ σου.
\VS{18}Οὐ μὴ ὑπάρξῃ ἐξηγητὴς ἔθνει παρανόμῳ, ὁ δὲ φυλάσσων τὸν νόμον μακαριστός.
\VS{19}Λόγοις οὐ παιδευθήσεται οἰκέτης σκληρός· ἐὰν γὰρ καὶ νοήσῃ, ἀλλʼ οὐχ ὑπακούσεται.
\VS{20}Ἐὰν ἴδῃς ἄνδρα ταχὺν ἐν λόγοις, γίνωσκε ὅτι ἐλπίδα ἔχει μᾶλλον ὁ ἄφρων αὐτοῦ.
\VS{21}Ὃς κατασπαταλᾷ ἐκ παιδὸς, οἰκέτης ἔσται, ἔσχατον δὲ ὀδυνηθήσεται ἐφʼ ἑαυτῷ·
\VS{22}Ἀνὴρ θυμώδης ἐγείρει νεῖκος, ἀνὴρ δὲ ὀργίλος ἐξώρυξεν ἁμαρτίαν.
\VS{23}Ὕβρις ἄνδρα ταπεινοῖ, τοὺς δὲ ταπεινόφρονας ἐρείδει δόξῃ Κύριος.
\par }{\PP \VS{24}Ὃς μερίζεται κλέπτῃ, μισεῖ τὴν ἑαυτοῦ ψυχήν· ἐὰν δὲ ὅρκου προτεθέντος ἀκούσαντες μὴ ἀναγγείλωσι,
\VS{25}φοβηθέντες καὶ αἰσχυνθέντες ἀνθρώπους ὑπεσκελίσθησαν, ὁ δὲ πεποιθὼς ἐπὶ Κυρίῳ εὐφρανθήσεται· ἀσέβεια ἀνδρὶ δίδωσι σφάλμα, ὃς δὲ πέποιθεν ἐπὶ τῷ δεσπότῃ σωθήσεται.
\VS{26}Πολλοὶ θεραπεύουσι πρόσωπα ἡγουμένων, παρὰ δὲ Κυρίου γίνεται τὸ δίκαιον ἀνδρί.
\VS{27}Βδέλυγμα δίκαιος ἀνὴρ ἀνδρὶ ἀδίκῳ, βδέλυγμα δὲ ἀνόμῳ κατευθύνουσα ὁδός.


\par }\Chap{31}{\PP \VS{10}Γυναῖκα ἀνδρείαν τίς εὑρήσει; τιμιωτέρα δέ ἐστι λίθων πολυτελῶν ἡ τοιαύτη.
\VS{11}Θαρσεῖ ἐπʼ αὐτῇ ἡ καρδία τοῦ ἀνδρὸς αὐτῆς· ἡ τοιαύτη καλῶν σκύλων οὐκ ἀπορήσει.
\VS{12}Ἐνεργεῖ γὰρ τῷ ἀνδρὶ εἰς ἀγαθὰ πάντα τὸν βίον.
\VS{13}Μηρυομένη ἔρια καὶ λινὸν, ἐποίησεν εὔχρηστον ταῖς χερσὶν αὐτῆς.
\VS{14}Ἐγένετο ὡσεὶ ναῦς ἐμπορευομένη μακρόθεν, συνάγει δὲ αὕτη τὸν βίον.
\VS{15}Καὶ ἀνίσταται ἐκ νυκτῶν, καὶ ἔδωκε βρώματα τῷ οἴκῳ, καὶ ἔργα ταῖς θεραπαίναις.
\VS{16}Θεωρήσασα γεώργιον ἐπρίατο, ἀπὸ δὲ καρπῶν χειρῶν αὐτῆς κατεφύτευσεν κτῆμα.
\VS{17}Ἀναζωσαμένη ἰσχυρῶς τὴν ὀσφῦν αὐτῆς ἤρεισε τοὺς βραχίονας αὐτῆς εἰς ἔργον.
\VS{18}Καὶ ἐγεύσατο ὅτι καλόν ἐστι τὸ ἐργάζεσθαι, καὶ οὐκ ἀποσβέννυται ὁ λύχνος αὐτῆς ὅλην τὴν νύκτα.
\VS{19}τοὺς πήχεις αὐτῆς ἐκτείνει ἐπὶ τὰ συμφέροντα, τὰς δὲ χεῖρας αὐτῆς ἐρείδει εἰς ἄτρακτον.
\VS{20}Χεῖρας δὲ αὐτῆς διήνοιξε πένητι, καρπὸν δὲ ἐξέτεινεν πτωχῷ.
\par }{\PP \VS{21}Οὐ φροντίζει τῶν ἐν οἴκῳ ὁ ἀνὴρ αὐτῆς ὅταν που χρονίζῃ, πάντες γὰρ οἱ παρʼ αὐτῆς ἐνδεδυμένοι εἰαί.
\VS{22}Δισσὰς χλαίνας ἐποίησε τῷ ἀνδρὶ αὐτῆς, ἐκ δὲ βύσσου καὶ πορφύρας ἑαυτῇ ἐνδύματα.
\VS{23}Περίβλεπτος δὲ γίνεται ὁ ἀνὴρ αὐτῆς ἐν πύλαις, ἡνίκα ἂν καθίσῃ ἐν συνεδρίῳ μετὰ τῶν γερόντων κατοίκων τῆς γῆς.
\VS{24}Σινδόνας ἐποίησε καὶ ἀπέδοτο περιζώματα τοῖς Χαναναίοις. στόμα αὐτῆς διήνοιξε προσεχόντως καὶ ἐννόμως, καὶ τάξιν ἐστείλατο τῇ γλώσσῃ αὐτῆς.
\VS{25}Ἰσχὺν καὶ εὐπρέπειαν ἐνεδύσατο, καὶ εὐφράνθη ἐν ἡμέραις ἐσχάταις.
\VS{26}Στεγναὶ διατριβαὶ οἴκων αὐτῆς, σῖτα δὲ ὀκνηρὰ οὐκ ἔφαγεν.
\VS{27}Τὸ στόμα δὲ ἀνοίγει σοφῶς καὶ νομοθέσμως. Ἡ δὲ ἐλεημοσύνη αὐτῆς
\VS{28}ἀνέστησε τὰ τέκνα αὐτῆς καὶ ἐπλούτησαν, καὶ ὁ ἀνὴρ αὐτῆς ᾔνεσεν αὐτήν.
\VS{29}Πολλαὶ θυγατέρες ἐκτήσαντο πλοῦτον, πολλαὶ ἐποίησαν δύναμιν· σὺ δὲ ὑπέρκεισαι, ὑπερῇρας πάσας.
\VS{30}Ψευδεῖς ἀρέσκειαι, καὶ μάταιον κάλλος γυναικος· γυνὴ γὰρ συνετὴ εὐλογεῖται, φόβον δὲ Κυρίου αὕτη αἰνείτω.
\VS{31}Δότε αὐτῇ ἀπὸ καρπῶν χειλέων αὐτῆς, καὶ αἰνείσθω ἐν πύλαις ὁ ἀνὴρ αὐτῆς.
\par }