\NormalFont\ShortTitle{ΕΣΔΡΑΣ. Αʹ}
{\MT ΕΣΔΡΑΣ. Αʹ

\par }\ChapOne{1}{\PP \VerseOne{1}ΚΑΙ ἤγαγεν Ἰωσίας τὸ πάσχα ἐν Ἱερουσαλὴμ τῷ Κυρίῳ αὐτοῦ, καὶ ἔθυσε τὸ πάσχα τῇ τεσσαρεσκαιδεκάτῃ ἡμέρᾳ τοῦ μηνὸς τοῦ πρώτου·
\VS{2}στήσας τοὺς ἱερεῖς κατʼ ἐφημερίας ἐστολισμένους ἐν τῷ ἱερῷ τοῦ Κυρίου.
\par }{\PP \VS{3}Καὶ εἶπε τοῖς Λευίταις ἱεροδούλοις τοῦ Ἰσραὴλ, ἁγιάσαι ἑαυτοὺς τῷ Κυρίῳ ἐν τῇ θέσει τῆς ἁγίας κιβωτοῦ τοῦ Κυρίου ἐν τῷ οἴκῳ ᾧ ᾠκοδόμησε Σαλωμὼν ὁ τοῦ Δαυὶδ ὁ βασιλεύς·
\VS{4}οὐκ ἔσται ὑμῖν ἆραι ἐπʼ ὤμων αὐτήν· καὶ νῦν λατρεύετε τῷ Κυρίῳ Θεῷ ὑμῶν, καὶ θεραπεύετε τὸ ἔθνος αὐτοῦ Ἰσραήλ, καὶ ἑτοιμάσατε κατὰ τὰς πατριὰς καὶ τὰς φυλὰς ὑμῶν,
\VS{5}κατὰ τὴν γραφὴν Δαυὶδ βασιλέως Ἰσραὴλ, καὶ κατὰ τὴν μεγαλειότητα Σαλωμὼν τοῦ υἱοῦ αὐτοῦ· καὶ στάντες ἐν τῷ ἁγίῳ κατὰ τὴν μεριδαρχίαν τὴν πατρικὴν ὑμῶν τῶν Λευιτῶν, τῶν ἔμπροσθεν τῶν ἀδελφῶν ὑμῶν υἱῶν Ἰσραὴλ,
\VS{6}ἐν τάξει θύσατε τὸ πάσχα, καὶ τὰς θυσίας ἑτοιμάσατε τοῖς ἀδελφοῖς ὑμῶν, καὶ ποιήσατε τὸ πάσχα κατὰ τὸ πρόσταγμα τοῦ Κυρίου τὸ δοθὲν τῷ Μωυσῇ.
\par }{\PP \VS{7}Καὶ ἐδωρήσατο Ἰωσίας τῷ λαῷ τῷ εὑρεθέντι ἀρνῶν καὶ ἐρίφων τριάκοντα χιλιάδας, μόσχους τρισχιλίους· ταῦτα ἐκ τῶν βασιλικῶν ἐδόθη κατʼ ἐπαγγελίαν τῷ λαῷ, καὶ τοῖς ἱερεῦσι, καὶ Λευίταις.
\VS{8}Καὶ ἔδωκε Χελκίας, καὶ Ζαχαρίας, καὶ Συῆλος οἱ ἐπιστάται τοῦ ἱεροῦ τοῖς ἱερεῦσιν εἰς πάσχα πρόβατα δισχίλια ἑξακόσια, μόσχους τριακοσίους.
\VS{9}Καὶ Ἰεχονὶας, καὶ Σαμαίας, καὶ Ναθαναὴλ ὁ ἀδελφὸς, καὶ Ἀσαβίας, καὶ Ὀχίηλος, καὶ Ἰωρὰμ χιλίαρχοι ἔδωκαν τοῖς Λευίταις εἰς πάσχα πρόβατα πεντακισχίλια, μόσχους ἑπτακοσίους.
\par }{\PP \VS{10}Καὶ ταῦτα τὰ γενόμενα, εὐπρεπῶς ἔστησαν οἱ ἱερεῖς καὶ οἱ Λευῖται,
\VS{11}ἔχοντες τὰ ἄζυμα κατὰ τὰς φυλὰς καὶ κατὰ τὰς μεριδαρχίας τῶν πατέρων ἔμπροσθεν τοῦ λαοῦ, προσενεγκεῖν τῷ Κυρίῳ κατὰ τὰ γεγραμμένα ἐν βιβλίῳ Μωυσῆ· καὶ οὕτως τὸ πρωϊνόν.
\VS{12}Καὶ ὤπτησαν τὸ πάσχα πυρὶ ὡς καθήκει, καὶ τὰς θυσίας ἥψησαν ἐν τοῖς χαλκείοις καὶ λέβησι μετʼ εὐωδίας, καὶ ἀπήνεγκαν πᾶσι τοῖς ἐκ τοῦ λαοῦ·
\VS{13}μετὰ δὲ ταῦτα ἡτοίμασαν ἑαυτοῖς τε καὶ τοῖς ἱερεῦσιν ἀδελφοῖς αὐτῶν υἱοῖς Ἀαρών·
\VS{14}οἱ γὰρ ἱερεῖς ἀνέφερον τὰ στέατα ἕως ἀωρίας· καὶ οἱ Λευῖται ἡτοίμασαν ἑαυτοῖς καὶ τοῖς ἱερεῦσιν ἀδελφοῖς αὐτῶν υἱοῖς Ἀαρών.
\VS{15}Καὶ οἱ ἱεροψάλται υἱοὶ Ἀσὰφ ἦσαν ἐπὶ τῆς τάξεως αὐτῶν, κατὰ τὰ ὑπὸ Δαυὶδ τεταγμένα, καὶ Ἀσὰφ, καὶ Ζαχαρίας, καὶ Ἐδδινοῦς ὁ παρὰ τοῦ βασιλέως.
\VS{16}Καὶ οἱ θυρωροὶ ἐφʼ ἑκάστου πυλῶνος· οὐκ ἔστι παραβῆναι ἕκαστον τὴν ἑαυτοῦ ἐφημερίαν· οἱ γὰρ ἀδελφοὶ αὐτῶν οἱ Λευῖται ἡτοίμασαν αὐτοῖς,
\VS{17}καὶ συνετελέσθη τὰ τῆς θυσίας τοῦ Κυρίου ἐν ἐκείνῃ τῇ ἡμέρᾳ ἀχθῆναι τὸ πάσχα,
\VS{18}καὶ προσαχθῆναι τὰς θυσίας ἐπὶ τὸ τοῦ Κυρίου θυσιαστήριον, κατὰ τὴν ἐπιταγὴν τοῦ βασιλέως Ἰωσίου.
\par }{\PP \VS{19}Καὶ ἠγάγοσαν οἱ υἱοὶ Ἰσραὴλ οἱ εὑρεθέντες ἐν τῷ καιρῷ τούτῳ τὸ πάσχα καὶ τὴν ἑορτὴν τῶν ἀζύμων ἡμέρας ἑπτά.
\VS{20}Καὶ οὐκ ἤχθη τὸ πάσχα τοιοῦτον ἐν τῷ Ἰσραὴλ ἀπὸ τῶν χρόνων Σαμουὴλ τοῦ προφήτου.
\VS{21}Καὶ πάντες οἱ βασιλεῖς τοῦ Ἰσραὴλ οὐκ ἠγάγοσαν πάσχα τοιοῦτον, οἷον ἤγαγεν Ἰωσίας, καὶ οἱ ἱερεῖς, καὶ οἱ Λευῖται, καὶ οἱ Ἰουδαῖοι, καὶ πᾶς Ἰσραὴλ ὁ εὑρεθεὶς ἐν τῇ κατοικήσει αὐτῶν ἐν Ἱερουσαλήμ.
\VS{22}Ὀκτωκαιδεκάτῳ ἔτει βασιλεύοντος Ἰωσίου ἤχθη τὸ πάσχα τοῦτο.
\par }{\PP \VS{23}Καὶ ὠρθώθη τὰ ἔργα Ἰωσίου ἐνώπιον τοῦ Κυρίου αὐτοῦ ἐν καρδίᾳ πλήρει εὐσεβείας.
\VS{24}Καὶ τὰ κατʼ αὐτὸν δὲ ἀναγέγραπται ἐν τοῖς ἔμπροσθεν χρόνοις, περὶ τῶν ἡμαρτηκότων καὶ ἠσεβηκότων εἰς τὸν Κύριον παρὰ πᾶν ἔθνος καὶ βασιλείαν, καὶ ἃ ἐλύπησαν αὐτὸν, ἔστι, καὶ οἱ λόγοι τοῦ Κυρίου ἀνέστησαν ἐπὶ Ἰσραήλ.
\par }{\PP \VS{25}Καὶ μετὰ πᾶσαν τὴν πρᾶξιν ταύτην Ἰωσίου, συνέβη Φαραὼ βασιλέα Αἰγύπτου ἐλθόντα πόλεμον ἐγεῖραι ἐν Χαρκαμὺς ἐπὶ τοῦ Εὐφράτου· καὶ ἐξῆλθεν εἰς ἀπάντησιν αὐτῷ Ἰωσίας.
\VS{26}Καὶ διεπέμψατο πρὸς αὐτὸν βασιλεὺς Αἰγύπτου, λέγων, τί ἐμοὶ καὶ σοί ἐστι, βασιλεῦ τῆς Ἰουδαίας;
\VS{27}Οὐχί πρὸς σὲ ἐξαπέσταλμαι ὑπὸ Κυρίου τοῦ Θεοῦ· ἐπὶ γὰρ τοῦ Εὐφράτου ὁ πόλεμός μου ἐστί· καὶ νῦν Κύριος μετʼ ἑμοῦ ἐστι, καὶ Κύριος μετʼ ἐνοῦ ἑπισπεύδων ἐστίν· ἀπόστηθι, καὶ μὴ ἐναντιοῦ τῷ Κυρίῳ.
\par }{\PP \VS{28}Καὶ οὐκ ἀπέστρεψεν ἑαυτὸν Ἰωσίας ἐπὶ τὸ ἅρμα αὐτοῦ, ἀλλὰ πολεμεῖν αὐτὸν ἐπεχείρει, οὐ προσέχων ῥήμασιν Ἰερεμίου προφήτου ἐκ στόματος Κυρίου.
\VS{29}Ἀλλὰ συνεστήσατο πρὸς αὐτὸν πόλεμον ἐν τῷ πεδίῳ Μαγεδδώ· καὶ κατέβησαν οἱ ἄρχοντες πρὸς τὸν βασιλέα Ἰωσίαν.
\VS{30}Καὶ εἶπεν ὁ βασιλεὺς τοῖς παισὶν ἑαυτοῦ, ἀποστήσατέ με ἀπὸ τῆς μάχης, ἠσθένησα γὰρ λίαν· καὶ εὐθέως ἀπέστησαν αὐτὸν οἱ παῖδες αὐτοῦ ἀπὸ τῆς παρατάξεως.
\VS{31}Καὶ ἀνέβη ἐπὶ τὸ ἅρμα τὸ δευτέριον αὐτοῦ, καὶ ἀποκατασταθεὶς, εἰς Ἰερουσαλὴμ, μετήλλαξε τὸν βίον αὐτοῦ, καὶ ἐτάφη ἐν τῷ πατρικῷ τάφῳ.
\VS{32}Καὶ ἐν ὅλῃ τῇ Ἰουδαίᾳ ἐπένθησαν τὸν Ἰωσίαν, καὶ ἐθρήνησεν Ἱερεμίας ὁ προφήτης ὑπὲρ Ἰωσίον, ὑαι οἱ προκαθήμενοι σὺν γυναιξὶν ἐθρηνοῦσαν αὐτὸν ἕως τῆς ἡμέρας ταύτης· καὶ ἐξεδόθη τοῦτο γίνεσθαι ἀεὶ εἰς ἅπαν τὸ γένος Ἰσραήλ.
\par }{\PP \VS{33}Ταῦτα δὲ ἀναγέγραπται ἐν τῇ βίβλῳ τῶν ἱστορουμένων περὶ τῶν βασιλέων τῆς Ἰουδαίας, καὶ τὸ καθʼ ἓν πραχθὲν τῆς πράξεως Ἰωσίου, καὶ τῆς δόξης αὐτοῦ, καὶ τῆς συνέσεως αὐτοῦ ἐν τῷ νόμῳ Κυρίου· τά τε προπραχθέντα ὑπʼ αὐτοῦ καὶ τὰ νῦν, ἱστόρηται ἐν τῷ βιβλίῳ τῶν βασιλέων Ἰσραὴλ καὶ Ἰούδα.
\par }{\PP \VS{34}Καὶ ἀναλαβόντες οἱ ἐκ τοῦ ἔθνους τὸν Ἰεχονίαν υἱὸν Ἰωσίου, ἀνέδειξαν βασιλέα ἀντὶ Ἰωσείου τοῦ πατρὸς αὐτοῦ, ὄντα ἐτῶν εἴκοσι τριῶν.
\VS{35}Καὶ ἐβασίλευσεν ἐν Ἰσραὴλ καὶ Ἱερουσαλὴμ μῆνας τρεῖς· καὶ ἀπέστησεν αὐτὸν βασιλεὺς Αἰγύπτου τοῦ μὴ βασιλεύειν ἐν Ἱερουσαλὴμ,
\VS{36}καὶ ἐζημίωσε τὸ ἔθνος ἀργυρίου ταλάντοις ἑκατὸν καὶ χρυσίου ταλάντῳ ἑνί.
\par }{\PP \VS{37}Καὶ ἀνέδειξε βασιλεὺς Αἰγύπτου βασιλέα Ἰωακὶμ τὸν ἀδελφὸν αὐτοῦ βασιλέα τῆς Ἰουδαίας καὶ Ἱερουσαλήμ.
\VS{38}Καὶ ἔδησεν Ἰωακὶμ τοὺς μεγιστᾶνας, Ζαράκην δὲ τὸν ἀδελφὸν αὐτοῦ συλλαβὼν ἀνήγαγεν ἐξ Αἰγύπτου.
\VS{39}Ἐτῶν δὲ ἦν εἰκοσιπέντε Ἰωακὶμ ὅτε ἐβασίλευσε τῆς Ἰουδαίας καὶ Ἱερουσαλήμ· καὶ ἐποίησε τὸ πονηρὸν ἐνώπιον Κυρίου.
\VS{40}Μετʼ αὐτὸν δὲ ἀνέβη Ναβουχοδονόσορ ὁ βασιλεὺς Βαβυλῶνος, καὶ ἔδησεν αὐτὸν ἐν χαλκείῳ δεσμῷ, καὶ ἀπήγαγεν εἰς Βαβυλῶνα.
\VS{41}Καὶ ἀπὸ τῶν ἱερῶν σκευῶν τοῦ Κυρίου λαβὼν Ναβουχοδονόσορ καὶ ἀπενέγκας, ἀπηρείσατο ἐν τῷ ναῷ αὐτοῦ ἐν Βαβυλῶνι.
\VS{42}Τὰ δὲ ἱστορηθέντα περὶ αὐτοῦ, καὶ τῆς ἀκαθαρσίας αὐτοῦ καὶ δυσσεβείας, ἀναγέγραπται ἐν τῇ βιβλῳ τῶν χρόνων τῶν βασιλέων.
\par }{\PP \VS{43}Καὶ ἐβασίλευσεν ἀντʼ αὐτοῦ Ἰωακὶμ ὁ υἱὸς αὐτοῦ· ὅτε γὰρ ἀνεδείχθη, ἦν ἐτῶν ὀκτώ.
\VS{44}Βασιλεύει δὲ μῆνας τρεῖς καὶ ἡμέρας δέκα ἐν Ἱερουσαλὴμ, καὶ ἐποίησε τὸ πονηρὸν ἔναντι Κυρίου.
\par }{\PP \VS{45}Καὶ μετʼ ἐνιαυτὸν ἀποστείλας Ναβουχοδονόσορ μετήγαγεν αὐτὸν εἰς Βαβυλῶνα, ἅμα τοῖς ἱεροῖς σκεύεσι τοῦ Κυρίου,
\VS{46}καὶ ἀνέδειξε Σεδεκίαν βασιλέα τῆς Ἰουδαίας καὶ Ἱερουσαλὴμ, ὄντα ἐτῶν εἴκοσι ἑνός· βασιλεύει δὲ ἔτη ἕνδεκα,
\VS{47}καὶ ἐποίησε τὸ πονηρὸν ἐνώπιον Κυρίου, καὶ οὐκ ἐνετράπη ἀπὸ τῶν ῥηθέντων λόγων ὑπὸ Ἱερεμίου τοῦ προφήτου ἐκ στόματος τοῦ Κυρίου.
\VS{48}Καὶ ὁρκισθεὶς ἀπὸ τοῦ βασιλέως Ναβουχοδονόσορ τῷ ὀνόματι Κυρίου, ἐπιορκήσας ἀπέστη· καὶ σκληρύνας αὐτοῦ τὸν τράχηλον καὶ τὴν καρδίαν αὐτοῦ, παρέβη τὰ νόμιμα Κυρίου Θεοῦ Ἰσραήλ.
\VS{49}Καὶ οἱ ἡγούμενοι δὲ τοῦ λαοῦ καὶ τῶν ἱερέων πολλὰ ἠσέβησαν καὶ ὑπὲρ πάσας τὰς ἀκαθαρσίας πάντων τῶν ἐθνῶν, καὶ ἐμίαναν τὸ ἱερὸν τοῦ Κυρίον τὸ ἁγιαζόμενον ἐν Ἱερουσαλήμ.
\par }{\PP \VS{50}Καὶ ἀπέστειλεν ὁ Θεὸζ τῶν πατέρων αὐτῶν διὰ τοῦ ἀγγέλου αὐτοῦ μετακαλέσαι αὐτοὺς, καθότι ἐφείδετο αὐτῶν καὶ τοῦ σκηνώματος αὐτοῦ.
\VS{51}Αὐτοὶ δὲ ἐμυκτήρισαν ἐν τοῖς ἀγγέλοις αὐτοῦ·
\VS{52}καὶ ᾗ ἡμέρᾳ ἐλάλησε Κύριος, ἦσαν ἐκπαίζοντες τοὺς προφήτας αὐτοῦ, ἕως οὗ θυμῶντα αὐτὸν ἐπὶ τῷ ἔθνει αὐτοῦ διὰ τὰ δυσσεβήματα, προστάξαι ἀναβιβάσαι ἐπʼ αὐτοὺς τοὺς βασιλεῖς τῶν Χαλδαίων.
\VS{53}Οὗτοι ἀπέκτειναν τοὺς νεανίσκους αὐτῶν ἐν ῥομφαία, περικύκλῳ τοῦ ἁγίου αὐτῶν ἱεροῦ· καὶ οὐκ ἐφείσαντο νεανίσκου καὶ παρθένου, καὶ πρεσβύτου καὶ νεωτέρου, ἀλλὰ πάντας παρέδωκαν εἰς τὰς χεῖρας αὐτῶν.
\VS{54}Καὶ πάντα τὰ ἱερὰ σκεύη τοῦ Κυρίου τὰ μεγάλα καὶ τὰ μικρά, καὶ τὰς κιβωτοὺς τοῦ Κυρίου, καὶ τὰς βασιλικὰς ἀποθήκας ἀναλαβόντες ἀπήνεγκαν εἰς Βαβυλῶνα.
\VS{55}Καὶ ἐνεπύρισαν τὸν οἶκον τοῦ Κυρίου, καὶ ἔλυσαν τὰ τείχη Ἱερουσαλὴμ, καὶ τοὺς πύργους αὐτῆς ἐνεπύρισαν ἐν πυρί,
\VS{56}καὶ συνετέλεσαν πάντα τὰ ἔνδοξα αὐτῆς ἀχρειῶσαι, καὶ τοὺς ἐπιλοίπους ἀπήγαγε μετὰ ῥομφαίας εἰς Βαβυλῶνα.
\VS{57}Καὶ ἦσαν παῖδες αὐτῷ καὶ τοῖς υἱοῖς αὐτοῦ, μέχρις οὗ βασιλεῦσαι Πέρσας, εἰς ἀναπλήρωσιν ῥήματος τοῦ Κυρίου ἐν στόματι Ἱερεμίου·
\VS{58}ἕως τοῦ εὐδοκῆσαι τὴν γῆν τὰ σάββατα αὐτῆς, πάντα τὸν χρόνον τῆς ἐρημώσεως αὐτῆς, σαββατιεῖ εἰς συμπλήρωσιν ἐτῶν ἑβδομήκοντα.

\par }\Chap{2}{\PP \VerseOne{1}Βασιλεύοντος Κύρου Περσῶν ἔτους πρώτου, εἰς συντέλειαν ῥήματος Κυρίου ἐν στόματι Ἱερεμίου,
\VS{2}ἤγειρε Κύριος τὸ πνεῦμα Κύρου βασιλέως Περσῶν, καὶ ἐκήρυξεν ἐν ὅλῃ τῇ βασιλείᾳ αὐτοῦ, καὶ ἅμα διὰ γραπτῶν, λέγων,
\VS{3}τάδε λέγει ὁ βασιλεὺς Περσῶν Κῦρος, ἐμὲ ἀνέδειξε βασιλέα τῆς οἰκουμένης ὁ Κύριος τοῦ Ἰσραὴλ, Κύριος ὁ ὕψιστος.
\VS{4}Καὶ ἐσήμῃνέ μοι οἰκοδομῆσαι αὐτῷ οἶκον ἐν Ἱερουσαλὴμ, τῇ ἐν τῇ Ἰουδαίᾳ.
\par }{\PP \VS{5}Εἴ τις ἐστὶν οὖν ὑμῶν ἐκ τοῦ ἔθνους αὐτοῦ, ἔστω ὁ Κύριος αὐτοῦ μετʼ αὐτοῦ, καὶ ἀναβὰς εἰς τὴν Ἱερουσαλὴμ τὴν ἐν τῇ Ἰουδαίᾳ, οἰκοδομείτω τὸν οἶκον τοῦ Κυρίου τοῦ Ἰσραήλ· οὗ ὁ Κύριος, ὁ κατασκηνώσας ἐν Ἱερουσαλήμ.
\VS{6}Ὅσοι οὖν κατὰ τοὺς τόπους οἰκοῦσι, βοηθείτωσαν αὐτῷ οἱ ἐν τῷ τόπῳ αὐτοῦ, ἐν χρυσίῳ καὶ ἐν ἀργυρίῳ, ἐν δὀσεσι,
\VS{7}μεθʼ ἵππων καὶ κτηνῶν, σὺν τοῖς ἄλλοις τοῖς κατʼ εὐχὰς προστεθειμένοις εἰς τὸ ἱερὸν τοῦ Κυρίου τὸ ἐν Ἱερουσαλήμ.
\par }{\PP \VS{8}Καὶ καταστήσαντες οἱ ἀρχίφυλοι τῶν πατριῶν τῆς Ἰούδα καὶ Βενιαμὶν φυλῆς, καὶ οἱ ἱερεῖς καὶ οἱ Λευῖται, καὶ πάντων ὧν ἤγειρε Κύριος τό πνεῦμα, ἀναβῆναι οἰκοδομῆσαι οἶκον τῷ Κυρίῳ τὸν ἐν Ἱερουσαλήμ·
\VS{9}καὶ οἱ περικύκλῳ αὐτῶν ἐβοήθησαν ἐν πᾶσιν, ἐν ἀργυρίῳ καὶ χρυσίῳ, ἵπποις, κτήνεσι, καὶ εὐχαῖς ὡς πλείσταις πολλῶν, ὧν ὁ νοῦς ἠγέρθη.
\VS{10}Καὶ ὁ βασιλεὺς Κύρος ἐξήνεγκε τὰ ἱερὰ σκεύη τοῦ Κυρίου, ἃ μετήνεγκε Ναβουχοδονόσορ ἐξ Ἱερουσαλὴμ, καὶ ἀπηρείσατο αὐτὰ ἐν τῷ εἰδωλείῳ αὐτοῦ.
\par }{\PP \VS{11}Ἐξενέγκας δὲ αὐτὰ Κύρος ὁ βασιλεὺς Περσῶν παρέδωκεν αὐτὰ Μιθρδάτῃ τῷ ἑαυτοῦ γαζοφύλακι·
\VS{12}Διὰ δὲ τούτου παρεδόθησαν Σαμανασσάρῳ προστάτῃ τῆς Ἰουδαίας.
\VS{13}Ὁ δὲ τούτων ἀριθμὸς ἦν, σπονδεῖα χρυσᾶ χίλια, σπονδεῖα ἀργυρᾶ χίλια, θυΐσκαι ἀργυραῖ εἰκοσιεννέα, φιάλαι χρυσαῖ τριάκοντα, ἀργυραῖ διοχίλιαι τετρακόσιαι δέκα, καὶ ἄλλα σκεύη χίλια.
\VS{14}Τὰ δὲ πάντα σκεύη ἐκομίσθη, χρυσᾶ καὶ ἀργυρᾶ, πεντακισχίλια τετρακόσια ἑξηκονταεννέα.
\VS{15}Ἀνηνέχθη δὲ ὑπὸ Σαμανασσάρον ἅμα τοῖς ἐκ τῆς αἰχμαλωσίας ἐκ Βαβυλῶνος εἰς Ἰερουσαλήμ.
\par }{\PP \VS{16}Ἐν δὲ τοῖς ἐπὶ Ἀρταξέρξου τῶν Περσῶν βασιλέως χρόνοις κατέγραψαν αὐτῷ κατὰ τῶν κατοικούντων ἐν τῇ Ἰουδαίᾳ καὶ Ἱερουσαλὴμ, Βήλεμος, καὶ Μιθραδάτης, καὶ Ταβέλλιος, καὶ Ῥάθυμος, καὶ Βεέλτεθμος, καὶ Σαμέλλιος ὁ γραμματεὺς, καὶ οἱ λοιποὶ οἱ τούτοις συντασσόμενοι, οἰκοῦντες δὲ ἐν Σαμαρείᾳ καὶ τοῖς ἄλλοις τόποις, τὴν ὑπογεγραμμένην ἐπιστολήν·
\VS{17}Βασιλεῖ Ἀρταξέρξῃ κυρίῳ οἱ παῖδές σου, Ῥάθυμος ὁ τὰ προσπίπτοντα, καὶ Σαμέλλιος ὁ γραμματεὺς, καὶ οἱ ἐπίλοιποι τῆς βουλῆς αὐτῶν, καὶ κριταὶ οἱ ἐν κοίλῃ Συρίᾳ καὶ Φοινίκῃ.
\VS{18}Καὶ νῦν γνωστὸν ἔστω τῷ κυρίῳ βασιλεῖ, ὅτι οἱ Ἰουδαῖοι ἀναβάντες παρʼ ὑμῶν πρὸς ἡμᾶς ἐλθόντες εἰς Ἱερουσαλήμ, τὴν πόλιν τήν ἀποστάτιν καὶ πονηρὰν, οἰκοδομοῦσι τάς τε ἀγορὰς αὐτῆς, καὶ τὰ τείχη θεραπεύουσι, καὶ ναὸν ὑποβάλλονται.
\VS{19}Ἐὰν οὖν ἡ πόλις αὕτη οἰκοδομηθῇ, καὶ τὰ τείχη συντελεσθῇ, φορολογίαν οὐ μὴ ὑπομείνωσι δοῦναι, ἀλλὰ καὶ βασιλεῦσιν ἀντιστήσονται.
\par }{\PP \VS{20}Καὶ ἐπεὶ ἐνεργεῖται τὰ κατὰ τὸν ναὸν, καλῶς ἔχειν ὑπολαμβάνομεν μὴ ὑπεριδεῖν τὸ τοιοῦτο,
\VS{21}ἀλλὰ προσφωνῆσαι τῷ κυρίῳ βασιλεῖ, ὅπως ἂν φαίνηταί σοι, ἐπισκεφθῇ ἐν τοῖς ἀπὸ τῶν πατέρων σου βιβλίοις.
\VS{22}Καὶ εὑρήσεις ἐν τοῖς ὑπομνηματισμοῖς γεγραμμένα περὶ τούτων, καὶ γνώσῃ ὅτι ἡ πόλις ἐκείνη ἦν ἀποστάτις,
\VS{23}καὶ βασιλεῖς καὶ πόλεις ἐνοχλοῦσα, καὶ οἱ Ἰουδαῖοι ἀποστάται καὶ πολιορκίας συνιστάμενοι ἐν αὐτῇ ἔτι ἐξ αἰῶνος, διʼ ἣν αἰτίαν καὶ ἡ πόλις αὕτη ἠρημώθη.
\VS{24}Νῦν οὖν ὑποδεικνύομέν σοι, κύριε βασιλεῦ, ὅτι ἐὰν ἡ πόλις αὕτη οἰκοδομηθῇ, καὶ τὰ ταύτης τείχη ἀνασταθῇ, κάθοδος οὐκ ἔτι σοι ἔσται εἰς κοίλην Συρίαν καὶ Φοινίκην.
\par }{\PP \VS{25}Τότε ἀντέγραψεν ὁ βασιλεὺς Ῥαθύμῳ τῷ γράφοντι τὰ προσπίπτοντα, καὶ Βεελτέθμῳ, καὶ Σαμελλίῳ γραμματεῖ, καὶ τοῖς λοιποῖς τοῖς συντασσομένοις καὶ οἰκοῦσιν ἐν τῇ Σαμαρείᾳ, καὶ Συρίᾳ, καὶ Φοινίκῃ, τὰ ὑπογεγραμμένα.
\VS{26}Ἀνέγνων τὴν ἐπιστολὴν ἣν πεπόμφατε πρὸς μέ· ἐπέταξα οὖν ἐπισκέψασθαι· καὶ εὑρέθη ὅτι ἡ πόλις ἐκείνη ἐστὶν ἐξ αἰῶνος βασιλεῦσιν ἀντιπαρατάσσουσα,
\VS{27}καὶ οἱ ἄνθρωποι ἀποστάσεις καὶ πολέμους ἐν αὐτῇ συντελοῦντες, καὶ βασιλεῖς ἰσχυροὶ καὶ σκληροὶ ἦσαν ἐν Ἱερουσαλὴμ κυριεύοντες καὶ φορολογοῦντες κοίλην Συρίαν καὶ Φοινίκην.
\VS{28}Νῦν οὖν ἐπέταξα ἀποκωλῦσαι τοὺς ἀνθρώπους ἐκείνους τοῦ οἰκοδομῆσαι τὴν πόλιν, καὶ προνοηθῆναι ὅπως μηδὲν παρὰ ταῦτα γένηται.
\VS{29}Καὶ μὴ προβῇ ἐπὶ πλεῖον τὰ τῆς κακίας εἰς τὸ βασιλεῖς ἐνοχλῆσαι.
\par }{\PP \VS{30}Τότε ἀναγνωσθέντων τῶν παρὰ τοῦ βασιλέως Ἀρταξέρξου γραφέντων, Ῥάθυμος, καὶ Σαμέλλιος ὁ γραμματεὺς, καὶ οἱ τούτοις συντασσόμενοι, ἀναζεύξαντες εἰς Ἱερουσαλὴμ κατὰ σπουδὴν μεθʼ ἵππου καὶ ὄχλου παρατάξεως, ἤρξαντο κωλύειν τοὺς οἰκοδομοῦντας, καὶ ἤργει ἡ οἰκοδομὴ τοῦ ἱεροῦ τοῦ ἐν Ἱερουσαλὴμ μέχρι τοῦ δευτέρου ἔτους τῆς βασιλείας Δαρείου τοῦ Περσῶν βασιλέως.

\par }\Chap{3}{\PP \VerseOne{1}Καὶ βασιλεὺς Δαρεῖος ἐποίησε δοχὴν μεγάλην πᾶσι τοῖς ὑπʼ αὐτὸν, καὶ πᾶσι τοῖς οἰκογενέσιν αὐτοῦ, καὶ πᾶσι τοῖς μεγιστᾶσι τῆς Μηδίας καὶ τῆς Περσίδος,
\VS{2}καὶ πᾶσι τοῖς σατράπαις καὶ στρατηγοῖς καὶ τοπάρχαις τοῖς ὑπʼ αὐτὸν, ἀπὸ τῆς Ἰνδικῆς μέχρις Αἰθιοπίας, ἐν ταῖς ἑκατὸν εἰκοσιεπτὰ σατραπείαις.
\VS{3}Καὶ ἐφάγοσαν καὶ ἐπίοσαν, καὶ ἐμπλησθέντες ἀνέλυσαν· ὁ δὲ Δαρεῖος ὁ βασιλεὺς ἀνέλυσεν εἰς τὸν κοιτῶνα ἑαυτοῦ, καὶ ἐκοιμήθη, καὶ ἔξυπνος ἐγένετο.
\par }{\PP \VS{4}Τότε οἱ τρεῖς νεανίσκοι οἱ σωματοφύλακες οἱ φυλάσσοντες τὸ σῶμα τοῦ βασιλέως,
\VS{5}εἶπαν ἕτερος πρὸς τὸν ἕτερον, εἴπωμεν ἕκαστος ἡμῶν ἕνα λόγον, ὃς ὑπερισχύσει· καὶ οὗ ἐὰν φανῇ τὸ ῥῆμα αὐτοῦ σοφώτερον τοῦ ἑτέρου, δώσει αὐτῷ Δαρεῖος ὁ βασιλεὺς δωρεὰς μεγάλας, καὶ ἐπινίκια μεγάλα,
\VS{6}καὶ πορφύραν περιβαλέσθαι, καὶ ἐν χρυσώμασι πίνειν, καὶ ἐπὶ χρυσῷ καθεύδειν, καὶ ἅρμα χρυσοχάλινον, καὶ κίδαριν βυσσίνην, καὶ μανιάκην περὶ τὸν τράχηλον,
\VS{7}καὶ δεύτερος καθιεῖται Δαρείου διὰ τὴν σοφίαν αὐτοῦ, καὶ συγγενὴς Δαρείου κληθήσεται.
\par }{\PP \VS{8}Καὶ τότε γράψαντες ἕκαστος τὸν ἑαυτοῦ λόγον, ἐσφαγίσαντο καὶ ἔθηκαν ὑπὸ τὸ προσκεφάλαιον Δαρείου τοῦ βασιλέως,
\VS{9}καὶ εἶπαν, ὅταν ἐγερθῇ ὁ βασιλεὺς, δώσουσιν αὐτῷ τὸ γράμμα, καὶ ὃν ἂν κρίνῃ ὁ βασιλεὺς καὶ οἱ τρεῖς μεγιστᾶνες τῆς Περσίδος, ὅτι οὗ ὁ λόγος αὐτοῦ σοφώτερος, αὐτῷ δοθήσεται τὸ νῖκος καθὼς γέγραπται.
\VS{10}Ὁ εἷς ἔγραψεν, ὑπερισχύει ὁ οἶνος.
\VS{11}Ὁ ἕτερος ἔγραψεν, ὑπερισχύει ὁ βασιλεύς.
\VS{12}Ὁ τρίτος ἔγραψεν, ὑπερισχύουσιν αἱ γυναῖκες, ὑπὲρ δὲ πάντα νικᾷ ἡ ἀλήθεια.
\par }{\PP \VS{13}Καὶ ὅτε ἐξηγέρθη ὁ βασιλεὺς, λαβόντες τὸ γράμμα ἔδωκαν αὐτῷ, καὶ ἀνέγνω.
\VS{14}Καὶ ἐξαποστείλας ἐκάλεσε πάντας τοὺς μεγιστᾶνας τῆς Περσίδος καὶ τῆς Μηδείας, καὶ τοὺς σατράπας καὶ στρατηγοὺς, καὶ τοπάρχας καὶ ὑπάτους,
\VS{15}καὶ ἐκάθισεν ἐν τῷ χρηματιστηρίῳ, καὶ ἀνεγνώσθη τὸ γράμμα ἐνώπιον αὐτῶν.
\VS{16}Καὶ εἶπε, καλέσατε τοὺς νεανίσκους, καὶ αὐτοὶ δηλώσουσι τοὺς λόγους ἑαυτῶν· καὶ ἐκλήθησαν, καὶ εἰσήλθοσαν.
\VS{17}Καὶ εἶπαν αὐτοῖς, ἀπαγγείλατε ἡμῖν περὶ τῶν γεγραμμένων.
\par }{\PP \VS{18}Καὶ ἤρξατο ὁ πρῶτος ὁ εἴπας περὶ τῆς ἰσχύος τοῦ οἴνου, καὶ ἔφη οὕτως, ἄνδρες,
\VS{19}πῶς ὑπερισχύει ὁ οἶνος; πάντας τοὺς ἀνθρώπους τοὺς πιόντας αὐτὸν πλανᾷ, τὴν διάνοιαν τοῦ τε βασιλέως καὶ τοῦ ὀρφανοῦ ποιεῖ τὴν διάνοιαν μίαν, τήν τε τοῦ οἰκέτου καὶ τὴν τοῦ ἐλευθέρου, τήν τε τοῦ πένητος καὶ τὴν τοῦ πλουσίου·
\VS{20}καὶ πᾶσαν διάνοιαν μεταστρέφει εἰς εὐωχίαν καὶ εὐφροσύνην, καὶ οὐ μέμνηται πᾶσαν λύπην καὶ πᾶν ὀφείλημα·
\VS{21}καὶ πάσας καρδίας ποιεῖ πλουσιας, καὶ οὐ μέμνηται βασιλέα οὐδὲ σατράπην· καὶ πάντα διὰ ταλάντων ποιεῖ λαλεῖν.
\VS{22}Καὶ οὐ μέμνηνται, ὅταν πίνωσι, φιλιάζειν φίλοις καὶ ἀδελφοῖς, καὶ μετʼ οὐ πολὺ σπῶνται τὰς μαχαίρας.
\VS{23}Καὶ ὅταν ἀπὸ τοῦ οἴνου ἐγερθῶσιν, οὐ μέμνηνται ἃ ἔπραξαν.
\VS{24}Ὦ ἄνδρες, οὐχ ὑπερισχύει ὁ οἶνος, ὅτι οὕτως ἀναγκάζει ποιεῖν; καὶ ἐσίγησεν οὕτως εἴπας.

\par }\Chap{4}{\PP \VerseOne{1}Καὶ ἤρξατο ὁ δεύτερος λαλεῖν, ὁ εἴπας περὶ τῆς ἰσχύος τοῦ βασιλέως.
\VS{2}Ὦ ἄνδρες, οὐχ ὑπερισχύουσιν οἱ ἄνθρωποι, τὴν γῆν καὶ τὴν θάλασσαν κατακρατοῦντες καὶ πάντα τὰ ἐν αὐτοῖς;
\VS{3}Ὁ δὲ βασιλεὺς ὑπερισχύει, καὶ κυριεύει αὐτῶν καὶ δεσπόζει αὐτῶν, καὶ πᾶν ὃ ἐὰν εἴπῃ αὐτοῖς, ἐνακούουσιν.
\VS{4}Ἐὰν εἴπῃ αὐτοῖς ποιῆσαι πόλεμον ἕτερος πρὸς τὸν ἕτερον, ποιοῦσιν· ἐὰν δὲ ἐξαποστείλῃ αὐτοὺς πρὸς τοὺς πολεμίους, βαδίζουσι καὶ κατεργάζονται τὰ ὄρη καὶ τὰ τείχη καὶ τοὺς πύργους,
\VS{5}φονεύουσι καὶ φονεύονται, καὶ τὸν λόγον τοῦ βασιλέως οὐ παραβαίνουσιν· ἐὰν δὲ νικήσωσι, τῷ βασιλεῖ κομίζουσι πάντα, καὶ ἐὰν προνομεύσωσι, καὶ τὰ ἄλλα πάντα.
\par }{\PP \VS{6}Καὶ ὅσοι οὐ στρατεύονται οὐδὲ πολεμοῦσιν, ἀλλὰ γεωργοῦσι τὴν γῆν, πάλιν ὅταν σπείρωσι θερίσαντε, ἀναφέρουσι τῷ βασιλεῖ· καὶ ἕτερος τὸν ἕτερον ἀναγκάζοντες, ἀναφέρουσι τοὺς φόρους τῷ βασιλεῖ.
\VS{7}Καὶ αὐτὸς εἷς μόνος ἐστίν· ἐὰν εἴπῃ ἀποκτεῖναι, ἀποκτέννουσιν· ἐὰν εἴπῃ ἀφεῖναι, ἀφίουσιν·
\VS{8}Εἶπε πατάξαι, τύπτουσιν· εἶπεν ἐρημῶσαι, ἐρημοῦσιν· εἶπεν οἰκοδομῆσαι, οἰκοδομοῦσιν·
\VS{9}εἶπεν ἐκκόψαι, ἐκκόπτουσιν· εἶπε φυτεῦσαι, φυτεύουσι.
\VS{10}Καὶ πᾶς ὁ λαὸς αὐτοῦ καὶ αἱ δυνάμεις αὐτοῦ ἐνακούουσι· πρὸς δὲ τούτοις αὐτὸς ἀνάκειται, ἐσθίει καὶ πίνει καὶ καθεύδει,
\VS{11}αὐτοὶ δὲ τηροῦσι κύκλῳ περὶ αὐτόν· καὶ οὐ δύνανται ἕκαστος ἀπελθεῖν, καὶ ποιεῖν τὰ ἔργα αὐτοῦ, οὐδὲ παρακούουσιν αὐτοῦ.
\VS{12}Ὦ ἄνδρες, πῶς οὐχ ὑπερισχύει ὁ βασιλεὺς, ὅτι οὕτως ἐπάκουστός ἐστι; καὶ ἐσίγησεν.
\par }{\PP \VS{13}Ὁ δὲ τρίτος ὁ εἴπας περὶ τῶν γυναικῶν καὶ τῆς ἀληθείας, οὗτός ἐστι Ζοροβάβελ, ἤρξατο λαλεῖν·
\VS{14}Ἄνδρες, οὐ μέγας ὁ βασιλεὺς, καὶ πολλοὶ οἱ ἄνθρωποι, καὶ ὁ οἶνος ἰσχύει; τίς οὖν ὁ δεσπόζων αὐτῶν, ἢ τίς ὁ κυριεύων αὐτῶν; οὐχ αἱ γυναῖκες;
\VS{15}Αἱ γυναῖκες ἐγέννησαν τὸν βασιλέα καὶ πάντα τὸν λαὸν ὃς κυριεύει τῆς θαλάσσης καὶ τῆς γῆς,
\VS{16}καὶ ἐξ αὐτῶν ἐγένοντο· καὶ αὗται ἐξέθρεψαν αὐτοὺς τοὺς φυτεύσαντας τοὺς ἀμπελῶνας ἐξ ὧν ὁ οἶνος γίνεται.
\VS{17}Καὶ αὗται ποιοῦσι τὰς στολὰς τῶν ἀνθρώπων, καὶ αὗται ποιοῦσι δόξαν τοῖς ἀνθρώποις, καὶ οὐ δύνανται οἱ ἄνθρωποι χωρὶς τῶν γυναικῶν εἶναι.
\VS{18}Ἐὰν δὲ συναγάγωσι χρυσίον καὶ ἀργύριον καὶ πᾶν πρᾶγμα ὡραῖον, καὶ ἴδωσι γυναῖκα μίαν καλὴν τῷ εἴδει καὶ τῷ κάλλει,
\VS{19}ταῦτα πάντα ἀφέντες, εἰς αὐτὴν ἐκκέχῃναν, καὶ χάσκοντες τὸ στόμα θεωροῦσιν αὐτὴν, καὶ πάντες αὐτὴν αἱρετίζουσι μᾶλλον ἢ τὸ χρυσίον καὶ τὸ ἀργύριον καὶ πᾶν πρᾶγμα ὡραῖον.
\par }{\PP \VS{20}Ἄνθρωπος τὸν ἑαυτοῦ πατέρα ἐγκαταλείπει ὃς ἐξέθρεψεν αὐτὸν, καὶ τὴν ἰδίαν χώραν, καὶ πρὸς τὴν ἰδίαν γυναῖκα κολλᾶται,
\VS{21}καὶ μετὰ τῆς γυναικὸς ἀφίησι τὴν ψυχήν, καὶ οὔτε τὸν πατέρα μέμνηται, οὔτε τὴν μητέρα, οὔτε τὴν χώραν.
\VS{22}Καὶ ἐντεῦθεν δεῖ ὑμᾶς γνῶναι ὅτι αἱ γυναῖκες κυριεύουσιν ὑμῶν· οὐχὶ πονεῖτε, καὶ μοχθεῖτε, καὶ πάντα ταῖς γυναιξὶ δίδοτε, καὶ φέρετε;
\VS{23}Και λαμβάνει ὁ ἄνθρωπος τὴν ῥομφαίαν αὐτοῦ, καὶ ἐκπορεύεται ἐξοδεύειν καὶ λῃστεύειν καὶ κλέπτειν, καὶ εἰς τὴν θάλασσαν πλεῖν, καὶ ποταμοὺς,
\VS{24}καὶ τὸν λέοντα θεωρεῖ, καὶ ἐν σκότει βαδίζει· καὶ ὅταν κλέψῃ καὶ ἁρπάσῃ καὶ λωποδυτήσῃ, τῇ ἐρωμένῃ ἀποφέρει.
\VS{25}Καὶ πλεῖον ἀγαπᾷ ἄνθρωπος τὴν ἰδίαν γυναῖκα μᾶλλον ἢ τὸν πατέρα καὶ τὴν μητέρα.
\VS{26}Καὶ πολλοὶ ἀπενοήθησαν ταῖς ἰδίαις διανοίαις διὰ τὰς γυναῖκας, καὶ δοῦλοι ἐγένοντο διʼ αὐτάς·
\VS{27}καὶ πολλοὶ ἀπώλοντο καὶ ἐσφάλησαν καὶ ἡμάρτοσαν διὰ τὰς γυναῖκας.
\par }{\PP \VS{28}Καὶ νῦν οὐ πιστεύετέ μοι; οὐχὶ μέγας ὁ βασιλεὺς τῇ ἐξουσίᾳ αὐτοῦ; οὐχὶ πᾶσαι αἱ χῶραι εὐλαβοῦνται ἅψασθαι αὐτοῦ;
\VS{29}Ἐθεώρουν αὐτὸν, καὶ Ἀπάμην τὴν θυγατέρα Βαρτάκου τοῦ θαυμαστοῦ, τὴν παλλακὴν τοῦ βασιλέως, καθημένην ἐν δεξιᾷ τοῦ βασιλέως,
\VS{30}καὶ ἀφαιροῦσαν τὸ διάδημα ἀπὸ τῆς κεφαλῆς τοῦ βασιλέως, καὶ ἐπιτιθοῦσαν ἑαυτῇ· καὶ ἐῤῥάπιζε τὸν βασιλέα τῇ ἀριστερᾷ.
\VS{31}Καὶ πρὸς τούτοις ὁ βασιλεὺς χάσκων τὸ στόμα ἐθεώρει αὐτήν· καὶ ἐὰν προσγελάσῃ αὐτῷ, γελᾷ· ἐὰν δὲ πικρανθῇ ἐπʼ αὐτὸν, κολακεύει αὐτὴν, ὅπως διαλλαγῇ αὐτῷ.
\VS{32}Ὦ ἄνδρες, πῶς οὐχὶ ἰσχυραὶ αἱ γυναῖκες, ὅτι οὕτως πράσσουσι;
\par }{\PP \VS{33}Καὶ τότε ὁ βασιλεὺς καὶ οἱ μεγιστᾶνες ἔβλεπον εἷς τὸν ἕτερον· καὶ ἤρξατο λαλεῖν περὶ τῆς ἀληθείας·
\VS{34}Ἄνδρες, οὐχὶ ἰσχυραὶ αἱ γυναῖκες; μεγάλη ἡ γῆ, καὶ ὑψηλὸς ὁ οὐρανὸς, καὶ ταχὺς τῷ δρόμῳ ὁ ἥλιος, ὅτι στρέφεται ἐν τῷ κύκλῳ τοῦ οὐρανοῦ, καὶ πάλιν ἀποτρέχει εἰς τὸν ἑαυτοῦ τόπον ἐν μιᾷ ἡμέρᾳ.
\VS{35}Οὐχὶ μέγας ὃς ταῦτα ποιεῖ; καὶ ἡ ἀλήθεια μεγάλη καὶ ἰσχυροτέρα παρὰ πάντα.
\VS{36}Πᾶσα ἡ γῆ τὴν ἀλὴθειαν καλεῖ, καὶ ὁ οὐρανὸς αὐτὴν εὐλογεῖ, καὶ πάντα τὰ ἔργα σείεται καὶ τρέμει, καὶ οὐκ ἔστι μετʼ αὐτῆς ἄδικον οὐδέν.
\VS{37}Ἄδικος ὁ οἶνος, ἄδικος ὁ βασιλεὺς, ἄδικοι αἱ γυναῖκες, ἄδικοι πάντες οἱ υἱοὶ τῶν ἀνθρώπων, καὶ ἄδικα πάντα τὰ ἔργα αὐτῶν τὰ τοιαῦτα, καὶ οὐκ ἔστιν ἐν αὐτοῖς ἀλήθεια, καὶ ἐν τῇ ἀδικίᾳ αὐτῶν ἀπολοῦνται.
\par }{\PP \VS{38}Καὶ ἡ ἀλήθεια μένει καὶ ἰσχύει εἰς τὸν αἰῶνα, καὶ ζῇ καὶ κρατεῖ εἰς τὸν αἰῶνα τοῦ αἰῶνος.
\VS{39}Καὶ οὐκ ἔστι παρʼ αὐτὴν λαμβάνειν πρόσωπα, οὐδὲ διάφορα, ἀλλὰ τὰ δίκαια ποιεῖ ἀπὸ πάντων τῶν ἀδίκων καὶ πονηρῶν· καὶ πάντες εὐδοκοῦσι τοῖς ἔργοις αὐτῆς,
\VS{40}καὶ οὐκ ἔστιν ἐν τῇ κρίσει αὐτῆς οὐδὲν ἄδικον· καὶ αὕτη, ἡ ἰσχὺς, καὶ τὸ βασίλειον, καὶ ἡ ἐξουσία, καὶ ἡ μεγαλειότης τῶν πάντων αἰώνων· εὐλογητὸς ὁ Θεὸς τῆς ἀληθείας.
\par }{\PP \VS{41}Καὶ ἐσιώπησε τοῦ λαλεῖν· καὶ πᾶς ὁ λαὸς τότε ἐφώνησε· καὶ τότε εἶπον, μεγάλη ἡ ἀλήθεια, καὶ ὑπερισχύει·
\VS{42}τότε ὁ βασιλεὺς εἶπεν αὐτῷ, αἴτησαι ὃ θέλεις πλείω τῶν γεγραμμένων, καὶ δώσομέν σοι ὅν τρόπον εὑρέθης σοφώτερος, καὶ ἐχόμενός μου καθήσῃ, καὶ συγγενής μου κληθήθῃ.
\VS{43}Τότε εἶπε τῷ βασιλεῖ, μνήσθητι τὴν εὐχὴν, ἣν ηὔξω, οἰκοδομῆσαι τὴν Ἱερουσαλὴμ ἐν τῇ ἡμέρᾳ ᾗ τὸ βασίλειόν σου παρέλαβες,
\VS{44}καὶ πάντα τὰ σκεύη τὰ ληφθέντα ἐξ Ἱερουσαλὴμ, καὶ ἐκπέμψαι ἃ ἐχώρισε Κύρος, ὅτε ηὔξατο ἐκκόψαι Βαβυλῶνα, καὶ ηὔξατο ἐξαποστεῖλαι ἐκεῖ.
\VS{45}Καὶ σὺ ηὔξω οἰκοδομῆσαι τὸν ναὸν ὃν ἐνεπύρισαν οἱ Ἰδουμαῖοι, ὅτε ἠρημώθη ἡ Ἰουδαία ὑπὸ τῶν Χαλδαίων.
\VS{46}Καὶ νῦν τοῦτό ἐστιν ὅ σε ἀξιῶ, κύριε βασιλεῦ, καὶ ὃ αἰτοῦμαί σε, καὶ αὕτη ἐστὶν, ἡ μεγαλωσύνη ἡ παρὰ σοῦ· δέομαι οὖν ἵνα ποιήσῃς τὴν εὐχὴν, ἣν ηὔξω τῷ βασιλεῖ τοῦ οὐρανοῦ, ποιῆσαι ἐκ στόματός σου.
\par }{\PP \VS{47}Τότε ἀναστὰς Δαρεῖος ὁ βασιλεὺς κατεφίλησεν αὐτὸν, καὶ ἔγραψεν αὐτῷ τὰς ἐπιστολὰς πρὸς πάντας τοὺς οἰκονόμους, καὶ τοπάρχας, καὶ στρατηγοὺς, καὶ σατράπας, ἵνα προπέμψωσιν αὐτὸν καὶ τοὺς μετʼ αὐτοῦ πάντας ἀναβαίνοντας οἰκοδομῆσαι τὴν Ἱερουσαλήμ.
\VS{48}Καὶ πᾶσι τοῖς τοπάρχαις ἐν κοίλῃ Συρίᾳ, καὶ Φοινίκῃ, καὶ τοῖς ἐν τῷ Λιβάνω ἔγραψεν ἐπιστολὰς, μεταφέρειν ξύλα κέδρινα ἀπὸ τοῦ Λιβάνου εἰς Ἱερουσαλὴμ, καὶ ὅπως οἰκοδομήσωσι μετʼ αὐτοῦ τὴν πόλιν.
\par }{\PP \VS{49}Καὶ ἔγραψε πᾶσι τοῖς Ἰουδαίοις τοῖς ἀναβαίνουσιν ἀπὸ τῆς βασιλείας εἰς τὴν Ἰουδαίαν ὑπὲρ τῆς ἐλευθερίας, πάντα δυνατὸν, καὶ τοπάρχην, καὶ σατράπην, καὶ οἰκονόμον μὴ ἐπελεύσεσθαι ἑπὶ τὰς θύρας αὐτῶν,
\VS{50}καὶ πᾶσαν τὴν χώραν ἣν κρατοῦσιν, ἀφορολόγητον αὐτοῖς ὑπάρχειν· καὶ ἵνα οἱ Ἰδουμαῖοι ἀφίωσι τὰς κώμας ἃ διακρατοῦσι τῶν Ἰουδαίων·
\VS{51}καὶ εἰς τὴν οἰκοδομὴν τοῦ ἱεροῦ δοθῆναι κατʼ ἐνιαυτὸν τάλαντα εἴκοσι, μέχρι τοῦ οἰκοδομηθήναι·
\VS{52}καὶ ἐπὶ τὸ θυσιαστήριον ὁλοκαυτώματα καρποῦσθαι καθʼ ἡμέραν, καθὰ ἔχουσιν ἐντολὴν, ἑπτακαίδεκα προσφέρειν ἄλλα τάλαντα, δέκα κατʼ ἐνιαυτόν.
\VS{53}καὶ πᾶσι τοῖς προσβαίνουσιν ἀπὸ τῆς Βαβυλωνίας κτίσαι τὴν πόλιν, ὑπάρχειν τὴν ἐλευθερίαν αὐτοῖς τε καὶ τοῖς ἐκγόνοις αὐτῶν, καὶ πᾶσι τοῖς ἱερεῦσι τοῖς προσβαίνουσιν.
\VS{54}Ἔγραψε δὲ καὶ τὴν χορηγίαν καὶ τὴν ἱερατικὴν στολὴν ἐν τίνι λατρεύουσιν ἐν αὐτῇ.
\VS{55}Καὶ τοῖς Λευίταις ἔγραψε δοῦναι τὴν χορηγίαν, ἕως τῆς ἡμέρας ἡς ἐπιτελεσθῇ ὁ οἶκος καὶ Ἱερουσαλὴμ οἰκοδομηθῆναι.
\VS{56}Καὶ πᾶσι τοῖς φρουροῦσι τὴν πόλιν ἔγραψε δοῦναι αὐτοῖς κλήρους καὶ ὀψώνια.
\VS{57}Καὶ ἐξαπέστειλε πάντα τὰ σκεύη ἃ ἐχώρισε Κύρος ἀπὸ Βαβυλῶνος· καὶ πάντα ὅσα εἶπε Κύρος ποιῆσαι, καὶ αὐτὸς ἐπέταξε ποιῆσαι, καὶ ἐξαποστεῖλαι εἰς Ἱερουσαλήμ.
\par }{\PP \VS{58}Καὶ ὅτε ἐξῆλθεν ὁ νεανίσκος, ἄρας τὸ πρόσωπον εἰς τὸν οὐρανὸν ἐναντίον Ἱερουσαλὴμ, εὐλόγησε τῷ βασιλεῖ τοῦ οὐρανοῦ, λέγων,
\VS{59}παρὰ σοῦ νίκη, καὶ παρὰ σοῦ ἡ σοφία, καὶ σὴ ἡ δόξα, καὶ ἐγὼ σὸς οἰκέτης.
\VS{60}Εὐλογητὸς εἶ, ὃς ἔδωκάς μοι σοφίαν, καὶ σοὶ ὁμολογῶ, δέσποτα τῶν πατέρων.
\VS{61}Καὶ ἔλαβε τὰς ἐπιστολὰς, καὶ ἐξῆλθε, καὶ ἦλθεν εἰς Βαβυλῶνα, καὶ ἀπήγγειλε τοῖς ἀδελφοῖς αὐτοῦ πᾶσι.
\VS{62}Καὶ εὐλόγησαν τὸν Θεὸν τῶν πατέρων αὐτῶν, ὅτι ἔδωκεν αὐτοῖς ἄνεσιν καὶ ἄφεσιν,
\VS{63}ἀναβῆναι καὶ οἰκοδομῆσαι τὴν Ἱερουσαλὴμ καὶ τὸ ἱερὸν, οὗ ὠνομάσθη τὸ ὄνομα αὐτοῦ ἐπʼ αὐτῷ· καὶ ἐκωθωνίζοντο μετὰ μουσικῶν καὶ χαρᾶς ἡμέρας ἑπτά.

\par }\Chap{5}{\PP \VerseOne{1}Μετὰ δὲ ταῦτα ἐξελέγησαν ἀναβῆναι ἀρχηγοὶ οἴκου πατριῶν κατὰ φυλὰς αὐτῶν, καὶ αἱ γυναῖκες αὐτῶν, καὶ οἱ υἱοὶ αὐτῶν, καὶ αἱ θυγατέρες, καὶ οἱ παῖδες αὐτῶν, καὶ αἱ παιδίσκαι, καὶ τὰ κτήνη αὐτῶν.
\VS{2}Καὶ Δαρεῖος συναπέστειλε μετʼ αὐτῶν ἱππεῖς χιλίους, ἕως τοῦ ἀποκαταστῆσαι αὐτοὺς εἰς Ἱερουσαλὴμ μετʼ εἰρήνης, καὶ μετὰ μουσικῶν, τυμπάνων, καὶ αὐλῶν.
\VS{3}Καὶ πάντες οἱ ἀδελφοὶ αὐτῶν παίζοντες, καὶ ἐποίησεν αὐτοὺς συναναβῆναι μετʼ ἐκείνων.
\par }{\PP \VS{4}Καὶ ταῦτα τὰ ὀνόματα τῶν ἀνδρῶν τῶν ἀναβαινόντων κατὰ πατριὰς αὐτῶν εἰς τὰς φυλὰς, ἐπὶ τὴν μεριδαρχίαν αὐτῶν.
\VS{5}Οἱ ἱερεῖς υἱοὶ Φινεὲς, υἱοὶ Ἀαρὼν, Ἰησοῦς ὁ τοῦ Ἰωσεδὲκ τοῦ Σαραίου, καὶ Ἰωακὶμ ὁ τοῦ Ζοροβάβελ τοῦ Σαλαθιὴλ ἐκ τοῦ οἴκου τοῦ Δαυὶδ, ἐκ τῆς γενεᾶς Φαρὲς, φυλῆς δὲ Ἰούδα,
\VS{6}ὃς ἐλάλησεν ἐπὶ Δαρείου τοῦ βασιλέως Περσῶν λόγους σοφοὺς ἐν τῷ δευτέρῳ ἔτει τῆς βασιλείας αὐτοῦ, μηνὶ Νισὰν τοῦ πρώτου μηνός.
\VS{7}Εἰσὶ δὲ οὗτοι οἱ ἐκ τῆς γῆς Ἰουδαίας ἀναβάντες ἐκ τῆς αἰχμαλωσίας τῆς παροικίας, οὓς μετῴκισε Ναβουχοδονόσορ βασιλεὺς Βαβυλῶνος εἰς Βαβυλῶνα.
\VS{8}Καὶ ἐπέστρεψαν εἰς Ἱερουσαλὴμ καὶ τὴν λοιπὴν Ἰουδαίαν ἕκαστος εἰς τὴς ἰδίαν πόλιν, οἱ ἐλθόντες μετὰ Ζοροβάβελ, καὶ Ἰησοῦ, Νεεμίου, Ζαραίου, Ῥησαίου, Ἐνηνέος, Μαρδοχαίου, Βεελσάρου, Ἀσφαράσου, Ῥεελίου, Ῥοΐμου, Βαανὰ, τῶν προηγουμένων αὐτῶν.
\par }{\PP \VS{9}Ἀριθμὸς τῶν ἀπὸ τοῦ ἔθνους καὶ οἱ προηγούμενοι αὐτῶν· υἱοὶ Φόρος, δύο χιλιάδες καὶ ἑκατὸν ἑβδομηκονταδύο· υἱοὶ Σαφὰτ, τετρακόσιοι ἑβδομηκονταδύο.
\par }{\PP \VS{10}Υἱοὶ Ἀρὲς, ἑπτακόσιοι πεντηκονταέξ.
\par }{\PP \VS{11}Υἱοὶ Φαὰθ Μωὰβ εἰς τοὺς υἱοὺς Ἰησοῦ καὶ Ἰωὰβ, δισχίλιοι ὀκτακόσιοι δεκαδύο.
\par }{\PP \VS{12}Υἱοὶ Ἠλὰμ, χίλιοι διακόσιοι πεντηκοντατέσσαρες· υἱοὶ Ζαθουῒ, ἐννακόσιοι ἑβδομηκονταπέντε· υἱοὶ Χορβὲ, ἑπτακόσιοι πέντε· υἱοὶ Βανί, ἑξακόσιοι τεσσαρακονταοκτώ.
\par }{\PP \VS{13}Υἱοὶ Βηβαὶ, ἑξακόσιοι τριακοντατρεῖς· υἱοὶ Ἀργαὶ, χίλιοι τριακόσιοι εἰκοσιδύο.
\par }{\PP \VS{14}Υἱοὶ Ἀδωνικὰν, ἑξακόσιοι τριακονταεπτά. υἱοὶ Βαγοῒ, δισχίλιοι ἑξακόσιοι ἕξ· υἱοὶ Ἀδινοὺ, τετρακόσιοι πεντηκοντατέσσαρες·
\par }{\PP \VS{15}Υἱοὶ Ἀτὴρ Ἐζεκίου, ἐννενηκονταδύο· υἱοὶ Κιλὰν, καὶ Ἀζηνὰν, ἑξηκονταεπτά· υἱοὶ Ἀζαροὺ, τετρακόσιοι τριακονταδύο.
\par }{\PP \VS{16}Υἱοὶ Ἁννὶς, ἑκατὸν εἷς· υἱοὶ Ἀρὸμ, τριακονταδύο· υἱοὶ Βασσαὶ, τριακόσιοι εἰκοσιτρεῖς· υἱοὶ Ἀρσιφουρὶθ, ἑκατὸν δύο.
\par }{\PP \VS{17}Υἱοὶ Βαιτηροὺς, τρισχίλιοι πέντε· υἱοὶ ἐκ Βαιθλωμῶν, ἑκατὸν εἰκοσιτρεῖς.
\par }{\PP \VS{18}Οἱ ἐκ Νετωφὰς, πεντηκονταπέντε· οἱ ἐξ Ἀναθὼθ, ἑκατὸν πεντηκονταοκτώ· οἱ ἐκ Βαιθασμὼν, τεσσαρακονταδύο.
\par }{\PP \VS{19}Οἱ ἐκ Καριαθιρὶ, εἰκοσιπέντε· οἱ ἐκ Καφείρας, καὶ Βηρὼγ, ἑπτακόσιοι τεσσαρακοντατρεῖς.
\par }{\PP \VS{20}Οἱ Χαδιασαὶ καὶ Ἁμμίδιοι, τετρακόσιοι εἰκοσιδύο· οι ἐκ Κιραμᾶς καὶ Γαββῆς, ἑξακόσιοι εἴκοσι εἷς.
\par }{\PP \VS{21}Οἱ ἐκ Μακαλὼν, ἑκατὸν εἰκοσιδύο· οἱ ἐκ Βετολίῳ, πεντηκονταδύο· υἱοὶ Νιφὶς, ἑκατὸν πεντηκονταέξ.
\par }{\PP \VS{22}Υἱοὶ Καλαμωλάλου, καὶ Ὠνοὺς, ἑπτακόσιοι εἰκοσιπέντε· υἱοὶ Ἱερεχοὺ, διακόσιοι τεσσαρακονταπέντε.
\par }{\PP \VS{23}Υἱοὶ Σανάας, τρισχίλιοι τριακόσιοι εἷς.
\par }{\PP \VS{24}Οἱ ἱερεῖς οἱ υἱοὶ Ἰεδδοὺ τοῦ Ἰησοῦ εἰς τοὺς υἱοὺς Σανασὶβ, ὀκτακόσιοι ἑβδομηκονταδύο· υἱοὶ Ἐμμηροὺθ, διακόσιοι πεντηκονταδύο.
\par }{\PP \VS{25}Υἱοὶ Φασσούρου, χίλιοι τεσσαρακονταεπτά· υἱοὶ Χαρμι, διακόσιοι δεκαεπτά.
\par }{\PP \VS{26}Οἱ Λευῖται οἱ υἱοὶ Ἰησοῦ, καὶ Καδοήλου, καὶ Βάννου, καὶ Σουδίου, ἑβδομηκοντατέσσαρες·
\par }{\PP \VS{27}Οἱ ἱεροψάλται υἱοὶ Ἀσὰρ, ἑκατὸν εἰκοσιοκτώ.
\par }{\PP \VS{28}Οἱ θυρωροὶ υἱοὶ Σαλοὺμ, υἱοὶ Ἀτὰρ, υἱοὶ Τολμὰν, υἱοὶ Δακοὺβ, Ἀτητὰ, υἱοὶ Τωβὶς, πάντες ἑκατὸν τριακονταεννέα.
\par }{\PP \VS{29}Οἱ ἱερόδουλοι, υἱοὶ Ἡσαὺ, υἱοὶ Ἀσιφὰ, υἱοὶ Ταβαὼθ, υἱοὶ Κηρὰς, υἱοὶ Σουδὰ, υἱοὶ Φαλαίου, υἱοὶ Λαβανὰ, υἱοὶ Ἀγραβὰ,
\par }{\PP \VS{30}Υἱοὶ Ἀκοὺδ, υἱοὶ Οὐτὰ, υἱοὶ Κητὰβ, υἱοὶ Ἀκκαβὰ, υἱοὶ Συβαῒ, υἱοὶ Ἀνὰν, υἱοὶ Καθουὰ, υἱοὶ Γεδδοὺρ,
\par }{\PP \VS{31}Υἱοὶ Ἰαΐρου, υἱοὶ Δαισὰν, υἱοὶ Νοεβὰ, υἱοὶ Χασεβά, υἱοὶ Καζηρὰ, υἱοὶ Ὀζίου, υἱοὶ Φινοὲ, υἱοὶ Ἀσαρὰ, υἱοὶ Βασθαῒ, υἱοὶ Ἀσσανὰ, υἱοὶ Μανὶ, υἱοὶ Ναφισὶ, υἱοὶ Ἀκοὺφ, υἱοὶ Ἀχιβὰ, υἱοὶ Ἀσοὺβ, υἱοὶ Φαρακὲμ, υἱοὶ Βασαλὲμ,
\par }{\PP \VS{32}Υἱοὶ Μεεδδὰ, υἱοὶ Κουθὰ, υἱοὶ Χαρέα, υἱοὶ Βαρχουὲ, υἱοὶ Σερὰρ, υἱοὶ Θομοῒ, υἱοὶ Νασί, υἱοὶ Ἀτεφά·
\par }{\PP \VS{33}Υἱοὶ παίδων Σαλωμὼν, υἱοὶ Ἀσσαπφιὼθ, υἱοὶ Φαριρὰ, υἱοὶ Ἰειηλὶ, υἱοὶ Λοζὼν, υἱοὶ Ἰσδαὴλ, υἱοὶ Σαφυῒ,
\par }{\PP \VS{34}Υἱοὶ Ἁγιὰ, υἱοὶ Φακαρὲθ, υἱοὶ Σαβιὴ, υἱοὶ Σαρωθὶ, υἱοὶ Μισαίας, υἱοὶ Γὰς, υἱοὶ Ἀδδοὺς, υἱοὶ Σουβὰ, υἱοὶ Ἀφεῤῥὰ, υἱοὶ Βαρωδὶς, υἱοὶ Σαφὰγ, υἱοὶ Ἀλλώμ·
\par }{\PP \VS{35}Πάντες οἱ ἱερόδουλοι, καὶ οἱ υἱοὶ τῶν παίδων Σαλωμὼν τριακόσιοι ἑβδομηκονταδύο.
\par }{\PP \VS{36}Οὗτοι ἀναβάντες ἀπὸ Θερμελὲθ, καὶ Θελερσὰς, ἡγούμενος αὐτῶν Χαρααθαλὰν, καὶ Ἀλάρ.
\VS{37}Καὶ οὐκ ἠδύναντο ἀπαγγεῖλαι τὰς πατριὰς αὐτῶν καὶ γενεὰς, ὡς ἐκ τοῦ Ἰσραήλ εἰσιν· υἱοὶ Δαλὰν τοῦ υἱοῦ τοῦ Βαενὰν, υἱοὶ Νεκωδὰν, ἑξακόσιοι πεντη· κονταδύο.
\par }{\PP \VS{38}Καὶ ἐκ τῶν ἱερέων οἱ ἐμποιούμενοι ἱερωσύνης, καὶ οὐχ εὑρέθησαν, υἱοὶ Ὀβδία, υἱοὶ Ἁκβὼς, υἱοὶ Ἰαδδοὺ τοῦ λαβόντος Αὐγίαν γυναῖκα τῶν θυγατέρων Φαηζελδαίου, καὶ ἐκλήθη ἐπὶ τῷ ὀνόματι αὐτοῦ.
\par }{\PP \VS{39}Καὶ τούτων ζητηθείσης τῆς γενικῆς γραφῆς ἐν τῷ καταλοχισμῷ καὶ μὴ εὑρεθείσης, ἐχωρίσθησαν τοῦ ἱερατεύειν.
\VS{40}Καὶ εἶπεν αὐτοῖς Νεεμίας καὶ Ἀτθαρίας, μὴ μετέχειν τῶν ἁγίων ἕως ἀναστῇ ἀρχιερεὺς ἐνδεδυμένος τὴν δήλωσιν καὶ τὴν ἀλήθειαν.
\par }{\PP \VS{41}Οἱ δὲ πάντες Ἰσραὴλ ἦσαν ἀπὸ δωδεκαετοῦς καὶ ἐπάνω χωρὶς παίδων καὶ παιδισκῶν, μυριάδες τέσσαρες δισχίλιοι τριακόσιοι ἑξήκοντα.
\VS{42}Παῖδες τούτων καὶ παιδίσκαι, ἑπτακισχίλιοι τριακόσιοι τριακονταεπτά. ψάλται καὶ ψαλτῳδοὶ, διακόσιοι τεσσαρακονταπέντε·
\VS{43}Κάμηλοι τετρακόσιοι τριακονταπέντε, καὶ ἵπποι ἑπτακισχίλιοι τριακονταὲξ, ἡμίονοι διακόσιοι τεσσαράκοντα πέντε, ὑποζύλια πεντακισχίλια πεντακόσια εἰκοσιπέντε.
\par }{\PP \VS{44}Καὶ ἐκ τῶν ἡγουμένων κατὰ τὰς πατριὰς ἐν τῷ παραγίνεσθαι αὐτούς εἰς τὸ ἱερὸν τοῦ Θεοῦ τὸ ἐν Ἱερουσαλὴμ, ηὔξαντο ἐγεῖραι τὸν οἶκον ἐπὶ τοὺ τοῦ τόπου αὐτοῦ κατὰ τὴν αὐτῶν δύναμιν,
\VS{45}καὶ δοῦναι εἰς τὸ ἱερὸν γαζοφυλάκιον τῶν ἔργων, χρυσίου μνᾶς χιλίας καὶ ἀργυρίου μνᾶς πεντακισχιλίας, καὶ στολὰς ἱερατικὰς ἑκατόν.
\VS{46}Καὶ κατῳκίσθησαν οἱ ἱερεῖς, καὶ οἱ Λευεῖται, καὶ οἱ ἐκ τοῦ λαοῦ αὐτοῦ ἐν Ἱερουσαλὴμ καὶ τῇ χώρᾳ, οἵ τε ἱεροψάλται, καὶ οἱ θυρωροὶ, καὶ πᾶς Ἰσραὴλ ἐν ταῖς κώμαις αὐτῶν.
\par }{\PP \VS{47}Ἐνστάντος δὲ τοῦ ἑβδόμον μηνὸς, καὶ ὄντων τῶν υἱῶν Ἰσραὴλ ἑκάστου ἐν τοῖς ἰδίοις, συνήχθησαν ὁμοθυμαδὸν εἰς τὸ εὐρύχωρον τοῦ πρώτου πυλῶνος τοῦ πρὸς τῇ ἀνατολῇ.
\VS{48}Καὶ καταστὰς Ἰησοῦς ὁ τοῦ Ἰωσεδὲκ καὶ οἱ ἀδελφοὶ αὐτοῦ οἱ ἱερεῖς, καὶ Ζοροβάβελ ὁ τοῦ Σαλαθιὴλ καὶ οἱ τούτου ἀδελφοὶ, ἡτοίμασαν τὸ θυσιαστήριον τοῦ Θεοῦ Ἰσραὴλ,
\VS{49}προσενέγκαι ἐπʼ αὐτοῦ ὁλοκαυτώσεις, ἀκολούθως τοῖς ἐν τῇ Μωσέως βίβλῳ τοῦ ἀνθρώπου τοῦ Θεοῦ διηγορευμένοις.
\par }{\PP \VS{50}Καὶ ἐπισυνήχθησαν αὐτοῖς ἐκ τῶν ἄλλων ἐθνῶν τῆς γῆς, καὶ κατώρθωσαν τὸ θυσιαστήριον ἐπὶ τοῦ τόπου αὐτῶν, ὅτι ἐν ἔχθρᾳ ἦσαν αὐτοῖς, καὶ κατίσχυσαν αὐτοὺς πάντα τὰ ἔθνη τὰ ἐπὶ τῆς γῆς· καὶ ἀνέφερον θυσίας κατὰ τὸν καιρὸν, καὶ ὁλοκαυτώματα Κυρίῳ τὸ πρωϊνὸν καὶ τὸ δειλινόν.
\VS{51}Καὶ ἐγάγοσαν τὴν τῆς σκηνοπηγίας ἑορτὴν, ὡς ἐπιτέτακται ἐν τῷ νόμῳ, καὶ θυσίας καθʼ ἡμέραν, ὡς προσῆκον ἦν·
\VS{52}καὶ μετὰ ταῦτα προσφορὰς ἐνδελεχισμοῦ, καὶ θυσίας σαββάτων καὶ νουμηνιῶν καὶ ἑορτῶν πασῶν ἡγιασμένων.
\par }{\PP \VS{53}Καὶ ὅσοι ηὔξαντο εὐχὴν τῷ Θεῷ ἀπὸ τῆς νουμηνίας τοῦ ἑβδόμου μηνὸς, ἤρξαντο προσφέρειν θυσίας τῷ Θεῷ, καὶ ὁ ναὸς τοῦ Θεοῦ οὔπω ᾠκοδόμητο.
\par }{\PP \VS{54}Καὶ ἔδωκαν ἀργύριον τοῖς λατόμοις καὶ τέκτοσι, καὶ ποτὰ καὶ βρωτὰ,
\VS{55}καὶ χάῤῥα τοῖς Σιδωνίοις καὶ Τυρίοις εἰς τὸ παράγειν αὐτοὺς ἐκ τοῦ Λιβάνου ξύλα κέδρινα, διαφέρειν σχεδίας εἰς τὸν Ἰόππης λιμένα, κατὰ τὸ πρόσταγμα τὸ γραφὲν αὐτοῖς παρὰ Κύρου τοῦ Περσῶν βασιλέως.
\par }{\PP \VS{56}Καὶ τῷ δευτέρῳ ἔτει παραγενόμενος εἰς τὸ ἰερὸν τοῦ Θεοῦ εἰς Ἱερονσαλὴμ μηνὸς δευτέρου, ἤρξατο Ζοροβάβελ ὁ τοῦ Σαλαθιὴλ, καὶ Ἰησοῦς ὁ τοῦ Ἰωσεδὲκ, καὶ οἱ ἀδελφοὶ αὐτῶν, καὶ οἱ ἱερεῖς οἱ Λευῖται, καὶ πάντες οἱ παραγενόμενοι ἐκ τῆς αἰχμαλωσίας εἰς Ἱερουσαλὴμ,
\VS{57}καὶ ἐθεμελίωσαν τὸν ναὸν τοῦ Θεοῦ τῇ νουμηνίᾳ τοῦ δευτέρου μηνὸς τοῦ δευτέρου ἔτους, ἐν τῷ ἐλθεῖν εἰς τὴν Ἰουδαίαν καὶ Ἱερουσαλήμ.
\VS{58}Καὶ ἔστησαν τοὺς Λευίτας ἀπὸ εἰκοσαετοῦς ἐπὶ τῶν ἔργων τοῦ Κυρίου· καὶ ἔστη Ἰησοῦς, καὶ οἱ υἱοὶ, καὶ οἱ ἀδελφοὶ, καὶ Καδμιὴλ ὁ ἀδελφὸς, καὶ οἱ υἱοὶ Ἠμαδαβοὺν, καὶ οἱ υἱοὶ Ἰωδὰ τοῦ Ἡλιαδοὺδ σὺν τοῖς υἱοῖς καὶ ἀδελφοῖς, πάντες οἱ Λευῖται ὁμοθυμαδὸν ἐργοδιῶκται, ποιοῦντες εἰς τὰ ἔργα ἐν τῷ οἴκῳ τοῦ Κυρίου· καὶ ᾠκοδόμησαν οἱ οἰκοδόμοι τὸν ναὸν τοῦ Κυριου.
\par }{\PP \VS{59}Καὶ ἔστησαν οἱ ἱερεῖς ἐστολισμενοι μετὰ μουσικῶν καὶ σαλπίγγων,
\VS{60}καὶ οἱ Λευῖται υἱοὶ Ἀσὰφ ἔχοντες τὰ κύμβαλα ὑμνοῦντες τῶ Κυρίῳ, καὶ εὐλογοῦντες κατὰ Δαυὶδ βασιλέα τοῦ Ἰσραήλ.
\par }{\PP \VS{61}Καὶ ἐφώνησαν διʼ ὕμνων εὐλογοῦντες τῷ Κυρίῳ, ὅτι ἡ χρηστότης αὐτοῦ καὶ ἡ δόξα εἰς τοὺς αἰῶνας ἐν παντὶ Ἰσραήλ.
\VS{62}Καὶ πᾶς ὁ λαὸς ἐσάλπισαν καὶ ἐβόησαν φωνῇ μεγάλῃ, ὑμνοῦντες τῷ Κυρίῳ ἐπὶ τῇ ἐγέρσει τοῦ οἴκου Κυρίου.
\par }{\PP \VS{63}Καὶ ἤλθοσαν ἐκ τῶν ἱερεων τῶν Λευιτῶν καὶ τῶν προκαθημένων κατὰ τὰς πατριὰς αὐτῶν, οἱ πρεσβύτεροι οἱ ἑωρακότες τὸν πρὸ τούτου οἶκον, πρὸς τὴν τουτου οἰκοδομὴν μετὰ κλαυθμοῦ καὶ κραυγῆς μεγάλης,
\VS{64}καὶ πολλοὶ διὰ σαλπιγγων καὶ χαρᾶς μεγάλῃ τῇ φωνῇ,
\VS{65}ὥστε τὸν λαὸν μὴ ἀκούειν τῶν σαλπίγγων διὰ τὸν κλαυθμὸν τοῦ λαοῦ· ὁ γὰρ ὄχλος ἦν ὁ σαλπίζων μεγάλως, ὥστε μακρόθεν ἀκούεσθαι.
\par }{\PP \VS{66}Καὶ ἀκούσαντες οἱ ἑχθροί τῆς φυλῆς Ἰούδα καὶ Βενιαμὶν, ἤλθοσαν ἐπιγνῶναι τίς ἡ φωνὴ τῶν σαλπίγγων.
\VS{67}Καὶ ἐπέγνωσαν ὅτι οἱ ἐκ τῆς αἰχμαλωσίας οἰκοδομοῦσι τὸν ναὸν τῷ Κυρίῳ Θεῷ Ἰσραήλ.
\VS{68}Καὶ προσελθόντες τῷ Ζοροβάβελ, καὶ Ἰησοῦ, καὶ τοῖς ἡγουμένοις τῶν πατριῶν, λέγουσιν αὐτοῖς, συνοικοδομήσωμεν ὑμῖν.
\VS{69}Ὁμοίως γὰρ ὑμῖν ἀκούομεν τοῦ Κυρίου ὑμῶν, καὶ αὐτῷ ἐπιθύομεν ἀφʼ ἡμερῶν Ἀσβακαφὰς βασιλέως Ἀσσυρίων, ὃς μετήγαγεν ἡμᾶς ἐνταῦθα.
\par }{\PP \VS{70}Καὶ εἶπεν αὐτοῖς Ζοροβάβελ καὶ Ἰησοῦς καὶ οἱ ἡγούμενοι τῶν πατριῶν τοῦ Ἰσραὴλ, οὐχ ἡμῖν καὶ ὑμῖν τοῦ οἰκοδομῆσαι τὸν οἶκον Κυρίῳ Θεῷ ἡμῶν.
\VS{71}Ἡμεῖς γὰρ μόνοι οἰκοδομήσωμεν τῷ Κυρίῳ τοῦ Ἰσραὴλ, ἀκολούθως οἷς προσέταξεν ἡμῖν Κύρος ὁ βασιλεὺς Περσῶν.
\VS{72}Τὰ δὲ ἔθνη τῆς γῆς ἐπικοιμώμενα τοῖς ἐν τῇ Ἰουδαίᾳ καὶ πολιορκοῦντες, εἶργον τοῦ οἰκοδομεῖν,
\VS{73}καὶ βουλὰς δημαγωγοῦντες, καὶ συστάσεις ποιούμενοι, ἀπεκώλυσαν τοῦ ἀποτελεσθῆναι τὴν οἰκοδομὴν πάντα τὸν χρόνον τῆς ζωῆς τοῦ βασιλέως Κύρου· καὶ εἴρχθησαν τῆς οἰκοδομῆς ἔτη δύο ἕως τῆς Δαρείου βασιλείας.

\par }\Chap{6}{\PP \VerseOne{1}Ἐν δὲ τῷ δευτέρῳ ἔτει τῆς Δαρείου βασιλείας, ἐπροφήτευσεν Ἀγγαῖος καὶ Ζαχαρίας ὁ τοῦ Ἀδδὼ οἱ προφῆται ἐπὶ τοὺς τοὺς Ἰουδαίους τοὺς ἐν τῇ Ἰουδαίᾳ καὶ Ἱερουσαλὴμ, ἐπὶ τῷ ὀνόματι Κυρίου Θεοῦ Ἰσραὴλ ἐπʼ αὐτούς.
\par }{\PP \VS{2}Τότε στὰς Ζοροβάβελ ὁ τοῦ Σαλαθιὴλ καὶ Ἰησοῦς ὁ τοῦ Ἰωσεδὲκ, ἥρξαντο οἰκοδομεῖν τὸν οἶκον τοῦ Κυρίου, τὸν ἐν Ἱερουσαλὴμ, συνόντων τῶν προφητῶν τοῦ Κυρίου, βοηθούντων αὐτοῖς.
\VS{3}Ἐν αὐτῷ τῷ χρόνῳ παρῆν πρὸς αὐτοὺς Σισίννης ὁ ἔπαρχος Συρίας καὶ Φοινίκης, καὶ Σαθραβουζάνης καὶ οἱ συνεταῖροι,
\VS{4}καὶ εἶπαν αὐτοῖς, τίνος ὑμῖν συντάξαντος τὸν οἶκον τοῦτον οἰκοδομεῖτε, καὶ τὴν στέγην ταύτην καὶ τὰ ἄλλα πάντα ἐπιτελεῖτε; καὶ τίνες εἰσὶν οἰκοδόμοι οἱ ταῦτα ἐπιτελοῦντες;
\par }{\PP \VS{5}Καὶ ἔσχοσαν χάριν, ἐπισκοπῆς γενομένης ἐπὶ τὴν αἰχμαλωσίαν, παρὰ τοῦ Κυρίου οἱ πρεσβύτεροι τῶν Ἰουδαίων,
\VS{6}καὶ οὐκ ἐκωλύθησαν τῆς οἰκοδομῆς, μέχρις οὗ ἀποσημανθῆναι Δαρείῳ περὶ αὐτῶν, καὶ προσφωνηθῆναι.
\par }{\PP \VS{7}ἈΝΤΙΓΡΑΦΟΝ ἘΠΙΣΤΟΛΗΣ ἯΣ ἜΓΡΑΨΕΔΑΡΕΙΩ, ΚΑΙ ἈΠΕΣΤΕΙΛΑΝ. Σισίννης ὁ ἔπαρχος Συρίας καὶ Φοινίκης, καὶ Σαθραβουζάνης, καὶ οἱ συνεταῖροι οἱ ἐν Συρίᾳ καὶ Φοινίκῃ ἡγεμόνες, βασιλεῖ Δαρείῳ χαίρειν.
\VS{8}Πάντα γνωστὰ ἔστω τῶ κυρίῳ ἡμῶν τῷ βασιλεῖ, ὅτι παραγενόμενοι εἰς τὴν χώραν τῆς Ἰουδαίας, καὶ ἐλθόντες εἰς Ἱερουσαλὴμ τὴν πόλιν, κατελάβομεν τῆς αἰχμαλωσίας τοὺς πρεσβυτέρους τῶν Ἰουδαίων ἐν Ἱερουσαλὴμ τῇ πόλει οἰκοδομοῦντας οἶκον τῷ Κυρίῳ μέγαν,
\VS{9}καινὸν διὰ λίθων ξυστῶν πολυτελῶν, ξύλων τιθεμένων ἐν τοῖς τοίχοις,
\VS{10}καὶ τὰ ἔργα ἐκεῖνα ἐπὶ σπουδῆς γινόμενα, καὶ εὐοδούμενον τὸ ἔργον ἐν ταῖς χερσὶν αὐτῶν, καὶ ἐν πάσῃ δόξῃ καὶ ἐπιμελείᾳ συντελούμενον.
\par }{\PP \VS{11}Τότε ἐπυνθανόμεθα τῶν πρεσβυτέρων τούτων, λέγοντες, τίνος ὑμῖν προστάξαντος οἰκοδομεῖτε τὸν οἶκον τοῦτον, καὶ τὰ ἔργα ταῦτα θεμελιοῦτε;
\VS{12}Ἐπηρωτήσαμεν οὖν αὐτοὺς, εἵνεκεν τοῦ γνωρίσαι σοι, καὶ γράψαι σοι τοὺς ἀνθρώπους τοὺς ἀφηγουμένους, καὶ τὴν ὀνοματογραφίαν ᾐτοῦμεν αὐτοὺς τῶν προκαθηγουμένων.
\VS{13}Οἱ δὲ ἀπεκρίθησαν ἡμῖν, λέγοντες, ἐσμὲν παῖδες τοῦ Κυρίου τοῦ κτίσαντος τὸν οὐρανὸν καὶ τὴν γῆν·
\VS{14}καὶ ᾠκοδόμητο οἶκος ἔμπροσθεν ἐτῶν πλειόνων διὰ βασιλέως τοῦ Ἰσραὴλ μεγάλου καὶ ἰσχυροῦ, καὶ ἐπετελέσθη.
\VS{15}Καὶ ἐπεὶ οἱ πατέρες ἡμῶν παραπικράναντες ἥμαρτον εἰς τὸν Κύριον τοῦ Ἰσραὴλ τὸν οὐράνιον, παρέδωκεν αὐτοὺς εἰς χεῖρας Ναβουχοδονόσορ βασιλέως Βαβυλῶνος βασιλέως τῶν Χαλδαίων.
\VS{16}Τόν τε οἶκον καθελόντες ἐνεπύρισαν, καὶ τὸν λαὸν ᾐχμαλώτευσαν εἰς Βαβυλῶνα.
\par }{\PP \VS{17}Ἐν δὲ τῷ πρώτῳ ἔτει βασιλεύοντος Κύρου χώρας Βαβυλωνίας, ἔγραψεν ὁ βασιλεὺς Κύρος τὸν οἶκον τοῦτον οἰκοδομῆσαι·
\VS{18}Καὶ τὰ ἱερὰ σκεύη τὰ χρυσᾶ καὶ τὰ ἀργυρᾶ, ἃ ἐξήνεγκε Ναβουχοδονόσορ ἐκ τοῦ οἴκου τοῦ ἐν Ἱερουσαλὴμ, καὶ ἀπηρείσατο αὐτὰ ἐν τῷ αὐτοῦ ναῷ, πάλιν ἐξήνεγκεν αὐτὰ Κύρος ὁ βασιλεὺς ἐκ τοῦ ναοῦ τοῦ ἐν Βαβυλωνίᾳ, καὶ παρεδόθη Σαβανασσάρῳ Ζοροβάβελ τῷ ἐπάρχῳ,
\VS{19}καὶ ἐπετάγη αὐτῷ, καὶ ἀπήνεγκε πάντα τὰ σκεύη ταῦτα ἀποθεῖναι ἐν τῷ ναῷ τῷ ἐν Ἱερουσαλὴμ, καὶ τὸν ναὸν τοῦ Κυρίου οἰκοδομηθῆναι ἐπὶ τοῦ τόπου.
\VS{20}Τότε ὁ Σαβανάσσαρος παραγενόμενος ἐνεβάλετο τοὺς θεμελίους τοῦ οἴκου Κυρίου τοῦ ἐν Ἱερουσαλὴμ, καὶ ἀπʼ ἐκείνου μέχρι τοῦ νῦν οἰκοδομούμενος οὐκ ἔλαβε συντέλειαν.
\par }{\PP \VS{21}Νῦν οὖν εἰ κρίνεται, βασιλεῦ, ἐπισκεπήτω ἐν τοῖς βασιλικοῖς βιβλιοφυλακίοις τοῦ Κύρου,
\VS{22}καὶ ἐὰν εὑρίσκητε; μετὰ τῆς γνώμης Κύρου τοῦ βασιλέως γενομένην τὴν οἰκοδομὴν τοῦ οἴκου Κυρίου τοῦ ἐν Ἱερουσαλὴμ, καὶ κρίνηται τῷ κυρίῳ βασιλεῖ ἡμῶν, προσφωνησάτω ἡμῖν περὶ τοῦτων.
\par }{\PP \VS{23}Τότε ὁ βασιλεὺς Δαρεῖος προσέταξεν ἐπισκέψασθαι ἐν τοῖς βιβλιοφυλακίοις τοῖς κειμένοις ἐν Βαβυλῶνι· καὶ εὑρέθη ἐν Ἐκβατάνοις τῇ βάρει τῇ ἐν Μηδίᾳ χώρᾳ τόπος εἷς, ἐν ῷ ὑπομνημάτιστο τάδε.
\VS{24}Ἔτους πρώτου βασιλεύοντος Κύρου, βασιλεὺς Κύρος προσέταξε τὸν οἶκον τοῦ Κυρίου τὸν ἐν Ἱερουσαλὴμ οἰκοδομῆσαι, ὅπου ἐπιθύουσι διὰ πυρὸς ἐνδελεχοῦς, οὗ τὸ ὕψος πηχῶν ἑξήκοντα,
\VS{25}πλάτος πηχῶν ἑξήκοντα διὰ δόμων λιθίνων ξυστῶν τριῶν, καὶ δόμου ξυλίνου ἐγχωρίου καινοῦ ἑνὸς, καὶ τὸ δαπάνημα δοθῆναι ἐκ τοῦ οἴκου Κύρου τοῦ βασιλέως.
\VS{26}Καὶ τὰ ἱερὰ σκεύη τοῦ οἴκου Κυρίου τά τε χρυσᾶ καὶ ἀργυρᾶ, ἃ ἐξήνεγκε Ναβουχοδονόσορ ἐκ τοῦ οἴκου τοῦ ἐν Ἱερουσαλὴμ, καὶ ἀπήνεγκεν εἰς Βαβυλῶνα, ἀποκατασταθῆναι εἰς τὸν οἶκον τὸν ἐν Ἱερουσαλὴμ, οὗ ἦν κείμενα, ὅπως τεθῇ ἐκεῖ.
\par }{\PP \VS{27}Προσέταξε δὲ ἐπιμεληθῆναι Σισίννῃ ἐπάρχῳ Συρίας καὶ Φοινίκης, καὶ Σαθραβουζάνῃ, καὶ τοῖς συνεταίροις, καὶ τοῖς ἀποτεταγμένοις ἐν Συρίᾳ καὶ Φοινίκῃ ἡγεμόσιν ἀπέχεσθαι τοῦ τόπου, ἐᾶσαι δὲ τὸν παῖδα Κυρίου Ζοροβάβελ, ἔπαρχον δὲ τῆς Ἰουδαίας, καὶ τοὺς πρεσβυτέρους τῶν Ἰουδαίων, τὸν οἶκον τοῦ Κυρίου ἐκεῖνον οἰκοδομεῖν ἐπὶ τοῦ τόπου.
\VS{28}Καὶ ἐγὼ δὲ ἐπέταξα ὁλοσχερῶς οἰκοδομῆσαι, καὶ ἀτενίσαι ἵνα συμποιῶσι τοῖς ἐκ τῆς αἰχμαλωσίας τῆς Ἰουδαίας, μέχρι τοῦ ἐπιτελεσθῆναι τὸν οἶκον τοῦ Κυρίου·
\VS{29}καὶ ἀπὸ τῆς φορολογίας κοίλης Συρίας καὶ Φοινίκης ἐπιμελῶς σύνταξιν δίδοσθαι τούτοις τοῖς ἀνθρώποις εἰς θυσίαν τῷ Κυρίῳ, Ζοροβάβελ ἐπάρεχῳ εἰς ταύρους, καὶ κριοὺς, καὶ ἄρνας,
\VS{30}ὁμοίως δὲ καὶ πυρὸν, καὶ ἅλα, καὶ οἶνον, καὶ ἔλαιον ἐνδελεχῶς κατʼ ἐνιαυτὸν, καθὼς ἂν οἱ ἱερεῖς οἱ ἐν Ἱερουσαλὴμ ὑπαγορεύσωσιν ἀναλίσκεσθαι καθʼ ἡμέραν, ἀναμφισβητήτως,
\VS{31}ὅπως προσφέρωνται σπονδαὶ τῷ Θεῷ τῷ ὑψίστῳ ὑπὲρ τοῦ βασιλέως καὶ τῶν παίδων, καὶ προσεύχωνται περὶ τῆς αὐτῶν ζωῆς·
\VS{32}καὶ προστάξαι ἵνα ὅσοι ἐὰν παραβῶσί τὶ τῶν γεγραμμένων καὶ ἀκυρώσωσι, ληφθῆναι ξύλον ἐκ τῶν ἰδίων αὐτοῦ, καὶ ἐπʼ αὐτοῦ κρεμασθῆναι, καὶ τὰ ὑπάρχοντα αὐτοῦ εἶναι βασιλικά.
\par }{\PP \VS{33}Διὰ ταῦτα καὶ ὁ Κύριος, οὗ τὸ ὄνομα αὐτοῦ ἐπικέκληται ἐκεῖ, ἀφανίσαι πάντα βασιλέα καὶ ἔθνος, ὃς ἐκτενεῖ τὴν χεῖρα αὐτοῦ κωλῦσαι ἢ κακοποιῆσαι τὸν οἶκον Κυρίου ἐκεῖνον τὸν ἐν Ἱερουσαλήμ.
\VS{34}Ἐγὼ βασιλεὺς Δαρεῖος δεδογμάτικα ἐπιμελῶς κατὰ ταῦτα γίνεσθαι.

\par }\Chap{7}{\PP \VerseOne{1}Τότε Σισίννης ἔπαρχος κοίλης Συρίας καὶ Φοινίκης, καὶ Σαθραβουζάνης, καὶ οἱ συνεταῖροι κατακολουθήσαντες τοῖς ὑπὸ τοῦ βασιλέως Δαρείου προσταγεῖσιν,
\VS{2}ἐπεστάτουν τῶν ἱερῶν ἔργων ἐπιμελέστερον συνεργοῦντες τοῖς πρεσβυτέροις τῶν Ἰουδαίων καὶ ἱεροστάταις.
\VS{3}Καὶ εὔοδα ἐγίνετο τὰ ἱερὰ ἔργα, προφητευόντων Ἀγγαίου καὶ Ζαχαρίου τῶν προφητῶν.
\par }{\PP \VS{4}Καὶ συνετέλεσαν ταῦτα διὰ προστάγματος Κυρίου Θεοῦ Ἰσραήλ· καὶ μετὰ τῆς γνώμης τοῦ Κύρου καὶ Δαρείου καὶ Ἀρταξέρξου βασιλέων Περσῶν,
\VS{5}συνετελέσθη ὁ οἶκος ὁ ἅγιος ἕως τρίτης καὶ εἰκάδος μηνὸς Ἄδαρ, τοῦ ἕκτου ἔτους βασιλέως Δαρείου.
\par }{\PP \VS{6}Καὶ ἐποίησαν οἱ υἱοὶ Ἰσραὴλ, καὶ οἱ ἱερεῖς καὶ οἱ Λευῖται καὶ οἱ λοιποὶ οἱ ἐκ τῆς αἰχμαλωσίας οἱ προστεθέντες, ἀκολούθως τοῖς ἐν τῇ Μωυσέως βίβλῳ.
\VS{7}Καὶ προσήνεγκαν εἰς τὸν ἐγκαινισμὸν τοῦ ἱεροῦ τοῦ Κυρίου ταύρους ἑκατὸν, κριοὺς διακοσίους, ἄρνας τετρακοσίους,
\VS{8}χιμάρους ὑπὲρ ἁμαρτίας παντὸς τοῦ Ἰσραὴλ δώδεκα πρὸς ἀριθμὸν, ἐκ τῶν φυλάρχων τοῦ Ἰσραὴλ δώδεκα.
\VS{9}Καὶ ἔστησαν οἱ ἱερεῖς καὶ οἱ Λενῖται κατὰ φυλὰς ἐστολισμένοι ἐπὶ τῶν ἔργων Κυρίου Θεοῦ Ἰσραὴλ ἀκολούθως τῇ Μωυσέως βίβλῳ, καὶ οἱ θυρωροὶ ἐφʼ ἑκάστου πυλῶνος.
\par }{\PP \VS{10}Καὶ ἠγάγοσαν οἱ υἱοὶ Ἰσραὴλ τῶν ἐκ τῆς αἰχμαλωσίας τὸ πάσχα ἐν τῇ τεσσαρεσκαιδεκάτῃ τοῦ πρώτου μηνὸς, ὅτε ἡγνίσθησαν οἱ ἱερεῖς καὶ οἱ Λευῖται,
\VS{11}ἅμα καὶ πάντες οἱ υἱοὶ τῆς αἰχμαλωσίας, ὅτι ἡγνίσθησαν· ὅτι οἱ Λευῖται ἅμα πάντες ἡγνίσθησαν.
\VS{12}Καὶ ἔθυσαν τὸ πάσχα πᾶσι τοῖς υἱοῖς τοῖς αἰχμαλωσίας, καὶ τοῖς ἀδελφοῖς αὐτῶν τοῖς ἱερεῦσι, καὶ ἑαυτοῖς.
\VS{13}Καὶ ἐφάγοσαν οἱ υἱοὶ Ἰσραὴλ οἱ ἐκ τῆς αἰχμαλωσίας, πάντες οἱ χωρισθέντες ἀπὸ τῶν βδελυγμάτων τῶν ἐθνῶν τῆς γῆς, ζητοῦντες τὸν Κύριον·
\VS{14}Καὶ ἠγάγοσαν τὴν ἑορτὴν τῶν ἀζύμων ἑπτὰ ἡμέρας εὐφραινόμενοι ἔναντι Κυρίου,
\VS{15}ὅτι μετέστρεψε τὴν βουλὴν τοῦ βασιλέως Ἀσσυρίων ἐπʼ αὐτοὺς, κατισχῦσαι τὰς χεῖρας αὐτῶν ἐπὶ τὰ ἔργα Κυρίου Θεοῦ Ἰσραήλ.

\par }\Chap{8}{\PP \VerseOne{1}Καὶ μεταγενέστερος τούτων ἐστὶ, βασιλεύοντος Ἀρταξέρξου τοῦ Περσῶν βασιλέως, προσέβη Ἔσδρας Ἀζαραίου, τοῦ Ζεχρίου, τοῦ Χελκίου, τοῦ Σαλήμου,
\VS{2}τοῦ Σαδδούκου, τοῦ Ἀχιτὼβ, τοῦ Ἀμαρίου, τοῦ Ὀζίου, τοῦ Βοκκὰ, τοῦ Ἀβισαῒ, τοῦ Φινεὲς, τοῦ Ἐλεάζαρ, τοῦ Ἀαρὼν, τοῦ ἱερέως τοῦ πρώτου·
\VS{3}οὗτος Ἔσδρας ἀνέβη ἐκ Βαβυλῶνος ὡς γραμματεὺς εὐφυὴς ὢν ἐν τῷ Μωυσέως νόμω τῷ ἐκδεδομένῳ ὑπὸ τοῦ Θεοῦ τοῦ Ἰσραήλ.
\VS{4}Καὶ ἔδωκεν αὐτῷ ὁ βασιλεὺς δόξαν, εὑρόντος χάριν ἐνώπιον αὐτοῦ ἐπὶ πάντα τὰ ἀξιώματα αὐτοῦ.
\par }{\PP \VS{5}Καὶ συνανέβησαν ἐκ τῶν υἱῶν Ἰσραὴλ, καὶ τῶν ἱερέων, καὶ Λευιτῶν, καὶ ἱεροψαλτῶν, καὶ θυρωρῶν, καὶ ἱεροδούλων εἰς Ἱερουσαλὴμ,
\VS{6}ἔτους ἑβδόμου βασιλεύοντος Ἀρταξέρξου ἐν τῷ πέμπτῳ μηνί· οὗτος ἐνιαυτὸς ἕβδομος τῷ βασιλεῖ· ἐξελθόντες γὰρ ἐκ Βαβυλῶνος τῇ νουμηνίᾳ τοῦ πρώτου μηνὸς, παρεγένοντο εἰς Ἱερουσαλὴμ κατὰ τὴν δοθεῖσαν αὐτοῖς εὐοδίαν παρὰ τοῦ Κυρίου ἐπʼ αὐτῷ.
\VS{7}Ὁ γὰρ Ἔσδρας πολλὴν ἐπιστήμην περιεῖχεν εἰς τὸ μηδὲν παραλιπεῖν τῶν ἐκ τοῦ νόμου Κυρίου καὶ ἐκ τῶν ἐντολῶν, διδάξαι πάντα τὸν Ἰσραὴλ δικαιώματα καὶ κρίματα.
\par }{\PP \VS{8}Προσπεσόντος δὲ τοῦ γραφέντος προστάγματος παρὰ Αρταξέρξου βασιλέως πρὸς Ἔσδραν τὸν ἱερέα καὶ ἀναγνώστην τοῦ νόμου Κυρίου, οὗ ἐστιν ἀντίγραφον τὸ ὑποκείμενον·
\par }{\PP \VS{9}Βασιλεὺς Ἀρταξέρξης Ἔσδρᾳ τῷ ἱερεῖ καὶ ἀναγνώστῃ τοῦ νόμου Κυρίου χαίρειν.
\VS{10}Καὶ τὰ φιλάνθρωπα ἐγὼ κρίνας προσέταξα τοὺς βουλομένους ἐκ τοῦ ἔθνους τῶν Ἰουδαίων αἱρετίζοντας, καὶ τῶν ἱερέων καὶ τῶν Λευιτῶν, καὶ τῶνδε ἐν τῇ ἡμετέρᾳ βασιλείᾳ, συμπορεύεσθαί σοι εἰς Ἱερουσαλήμ.
\VS{11}Ὅσοι οὖν ἐνθυμοῦνται, συνεξορμάσθωσαν καθάπερ δέδοκται ἐμοί τε, καὶ τοῖς ἑπτὰ φίλοις συμβουλευταῖς,
\VS{12}ὅπως ἐπισκέψωνται τὰ κατὰ τὴν Ἰουδαίαν καὶ Ἱερουσαλὴμ ἀκολούθως ᾧ ἔχει ἐν τῷ νόμῳ Κυρίου,
\VS{13}καὶ ἀπενεγκεῖν δῶρα τῷ Κυρίῳ τοῦ Ἰσραὴλ, ἃ ηὐξάμην ἐγώ τε καὶ οἱ φίλοι, εἰς Ἱερουσαλήμ· καὶ πᾶν χρυσίον καὶ ἀργύριον ὃ ἐὰν εὑρεθῇ ἐν τῇ χώρᾳ τῆς Βαβυλωνίας τῷ Κυρίῳ εἰς Ἱερουσαλὴμ
\VS{14}σὺν τῷ δεδωρημένῳ ὑπὸ τοῦ ἔθνους εἰς τὸ ἱερὸν τοῦ Κυρίου Θεοῦ αὐτῶν τὸ ἐν Ἱερουσαλὴμ, συναχθῆναι τό, τε χρυσίον καὶ τὸ ἀργύριον εἰς ταύρους καὶ κριοὺς καὶ ἄρνας, καὶ τὰ τούτοις ἀκόλουθα,
\VS{15}ὥστε προσενεγκεῖν θυσίας τῷ Κυρίῳ ἐπὶ τὸ θυσιαστήριον τοῦ Κυρίον Θεοῦ αὐτῶν τὸ ἐν Ἱερουσαλήμ.
\par }{\PP \VS{16}Καὶ πάντα ὅσα ἐὰν βούλῃ μετὰ τῶν ἀδελφῶν σου ποιῆσαι χρυσίῳ καὶ ἀργυρίῳ, ἐπιτέλει κατὰ τὸ θέλημα τοῦ Θεοῦ σου.
\VS{17}Καὶ τὰ ἱερὰ σκεύη τοῦ Κυρίου τὰ διδόμενά σοι εἰς τὴν χρείαν τοῦ ἱεροῦ τοῦ Θεοῦ σου,
\VS{18}δώσεις ἐκ τοῦ βασιλικοῦ γαζοφυλακίου.
\par }{\PP \VS{19}Κᾀγὼ ἰδοὺ Ἀρταξέρξης βασιλεὺς προσέταξα τοῖς γαζοφύλαξι Συρίας καὶ Φοινίκης, ἵνα ὅσα ἐὰν ἀποστείλῃ Ἔσδρας ὁ ἱερεὺς καὶ ἀναγνώστης τοῦ νόμου τοῦ Θεοῦ τοῦ ὑψίστου, ἐπιμελῶς διδῶσιν αὐτῷ ἕως ἀργυρίου ταλάντων ἑκατὸν,
\VS{20}ὁμοίως δὲ καὶ ἕως πυροῦ κόρων ἑκατὸν, καὶ οἴνου μετρητῶν ἑκατόν·
\VS{21}καὶ ἄλλα ἐκ πλήθους πάντα κατὰ τὸν τοῦ Θεοῦ νόμον ἐπιτελεσθήτω ἐπιμελῶς τῷ Θεῷ τῷ ὑψίστῳ, ἕνεκα τοῦ μὴ γενέσθαι ὀργὴν εἰς τὴν βασιλείαν τοῦ βασιλέως καὶ τῶν υἱῶν αὐτοῦ.
\VS{22}Καὶ ὑμῖν δὲ λέγεται ὅπως πᾶσι τοῖς ἱερεῦσι, καὶ τοῖς Λευίταις, καὶ ἱεροψάλταις, καὶ θυρωροῖς, καὶ ἱεροδούλοις, καὶ πραγματικοῖς τοῦ ἱεροῦ τούτου μηδὲ μία φορολογία, μηδὲ ἄλλη ἐπιβουλὴ γίνηται, καὶ μηδένα ἔχειν ἐξουσίαν ἐπιβαλεῖν τι τούτοις.
\par }{\PP \VS{23}Καὶ σὺ, Ἔσδρα, κατὰ τὴν σοφίαν τοῦ Θεοῦ, ἀνάδειξον κριτὰς καὶ δικαστὰς, ὅπως δικάζωσιν ἐν ὅλῃ Συρίᾳ καὶ Φοινίκῃ πάντας τοὺς ἐπισταμένους τὸν νόμον τοῦ Θεοῦ σου, καὶ τοὺς μὴ ἐπισταμένους διδάξεις.
\VS{24}Καὶ πάντες ὅσοι ἂν παραβαίνωσι τὸν νόμον τοῦ Θεοῦ σου καὶ τὸς βασιλικὸν, ἐπιμελῶς κολασθήσονται, ἐάν τε καὶ θανάτῳ, ἐάν τε καὶ τιμωρίᾳ, ἢ ἀργυρικῇ ζημίᾳ, ἢ ἀπαγωγῇ.
\par }{\PP \VS{25}Καὶ εἶπεν Ἔσδρας ὁ γραμματεὺς, εὐλογητὸς μόνος Κύριος ὁ Θεὸς τῶν πατέρων μου, ὁ δοὺς ταῦτα εἰς τὴν καρδίαν τοῦ βασιλέως, δοξάσαι τὸν οἶκον αὐτοῦ τὸν ἐν Ἱερουσαλὴμ,
\VS{26}καὶ ἐμὲ ἐτίμησεν ἐναντίον τοὺ βασιλέως, καὶ τῶν συμβουλευόντων, καὶ πάντων τῶν φίλων, καὶ μεγιστάνων αὐτοῦ.
\VS{27}Καὶ ἐγὼ εὐθαρσὴς ἐγενόμην κατὰ τὴν ἀντίλμψιν Κυρίου τοῦ Θεοῦ μου, καὶ συνήγαγον ἄνδρας ἐκ τοῦ Ἰσραὴλ ὥστε συναναβῆναί μοι.
\par }{\PP \VS{28}Καὶ οὗτοι οἱ προηγούμενοι κατὰ τὰς πατριὰς αὐτῶν καὶ τὰς μεριδαρχίας, οἱ ἀναβάντες μετʼ ἐμοῦ ἐκ Βαβυλῶνος ἐν τῇ βασιλείᾳ Ἀρταξέρξου τοῦ βασιλέως.
\VS{29}Ἐκ τῶν υἱῶν Φινεὲς, Γηρσών· ἐκ τῶν υἱῶν Ἰαθαμάρου, Γαμαλιήλ· ἐκ τῶν υἱῶν Λαυὶδ, Λαττοὺς ὁ Σεχενίου·
\VS{30}ἐκ τῶν υἱῶν Φόρος, Ζαχαρίας, καὶ μετʼ αὐτοῦ ἀπεγράφησαν ἄνδρες ἑκατὸν πεντήκοντα·
\VS{31}ἐκ τῶν υἱῶν Φαὰθ Μωὰβ, Ἐλιαωνίας Ζαραίου, καὶ μετʼ αὐτοῦ ἄνδρες διακόσιοι·
\VS{32}ἐκ τῶν υἱῶν Ζαθόης, Ζεχενίας Ἰεζήλου, καὶ μετʼ αὐτοῦ ἄνδρες τριακόσιοι· ἐκ τῶν υἱῶν Ἀδὶν, Ὠβὴθ Ἰωνάθου, καὶ μετʼ αὐτοῦ ἄνδρες διακόσιοι πεντήκοντα·
\VS{33}ἐκ τῶν υἱῶν Ἠλὰμ, Ἰεσίαν Γοθολίου, καὶ μετʼ αὐτοῦ ἄνδρες ἑβδομήκοντα·
\VS{34}ἐκ τῶν υἱῶν Σαφατίου, Ζαραΐας Μιχαήλου, καὶ μετʼ αὐτοῦ ἄνδρες ἑβδομήκοντα·
\VS{35}ἐκ τῶν υἱῶν Ἰωὰβ, Ἀβαδίας Ἰεζήλου, καὶ μετʼ αὐτοῦ ἄνδρες διακόσιοι δεκαδύο·
\VS{36}Ἐκ τῶν υἱῶν Βανίας, Σαλιμὼθ Ἰωσαφίου, καὶ μετʼ αὐτοῦ ἄνδρες ἑξήκοντα καὶ ἑκατόν·
\VS{37}ἐκ τῶν υἱῶν Βαβὶ, Ζαχαρίας Βηβαῒ, καὶ μετʼ αὐτοῦ ἄνδρες εἰκοσιοκτώ·
\VS{38}ἐκ τῶν υἱῶν Ἀστὰθ, Ἰωάννης Ἀκατὰν, καὶ μετʼ αὐτοῦ ἄνδρες ἑκατὸν δέκα·
\VS{39}ἐκ τῶν υἱῶν Ἀδωνικὰμ, οἱ ἔσχατοι· καὶ ταῦτα τὰ ὀνόματα αὐτῶν· Ἐλιφαλὰ τοῦ Γεουὴλ, καὶ Σαμαίας, καὶ μετʼ αὐτῶν ἄνδρες ἑβδομήκοντα·
\VS{40}ἐκ τῶν υἱῶν Βαγὼ, Οὐθὶ ὁ τοῦ Ἱσταλκούρου, καὶ μετʼ αὐτοῦ ἄνδρες ἑβδομήκοντα.
\par }{\PP \VS{41}Καὶ συνήγαγον αὐτοὺς ἐπὶ τὸν λεγόμενον Θερὰν ποταμὸν, καὶ παρενεβάλομεν ἡμέρας τρεῖς αὐτόθι, καὶ κατέμαθον αὐτούς.
\VS{42}Καὶ ἐκ τῶν ἱερέων καὶ ἐκ τῶν Λευιτῶν οὐχ εὑρὼν ἐκεῖ,
\VS{43}ἀπέστειλα πρὸς Ἐλεάζαρον, καὶ Ἱδουῆλον, καὶ Μαιὰ, καὶ Μασμὰν, καὶ Ἀλναθὰν, καὶ Σαμαίαν, καὶ Ἰώριβον, Νάθαν,
\VS{44}Ἐννατὰν, Ζαχαρίαν, καὶ Μοσόλλαμον τοὺς ἡγουμένους καὶ ἐπιοτήμονας,
\VS{45}καὶ εἶπα αὐτοῖς ἐλθεῖν πρὸς Λοδδαῖον τὸν ἡγούμενον τὸν ἐν τῷ τόπῳ τοῦ γαζοφυλακίου,
\VS{46}ἐντειλάμενος αὐτοῖς διαλεχθῆναι Λοδδαίῳ, καὶ τοῖς ἀδελφοῖς αὐτοῦ, καὶ τοῖς ἐν τῷ τόπῳ γαζοφύλαξιν, ἀποστεῖλαι ἡμῖν τοὺν ἱερατεύσοντας ἐν τῷ οἴκῳ τοῦ Κυρίου ἡμῶν·
\par }{\PP \VS{47}Καὶ ἤγαγον ἡμῖν κατὰ τὴν κραταιὰν χεῖρα τοῦ Κυρίου ἡμῶν ἄνδρας ἐπιστήμονας τῶν υἱῶν Μοολὶ τοῦ Λευὶ τοῦ Ἰσραὴλ, Ἀσεβηβίαν, καὶ τοὺς υἱοὺς αὐτοῦ, καὶ τοὺς ἀδελφοὺς, ὄντας δέκα καὶ ὀκτώ·
\VS{48}καὶ Ἀσεβίαν, καὶ Ἄννουον, καὶ Ὠσαίαν ἀδελφὸν ἐκ τῶν υἱῶν Χανουναίου, καὶ οἱ υἱοὶ αὐτῶν εἴκοσι ἄνδρες·
\VS{49}καὶ ἐκ τῶν ἱεροδούλων ὧν ἔδωκε Δαυὶδ, καὶ οἱ ἡγούμενοι εἰς τὴν ἐργασίαν τῶν Λευιτῶν, ἱεροδούλους διακοσίους καὶ εἴκοσι· πάντων ἐσημάνθη ἡ ὀνοματογραφία.
\par }{\PP \VS{50}Καὶ ηὐξάμην ἐκεῖ νηστείαν τοῖς νεανίσκοις ἔναντι Κυρίου ἡμῶν, ζητῆσαι παρʼ αὐτοῦ εὐοδίαν ἡμῖν τε καὶ τοῖς συνοῦσιν ἡμῖν, τέκνοις ἡμῶν, καὶ κτήνεσιν.
\VS{51}Ἐνετράπην γὰρ αἰτῆσαι τὸν βασιλέα, πεζούς τε καὶ ἱππεῖς, και προπομπὴν ἕνεκεν ἀσφαλείας τῆς πρὸς τοὺς ἐναντιουμένους ἡμῖν.
\VS{52}Εἴπαμεν γὰρ τῷ βασιλεῖ, ὅτι ἡ ἰσχὺς τοῦ Κυρίου ἡμῶν ἔσται μετὰ τῶν ἐπιζητούντων αὐτὸν εἰς πᾶσαν ἐπανόρθωσιν.
\VS{53}Καὶ πάλιν ἐδεήθημεν τοῦ Κυρίου ἡμῶν πάντα ταῦτα, καὶ ἐτύχομεν εὐιλάτου.
\par }{\PP \VS{54}Καὶ ἐχώρισα τῶν φυλάρχων τῶν ἱερέων ἄνδρας δεκαδύο, καὶ Ἐσερεβίαν καὶ Σαμίαν, καὶ μετʼ αὐτῶν ἐκ τῶν ἀδελφῶν αὐτῶν ἄνδρας δώδεκα.
\VS{55}Καὶ ἔστησα αὐτοῖς τὸ ἀργύριον, καὶ τὸ χρυσίον, καὶ τὰ ἱερὰ σκεύη τοῦ οἴκου τοῦ Κυρίου ἡμῶν, ἃ ἐδωρήσατο ὁ βασιλεὺς, καὶ οἱ σύμβουλοι αὐτοὺ, καὶ οἱ μεγιστᾶνες, καὶ πᾶς Ἰσραήλ.
\VS{56}Καὶ στήσας παρέδωκα αὑτοῖς ἀργυρίου τάλαντα ἑξακόσια πεντήκοντα, καὶ σκεύη ἀργυρᾶ ταλάντων ἑκατὸν, καὶ χρυσίου τάλαντα ἑκατὸν,
\VS{57}καὶ χρυσώματα εἴκοσι, καὶ σκεύη χάλκεα ἀπὸ χρηστοῦ χαλκοῦ στίλβοντα χρυσοειδῆ σκεύη δώδεκα.
\par }{\PP \VS{58}Καὶ εἶπα αὐτοῖς καὶ ὑμεῖς ἅγιοι ἐστὲ τῷ Κυρίῳ, καὶ τὰ σκεύη τὰ ἅγια, καὶ τὸ χρυσίον, καὶ τὸ ἀργύριον, εὐχὴ τῷ Κυρίῳ, Κυρίῳ τῶν πατέρων ἡμῶν.
\VS{59}Ἀγρυπνεῖτε, καὶ φυλάσσετε ἕως τοῦ παραδοῦναι ὑμᾶς αὐτὰ τοῖς φυλάρχοις τῶν ἱερέων καὶ τῶν Λευιτῶν, καὶ τοῖς ἡγουμένοις τὼν πατριῶν τοῦ Ἰσραὴλ ἐν Ἰερουσαλὴμ, ἐν τοῖς παστοφορίοις τοῦ οἴκου τοῦ Θεοῦ ἡμῶν.
\VS{60}Καὶ οἱ παραλαβόντες οἱ ἱερεῖς καὶ οἱ Λευῖται τὸ ἀργύριον, καὶ τὸ χρυσίον, καὶ τὰ σκεύη τὰ ἐν Ἱερουσαλὴμ, εἰσήνεγκαν εἰς τὸ ἱερὸν τοῦ Κυρίου.
\par }{\PP \VS{61}Καὶ ἀναζεύξαντες ἀπὸ τοῦ ποταμοῦ Θερὰ τῇ δωδεκάτῃ τοῦ πρώτου μηνὸς, ἕως εἰσήλθομεν εἰς Ἱερουσαλὴμ κατὰ τὴν κραταιὰν χεῖρα τοῦ Κυρίου ἡμῶν τὴν ἐφʼ ἡμῖν· καὶ ἐῤῥύσατο ἡμᾶς ἀπὸ τῆς εἰσόδου ἀπὸ παντὸς ἐχθροῦ, καὶ ἦλθομεν εἰς Ἱερουσαλήμ.
\VS{62}Καὶ γενομένης αὐτόθι ἡμέρας τρίτης, τῇ ἡμέρᾳ τῇ τετάρτῃ σταθὲν τὸ ἀργύριον καὶ τὸ χρυσίον παρεδόθη ἐν τῷ οἴκῳ Κυρίου ἡμῶν Μαρμωθὶ Οὐρία ἱερεῖ.
\VS{63}Καὶ μετʼ αὐτοῦ Ἐλεάζαρ ὁ τοῦ Φινεὲς, καὶ ἦσαν μετʼ αὐτοῦ Ἰωσαβδὸς Ἰησοῦ, καὶ Μωὲθ Σαβάννου· οἱ δὲ Λευῖται, πρὸς ἀριθμὸν καὶ ὁλκὴν ἅπαντα.
\VS{64}Καὶ ἐγράφη πᾶσα ἡ ὁλκὴ αὐτῶν αὐτῇ τῇ ὥρᾳ.
\par }{\PP \VS{65}Οἱ δὲ παραγενόμενοι ἐκ τῆς αἰχμαλωσίας προσήνεγκαν θυσίας τῷ Θεῷ τοῦ Ἰσραὴλ Κυρίῳ, ταύρους δώδεκα ὑπὲρ παντὸς Ἰσραὴλ,
\VS{66}κριοὺς ἐνενηκονταὲξ, ἄρνας ἑβδομηκονταδύο, τράγους ὑπὲρ σωτηρίου δώδεκα, ἅπαντα θυσίαν τῷ Κυρίῳ.
\VS{67}Καὶ ἀπέδωκαν τὰ προστάγματα τοῦ βασιλέως τοῖς βασιλικοῖς οἰκονόμοις καὶ τοῖς ἐπάρχοις κοίλης Συρίας καὶ Φοινίκης, καὶ ἐδόξασαν τὸ ἔθνος, καὶ τὸ ἱερὸν τοῦ Κυρίου.
\par }{\PP \VS{68}Καὶ τούτων τελεσθέντων, προσήλθοσάν μοι οἱ ἡγούμενοι, λέγοντες,
\VS{69}οὐκ ἐχώρισαν τὸ ἔθνος τοῦ Ἰσραὴλ καὶ οἱ ἄρχοντες καὶ οἱ ἱερεῖς καὶ οἱ Λευῖται τὰ ἀλλογενῆ ἔθνη τῆς γῆς καὶ τὰς ἀκαθαρσίας αὐτῶν, ἀπὸ τῶν ἐθνῶν τῶν Χαναναίων, καὶ Χετταίων, καὶ Φερεζαίων, καὶ Ἰεβουσαίων, καὶ Μωαβιτῶν, καὶ Αἰγυπτίων, καὶ Ἰδουμαίων.
\VS{70}Συνῴκησαν γὰρ μετὰ τῶν θυγατέρων αὐτῶν καὶ αὐτοὶ καὶ οἱ υἱοὶ αὐτῶν, καὶ ἐπεμίγη τὸ σπέρμα τὸ ἅγιον εἰς τὰ ἀλλογενῆ ἔθνη τῆς γῆς, καὶ μετεῖχον οἱ προηγούμενοι καὶ οἱ μεγιστᾶνες τῆς ἀνομίας ταύτης ἀπὸ τῆς ἀρχῆς τοῦ πράγματος.
\par }{\PP \VS{71}Καὶ ἅμα τῷ ἀκοῦσαί με ταῦτα, διέῤῥηξα τὰ ἱμάτια καὶ τὴν ἱερὰν ἐσθῆτα, καὶ κατέτιλα τοῦ τριχώματος τῆς κεφαλῆς καὶ τοῦ πώγωνος, καὶ ἐκάθισα σύννους καὶ περίλυπος.
\VS{72}Καὶ ἐπισυνήχθησαν πρὸς μὲ ὅσοι ποτὲ ἐπεκινοῦντο ἐπὶ τῷ ῥήματι Κυρίου Θεοῦ τοῦ Ἰσραὴλ, ἐμοῦ πενθοῦντος ἐπὶ τῇ ἀνομίᾳ· καὶ ἐκαθήμην περίλυπος ἕως τῆς δειλινῆς θυσίας.
\par }{\PP \VS{73}Καὶ ἐξεγερθεὶς ἐκ τῆς νηστείας διεῤῥηγμένα ἔχων τὰ ἱμάτια καὶ τὴν ἱερὰν ἐσθῆτα, κάμψας τὰ γόνατα, καὶ ἐκτείνας τὰς χεῖρας πρὸς τὸν Κύριον·
\VS{74}ἔλεγον, Κύριε, ᾔσχυμμαι καὶ ἐντέτραμμαι κατὰ πρόσωπόν σου.
\VS{75}Αἱ γὰρ ἁμαρτίαι ἡμῶν ἐπλεόνασαν ὑπὲρ τὰς κεφαλὰς ἡμῶν, καὶ αἱ ἄγνοιαι ἡμῶν ὑπερήνεγκαν ἕως τοῦ οὐρανοῦ,
\VS{76}ἔτι ἀπὸ τῶν χρόνων τῶν πατέρων ἡμῶν, καὶ ἐσμὲν ἐν μεγάλῃ ἁμαρτίᾳ ἕως τῆς ἡμέρας ταύτης.
\VS{77}Καὶ διὰ τὰς ἁμαρτίας ἡμῶν καὶ τῶν πατέρων ἡμῶν παρεδόθημεν σὺν τοῖς ἀδελφοῖς ἡμῶν, καὶ σὺν τοῖς βασιλεῦσιν ἡμῶν, καὶ σὺν τοῖς ἱερεῦσιν ἡμῶν, τοῖς βασιλεῦσι τῆς γῆς εἰς ῥομφαίαν καὶ αἰχμαλωσίαν καὶ προνομὴν μετὰ αἰσχύνης μέχρι τῆς σήμερον ἡμέρας.
\par }{\PP \VS{78}Καὶ νῦν κατὰ πόσον τι ἐγενήθη ἡμῖν ἔλεος παρὰ τοῦ Κυρίου Κυρίου, καταλειφθῆναι ἡμῖν ῥίζαν καὶ ὄνομα ἐν τῷ τόπῳ ἁγιάσματός σου,
\VS{79}καὶ τοῦ ἀνακαλύψαι φωστῆρα ἡμῖν ἐν τῷ οἴκῳ Κυρίου τοῦ Θεοῦ ἡμῶν, δοῦναι ἡμῖν τροφὴν ἐν τῷ καιρῷ τῆς δουλείας ἡμῶν;
\VS{80}Καὶ ἐν τῷ δουλεύειν ἡμᾶς οὐκ ἐγκατελείφθημεν ὑπὸ τοῦ Κυρίου ἡμῶν, ἀλλὰ ἐποίησεν ἡμᾶς ἐν χάριτι ἐνώπιον τῶν βασιλέων Περσῶν,
\VS{81}δοῦναι ἡμῖν τροφὴν, καὶ δοξάσαι τὸ ἱερὸν τοῦ Κυρίου ἡμῶν, καὶ ἐγεῖραι τὴν ἔρημον Σιὼν, δοῦναι ἡμῖν στερέωμα ἐν τῇ Ἰουδαίᾳ καὶ Ἱερουσαλήμ.
\par }{\PP \VS{82}Καὶ νῦν τί ἐροῦμεν, Κύριε, ἔχοντες ταῦτα; παρέβημεν γὰρ τὰ προστάγματά σου, ἃ ἔδωκας ἐν χειρὶ τῶν παίδων σου τῶν προφητῶν,
\VS{83}λέγων, ὅτι ἡ γῆ, εἰς ἣν εἰσέρχεσθε κληρονομῆσαι, ἔστι γῆ μεμολυσμένη μολυσμῷ τῶν ἀλλογενῶν τῆς γῆς, καὶ τῆς ἀκαθαρσίας αὐτῶν ἐνέπλησαν αὐτήν.
\VS{84}Καὶ νῦν τὰς θυγατέρας ὑμῶν μὴ συνοικήσητε τοῖς υἱοῖς αὐτῶν, καὶ τὰς θυγατέρας αὐτῶν μὴ λάβητε τοῖς υἱοῖς ὑμῶν,
\VS{85}καὶ οὐ ζητήσετε εἰρηνεῦσαι τὰ πρὸς αὐτοὺς τὸν ἅπαντα χρόνον, ἵνα ἰσχύσαντες φάγητε τὰ ἀγαθὰ τῆς γῆς, καὶ κατακληρονομήσητε τοῖς τέκνοις ὑμῶν ἕως αἰῶνος.
\par }{\PP \VS{86}Καὶ τὰ συμβαίνοντα πάντα ἡμῖν γίνεται διὰ τὰ ἔργα ἡμῶν τὰ πονηρὰ, καὶ τὰς μεγάλας ἁμαρτίας ἡμῶν·
\VS{87}σὺ γὰρ Κύριε ὁ κουφίσας τὰς ἁμαρτίας ἡμῶν, ἔδωκας ἡμῖν τοιαύτην ῥίζαν· πάλιν ἀνεκάμψαμεν παραβῆναι τὸν νόμον σου εἰς τὸ ἐπιμιγῆναι τῇ ἀκαθαρσίᾳ τῶν ἐθνῶν τῆς γῆς.
\VS{88}Οὐχὶ ὠργίσθης ἡμῖν ἀπολέσαι ἡμᾶς, ἕως τοῦ μὴ καταλιπεῖν ῥίζαν καὶ σπέρμα καὶ ὄνομα ἡμῶν;
\par }{\PP \VS{89}Κύριε τοῦ Ἰσραὴλ, ἀληθινὸς εἶ· κατελείφθημεν γὰρ ῥίζα ἐν τῇ σήμερον.
\VS{90}Ἰδοὺ νῦν ἐσμὲν ἐνώπιόν σου ἐν ταῖς ἀνομίαις ἡμῶν· οὐ γὰρ ἐστι στῆναι ἔτι ἔμπροσθέν σου ἐπὶ τούτοις.
\VS{91}Καὶ ὅτε προσευχόμενος Ἔσδρας ἀνθωμολογεῖτο κλαίων χαμαιπετὴς ἔμπροσθεν τοῦ ἱεροῦ, ἐπισυνήχθησαν πρὸς αὐτὸν ἀπὸ Ἱερουσαλὴμ ὄχλος πολὺς σφόδρα, ἄνδρες, καὶ γυναῖκες, καὶ νεανίαι· κλαυθμὸς γὰρ ἦν μέγας ἐν τῷ πλήθει.
\par }{\PP \VS{92}Καὶ φωνήσας Ἰεχονίας Ἰεήλου τῶν υἱῶν Ἰσραὴλ, εἶπεν, Ἔσδρα, ἡμεῖς ἡμάρτομεν εἰς τὸν Κύριον· συνῳκίσαμεν γυναῖκας ἀλλογενεῖς ἐκ τῶν ἐθνῶν τῆς γῆς· καὶ νῦν ἐστιν ἐπάνω πᾶς Ἰσραήλ.
\VS{93}Ἐν τούτῳ γινέσθω ἡμῖν ὁρκωμοσία πρὸς τὸν Κύριον, ἐκβαλεῖν πάσας τὰς γυναῖκας ἡμῶν τὰς ἐκ τῶν ἀλλογενῶν σὺν τοῖς τέκνοις αὐτῶν,
\VS{94}ὡς ἐκρίθη σοι, καὶ ὅσοι πειθαρχοῦσι τοῦ νόμου Κυρίου.
\VS{95}Ἀναστὰς ἐπιτέλει· πρὸς σὲ γὰρ τὸ πρᾶγμα, καὶ ἡμεῖς μετὰ σοῦ ἰσχὺν ποιεῖν.
\VS{96}Καὶ ἀναστὰς Ἔσδρας ὥρκισε τοὺς φυλάρχους τῶν ἱερέων καὶ Λευιτῶν παντὸς τοῦ Ἰσραὴλ, ποιῆσαι κατὰ ταῦτα· καὶ ὤμοσαν.

\par }\Chap{9}{\PP \VerseOne{1}Καὶ ἀναστὰς Ἔσρας ἀπὸ τῆς αὐλῆς τοῦ ἱεροῦ, ἐπορεύθη εἰς τὸ παστοφόριον Ἰωνὰν τοῦ Ἐλιασίβου.
\VS{2}Καὶ αὐλισθεὶς ἐκεῖ, ἄρτου οὐκ ἐγεύσατο οὐδὲ ὕδωρ ἔπιε, πενθῶν ἐπὶ τῶν ἀνομιῶν τῶν μεγάλων τοῦ πλήθους.
\VS{3}Καὶ ἐγένετο κήρυγμα ἐν ὅλῃ τῇ Ἰουδαίᾳ καὶ Ἱερουσαλὴμ πᾶσι τοῖς ἐκ τῆς αἰχμαλωσίας, συναχθῆναι εἰς Ἱερουσαλήμ.
\VS{4}Καὶ ὅσοι ἂν μὴ ἀπαντήσωσιν ἐν δυσὶν ἢ τρισὶν ἡμέραις, κατὰ τὸ κρίμα τῶν προκαθημένων πρεσβυτέρων, ἀνιερωθήσονται τὰ κτήνη αὐτῶν, καὶ αὐτὸς ἀλλοτριωθήσεται ἀπὸ τοῦ πλήθους τῆς αἰχμαλωσίας.
\par }{\PP \VS{5}Καὶ ἐπισυνήχθησαν πάντες οἱ ἐκ τῆς φυλῆς Ἰούδα καὶ Βενιαμεὶν ἐν τρισὶν ἡμέραις εἰς Ἱερουσαλήμ· οὗτος ὁ μὴν ἔννατος, τῇ εἰκάδι τοῦ μηνός.
\VS{6}Καὶ συνεκάθισαν πᾶν τὸ πλῆθος ἐν τῷ εὐρυχώρῳ τοῦ ἱεροῦ, τρέμοντες διὰ τὸν ἐνεστῶτα χειμῶνα.
\par }{\PP \VS{7}Καὶ ἀναστὰς Ἔσδρας εἶπεν αὐτοῖς, ὑμεῖς ἠνομήσατε καὶ συνῳκίσατε γυναιξὶν ἀλλογενέσι, τοῦ προσθεῖναι ἁμαρτίαν τῷ Ἰσραήλ.
\VS{8}Καὶ νῦν δότε ὁμολογίαν δόξαν τῷ Κυρίῳ Θεῷ τῶν πατέρων ἡμῶν,
\VS{9}καὶ ποιήσατε τὸ θέλημα αὐτοῦ, καὶ χωρίσθητε ἀπὸ τῶν ἐθνῶν τῆς γῆς, καὶ ἀπὸ τῶν γυναικῶν τῶν ἀλλογενῶν.
\par }{\PP \VS{10}Καὶ ἐφώνησεν ἅπαν τὸ πλῆθος, καὶ εἶπον μεγάλῃ τῇ φωνῇ, οὕτως ὡς εἴρηκας, ποιήσομεν.
\VS{11}Ἀλλὰ τὸ πλῆθος πολὺ καὶ ὥρα χειμερινὴ, καὶ οὐκ ἰσχύομεν στῆναι αἴθριοι· καὶ τὸ ἔργον οὐκ ἔστιν ἡμῖν ἡμέρας μιᾶς οὐδὲ δύο, ἐπὶ πλεῖον γὰρ ἡμάρτομεν ἐν τούτοις.
\VS{12}Στήτωσαν δὲ οἱ προηγούμενοι τοῦ πλήθους, καὶ πάντες οἱ ἐκ τῶν κατοικιῶν ἡμῶν ὅσοι ἔχουσι γυναῖκας ἀλλογενεῖς, παραγενηθήτωσαν λαβόντες χρόνον,
\VS{13}ἑκάστου δὲ τόπου τοὺς πρεσβυτέρους καὶ τοὺς κριτὰς, ἕως τοῦ λῦσαι τὴν ὀργὴν Κυρίου ἀφʼ ἡμῶν τοῦ πράγματος τούτου.
\par }{\PP \VS{14}Ἰωνάθας Ἀζαήλου, καὶ Ἐζεκίας Θεωκανοῦ ἐπεδέξαντο κατὰ ταῦτα· καὶ Μοσόλλαμος, καὶ Λευὶς, καὶ Σαββαταῖος συνεβράβευσαν αὐτοῖς.
\VS{15}Καὶ ἐποίησαν κατὰ πάντα ταῦτα οἱ ἐκ τῆς αἰχμαλωσίας·
\VS{16}καὶ ἐπελέξατο αὐτῷ Ἔσδρας ὁ ἱερεὺς ἄνδρας ἡγουμένους τῶν πατριῶν αὐτῶν πάντας κατʼ ὄνομα, καὶ συνεκλείσθησαν τῇ νουμηνίᾳ τοῦ μηνὸς τοῦ δεκάτου, ἐτάσαι τὸ πρᾶγμα.
\VS{17}Καὶ ἤχθη ἐπὶ πέρας τὰ κατὰ τοὺς ἄνδρας τοὺς ἐπισυνέχοντας γυναῖκας ἀλλογενεῖς, ἕως τῆς νουμηνίας τοῦ πρώτου μηνός.
\par }{\PP \VS{18}Καὶ εὑρέθησαν τῶν ἱερέων οἱ ἐπισυναχθέντες ἀλλογενεῖς γυναῖκας ἔχοντες,
\VS{19}ἐκ τῶν υἱῶν Ἰησοῦ τοῦ Ἰωσεδὲκ, καὶ τῶν ἀδελφῶν αὐτοῦ, Μαθήλας, καὶ Ἐλεάζαρος, καὶ Ἰώριβος, καὶ Ἰωαδάνος.
\VS{20}Καὶ ἐπέβαλον τὰς χεῖρας ἐκβαλεῖν τὰς γυναῖκας αὐτῶν· καὶ εἰς ἐξιλασμὸν κριοὺς ὑπὲρ τῆς ἀγνοίας αὐτῶν.
\par }{\PP \VS{21}Καὶ ἐκ τῶν υἱῶν Ἐμμὴρ, Ἀνανίας, καὶ Ζαβδαίος, καὶ Μάνης, καὶ Σαμαῖος, καὶ Ἱερεὴλ, καὶ Ἀζαρίας·
\VS{22}καὶ ἐκ τῶν υἱῶν Φαισοὺρ, Ἐλιωναῒς, Μασσίας, Ἰσμαῆλος, καὶ Ναθαναῆλος, καὶ Ὠκόδηλος, καὶ Σαλόας.
\par }{\PP \VS{23}Καὶ ἐκ τῶν Λευιτῶν, Ἰωζαβάδος, καὶ Σεμεῒς, καὶ Κώϊος (οὗτός ἐστι Καλιτὰς), καὶ Παθαῖος, καὶ Ἰούδας, καὶ Ἰωνάς.
\VS{24}Ἐκ τῶν ἱεροψαλτῶν, Ἐλιάσαβος, Βακχοῦρος.
\VS{25}Ἐκ τῶν θυρωρῶν, Σαλοῦμος, καὶ Τολβάνης.
\par }{\PP \VS{26}Ἐκ τοῦ Ἰσραὴλ ἐκ τῶν υἱῶν Φόρος, Ἱερμὰς, καὶ Ἰεζίας, καὶ Μελχίας, καὶ Μαῆλος, καὶ Ἐλεάζαρος, καὶ Ἀσεβίας, καὶ Βαναίας.
\VS{27}Ἐκ τῶν υἱῶν Ἠλὰ, Ματθανίας, Ζαχαρίας, καὶ Ἰεζριῆλος, καὶ Ἰωαβδίος, καὶ Ἱερεμὼθ, καὶ Ἀϊδίας.
\VS{28}Καὶ ἐκ τῶν υἱῶν Ζαμὼθ, Ἐλιαδὰς, Ἐλιάσιμος, Ὀθονίας, Ἰαριμὼθ, καὶ Σάβαθος, καὶ Ζεραλίας.
\VS{29}Καὶ ἐκ τῶν υἱῶν Βηβαῒ, Ἰωάννης, καὶ Ἀνανίας, καὶ Ἰωζάβδος, καὶ Ἀμαθίας.
\VS{30}Ἐκ τῶν υἱῶν Μανὶ, Ὠλαμὸς, Μαμοῦχος, Ἰεδαῖος, Ἰασούβος, καὶ Ἰασαῆλος, καὶ Ἱερεμώθ.
\VS{31}Καὶ ἐξ υἱῶν Ἀδδὶ, Νάαθος, καὶ Μοοσίας, Λακκοῦνος, καὶ Ναΐδος, Ματθανίας, καὶ Σεσθὴλ, καὶ Βαλνούος, καὶ Μανασσίας.
\VS{32}Καὶ ἐκ τῶν υἱῶν Ἀνὰν, Ἐλιωνὰς, καὶ Ἀσαΐας, καὶ Μελχίας, καὶ Σαββαῖος, καὶ Σίμων Χοσαμαίος.
\VS{33}Καὶ ἐκ τῶν υἱῶν Ἀσὸμ, Ἀλταναῖος, καὶ Ματταθίας, καὶ Σαβανναῖος, καὶ Ἐλιφαλὰτ, καὶ Μανασσῆς, καὶ Σεμεΐ.
\VS{34}Καὶ ἐκ τῶν υἱῶν Βαανὶ, Ἱερεμίας, Μομδίος, Ἰσμαῆρος, Ἰουὴλ, Μαβδαῒ, καὶ Πεδίας, καὶ Ἄνως, Ῥαβασίων, καὶ Ἐνάσιβος, καὶ Μαμνιτάναιμος, Ἐλίασις, Βαννοὺς, Ἐλιαλὶ, Σομεῒς, Σελεμίας, Ναθανίας· καὶ ἐκ τῶν υἱῶν Ἐζωρὰ, Σεσὶς, Ἐσρὶλ, Ἀζαῆλος, Σαματὸς, Ζαμβρὶ, Ἰώσηφος.
\VS{35}Καὶ ἐκ τῶν υἱῶν Ἐθμὰ, Μαζιτίας, Ζαβαδαίας, Ἠδαῒς, Ἰουὴλ, Βαναίας.
\par }{\PP \VS{36}Πάντες οὗτοι συνῴκισαν γυναῖκας ἀλλογενεῖς, καὶ ἀπέλυσαν αὐτὰς σὺν τέκνοις.
\par }{\PP \VS{37}Καὶ κατῴκησαν οἱ ἱερεῖς, καὶ οἱ Λευῖται, καὶ οἱ ἐκ τοῦ Ἰσραὴλ ἐν Ἱερουσαλὴμ καὶ ἐν τῇ χώρᾳ τῇ νουμηνίᾳ τοῦ μηνὸς τοῦ ἑβδόμου, καὶ οἱ υἱοὶ Ἰσραὴλ ἐν ταῖς κατοικίαις αὐτων.
\par }{\PP \VS{38}Καὶ συνήχθη πᾶν τὸ πλῆθος ὁμοθυμαδὸν ἐπὶ τὸ εὐρύχωρον τοῦ πρὸς ἀνατολὰς τοῦ ἱεροῦ πυλῶνος,
\VS{39}καὶ εἶπεν Ἔσδρᾳ τῷ ἱερεῖ καὶ ἀναγνώστῃ, κόμισαι τὸν νόμον Μωυσῆ, τὸν παραδοθέντα ὑπὸ Κυρίου Θεοῦ Ἰσραήλ.
\VS{40}Καὶ ἐκόμισεν Ἔσδρας ὁ ἀρχιερεὺς τὸν νόμον παντὶ τῷ πλήθει ἀπὸ ἀνθρώπου ἕως γυναικὸς, καὶ πᾶσι τοῖς ἱερεῦσιν, ἀκοῦσαι τοῦ νόμου νουμηνίᾳ τοῦ ἑβδόμου μηνός.
\VS{41}Καὶ ἀνεγίνωσκεν ἐν τῷ πρὸ τοῦ ἱεροῦ πυλῶνος εὐρυχώρῳ, ἐξ ὄρθρου ἕως μέσης ἡμέρας, ἐνώπιον ἀνδρῶν τε καὶ γυναικῶν· καὶ ἐπέδωκαν πᾶν τὸ πλῆθος τὸν νοῦν εἰς τὸν νόμον.
\par }{\PP \VS{42}Καὶ ἔστη Ἔσδρας ὁ ἱερεὺς καὶ ἀναγνώστης τοῦ νόμου ἐπὶ τοῦ ξυλίνου βήματος τοῦ κατασκευασθέντος.
\VS{43}Καὶ ἔστησαν παρʼ αὐτῷ Ματταθίας, Σαμμοὺς, Ἀνανίας, Ἀζαρίας, Οὐρίας, Ἐζεκίας, Βαάλσαμος, ἐκ δεξιῶν·
\VS{44}καὶ ἐξ εὐωνύμων Φαλδαῖος, καὶ Μισαὴλ, Μελχίας, Λωθάσουβος, Ναβαρίας, Ζαχαρίας.
\par }{\PP \VS{45}Καὶ ἀναλαβὼν Ἔσδρας τὸ βιβλίον ἐνώπιον τοῦ πλήθους, προεκάθητο ἐπιδόξως ἐνώπιον πάντων.
\VS{46}Καὶ ἐν τῷ λῦσαι τὸν νόμον, πάντες ὀρθοὶ ἔστησαν· καὶ εὐλόγησεν Ἔσδρας τῷ Κυρίῳ Θεῷ ὑψίστῳ Θεῷ σαβαὼθ παντοκράτορι.
\VS{47}Καὶ ἐπεφώνησε πᾶν τὸ πλῆθος, ἀμήν· καὶ ἄραντες ἄνω τὰς χεῖρας, προσπεσόντες ἐπὶ τὴν γῆν, προσεκύνησαν τῷ Κυρίῳ.
\par }{\PP \VS{48}Ἰησοῦς, καὶ Ἀννιοὺθ, καὶ Σαραβίας, καὶ Ἰαδινὸς, καὶ Ἰάκουβος, Σαβαταῖος, Αὐταίας, Μαιάννας, καὶ Καλίτας, Ἀζαρίας, καὶ Ἰώζαβδος, καὶ Ἀνανίας, Φαλίας, οἱ Λευῖται, ἐδίδασκον τὸν νόμον τοῦ Κυρίου, καὶ πρὸς τὸ πλῆθος ἀνεγίνωσκον τὸν νόμον τοῦ Κυρίου, ἐμφυσιοῦντες ἅμα τὴν ἀνάγνωσιν.
\par }{\PP \VS{49}Καὶ εἶπεν Ἀτθαράτης Ἔσδρᾳ τῷ ἀρχιερεῖ καὶ ἀναγνώστῃ, καὶ τοῖς Λευίταις τοῖς διδάσκουσι τὸ πλῆθος ἐπὶ πάντας,
\VS{50}ἡ ἡμέρα αὕτη ἐστὶν ἁγία τῷ Κυρίῳ· καὶ πάντες ἔκλαιον ἐν τῷ ἀκοῦσαι τοῦ νόμου·
\VS{51}βαδίσαντες οὖν φάγετε λιπάσματα, καὶ πίετε γλυκάσματα, καὶ ἀποστείλατε ἀποστολὰς τοῖς μὴ ἔχουσιν·
\VS{52}ἁγία γὰρ ἡ ἡμέρα τῷ Κυρίῳ· καὶ μὴ λυπεῖσθε, ὁ γὰρ Κύριος δοξάσει ὑμᾶς.
\par }{\PP \VS{53}Καὶ οἱ Λευῖται ἐκέλευον παντὶ τῷ δήμῳ, λέγοντες ἡ ἡμέρα αὕτη ἁγία, μὴ λυπεῖσθε.
\VS{54}Καὶ ᾤχοντο πάντες φαγεῖν καὶ πιεῖν καὶ εὐφραίνεσθαι, καὶ δοῦναι ἀποστολὰς τοῖς μὴ ἔχουσι, καὶ εὐφρανθῆναι μεγάλως,
\VS{55}ὅτι γὰρ ἐνεφυσιώθησαν ἐν τοῖς ῥήμασιν οἷς ἐδιδάχθησαν, καὶ ἐπισυνήχθησαν.
\par }