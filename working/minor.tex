% !TeX program = lualatex
% Layout
\documentclass[9pt,openany]{book}
\usepackage[paperwidth=6.5in,paperheight=9.5in,top=0.3in,bottom=0.7in,left=0.4in,right=0.4in,bindingoffset=0.2in,includehead,headsep=7pt]{geometry}
\usepackage{setspace} % for line spacing
\usepackage{microtype} % helps with formatting
\usepackage{emptypage}
\usepackage[greek]{babel}
\usepackage[all]{nowidow}
\usepackage{ragged2e} % Nicer ragged edges. Use \Center to get nice balanced lines
\usepackage{bookmark} %add PDF bookmarks for each \chapter
\usepackage{hyperref} % Make the TOC clickable
\usepackage{fontspec}
\setmainfont{Gentium Plus}

% % Use a different font for Greek (with Latin fallback)
% % https://tex.stackexchange.com/a/619560/319804
% \directlua{luaotfload.add_fallback("myfallback",{"Gentium Plus:mode=harf;",})}
% \setmainfont{GFS Porson}[RawFeature={fallback=myfallback}]

\flushbottom
\frenchspacing % stops extra space after a period or colon
\hyphenpenalty=2000
\emergencystretch=1pt
\setlength{\parindent}{.2in} % Set paragraph indentation

\usepackage{titlesec}
\renewcommand{\thechapter}{} % remove numbers from chapter
% Show chapter titles with a line after them
\titleformat{\chapter}[display]
{\filcenter\normalfont\Huge}{}{0pt}{}
[\vspace{.5ex}\rule{1.5in}{0.4pt}]

% Add the multicol package for two-column layout
\usepackage{multicol}
\setlength{\columnsep}{.4cm} % Set the columnn width

% Adjust spacing and format of TOC 
\usepackage{tocloft}
\renewcommand{\cftchapfont}{\normalfont} % Set chapter entries to normal font
\renewcommand{\cftchappagefont}{\normalfont} % Set chapter page numbers to normal font
\renewcommand{\cftdotsep}{2} % change the default spacing of dots
\renewcommand{\cftchapleader}{\cftdotfill{\cftdotsep}} % Add leader dots for chapters
\setlength{\cftbeforechapskip}{1pt} % Adjust the space before each TOC chapter entry

% Adjust the width of the TOC
\renewcommand{\cftchapafterpnum}{\hspace*{9em}} % increase right margin
\setlength{\cftchapindent}{9em} % increase left margin
\setlength{\cftchapnumwidth}{0em} % remove space reserved for chapter numbers

% Change the title for the TOC
% https://tex.stackexchange.com/a/28518/319804
\addto\captionsgreek{%
  \renewcommand{\contentsname}%
    {Order of Books}%
}

% Center the TOC title
\renewcommand{\cfttoctitlefont}{\hfill\huge}
\renewcommand{\cftaftertoctitle}{\hfill\hfill}

\newcommand{\psalmheading}[1]{%
    \begin{Center}%
        {#1}%
    \end{Center}%
}

% Set up chapter/verse references in headers
% https://tex.stackexchange.com/a/657575/319804
\usepackage{fancyhdr}
\pagestyle{fancy}
\newcounter{mychapter}
\newcounter{verse}
\def\book{}
\fancyhead{}
\fancyhead[LE]{\rightmark}
\fancyhead[OR]{\leftmark}
\fancyfoot{}
\fancyfoot[C]{\thepage}
\renewcommand{\headrulewidth}{0pt}

\newcommand{\ch}[1]{%
  \setcounter{mychapter}{#1}%
  \markboth{\book\ \themychapter:1}{\book\ \themychapter:1}%
  \textbf{#1}%
}
\newcommand{\vs}[1]{%
  \textsuperscript{#1}%
  \markboth{\book\ \themychapter:#1}{\book\ \themychapter:#1}%
}

\let\biblebook\chapter % rename \chapter so it's not confusing

\newenvironment{psalmhead}[1]{%
    \par\addvspace{\baselineskip}%
    \noindent%
    \begin{minipage}[b]{\columnwidth}%
        \begin{Center}%
            {#1}%
        \end{Center}%
}{%
    \end{minipage}%
}

\title{Η ΠΑΛΑΙΑ ΔΙΑΘΗΚΗ}
\author{Ἡ μετάφρασις τῶν Ἑβδομήκοντα}
\date{}

\begin{document}
\begin{spacing}{1.1}
\maketitle

\tableofcontents

\begin{multicols}{2}


\end{multicols}
\chapter{ΩΣΗΕ}
\begin{multicols}{2}

\ch{1}
ΛΟΓΟΣ Κυρίου, ὃς ἐγενήθη πρὸς Ὠσηὲ τὸν τοῦ Βεηρεὶ, ἐν ἡμέραις Ὀζίου, καὶ Ἰωάθαμ, καὶ Ἄχαζ, καὶ Ἐζεκίου βασιλέων Ἰούδα, καὶ ἐν ἡμέραις Ἱεροβοὰμ υἱοῦ Ἰωὰς βασιλέως Ἰσραήλ.

\vs{2}Ἀρχὴ λόγου Κυρίου ἐν Ὠσῆέ· καὶ εἶπε Κύριος πρὸς Ὠσηὲ, βάδιζε, λάβε σεαυτῷ γυναῖκα πορνείας, καὶ τέκνα πορνείας, διότι ἐκπορνεύουσα ἐκπορνεύσει ἡ γῆ ἀπὸ ὄπισθεν τοῦ Κυρίου.

\vs{3}Καὶ ἐπορεύθη, καὶ ἔλαβε τὴν Γόμερ, θυγατέρα Δεβηλαΐμ· καὶ συνέλαβε καὶ ἔτεκεν αὐτῷ υἱόν.
\vs{4}Καὶ εἶπε Κύριος πρὸς αὐτὸν, κάλεσον τὸ ὄνομα αὐτοῦ Ἰεζραὲλ, διότι ἔτι μικρὸν, καὶ ἐκδικήσω τὸ αἷμα τοῦ Ἰεζραὲλ ἐπὶ τὸν οἶκον Ἰούδα, καὶ καταπαύσω βασιλείαν οἴκου Ἰσραήλ.
\vs{5}Καὶ ἔσται, ἐν τῇ ἡμέρᾳ ἐκείνῃ, συντρίψω τὸ τόξον τοῦ Ἰσραὴλ ἐν κοιλάδι τοῦ Ἰεζραέλ.

\vs{6}Καὶ συνέλαβεν ἔτι, καὶ ἔτεκε θυγατέρα· καὶ εἶπεν αὐτῷ, κάλεσον τὸ ὄνομα αὐτῆς, οὐκ ἠλεημένη· διότι οὐ μὴ προσθήσω ἔτι ἐλεῆσαι τὸν οἶκον Ἰσραὴλ, ἀλλʼ ἢ ἀντιτασσόμενος ἀντιτάξομαι αὐτοῖς.
\vs{7}Τοὺς δὲ υἱοὺς Ἰούδα ἐλεήσω, καὶ σώσω αὐτοὺς ἐν Κυρίῳ Θεῷ αὐτῶν, καὶ οὐ σώσω αὐτοὺς ἐν τόξῳ, οὐδὲ ἐν ῥομφαίᾳ, οὐδὲ ἐν πολέμῳ, οὐδὲ ἐν ἵπποις, οὐδὲ ἐν ἱππεῦσι.
\vs{8}Καὶ ἀπεγαλάκτισε τὴν οὐκ ἠλεημένην· καὶ συνέλαβεν ἔτι, καὶ ἔτεκεν υἱόν.
\vs{9}Καὶ εἶπε, κάλεσον τὸ ὄνομα αὐτοῦ, οὐ λαός μου· διότι ὑμεῖς οὐ λαός μου, καὶ ἐγὼ οὐκ εἰμὶ ὑμῶν.

\ch{2}Καὶ ἦν ὁ ἀριθμὸς τῶν υἱῶν Ἰσραὴλ, ὡς ἡ ἄμμος τῆς θαλάσσης, ἣ οὐκ ἐκμετρηθήσεται, οὐδὲ ἐξαριθμηθήσεται· καὶ ἔσται, ἐν τῷ τόπῳ, οὗ ἐῤῥέθη αὐτοῖς, οὐ λαός μου ὑμεῖς, κληθήσονται καὶ αὐτοὶ υἱοὶ Θεοῦ ζῶντος.
\vs{2}Καὶ συναχθήσονται υἱοὶ Ἰούδα, καὶ οἱ υἱοὶ Ἰσραὴλ ἐπιτοαυτὸ, καὶ θήσονται ἑαυτοῖς ἀρχὴν μίαν, καὶ ἀναβήσονται ἐκ τῆς γῆς, ὅτι μεγάλη ἡ ἡμέρα τοῦ Ἰεζραέλ.

\vs{3}Εἴπατε τῷ ἀδελφῷ ὑμῶν, λαός μου, καὶ τῇ ἀδελφῇ ὑμῶν, ἠλεημένη.
\vs{4}Κρίθητε πρὸς τὴν μητέρα ὑμῶν, κρίθητε, ὅτι αὕτη οὐ γυνή μου, καὶ ἐγὼ οὐκ ἀνὴρ αὐτῆς· καὶ ἐξαρῶ τὴν πορνείαν αὐτῆς ἐκ προσώπου μου, καὶ τὴν μοιχείαν αὐτῆς ἐκ μέσου μαστῶν αὐτῆς,
\vs{5}ὅπως ἂν ἐκδύσω αὐτὴν γυμνὴν, καὶ ἀποκαταστήσω αὐτὴν καθὼς ἡμέρα γενέσεως αὐτῆς· καὶ θήσω αὐτὴν ἔρημον, καὶ τάξω αὐτὴν ὡς γῆν ἄνυδρον, καὶ ἀποκτενῶ αὐτὴν ἐν δίψει.
\vs{6}Καὶ τὰ τέκνα αὐτῆς οὐ μὴ ἐλεήσω, ὅτι τέκνα πορνείας ἐστίν.
\vs{7}Ὅτι ἐξεπόρνευσεν ἡ μήτηρ αὐτῶν, κατῄσχυνεν ἡ τεκοῦσα αὐτά· ὅτι εἶπε, πορεύσομαι ὀπίσω τῶν ἐραστῶν μου, τῶν διδόντων μοι τοὺς ἄρτους μου, καὶ τὸ ὕδωρ μου, καὶ τὰ ἱμάτιά μου, καὶ τὰ ὀθόνιά μου, τὸ ἔλαιόν μου, καὶ πάντα ὅσα μοι καθήκει.

\vs{8}Διὰ τοῦτο ἰδοὺ ἐγὼ φράσσω τὴν ὁδὸν αὐτῆς ἐν σκόλοψι, καὶ ἀνοικοδομήσω τὰς ὁδοὺς, καὶ τὴν τρίβον αὐτῆς οὐ μὴ εὕρῃ·
\vs{9}Καὶ καταδιώξεται τοὺς ἐραστὰς αὐτῆς, καὶ οὐ μὴ καταλάβῃ αὐτούς· καὶ ζητήσει αὐτοὺς, καὶ οὐ μὴ εὕρῃ αὐτούς· καὶ ἐρεῖ, πορεύσομαι, καὶ ἐπιστρέψω πρὸς τὸν ἄνδρα μου τὸν πρότερον, ὅτι καλῶς μοι ἦν τότε, ἢ νῦν.

\vs{10}Καὶ αὕτη οὐκ ἔγνω ὅτι ἐγὼ ἔδωκα αὐτῇ τὸν σῖτον, καὶ τὸν οἶνον, καὶ τὸ ἔλαιον, καὶ ἀργύριον ἐπλήθυνα αὐτῇ· αὕτη δὲ ἀργυρᾶ καὶ χρυσᾶ ἐποίησεν τῇ Βάαλ.
\vs{11}Διὰ τοῦτο ἐπιστρέψω, καὶ κομιοῦμαι τὸν σῖτόν μου καθʼ ὥραν αὐτοῦ, καὶ τὸν οἶνόν μου ἐν καιρῷ αὐτοῦ· καὶ ἀφελοῦμαι τὰ ἱμάτιά μου, καὶ τὰ ὀθόνιά μου, τοῦ μὴ καλύπτειν τὴν ἀσχημοσύνην αὐτῆς·
\vs{12}Καὶ νῦν ἀποκαλύψω τὴν ἀκαθαρσίαν αὐτῆς ἐνώπιον τῶν ἐραστῶν αὐτῆς, καὶ οὐθεὶς οὐ μὴ ἐξέληται αὐτὴν ἐκ χειρός μου.
\vs{13}Καὶ ἀποστρέψω πάσας τὰς εὐφροσύνας αὐτῆς, ἑορτὰς αὐτῆς, καὶ τὰς νουμηνίας αὐτῆς, καὶ τὰ σάββατα αὐτῆς, καὶ πάσας τὰς πανηγύρεις αὐτῆς.
\vs{14}Καὶ ἀφανιῶ ἄμπελον αὐτῆς, καὶ τὰς συκᾶς αὐτῆς, ὅσα εἶπε, μισθώματά μου ταῦτά ἐστιν ἃ ἔδωκάν μοι οἱ ἐρασταί μου· καὶ θήσομαι αὐτὰ εἰς μαρτύριον, καὶ καταφάγεται αὐτὰ τὰ θηρία τοῦ ἀγροῦ, καὶ τὰ πετεινὰ τοῦ οὐρανου, καὶ τὰ ἑρπετὰ τῆς γῆς.
\vs{15}Καὶ ἐκδικήσω ἐπʼ αὐτὴν τὰς ἡμέρας τῶν Βααλεὶμ, ἐν αἷς ἐπέθυεν αὐτοῖς· καὶ περιετίθετο τὰ ἐνώτια αὐτῆς, καὶ τὰ καθόρμια αὐτῆς, καὶ ἐπορεύετο ὀπίσω τῶν ἐραστῶν αὐτῆς, ἐμοῦ δὲ ἐπελάθετο, λέγει Κύριος.

\vs{16}Διὰ τοῦτο ἰδοὺ ἐγὼ πλανῶ αὐτὴν, καὶ τάξω αὐτὴν ὡς ἔρημον, καὶ λαλήσω ἐπὶ τὴν καρδίαν αὐτῆς,
\vs{17}καὶ δώσω αὐτῇ τὰ κτήματα αὐτῆς ἐκεῖθεν, καὶ τὴν κοιλάδα Ἀχὼρ διανοῖξαι σύνεσιν αὐτῆς· καὶ ταπεινωθήσεται ἐκεῖ κατὰ τὰς ἡμέρας νηπιότητος αὐτῆς, καὶ κατὰ τὰς ἡμέρας ἀναβάσεως αὐτῆς ἐκ γῆς Αἰγύπτου.

\vs{18}Καὶ ἔσται ἐν τῇ ἡμέρᾳ ἐκεῖνῃ, λέγει Κύριος, καλέσει με ὁ ἀνήρ μου, καὶ οὐ καλέσει με ἔτι Βααλείμ.
\vs{19}Καὶ ἐξαρῶ τὰ ὀνόματα τῶν Βααλεὶμ ἐκ στόματος αὐτῆς, καὶ οὐ μὴ μνησθῶσιν οὐκέτι τὰ ὀνόματα αὐτῶν.
\vs{20}Καὶ διαθήσομαι αὐτοῖς διαθήκην ἐν τῇ ἡμέρᾳ ἐκείνῃ μετὰ τῶν θηρίων τοῦ ἀγροῦ, καὶ μετὰ τῶν πετεινῶν τοῦ οὐρανοῦ, καὶ τῶν ἑρπετῶν τῆς γῆς· καὶ τόξον, καὶ ῥομφαίαν, καὶ πόλεμον συντρίψω ἀπὸ τῆς γῆς, καὶ κατοικιῶ σε ἐπʼ ἐλπίδι.
\vs{21}Καὶ μνηστεύσομαί σε ἐμαυτῷ εἰς τὸν αἰῶνα· καὶ μνηστεύσομαί σε ἐμαυτῷ ἐν δικαιοσύνῃ καὶ ἐν κρίματι, καὶ ἐν ἐλέει, καὶ ἐν οἰκτιρμοῖς,
\vs{22}καὶ μνηστεύσομαί σε ἐμαυτῷ ἐν πίστει, καὶ ἐπιγνώσῃ τὸν Κύριον.

\vs{23}Καὶ ἔσται ἐν ἐκείνῃ τῇ ἡμέρᾳ, λέγει Κύριος, ἐπακούσομαι τῷ οὐρανῷ, καὶ αὐτὸς ἐπακούσεται τῇ γῇ,
\vs{24}καὶ ἡ γῆ ἐπακούσεται τὸν σῖτον, καὶ τὸν οἶνον, καὶ τὸ ἔλαιον, καὶ αὐτὰ ἐπακούσεται τῷ Ἰεζραέλ.
\vs{25}Καὶ σπερῶ αὐτὴν ἐμαυτῷ ἐπὶ τῆς γῆς, καὶ ἀγαπήσω τὴν οὐκ ἠγαπημένην, καὶ ἐρῶ τῷ οὐ λαῷ μου, λαός μου εἶ σύ· καὶ αὐτος ἐρεὶ, Κύριος ὁ Θεός μου εἶ σύ.

\ch{3}
Καὶ εἶπε, Κύριος πρὸς μὲ, ἔτι πορεύθητι, καὶ ἀγάπησον γυναῖκα ἀγαπῶσαν πονηρὰ, καὶ μοιχαλὶν, καθὼς ἀγαπᾷ ὁ Θεὸς τοὺς υἱοὺς Ἰσραὴλ, καὶ αὐτοὶ ἐπιβλέπουσιν ἐπὶ θεοὺς αλλοτρίους, καὶ φιλοῦσι πέμματα μετὰ σταφίδος.
\vs{2}Καὶ ἐμισθωσάμην ἐμαυτῷ πεντεκαίδεκα ἀργυρίου, καὶ γομὸρ κριθῶν, καὶ νέβελ οἴνου.
\vs{3}Καὶ εἶπα πρὸς αὐτὴν, ἡμέρας πολλὰς καθήσῃ ἐπʼ ἐμοὶ, καὶ οὐ μὴ πορνεύσῃς, οὐδὲ μὴ γένῃ ἀνδρὶ, καὶ ἐγὼ ἐπὶ σοί.

\vs{4}Διότι ἡμέρας πολλὰς καθήσονται οἱ υἱοὶ Ἰσραὴλ, οὐκ ὄντος βασιλέως, οὐδὲ ὄντος ἄρχοντος, οὐδὲ οὔσης θυσίας, οὐδὲ ὄντος θυσιαστηρ ίου, οὐδὲ ἱερατείας, οὐδὲ δήλων.
\vs{5}Καὶ μετὰ ταῦτα ἐπιστρέψουσιν οἱ υἱοὶ Ἰσραὴλ, καὶ ἐπιζητήσουσι Κύριον τὸν Θεὸν αὐτῶν, καὶ Δαυὶδ τὸν βασιλέα αὐτῶν, καὶ ἐκστήσονται ἐπὶ τῷ Κυρίῳ, καὶ ἐπὶ τοῖς ἀγαθοῖς αὐτοῦ ἐπʼ ἐσχάτων τῶν ἡμερῶν.

\ch{4}
Ἀκούσατε λόγον Κυρίου υἱοὶ Ἰσραὴλ, ὅτι κρίσις τῷ Κυρίῳ πρὸς τοὺς κατοικοῦντας τὴν γῆν, διότι οὐκ ἔστιν ἀλήθεια, οὐδὲ ἔλεος, οὐδὲ ἐπίγνωσις Θεοῦ ἐπὶ τῆς γῆς.
\vs{2}Ἀρὰ, καὶ ψεῦδος, καὶ φόνος καὶ κλοπὴ, καὶ μοιχεία κέχυται ἐπὶ τῆς γῆς, καὶ αἵματα ἐφʼ αἵμασι μίσγουσι.
\vs{3}Διὰ τοῦτο πενθήσει ἡ γῆ, καὶ σμικρυνθήσεται σὺν πᾶσι τοῖς κατοικοῦσιν αὐτὴν, σὺν τοῖς θηρίοις τοῦ ἀγροῦ, καὶ σὺν τοῖς ἑρπετοῖς τῆς γῆς, καὶ σὺν τοῖς πετεινοῖς τοῦ οὐρανοῦ, καὶ οἱ ἰχθύες τῆς θαλάσσης ἐκλείψουσιν,
\vs{4}ὅπως μηδεὶς μήτε δικάζηται, μήτε ἐλέγχῃ μηδείς· ὁ δὲ λαός μου ὡς ἀντιλεγόμενος ἱερεύς.
\vs{5}Καὶ ἀσθενήσει ἡμέρας, καὶ ἀσθενήσει ὁ προφήτης μετὰ σοῦ· νυκτὶ ὡμοίωσα τὴν μητέρα σου.

\vs{6}Ὡμοιώθη ὁ λαός μου, ὡς οὐκ ἔχων γνῶσιν· ὅτι σὺ ἐπίγνωσιν ἀπώσω, κᾀγὼ ἀπώσομαί σε, τοῦμὴ ἱερατεύειν μοι· καὶ ἐπελάθου νόμον Θεοῦ σου, κᾀγὼ ἐπιλήσομαι τέκνων σου.
\vs{7}Κατὰ τὸ πλῆθος αὐτῶν, οὕτως ἥμαρτόν μοι· τὴν δόξαν αὐτῶν εἰς ἀτιμίαν θήσομαι.
\vs{8}Ἁμαρτίας λαοῦ μου φάγονται· καὶ ἐν ταῖς ἀδικίαις αὐτῶν λήψονται τὰς ψυχὰς αὐτῶν.
\vs{9}Καὶ ἔσται καθὼς ὁ λαὸς, οὕτως καὶ ὁ ἱερεύς· καὶ ἐκδικήσω ἐπʼ αὐτὸν τὰς ὁδοὺς αὐτοῦ, καὶ τὰ διαβούλια αὐτοῦ ἀνταποδώσω αὐτῷ.
\vs{10}Καὶ φάγονται, καὶ οὐ μὴ ἐμπλησθῶσιν· ἐπόρνευσαν, καὶ οὐ μὴ κατευθύνωσι· διότι τὸν Κύριον ἐγκατέλιπον τοῦ φυλάξαι.

\vs{11}Πορνείαν καὶ οἶνον καὶ μέθυσμα ἐδέξατο καρδία λαοῦ μου·
\vs{12}ἐν συμβόλοις ἐπηρώτων, καὶ ἐν ῥάβδοις αὐτοῦ ἀπήγγελλον αὐτῷ· πνεύματι πορνείας ἐπλανήθησαν, καὶ ἐξεπόρνευσαν ἀπὸ τοῦ Θεοῦ αὐτῶν.
\vs{13}Ἐπὶ τὰς κορυφὰς τῶν ὀρέων ἐθυσίαζον, καὶ ἐπὶ τοὺς βουνοὺς ἔθυον ὑποκάτω δρυὸς, καὶ λεύκης, καὶ δένδρου συσκιάζοντος, ὅτι καλὸν σκέπη. διὰ τοῦτο ἐκπορνεύσουσιν αἱ θυγατέρες ὑμῶν, καὶ αἱ νύμφαι ὑμῶν μοιχεύσουσι.
\vs{14}Καὶ οὐ μὴ ἐπισκέψωμαι ἐπὶ τὰς θυγατέρας ὑμῶν ὅταν πορνεύσωσι, καὶ ἐπὶ τὰς νύμφας ὑμῶν ὅταν μοιχεύωσιν· ὅτι αὐτοὶ μετὰ τῶν πορνῶν συνεφύροντο, καὶ μετὰ τῶν τετελεσμένων ἔθυον, καὶ ὁ λαὸς ὁ μὴ συνιὼν συνεπλέκετο μετὰ πόρνης.

\vs{15}Σὺ δὲ Ἰσραὴλ μὴ ἀγνόει, καὶ Ἰούδα μὴ εἰσπορεύεσθε εἰς Γάλγαλα, καὶ μὴ ἀναβαίνετε εἰς τὸν οἶκον Ὦν, καὶ μὴ ὀμνύετε ζῶντα Κύριον.
\vs{16}Διότι ὡς δάμαλις παροιστρῶσα παροίστρησεν Ἰσραήλ· νῦν νεμήσει αὐτοὺς Κύριος ὡς ἀμνὸν ἐν εὐρυχώρῳ.
\vs{17}Μέτοχος εἰδώλων Ἐφραὶμ ἔθηκεν ἑαυτῷ σκάνδαλα,
\vs{18}ἠρέτισε Χαναναίους· πορνεύοντες ἐξεπόρνευσαν, ἠγάπησαν ἀτιμίαν ἐκ φρυάγματος αὐτῆς.
\vs{19}Συστροφὴ πνεύματος σὺ εἶ ἐν ταῖς πτέρυξιν αὐτῆς, καὶ καταισχυνθήσονται ἐκ τῶν θυσιαστηρίων αὐτῶν.

\ch{5}
Ἀκούσατε ταῦτα οἱ ἱερεῖς, καὶ προσέχετε οἶκος Ἰσραήλ, καὶ ὁ οἶκος τοῦ βασιλέως ἐνωτίζεσθε, διότι πρὸς ὑμᾶς ἐστι τὸ κρίμα, ὅτι παγὶς ἐγενήθητε τῇ Σκοπιᾷ, καὶ ὡς δίκτυον ἐκτεταμένον ἐπὶ τὸ Ἰταβύριον,
\vs{2}ὃ οἱ ἀγρεύοντες τὴν θήραν κατέπηξαν· ἐγὼ δὲ παιδευτὴς ὑμῶν,
\vs{3}ἐγὼ ἔγνων τὸν Ἐφραὶμ, καὶ Ἰσραὴλ οὐκ ἀπέστη ἀπʼ ἐμοῦ· διότι νῦν ἐξεπόρνευσεν Ἐφραὶμ, ἐμιάνθη Ἰσραήλ.
\vs{4}Οὐκ ἔδωκαν τὰ διαβούλια αὐτῶν τοῦ ἐπιστρέψαι πρὸς τὸν Θεὸν αὐτῶν, ὅτι πνεῦμα πορνίας ἐν αὐτοῖς ἐστίν· τὸν δὲ Κύριον οὐκ ἐπέγνωσαν.

\vs{5}Καὶ ταπεινωθήσεται ἡ ὕβρις τοῦ Ἰσραὴλ εἰς πρόσωπον αὐτοῦ· καὶ Ἰσραὴλ καὶ Ἐφραὶμ ἀσθενήσουσιν ἐν ταῖς ἀδικίαις αὐτῶν· καὶ ἀσθενήσει καὶ Ἰούδας μετʼ αὐτῶν.
\vs{6}Μετὰ προβάτων καὶ μόσχων πορεύσονται τοῦ ἐκζητῆσαι τὸν Κύριον, καὶ οὐ μὴ εὕρωσιν αὐτόν· ὅτι ἐκκέκλικεν ἀπʼ αὐτῶν,
\vs{7}ὅτι τὸν Κύριον ἐγκατέλιπον, ὅτι τέκνα ἀλλότρια ἐγεννήθησαν αὐτοῖς· νῦν καταφάγεται αὐτοὺς ἡ ἐρυσίβη, καὶ τοὺς κλήρους αὐτῶν.

\vs{8}Σαλπίσατε σάλπιγγι ἐπὶ τοὺς βουνοὺς, ἠχήσατε ἐπὶ τῶν ὑψηλῶν, κηρύξατε ἐν τῷ οἴκῳ Ὦν, ἐξέστη Βενιαμὶν,
\vs{9}Ἐφραὶμ εἰς ἀφανισμὸν ἐγένετο ἐν ἡμέραις ἐλέγχου· ἐν ταῖς φυλαῖς τοῦ Ἰσραὴλ ἔδειξα πιστά.
\vs{10}Ἐγένοντο οἱ ἄρχοντες Ἰούδα ὡς μετατιθέντες ὅρια, ἐπʼ αὐτοὺς ἐκχεῶ ὡς ὕδωρ τὸ ὅρμημά μου.

\vs{11}Κατεδυνάστευσεν Ἐφραὶμ τὸν ἀντίδικον αὐτοῦ, κατεπάτησε τὸ κρίμα, ὅτι ἤρξατο πορεύεσθαι ὀπίσω τῶν ματαίων.
\vs{12}Καὶ ἐγὼ ὡς ταραχὴ τῷ Ἐφραὶμ, καὶ ὡς κέντρον τῷ οἴκῳ Ἰούδα.
\vs{13}Καὶ εἶδεν Ἐφραὶμ τὴν νόσον αὐτοῦ, καὶ Ἰούδας τὴν ὀδύνην αὐτοῦ· καὶ ἐπορεύθη Ἐφραὶμ πρὸς Ἀσσυρίους, καὶ ἀπέστειλε πρέσβεις πρὸς βασιλέα Ἰαρείμ· καὶ οὗτος οὐκ ἠδυνάσθη ἰάσασθαι ὑμᾶς, καὶ οὐ μὴ διαπαύσῃ ἐξ ὑμῶν ὀδύνη.
\vs{14}Διότι ἐγώ εἰμι ὡς πανθὴρ τῷ Ἐφραὶμ, καὶ ὡς λέων τῷ οἴκῳ Ἰούδα· καὶ ἐγὼ ἁρπῶμαι καὶ πορεύσομαι, καὶ λήψομαι, καὶ οὐκ ἔσται ὁ ἐξαιρούμενος.

\vs{15}Πορεύσομαι καὶ ἐπιστρέψω εἰς τὸν τόπον μου, ἕως οὗ ἀφανισθῶσι, καὶ ζητήσουσι τὸ πρόσωπόν μου.

Ἐν θλίψει αὐτῶν ὀρθριοῦσι πρὸς μὲ, λέγοντες,

\ch{6}πορευθῶμεν, καὶ ἐπιστρέψωμεν πρὸς Κύριον τὸν Θεὸν ἡμῶν, ὅτι αὐτὸς ἥρπακε, καὶ ἰάσεται ἡμᾶς· πατάξει,
\vs{2}καὶ μοτώσει ἡμᾶς, ὑγιάσει ἡμᾶς μετὰ δύο ἡμέρας· ἐν τῇ ἡμέρᾳ τῇ τρίτῃ ἐξαναστησόμεθα,
\vs{3}καὶ ζησόμεθα ἐνώπιον αὐτοῦ, καὶ γνωσόμεθα· διώξωμεν τοῦ γνῶναι τὸν Κύριον· ὡς ὄρθρον ἕτοιμον εὑρήσομεν αὐτὸν, καὶ ἥξει ὡς ὑετὸς ἡμῖν πρώϊμος καὶ ὄψιμος γῇ.

\vs{4}Τί σοι ποιήσω Ἐφραίμ; τί σοι ποιήσω Ἰούδα; τὸ δὲ ἔλεος ὑμῶν ὡς νεφέλη πρωϊνὴ, καὶ ὡς δρόσος ὀρθρινὴ πορευομένη.
\vs{5}Διὰ τοῦτο ἀπεθέρισα τοὺς προφήτας ὑμῶν· ἀπέκτεινα αὐτοὺς ἐν ῥήματι στόματός μου· καὶ τὸ κρίμα μου ὡς φῶς ἐξελεύσεται.

\vs{6}Διότι ἔλεος θέλω ἢ θυσίαν, καὶ ἐπίγνωσιν Θεοῦ ἢ ὁλοκαυτώματα.
\vs{7}Αὐτοὶ δέ εἰσιν ὡς ἄνθρωπος παραβαίνων διαθήκην· ἐκεῖ κατεφρόνησέ μου Γαλαὰδ πόλις,
\vs{8}ἐργαζομένη μάταια, ταράσσουσα ὕδωρ,
\vs{9}καὶ ἡ ἰσχύς σου ἀνδρὸς πειρατοῦ· ἔκρυψαν ἱερεῖς ὁδόν, ἐφόνευσαν Σίκιμα, ὅτι ἀνομίαν ἐποίησαν ἐν τῷ οἴκῳ τοῦ Ἰσραήλ· εἶδον φρικώδη ἐκεῖ,
\vs{10}πορνείαν τοῦ Ἐφραίμ· ἐμιάνθη Ἰσραὴλ καὶ Ἰούδα· ἄρχου τρυγᾷν σεαυτῷ,
\vs{11}ἐν τῷ ἐπιστρέφειν με τὴν αἰχμαλωσίαν τοῦ λαοῦ μου.

\ch{7}
Ἐν τῷ ἰάσασθαί με τὸν Ἰσραὴλ, καὶ ἀποκαλυφθήσεται ἡ ἀδικία Ἐφραὶμ, καὶ ἡ κακία Σαμαρείας, ὅτι εἰργάσαντο ψευδῆ· καὶ κλέπτης πρὸς αὐτὸν εἰσελεύσεται, ἐκδιδύσκων λῃστὴς ἐν τῇ ὁδῷ αὐτοῦ,
\vs{2}ὅπως συνᾴδωσιν ὡς ᾄοντες τῇ καρδίᾳ αὐτῶν· πάσας τὰς κακίας αὐτῶν ἐμνήσθην· νῦν ἐκύκλωσαν αὐτοὺς τὰ διαβούλια αὐτῶν, ἀπέναντι τοῦ προσώπου μου ἐγένοντο.
\vs{3}Ἐν ταῖς κακίαις αὐτῶν εὔφραναν βασιλεῖς, καὶ ἐν τοῖς ψεύδεσιν αὐτῶν ἄρχοντας.
\vs{4}Πάντες μοιχεύοντες ὡς κλίβανος καιόμενος εἰς πέψιν κατακαύματος ἀπὸ τῆς φλογὸς, ἀπὸ φυράσεως στέατος, ἕως τοῦ ζυμωθῆναι αὐτό.
\vs{5}Ἡμέραι τῶν βασιλέων ὑμῶν, ἤρξαντο οἱ ἄρχοντες θυμοῦσθαι ἐξ οἴνου, ἐξέτεινε τὴν χεῖρα αὐτοῦ μετὰ λοιμῶν.
\vs{6}Διότι ἀνεκαύθησαν ὡς κλίβανος αἱ καρδίαι αὐτῶν, ἐν τῷ καταράσσειν αὐτοὺς ὅλην τὴν νύκτα· ὕπνου Ἐφραὶμ ἐνεπλήσθη, πρωῒ ἐνεγενήθη, ἀνεκαύθη ὡς πυρὸς φέγγος.
\vs{7}Πάντες ἐθερμάνθησαν ὡς κλίβανος, καὶ κατέφαγον τοὺς κριτὰς αὐτῶν· πάντες οἱ βασιλεῖς αὐτῶν ἔπεσαν, οὐκ ἦν ἐν αὐτοῖς ὁ ἐπικαλούμενος πρὸς μέ.

\vs{8}Ἐφραὶμ ἐν τοῖς λαοῖς αὐτοῦ συνεμίγνυτο, Ἐφραὶμ ἐγένετο ἐγκρυφίας, οὐ μεταστρεφόμενος.
\vs{9}Κατέφαγον ἀλλότριοι τὴν ἰσχὺν αὐτοῦ, αὐτὸς δὲ οὐκ ἔγνω· καὶ πολιαὶ ἐξήνθησαν αὐτῷ, καὶ αὐτὸς οὐκ ἔγνω.
\vs{10}Καὶ ταπεινωθήσεται ἡ ὕβρις Ἰσραὴλ εἰς πρόσωπον αὐτοῦ· καὶ οὐκ ἐπέστρεψαν πρὸς Κύριον τὸν Θεὸν αὐτῶν, καὶ οὐκ ἐξεζήτησαν αὐτὸν ἐν πᾶσι τούτοις.

\vs{11}Καὶ ἦν Ἐφραὶμ ὡς περιστερὰ ἄνους, οὐκ ἔχουσα καρδίαν· Αἴγυπτον ἐπεκαλεῖτο, καὶ εἰς Ἀσσυρίους ἐπορεύθησαν·
\vs{12}Καθὼς ἂν πορεύωνται, ἐπιβαλῶ ἐπʼ αὐτοὺς τὸ δίκτυόν μου, καθὼς τὰ πετεινὰ τοῦ οὐρανοῦ κατάξω αὐτοὺς, παιδεύσω αὐτοὺς ἐν τῇ ἀκοῇ τῆς θλίψεως αὐτῶν.

\vs{13}Οὐαὶ αὐτοῖς, ὅτι ἀπεπήδησαν ἀπʼ ἐμοῦ· δείλαιοί εἰσιν, ὅτι ἠσέβησαν εἰς ἐμέ· ἐγὼ δὲ ἐλυτρωσάμην αὐτοὺς, αὐτοὶ δὲ κατελάλησαν κατʼ ἐμοῦ ψευδῆ.
\vs{14}Καὶ οὐκ ἐβόησαν πρὸς μὲ αἱ καρδίαι αὐτῶν, ἀλλʼ ἢ ὠλόλυζον ἐν ταῖς κοίταις αὐτῶν· ἐπὶ σίτῳ καὶ οἴνῳ κατετέμνοντο.
\vs{15}Ἐπαιδεύθησαν ἐν ἐμοὶ, κᾀγὼ κατίσχυσα τοὺς βραχίονας αὐτῶν, καὶ εἰς ἐμὲ ἐλογίσαντο πονηρά.
\vs{16}Ἀπεστράφησαν εἰς οὐδὲν, ἐγένοντο ὡς τόξον ἐντεταμένον· πεσοῦνται ἐν ῥομφαίᾳ οἱ ἄρχοντες αὐτῶν διʼ ἀπαιδευσίαν γλώσσης αὐτῶν· οὗτος ὁ φαυλισμὸς αὐτῶν ἐν γῇ Αἰγύπτῳ.

\ch{8}
Εἰς κόλπον αὐτῶν, ὡς γῆ, ὡς ἀετὸς ἐπʼ οἶκον Κυρίου, ἀνθʼ ὧν παρέβησαν τὴν διαθήκην μου, καὶ κατὰ τοῦ νόμου μου ἠσέβησαν.
\vs{2}Ἐμὲ κεκράξονται, ὁ Θεὸς, ἐγνώκαμέν σε·
\vs{3}Ὅτι Ἰσραὴλ ἀπεστρέψατο ἀγαθὰ, ἐχθρὸν κατεδίωξαν.
\vs{4}Ἑαυτοῖς ἐβασίλευσαν, καὶ οὐ διʼ ἐμοῦ, ἦρξαν, καὶ οὐκ ἐγνώρισάν μοι· τὸ ἀργύριον αὐτῶν καὶ τὸ χρυσίον αὐτῶν ἐποίησαν ἑαυτοῖς εἴδωλα, ὅπως ἐξολοθρευθῶσιν.

\vs{5}Ἀπότριψαι τὸν μόσχον σου Σαμάρεια, παρωξύνθη ὁ θυμός μου ἐπʼ αὐτούς· ἕως τίνος οὐ μὴ δύνωνται καθαρισθῆναι
\vs{6}ἐν τῷ Ἰσραήλ; καὶ αὐτὸ τέκτων ἐποίησε, καὶ οὐ Θεός ἐστι· διότι πλανῶν ἦν ὁ μόσχος σου, Σαμάρεια.
\vs{7}Ὅτι ἀνεμόφθορα ἔσπειραν, καὶ ἡ καταστροφὴ αὐτῶν ἐκδέξεται αὐτά· δράγμα οὐκ ἔχον ἰσχὺν τοῦ ποιῆσαι ἄλευρον· ἐὰν δὲ καὶ ποιήσῃ, ἀλλότριοι καταφάγονται αὐτό.
\vs{8}Κατεπόθη Ἰσραὴλ, νῦν ἐγένετο ἐν τοῖς ἔθνεσιν ὡς σκεῦος ἄχρηστον,
\vs{9}ὅτι αὐτοὶ ἀνέβησαν εἰς Ἀσσυρίους· ἀνέθαλε καθʼ ἑαυτὸν Ἐφραίμ· δῶρα ἠγάπησαν,
\vs{10}διὰ τοῦτο παραδοθήσονται ἐν τοῖς ἔθνεσι· νῦν εἰσδέξομαι αὐτοὺς, καὶ κοπάσουσι μικρὸν τοῦ χρίειν βασιλέα καὶ ἄρχοντας.

\vs{11}Ὅτι ἐπλήθυνεν Ἐφραὶμ θυσιαστήρια, εἰς ἁμαρτίας ἐγένοντο αὐτῷ θυσιαστήρια ἠγαπημένα.
\vs{12}Καταγράψω αὐτῷ πλῆθος, καὶ τὰ νόμιμα αὐτοῦ εἰς ἀλλότρια ἐλογίσθησαν, θυσιαστήρια τὰ ἠγαπημένα.
\vs{13}Διότι ἐὰν θύσωσι θυσίαν, καὶ φάγωσι κρέα, Κύριος οὐ προσδέξεται αὐτά· νῦν μνησθήσεται τὰς ἀδικίας αὐτῶν, καὶ ἐκδικήσει τὰς ἁμαρτίας αὐτῶν· αὐτοὶ εἰς Αἴγυπτον ἀπέστρεψαν, καὶ ἐν Ἀσσυρίοις ἀκάθαρτα φάγονται.
\vs{14}Καὶ ἐπελάθετο Ἰσραὴλ τοῦ ποιήσαντος αὐτόν, καὶ ᾠκοδόμησαν τεμένη· καὶ Ἰούδας ἐπλήθυνε πόλεις τετειχισμένας· καὶ ἐξαποστελῶ πῦρ εἰς τὰς πόλεις αὐτοῦ, καὶ καταφάγεται τὰ θεμέλια αὐτῶν.

\ch{9}
Μὴ χαῖρε Ἰσραήλ, μηδὲ εὐφραίνου καθὼς οἱ λαοὶ, διότι ἐπόρνευσας ἀπὸ τοῦ Θεοῦ σου· ἠγάπησας δόματα ἐπὶ πάντα ἅλωνα σίτου.
\vs{2}Ἅλων καὶ ληνὸς οὐκ ἔγνω αὐτούς, καὶ ὁ οἶνος ἐψεύσατο αὐτούς.
\vs{3}Οὐ κατῴκησαν ἐν τῇ γῇ τοῦ Κυρίου· κατῴκησεν Ἐφραὶμ Αἴγυπτον, καὶ ἐν Ἀσσυρίοις ἀκάθαρτα φάγονται.
\vs{4}Οὐκ ἔσπεισαν τῷ Κυρίῳ οἶνον, καὶ οὐχ ἥδυναν αὐτῷ αἱ θυσίαι αὐτῶν, ὡς ἄρτος πένθους αὐτοῖς· πάντες οἱ ἐσθίοντες αὐτὰ μιανθήσονται, διότι οἱ ἄρτοι αὐτῶν ταῖς ψυχαῖς αὐτῶν οὐκ εἰσελεύσονται εἰς τὸν οἶκον Κυρίου.

\vs{5}Τί ποιήσετε ἐν ἡμέραις πανηγύρεως, καὶ ἐν ἡμέρᾳ ἑορτῆς τοῦ Κυρίου;
\vs{6}Διὰ τοῦτο ἰδοὺ πορεύονται ἐκ ταλαιπωρίας Αἰγύπτου, καὶ ἐκδέξεται αὐτοὺς Μέμφις, καὶ θάψει αὐτοὺς Μαχμάς· τὸ ἀργύριον αὐτῶν ὄλεθρος κληρονομήσει αὐτὸ, ἄκανθαι ἐν τοῖς σκηνώμασιν αὐτῶν.

\vs{7}Ἥκασιν αἱ ἡμέραι τῆς ἐκδικήσεως, ἥκασιν αἱ ἡμέραι τῆς ἀνταποδόσεώς σου, καὶ κακωθήσεται Ἰσραὴλ ὥσπερ ὁ προφήτης ὁ παρεξεστηκώς, ἄνθρωπος ὁ πνευματοφόρος· ὑπὸ τοῦ πλήθους τῶν ἀδικιῶν σου ἐπληθύνθη μανία σου.
\vs{8}Σκοπὸς Ἐφραὶμ μετὰ Θεοῦ· προφήτης παγὶς σκολιὰ ἐπὶ πάσας τὰς ὁδοὺς αὐτοῦ, μανίαν ἐν οἴκῳ Θεοῦ κατέπηξαν.
\vs{9}Ἐφθάρησαν κατὰ τὰς ἡμέρας τοῦ βουνοῦ, μνησθήσεται ἀδικίας αὐτῶν, ἐκδικήσει ἁμαρτίας αὐτῶν.

\vs{10}Ὡς σταφυλὴν ἐν ἐρήμῳ εὗρον τὸν Ἰσραὴλ, καὶ ὡς σκοπὸν ἐν συκῇ πρώϊμον πατέρας αὐτῶν εἶδον· αὐτοὶ εἰσῆλθον πρὸς τὸν Βεελφεγὼρ, καὶ ἀπηλλοτριώθησαν εἰς αἰσχύνην, καὶ ἐγένοντο οἱ ἐβδελυγμένοι ὡς οἱ ἠγαπημένοι.
\vs{11}Ἐφραὶμ ὡς ὄρνεον ἐξεπετάσθη, αἱ δόξαι αὐτῶν ἐκ τόκων καὶ ὠδίνων καὶ συλλήψεων·
\vs{12}Διότι καὶ ἐὰν ἐκθρέψωσι τὰ τέκνα αὐτῶν, ἀτεκνωθήσονται ἐξ ἀνθρώπων· διότι καὶ οὐαὶ αὐτοῖς ἐστι· σάρξ μου ἐξ αὐτῶν.
\vs{13}Ἐφραὶμ, ὃν τρόπον εἶδον, εἰς θήραν παρέστησαν τὰ τέκνα αὐτῶν, καὶ Ἐφραὶμ, τοῦ ἐξαγαγεῖν εἰς ἀποκέντησιν τὰ τέκνα αὐτοῦ.

\vs{14}Δὸς αὐτοῖς Κύριε, τί δώσεις αὐτοῖς; μήτραν ἀτεκνοῦσαν, καὶ μαστοὺς ξηρούς.
\vs{15}Πᾶσαι αἱ κακίαι αὐτῶν ἐν Γαλγάλ, ὅτι ἐκεῖ ἐμίσησα αὐτούς· διὰ τὰς κακίας τῶν ἐπιτηδευμάτων αὐτῶν, ἐκ τοῦ οἴκου μου ἐκβαλῶ αὐτοὺς, οὐ μὴ προσθήσω τοῦ ἀγαπῆσαι αὐτούς· πάντες οἱ ἄρχοντες αὐτῶν ἀπειθοῦντες.
\vs{16}Ἐπόνεσεν Ἐφραίμ· τὰς ῥίζας αὐτοῦ ἐξηράνθη, καρπὸν οὐκ ἔτι μὴ ἐνέγκῃ· διότι καὶ ἐὰν γεννήσωσιν, ἀποκτενῶ τὰ ἐπιθυμήματα κοιλίας αὐτῶν.
\vs{17}Ἀπώσεται αὐτοὺς ὁ Θεός, ὅτι οὐκ εἰσήκουσαν αὐτοῦ, καὶ ἔσονται πλανῆται ἐν τοῖς ἔθνεσιν.

\ch{10}
Ἄμπελος εὐκηματοῦσα Ἰσραὴλ, ὁ καρπὸς εὐθηνῶν αὐτῆς· κατὰ τὸ πλῆθος τῶν καρπῶν αὐτῆς, ἐπλήθυνε τὰ θυσιαστήρια· κατὰ τὰ ἀγαθὰ τῆς γῆς αὐτοῦ, ᾠκοδόμησε στήλας.
\vs{2}Ἐμέρισαν καρδίας αὐτῶν, νῦν ἀφανισθήσονται· αὐτὸς κατασκάψει τὰ θυσιαστήρια αὐτῶν, ταλαιπωρήσουσιν αἱ στῆλαι αὐτῶν.

\vs{3}Διότι νῦν ἐροῦσιν, οὐκ ἔστι βασιλεὺς ἡμῖν, ὅτι οὐκ ἐφοβήθημεν τὸν Κύριον· ὁ δὲ βασιλεὺς τί ποιήσει ἡμῖν,
\vs{4}λαλῶν ῥήματα προφάσεις ψευδεῖς; διαθήσεται διαθήκην, ἀνατελεῖ ὡς ἄγρωστις κρίμα ἐπὶ χέρσον ἀγροῦ.
\vs{5}Τῷ μόσχῳ τοῦ οἴκου Ὦν παροικήσουσιν οἱ κατοικοῦντες Σαμάρειαν, ὅτι ἐπένθησε λαὸς αὐτοῦ ἐπʼ αὐτόν· καὶ καθὼς παρεπίκραναν αὐτὸν, ἐπιχαροῦνται ἐπὶ τὴν δόξαν αὐτοῦ, ὅτι μετῳκίσθη ἀπʼ αὐτοῦ.
\vs{6}Καὶ αὐτὸν εἰς Ἀσσυρίους δήσαντες, ἀπήνεγκαν ξένια τῷ βασιλεῖ Ἰαρείμ· ἐν δόματι Ἐφραὶμ δέξεται, καὶ αἰσχυνθήσεται Ἰσραὴλ ἐν τῇ βουλῇ αὐτοῦ.
\vs{7}Ἀπέῤῥιψεν Σαμάρεια βασιλέα αὐτῆς ὡς φρύγανον ἐπὶ προσώπου ὕδατος·
\vs{8}Καὶ ἐξαρθήσονται βωμοὶ Ὦν ἁμαρτήματα τοῦ Ἰσραὴλ, ἄκανθαι καὶ τρίβολοι ἀναβήσονται ἐπὶ τὰ θυσιαστήρια αὐτῶν· καὶ ἐροῦσι τοῖς ὄρεσι, καλύψατε ἡμᾶς, καὶ τοῖς βουνοῖς, πέσατε ἐφʼ ἡμᾶς.

\vs{9}Ἀφʼ οὗ οἱ βουνοὶ, ἥμαρτεν Ἰσραήλ, ἐκεῖ ἔστησαν· οὐ μὴ καταλάβῃ αὐτοὺς ἐν τῷ βουνῷ πόλεμος ἐπὶ τὰ τὲκνα ἀδικίας
\vs{10}παιδεῦσαι αὐτούς· καὶ συναχθήσονται ἐπʼ αὐτοὺς λαοὶ, ἐν τῷ παιδεύεσθαι αὐτοὺς ἐν ταῖς δυσὶν ἀδικίαις αὐτῶν.
\vs{11}Ἐφραὶμ δάμαλις δεδιδαγμένη ἀγαπᾷν νῖκος, ἐγὼ δὲ ἐπελεύσομαι ἐπὶ τὸ κάλλιστον τοῦ τραχήλου αὐτῆς· ἐπιβιβῶ Ἐφραὶμ, παρασιωπήσομαι Ἰούδαν, ἐνισχύσει αὐτῷ Ἰακώβ.

\vs{12}Σπείρατε ἑαυτοῖς εἰς δικαιοσύνην, τρυγήσατε εἰς καρπὸν ζωῆς, φωτίσατε ἑαυτοῖς φῶς γνώσεως, ἐκζητήσατε τὸν Κύριον ἕως τοῦ ἐλθεῖν γεννήματα δικαιοσύνης ὑμῖν.
\vs{13}Ἱνατί παρεσιωπήσατε ἀσέβειαν, καὶ τὰς ἀδικίας αὐτῆς ἐτρυγήσατε; ἐφάγετε καρπὸν ψευδῆ, ὅτι ἤλπισας ἐν τοῖς ἁμαρτήμασί σου, ἐν πλήθει δυνάμεώς σου.
\vs{14}Καὶ ἐξαναστήσεται ἀπώλεια ἐν τῷ λαῷ σου, καὶ πάντα τὰ περιτετειχισμένα σου οἰχήσεται· ὡς ἄρχεν Σαλαμὰν ἐκ τοῦ οἴκου τοῦ Ἱεροβοὰμ, ἐν ἡμέραις πολέμου μητέρα ἐπὶ τέκνοις ἠδάφισαν,
\vs{15}οὕτως ποιήσω ὑμῖν οἶκος τοῦ Ἰσραὴλ ἀπὸ προσώπου ἀδικίας κακιῶν ὑμῶν.

\ch{11}
Ὄρθρου ἀπεῤῥίφησαν, ἀπεῤῥίφη βασιλεὺς Ἰσραήλ· ὅτι νήπιος Ἰσραὴλ, καὶ ἐγὼ ἠγάπησα αὐτόν, καὶ ἐξ Αἰγύπτου μετεκάλεσα τὰ τέκνα αὐτοῦ.
\vs{2}Καθὼς μετεκάλεσα αὐτούς, οὕτως ἀπῴχοντο ἐκ προσώπου μου· αὐτοὶ τοῖς Βααλεὶμ ἔθυον, καὶ τοῖς γλυπτοῖς ἐθυμίων.
\vs{3}Καὶ ἐγὼ συνεπόδισα τὸν Ἐφραὶμ, ἀνέλαβον αὐτὸν ἐπὶ τὸν βραχίονά μου, καὶ οὐκ ἔγνωσαν ὅτι ἴαμαι αὐτούς.
\vs{4}Ἐν διαφθορᾷ ἀνθρώπων ἐξέτεινα αὐτοὺς ἐν δεσμοῖς ἀγαπήσεώς μου, καὶ ἔσομαι αὐτοῖς ὡς ῥαπίζων ἄνθρωπος ἐπὶ τὰς σιαγόνας αὐτοῦ· καὶ ἐπιβλέψομαι πρὸς αὐτόν, δυνήσομαι αὐτῷ.

\vs{5}Κατῴκησεν Ἐφραὶμ ἐν Αἰγύπτῳ, καὶ Ἀσσοὺρ αὐτὸς βασιλεὺς αὐτοῦ· ὅτι οὐκ ἠθέλησεν ἐπιστρέψαι,
\vs{6}καὶ ἠσθένησεν ἐν ῥομφαίᾳ ἐν ταῖς πόλεσιν αὐτοῦ· καὶ κατέπαυσεν ἐν ταῖς χερσὶν αὐτοῦ, καὶ φάγονται ἐκ τῶν διαβουλίων αὐτῶν,
\vs{7}καὶ ὁ λαὸς αὐτοῦ ἐπικρεμάμενος ἐκ τῆς κατοικίας αὐτοῦ· καὶ ὁ Θεὸς ἐπὶ τὰ τίμια αὐτοῦ θυμωθήσεται, καὶ οὐ μὴ ὑψώσῃ αὐτόν.

\vs{8}Τί σε διαθῶμαι Ἐφραίμ; ὑπερασπιῶ σου Ἰσραήλ; τί σε διαθῶ; ὡς Ἀδάμα θήσομαί σε, καὶ ὡς Σεβοείμ· μετεστράφη ἡ καρδία μου ἐν τῷ αὐτῷ, συνεταράχθη ἡ μεταμέλειά μου·
\vs{9}Οὐ μὴ ποιήσω κατὰ τὴν ὀργὴν τοῦ θυμοῦ μου· οὐ μὴ ἐγκαταλίπω τοῦ ἐξαλειφθῆναι τὸν Ἐφραὶμ· διότι Θεὸς ἐγώ εἰμι, καὶ οὐκ ἄνθρωπος, ἐν σοὶ ἅγιος, καὶ οὐκ εἰσελεύσομαι εἰς πόλιν.
\vs{10}Ὀπίσω Κυρίου πορεύσομαι· ὡς λέων ἐρεύξεται, ὅτι αὐτὸς ὠρύσεται, καὶ ἐκστήσονται τέκνα ὑδάτων.
\vs{11}Ἐκστήσονται ὡς ὄρνεον ἐξ Αἰγύπτου, καὶ ὡς περιστερὰ ἐκ γῆς Ἀσσυρίων· καὶ ἀποκαταστήσω αὐτοὺς εἰς τοὺς οἴκους αὐτῶν, λέγει Κύριος.

\ch{12}
Ἐκύκλωσέ με ἐν ψεύδει Ἐφραὶμ, καὶ ἐν ἀσεβείαις οἶκος Ἰσραὴλ, καὶ Ἰούδα· νῦν ἔγνω αὐτοὺς ὁ Θεὸς, καὶ ὁ λαὸς ἅγιος κεκλήσεται Θεοῦ.

\vs{2}Ὁ δὲ Ἐφραὶμ πονηρὸν πνεῦμα, ἐδίωξε καύσωνα ὅλην τὴν ἡμέραν· κενὰ καὶ μάταια ἐπλήθυνε, καὶ διαθήκην μετὰ Ἀσσυρίων διέθετο, καὶ ἔλαιον εἰς Αἴγυπτον ἐνεπορεύετο.
\vs{3}Καὶ κρίσις τῷ Κυρίῳ πρὸς Ἰούδαν, τοῦ ἐκδικῆσαι τὸν Ἰακώβ· κατὰ τὰς ὁδοὺς αὐτοῦ καὶ κατὰ τὰ ἐπιτηδεύματα αὐτοῦ ἀποδώσει αὐτῷ.

\vs{4}Ἐν τῇ κοιλίᾳ ἐπτέρνισε τὸν ἀδελφὸν αὐτοῦ, καὶ ἐν κόποις αὐτοῦ ἐνίσχυσε πρὸς Θεόν.
\vs{5}Καὶ ἐνίσχυσε μετὰ ἀγγέλου, καὶ ἠδυνάσθη· ἔκλαυσαν, καὶ ἐδεήθησάν μου· ἐν τῷ οἴκῳ Ὦν εὕροσάν με, καὶ ἐκεῖ ἐλαλήθη πρὸς αὐτούς.
\vs{6}Ὁ δὲ Κύριος ὁ Θεὸς ὁ παντοκράτωρ ἔσται μνημόσυνον αὐτοῦ.
\vs{7}Καὶ σὺ ἐν Θεῷ σου ἐπιστρέψεις, ἔλεον καὶ κρίμα φυλάσσου, καὶ ἔγγιζε πρὸς τὸν Θεόν σου διαπαντός.

\vs{8}Χαναὰν, ἐν χειρὶ αὐτοῦ ζυγὸς ἀδικίας, καταδυναστεύειν ἠγάπησε.
\vs{9}Καὶ εἶπεν Ἐφραὶμ, πλὴν πεπλούτηκα, εὕρηκα ἀναψυχὴν ἐμαυτῷ· πάντες οἱ πόνοι αὐτοῦ οὐχ εὑρεθήσονται αὐτῷ, διʼ ἀδικίας ἃς ἥμαρτεν.
\vs{10}Ἐγὼ δὲ Κύριος ὁ Θεός σου ἀνήγαγόν σε ἐκ γῆς Αἰγύπτου, ἔτι κατοικιῶ σε ἐν σκηναῖς, καθὼς ἡμέραι ἑορτῆς.
\vs{11}Καὶ λαλήσω πρὸς προφήτας, καὶ ἐγὼ ὁράσεις ἐπλήθυνα, καὶ ἐν χερσὶ προφητῶν ὡμοιώθην.
\vs{12}Εἰ μὴ Γαλαάδ ἐστιν, ἄρα ψευδεῖς ἦσαν ἐν Γαλαὰδ ἄρχοντες θυσιάζοντες, καὶ τὰ θυσιαστήρια αὐτῶν, ὡς χελῶναι ἐπὶ χέρσον ἀγροῦ.

\vs{13}Καὶ ἀνεχώρησεν Ἰακὼβ εἰς πεδίον Συρίας, καὶ ἐδούλευσεν Ἰσραὴλ ἐν γυναικὶ, καὶ γυναικὶ ἐφυλάξατο.
\vs{14}Καὶ ἐν προφήτῃ ἀνήγαγεν Κύριος τὸν Ἰσραὴλ ἐκ γῆς Αἰγύπτου, καὶ ἐν προφήτῃ διεφυλάχθη.
\vs{15}Ἐθύμωσεν Ἐφραὶμ, καὶ παρώργισε, καὶ τὸ αἷμα αὐτοῦ ἐπʼ αὐτὸν ἐκχυθήσεται, καὶ τὸν ὀνειδισμὸν αὐτοῦ ἀνταποδώσει Κύριος αὐτῷ.

\ch{13}
Κατὰ τὸν λόγον Ἐφραὶμ δικαιώματα ἔλαβεν αὐτὸς ἐν τῷ Ἰσραήλ, καὶ ἔθετο αὐτὰ τῇ Βάαλ καὶ ἀπέθανε.
\vs{2}Καὶ νῦν προσέθεντο τοῦ ἁμαρτάνειν, καὶ ἐποίησαν ἑαυτοῖς χώνευμα ἐκ τοῦ ἀργυρίου αὐτῶν, κατʼ εἰκόνα εἰδώλων, ἔργα τεκτόνων συντετελεσμένα αὐτοῖς· αὐτοὶ λέγουσι, θύσατε ἀνθρώπους, μόσχοι γὰρ ἐκλελοίπασι.
\vs{3}Διὰ τοῦτο ἔσονται ὡς νεφέλη πρωϊνὴ καὶ ὡς δρόσος ὀρθρινὴ πορευομένη, ὡς χνοῦς ἀποφυσώμενος ἀφʼ ἅλωνος, καὶ ὡς ἀτμὶς ἀπὸ δακρύων.
\vs{4}Ἐγὼ δὲ Κύριος ὁ Θεός σου ὁ στερεῶν τὸν οὐρανὸν, καὶ κτίζων γῆν, οὗ αἱ χεῖρες ἔκτισαν πᾶσαν τὴν στρατιὰν τοῦ οὐρανοῦ, καὶ οὐ παρέδειξά σοι αὐτὰ τοῦ πορεύεσθαι ὀπίσω αὐτῶν· καὶ ἐγὼ ἀνήγαγόν σε ἐκ γῆς Αἰγύπτου, καὶ θεὸν πλὴν ἐμοῦ οὐ γνώσῃ, καὶ σώζων οὐκ ἔστι πάρεξ ἐμοῦ.
\vs{5}Ἐγὼ ἐποίμαινόν σε ἐν τῇ ἐρήμῳ, ἐν γῇ ἀοικήτῳ
\vs{6}κατὰ τὰς νομὰς αὐτῶν, καὶ ἐνεπλήσθησαν εἰς πλησμονήν, καὶ ὑψώθησαν αἱ καρδίαι αὐτῶν· ἕνεκα τούτου ἐπελάθοντό μου.
\vs{7}Καὶ ἔσομαι αὐτοῖς ὡς πανθὴρ, καὶ ὡς πάρδαλις· κατὰ τὴν ὁδὸν Ἀσσυρίων
\vs{8}ἀπαντήσομαι αὐτοῖς ὡς ἄρκτος ἡ ἀπορουμένη, καὶ διαῤῥήξω συγκλεισμὸν καρδίας αὐτῶν, καὶ καταφάγονται αὐτοὺς ἐκεῖ σκύμνοι δρυμοῦ, θηρία ἀγροῦ διασπάσει αὐτούς.

\vs{9}Τῇ διαφθορᾷ σου Ἰσραήλ τίς βοηθήσει;
\vs{10}Ποῦ ὁ βασιλεύς σου οὗτος; καὶ διασωσάτω σε ἐν πάσαις ταῖς πόλεσί σου· κρινάτω σε ὃν εἶπας, δός μοι βασιλέα καὶ ἄρχοντα.
\vs{11}Καὶ ἔδωκά σοι βασιλέα ἐν ὀργῇ μου, καὶ ἔσχον ἐν τῷ θυμῷ μου.

\vs{12}Συστροφὴν ἀδικίας Ἐφραὶμ, ἐγκεκρυμμένη ἡ ἁμαρτία αὐτοῦ,
\vs{13}ὠδῖνες ὡς τικτούσης ἥξουσιν αὐτῷ· οὗτος ὁ υἱός σου ὁ φρόνιμος, διότι οὐ μὴ ὑποστῇ ἐν συντριβῇ τέκνων.
\vs{14}Ἐκ χειρὸς ᾅδου ῥύσομαι, καὶ ἐκ θανάτου λυτρώσομαι αὐτούς· ποῦ ἡ δίκη σου θάνατε; ποῦ τὸ κέντρον σου ᾅδη; παράκλησις κέκρυπται ἀπὸ ὀφθαλμῶν μου.

\vs{15}Διότι οὗτος ἀναμέσον ἀδελφῶν διαστελεῖ, ἐπάξει καύσωνα ἄνεμον Κύριος ἐκ τῆς ἐρήμου ἐπʼ αὐτόν, καὶ ἀναξηρανεῖ τὰς φλέβας αὐτοῦ, ἐξερημώσει τὰς πηγὰς αὐτοῦ· αὐτὸς καταξηρανεῖ τὴν γῆν αὐτοῦ, καὶ πάντα τὰ σκεύη τὰ ἐπιθυμητὰ αὐτοῦ.

\ch{14}
Ἀφανισθήσεται Σαμάρεια, ὅτι ἀντέστη πρὸς τὸν Θεὸν αὐτῆς· ἐν ῥομφαίᾳ πεσοῦνται αὐτοὶ, καὶ τὰ ὑποτίτθια αὐτῶν ἐδαφισθήσονται, καὶ αἱ ἐν γαστρὶ ἔχουσαι αὐτῶν διαῤῥαγήσονται.

\vs{2}Ἐπιστράφηθι Ἰσραὴλ πρὸς Κύριον τὸν Θεόν σου, διότι ἠσθένησαν ἐν ταῖς ἀδικίαις σου.
\vs{3}Λάβετε μεθʼ ἑαυτῶν λόγους, καὶ ἐπιστράφητε πρὸς Κύριον τὸν Θεὸν ὑμῶν· εἴπατε αὐτῷ, ὅπως μὴ λάβητε ἀδικίαν, καὶ λάβητε ἀγαθὰ, καὶ ἀνταποδώσομεν καρπὸν χειλέων ἡμῶν.
\vs{4}Ἀσσοὺρ οὐ μὴ σώσῃ ἡμᾶς, ἐφʼ ἵππον οὐκ ἀναβησόμεθα· οὐκέτι μὴ εἴπωμεν, θεοὶ ἡμῶν, τοῖς ἔργοις τῶν χειρῶν ἡμῶν· ὁ ἐν σοὶ ἐλεήσει ὀρφανόν.

\vs{5}Ἰάσομαι τὰς κατοικίας αὐτῶν, ἀγαπήσω αὐτοὺς ὁμολόγως, ὅτι ἀπέστρεψε τὴν ὀργήν μου ἀπʼ αὐτοῦ.
\vs{6}Ἔσομαι ὡς δρόσος τῷ Ἰσραὴλ, ἀνθήσει ὡς κρίνον, καὶ βαλεῖ τὰς ῥίζας αὐτοῦ ὡς ὁ Λίβανος·
\vs{7}Πορεύσονται οἱ κλάδοι αὐτοῦ, καὶ ἔσται ὡς ἐλαία κατάκαρπος, καὶ ἡ ὀσφρασία αὐτοῦ ὡς Λιβάνου.
\vs{8}ἐπιστρέψουσι καὶ καθιοῦνται ὑπὸ τὴν σκέπην αὐτοῦ, ζήσονται καὶ μεθυσθήσονται σίτῳ· καὶ ἐξανθήσει ὡς ἄμπελος· μνημόσυνον αὐτοῦ, ὡς οἶνος Λιβάνου
\vs{9}τῷ Ἐφραίμ· τί αὐτῷ ἔτι καὶ εἰδώλοις; ἐγὼ ἐταπείνωσα αὐτὸν, καὶ κατισχύσω αὐτόν· ἐγὼ ὡς ἄρκευθος πυκάζουσα, ἐξ ἐμοῦ ὁ καρπός σου εὕρηται.

\vs{10}Τίς σοφὸς καὶ συνήσει ταῦτα; ἢ συνετὸς καὶ ἐπιγνώσεται αὐτά; ὅτι εὐθεῖαι αἱ ὁδοὶ τοῦ Κυρίου, καὶ δίκαιοι πορεύσονται ἐν αὐταῖς, οἱ δὲ ἀσεβεῖς ἀσθενήσουσιν ἐν αὐταῖς.


\end{multicols}
\chapter{ΙΩΗΛ}
\begin{multicols}{2}

\ch{1}
ΛΟΓΟΣ Κυρίου ὃς ἐγενήθη πρὸς Ἰωὴλ τὸν τοῦ Βαθουήλ.

\vs{2}Ἀκούσατε ταῦτα, οἱ πρεσβύτεροι, καὶ ἑνωτίσασθε, πάντες οἱ κατοικοῦντες τὴν γῆν. εἰ γέγονεν τοιαῦτα ἐν ταῖς ἡμέραις ἡμῶν, ἢ ἐν ταῖς ἡμέραις τῶν πατέρων ὑμῶν;
\vs{3}ὑπὲρ αὐτῶν τοῖς τέκνοις ὑμῶν διηγήσασθε, καὶ τὰ τέκνα ὑμῶν τοῖς τέκνοις αὐτῶν, καὶ τὰ τέκνα αὐτῶν εἰς γενεὰν ἑτέραν.
\vs{4}τὰ κατάλοιπα τῆς κάμπης κατέφαγεν ἡ ἀκρίς, καὶ τὰ κατάλοιπα τῆς ἀκρίδος κατέφαγεν ὁ βροῦχος, καὶ τὰ κατάλοιπα τοῦ βρούχου κατέφαγεν ἡ ἐρυσίβη.

\vs{5}Ἐκνήψατε, οἱ μεθύοντες, ἐξ οἴνου αὐτῶν καὶ κλαύσατε· θρηνήσατε πάντες οἱ πίνοντες οἶνον εἰς μέθην, ὅτι ἐξῄρθη ἐκ στόματος ὑμῶν εὐφροσύνη καὶ χαρά.
\vs{6}Ὅτι ἔθνος ἀνέβη ἐπὶ τὴν γῆν μου ἰσχυρὸν καὶ ἀναρίθμητον, οἱ ὀδόντες αὐτοῦ ὀδόντες λέοντος, καὶ αἱ μῦλαι αὐτοῦ σκύμνου·
\vs{7}Ἔθετο τὴν ἄμπελόν μου εἰς ἀφανισμὸν, καὶ τὰς συκάς μου εἰς συγκλασμόν· ἐρευνῶν ἐξηρεύνησεν αὐτὴν, καὶ ἔῤῥιψεν· ἐλεύκανε τὰ κλήματα αὐτῆς.

\vs{8}Θρήνησον πρὸς μὲ ὑπὲρ νύμφην περιεζωσμένην σάκκον, ἐπὶ τὸν ἄνδρα αὐτῆς τὸν παρθενικόν.
\vs{9}Ἐξῇρται θυσία καὶ σπονδὴ ἐξ οἴκου Κυρίου· πενθεῖτε οἱ ἱερεῖς οἱ λειτουργοῦντες θυσιαστηρίῳ Κυρίου,
\vs{10}ὅτι τεταλαιπώρηκε τὰ πεδία· πενθείτω ἡ γῆ, ὅτι τεταλαιπώρηκεν σῖτος· ἐξηράνθη οἶνος, ὠλιγώθη ἔλαιον·
\vs{11}Ἐξηράνθησαν γεωργοί· θρηνεῖτε κτήματα ὑπὲρ πυροῦ καὶ κριθῆς, ὅτι ἀπόλωλε τρυγητὸς ἐξ ἀγροῦ·
\vs{12}Ἡ ἄμπελος ἐξηράνθη, καὶ αἱ συκαῖ ὠλιγώθησαν· ῥοὰ, καὶ φοῖνιξ, καὶ μῆλον, καὶ πάντα τὰ ξύλα τοῦ ἀγροῦ ἐξηράνθησαν, ὅτι ᾔσχυναν χαρὰν οἱ υἱοὶ τῶν ἀνθρώπων.

\vs{13}Περιζώσασθε καὶ κόπτεσθε οἱ ἱερεῖς· θρηνεῖτε οἱ λειτουργοῦντες θυσιαστηρίῳ· εἰσέλθετε, ὑπνώσατε ἐν σάκκοις λειτουργοῦντες Θεῷ, ὅτι ἀπέσχηκεν ἐξ οἴκου Θεοῦ ὑμῶν θυσία καὶ σπονδή.

\vs{14}Ἁγιάσατε νηστείαν, κηρύξατε θεραπείαν, συναγάγετε πρεσβυτέρους, πάντας κατοικοῦντας γῆν εἰς οἶκον Θεοῦ ὑμῶν, καὶ κεκράξετε πρὸς Κύριον ἐκτενῶς,

\vs{15}Οἴμοι, οἴμοι, οἴμοι εἰς ἡμέραν, ὅτι ἐγγὺς ἡ ἡμέρα Κυρίου, καὶ ὡς ταλαιπωρία ἐκ ταλαιπωρίας ἥξει.
\vs{16}Κατέναντι τῶν ὀφθαλμῶν ὑμῶν βρώματα ἐξωλθρεύθη, ἐξ οἴκου Θεοῦ ὑμῶν εὐφροσύνη καὶ χαρά.
\vs{17}Ἐσκίρτησαν δαμάλεις ἐπὶ ταῖς φάτναις αὐτῶν, ἠφανίσθησαν θησαυροὶ, κατεσκάφησαν ληνοὶ, ὅτι ἐξηράνθη σῖτος.
\vs{18}Τί ἀποθήσομεν ἑαυτοῖς; ἔκλαυσαν βουκόλια βοῶν, ὅτι οὐχ ὑπῆρχε νομὴ αὐτοῖς· καὶ τὰ ποίμνια τῶν προβάτων ἠφανίσθησαν.
\vs{19}Πρὸς σὲ Κύριε βοήσομαι, ὅτι πῦρ ἀνήλωσε τὰ ὡραῖα τῆς ἐρήμου, καὶ φλὸξ ἀνῆψε πάντα τὰ ξύλα τοῦ ἀγροῦ,
\vs{20}καὶ τὰ κτήνη τοῦ πεδίου ἀνέβλεψαν πρὸς σὲ, ὅτι ἐξηράνθησαν ἀφέσεις ὑδάτων, καὶ πῦρ κατέφαγε τὰ ὡραῖα τῆς ἐρήμου.

\ch{2}
Σαλπίσατε σάλπιγγι ἐν Σιὼν, κηρύξατε ἐν ὄρει ἁγίῳ μου, καὶ συγχυθήτωσαν πάντες οἱ κατοικοῦντες τὴν γῆν, διότι πάρεστιν ἡμέρα Κυρίου, ὅτι ἐγγὺς
\vs{2}ἡμέρα σκότους καὶ γνόφου, ἡμέρα νεφέλης καὶ ὁμίχλης· ὡς ὄρθρος χυθήσεται ἐπὶ τὰ ὄρη λαὸς πολὺς καὶ ἰσχυρὸς, ὅμοιος αὐτῷ οὐ γέγονεν ἀπὸ τοῦ αἰῶνος, καὶ μετʼ αὐτὸν οὐ προστεθήσεται ἕως ἐτῶν εἰς γενεὰς γενεῶν.
\vs{3}Τὰ ἔμπροσθεν αὐτοῦ πῦρ ἀναλίσκον, καὶ τὰ ὀπίσω αὐτοῦ ἀναπτομένη φλόξ· ὡς παράδεισος τρυφῆς ἡ γῆ πρὸ προσώπου αὐτοῦ, καὶ τὰ ὄπισθεν αὐτοῦ πεδίον ἀφανισμοῦ, καὶ ἀνασωζόμενος οὐκ ἔσται αὐτῷ.

\vs{4}Ὡς ὅρασις ἵππων ἡ ὅρασις αὐτῶν, καὶ ὡς ἱππεῖς οὕτως καταδιώξονται.
\vs{5}Ὡς φωνὴ ἁρμάτων ἐπὶ τὰς κορυφὰς τῶν ὀρέων ἐξαλοῦνται, καὶ ὡς φωνὴ φλογὸς πυρὸς κατεσθιούσης καλάμην, καὶ ὡς λαὸς πολὺς καὶ ἰσχυρὸς παρατασσόμενος εἰς πόλεμον.
\vs{6}Ἀπὸ προσώπου αὐτοῦ συντριβήσονται λαοὶ, πᾶν πρόσωπον ὡς πρόσκαυμα χύτρας.
\vs{7}Ὡς μαχηταὶ δραμοῦνται καὶ ὡς ἄνδρες πολεμισταὶ ἀναβήσονται ἐπὶ τὰ τείχη, καὶ ἕκαστος ἐν τῇ ὁδῷ αὐτοῦ πορεύσεται, καὶ οὐ μὴ ἐκκλίνωσι τὰς τρίβους αὐτῶν,
\vs{8}καὶ ἕκαστος ἀπὸ τοῦ ἀδελφοῦ αὐτοῦ οὐκ ἀφέξεται· καταβαρυνόμενοι ἐν τοῖς ὅπλοις αὐτῶν πορεύσονται, καὶ ἐν τοῖς βέλεσιν αὐτῶν πεσοῦνται, καὶ οὐ μὴ συντελεσθῶσι.
\vs{9}Τῆς πόλεως ἐπιλήψονται, καὶ ἐπὶ τῶν τειχέων δραμοῦνται, καὶ ἐπὶ ταῖς οἰκίαις ἀναβήσονται, καὶ διὰ θυρίδων εἰσελεύσονται ὡς κλέπται.
\vs{10}Πρὸ προσώπου αὐτοῦ συγχυθήσεται ἡ γῆ, καὶ σεισθήσεται ὁ οὐρανός· ὁ ἥλιος καὶ ἡ σελήνη συσκοτάσουσι, καὶ ἄστρα δύσουσι τὸ φέγγος αὐτῶν.
\vs{11}Καὶ Κύριος δώσει φωνὴν αὐτοῦ πρὸ προσώπου δυνάμεως αὐτοῦ, ὅτι πολλή ἐστι σφόδρα ἡ παρεμβολὴ αὐτοῦ, ὅτι ἰσχυρὰ ἔργα λόγων αὐτοῦ· διότι μεγάλη ἡ ἡμέρα Κυρίου, ἐπιφανὴς σφόδρα, καὶ τίς ἔσται ἱκανὸς αὐτῇ;

\vs{12}Καὶ νῦν λέγει Κύριος ὁ Θεὸς ὑμῶν, ἐπιστράφητε πρὸς μὲ ἐξ ὅλης τῆς καρδίας ὑμῶν, καὶ ἐν νηστείᾳ, καὶ ἐν κλαυθμῷ, καὶ ἐν κοπετῷ,
\vs{13}καὶ διαῤῥήξατε τὰς καρδίας ὑμῶν, καὶ μὴ τὰ ἱμάτια ὑμῶν· καὶ ἐπιστράφητε πρὸς Κύριον τὸν Θεὸν ὑμῶν, ὅτι ἐλεήμων καὶ οἰκτίρμων ἐστὶ, μακρόθυμος καὶ πολυέλεος, καὶ μετανοῶν ἐπὶ ταῖς κακίαις.
\vs{14}Τίς οἶδεν εἰ ἐπιστρέψει, καὶ μετανοήσει, καὶ ὑπολείψεται ὀπίσω αὐτοῦ εὐλογίαν, καὶ θυσίαν, καὶ σπονδὴν Κυρίῳ τῷ Θεῷ ὑμῶν;

\vs{15}Σαλπίσατε σάλπιγγι ἐν Σιὼν, ἁγιάσατε νηστείαν. κηρύξατε θεραπείαν,
\vs{16}συναγάγετε λαὸν, ἁγιάσατε ἐκκλησίαν, ἐκλέξασθε πρεσβυτέρους, συναγάγετε νήπια θηλάζοντα μαστοὺς, ἐξελθέτω νυμφίος ἐκ τοῦ κοιτῶνος αὐτοῦ, καὶ νύμφη ἐκ τοῦ παστοῦ αὐτῆς.
\vs{17}Ἀναμέσον τῆς κρηπίδος τοῦ θυσιαστηρίου, κλαύσονται οἱ ἱερεῖς οἱ λειτουργοῦντες τῷ Κυρίῳ, καὶ ἐροῦσι, φεῖσαι Κύριε, τοῦ λαοῦ σου, καὶ μὴ δῷς τὴν κληρονομίαν σου εἰς ὄνειδος, τοῦ κατάρξαι αὐτῶν ἔθνη, ὅπως μὴ εἴπωσιν ἐν τοῖς ἔθνεσι, ποῦ ἐστιν ὁ Θεὸς αὐτῶν;

\vs{18}Καὶ ἐζήλωσε Κύριος τὴν γῆν αὐτοῦ, καὶ ἐφείσατο τοῦ λαοῦ αὐτοῦ.
\vs{19}Καὶ ἀπεκρίθη Κύριος, καὶ εἶπε τῷ λαῷ αὐτοῦ, ἰδοὺ ἐγὼ ἐξαποστέλλω ὑμῖν τὸν σῖτον καὶ τὸν οἶνον καὶ τὸ ἔλαιον, καὶ ἐμπλησθήσεσθε αὐτῶν, καὶ οὐ δώσω ὑμᾶς οὐκ ἔτι εἰς ὀνειδισμὸν ἐν τοῖς ἔθνεσι.
\vs{20}Καὶ τὸν ἀπὸ Βοῤῥᾶ ἐκδιώξω ἀφʼ ὑμῶν, καὶ ἐξώσω αὐτὸν εἰς γῆν ἄνυδρον, καὶ ἀφανιῶ τὸ πρόσωπον αὐτοῦ εἰς τὴν θάλασσαν τὴν πρώτην, καὶ τὰ ὀπίσω αὐτοῦ εἰς τὴν θάλασσαν τὴν ἐσχάτην· καὶ ἀναβήσεται ἡ σαπρία αὐτοῦ, καὶ ἀναβήσεται ὁ βρόμος αὐτοῦ, ὅτι ἐμεγάλυνε τὰ ἔργα αὐτοῦ.

\vs{21}Θάρσει γῆ, χαῖρε καὶ εὐφραίνου, ὅτι ἐμεγάλυνε Κύριος τοῦ ποιῆσαι.
\vs{22}Θαρσεῖτε κτήνη τοῦ πεδίου, ὅτι βεβλάστηκε τὰ πεδία τῆς ἐρήμου, ὅτι ξύλον ἤνεγκεν τὸν καρπὸν αὐτοῦ, συκῆ καὶ ἄμπελος ἔδωκαν τὴν ἰσχὺν αὐτῶν.
\vs{23}Καὶ τὰ τέκνα Σιὼν χαίρετε καὶ εὐφραίνεσθε ἐπὶ τῷ Κυρίῳ Θεῷ ὑμῶν, διότι ἔδωκεν ὑμῖν τὰ βρώματα εἰς δικαιοσύνην, καὶ βρέξει ὑμῖν ὑετὸν πρώϊμον καὶ ὄψιμον, καθὼς ἔμπροσθεν,
\vs{24}καὶ πλησθήσονται αἱ ἅλωνες σίτου, καὶ ὑπερχυθήσονται αἱ ληνοὶ οἴνου καὶ ἐλαίου.
\vs{25}Καὶ ἀνταποδώσε ὑμῖν ἀντὶ τῶν ἐτῶν ὧν κατέφαγεν ἡ ἀκρὶς, καὶ ὁ βροῦχος, καὶ ἡ ἐρυσίβη, καὶ ἡ κάμπη, ἡ δύναμίς μου ἡ μεγάλη, ἣν ἐξαπέστειλα εἰς ὑμᾶς.
\vs{26}Καὶ φάγεσθε ἐσθίοντες, καὶ ἐμπλησθήσεσθε, καὶ αἰνέσετε τὸ ὄνομα Κυρίου τοῦ Θεοῦ ὑμῶν, ἃ ἐποίησε μεθʼ ὑμῶν εἰς θαυμάσια· καὶ οὐ μὴ καταισχυνθῇ ὁ λαός μου εἰς τὸν αἰῶνα.
\vs{27}Καὶ ἐπιγνώσεσθε ὅτι ἐν μέσῳ τοῦ Ἰσραὴλ ἐγώ εἰμι, καὶ ἐγὼ Κύριος ὁ Θεὸς ὑμῶν, καὶ οὐκ ἕστιν ἔτι πλὴν ἐμοῦ· καὶ οὐ μὴ καταισχυνθῶσιν ἔτι ὁ λαός μου εἰς τὸν αἰῶνα.

\ch{3}
Καὶ ἔσται μετὰ ταῦτα, καὶ ἐκχεῶ ἀπὸ τοῦ πνεύματός μου ἐπὶ πᾶσαν σάρκα, καὶ προφητεύσουσιν οἱ υἱοὶ ὑμῶν, καὶ αἱ θυγατέρες ὑμῶν, καὶ οἱ πρεσβύτεροι ὑμῶν ἐνύπνια ἐνυπνιασθήσονται, καὶ οἱ νεανίσκοι ὑμῶν ὁράσεις ὄψονται.
\vs{2}Καὶ ἐπὶ τοὺς δούλους μου καὶ ἐπὶ τὰς δούλας ἐν ταῖς ἡμέραις ἐκείναις ἐκχεῶ ἀπὸ τοῦ πνεύματός μου.
\vs{3}Καὶ δώσω τέρατα ἐν οὐρανῷ, καὶ ἐπὶ τῆς γῆς αἷμα καὶ πῦρ καὶ ἀτμίδα καπνοῦ.
\vs{4}Ὁ ἥλιος μεταστραφήσεται εἰς σκότος, καὶ ἡ σελήνη εἰς αἷμα, πρὶν ἐλθεῖν τὴν ἡμέραν Κυρίου τὴν μεγάλην, καὶ ἐπιφανῆ.

\vs{5}Καὶ ἔσται πᾶς ὃς ἂν ἐπικαλέσηται τὸ ὄνομα Κυρίου, σωθήσεται· ὅτι ἐν τῷ ὄρει Σιὼν καὶ ἐν Ἱερουσαλὴμ ἔσται ἀνασωζόμενος καθότι εἶπε Κύριος, καὶ εὐαγγελιζόμενοι οὓς Κύριος προσκέκληται.

\ch{4}
Ὅτι ἰδοὺ ἐγὼ ἐν ταῖς ἡμέραις ἐκείναις καὶ ἐν τῷ καιρῷ ἐκείνῳ, ὅταν ἐπιστρέψω τὴν αἰχμαλωσίαν Ἰούδα καὶ Ἱερουσαλὴμ,
\vs{2}καὶ συνάξω πάντα τὰ ἔθνη, καὶ κατάξω αὐτὰ εἰς τὴν κοιλάδα Ἰωσαφὰτ, καὶ διακριθήσομαι πρὸς αὐτοὺς ἐκεῖ ὑπὲρ τοῦ λαοῦ μου καὶ τῆς κληρονομίας μου Ἰσραὴλ, οἳ διεσπάρησαν ἐν τοῖς ἔθνεσι, καὶ τὴν γῆν μου κατεδιείλαντο,
\vs{3}καὶ ἐπὶ τὸν λαόν μου ἔβαλον κλήρους, καὶ ἔδωκαν τὰ παιδάρια πόρναις, καὶ τὰ κοράσια ἐπώλουν ἀντὶ τοῦ οἴνου, καὶ ἔπινον.

\vs{4}Καὶ τί ὑμεῖς ἐμοὶ Τύρος, καὶ Σιδὼν, καὶ πᾶσα Γαλιλαία ἀλλοφύλων; μὴ ἀνταπόδομα ὑμεῖς ἀνταποδίδοτέ μοι; ἢ μνησικακεῖτε ὑμεῖς ἐπʼ ἐμοί; ὀξέως, καὶ ταχέως ἀνταποδώσω τὸ ἀνταπόδομα ὑμῶν εἰς κεφαλὰς ὑμῶν,
\vs{5}ἀνθʼ ὧν τὸ ἀργύριόν μου καὶ τὸ χρυσίον μου ἐλάβετε, καὶ τὰ ἐπίλεκτά μου τὰ καλὰ εἰσηνέγκατε εἰς τοὺς ναοὺς ὑμῶν,
\vs{6}καὶ τοὺς υἱοὺς Ἰούδα καὶ τοὺς υἱοὺς Ἱερουσαλὴμ ἀπέδοσθε τοῖς υἱοῖς τῶν Ἑλλήνων, ὅπως ἐξώσητε αὐτοὺς ἐκ τῶν ὁρίων αὐτῶν.
\vs{7}Καὶ ἰδοὺ ἐγὼ ἐξεγείρω αὐτοὺς ἐκ τοῦ τόπου οὗ ἀπέδοσθε αὐτοὺς ἐκεῖ, καὶ ἀνταποδώσω τὸ ἀνταπόδομα ὑμῶν εἰς κεφαλὰς ὑμῶν,
\vs{8}καὶ ἀποδώσομαι τοὺς υἱοὺς ὑμῶν καὶ τὰς θυγατέρας ὑμῶν εἰς χεῖρας τῶν υἱῶν Ἰούδα, καὶ ἀποδώσονται αὐτοὺς εἰς αἰχμαλωσίαν εἰς ἔθνος μακρὰν ἀπέχον, ὅτι Κύριος ἐλάλησε.

\vs{9}Κηρύξατε ταῦτα ἐν τοῖς ἔθνεσιν, ἁγιάσατε πόλεμον, ἐξεγείρατε τοὺς μαχητὰς, προσαγάγετε καὶ ἀναβαίνετε πάντες ἄνδρες πολεμισταὶ,
\vs{10}συγκόψατε τὰ ἄροτρα ὑμῶν εἰς ῥομφαίας, καὶ τὰ δρέπανα ὑμῶν εἰς σειρομάστας· ὁ ἀδύνατος λεγέτω, ὅτι ἰσχύω ἐγώ.
\vs{11}Συναθροίζεσθε, καὶ εἰσπορεύεσθε πάντα τὰ ἔθνη κυκλόθεν, καὶ συνάχθητε ἐκεῖ· ὁ πρᾳῢς ἔστω μαχητής.
\vs{12}Ἐξεγειρέσθωσαν, ἀναβαινέτωσαν πάντα τὰ ἔθνη εἰς τὴν κοιλάδα Ἰωσαφὰτ, διότι ἐκεῖ καθιῶ τοῦ διακρῖναι πάντα τὰ ἔθνη κυκλόθεν.

\vs{13}Ἐξαποστείλατε δρέπανα, ὅτι παρέστηκεν ὁ τρυγητός· εἰσπορεύεσθε, πατεῖτε, διότι πλήρης ἡ ληνός· ὑπερεκχεῖτε τὰ ὑπολήνια, ὅτι πεπλήθυνται τὰ κακὰ αὐτῶν.
\vs{14}Ἦχοι ἐξήχησαν ἐν τῇ κοιλάδι τῆς δίκης, ὅτι ἐγγὺς ἡμέρα Κυρίου ἐν τῇ κοιλάδι τῆς δίκης.
\vs{15}Ὁ ἥλιος καὶ ἡ σελήνη συσκοτάσουσι, καὶ οἱ ἀστέρες δύσουσι φέγγος αὐτῶν.

\vs{16}Ὁ δὲ Κύριος ἐκ Σιὼν ἀνακεκράξεται, καὶ ἐξ Ἱερουσαλὴμ δώσει φωνὴν αὐτοῦ, καὶ σεισθήσεται ὁ οὐρανὸς καὶ ἡ γῆ· ὁ δὲ Κύριος φείσεται τοῦ λαοῦ αὐτοῦ, καὶ ἐνισχύσει τοὺς υἱοὺς Ἰσραήλ.
\vs{17}Καὶ ἐπιγνώσεσθε διότι ἐγὼ Κύριος ὁ Θεὸς ὑμῶν, ὁ κατασκηνῶν ἐν Σιὼν ὄρει ἁγίῳ μου· καὶ ἔσται Ἱερουσαλὴμ ἁγία, καὶ ἀλλογενεῖς οὐ διελεύσονται διʼ αὐτῆς οὐκέτι.

\vs{18}Καὶ ἔσται ἐν τῇ ἡμέρᾳ ἐκείνῃ, ἀποσταλάξει τὰ ὄρη γλυκασμὸν, καὶ οἱ βουνοὶ ῥυήσονται γάλα, καὶ πᾶσαι αἱ ἀφέσεις Ἰούδα ῥυήσονται ὕδατα, καὶ πηγὴ ἐξ οἴκου Κυρίου ἐξελεύσεται, καὶ ποτιεῖ τὸν χειμάῤῥουν τῶν σχοίνων.

\vs{19}Αἴγυπτος εἰς ἀφανισμὸν ἔσται, καὶ ἡ Ἰδουμαία εἰς πεδίον ἀφανισμοῦ ἔσται, ἐξ ἀδικιῶν υἱῶν Ἰούδα, ἀνθʼ ὧν ἐξέχεαν αἷμα δίκαιον ἐν τῇ γῇ αὐτῶν.
\vs{20}Ἡ δὲ Ἰουδαία εἰς τὸν αἰῶνα κατοικηθήσεται, καὶ Ἱερουσαλὴμ εἰς γενεὰς γενεῶν.
\vs{21}Καὶ ἐκζητήσω τὸ αἷμα αὐτῶν, καὶ οὐ μὴ ἀθωώσω, καὶ Κύριος κατασκηνώσει ἐν Σιών.


\end{multicols}
\chapter{ΑΜΩΣ}
\begin{multicols}{2}

\ch{1}
ΛΟΓΟΙ Αμὼς οἳ ἐγένοντο ἐν Ἀκκαρεὶμ ἐν Θεκουὲ, οὓς εἶδεν ὑπὲρ Ἱερουσαλὴμ, ἐν ἡμέραις Ὀζίου βασιλέως Ἰούδα, καὶ ἐν ἡμέραις Ἱεροβοὰμ τοῦ Ἰωὰς βασιλέως Ἰσραὴλ, πρὸ δύο ἐτῶν τοῦ σεισμοῦ.

\vs{2}Καὶ εἶπε, Κύριος ἐκ Σιὼν ἐφθέγξατο, καὶ ἐξ Ἱερουσαλὴμ ἔδωκε φωνὴν αὐτοῦ· καὶ ἐπένθησαν αἱ νομαὶ τῶν ποιμένων, καὶ ἐξηράνθη ἡ κορυφὴ τοῦ Καρμήλου.

\vs{3}Καὶ εἶπε Κύριος, ἐπὶ ταῖς τρισὶν ἀσεβείαις Δαμασκοῦ, καὶ ἐπὶ ταῖς τέσσαρσιν οὐκ ἀποστραφήσομαι αὐτὸν, ἀνθʼ ὧν ἔπριζον πριοσι σιδηροῖς τὰς ἐν γαστρὶ ἐχούσας τῶν ἐν Γαλαάδ.
\vs{4}Καὶ ἀποστελῶ πῦρ εἰς τὸν οἶκον Ἀζαὴλ, καὶ καταφάγεται τὰ θεμέλια υἱοῦ Ἄδερ.
\vs{5}Καὶ συντρίψω μοχλοὺς Δαμασκοῦ, καὶ ἐξολοθρεύσω κατοικοῦντας ἐκ πεδίου Ὦν, καὶ κατακόψω φυλὴν ἐξ ἀνδρῶν Χαῤῥὰν, καὶ αἰχμαλωτευθήσεται λαὸς Συρίας ἐπίκλητος, λέγει Κύριος.

\vs{6}Τάδε λέγει Κύριος, ἐπὶ ταῖς τρισὶν ἀσεβείαις Γάζης, καὶ ἐπὶ ταῖς τέσσαρσιν οὐκ ἀποστραφήσομαι αὐτοὺς, ἕνεκεν τοῦ αἰχμαλωτεῦσαι αὐτοὺς αἰχμαλωσίαν τοῦ Σαλωμὼν, τοῦ συγκλεῖσαι εἰς τὴν Ἰδουμαίαν.
\vs{7}Καὶ ἐξαποστελῶ πῦρ ἐπὶ τὰ τείχη Γάζης, καὶ καταφάγεται τὰ θεμέλια αὐτῆς.
\vs{8}Καὶ ἐξολοθρεύσω κατοικοῦντας ἐξ Ἀζώτου, καὶ ἐξαρθήσεται φυλὴ ἐξ Ἀσκάλωνος, καὶ ἐπάξω τὴν χεῖρά μου ἐπὶ Ἀκκάρων, καὶ ἀπολοῦνται οἱ κατάλοιποι τῶν ἀλλοφύλων, λέγει Κύριος.

\vs{9}Τάδε λέγει Κύριος, ἐπὶ ταῖς τρισὶν ἀσεβείαις Τύρου, καὶ ἐπὶ ταῖς τέσσαρσιν οὐκ ἀποστραφήσομαι αὐτὴν, ἀνθʼ ὧν συνέκλεισαν αἰχμαλωσίαν τοῦ Σαλωμὼν εἰς τὴν Ἰδουμαίαν, καὶ οὐκ ἐμνήσθησαν διαθήκης ἀδελφῶν.
\vs{10}Καὶ ἐξαποστελῶ πῦρ ἐπὶ τὰ τείχη Τύρου, καὶ καταφάγεται τὰ θεμέλια αὐτῆς.

\vs{11}Τάδε λέγει Κύριος, ἐπὶ ταῖς τρισὶν ἀσεβείαις τῆς Ἰδουμαίας, καὶ ἐπὶ ταῖς τέσσαρσιν οὐκ ἀποστραφήσομαι αὐτοὺς, ἕνεκα τοῦ διῶξαι αὐτοὺς ἐν ῥομφαίᾳ τὸν ἀδελφὸν αὐτοῦ, καὶ ἐλυμήνατο μητέρα ἐπὶ γῆς, καὶ ἥρπασεν εἰς μαρτύριον φρίκην αὐτοῦ, καὶ τὸ ὅρμημα αὐτοῦ ἐφύλαξεν εἰς νῖκος.
\vs{12}Καὶ ἐξαποστελῶ πῦρ εἰς Θαμὰν, καὶ καταφάγεται θεμέλια τειχέων αὐτῆς.

\vs{13}Τάδε λέγει Κύριος, ἐπὶ ταῖς τρισὶν ἀσεβείαις υἱῶν Ἀμμὼν, καὶ ἐπὶ ταῖς τέσσαρσιν οὐκ ἀποστραφήσομαι αὐτὸν, ἀνθʼ ὧν ἀνέσχιζον τὰς ἐν γαστρὶ ἐχούσας τῶν Γαλααδιτῶν, ὅπως ἐνπλατύνωσι τὰ ὅρια ἑαυτῶν.
\vs{14}Καὶ ἀνάψω πῦρ ἐπὶ τεὶχη Ῥαββὰθ, καὶ καταφάγεται θεμέλια αὐτῆς μετὰ κραυγῆς ἐν ἡμέρᾳ πολέμου, καὶ σεισθήσεται ἐν ἡμέραις συντελείας αὐτῆς,
\vs{15}καὶ πορεύσονται οἱ βασιλεῖς αὐτῆς ἐν αἰχμαλωσίᾳ, οἱ ἱερεῖς αὐτῶν καὶ οἱ ἄρχοντες αὐτῶν ἐπιτοαυτὸ, λέγει Κύριος.

\ch{2}
Τάδε λέγει Κύριος, ἐπὶ ταῖς τρισὶν ἀσεβείαις Μωὰβ, καὶ ἐπὶ ταῖς τέσσαρσιν οὐκ ἀποστραφήσομαι αὐτὸν, ἀνθʼ ὧν κατέκαυσαν τὰ ὀστᾶ βασιλέως τῆς Ἰδουμαίας εἰς κονίαν.
\vs{2}Καὶ ἐξαποστελῶ πῦρ εἰς Μωὰβ, καὶ καταφάγεται τὰ θεμέλια τῶν πόλεων αὐτῆς, καὶ ἀποθανεῖται ἐν ἀδυναμίᾳ Μωὰβ μετὰ κραυγῆς καὶ μετὰ φωνῆς σάλπιγγος.
\vs{3}Καὶ ἐξολοθρεύσω κριτὴν ἐξ αὐτῆς, καὶ πάντας [ἄρχοντας] αὐτῆς ἀποκτενῶ μετʼ αὐτοῦ, λέγει Κύριος.

\vs{4}Τάδε λέγει Κύριος, ἐπὶ ταῖς τρισὶν ἀσεβείαις υἱῶν Ἰούδα, καὶ ἐπὶ ταῖς τέσσαρσιν οὐκ ἀποστραφήσουμαι αὐτὸν, ἕνεκα τοῦ ἀπώσασθαι αὐτοὺς τὸν νόμον τοῦ Κυρίου, καὶ τὰ προστάγματα αὐτοῦ οὐκ ἐφυλάξαντο, καὶ ἐπλάνησεν αὐτοὺς τὰ μάταια αὐτῶν ἃ ἐποίησαν, οἷς ἐξηκολούθησαν οἱ πατέρες αὐτῶν ὀπίσω αὐτῶν.
\vs{5}Καὶ ἐξαποστελῶ πῦρ ἐπὶ Ἰούδαν, καὶ καταφάγεται θεμέλια Ἱερουσαλήμ.

\vs{6}Τάδε λέγει Κύριος, ἐπὶ ταῖς τρισὶν ἀσεβείαις Ἰσραὴλ, καὶ ἐπὶ ταῖς τέσσαρσιν οὐκ ἀποστραφήσομαι αὐτὸν, ἀνθʼ ὧν ἀπέδοντο ἀργυρίου δίκαιον, καὶ πένητα ἕνεκεν ὑποδημάτων,
\vs{7}τὰ πατοῦντα ἐπὶ τὸν χοῦν τῆς γῆς, καὶ ἐκονδύλιζον εἰς κεφαλὰς πτωχῶν, καὶ ὁδὸν ταπεινῶν ἐξέκλιναν, καὶ υἱὸς καὶ πατὴρ αὐτοῦ εἰσεπορεύοντο πρὸς τὴν αὐτὴν παιδίσκην, ὅπως βεβηλῶσι τὸ ὄνομα τοῦ Θεοῦ αὐτῶν.
\vs{8}Καὶ τὰ ἱμάτια αὐτῶν δεσμεύοντες σχοινίοις, παραπετάσματα ἐποίουν ἐχόμενα τοῦ θυσιαστηρίου, καὶ οἶνον ἐκ συκοφαντιῶν ἔπινον ἐν τῷ οἴκῳ τοῦ Θεοῦ αὐτῶν.

\vs{9}Ἐγὼ δὲ ἐξῇρα τὸν Ἀμοῤῥαῖον ἐκ προσώπου αὐτῶν, οὗ ἦν, καθὼς ὕψος κέδρου, τὸ ὕψος αὐτοῦ, καὶ ἰσχυρὸς ἦν ὡς δρῦς, καὶ ἐξήρανα τὸν καρπὸν αὐτοῦ ἐπάνωθεν, καὶ τὰς ῥίζας αὐτοῦ ὑποκάτωθεν.
\vs{10}Καὶ ἐγὼ ἀνήγαγον ὑμᾶς ἐκ γῆς Αἰγύπτου, καὶ περιήγαγον ὑμᾶς ἐν τῇ ἐρήμῳ τεσσαράκοντα ἔτη, τοῦ κατακληρονομῆσαι τὴν γῆν τῶν Ἀμοῤῥαίων.
\vs{11}Καὶ ἔλαβον ἐκ τῶν υἱῶν ὑμῶν εἰς προφήτας, καὶ ἐκ τῶν νεανίσκων ὑμῶν εἰς ἁγιασμόν· μὴ οὐκ ἔστι ταῦτα υἱοὶ Ἰσραήλ; λέγει Κύριος.
\vs{12}Καὶ ἐποτίζετε τοὺς ἡγιασμένους οἶνον, καὶ τοῖς προφήταις ἐνετέλλεσθε λέγοντες, οὐ μὴ προφητεύσητε.

\vs{13}Διατοῦτο ἰδοὺ ἐγὼ κυλίω ὑποκάτω ὑμῶν, ὃν τρόπον κυλίεται ἡ ἅμαξα ἡ γέμουσα καλάμης.
\vs{14}Καὶ ἀπολεῖται φυγὴ ἐκ δρομέως, καὶ ὁ κραταιὸς οὐ μὴ κρατήσῃ τῆς ἰσχύος αὐτοῦ, καὶ ὁ μαχητὴς οὐ μὴ σώσῃ τὴν ψυχὴν αὐτοῦ.
\vs{15}Καὶ ὁ τοξότης οὐ μὴ ὑποστῇ καὶ ὁ ὀξὺς τοῖς ποσὶν αὐτοῦ οὐ μὴ διασωθῇ, καὶ ὁ ἱππεὺς οὐ μὴ σώσῃ τὴν ψυχὴν αὐτοῦ,
\vs{16}καὶ ὁ κραταιὸς οὐ μὴ εὑρήσει τὴν καρδίαν αὐτοῦ ἐν δυναστείαις, ὁ γυμνὸς διώξεται ἐν ἐκείνῃ τῇ ἡμέρᾳ, λέγει Κύριος.

\ch{3}
Ἀκούσατε τὸν λόγον τοῦτον, ὃν ἐλάλησε Κύριος ἐφʼ ὑμᾶς, οἶκος Ἰσραὴλ, καὶ κατὰ πάσης φυλῆς, ἧς ἀνήγαγον ἐκ γῆς Αἰγύπτου, λέγων,
\vs{2}πλὴν ὑμᾶς ἔγνων ἐκ πασῶν τῶν φυλῶν τῆς γῆς, διατοῦτο ἐκδικήσω ἐφʼ ὑμᾶς πάσας τὰς ἁμαρτίας ὑμῶν.

\vs{3}Εἰ πορεύσονται δύο ἐπιτοαυτὸ καθόλου, ἐὰν μὴ γνωρίσωσιν ἑαυτούς;
\vs{4}Εἰ ἐρεύξεται λέων ἐκ τοῦ δρυμοῦ αὐτοῦ θήραν οὐκ ἔχων; εἰ δώσει σκύμνος φωνὴν αὐτοῦ ἐκ τῆς μάνδρας αὐτοῦ καθόλου, ἐὰν μὴ ἁρπάσῃ τί;
\vs{5}Εἰ πεσεῖται ὄρνεον ἐπὶ τῆς γῆς ἄνευ ἰξευτοῦ; εἰ σχασθήσεται παγὶς ἐπὶ τῆς γῆς ἄνευ τοῦ συλλαβεῖν τί;
\vs{6}Εἰ φωνήσει σάλπιγξ ἐν πόλει, καὶ λαὸς οὐ πτοηθήσεται; εἰ ἔσται κακία ἐν πόλει ἣν Κύριος οὐκ ἐποίησε;
\vs{7}Διότι οὐ μὴ ποιήσῃ Κύριος ὁ Θεὸς πρᾶγμα ἐὰν μὴ ἀποκαλύψῃ παιδείαν πρὸς τοὺς δούλους αὐτοῦ τοὺς προφήτας.
\vs{8}Λέων ἐρεύξεται, καὶ τίς οὐ φοβηθήσεται; Κύριος ὁ Θεὸς ἐλάλησε, καὶ τίς οὐ προφητεύσει;

\vs{9}Ἀναγγείλατε χώραις ἐν Ἀσσυρίοις, καὶ ἐπὶ τὰς χώρας τῆς Αἰγύπτου, καὶ εἴπατε, συνάχθητε ἐπὶ τὸ ὄρος Σαμαρείας, καὶ ἴδετε θαυμαστὰ πολλὰ ἐν μέσῳ αὐτῆς, καὶ καταδυναστείαν τὴν ἐν αὐτῇ.
\vs{10}Καὶ οὐκ ἔγνω ἃ ἔσται ἐναντίον αὐτῆς, λέγει Κύριος, οἱ θησαυρίζοντες ἀδικίαν καὶ ταλαιπωρίαν ἐν ταῖς χώραις αὐτῶν.
\vs{11}Διατοῦτο τάδε λέγει Κύριος ὁ Θεὸς, Τύρος κυκλόθεν ἡ γῆ σου ἐρημωθήσεται, καὶ κατάξει ἐκ σοῦ ἰσχύν σου, καὶ διαρπαγήσονται αἱ χῶραί σου.
\vs{12}Τάδε λέγει Κύριος, ὃν τρόπον ὅταν ἐκσπάσῃ ὁ ποιμὴν ἐκ στόματος τοῦ λέοντος δύο σκέλη ἢ λοβὸν ὠτίου, οὕτως ἐκσπασθήσονται οἱ υἱοὶ Ἰσραὴλ οἱ κατοικοῦντες ἐν Σαμαρείᾳ κατέναντι τῆς φυλῆς, καὶ ἐν Δαμασκῷ.

\vs{13}Ἰερεῖς ἀκούσατε, καὶ ἐπιμαρτύρασθε τῷ οἴκῳ Ἰακὼβ, λέγει Κύριος ὁ Θεὸς ὁ παντοκράτωρ.
\vs{14}Διότι ἐν τῇ ἡμέρᾳ ὅταν ἐκδικῶ ἀσεβείας τοῦ Ἰσραὴλ ἐπʼ αὐτὸν, καὶ ἐκδικήσω ἐπὶ τὰ θυσιαστήρια Βαιθήλ· καὶ κατασκαφήσεται τὰ κέρατα τοῦ θυσιαστηρίου, καὶ πεσοῦνται ἐπὶ τὴν γῆν·
\vs{15}Συγχεῶ καὶ πατάξω τὸν οἶκον τὸν περίπτερον ἐπὶ τὸν οἶκον τὸν θερινὸν, καὶ ἀπολοῦνται οἶκοι ἐλεφάντινοι, καὶ προστεθήσονται ἕτεροι οἶκοι πολλοὶ, λέγει Κύριος.

\ch{4}
Ἀκούσατε τὸν λόγον τοῦτον δαμάλεις τῆς Βασανίτιδος, αἱ ἐν τῷ ὄρει τῆς Σαμαρείας, αἱ καταδυναστεύουσαι πτωχοὺς, καὶ καταπατοῦσαι πένητας, αἱ λέγουσαι τοῖς κυρίοις αὐτῶν, ἐπίδοτε ἡμῖν ὅπως πίωμεν.

\vs{2}Ομνύει Κύριος κατὰ τῶν ἁγίων αὐτοῦ, διότι ἰδοὺ ἡμέραι ἔρχονται ἐφʼ ὑμᾶς, καὶ λήψονται ὑμᾶς ἐν ὅπλοις, καὶ τοὺς μεθʼ ὑμῶν εἰς λέβητας ὑποκαιομένους ἐμβαλοῦσιν ἔμπυροι λοιμοὶ,
\vs{3}καὶ ἐξενεχθήσεσθε γυμναὶ κατέναντι ἀλλήλων, καὶ ἀποῤῥιφήσεσθε εἰς τὸ ὄρος τὸ Ῥομμὰν, λέγει Κύριος.

\vs{4}Εἰσήλθατε εἰς Βαιθὴλ, καὶ ἠσεβήσατε, καὶ εἰς Γάλγαλα ἐπληθύνατε τοῦ ἀσεβῆσαι· καὶ ἠνέγκατε εἰς τοπρωῒ θυσίας ὑμῶν, εἰς τὴν τριημερίαν τὰ ἐπιδέκατα ὑμῶν.
\vs{5}Καὶ ἀνέγνωσαν ἔξω νόμον, καὶ ἐπεκαλέσαντο ὁμολογίας· ἀναγγείλατε ὅτι ταῦτα ἠγάπησαν οἱ υἱοὶ Ἰσραὴλ, λέγει Κύριος.

\vs{6}Καὶ ἐγὼ δώσω ὑμῖν γομφιασμὸν ὀδόντων ἐν πάσαις ταῖς πόλεσιν ὑμῶν, καὶ ἔνδειαν ἄρτων ἐν πᾶσι τοῖς τόποις ὑμῶν· καὶ οὐκ ἐπεστρέψατε πρὸς μὲ, λέγει Κύριος.
\vs{7}Καὶ ἐγὼ ἀνέσχον ἐξ ὑμῶν τὸν ὑετὸν πρὸ τριῶν μηνῶν τοῦ τρυγητοῦ, καὶ βρέξω ἐπὶ πόλιν μίαν, ἐπὶ δὲ πόλιν μίαν οὐ βρέξω· μερὶς μία βραχήσεται, καὶ μερὶς, ἐφʼ ἣν οὐ βρέξω, ξηρανθήσεται.
\vs{8}Καὶ συναθροισθήσονται δύο καὶ τρεῖς πόλεις εἰς πόλιν μίαν τοῦ πιεῖν ὕδωρ, καὶ οὐ μὴ ἐμπλησθῶσι· καὶ οὐκ ἐπεστράφητε πρὸς μὲ, λέγει Κύριος.
\vs{9}Ἐπάταξα ὑμᾶς ἐν πυρώσει, καὶ ἐν ἰκτέρῳ· ἐπληθύνατε κήπους ὑμῶν, ἀμπελῶνας ὑμῶν, καὶ συκεῶνας ὑμῶν· καὶ ἐλαιῶνας ὑμῶν κατέφαγεν ἡ κάμπη· καὶ οὐδʼ ὡς ἐπεστρέψατε πρὸς μὲ, λέγει Κύριος.
\vs{10}Ἐξαπέστειλα εἰς ὑμᾶς θάνατον ἐν ὁδῷ Αἰγύπτου, καὶ ἀπέκτεινα ἐν ῥομφαίᾳ τοὺς νεανίσκους ὑμῶν, μετὰ αἰχμαλωσίας ἵππων σου, καὶ ἀνήγαγον ἐν πυρὶ τὰς παρεμβολὰς ἐν τῇ ὀργῇ ὑμῶν· καὶ οὐδʼ ὡς ἐπεστρέψατε πρὸς μὲ, λέγει Κύριος.
\vs{11}Κατέστρεψα ὑμᾶς, καθὼς κατέστρεψεν ὁ Θεὸς Σόδομα καὶ Γόμοῤῥα, καὶ ἐγένεσθε ὡς δαλὸς ἐξέσπασμένος ἐκ πυρός· καὶ οὐδʼ ὡς ἐπεστρέψατε πρὸς μὲ, λέγει Κύριος.

\vs{12}Διατοῦτο οὕτως ποιήσω σοι Ἰσραήλ· πλὴν ὅτι οὕτως ποιήσω σοι, ἑτοιμάζου τοῦ ἐπικαλεῖσθαι τὸν Θεόν σου Ἰσραήλ.
\vs{13}Διότι ἰδοὺ ἐγὼ στερεῶν βροντὴν, καὶ κτίζων πνεῦμα, καὶ ἀπαγγέλλων εἰς ἀνθρώπους τὸν χριστὸν αὐτοῦ, ποιῶν ὄρθρον καὶ ὁμίχλην, καὶ ἐπιβαίνων ἐπὶ τὰ ὑψηλὰ τῆς γῆς· Κύριος ὁ Θεὸς ὁ παντοκράτωρ ὄνομα αὐτῷ.

\ch{5}
Ἀκούσατε τὸν λόγον Κυρίου τοῦτον, ὃν ἐγὼ λαμβάνω ἐφʼ ὑμᾶς, θρῆνον. Οἶκος Ἰσραὴλ
\vs{2}ἔπεσεν, οὐκέτι μὴ προσθήσει τοῦ ἀναστῆναι. Παρθένος τοῦ Ἰσραὴλ ἔσφαλεν ἐπὶ τῆς γῆς αὐτοῦ, οὐκ ἔστιν ὁ ἀναστήσων αὐτήν.
\vs{3}Διατοῦτο τάδε λέγει Κύριος Κύριος, ἡ πόλις ἐξ ἧς ἐξεπορεύοντο χίλιοι, ὑπολειφθήσονται ἑκατόν· καὶ ἐξ ἧς ἐξεπορεύοντο ἑκατὸν, ὑπολειφθήσονται δέκα τῷ οἴκῳ Ἰσραήλ.

\vs{4}Διότι τάδε λέγει Κύριος πρὸς τὸν οἶκον Ἰσραὴλ, ἐκζητήσατέ με, καὶ ζήσεσθε.
\vs{5}Καὶ μὴ ἐκζητεῖτε Βαιθὴλ, καὶ εἰς Γάλγαλα μὴ εἰσπορεύεσθε, καὶ ἐπὶ τὸ φρέαρ τοῦ ὅρκου μὴ διαβαίνετε, ὅτι Γάλγαλα αἰχμαλωτευομένη αἰχμαλωτευθήσεται, καὶ Βαιθὴλ ἔσται ὡς οὐχ ὑπάρχουσα.
\vs{6}Ἐκζητήσατε τὸν Κύριον, καὶ ζήσατε, ὅπως μὴ ἀναλάμψῃ ὡς πῦρ ὁ οἶκος Ἰωσὴφ, καὶ καταφάγῃ αὐτὸν, καὶ οὐκ ἔσται ὁ σβέσων τῷ οἴκῳ Ἰσραήλ.

\vs{7}Ὁ ποιῶν εἰς ὕψος κρίμα, καὶ δικαιοσύνην εἰς γῆν ἔθηκεν·
\vs{8}ὁ ποιῶν πάντα καὶ μετασκευάζων, καὶ ἐκτρέπων εἰς τοπρωῒ σκιάν, καὶ ἡμέραν εἰς νύκτα συσκοτάζων, ὁ προσκαλούμενος τὸ ὕδωρ τῆς θαλάσσης, καὶ ἐκχέων αὐτὸ ἐπὶ πρόσωπον τῆς γῆς· Κύριος ὄνομα αὐτῷ·
\vs{9}ὁ διαιρῶν συντριμμὸν ἐπὶ ἰσχὺν, καὶ ταλαιπωρίαν ἐπὶ ὀχύρωμα ἐπάγων.

\vs{10}Ἐμίσησαν ἐν πύλαις ἐλέγχοντα, καὶ λόγον ὅσιον ἐβδελύξαντο.
\vs{11}Διατοῦτο ἀνθʼ ὧν κατεκονδύλιζον πτωχοὺς, καὶ δῶρα ἐκλεκτὰ ἐδέξασθε παρʼ αὐτῶν, οἴκους ξεστοὺς ᾠκοδομήσατε, καὶ οὐ μὴ κατοικήσητε ἐν αὐτοῖς· ἀμπελῶνας ἐπιθυμητοὺς ἐφυτεύσατε, καὶ οὐ μὴ πίητε τὸν οἶνον αὐτῶν.
\vs{12}Ὅτι ἔγνων πολλὰς ἀσεβείας ὑμῶν, καὶ ἰσχυραὶ αἱ ἁμαρτίαι ὑμῶν, καταπατοῦντες δίκαιον, λαμβάνοντες ἀλλάγματα, καὶ πένητας ἐν πύλαις ἐκκλίνοντες.

\vs{13}Διατοῦτο ὁ συνιὼν ἐν τῷ καιρῷ ἐκείνῳ σιωπήσεται, ὅτι καιρὸς πονηρῶν ἐστιν.
\vs{14}Ἐκζητήσατε τὸ καλὸν, καὶ μὴ πονηρὸν, ὅπως ζήσητε, καὶ ἔσται οὕτως μεθʼ ὑμῶν Κύριος ὁ Θεὸς ὁ παντοκράτωρ, ὃν τρόπον εἴπατε,
\vs{15}μεμισήκαμεν τὰ πονηρὰ, καὶ ἠγαπήσαμεν τὰ καλά· καὶ ἀποκαταστήσατε ἐν πύλαις κρίμα, ὅπως ἐλεήσῃ Κύριος ὁ Θεὸς ὁ παντοκράτωρ τοὺς περιλοίπους τοῦ Ἰωσήφ.

\vs{16}Διατοῦτο τάδε λέγει Κύριος ὁ Θεὸς ὁ παντοκράτωρ, ἐν πάσαις ταῖς πλατείαις κοπετὸς, καὶ ἐν πάσαις ταῖς ὁδοῖς ῥηθήσεται οὐαὶ, οὐαί· κληθήσεται γεωργὸς εἰς πένθος καὶ κοπετὸν, καὶ εἰς εἰδότας θρῆνον.
\vs{17}Καὶ ἐν πάσαις ὁδοῖς κοπετὸς, διότι ἐλεύσομαι διὰ μέσου σου, εἶπε Κύριος.

\vs{18}Οὐαὶ οἱ ἐπιθυμοῦντες τὴν ἡμέραν Κυρίου· ἱνατί αὕτη ὑμῖν ἡ ἡμέρα τοῦ Κυρίου; καὶ αὕτη ἐστὶ σκότος καὶ οὐ φῶς.
\vs{19}Ὃν τρόπον ἐὰν φύγῃ ἄνθρωπος ἐκ προσώπου τοῦ λέοντος, καὶ ἐμπέσῃ αὐτῷ ἡ ἄρκος, καὶ εἰσπηδήσῃ εἰς τὸν οἶκον αὐτοῦ, καὶ ἀπερείσηται τὰς χεῖρας αὐτοῦ ἐπὶ τὸν τοῖχον, καὶ δάκῃ αὐτὸν ὄφις.
\vs{20}Οὐχὶ σκότος ἡ ἡμέρα τοῦ Κυρίου, καὶ οὐ φῶς, καὶ γνόφος οὐκ ἔχων φέγγος αὕτη;

\vs{21}Μεμίσηκα, ἀπῶσμαι ἑορτὰς ὑμῶν, καὶ οὐ μὴ ὀσφρανθῶ θυσίας ἐν ταῖς πανηγύρεσιν ὑμῶν.
\vs{22}Διότι ἐὰν ἐνέγκητέ μοι ὁλοκαυτώματα καὶ θυσίας ὑμῶν, οὐ προσδέξομαι, καὶ σωτηρίους ἐπιφανείας ὑμῶν οὐκ ἐπιβλέψομαι.
\vs{23}Μετάστησον ἀπʼ ἐμοῦ ἦχον ᾠδῶν σου, καὶ ψαλμὸν ὀργάνων σου οὐκ ἀκούσομαι.
\vs{24}Καὶ κυλισθήσεται ὡς ὕδωρ κρίμα, καὶ δικαιοσύνη ὡς χειμάῤῥους ἄβατος.
\vs{25}Μὴ σφάγια καὶ θυσίας προσηνέγκατέ μοι οἶκος Ἰσραὴλ τεσσαράκοντα ἔτη ἐν τῇ ἐρήμῳ;
\vs{26}Καὶ ἀνελάβετε τὴν σκηνὴν τοῦ Μολὸχ, καὶ τὸ ἄστρον τοῦ θεοῦ ὑμῶν Ῥαιφὰν, τοὺς τύπους αὐτῶν οὓς ἐποιήσατε ἑαυτοῖς.
\vs{27}Καὶ μετοικιῶ ὑμᾶς ἐπέκεινα Δαμασκοῦ, λέγει Κύριος· ὁ Θεὸς ὁ παντοκράτωρ ὄνομα αὐτῷ.

\ch{6}
Οὐαὶ τοῖς ἐξουθενοῦσι Σιὼν, καὶ τοῖς πεποιθόσιν ἐπὶ τὸ ὄρος Σαμαρείας, ἀπετρύγησαν ἀρχὰς ἐθνῶν, καὶ εἰσῆλθον αὐτοί. οἶκος τοῦ Ἰσραὴλ
\vs{2}διάβητε πάντες καὶ ἴδετε, καὶ διέλθατε ἐκεῖθεν εἰς Ἐματραββὰ, καὶ κατάβητε ἐκεῖθεν εἰς Γὲθ ἀλλοφύλων, τὰς κρατίστας ἐκ πασῶν τῶν βασιλειῶν τούτων, εἰ πλείονα τὰ ὅρια αὐτῶν ἐστι τῶν ὑμετέρων ὁρίων.

\vs{3}Οἱ ἐρχόμενοι εἰς ἡμέραν κακὴν, οἱ ἐγγίζοντες καὶ ἐφαπτόμενοι σαββάτων ψευδῶν,
\vs{4}οἱ καθεύδοντες ἐπὶ κλινῶν ἐλεφαντίνων, καὶ κατασπαταλῶντες ἐπὶ ταῖς στρωμναῖς αὐτῶν, καὶ ἔσθοντες ἐρίφους ἐκ ποιμνίων, καὶ μοσχάρια ἐκ μέσου βουκολίων γαλαθηνὰ,
\vs{5}οἱ ἐπικροτοῦντες πρὸς τὴν φωνὴν τῶν ὀργάνων, ὡς ἑστηκότα ἐλογίσαντο, καὶ οὐχ ὡς φεύγοντα,
\vs{6}οἱ πίνοντες τὸν διυλισμένον οἶνον, καὶ τὰ πρῶτα μῦρα χριόμενοι, καὶ οὐκ ἔπασχον οὐδὲν ἐπὶ τῇ συντριβῇ Ἰωσήφ.
\vs{7}Διὰ τοῦτο νῦν αἰχμάλωτοι ἔσονται ἀπʼ ἀρχῆς δυναστῶν, καὶ ἐξαρθήσεται χρεμετισμὸς ἵππων ἐξ Ἐφραίμ·

\vs{8}Ὅτι ὤμοσε Κύριος καθʼ ἑαυτοῦ, διότι βδελύσσομαι ἐγὼ πᾶσαν τὴν ὕβριν Ἰακὼβ, καὶ τὰς χώρας αὐτοῦ μεμίσηκα, καὶ ἐξαρῶ πόλιν σὺν πᾶσι τοῖς κατοικοῦσιν αὐτήν.

\vs{9}Καὶ ἔσται, ἐὰν ὑπολειφθῶσι δέκα ἄνδρες ἐν οἰκίᾳ μιᾷ, καὶ ἀποθανοῦνται, καὶ ὑπολειφθήσονται οἱ κατάλοιποι,
\vs{10}καὶ λήψονται οἱ οἰκεῖοι αὐτῶν, καὶ παραβιῶνται τοῦ ἐξενέγκαι τὰ ὀστᾶ αὐτῶν ἐκ τοῦ οἴκου· καὶ ἐρεῖ τοῖς προεστηκόσι τῆς οἰκίας, εἰ ἔτι ὑπάρχει παρὰ σοί; Καὶ ἐρεῖ, οὐκ ἔτι· καὶ ἐρεῖ, σίγα ἕνεκα τοῦ μὴ ὀνομάσαι τὸ ὄνομα Κυρίου.

\vs{11}Διότι ἰδοὺ Κύριος ἐντέλλεται, καὶ πατάξει τὸν οἶκον τὸν μέγαν θλάσμασι, καὶ τὸν οἶκον τὸν μικρὸν ῥάγμασιν.

\vs{12}Εἰ διώξονται ἐν πέτραις ἵπποι; εἰ παρασιωπήσονται ἐν θηλείαις; ὅτι ἐξεστρέψατε εἰς θυμὸν κρίμα, καὶ καρπὸν δικαιοσύνης εἰς πικρίαν,
\vs{13}οἱ εὐφραινόμενοι ἐπʼ οὐδενὶ λόγῳ, οἱ λέγοντες, οὐκ ἐν τῃ ἰσχύϊ ἡμῶν ἔσχομεν κέρατα;
\vs{14}Διότι ἰδοὺ ἐγὼ ἐπεγερῶ ἐφʼ ὑμᾶς οἶκος Ἰσραὴλ ἔθνος, λέγει Κύριος τῶν δυνάμεων, καὶ ἐκθλίψουσιν ὑμᾶς τοῦ μὴ εἰσελθεῖν εἰς Αἰμὰθ, καὶ ὡς τοῦ χειμάῤῥου τῶν δυσμῶν.

\ch{7}
Οὕτως ἔδειξέ μοι Κύριος ὁ Θεός· καὶ ἰδοὺ ἐπιγονὴ ἀκρίδων ἐρχομένη ἑωθινὴ, καὶ ἰδοὺ βροῦχος εἷς, Γὼγ ὁ βασιλεύς.
\vs{2}Καὶ ἔσται ἐὰν συντελέσῃ τοῦ καταφαγεῖν τὸν χόρτον τῆς γῆς, καὶ εἶπα, Κύριε Κύριε, ἵλεως γενοῦ· τίς ἀναστήσει τὸν Ἰακώβ; ὅτι ὀλιγοστός ἐστι.
\vs{3}Μετανόησον Κύριε ἐπὶ τούτῳ. Καὶ τοῦτο οὐκ ἔσται, λέγει Κύριος.

\vs{4}Οὕτως ἔδειξέ μοι Κύριος· καὶ ἰδοὺ ἐκάλεσε τὴν δίκην ἐν πυρὶ Κύριος, καὶ κατέφαγε τὴν ἄβυσσον τὴν πολλὴν, καὶ κατέφαγε τὴν μερίδα Κυρίου.
\vs{5}Καὶ εἶπα, Κύριε κόπασον δὴ, τίς ἀναστήσει τὸν Ἰακώβ; ὅτι ὀλιγοστός ἐστι.
\vs{6}Μετανόησον Κύριε ἐπὶ τούτῳ. Καὶ τοῦτο οὐ μὴ γένηται, λέγει Κύριος.

\vs{7}Οὕτως ἔδειξέ μοι Κύριος· καὶ ἰδοὺ ἑστηκὼς ἐπὶ τείχους ἀδαμαντίνου, καὶ ἐν τῇ χειρὶ αὐτοῦ ἀδάμας.
\vs{8}Καὶ εἶπε Κύριος πρὸς μὲ, τί σὺ ὁρᾷς Ἀμώς; καὶ εἶπα, ἀδάμαντα· καὶ εἶπε Κύριος πρὸς μὲ, ἰδοὺ ἐγὼ ἐντάσσω ἀδάμαντα ἐν μέσῳ λαοῦ μου Ἰσραὴλ, οὐκ ἔτι μὴ προσθῶ τοῦ παρελθεῖν αὐτόν.
\vs{9}Καὶ ἀφανισθήσονται βωμοὶ τοῦ γέλωτος, καὶ αἱ τελεταὶ τοῦ Ἰσραὴλ ἐρημωθήσονται, καὶ ἀναστήσομαι ἐπὶ τὸν οἶκον Ἱεροβοὰμ ἐν ῥομφαίᾳ.

\vs{10}Καὶ ἐξαπέστειλεν Ἀμασίας ὁ ἱερεὺς Βαιθὴλ πρὸς Ἱεροβοὰμ βασιλέα Ἰσραὴλ, λέγων, συστροφὰς ποιεῖται κατὰ σοῦ Ἀμὼς ἐν μέσῳ οἴκου Ἰσραὴλ, οὐ μὴ δύνηται ἡ γῆ ὑπενεγκεῖν πάντας τοὺς λόγους αὐτοῦ.
\vs{11}Διότι τάδε λέγει Ἀμὼς, ἐν ῥομφαίᾳ τελευτήσει Ἱεροβοὰμ, ὁ δὲ Ἰσραὴλ αἰχμάλωτος ἀχθήσεται ἀπὸ τῆς γῆς αὐτοῦ.

\vs{12}Καὶ εἶπεν Ἀμασίας πρὸς Ἀμὼς ὁ ὁρῶν βάδιζε, ἐκχῶρησον σὺ εἰς γῆν Ἰούδα, καὶ ἐκεῖ καταβίου, καὶ ἐκεῖ προφητεύσεις,
\vs{13}εἰς δὲ Βαιθὴλ οὐκ ἔτι προσθήσεις τοῦ προφητεῦσαι, ὅτι ἁγίασμα βασιλέως ἐστὶ, καὶ οἶκος βασιλείας ἐστί.

\vs{14}Καὶ ἀπεκρίθη Ἀμὼς καὶ εἶπε πρὸς Ἀμασίαν, οὐκ ἤμην προφήτης ἐγὼ, οὐδὲ υἱὸς προφήτου, ἀλλʼ ἢ αἰπόλος ἤμην, καὶ κνίζων συκάμινα·
\vs{15}καὶ ἀνέλαβέ με Κύριος ἐκ τῶν προβάτων, καὶ εἶπε Κύριος πρὸς μὲ, βάδιζε, καὶ προφήτευσον ἐπὶ τὸν λαόν μου Ἰσραήλ.
\vs{16}Καὶ νῦν ἄκουε λόγον Κυρίου· σὺ λέγεις, μὴ προφήτευε ἐπὶ τὸν Ἰσραὴλ, καὶ οὐ μὴ ὀχλαγωγήσῃς ἐπὶ τὸν οἶκον Ἰακώβ.
\vs{17}Διὰ τοῦτο τάδε λέγει Κύριος, ἡ γυνή σου ἐν τῇ πόλει πορνεύσει, καὶ οἱ υἱοί σου καὶ αἱ θυγατέρες σου ἐν ῥομφαίᾳ πεσοῦνται, καὶ ἡ γῆ σου ἐν σχοινίῳ καταμετρηθήσεται, καὶ σὺ ἐν γῇ ἀκαθάρτῳ τελευτήσεις, ὁ δὲ Ἰσραὴλ αἰχμάλωτος ἀχθήσεται ἀπὸ τῆς γῆς αὐτοῦ· οὕτως ἔδειξέ μοι Κύριος Κύριος.

\ch{8}
Καὶ ἰδοὺ ἄγγος ἰξευτοῦ. Καὶ εἶπε, τί σὺ βλέπεις Ἀμώς; καὶ εἶπα, ἄγγος ἰξευτοῦ·
\vs{2}καὶ εἶπε Κύριος πρὸς μὲ, ἥκει τὸ πέρας ἐπὶ τὸν λαόν μου Ἰσραὴλ, οὐ προσθήσω ἔτι τοῦ παρελθεῖν αὐτόν.
\vs{3}Καὶ ὀλολύξει τὰ φατνώματα τοῦ ναοῦ ἐν τῇ ἡμέρᾳ ἐκείνῃ, λέγει Κύριος Κύριος· πολὺς ὁ πεπτωκὼς ἐν παντὶ τόπῳ, ἐπιῤῥίψω σιωπήν.

\vs{4}Ἀκούσατε δὴ ταῦτα οἱ ἐκτρίβοντες εἰς τοπρωῒ πένητα, καὶ καταδυναστεύοντες πτωχοὺς ἀπὸ τῆς γῆς,
\vs{5}λέγοντες, πότε διελεύσεται ὁ μὴν, καὶ ἐμπολήσομεν, καὶ τὰ σάββατα, καὶ ἀνοίξομεν θησαυρὸν τοῦ ποιῆσαι μέτρον μικρὸν, καὶ τοῦ μεγαλῦναι στάθμιον, καὶ ποιῆσαι ζυγὸν ἄδικον,
\vs{6}τοῦ κτᾶσθαι ἐν ἀργυρίῳ καὶ πτωχοὺς, καὶ πένητα ἀντὶ ὑποδημάτων, καὶ ἀπὸ παντὸς γεννήματος ἐμπορευσόμεθα.
\vs{7}Ὀμνύει Κύριος κατὰ τῆς ὑπερηφανίας Ἰακὼβ, εἰ ἐπιλησθήσεται εἰς νῖκος πάντα τὰ ἔργα ὑμῶν,
\vs{8}Καὶ ἐπὶ τούτοις οὐ ταραχθήσεται ἡ γῆ, καὶ πενθήσει πᾶς ὁ κατοικῶν ἐν αὐτῇ; καὶ ἀναβήσεται ὡς ποταμὸς συντέλεια, καὶ καταβήσεται ὡς ποταμὸς Αἰγύπτου.

\vs{9}Καὶ ἔσται ἐν τῇ ἡμέρᾳ ἐκείνῃ, λέγει Κύριος Κύριος, δύσεται ὁ ἥλιος μεσημβρίας, καὶ συσκοτάσει ἐπὶ τῆς γῆς ἐν ἡμέρᾳ τὸ φῶς,
\vs{10}καὶ μεταστρέψω τὰς ἑορτὰς ὑμῶν εἰς πένθος, καὶ πάσας τὰς ᾠδὰς ὑμῶν εἰς θρῆνον, καὶ ἀναβιβῶ ἐπὶ πᾶσαν ὀσφὺν σάκκον, καὶ ἐπὶ πᾶσαν κεφαλὴν φαλάκρωμα· καὶ θήσομαι αὐτὸν ὡς πένθος ἀγαπητοῦ, καὶ τοὺς μετʼ αὐτοῦ ὡς ἡμέραν ὀδύνης.

\vs{11}Ἰδοὺ ἡμέραι ἔρχονται, λέγει Κύριος, καὶ ἐξαποστελῶ λιμὸν ἐπὶ τὴν γῆν, οὐ λιμὸν ἄρτων, οὐδὲ δίψαν ὕδατος, ἀλλὰ λιμὸν τοῦ ἀκοῦσαι τὸν λόγον Κυρίου.
\vs{12}Καὶ σαλευθήσονται ὕδατα ἀπὸ τῆς θαλάσσης ἕως θαλάσσης, καὶ ἀπὸ Βοῤῥᾶ ἕως ἀνατολῶν περιδραμοῦνται ζητοῦντες τὸν λόγον τοῦ Κυρίου, καὶ οὐ μὴ εὕρωσιν.
\vs{13}Ἐν τῇ ἡμέρᾳ ἐκείνῃ ἐκλείψουσιν αἱ παρθένοι αἱ καλαὶ, καὶ οἱ νεανίσκοι ἐν δίψει,
\vs{14}οἱ ὀμνύοντες κατὰ τοῦ ἱλασμοῦ Σαμαρείας, καὶ οἱ λέγοντες, ζῇ ὁ θεός σου Δὰν, καὶ ζῇ ὁ θεός σου Βηρσαβεέ· καὶ πεσοῦνται, καὶ οὐ μὴ ἀναστῶσιν ἔτι.

\ch{9}
Εἶδον τὸν Κύριον ἐφεστῶτα ἐπὶ τοῦ θυσιαστηρίου, καὶ εἶπε,

Πάταξον ἐπὶ τὸ ἱλαστήριον, καὶ σεισθήσεται τὰ πρόπυλα, καὶ διάκοψον εἰς κεφαλὰς πάντων· καὶ τοὺς καταλοίπους αὐτῶν ἐν ῥομφαίᾳ ἀποκτενῶ, οὐ μὴ διαφύγῃ ἐξ αὐτῶν φεύγων, καὶ οὐ μὴ διασωθῇ ἐξ αὐτῶν ἀνασωζόμενος.
\vs{2}Ἐὰν κατακρυβῶσιν εἰς ᾅδου, ἐκεῖθεν ἡ χείρ μου ἀνασπάσει αὐτούς· καὶ ἐὰν ἀναβῶσιν εἰς τὸν οὐρανὸν, ἐκεῖθεν κατάξω αὐτούς.
\vs{3}Ἐὰν ἐγκατακρυβῶσιν εἰς τὴν κορυφὴν τοῦ Καρμήλου, ἐκεῖθεν ἐξερευνήσω, καὶ λήψομαι αὐτούς· καὶ ἐὰν καταδύσωσιν ἐξ ὀφθαλμῶν μου εἰς τὰ βάθη τῆς θαλάσσης, ἐκεῖ ἐντελοῦμαι τῷ δράκοντι, καὶ δήξεται αὐτούς.
\vs{4}Καὶ ἐὰν πορευθῶσιν ἐν αἰχμαλωσίᾳ πρὸ προσώπου τῶν ἐχθρῶν αὐτῶν, ἐκεῖ ἐντελοῦμαι τῇ ῥομφαίᾳ, καὶ ἀποκτενεῖ αὐτούς· καὶ στηριῶ τοὺς ὀφθαλμούς μου ἐπʼ αὐτοὺς εἰς κακὰ, καὶ οὐκ εἰς ἀγαθά.

\vs{5}Καὶ Κύριος Κύριος ὁ Θεὸς ὁ παντοκράτωρ, ὁ ἐφαπτόμενος τῆς γῆς, καὶ σαλεύων αὐτὴν, καὶ πενθήσουσι πάντες οἱ κατοικοῦντες αὐτὴν, καὶ ἀναβήσεται ὡς ποταμὸς συντέλεια αὐτῆς, καὶ καταβήσεται ὡς ποταμὸς Αἰγύπτου·
\vs{6}Ὁ οἰκοδομῶν εἰς τὸν οὐρανὸν ἀνάβασιν αὐτοῦ, καὶ τὴν ἐπαγγελίαν αὐτοῦ ἐπὶ τῆς γῆς θεμελιῶν, ὁ προσκαλούμενος τὸ ὕδωρ τῆς θαλάσσης, καὶ ἐκχέων αὐτὸ ἐπὶ πρόσωπον τῆς γῆς· Κύριος παντοκράτωρ ὄνομα αὐτῷ.

\vs{7}Οὐχ ὡς υἱοὶ Αἰθιόπων ὑμεῖς ἐστὲ ἐμοὶ, υἱοὶ Ἰσραὴλ; λέγει Κύριος· οὐ τὸν Ἰσραὴλ ἀνήγαγον ἐκ γῆς Αἰγύπτου, καὶ τοὺς ἀλλοφύλους ἐκ Καππαδοκίας, καὶ τοὺς Σύρους ἐκ βόθρου;
\vs{8}Ἰδοὺ οἱ ὀφθαλμοὶ Κυρίου τοῦ Θεοῦ ἐπὶ τὴν βασιλείαν τῶν ἁμαρτωλῶν, καὶ ἐξαρῶ αὐτὴν ἀπὸ προσώπου τῆς γῆς· πλὴν ὅτι οὐκ εἰς τέλος ἐξαρῶ τὸν οἶκον Ἰακὼβ, λέγει Κύριος.
\vs{9}Διότι ἐγὼ ἐντέλλομαι, καὶ λικμήσω ἐν πᾶσι τοῖς ἔθνεσι τὸν οἶκον Ἰσραὴλ, ὃν τρόπον λικμᾶται ἐν τῷ λικμῷ, καὶ οὐ μὴ πέσῃ σύντριμμα ἐπὶ τὴν γῆν.
\vs{10}Ἐν ῥομφαίᾳ τελευτήσουσι πάντες ἁμαρτωλοὶ λαοῦ μου, οἱ λέγοντες, οὐ μὴ ἐγγίσῃ, οὐδὲ μὴ γένηται ἐφʼ ἡμᾶς τὰ κακά.

\vs{11}Ἐν τῇ ἡμέρᾳ ἐκείνῃ ἀναστήσω τὴν σκηνὴν Δαυὶδ τὴν πεπτωκυῖαν, καὶ ἀνοικοδομήσω τὰ πεπτωκότα αὐτῆς, καὶ τὰ κατεσκαμμένα αὐτῆς ἀναστήσω, καὶ ἀνοικοδομήσω αὐτὴν καθὼς αἱ ἡμέραι τοῦ αἰῶνος.
\vs{12}Ὅπως ἐκζητήσωσιν οἱ κατάλοιποι τῶν ἀνθρώπων καὶ πάντα τὰ ἔθνη, ἐφʼ οὓς ἐπικέκληται τὸ ὄνομά μου ἐπʼ αὐτοὺς, λέγει Κύριος ὁ ποιῶν πάντα ταῦτα.

\vs{13}Ἰδοὺ ἡμέραι ἔρχονται, λέγει Κύριος, καὶ καταλήψεται ὁ ἀμητὸς τὸν τρυγητὸν, καὶ περκάσει ἡ σταφυλὴ ἐν τῷ σπόρῳ, καὶ ἀποσταλάξει τὰ ὄρη γλυκασμὸν, καὶ πάντες οἱ βουνοὶ σύμφυτοι ἔσονται.
\vs{14}Καὶ ἐπιστρέψω τὴν αἰχμαλωσίαν τοῦ λαοῦ μου Ἰσραὴλ, καὶ οἰκοδομήσουσι πόλεις τὰς ἠφανισμένας, καὶ κατοικήσουσι, καὶ φυτεύσουσιν ἀμπελῶνας, καὶ πίονται τὸν οἶνον αὐτῶν, καὶ ποιήσουσι κήπους, καὶ φάγονται τὸν καρπὸν αὐτῶν·
\vs{15}Καὶ καταφυτεύσω αὐτοὺς ἐπὶ τῆς γῆς αὐτῶν, καὶ οὐ μὴ ἐκσπασθῶσιν οὐκέτι ἀπὸ τῆς γῆς, ἧς ἔδωκα αὐτοῖς, λέγει Κύριος ὁ Θεὸς παντοκράτωρ.


\end{multicols}
\chapter{ΟΒΔΕΙΟΥ}
\begin{multicols}{2}

\ch{1} 
ὍΡΑΣΙΣ Ὀβδίου. Τάδε λέγει Κύριος ὁ Θεὸς τῇ Ἰδουμαίᾳ, ἀκοὴν ἤκουσα παρὰ Κυρίου, καὶ περιοχὴν εἰς τὰ ἔθνη ἐξαπέστειλεν· ἀνάστητε, καὶ ἐξαναστῶμεν ἐπʼ αὐτὴν εἰς πόλεμον.

\vs{2}Ἰδοὺ ὀλιγοστὸν δέδωκά σε ἐν τοῖς ἔθνεσιν, ἠτιμωμένος εἶ σὺ σφόδρα.
\vs{3}Ὑπερηφανία τῆς καρδίας σου ἐπῇρέ σε κατασκηνοῦντα ἐν ταῖς ὀπαῖς τῶν πετρῶν· ὑψῶν κατοικίαν αὐτοῦ, λέγων ἐν καρδίᾳ αὐτοῦ, τίς κατάξει με ἐπὶ τὴν γῆν;
\vs{4}Ἐὰν μετεωρισθῇς, ὡς ἀετὸς, καὶ ἐὰν ἀναμέσον τῶν ἄστρων θῇς νοσσιάν σου, ἐκεῖθεν κατάξω σε, λέγει Κύριος.
\vs{5}Εἰ κλέπται εἰσῆλθον πρὸς σὲ, ἢ λῃσταὶ νυκτὸς, ποῦ ἂν ἀπεῤῥίφης; οὐκ ἂν ἔκλεψαν τὰ ἱκανὰ ἑαυτοῖς; καὶ εἰ τρυγηταὶ εἰσῆλθον πρὸς σὲ, οὐκ ἂν ἐπελείποντο ἐπιφυλλίδα;

\vs{6}Πῶς ἐξηρευνήθη Ἡσαῦ, καὶ κατελήφθη τὰ κεκρυμμένα αὐτοῦ;
\vs{7}Ἕως τῶν ὁρίων ἐξαπέστειλάν σε· πάντες οἱ ἄνδρες τῆς διαθήκης σου ἀντέστησάν σοι, ἠδυνάσθησαν πρὸς σὲ ἄνδρες εἰρηνικοί σου, ἔθηκαν ἔνεδρα ὑποκάτω σου, οὐκ ἔστι σύνεσις αὐτοῖς.

\vs{8}Ἐν τῇ ἡμέρᾳ ἐκείνῃ, λέγει Κύριος, ἀπολῶ σοφους ἐκ τῆς Ἰδουμαίας, καὶ σύνεσιν ἐξ ὄρους Ἡσαῦ.
\vs{9}Καὶ πτοηθήσονται οἱ μαχηταί σου οἱ ἐκ Θαιμὰν, ὅπως ἐξαρθῇ ἄνθρωπος ἐξ ὄρους Ἡσαῦ.
\vs{10}Διὰ τὴν σφαγὴν, καὶ τὴν ἀσέβειαν ἀδελφοῦ σου Ἰακὼβ, καλύψει σε αἰσχύνη, καὶ ἐξαρθήσῃ εἰς τὸν αἰῶνα.
\vs{11}Ἀφʼ ἧς ἡμέρας ἀντέστης ἐξεναντίας, ἐν ἡμέραις αἰχμαλωτευόντων ἀλλογενῶν δύναμιν αὐτοῦ, καὶ ἀλλότριοι εἰσῆλθον εἰς πύλας αὐτοῦ, καὶ ἐπὶ Ἱερουσαλὴμ ἔβαλον κλήρους, καὶ σὺ ἦς ὡς εἷς ἐξ αὐτῶν.

\vs{12}Καὶ μὴ ἐπίδῃς ἡμέραν ἀδελφοῦ σου ἐν ἡμέρᾳ ἀλλοτρίων, καὶ μὴ ἐπιχαρῇς ἐπὶ τοὺς υἱοὺς Ἰούδα ἐν ἡμέρᾳ ἀπωλείας αὐτῶν, καὶ μὴ μεγαλοῤῥημονῇ ἐν ἡμέρᾳ θλίψεως,
\vs{13}μηδὲ εἰσέλθῃς εἰς πύλας λαῶν ἐν ἡμέρᾳ πόνων αὐτῶν, μηδὲ ἐπίδῃς καὶ σὺ τὴν συναγωγὴν αὐτῶν ἐν ἡμέρᾳ ὀλέθρου αὐτῶν, καὶ μὴ συνεπιθῇ ἐπὶ τὴν δύναμιν αὐτῶν ἐν ἡμέρᾳ ἀπωλείας αὐτῶν,
\vs{14}μηδὲ ἐπιστῇς ἐπὶ τὰς διεκβολὰς αὐτῶν, ἐξολοθρεῦσαι τοὺς ἀνασωζομένους αὐτῶν, μηδὲ συγκλείσῃς τοὺς φεύγοντας αὐτοῦ ἐν ἡμέρᾳ θλίψεως.

\vs{15}Διότι ἐγγὺς ἡμέρα Κυρίου ἐπὶ πάντα τὰ ἔθνη· ὃν τρόπον ἐποίησας, οὕτως ἔσται σοι· τὸ ἀνταπόδομά σου ἀνταποδοθήσεται εἰς κεφαλήν σου.
\vs{16}Διότι ὃν τρόπον ἔπιες ἐπὶ τὸ ὄρος τὸ ἅγιόν μου, πίονται πάντα τὰ ἔθνη οἶνον, πίονται καὶ καταβήσονται, καὶ ἔσονται καθὼς οὐχ ὑπάρχοντες.

\vs{17}Ἑν δὲ τῷ ὄρει Σιὼν ἔσται σωτηρία, καὶ ἔσται ἅγιον· καὶ κατακληρονομήσουσιν ὁ οἶκος Ἰακὼβ τοῦς, κατακληρονομήσαντας αὐτούς.
\vs{18}Καὶ ἔσται ὁ οἶκος Ἰακὼβ πῦρ, ὁ δὲ οἶκος Ἰωσὴφ φλὸξ, ὁ δὲ οἶκος Ἡσαῦ εἰς καλάμην, καὶ ἐκκαυθήσονται εἰς αὐτοὺς, καὶ καταφάγονται αὐτοὺς, καὶ οὐκ ἔσται πυροφόρος τῷ οἴκῳ Ἡσαῦ, διότι Κύριος ἐλάλησε.
\vs{19}Καὶ κατακληρονομήσουσιν οἱ ἐν ναγὲβ τὸ ὄρος τὸ Ἡσαῦ, καὶ οἱ ἐν τῇ σεφηλὰ τοὺς ἀλλοφύλους· καὶ κατακληρονομήσουσι τὸ ὄρος Ἐφραὶμ, καὶ τὸ πεδίον Σαμαρείας, καὶ Βενιαμὶν, καὶ τὴν Γαλααδίτιν.

\vs{20}Καὶ τῆς μετοικεσίας ἡ ἀρχὴ αὕτη τοῖς υἱοῖς Ἰσραὴλ, γῆ τῶν Χαναναίων ἕως Σαρεπτῶν· καὶ ἡ μετοικεσία Ἱερουσαλὴμ ἕως Ἐφραθά· κληρονομήσουσι τὰς πόλεις τοῦ ναγέβ.

\vs{21}Καὶ ἀναβήσονται ἀνασωζόμενοι ἐξ ὄρους Σιὼν, τοῦ ἐκδικῆσαι τὸ ὄρος Ἡσαῦ, καὶ ἔσται τῷ Κυρίῳ ἡ βασιλεία.


\end{multicols}
\chapter{ΙΩΝΑΣ}
\begin{multicols}{2}

\ch{1}
ΚΑΙ ἐγένετο λόγος Κυρίου πρὸς Ἰωνᾶν τὸν τοῦ Ἀμαθὶ, λέγων,
\vs{2}ἀνάστηθι, καὶ πορεύθητι εἰς Νινευὴ τὴν πόλιν τὴν μεγάλην, καὶ κήρυξον ἐν αὐτῇ, ὅτι ἀνέβη ἡ κραυγὴ τῆς κακίας αὐτῆς πρὸς μέ.
\vs{3}Καὶ ἀνέστη Ἰωνᾶς τοῦ φυγεῖν εἰς Θαρσὶς ἐκ προσώπου Κυρίου· καὶ κατέβη εἰς Ἰόππην, καὶ εὗρε πλοῖον βαδίζον εἰς Θαρσὶς, καὶ ἔδωκε τὸ ναῦλον αὐτοῦ, καὶ ἀνέβη εἰς αὐτὸ, τοῦ πλεῦσαι μετʼ αὐτῶν εἰς Θαρσὶς ἐκ προσώπου Κυρίου.

\vs{4}Καὶ Κύριος ἐξήγειρε πνεῦμα ἐπὶ τὴν θάλασσαν, καὶ ἐγένετο κλύδων μέγας ἐν τῇ θαλάσσῃ, καὶ τὸ πλοῖον ἐκινδύνευε τοῦ συντριβῆναι.
\vs{5}Καὶ ἐφοβήθησαν οἱ ναυτικοὶ, καὶ ἀνεβόησαν ἕκαστος πρὸς τὸν θεὸν αὐτοῦ, καὶ ἐκβολὴν ἐποιήσαντο τῶν σκευῶν τῶν ἐν τῷ πλοίῳ εἰς τὴν θάλασσαν, τοῦ κουφισθῆναι ἀπʼ αὐτῶν· Ἰωνᾶς δὲ κατέβη εἰς τὴν κοίλην τοῦ πλοίου, καὶ ἐκάθευδε, καὶ ἔρεγχε.

\vs{6}Καὶ προσῆλθε πρὸς αὐτὸν ὁ πρωρεὺς, καὶ εἶπεν αὐτῷ, τί σὺ ῥέγχεις; ἀνάστα, καὶ ἐπικαλοῦ τὸν Θεόν σου, ὅπως διασώσῃ ὁ Θεὸς ἡμᾶς, καὶ οὐ μὴ ἀπολώμεθα.
\vs{7}Καὶ εἶπεν ἕκαστος πρὸς τὸν πλησίον αὐτοῦ, δεῦτε βάλωμεν κλήρους, καὶ ἐπιγνῶμεν, τίνος ἕνεκεν ἡ κακία αὕτη ἐστὶν ἐν ἡμῖν· καὶ ἔβαλον κλήρους, καὶ ἔπεσεν ὁ κλῆρος ἐπὶ Ἰωνᾶν.

\vs{8}Καὶ εἶπον πρὸς αὐτὸν, ἀπάγγειλον ἡμῖν, τίς σου ἡ ἐργασία ἐστὶ, καὶ πόθεν ἔρχῃ, καὶ ἐκ ποίας χώρας, καὶ ἐκ ποίου λαοῦ εἶ σύ;
\vs{9}Καὶ εἶπε πρὸς αὐτοὺς, δοῦλος Κυρίου εἰμὶ ἐγὼ, καὶ τὸν Κύριον Θεὸν τοῦ οὐρανοῦ ἐγὼ σέβομαι, ὃς ἐποίησε τὴν θάλασσαν καὶ τὴν ξηράν.
\vs{10}Καὶ ἐφοβήθησαν οἱ ἄνδρες φόβον μέγαν, καὶ εἶπον πρὸς αὐτὸν, τί τοῦτο ἐποίησας; διότι ἔγνωσαν οἱ ἄνδρες ὅτι ἐκ προσώπου Κυρίου ἦν φεύγων, ὅτι ἀπήγγειλεν αὐτοῖς·
\vs{11}καὶ εἶπον πρὸς αὐτὸν, τί ποιήσομέν σοι, καὶ κοπάσει ἡ θάλασσα ἀφʼ ἡμῶν; ὅτι ἡ θάλασσα ἐπορεύετο καὶ ἐξήγειρε μᾶλλον κλύδωνα.
\vs{12}Καὶ εἶπεν Ἰωνᾶς πρὸς αὐτοὺς, ἄρατέ με, καὶ ἐμβάλετέ με εἰς τὴν θάλασσαν, καὶ κοπάσει ἡ θάλασσα ἀφʼ ὑμῶν· διότι ἔγνωκα ἐγὼ, ὅτι διʼ ἐμὲ ὁ κλύδων ὁ μέγας οὗτος ἐφʼ ὑμᾶς ἐστι.

\vs{13}Καὶ παρεβιάζοντο οἱ ἄνδρες τοῦ ἐπιστρέψαι πρὸς τὴν γῆν, καὶ οὐκ ἠδύναντο, ὅτι ἡ θάλασσα ἐπορεύετο, καὶ ἐξηγείρετο μᾶλλον ἐπʼ αὐτούς.
\vs{14}Καὶ ἀνεβόησαν πρὸς Κύριον, καὶ εἶπαν, μηδαμῶς Κύριε· μὴ ἀπολώμεθα ἕνεκεν τῆς ψυχῆς τοῦ ἀνθρώπου τούτου, καὶ μὴ δῷς ἐφʼ ἡμᾶς αἷμα δίκαιον, διότι σὺ Κύριε, ὃν τρόπον ἐβούλου, πεποίηκας.
\vs{15}Καὶ ἔλαβον τὸν Ἰωνᾶν, καὶ ἐξέβαλον αὐτὸν εἰς τὴν θάλασσαν, καὶ ἔστη ἡ θάλασσα ἐκ τοῦ σάλου αὐτῆς.
\vs{16}Καὶ ἐφοβήθησαν οἱ ἄνδρες φόβῳ μεγάλῳ τὸν Κύριον, καὶ ἔθυσαν θυσίαν τῷ Κυρίῳ, καὶ ηὔξαντο τὰς εὐχάς.

\ch{2}
Καὶ προσέταξε Κύριος κήτει μεγάλῳ καταπιεῖν τὸν Ἰωνᾶν· καὶ ἦν Ἰωνᾶς ἐν τῇ κοιλίᾳ τοῦ κήτους τρεῖς ἡμέρας καὶ τρεῖς νύκτας.

\vs{2}Καὶ προσηύξατο Ἰωνᾶς πρὸς Κύριον τὸν Θεὸν αὐτοῦ ἐκ τῆς κοιλίας τοῦ κήτους,
\vs{3}καὶ εἶπεν,

Ἐβόησα ἐν θλίψει μου πρὸς Κύριον τὸν Θεόν μου, καὶ εἰσήκουσέ μου, ἐκ κοιλίας ᾅδου κραυγῆς μου, ἤκουσας φωνῆς μου, ἀπέῤῥιψάς με εἰς βάθη καρδίας θαλάσσης,
\vs{4}καὶ ποταμοί ἐκύκλωσάν με, πάντες οἱ μετεωρισμοί σου καὶ τὰ κύματά σου ἐπʼ ἐμὲ διῆλθον.
\vs{5}Καὶ ἐγὼ εἶπα, ἀπῶσμαι ἐξ ὀφθαλμῶν σου· ἆρα προσθήσω τοῦ ἐπιβλέψαι με πρὸς ναὸν τὸν ἅγιόν σου;
\vs{6}Περιεχύθη μοι ὕδωρ ἕως ψυχῆς, ἄβυσσος ἐκύκλωσέ με ἐσχάτη, ἔδυ ἡ κεφαλή μου εἰς σχισμὰς ὀρέων,
\vs{7}κατέβην εἰς γῆν, ἧς οἱ μοχλοὶ αὐτῆς κάτοχοι αἰώνιοι· καὶ ἀναβήτω φθορὰ ζωῆς μου Κύριε ὁ Θεός μου.

\vs{8}Ἐν τῷ ἐκλείπειν ἀπʼ ἐμοῦ τὴν ψυχήν μου, τοῦ Κυρίου ἐμνήσθην, καὶ ἔλθοι πρὸς σὲ ἡ προσευχή μου εἰς ναὸν τὸν ἅγιόν σου.
\vs{9}Φυλασσόμενοι μάταια καὶ ψευδῆ, ἔλεος αὐτῶν ἐγκατέλιπον.
\vs{10}Ἐγὼ δὲ μετὰ φωνῆς αἰνέσεως καὶ ἐξομολογήσεως θύσω σοι, ὅσα ηὐξάμην ἀποδώσω σοι σωτηρίου τῷ Κυρίῳ.

\vs{11}Καὶ προσετάγη ἀπὸ Κυρίου τῷ κήτει, καὶ ἐξέβαλε τὸν Ἰωνᾶν ἐπὶ τὴν ξηράν.

\ch{3}
Καὶ ἐγένετο λόγος Κυρίου πρὸς Ἰωνᾶν ἐκ δευτέρου, λέγων,
\vs{2}ἀνάστηθι, πορεύθητι εἰς Νινευὴ τὴν πόλιν τὴν μεγάλην; καὶ κήρυξον ἐν αὐτῇ κατὰ τὸ κήρυγμα τὸ ἔμπροσθεν, ὃ ἐγὼ ἐλάλησα πρὸς σέ.
\vs{3}Καὶ ἀνέστη Ἰωνᾶς, καὶ ἐπορεύθη εἰς Νινευὴ, καθὰ ἐλάλησε Κύριος· ἡ δὲ Νινευὴ ἦν πόλις μεγάλη τῷ Θεῷ, ὡσεὶ πορείας ὁδοῦ τριῶν ἡμερῶν·
\vs{4}καὶ ἤρξατο Ἰωνᾶς τοῦ εἰσελθεῖν εἰς τὴν πόλιν, ὡσεὶ πορείαν ἡμέρας μιᾶς· καὶ ἐκήρυξε, καὶ εἶπεν, ἔτι τρεῖς ἡμέραι, καὶ Νινευὴ καταστραφήσεται.

\vs{5}Καὶ ἐπίστευσαν οἱ ἄνδρες Νινευὴ τῷ Θεῷ, καὶ ἐκήρυξαν νηστείαν, καὶ ἐνεδύσαντο σάκκους ἀπὸ μεγάλου αὐτῶν ἕως μικροῦ αὐτῶν.
\vs{6}Καὶ ἤγγισεν ὁ λόγος πρὸς τὸν βασιλέα τῆς Νινευὴ, καὶ ἐξανέστη ἀπὸ τοῦ θρόνου αὐτοῦ, καὶ περιείλατο τὴν στολὴν αὐτοῦ ἀφʼ ἑαυτοῦ, καὶ περιεβάλετο σάκκον, καὶ ἐκάθισεν ἐπὶ σποδοῦ.
\vs{7}Καὶ ἐκηρύχθη, καὶ ἐῤῥέθη ἐν τῇ Νινευὴ παρὰ τοῦ βασιλέως καὶ παρὰ τῶν μεγιστάνων αὐτοῦ, λέγων, οἱ ἄνθρωποι, καὶ τὰ κτήνη, καὶ οἱ βόες, καὶ τὰ πρόβατα μὴ γευσάσθωσαν, μηδὲ νεμέσθωσαν, μηδὲ ὕδωρ πιέτωσαν.
\vs{8}Καὶ περιεβάλλοντο σάκκους οἱ ἄνθρωποι καὶ τὰ κτήνη, καὶ ἀνεβόησαν πρὸς τὸν Θεὸν ἐκτενῶς· καὶ ἀπέστρεψαν ἕκαστος ἀπὸ τῆς ὁδοῦ αὐτῶν τῆς πονηρᾶς, καὶ ἀπὸ τῆς ἀδικίας τῆς ἐν χερσὶν αὐτῶν, λέγοντες,
\vs{9}τίς οἶδεν εἰ μετανοήσει ὁ Θεὸς, καὶ ἀποστρέψει ἐξ ὀργῆς θυμοῦ αὐτοῦ, καὶ οὐ μὴ ἀπολώμεθα;

\vs{10}Καὶ εἶδεν ὁ Θεὸς τὰ ἔργα αὐτῶν, ὅτι ἀπέστρεψαν ἀπὸ τῶν ὁδῶν αὐτῶν τῶν πονηρῶν, καὶ μετενόησεν ὁ Θεὸς ἐπὶ τῇ κακίᾳ, ᾗ ἐλάλησε τοῦ ποιῆσαι αὐτοῖς, καὶ οὐκ ἐποίησε.

\ch{4}
Καὶ ἐλυπήθη Ἰωνᾶς λύπην μεγάλην· καὶ συνεχύθη,
\vs{2}καὶ προσεύξατο πρὸς Κύριον, καὶ εἶπεν, Κύριε, οὐχ οὗτοι οἱ λόγοι μου, ἔτι ὄντος μου ἐν τῇ γῇ μου; διατοῦτο προέφθασα τοῦ φυγεῖν εἰς Θαρσὶς, διότι ἔγνων ὅτι σὺ ἐλεήμων καὶ οἰκτίρμων, μακρόθυμος καὶ πολυέλεος, καὶ μετανοῶν ἐπὶ ταῖς κακίαις.
\vs{3}Καὶ νῦν, δέσποτα Κύριε, λάβε τὴν ψυχήν μου ἀπʼ ἐμοῦ, ὅτι καλὸν τὸ ἀποθανεῖν με ἢ ζῇν με.
\vs{4}Καὶ εἶπε Κύριος πρὸς Ἰωνᾶν, εἰ σφόδρα λελύπησαι σύ;

\vs{5}Καὶ ἐξῆλθεν Ἰωνᾶς ἐκ τῆς πόλεως, καὶ ἐκάθισεν ἀπέναντι τῆς πόλεως· καὶ ἐποίησεν αὐτῷ ἐκεῖ σκηνὴν, καὶ ἐκάθητο ὑποκάτω αὐτῆς, ἕως οὗ ἀπίδῃ τί ἔσται τῇ πόλει.
\vs{6}Καὶ προσέταξε Κύριος ὁ Θεὸς κολοκύνθῃ, καὶ ἀνέβη ὑπὲρ κεφαλῆς τοῦ Ἰωνᾶ, τοῦ εἶναι σκιὰν ὑπεράνω τῆς κεφαλῆς αὐτοῦ, τοῦ σκιάζειν αὐτῷ ἀπὸ τῶν κακῶν αὐτοῦ· καὶ ἐχάρη Ἰωνᾶς ἐπὶ τῇ κολοκύνθῃ χαρὰν μεγάλην.

\vs{7}Καὶ προσέταξεν ὁ Θεὸς σκώληκι ἑωθινῇ τῇ ἐπαυρίον, καὶ ἐπάταξε τὴν κολόκυνθαν, καὶ ἀπεξηράνθη.
\vs{8}Καὶ ἐγένετο ἅμα τῷ ἀνατεῖλαι τὸν ἥλιον, καὶ προσέταξεν ὁ Θεὸς πνεύματι καύσωνι συγκαίοντι, καὶ ἐπάταξεν ὁ ἥλιος ἐπὶ τὴν κεφαλὴν τοῦ Ἰωνᾶ· καὶ ὠλιγοψύχησε, καὶ ἀπελέγετο τὴν ψυχὴν αὐτοῦ, καὶ εἶπε, καλόν μοι ἀποθανεῖν με ἢ ζῇν.
\vs{9}Καὶ εἶπεν ὁ Θεὸς πρὸς Ἰωνᾶν, εἰ σφόδρα λελύπησαι σὺ ἐπὶ τῇ κολοκύνθῃ; καὶ εἶπε, σφόδρα λελύπημαι ἐγὼ ἕως θανάτου.

\vs{10}Καὶ εἶπε Κύριος, σὺ ἐφείσω ὑπὲρ τῆς κολοκύνθης, ὑπὲρ ἧς οὐκ ἐκακοπάθησας ἐπʼ αὐτὴν, καὶ οὐδὲ ἐξέθρεψας αὐτὴν, ἣ ἐγενήθη ὑπὸ νύκτα, καὶ ὑπὸ νύκτα ἀπώλετο·
\vs{11}ἐγὼ δὲ οὐ φείσομαι ὑπὲρ Νινευὴ τῆς πόλεως τῆς μεγάλης, ἐν ᾗ κατοικοῦσι πλείους ἢ δώδεκα μυριάδες ἀνθρώπων, οἵτινες οὐκ ἔγνωσαν δεξιὰν αὐτῶν ἢ ἀριστερὰν αὐτῶν, καὶ κτήνη πολλά;


\end{multicols}
\chapter{ΜΙΧΑΙΑΣ}
\begin{multicols}{2}

\ch{1}
ΚΑΙ ἐγένετο λόγος Κυρίου πρὸς Μιχαίαν τὸν τοῦ Μωρασθεὶ, ἐν ἡμέραις Ἰωάθαμ, καὶ Ἄχαζ, καὶ Ἐζεκίου βασιλέων Ἰούδα, ὑπὲρ ὧν εἶδε περὶ Σαμαρείας καὶ περὶ Ἱερουσαλήμ.

\vs{2}Ἀκούσατε λαοὶ λόγους, καὶ προσεχέτω ἡ γῆ, καὶ πάντες οἱ ἐν αὐτῇ· καὶ ἔσται Κύριος Κύριος ἐν ὑμῖν εἰς μαρτύριον, Κύριος ἐξ οἴκου ἁγίου αὐτοῦ.
\vs{3}Διότι ἰδοὺ Κύριος ἐκπορεύεται ἐκ τοῦ τόπου αὐτοῦ, καὶ καταβήσεται, καὶ ἐπιβήσεται ἐπὶ τὰ ὕψη τῆς γῆς,
\vs{4}καὶ σαλευθήσεται τὰ ὄρη ὑποκάτωθεν αὐτοῦ, καὶ αἱ κοιλάδες τακήσονται ὡσεὶ κηρὸς ἀπὸ προσώπου πυρὸς, καὶ ὡς ὕδωρ καταφερόμενον ἐν καταβάσει.

\vs{5}Διʼ ἀσέβειαν Ἰακὼβ πάντα ταῦτα, καὶ διʼ ἁμαρτίαν οἴκου Ἰσραήλ· τίς ἡ ἀσέβεια τοῦ Ἰακώβ; οὐχ ἡ Σαμάρεια; καὶ τίς ἡ ἁμαρτία οἴκου Ἰούδα; οὐχὶ Ἱερουσαλήμ;
\vs{6}Καὶ θήσομαι Σαμάρειαν εἰς ὀπωροφυλάκιον ἀγροῦ, καὶ εἰς φυτείαν ἀμπελῶνος, καὶ κατασπάσω εἰς χάος τοὺς λίθους αὐτῆς, καὶ τὰ θεμέλια αὐτῆς ἀποκαλύψω.
\vs{7}Καὶ πάντα τὰ γλυπτὰ αὐτῆς κατακόψουσι, καὶ πάντα τὰ μισθώματα αὐτῆς ἐμπρήσουσιν ἐν πυρὶ, καὶ πάντα τὰ εἴδωλα αὐτῆς θήσομαι εἰς ἀφανισμόν· διότι ἐκ μισθωμάτων πορνείας συνήγαγε, καὶ ἐκ μισθωμάτων πορνείας συνέστρεψεν.

\vs{8}Ἕνεκεν τούτου κόψεται, καὶ θρηνήσει, πορεύσεται ἀνυπόδετος, καὶ γυμνὴ ποιήσεται κοπετὸν ὡς δρακόντων, καὶ πένθος ὡς θυγατέρων σειρήνων.
\vs{9}Ὅτι κατεκράτησεν ἡ πληγὴ αὐτῆς, διότι ἦλθεν ἕως Ἰούδα, καὶ ἥψατο ἕως πύλης λαοῦ μου, ἕως Ἱερουσαλήμ.

\vs{10}Οἱ ἐν Γὲθ μὴ μεγαλύνεσθε, καὶ οἱ Ἐνακεὶμ μὴ ἀνοικοδομεῖτε ἐξ οἴκου κατὰ γέλωτα, γῆν καταπάσασθε καταγέλωτα ὑμῶν,
\vs{11}κατοικοῦσα καλῶς τὰς πόλεις αὐτῆς, οὐκ ἐξῆλθε κατοικοῦσα Σενναὰρ, κόψασθαι οἶκον ἐχόμενον αὐτῆς, λήψεται ἐξ ὑμῶν πληγὴν ὀδύνης.

\vs{12}Τίς ἥρξατο εἰς ἀγαθὰ κατοικούσῃ ὀδύνας; ὅτι κατέβη κακὰ παρὰ Κυρίου ἐπὶ πύλας Ἱερουσαλὴμ,
\vs{13}ψόφος ἁρμάτων καὶ ἱππευόντων· κατοικοῦσα Λαχεὶς, ἀρχηγὸς ἁμαρτίας αὕτη ἐστὶ τῇ θυγατρὶ Σιὼν, ὅτι ἐν σοὶ εὑρέθησαν ἀσέβειαι τοῦ Ἰσραήλ.
\vs{14}Διατοῦτο δώσει ἐξαποστελλομένους ἕως κληρονομίας Γὲθ, οἴκους ματαίους, εἰς κενὸν ἐγένοντο τοῖς βασιλεῦσι τοῦ Ἰσραὴλ,
\vs{15}ἕως τοὺς κληρονόμους ἀγάγωσι, κατοικοῦσα Λαχείς· κληρονομία ἕως Ὀδολλὰμ ἥξει, ἡ δόξα τὴς θυγατρὸς Ἰσραήλ.
\vs{16}Ξύρησαι, καὶ κεῖραι ἐπὶ τὰ τέκνα τὰ τρυφερά σου, ἐμπλάτυνον τὴν χηρείαν σου ὡς ἀετὸς, ὅτι ᾐχμαλωτεύθησαν ἀπὸ σοῦ.

\ch{2}
Ἐγένοντο λογιζόμενοι κόπους, καὶ ἐργαζόμενοι κακὰ ἐν ταῖς κοίταις αὐτῶν, καὶ ἅμα τῇ ἡμέρᾳ συνετέλουν αὐτὰ, διότι οὐκ ᾖραν πρὸς τὸν Θεὸν χεῖρας αὐτῶν.
\vs{2}Καὶ ἐπεθύμουν ἀγρούς, καὶ διήρπαζον ὀρφανούς, καὶ οἴκους κατεδυνάστευον, καὶ διήρπαζον ἄνδρα καὶ τὸν οἶκον αὐτοῦ, καὶ ἄνδρα καὶ τὴν κληρονομίαν αὐτοῦ.

\vs{3}Διατοῦτο τάδε λέγει Κύριος ἰδοὺ ἐγὼ λογίζομαι ἐπὶ τὴν φυλὴν ταύτην κακὰ, ἐξ ὧν οὐ μὴ ἄρητε τοὺς τραχήλους ὑμῶν, καὶ οὐ μὴ πορευθῆτε ὀρθοὶ ἐξαίφνης, ὅτι καιρὸς πονηρός ἐστιν.

\vs{4}Ἐν τῇ ἡμέρᾳ ἐκείνῃ ληφθήσεται ἐφʼ ὑμᾶς παραβολὴ, καὶ θρηνηθήσεται θρῆνος ἐν μέλει, λέγων, ταλαιπωρίᾳ ἐταλαιπωρήσαμεν· μερὶς λὰοῦ μου κατεμετρήθη ἐν σχοινίῳ, καὶ οὐκ ἦν ὁ κωλύων αὐτὸν τοῦ ἀποστρέψαι· οἱ ἀγροὶ ὑμῶν διεμερίσθησαν.
\vs{5}Διατοῦτο οὐκ ἔσται σοι βάλλων σχοινίον ἐν κλήρῳ· ἐν ἐκκλησίᾳ Κυρίου
\vs{6}μὴ κλαίετε δάκρυσι, μηδὲ δακρυέτωσαν ἐπὶ τούτοις· οὐδὲ γὰρ ἀπώσεται ὀνείδη,
\vs{7}ὁ λέγων, οἶκος Ἰακὼβ παρώργισε πνεῦμα Κυρίου. οὐ ταῦτα τὰ ἐπιτηδεύματα αὐτοῦ ἐστιν; οὐχ οἱ λόγοι αὐτοῦ εἰσι καλοὶ μετʼ αὐτοῦ; καὶ ὀρθοὶ πεπόρευνται;
\vs{8}Καὶ ἔμπροσθεν ὁ λαός μου εἰς ἔχθραν ἀντέστη, κατέναντι τῆς εἰρήνης αὐτοῦ· τὴν δορὰν αὐτοῦ ἐξέδειραν, τοῦ ἀφελέσθαι ἐλπίδας συντριμμὸν πολέμου.
\vs{9}Ἡγούμενοι λαοῦ μου ἀποῤῥιφήσονται ἐκ τῶν οἰκιῶν τρυφῆς αὐτῶν, διὰ τὰ πονηρὰ ἐπιτηδεύματα αὐτῶν ἐξώσθησαν· ἐγγίσατε ὄρεσιν αἰωνίοις.

\vs{10}Ἀνάστηθι καὶ πορεύου, ὅτι οὐκ ἔστι σοι αὕτη ἀνάπαυσις ἕνεκεν ἀκαθαρσίας· διεφθάρητε φθορᾷ,
\vs{11}κατεδιώχθητε οὐδενὸς διώκοντος· πνεῦμα ἔστησε ψεῦδος, ἐστάλαξέ σοι εἰς οἶνον καὶ μέθυσμα· καὶ ἔσται, ἐκ τῆς σταγόνος τοῦ λαοῦ τούτου
\vs{12}συναγόμενος συναχθήσεται Ἰακὼβ σὺν πᾶσιν· ἐκδεχόμενος ἐκδέξομαι τοὺς καταλοίπους τοῦ Ἰσραὴλ, ἐπιτοαυτὸ θήσομαι τὴν ἀποστροφὴν αὐτοῦ· ὡς πρόβατα ἐν θλίψει, ὡς ποίμνιον ἐν μέσῳ κοίτης αὐτῶν· ἐξαλοῦνται ἐξ ἀνθρώπων
\vs{13}διὰ τῆς διακοπῆς πρὸ προσώπου αὐτῶν· διέκοψαν, καὶ διῆλθον πύλην, καὶ ἐξῆλθον διʼ αὐτῆς, καὶ ἐξῆλθεν ὁ βασιλεὺς αὐτῶν πρὸ προσώπου αὐτῶν, ὁ δὲ Κύριος ἡγήσεται αὐτῶν.

\ch{3}
Καὶ ἐρεῖ, ἀκούσατε δὴ ταῦτα αἱ ἀρχαὶ οἴκου Ἰακὼβ, καὶ οἱ κατάλοιποι οἴκου Ἰσραήλ· οὐχ ὑμῖν ἐστι τοῦ γνῶναι τὸ κρίμα;
\vs{2}μισοῦντες τὰ καλὰ, καὶ ζητοῦντες τὰ πονηρὰ, ἁρπάζοντες τὰ δέρματα αὐτῶν ἀπʼ αὐτῶν, καὶ τὰς σάρκας αὐτῶν ἀπὸ τῶν ὀστέων αὐτῶν.
\vs{3}Ὃν τρόπον κατέφαγον τὰς σάρκας τοῦ λαοῦ μου, καὶ τὰ δέρματα αὐτῶν ἀπʼ αὐτῶν ἐξέδειραν, καὶ τὰ ὀστέα αὐτῶν συνέθλασαν, καὶ ἐμέλισαν ὡς σάρκας εἰς λέβητα, καὶ ὡς κρέα εἰς χύτραν,
\vs{4}οὕτως κεκράξονται πρὸς τὸν Κύριον, καὶ οὐκ εἰσακούσεται αὐτῶν· καὶ ἀποστρέψει τὸ πρόσωπον αὐτοῦ ἀπʼ αὐτῶν ἐν τῷ καιρῷ ἐκείνῳ, ἀνθʼ ὧν ἐπονηρεύσαντο ἐν τοῖς ἐπιτηδεύμασιν αὐτῶν ἐπʼ αὐτούς.

\vs{5}Τάδε λέγει Κύριος ἐπὶ τοὺς προφήτας τοὺς πλανῶντας τὸν λαόν μου, τοὺς δάκνοντας ἐν τοῖς ὀδοῦσιν αὐτῶν, καὶ κηρύσσοντας εἰρήνην ἐπʼ αὐτὸν, καὶ οὐκ ἐδόθη εἰς τὸ στόμα αὐτῶν, ἤγειραν ἐπʼ αὐτὸν πόλεμον·
\vs{6}διατοῦτο νὺξ ὑμῖν ἔσται ἐξ ὁράσεως, καὶ σκοτία ἔσται ὑμῖν ἐκ μαντείας, καὶ δύσεται ὁ ἥλιος ἐπὶ τοὺς προφήτας, καὶ συσκοτάσει ἐπʼ αὐτοὺς ἡ ἡμέρα·
\vs{7}Καὶ καταισχυνθήσονται οἱ ὁρῶντες τὰ ἐνύπνια, καὶ καταγελασθήσονται οἱ μάντεις, καὶ καταλαλήσουσι κατʼ αὐτῶν πάντες αὐτοὶ, διότι οὐκ ἔσται ὁ ἐπακούων αὐτῶν.
\vs{8}ἐὰν μὴ ἐγὼ ἐμπλήσω ἰσχὺν ἐν πνεύματι Κυρίου καὶ κρίματος καὶ δυναστείας, τοῦ ἀπαγγεῖλαι τῷ Ἰακὼβ ἀσεβείας αὐτοῦ, καὶ τῷ Ἰσραὴλ ἁμαρτίας αὐτοῦ.

\vs{9}Ἀκούσατε δὴ ταῦτα οἱ ἡγούμενοι οἴκου Ἰακὼβ, καὶ οἱ κατάλοιποι οἴκου Ἰσραὴλ, οἱ βδελυσσόμενοι κρίμα, καὶ πάντα τὰ ὀρθὰ διαστρέφοντες,
\vs{10}οἱ οἰκοδομοῦντες Σιὼν ἐν αἵμασι, καὶ Ἱερουσαλὴμ ἐν ἀδικίαις,
\vs{11}οἱ ἡγούμενοι αὐτῆς μετὰ δώρων ἔκρινον, καὶ οἱ ἱερεῖς αὐτῆς μετὰ μισθοῦ ἀπεκρίνοντο, καὶ οἱ προφῆται αὐτῆς μετὰ ἀργυρίου ἐμαντεύοντο, καὶ ἐπὶ τὸν Κύριον ἐπανεπαύοντο, λέγοντες, οὐχὶ ὁ Κύριος ἐν ἡμῖν ἐστιν; οὐ μὴ ἐπέλθῃ ἐφʼ ἡμᾶς κακά.
\vs{12}Διατοῦτο διʼ ὑμᾶς Σιὼν ὡς ἀγρὸς ἀροτριαθήσεται, καὶ Ἱερουσαλὴμ ὡς ὀπωροφυλάκιον ἔσται, καὶ τὸ ὄρος τοῦ οἴκου εἰς ἄλσος δρυμοῦ.

\ch{4}
Καὶ ἔσται ἐπʼ ἐσχάτων τῶν ἡμερῶν ἐμφανὲς τὸ ὄρος Κυρίου, ἕτοιμον ἐπὶ τὰς κορυφὰς τῶν ὀρέων, καὶ μετεωρισθήσεται ὑπεράνω τῶν βουνῶν· καὶ σπεύσουσι πρὸς αὐτὸ λαοὶ,
\vs{2}καὶ πορεύσονται ἔθνη πολλὰ καὶ ἐροῦσι, δεῦτε, ἀναβῶμεν εἰς τὸ ὄρος Κυρίου, καὶ εἰς τὸν οἶκον τοῦ Θεοῦ Ἰακώβ· καὶ δείξουσιν ἡμῖν τὴν ὁδὸν αὐτοῦ, καὶ πορευσόμεθα ἐν ταῖς τρίβοις αὐτοῦ· ὅτι ἐκ Σιὼν ἐξελεύσεται νόμος, καὶ λόγος Κυρίου ἐξ Ἱερουσαλήμ.
\vs{3}Καὶ κρινεῖ ἀναμέσον λαῶν πολλῶν, καὶ ἐξελέγξει ἔθνη ἰσχυρὰ ἕως εἰς μακράν· καὶ κατακόψουσι τὰς ῥομφαίας αὐτῶν εἰς ἄροτρα, καὶ τὰ δόρατα αὐτῶν εἰς δρέπανα, καὶ οὐκέτι μὴ ἀντάρῃ ἔθνος ἐπʼ ἔθνος ῥομφαίαν, καὶ οὐκέτι μὴ μάθωσι πολεμεῖν·
\vs{4}Καὶ ἀναπαύσεται ἕκαστος ὑποκάτω ἀμπέλου αὐτοῦ, καὶ ἕκαστος ὑποκάτω συκῆς αὐτοῦ, καὶ οὐκ ἔσται ὁ ἐκφοβῶν, διότι τὸ στόμα Κυρίου παντοκράτορος ἐλάλησε ταῦτα·
\vs{5}Ὅτι πάντες οἱ λαοὶ πορεύσονται ἕκαστος τὴν ὁδὸν αὐτοῦ, ἡμεῖς δὲ πορευσόμεθα ἐν ὀνόματι Κυρίου Θεοῦ ἡμῶν εἰς τὸν αἰῶνα, καὶ ἐπέκεινα.

\vs{6}Ἐν τῇ ἡμέρᾳ ἐκείνῃ, λέγει Κύριος, συνάξω τὴν συντετριμμένην, καὶ τὴν ἐξωσμένην εἰσδέξομαι, καὶ οὓς ἀπωσάμην.
\vs{7}Καὶ θήσομαι τὴν συντετριμμένην εἰς ὑπόλειμμα, καὶ τὴν ἀπωσμένην εἰς ἔθνος δυνατόν· καὶ βασιλεύσει Κύριος ἐπʼ αὐτοὺς ἐν ὄρει Σιὼν ἀπὸ τοῦ νῦν ἕως εἰς τὸν αἰῶνα.

\vs{8}Καὶ σὺ πύργος ποιμνίου αὐχμώδης, θυγάτηρ Σιὼν, ἐπὶ σὲ ἥξει, καὶ εἰσελεύσεται ἡ ἀρχὴ, ἡ πρώτη βασιλεία ἐκ Βαβυλῶνος τῇ θυγατρὶ Ἱερουσαλήμ.

\vs{9}Καὶ νῦν ἱνατί ἔγνως κακά; μὴ βασιλεὺς οὐκ ἦν σοι; ἢ ἡ βουλή σου ἀπώλετο, ὅτι κατεκράτησάν σου ὠδῖνες ὡς τικτούσης;
\vs{10}Ὤδινε καὶ ἀνδρίζου, καὶ ἔγγιζε θυγάτηρ Σιὼν ὡς τίκτουσα· διότι νῦν ἐξελεύσῃ ἐκ πόλεως, καὶ κατασκηνώσεις ἐν πεδίῳ, καὶ ἥξεις ἕως Βαβυλῶνος· ἐκεῖθεν ῥύσεταί σε, καὶ ἐκεῖθεν λυτρώσεταί σε Κύριος ὁ Θεός σου ἐκ χειρὸς ἐχθρῶν σου.

\vs{11}Καὶ νῦν ἐπισυνήχθησαν ἐπὶ σὲ ἔθνη πολλὰ, λέγοντες, ἐπιχαρούμεθα, καὶ ἐπόψονται ἐπὶ Σιὼν οἱ ὀφθαλμοὶ ἡμῶν.
\vs{12}Αὐτοὶ δὲ οὐκ ἔγνωσαν τὸν λογισμὸν Κυρίου, καὶ οὐ συνῆκαν τὴν βουλὴν αὐτοῦ, ὅτι συνήγαγεν αὐτοὺς ὡς δράγματα ἅλωνος.
\vs{13}Ἀνάστηθι, καὶ ἀλόα αὐτοὺς θυγάτηρ Σιὼν, ὅτι τὰ κέρατά σου θήσομαι σιδηρᾶ, καὶ τὰς ὁπλάς σου θήσομαι χαλκᾶς· καὶ κατατήξεις λαοὺς πολλοὺς, καὶ ἀναθήσεις τῷ Κυρίῳ τὸ πλῆθος αὐτῶν, καὶ τὴν ἰσχὺν αὐτῶν τῷ Κυρίῳ πάσης τῆς γῆς.

\vs{14}Νῦν ἐμφραχθήσεται θυγάτηρ ἐμφραγμῷ, συνοχὴν ἔταξεν ἐφʼ ἡμᾶς, ἐν ῥάβδῳ πατάξουσιν ἐπὶ σιαγόνα τὰς φυλὰς τοῦ Ἰσραήλ.

\ch{5}
Καὶ σὺ Βηθλεὲμ οἶκος Ἐφραθὰ, ὀλιγοστὸς εἶ τοῦ εἶναι ἐν χιλιάσιν Ἰούδα· ἐκ σοῦ μοι ἐξελεύσεται, τοῦ εἶναι εἰς ἄρχοντα τοῦ Ἰσραὴλ, καὶ ἔξοδοι αὐτοῦ ἀπʼ ἀρχῆς ἐξ ἡμερῶν αἰῶνος.

\vs{2}Διατοῦτο δώσει αὐτοὺς ἕως καιροῦ τικτούσης, τέξεται, καὶ οἱ ἐπίλοιποι τῶν ἀδελφῶν αὐτῶν ἐπιστρέψουσιν ἐπὶ τοὺς υἱοὺς Ἰσραήλ.
\vs{3}Καὶ στήσεται καὶ ὄψεται, καὶ ποιμανεῖ τὸ ποίμνιον αὐτοῦ ἐν ἰσχύϊ Κύριος, καὶ ἐν τῇ δόξῃ ὀνόματος Κυρίου Θεοῦ αὐτῶν ὑπάρξουσι, διότι νῦν μεγαλυνθήσονται ἕως ἄκρων τῆς γῆς.

\vs{4}Καὶ ἔσται αὐτῇ εἰρήνη, Ἀσσοὺρ ὅταν ἐπέλθῃ ἐπὶ τὴν γῆν ὑμῶν, καὶ ὅταν ἐπιβῇ ἐπὶ τὴν χώραν ὑμῶν, καὶ ἐπεγερθήσονται ἐπʼ αὐτὸν ἑπτὰ ποιμένες, καὶ ὀκτὼ δήγματα ἀνθρώπων,
\vs{5}καὶ ποιμανοῦσι τὸν Ἀσσοὺρ ἐν ῥομφαίᾳ, καὶ τὴν γῆν τοῦ Νεβρὼδ ἐν τῇ τάφρῳ αὐτῆς· καὶ ῥύσεται ἐκ τοῦ Ἀσσοὺρ ὅταν ἐπέλθῃ ἐπὶ τὴν γῆν ὑμῶν, καὶ ὅταν ἐπιβῇ ἐπὶ τὰ ὅρια ὑμῶν.

\vs{6}Καὶ ἔσται τὸ ὑπόλειμμα τοῦ Ἰακὼβ ἐν τοῖς ἔθνεσιν ἐν μέσῳ λαῶν πολλῶν, ὡς δρόσος παρὰ Κυρίου πίπτουσα, καὶ ὡς ἄρνες ἐπὶ ἄγρωστιν, ὅπως μὴ συναχθῇ μηδεὶς, μηδὲ ὑποστῇ ἐν υἱοῖς ἀνθρώπων.
\vs{7}Καὶ ἔσται τὸ ὑπόλειμμα Ἰακὼβ ἐν τοῖς ἔθνεσιν ἐν μέσῳ λαῶν πολλῶν, ὡς λέων ἐν κτήνεσιν ἐν τῷ δρυμῷ, καὶ ὡς σκύμνος ἐν ποιμνίοις προβάτων, ὃν τρόπον ὅταν διέλθῃ, καὶ διαστείλας ἁρπάσῃ, καὶ μὴ ᾖ ὁ ἐξαιρούμενος.
\vs{8}Ὑψωθήσεται ἡ χείρ σου ἐπὶ τοὺς θλίβοντάς σε, καὶ πάντες οἱ ἐχθροί σου ἐξολοθρευθήσονται.

\vs{9}Καὶ ἔσται ἐν τῇ ἡμέρᾳ ἐκείνῃ, λέγει Κύριος, ἐξολοθρεύσω τοὺς ἵππους ἐκ μέσου σου, καὶ ἀπολῶ τὰ ἅρματά σου,
\vs{10}καὶ ἐξολεθρεύσω τὰς πόλεις τῆς γῆς σου, καὶ ἐξαρῶ πάντα τὰ ὀχυρώματά σου·
\vs{11}καὶ ἐξολοθρεύσω τὰ φάρμακά σου ἐκ τῶν χειρῶν σου, καὶ ἀποφθεγγόμενοι οὐκ ἔσονται ἐν σοί·
\vs{12}καὶ ἐξολοθρεύσω τὰ γλυπτά σου, καὶ τὰς στηλὰς σου ἐκ μέσου σου, καὶ οὐκ ἔτι μὴ προσκυνήσεις τοῖς ἔργοις τῶν χειρῶν σου.
\vs{13}Καὶ ἐκκόψω τὰ ἄλση ἐκ μέσου σου, καὶ ἀφανιῶ τὰς πόλεις σου.
\vs{14}Καὶ ποιήσω ἐν ὀργῇ καὶ ἐν θυμῷ ἐκδίκησιν ἐν τοῖς ἔθνεσιν, ἀνθʼ ὧν οὐκ εἰσήκουσαν.

\ch{6}
Ἀκούσατε δὴ λόγον· Κύριος Κύριος εἶπεν, ἀνάστηθι, κρίθητι πρὸς τὰ ὄρη, καὶ ἀκουσάτωσαν βουνοὶ φωνήν σου.

\vs{2}Ἀκούσατε ὄρη τὴν κρίσιν τοῦ Κυρίου, καὶ αἱ φάραγγες θεμέλια τῆς γῆς, ὅτι κρίσις τῷ Κυρίῳ πρὸς τὸν λαὸν αὐτοῦ, καὶ μετὰ τοῦ Ἰσραὴλ διελεγχθήσεται.
\vs{3}Λαός μου, τί ἐποίησά σοι, ἢ τί ἐλύπησά σε, ἢ τί παρηνώχλησά σοι; ἀποκρίθητί μοι.
\vs{4}Διότι ἀνήγαγόν σε ἐκ γῆς Αἰγύπτου, καὶ ἐξ οἴκου δουλείας ἐλυτρωσάμην σε, καὶ ἐξαπέστειλα πρὸ προσώπου σου τὸν Μωυσῆν, καὶ Ἀαρὼν, καὶ Μαριάμ.

\vs{5}Λαός μου μνήσθητι δὴ, τί ἐβουλεύσατο κατὰ σοῦ Βαλὰκ βασιλεὺς Μωὰβ, καὶ τί ἀπεκρίθη αὐτῷ Βαλαὰμ, υἱὸς τοῦ Βεὼρ ἀπὸ τῶν σχοίνων ἕως τοῦ Γαλγὰλ, ὅπως γνωσθῇ ἡ δικαιοσύνη τοῦ Κυρίου.

\vs{6}Ἐν τίνι καταλάβω τὸν Κύριον, ἀντιλήψομαι Θεοῦ μου ὑψίστου; εἰ καταλήψομαι αὐτὸν ἐν ὁλοκαυτώμασιν, ἐν μόσχοις ἐνιαυσίοις;
\vs{7}Εἰ προσδέξεται Κύριος ἐν χιλιάσι κριῶν; ἢ ἐν μυριάσι χιμάρων πιόνων; εἰ δῶ πρωτότοκά μου ὑπὲρ ἀσεβείας, καρπὸν κοιλίας μου ὑπὲρ ἁμαρτίας ψυχῆς μου;
\vs{8}Εἰ ἀνηγγέλη σοι ἄνθρωπε τί καλόν; ἢ τί Κύριος ἐκζητεῖ παρὰ σοῦ, ἀλλʼ ἢ τοῦ ποιεῖν κρίμα, καὶ ἀγαπᾷν ἔλεον, καὶ ἕτοιμον εἶναι τοῦ πορεύεσθαι μετὰ Κυρίου Θεοῦ σου;

\vs{9}Φωνὴ Κυρίου τῇ πόλει ἐπικληθήσεται, καὶ σώσει φοβουμένους τὸ ὄνομα αὐτοῦ· ἄκουε φυλὴ, καὶ τίς κοσμήσει πόλιν;
\vs{10}Μὴ πῦρ καὶ οἶκος ἀνόμου θησαυρίζων θησαυροὺς ἀνόμους, καὶ μετὰ ὕβρεως ἀδικίας;
\vs{11}Εἰ δικαιωθήσεται ἐν ζυγῷ ἄνομος, καὶ ἐν μαρσίππῳ στάθμια δόλου,
\vs{12}ἐξ ὧν τὸν πλοῦτον αὐτῶν ἀσεβείας ἔπλησαν, καὶ οἱ κατοικοῦντες αὐτὴν ἐλάλουν ψεύδη, καὶ ἡ γλῶσσα αὐτῶν ὑψώθη ἐν τῷ στόματι αὐτῶν;

\vs{13}Καὶ ἐγὼ ἄρξομαι τοῦ πατάξαι σε, ἀφανιῶ σε ἐν ταῖς ἁμαρτίαις σου.
\vs{14}Σὺ φάγεσαι, καὶ οὐ μὴ ἐμπλησθῇς, καὶ συσκοτάσει ἐν σοὶ καὶ ἐκνεύσει, καὶ οὐ μὴ διασωθῇς, καὶ ὅσοι ἂν διασωθῶσιν, εἰς ῥομφαίαν παραδοθήσονται·
\vs{15}Σὺ σπερεῖς, καὶ οὐ μὴ ἀμήσῃς, σὺ πιέσεις ἐλαίαν, καὶ οὐ μὴ ἀλείψῃ ἔλαιον, καὶ οἶνον, καὶ οὐ μὴ πίητε, καὶ ἀφανισθήσεται νόμιμα λαοῦ μου.
\vs{16}Καὶ ἐφύλαξας τὰ δικαιώματα Ζαμβρὶ, καὶ πάντα τὰ ἔργα οἴκου Ἀχαὰβ, καὶ ἐπορεύθητε ἐν ταῖς ὁδοῖς αὐτῶν, ὅπως παραδῶ σε εἰς ἀφανισμὸν, καὶ τοὺς κατοικοῦντας αὐτὴν εἰς συρισμὸν, καὶ ὀνείδη λαῶν λήψεσθε.

\ch{7}
Οἴμοι, ὅτι ἐγενήθην ὡς συνάγων καλάμην ἐν ἀμητῷ, καὶ ὡς ἐπιφυλλίδα ἐν τρυγητῳ, οὐχ ὑπάρχοντος βότρυος τοῦ φαγεῖν τὰ πρωτόγονα. οἴμοι ψυχὴ,
\vs{2}ὅτι ἀπόλωλεν εὐσεβὴς ἀπὸ τῆς γῆς, καὶ κατορθῶν ἐν ἀνθρώποις οὐχ ὑπάρχει· πάντες εἰς αἵματα δικάζονται, ἕκαστος τὸν πλησίον αὐτοῦ ἐκθλίβουσιν ἐκθλιβῇ,
\vs{3}ἐπὶ τὸ κακὸν τὰς χεῖρας αὐτῶν ἑτοιμάζουσιν· ὁ ἄρχων αἰτεῖ, καὶ ὁ κριτὴς εἰρηνικοὺς λόγους ἐλάλησε, καταθύμιον ψυχῆς αὐτοῦ ἐστιν· καὶ ἐξελοῦμαι τὰ ἀγαθὰ αὐτῶν
\vs{4}ὡς σὴς ἐκτρώγων, καὶ βαδίζων ἐπὶ κανόνος ἐν ἡμέρᾳ σκοπιᾶς· οὐαὶ οὐαὶ, αἱ ἐκδικήσεις σου ἥκασι, νῦν ἔσονται κλαυθμοὶ αὐτῶν.
\vs{5}Μὴ καταπιστεύετε ἐν φίλοις, καὶ μὴ ἐλπίζετε ἐπὶ ἡγουμένοις· ἀπὸ τῆς συγκοίτου σου φύλαξαι, τοῦ ἀναθέσθαι τι αὐτῇ.
\vs{6}Διότι υἱὸς ἀτιμάζει πατέρα, θυγάτηρ ἐπαναστήσεται ἐπὶ τὴν μητέρα αὐτῆς, νύμφη ἐπὶ τὴν πενθερὰν αὐτῆς, ἐχθροὶ πάντες ἀνδρὸς οἱ ἐν τῷ οἴκῳ αὐτοῦ.

\vs{7}Ἐγὼ δὲ ἐπὶ τὸν Κύριον ἐπιβλέψομαι, ὑπομενῶ ἐπὶ τῷ Θεῷ τῷ σωτῆρί μου, εἰσακούσεταί μου ὁ Θεός μου.

\vs{8}Μὴ ἐπίχαιρέ μοι ἡ ἐχθρά μου, ὅτι πέπτωκα, καὶ ἀναστήσομαι· διότι ἐὰν καθίσω ἐν τῷ σκότει, Κύριος φωτιεῖ μοι.
\vs{9}Ὀργὴν Κυρίου ὑποίσω, ὅτι ἥμαρτον αὐτῷ, ἕως τοῦ δικαιῶσαι αὐτὸν τὴν δίκην μου· καὶ ποιήσει τὸ κρίμα μου, καὶ ἐξάξει με εἰς τὸ φῶς· ὄψομαι τὴν δικαιοσύνην αὐτοῦ,
\vs{10}καὶ ὄψεται ἡ ἐχθρά μου καὶ περιβαλεῖται αἰσχύνην, ἡ λέγουσα, ροῦ Κύριος ὁ Θεός σου; οἱ ὀφθαλμοί μου ἐπόψονται αὐτὴν, νῦν ἔσται εἰς καταπάτημα ὡς πηλὸς ἐν ταῖς ὁδοῖς.

\vs{11}Ἡμέρα ἀλοιφῆς πλίνθου, ἐξάλειψίς σου ἡ ἡμέρα ἐκείνη, καὶ ἀποτρίψεται νόμιμά σου ἡ ἡμέρα ἐκείνη.
\vs{12}Καὶ αἱ πόλεις σου ἥξουσιν εἰς ὁμαλισμὸν, καὶ εἰς διαμερισμὸν Ἀσσυρίων, καὶ αἱ πόλεις σου αἱ ὀχυραὶ εἰς διαμερισμὸν ἀπὸ Τύρου ἕως τοῦ ποταμοῦ, καὶ ἀπὸ θαλάσσης ἕως θαλάσσης, καὶ ἀπὸ ὄρους ἕως τοῦ ὄρους.
\vs{13}Καὶ ἔσται ἡ γῆ εἰς ἀφανισμὸν σὺν τοῖς κατοικοῦσιν αὐτὴν, ἀπὸ καρπῶν ἐπιτηδευμάτων αὐτῶν.

\vs{14}Ποίμαινε λαόν σου ἐν ῥάβδῳ σου, πρόβατα κληρονομίας σου, κατασκηνοῦντας καθʼ ἑαυτοὺς δρυμὸν ἐν μέσῳ τοῦ Καρμήλου· νεμήσονται τὴν Βασανίτιν, καὶ τὴν Γαλααδίτιν καθὼς αἱ ἡμέραι τοῦ αἰῶνος.

\vs{15}Καὶ κατὰ τὰς ἡμέρας ἐξοδίας σου ἐξ Αἰγύπτου, ὄψεσθε θαυμαστά.
\vs{16}Ὄψονται ἔθνη καὶ καταισχυνθήσονται, καὶ ἐκ πάσης τῆς ἰσχύος αὐτῶν, ἐπιθήσουσι χεῖρας ἐπὶ τὸ στόμα αὐτῶν, τὰ ὦτα αὐτῶν ἀποκωφωθήσεται,
\vs{17}λείξουσι χοῦν ὡς ὄφεις σύροντες γῆν, συγχυθήσονται ἐν συγκλεισμῷ αὐτῶν· ἐπὶ τῷ Κυρίῳ Θεῷ ἡμῶν ἐκστήσονται, καὶ φοβηθήσονται ἀπὸ σοῦ.

\vs{18}Τίς Θεὸς ὥσπερ σὺ; ἐξαίρων ἀνομίας, καὶ ὑπερβαίνων ἀσεβείας τοῖς καταλοίποις τῆς κληρονομίας αὐτοῦ; καὶ οὐ συνέσχεν εἰς μαρτύριον ὀργὴν αὐτοῦ, ὅτι θελητὴς ἐλέους ἐστίν.
\vs{19}Ἐπιστρέψει καὶ οἰκτειρήσει ἡμᾶς, καταδύσει τὰς ἀδικίας ἡμῶν, καὶ ἀποῤῥιφήσονται εἰς τὰ βάθη τῆς θαλάσσης πάσας τὰς ἁμαρτίας ἡμῶν.
\vs{20}Δώσει εἰς ἀλήθειαν τῷ Ἰακὼβ, ἔλεον τῷ Ἁβραάμ, καθότι ὤμοσας τοῖς πατράσιν ἡμῶν, κατὰ τὰς ἡμέρας τὰς ἔμπροσθεν.


\end{multicols}
\chapter{ΝΑΟΥΜ}
\begin{multicols}{2}

\ch{1}
ΛΗΜΜΑ Νινευὴ, βιβλίον ὁράσεως Ναοὺμ τοῦ Ἐλκεσαίου.

\vs{2}Θεὸς ζηλωτὴς, καὶ ἐκδικῶν Κύριος, ἐκδικῶν Κύριος μετὰ θυμοῦ, ἐκδικῶν Κύριος τοὺς ὑπεναντίους αὐτοῦ, καὶ ἐξαίρων αὐτὸς τοὺς ἐχθροὺς αὐτοῦ.
\vs{3}Κύριος μακρόθυμος, καὶ μεγάλη ἡ ἰσχὺς αὐτοῦ, καὶ ἀθῶον οὐκ ἀθωώσει Κύριος· ἐν συντελείᾳ, καὶ ἐν συσσεισμῷ ἡ ὁδὸς αὐτοῦ, καὶ νεφέλαι κονιορτὸς ποδῶν αὐτοῦ.
\vs{4}Ἀπειλῶν θαλάσσῃ, καὶ ξηραίνων αὐτὴν, καὶ πάντας τοὺς ποταμοὺς ἐξερημῶν· ὠλιγώθη ἡ Βασανίτις, καὶ ὁ Κάρμηλος, καὶ τὰ ἐξανθοῦντα τοῦ Λιβάνου ἐξέλιπε.
\vs{5}Τὰ ὄρη ἐσείσθησαν ἀπʼ αὐτοῦ, καὶ οἱ βουνοὶ ἐσαλεύθησαν· καὶ ἀνεστάλη ἡ γῆ ἀπὸ προσώπου αὐτοῦ, ἡ σύμπασα καὶ πάντες οἱ κατοικοῦντες ἐν αὐτῇ.
\vs{6}Ἀπὸ προσώπου ὀργῆς αὐτοῦ τίς ὑποστήσεται; καὶ τίς ἀντιστήσεται ἐν ὀργῇ θυμοῦ αὐτοῦ; ὁ θυμὸς αὐτοῦ τήκει ἀρχὰς, καὶ αἱ πέτραι διεθρύβησαν ἀπʼ αὐτοῦ.

\vs{7}Χρηστὸς Κύριος τοῖς ὑπομένουσιν αὐτὸν ἐν ἡμέρᾳ θλίψεως, καὶ γινώσκων τοὺς εὐλαβουμένους αὐτόν.
\vs{8}Καὶ ἐν κατακλυσμῷ πορείας συντέλειαν ποιήσεται, τοὺς ἐπεγειρομένους καὶ τοὺς ἐχθροὺς αὐτοῦ διώξεται σκότος.
\vs{9}Τί λογίζεσθε ἐπὶ τὸν Κύριον; συντέλειαν αὐτὸς ποιήσεται, οὐκ ἐκδικήσει δὶς ἐπιτοαυτὸ ἐν θλίψει.
\vs{10}Ὅτι ἕως θεμελίου αὐτοῦ χερσωθήσεται· καὶ ὡς σμῖλαξ περιπλεκομένη βρωθήσεται, καὶ ὡς καλάμη ξηρασίας μεστή.

\vs{11}Ἐκ σοῦ ἐξελεύσεται λογισμὸς κατὰ τοῦ Κυρίου, πονηρὰ βουλευόμενος ἐναντία.

\vs{12}Τάδε λέγει Κύριος κατάρχων ὑδάτων πολλῶν, καὶ οὕτως διασταλήσονται, καὶ ἡ ἀκοή σου οὐκ ἐνακουσθήσεται ἔτι.
\vs{13}Καὶ νῦν συντρίψω τὴν ῥάβδον αὐτοῦ ἀπὸ σοῦ, καὶ τοὺς δεσμοὺς διαῤῥήξω.

\vs{14}Καὶ ἐντελεῖται περὶ σοῦ Κύριος, οὐ σπαρήσεται ἐκ τοῦ ὀνόματός σου ἔτι· ἐξ οἴκου θεοῦ σου ἐξολοθρεύσω τὰ γλυπτὰ, καὶ χωνευτὰ, θήσομαι ταφήν σου, ὅτι ταχεῖς.

\ch{2}
Ἰδοὺ ἐπὶ τὰ ὄρη οἱ πόδες εὐαγγελιζομένου, καὶ ἀπαγγέλλοντος εἰρήνην· ἑόρταζε Ἰούδα τὰς ἑορτάς σου, ἀπόδος τὰς εὐχάς σου, διότι οὐ μὴ προσθήσωσιν ἔτι τοῦ διελθεῖν διὰ σοῦ εἰς παλαίωσιν.

\vs{2}Συντετέλεσται, ἐξῇρται· ἀνέβη ἐμφυσῶν εἰς πρόσωπόν σου, ἐξαιρούμενος ἐκ θλίψεως· σκόπευσον ὁδὸν, κράτησον ὀσφύος, ἄνδρισαι τῇ ἰσχύϊ σφόδρα.

\vs{3}Διότι ἀπέστρεψε Κύριος τὴν ὕβριν Ἰακὼβ, καθὼς ὕβριν τοῦ Ἰσραὴλ, διότι ἐκτινάσσοντες ἐξετίναξαν αὐτοὺς, καὶ τὰ κλήματα αὐτῶν διέφθειραν.
\vs{4}Ὅπλα δυναστείας αὐτῶν ἐξ ἀνθρώπων, ἄνδρας δυνατοὺς ἐμπαίζοντας ἐν πυρί· αἱ ἡνίαι τῶν ἁρμάτων αὐτῶν ἐν ἡμέρᾳ ἑτοιμασίας αὐτοῦ, καὶ οἱ ἱππεῖς θορυβηθήσονται
\vs{5}ἐν ταῖς ὁδοῖς, καὶ συγχυθήσονται τὰ ἅρματα, καὶ συμπλακήσονται ἐν ταῖς πλατείαις· ἡ ὅρασις αὐτῶν ὡς λαμπάδες πυρὸς, καὶ ὡς ἀστραπαὶ διατρέχουσαι.

\vs{6}Καὶ μνησθήσονται οἱ μεγιστᾶνες αὐτῶν, καὶ φεύξονται ἡμέρας, καὶ ἀσθενήσουσιν ἐν τῇ πορείᾳ αὐτῶν, καὶ σπεύσουσιν ἐπὶ τὰ τείχη αὐτῆς, καὶ ἑτοιμάσουσι τὰς προφυλακὰς αὐτῶν.
\vs{7}Πύλαι τῶν πόλεων διηνοίχθησαν, καὶ τὰ βασίλεια διέπεσε,
\vs{8}καὶ ἡ ὑπόστασις ἀπεκαλύφθη· καὶ αὕτη ἀνέβαινε, καὶ αἱ δοῦλαι αὐτῆς ἤγοντο, καθὼς περιστεραὶ φθεγγόμεναι ἐν καρδίαις αὐτῶν.
\vs{9}Καὶ Νινευὴ ὡς κολυμβήθρα ὕδατος τὰ ὕδατα αὐτῆς, καὶ αὐτοὶ φεύγοντες οὐκ ἔστησαν, καὶ οὐκ ἦν ὁ ἐπιβλέπων.

\vs{10}Διήρπαζον τὸ ἀργύριον, διήρπαζον τὸ χρυσίον, καὶ οὐκ ἦν πέρας τοῦ κόσμου αὐτῆς· βεβάρυνται ἐπὶ πάντα τὰ σκεὺη τὰ ἐπιθυμητὰ αὐτῆς.
\vs{11}Ἐκτιναγμὸς, καὶ ἀνατιναγμὸς, καὶ ἐκβρασμὸς, καὶ καρδίας θραυσμὸς, καὶ ὑπόλυσις γονάτων, καὶ ὠδίνες ἐπὶ πᾶσαν ὀσφύν· καὶ τὸ πρόσωπον πάντων ὡς πρόσκαυμα χύτρας.

\vs{12}Ποῦ ἐστι τὸ κατοικητήριον τῶν λεόντων, καὶ ἡ νομὴ ἡ οὖσα τοῖς σκύμνοις; ποῦ ἐπορεύθη λέων, τοῦ εἰσελθεῖν ἐκεῖ σκύμνον λέοντος, καὶ οὐκ ἦν ὁ ἐκφοβῶν;
\vs{13}Λέων ἥρπασε τὰ ἱκανὰ τοῖς σκύμνοις αὐτοῦ, καὶ ἀπέπνιξε τοῖς λέουσιν αὐτοῦ, καὶ ἔπλησε θήρας νοσσιὰν αὐτοῦ, καὶ τὸ κατοικητήριον αὐτοῦ ἁρπαγῆς.

\vs{14}Ἰδοὺ ἐγὼ ἐπὶ σὲ, λέγει Κύριος παντοκράτωρ, καὶ ἐκκαύσω ἐν καπνῷ πλῆθός σου, καὶ τοὺς λέοντάς σου καταφάγεται ῥομφαία, καὶ ἐξολοθρεύσω ἐκ τῆς γῆς τὴν θήραν σου, καὶ οὐ μὴ ἀκουσθῇ οὐκέτι τὰ ἔργα σου.

\ch{3}
Ὦ πόλις αἱμάτων, ὅλη ψευδὴς, ἀδικίας πλήρης, οὐ ψηλαφηθήσεται θήρα.
\vs{2}Φωνὴ μαστίγων, καὶ φωνὴ σεισμοῦ τροχῶν, καὶ ἵππου διώκοντος, καὶ ἅρματος ἀναβράσσοντος,
\vs{3}καὶ ἱππέως ἀναβαίνοντος, καὶ στιλβούσης ῥομφαίας, καὶ ἐξαστραπτόντων ὅπλων, καὶ πλήθους τραυματιῶν, καὶ βαρείας πτώσεως, καὶ οὐκ ἦν πέρας τοῖς ἔθνεσιν αὐτῆς· καὶ ἀσθενήσουσιν ἐν τοῖς σώμασιν αὐτῶν ἀπὸ πλήθους πορνείας·
\vs{4}πόρνη καλὴ, καὶ ἐπίχαρις ἡγουμένη φαρμάκων, ἡ πωλοῦσα ἔθνη ἐν τῇ πορνείᾳ αὐτῆς, καὶ λαοὺς ἐν τοῖς φαρμάκοις αὐτῆς.

\vs{5}Ἰδοὺ ἐγὼ ἐπὶ σὲ, λέγει Κύριος ὁ Θεὸς ὁ παντοκράτωρ, καὶ ἀποκαλύψω τὰ ὀπίσω σου ἐπὶ τὸ πρόσωπόν σου, καὶ δείξω ἔθνεσι τὴν αἰσχύνην σου, καὶ βασιλείαις τὴν ἀτιμίαν σου.
\vs{6}Καὶ ἐπιῤῥίψω ἐπὶ σὲ βδελυγμὸν κατὰ τὰς ἀκαθαρσίας σου, καὶ θήσομαί σε εἰς παράδειγμα.
\vs{7}Καὶ ἔσται, πᾶς ὁ ὁρῶν σε καταβήσεται ἀπὸ σοῦ, καὶ ἐρεῖ, δειλαία Νινευή· τίς στενάξει αὐτήν; πόθεν ζητήσω παράκλησιν αὐτῇ;

\vs{8}Ἑτοιμασὰι μερίδα, ἁρμόσαι χορδὴν, ἑτοιμάσαι μερίδα Ἀμμών· ἡ κατοικοῦσα ἐν ποταμοῖς, ὕδωρ κύκλῳ αὐτῆς, ἧς ἡ ἀρχὴ θάλασσα, καὶ ὕδωρ τὰ τείχη αὐτῆς,
\vs{9}καὶ Αἰθιοπία ἰσχὺς αὐτῆς, καὶ Αἴγυπτος· καὶ οὐκ ἔστη πέρας τῆς φυγῆς· καὶ Λίβυες ἐγένοντο βοηθοὶ αὐτῆς.
\vs{10}Καὶ αὐτὴ εἰς μετοικεσίαν πορεύσεται αἰχμάλωτος καὶ τὰ νήπια αὐτῆς ἐδαφιοῦσιν ἐπʼ ἀρχὰς πασῶν τῶν ὁδῶν αὐτῆς, καὶ ἐπὶ πάντα τὰ ἔνδοξα αὐτῆς βαλοῦσι κλήρους· καὶ πάντες οἱ μεγιστᾶνες αὐτῆς δεθήσονται χειροπέδαις.
\vs{11}Καὶ σὺ μεθυσθήσῃ, καὶ ἔσῃ ὑπερεωραμένη, καὶ σὺ ζητήσεις σεαυτῇ στάσιν ἐξ ἐχθρῶν.
\vs{12}Πάντα τὰ ὀχυρώματά σου συκαῖ σκοποὺς ἔχουσαι· ἐὰν σαλευθῶσι, πεσοῦνται εἰς στόμα ἔσθοντος.
\vs{13}Ἰδοὺ ὁ λαός σου ὡς γυναῖκες ἐν σοὶ, τοῖς ἐχθροῖς σου ἀνοιγόμεναι ἀνοιχθήσονται πύλαι τῆς γῆς σου, καταφάγεται πῦρ τοὺς μοχλούς σου.

\vs{14}Ὕδωρ περιοχῆς ἐπίσπασαι σεαυτῇ, καὶ κατακράτησον τῶν ὀχυρωμάτων σου· ἔμβηθι εἰς πηλὸν, καὶ συμπατήθητι ἐν ἀχύροις, κατακράτησον ὑπὲρ πλίνθον.
\vs{15}Ἐκεῖ καταφάγεταί σε πῦρ, ἐξολοθρεύσει σε ῥομφαία, καταφάγεταί σε ὡς ἀκρὶς, καὶ βαρυνθήσῃ ὡς βροῦχος.
\vs{16}Ἐπλήθυνας τὰς ἐμπορίας σου ὑπὲρ τὰ ἄστρα τοῦ οὐρανοῦ· βροῦχος ὥρμησε, καὶ ἐξεπετάσθη.
\vs{17}Ἐξήλατο ὡς ἀττέλεβος ὁ σύμμικτός σου, ὡς ἀκρὶς ἐπιβεβηκυῖα ἐπὶ φραγμὸν ἐν ἡμέρᾳ πάγους· ὁ ἥλιος ἀνέτειλε, καὶ ἀφήλατο, καὶ οὐκ ἔγνω τὸν τόπον αὐτῆς· οὐαὶ αὐτοῖς.

\vs{18}Ἐνύσταξαν οἱ ποιμένες σου, βασιλεὺς Ἀσσύριος ἐκοίμισε τοὺς δυνάστας σου, ἀπῇρεν ὁ λαός σου ἐπὶ τὰ ὄρη, καὶ οὐκ ἦν ὁ ἐκδεχόμενος.

\vs{19}Οὐκ ἔστιν ἴασις τῇ συντριβῇ σου, ἐφλέγμανεν ἡ πληγή σου, πάντες οἱ ἀκούοντες τὴν ἀγγελίαν σου κροτήσουσι χεῖρας ἐπὶ σέ· διότι ἐπὶ τίνα οὐκ ἐπῆλθεν ἡ κακία σου διαπαντός;


\end{multicols}
\chapter{ΑΜΒΑΚΟΥΜ}
\begin{multicols}{2}

\ch{1}
ΤΟ λῆμμα ὃ εἶδεν Ἀμβακοὺμ ὁ προφήτης.

\vs{2}Ἕως τίνος Κύριε κεκράξομαι, καὶ οὐ μὴ εἰσακουσεις; βοήσομαι πρὸς σὲ ἀδικούμενος, καὶ οὐ σώσεις;
\vs{3}Ἱνατί ἔδειξάς μοι κόπους καὶ πόνους ἐπιβλέπειν, ταλαιπωρίαν καὶ ἀσέβειαν; ἐξεναντίας μου γέγονε κρίσις, καὶ ὁ κριτὴς λαμβάνει·
\vs{4}Διατοῦτο διεσκέδασται νόμος, καὶ οὐ διεξάγεται εἰς τέλος κρίμα, ὅτι ἀσεβὴς καταδυναστεύει τὸν δίκαιον, ἕνεκεν τούτου ἐξελεύσεται τὸ κρίμα διεστραμμένον.

\vs{5}Ἴδετε οἱ καταφρονηταὶ, καὶ ἐπιβλέψατε, καὶ θαυμάσατε θαυμάσια, καὶ ἀφανίσθητε· διότι ἔργον ἐγὼ ἐργάζομαι ἐν ταῖς ἡμέραις ὑμῶν, ὃ οὐ μὴ πιστεύσητε, ἐάν τις ἐκδιηγῆται.
\vs{6}Διότι ἰδοὺ ἐγὼ ἐξεγείρω τοὺς Χαλδαίους, τὸ ἔνθος τὸ πικρὸν, καὶ τὸ ταχινὸν, τὸ πορευόμενον ἐπὶ τὰ πλάτη τῆς γῆς, τοῦ κατακληρονομῆσαι σκηνώματα οὐκ αὐτοῦ.
\vs{7}Φοβερὸς καὶ ἐπιφανής ἐστιν, ἐξ αὐτοῦ τὸ κρίμα αὐτοῦ ἔσται, καὶ τὸ λῆμμα αὐτοῦ ἐξ αὐτοῦ ἐξελεύσεται.
\vs{8}Καὶ ἐξαλοῦνται ὑπὲρ παρδάλεις οἱ ἵπποι αὐτοῦ, καὶ ὀξύτεροι ὑπὲρ τοὺς λύκους τῆς Ἀραβίας· καὶ ἐξιππάσονται οἱ ἱππεῖς αὐτοῦ, καὶ ὁρμήσουσι μακρόθεν, καὶ πετασθήσονται ὡς ἀετὸς πρόθυμος εἰς τὸ φαγεῖν.
\vs{9}Συντέλεια εἰς ἀσεβεῖς ἥξει, ἀνθεστηκότας προσώποις αὐτῶν ἐξεναντίας, καὶ συνάξει ὡς ἄμμον αἰχμαλωσίαν·
\vs{10}καὶ αὐτὸς ἐν βασιλεῦσιν ἐντρυφήσει, καὶ τύραννοι παίγνια αὐτοῦ, καὶ αὐτὸς εἰς πᾶν ὀχύρωμα ἐμπαίξεται, καὶ βαλεῖ χῶμα, καὶ κρατήσει αὐτοῦ·
\vs{11}τότε μεταβαλεῖ τὸ πνεῦμα, καὶ διελεύσεται, καὶ ἐξιλάσεται· αὕτη ἡ ἰσχὺς τῷ θεῷ μου.

\vs{12}Οὐχὶ σὺ ἀπʼ ἀρχῆς Κύριε ὁ Θεὸς ὁ ἅγιός μου; καὶ οὐ μὴ ἀποθάνωμεν· Κύριε εἰς κρίμα τέταχας αὐτὸ, καὶ ἔπλασέ με τοῦ ἐλέγχειν παιδείαν αὐτοῦ.
\vs{13}Καθαρὸς ὀφθαλμὸς τοῦ μὴ ὁρᾷν πονηρὰ, καὶ ἐπιβλέπειν ἐπὶ πόνους ὀδύνης· ἱνατί ἐπιβλέπεις ἐπὶ καταφρονοῦντας; παρασιωπήσῃ ἐν τῷ καταπίνειν ἀσεβῆ τὸν δίκαιον;
\vs{14}Καὶ ποιήσεις τοὺς ἀνθρώπους ὡς τοὺς ἰχθύας τῆς θαλάσσης, καὶ ὡς τὰ ἑρπετὰ τὰ οὐκ ἔχοντα ἡγούμενον;
\vs{15}Συντέλειαν ἐν ἀγκίστρῳ ἀνέσπασεν, καὶ εἵλκυσεν αὐτὸν ἐν ἀμφιβλήστρῳ, καὶ συνήγαγεν αὐτὸν ἐν ταῖς σαγήναις αὐτοῦ· ἕνεκεν τούτου εὐφρανθήσεται καὶ χαρήσεται ἡ καρδία αὐτοῦ.
\vs{16}Ἕνεκεν τούτου θύσει τῇ σαγήνῃ αὐτοῦ, καὶ θυμιάσει τῷ ἀμφιβλήστρῳ αὐτοῦ, ὅτι ἐν αὐτοῖς ἐλίπανε μερίδα αὐτοῦ, καὶ τὰ βρώματα αὐτοῦ ἐκλεκτά.
\vs{17}Διατοῦτο ἀμφιβαλεῖ τὸ ἀμφίβληστρον αὐτοῦ, καὶ διαπαντὸς ἀποκτέννειν ἔθνη οὐ φείσεται.

\ch{2}
Ἐπὶ τῆς φυλακῆς μου στήσομαι, καὶ ἐπιβήσομαι ἐπὶ πέτραν, καὶ ἀποσκοπεύσω τοῦ ἰδεῖν τί λαλήσει ἐν ἐμοὶ, καὶ τί ἀποκριθῶ ἐπὶ τὸν ἔλεγχόν μου.

\vs{2}Καὶ ἀπεκρίθη πρὸς μὲ Κύριος, καὶ εἶπε, γράψον ὅρασιν, καὶ σαφῶς εἰς πυξίον, ὅπως διώκῃ ὁ ἀναγινώσκων αὐτά.
\vs{3}Διότι ἔτι ὅρασις εἰς καιρὸν, καὶ ἀνατελεῖ εἰς πέρας, καὶ οὐκ εἰς κενόν· ἐὰν ὑστερήσῃ, ὑπὸμεινον αὐτὸν, ὅτι ἐρχόμενος ἥξει, καὶ οὐ μὴ χρονίσῃ.

\vs{4}Ἐὰν ὑποστείληται, οὐκ εὐδοκεῖ ἡ ψυχή μου ἐν αὐτῷ· ὁ δὲ δίκαιος ἐκ πίστεώς μου ζήσεται.
\vs{5}Ὁ δὲ κατοιόμενος, καὶ καταφρονητὴς, ἀνὴρ ἀλαζὼν, οὐθὲν μὴ περάνῃ· ὃς ἐπλάτυνε καθὼς ᾅδης τὴν ψυχὴν αὐτοῦ, καὶ οὗτος ὡς θάνατος οὐκ ἐμπιπλάμενος, καὶ ἐπισυνάξει ἐπʼ αὐτὸν πάντα τὰ ἔθνη, καὶ εἰσδέξεται πρὸς αὐτὸν πάντας τοὺς λαούς.
\vs{6}Οὐχὶ ταῦτα πάντα κατʼ αὐτοῦ παραβολὴν λήψονται, καὶ πρόβλημα εἰς διήγησιν αὐτοῦ; καὶ ἐροῦσιν, οὐαὶ ὁ πληθύνων ἑαυτῷ τὰ οὐκ ὄντα αὐτοῦ ἕως τίνος, καὶ βαρύνων τὸν κλοιὸν αὐτοῦ στιβαρῶς.
\vs{7}Ὅτι ἐξαίφνης ἀναστήσονται δάκνοντες αὐτὸν, καὶ ἐκνήψουσιν οἱ ἐπίβουλοί σου, καὶ ἔσῃ εἰς διαρπαγὴν αὐτοῖς,
\vs{8}διότι ἐσκύλευσας ἔθνη πολλὰ, σκυλεύσουσι πάντες οἱ ὑπολελειμμένοι λαοὶ, διʼ αἵματα ἀνθρώπων, καὶ ἀσεβείας γῆς καὶ πόλεως, καὶ πάντων τῶν κατοικούντων αὐτήν.

\vs{9}Ὢ ὁ πλεονεκτῶν πλεονεξίαν κακὴν τῷ οἴκῳ αὐτοῦ, τοῦ τάξαι εἰς ὕψος νοσσιὰν αὐτοῦ, τοῦ ἐκσπασθῆναι ἐκ χειρὸς κακῶν.
\vs{10}Ἐβουλεύσω αἰσχύνην τῷ οἴκῳ σου, συνεπέρανας πολλοὺς λαοὺς, καὶ ἐξήμαρτεν ἡ ψυχή σου.
\vs{11}Διότι λίθος ἐκ τοίχου βοήσεται, καὶ κάνθαρος ἐκ ξύλου φθέγξεται αὐτά.

\vs{12}Οὐαὶ ὁ οἰκοδομῶν πόλιν ἐν αἵμασι, καὶ ἑτοιμάζων πόλιν ἐν ἀδικίαις.
\vs{13}Οὐ ταῦτά ἐστι παρὰ Κυρίου παντοκράτορος; καὶ ἐξέλιπον λαοὶ ἱκανοὶ ἐν πυρὶ, καὶ ἔθνη πολλὰ ὠλιγοψύχησαν.
\vs{14}Ὅτι ἐμπλησθήσεται ἡ γῆ τοῦ γνῶναι τὴν δόξαν Κυρίου, ὡς ὕδωρ κατακαλύψει αὐτούς.

\vs{15}Ὢ ὁ ποτίζων τὸν πλησίον αὐτοῦ ἀνατροπῇ θολερᾷ, καὶ μεθύσκων ὅπως ἐπιβλέπῃ ἐπὶ τὰ σπήλαια αὐτῶν.
\vs{16}Πλησμονὴν ἀτιμίας ἐκ δόξης πίε καὶ σύ· καρδία σαλεύθητι, καὶ σείσθητι· ἐκύκλωσεν ἐπὶ σὲ ποτήριον δεξιᾶς Κυρίου, καὶ συνήχθη ἀτιμία ἐπὶ τὴν δόξαν σου.
\vs{17}Διότι ἀσέβεια τοῦ Λιβάνου καλύψει σε, καὶ ταλαιπωρία θηρίων πτοήσει σε, διʼ αἵματα ἀνθρώπων, καὶ ἀσεβείας γῆς καὶ πόλεως, καὶ πάντων τῶν κατοικούντων αὐτήν.

\vs{18}Τί ὠφελεῖ γλυπτὸν, ὅτι ἔγλυψαν αὐτό; ἔπλασεν αὐτὸ χώνευμα, φαντασίαν ψευδῆ, ὅτι πέποιθεν ὁ πλάσας ἐπὶ τὸ πλάσμα αὐτοῦ, τοῦ ποιῆσαι εἴδωλα κωφά.

\vs{19}Οὐαὶ ὁ λέγων τῷ ξύλῳ, ἔκνηψον, ἐξεγέρθητι· καὶ τῷ λίθῳ, ὑψώθητι· καὶ αὐτό ἐστι φαντασία· τοῦτο δέ ἐστιν ἔλασμα χρυσίου καὶ ἀργυρίου, καὶ πᾶν πνεῦμα οὐκ ἔστιν ἐν αὐτῷ.
\vs{20}Ὁ δὲ Κύριος ἐν ναῷ ἁγίῳ αὐτοῦ· εὐλαβείσθω ἀπὸ προσώπου αὐτοῦ πᾶσα ἡ γῆ.

\ch{3}
ΠΡΟΣΕΥΧ ἈΜΒΑΚΟΥΜ ΤΟΥ ΠΡΟΦΗΤΟΥ, ΜΕΤΑ Ὠ̣ΔΗΣ.

\vs{2}Κύριε εἰσακήκοα τὴν ἀκοήν σου, καὶ ἐφοβήθην· κατενόησα τὰ ἔργα σου, καὶ ἐξέστην· ἐν μέσῳ δύο ζώων γνωσθήσῃ, ἐν τῷ ἐγγίζειν τὰ ἔτη ἐπιγνωσθήσῃ· ἐν τῷ παρεῖναι τὸν καιρὸν ἀναδειχθήσῃ· ἐν τῷ ταραχθῆναι τὴν ψυχήν μου, ἐν ὀργῇ ἐλέους μνησθήσῃ.

\vs{3}Ὁ Θεὸς ἐκ Θαιμὰν ἥξει, καὶ ὁ ἅγιος ἐξ ὄρους Φαρὰν κατασκίου δασέος· διάψαλμα· ἐκάλυψεν οὐρανοὺς ἡ ἀρετὴ αὐτοῦ, καὶ αἰνέσεως αὐτοῦ πλήρης ἡ γῆ.
\vs{4}Καὶ φέγγος αὐτοῦ ὡς φῶς ἔσται· κέρατα ἐν χερσὶν αὐτοῦ, καὶ ἔθετο ἀγάπησιν κραταιὰν ἰσχύος αὐτοῦ.
\vs{5}Πρὸ προσώπου αὐτοῦ πορεύσεται λόγος, καὶ ἐξελεύσεται εἰς πεδία· κατὰ πόδας αὐτοῦ
\vs{6}ἔστη, καὶ ἐσαλεύθη ἡ γῆ· ἐπέβλεψε, καὶ διετάκη ἔθνη· διεθρύβη τὰ ὄρη βίᾳ, ἐτάκησαν βουνοὶ αἰώνιοι πορείας αἰωνίας αὐτοῦ.
\vs{7}Ἀντὶ κόπων εἶδον σκηνώματα Αἰθιόπων, πτοηθήσονται καὶ αἱ σκηναὶ γῆς Μαδιάμ.

\vs{8}Μὴ ἐν ποταμοῖς ὠργίσθης Κύριε; ἢ ἐν ποταμοῖς ὁ θυμός σου; ἢ ἐν θαλάσσῃ τὸ ὅρμημά σου; ὅτι ἐπιβήσῃ ἐπὶ τοὺς ἵππους σου, καὶ ἡ ἱππασία σου σωτηρία.
\vs{9}Ἐντείνων ἐνέτεινας τόξον σου ἐπὶ σκῆπτρα, λέγει Κύριος· διάψαλμα· ποταμῶν ῥαγήσεται γῆ.
\vs{10}Ὄψονταί σε, καὶ ὠδινήσουσι λαοὶ, σκορπίζων ὕδατα πορείας· ἔδωκεν ἡ ἄβυσσος φωνὴν αὐτῆς, ὕψος φαντασίας αὐτῆς.
\vs{11}Ἐπῄρθη ὁ ἥλιος, καὶ ἡ σελήνη ἔστη ἐν τῇ τάξει αὐτῆς· εἰς φῶς βολίδες σου πορεύσονται, εἰς φέγγος ἀστραπῆς ὅπλων σου.
\vs{12}Ἐν ἀπειλῇ ὀλιγώσεις γῆν, καὶ ἐν θυμῷ κατάξεις ἔθνη,
\vs{13}ἐξῆλθες εἰς σωτηρίαν λαοῦ σου, τοῦ σῶσαι τὸν χριστόν σου· βαλεῖς εἰς κεφαλὰς ἀνόμων θάνατον, ἐξήγειρας δεσμοὺς ἕως τραχήλου· διάψαλμα.
\vs{14}Διέκοψας ἐν ἐκστάσει κεφαλὰς δυναστῶν, σεισθήσονται ἐν αὐτῇ· διανοίξουσι χαλινοὺς αὐτῶν, ὡς ἔσθων πτωχὸς λάθρα.
\vs{15}Καὶ ἐπιβιβᾷς εἰς θάλασσαν τοὺς ἵππους σου, ταράσσοντας ὕδωρ πολύ.

\vs{16}Ἐφυλαξάμην, καὶ ἐπτοήθη ἡ κοιλία μου ἀπὸ φωνῆς προσευχῆς χειλέων μου, καὶ εἰσῆλθε τρόμος εἰς τὰ ὀστᾶ μου, καὶ ὑποκάτωθέν μου ἐταράχθη ἡ ἕξις μου· ἀναπαύσομαι ἐν ἡμέρᾳ θλίψεως, τοῦ ἀναβῆναι εἰς λαὸν παροικίας μου.

\vs{17}Διότι συκῆ οὐ καρποφορήσει, καὶ οὐκ ἔσται γεννήματα ἐν ταῖς ἀμπέλοις· ψεύσεται ἔργον ἐλαίας, καὶ τὰ πεδία οὐ ποιήσει βρῶσιν· ἐξέλιπεν ἀπὸ βρώσεως πρόβατα, καὶ οὐχ ὑπάρχουσι βόες ἐπὶ φάτναις·
\vs{18}ἐγὼ δὲ ἐν τῷ Κυρίῳ ἀγαλλιάσομαι, χαρήσομαι ἐπὶ τῷ Θεῷ τῷ σωτῆρί μου.
\vs{19}Κύριος ὁ Θεὸς δύναμίς μου, καὶ τάξει τοὺς πόδας μου εἰς συντέλειαν· ἐπὶ τὰ ὑψηλὰ ἐπιβιβᾷ με, τοῦ νικῆσαι ἐν τῇ ᾠδῇ αὐτοῦ.


\end{multicols}
\chapter{ΣΟΦΟΝΙΑΣ}
\begin{multicols}{2}

\ch{1}
ΛΟΓΟΣ Κυρίου, ὃς ἐγενήθη πρὸς Σοφονίαν τὸν τοῦ Χουσὶ, υἱὸν Γοδολίου, τοῦ Ἀμορίου τοῦ Ἐζεκίου, ἐν ἡμέραις Ἰωσίου υἱοῦ Ἀμὼν βασιλέως Ἰούδα.

\vs{2}Ἐκλείψει ἐκλιπέτω ἀπὸ προσώπου τῆς γῆς, λέγει Κύριος.
\vs{3}Ἐκλιπέτω ἄνθρωπος καὶ κτήνη, ἐκλιπέτω τὰ πετεινὰ τοῦ οὐρανοῦ καὶ οἱ ἰχθύες τῆς θαλάσσης· καὶ ἀσθενήσουσιν οἱ ἀσεβεῖς, καὶ ἐξαρῶ τοὺς ἀνόμους ἀπὸ προσώπου τῆς γῆς, λέγει Κύριος.
\vs{4}Καὶ ἐκτενῶ τὴν χεῖρά μου ἐπὶ Ἰούδα, καὶ ἐπὶ πάντας τοὺς κατοικοῦντας Ἱερουσαλήμ· καὶ ἐξαρῶ ἐκ τοῦ τόπου τούτου τὰ ὀνόματα τῆς Βάαλ, καὶ τὰ ὀνόματα τῶν ἱερέων,
\vs{5}καὶ τοὺς προσκυνοῦντας ἐπὶ τὰ δώματα τῇ στρατιᾷ τοῦ οὐρανοῦ, καὶ τοὺς προσκυνοῦντας καὶ τοὺς ὀμνύοντας κατὰ τοῦ Κυρίου, καὶ τοὺς ὀμνύοντας κατὰ τοῦ βασιλέως αὐτῶν,
\vs{6}καὶ τοὺς ἐκκλίνοντας ἀπὸ τοῦ Κυρίου, καὶ τοὺς μὴ ζητοῦντας τὸν Κύριον, καὶ τοὺς μὴ ἀντεχομένους τοῦ Κυρίου.

\vs{7}Εὐλαβεῖσθε ἀπὸ προσώπου Κυρίου τοῦ Θεοῦ· διότι ἐγγὺς ἡμέρα τοῦ Κυρίου, ὅτι ἡτοίμακε Κύριος τὴν θυσίαν αὐτοῦ, καὶ ἡγίακε τοὺς κλητοὺς αὐτοῦ.
\vs{8}Καὶ ἔσται, ἐν ἡμέρᾳ θυσίας Κυρίου, καὶ ἐκδικήσω ἐπὶ τοὺς ἄρχοντας, καὶ ἐπὶ τὸν οἶκον τοῦ βασιλέως, καὶ ἐπὶ πάντας τοὺς ἐνδεδυμένους ἐνδύματα ἀλλότρια.
\vs{9}Καὶ ἐκδικήσω ἐμφανῶς ἐπὶ τὰ πρόπυλα ἐν ἐκείνῃ τῇ ἡμέρᾳ, τοὺς πληροῦντας τὸν οἶκον Κυρίου Θεοῦ αὐτῶν ἀσεβείας καὶ δόλου.

\vs{10}Καὶ ἔσται ἐν τῇ ἡμέρᾳ ἐκείνῃ, λέγει Κύριος, φωνὴ κραυγῆς ἀπὸ πύλης ἀποκεντούντων, καὶ ὀλολυγμὸς ἀπὸ τῆς δευτέρας, καὶ συντριμμὸς μέγας ἀπὸ τῶν βουνῶν.
\vs{11}Θρηνήσατε οἱ κατοικοῦντες τὴν κατακεκομμένην, ὅτι ὡμοιώθη πᾶς ὁ λαὸς Χαναὰν, καὶ ἐξωλοθρεύθησαν πάντες οἱ ἐπῃρμένοι ἀργυρίῳ.

\vs{12}Καὶ ἔσται ἐν τῇ ἡμέρᾳ ἐκείνῃ, ἐξερευνήσω τὴν Ἱερουσαλὴμ μετὰ λύχνου, καὶ ἐκδικήσω ἐπὶ τοὺς ἄνδρας τοὺς καταφρονοῦντας ἐπὶ τὰ φυλάγματα αὐτῶν· οἱ δὲ λέγοντες ἐν ταῖς καρδίαις αὐτῶν, οὐ μὴ ἀγαθοποιήσῃ Κύριος, οὐδὲ μὴ κακώσῃ·
\vs{13}καὶ ἔσται ἡ δύναμις αὐτῶν εἰς διαρπαγὴν, καὶ οἱ οἶκοι αὐτῶν εἰς ἀφανισμόν· καὶ οἰκοδομήσουσιν οἰκίας, καὶ οὐ μὴ κατοικήσουσιν ἐν αὐταῖς· καὶ καταφυτεύσουσιν ἀμπελῶνας, καὶ οὐ μὴ πίωσι τὸν οἶνον αὐτῶν.

\vs{14}Ὅτι ἐγγὺς ἡμέρα Κυρίου ἡ μεγάλη, ἐγγὺς καὶ ταχεῖα σφόδρα· φωνὴ ἡμέρας Κυρίου πικρὰ καὶ σκληρὰ τέτακται·
\vs{15}δυνατὴ ἡμέρα ὀργῆς, ἡ ἡμέρα ἐκείνη, ἡμέρα θλίψεως καὶ ἀνάγκης, ἡμέρα ἀωρίας καὶ ἀφανισμοῦ, ἡμέρα γνόφου καὶ σκότους, ἡμέρα νεφέλης καὶ ὁμίχλης,
\vs{16}ἡμέρα σάλπιγγος καὶ κραυγῆς ἐπὶ τὰς πόλεις τὰς ὀχυρὰς, καὶ ἐπὶ τὰς γωνίας τὰς ὑψηλάς.
\vs{17}Καὶ ἐκθλίψω τοὺς ἀνθρώπους, καὶ πορεύσονται ὡς τυφλοὶ, ὅτι τῷ Κυρίῳ ἐξήμαρτον· καὶ ἐκχεεῖ τὸ αἷμα αὐτῶν ὡς χοῦν, καὶ τὰς σάρκας αὐτῶν ὡς βόλβιτα.
\vs{18}Καὶ τὸ ἀργύριον αὐτῶν καὶ τὸ χρυσίον αὐτῶν οὐ μὴ δύνηται ἐξελέσθαι αὐτοὺς ἐν ἡμέρᾳ ὀργῆς Κυρίου· καὶ ἐν πυρὶ ζήλου αὐτοῦ καταναλωθήσεται πᾶσα ἡ γῆ, διότι συντέλειαν καὶ σπουδὴν ποιήσει ἐπὶ πάντας τοὺς κατοικοῦντας τὴν γῆν.

\ch{2}
Συνάχθητε, καὶ συνδέθητε τὸ ἔθνος τὸ ἀπαίδευτον,
\vs{2}πρὸ τοῦ γενέσθαι ὑμᾶς ὡς ἄνθος παραπορευόμενον, πρὸ τοῦ ἐπελθεῖν ἐφʼ ὑμᾶς ὀργὴν Κυρίου, πρὸ τοῦ ἐπελθεῖν ἐφʼ ὑμᾶς ἡμέραν θυμοῦ Κυρίου.
\vs{3}Ζητήσατε τὸν Κύριον πάντες ταπεινοὶ γῆς, κρίμα ἐργάζεσθε, καὶ δικαιοσύνην ζητήσατε, καὶ ἀποκρίνασθε αὐτὰ, ὅπως σκεπασθῆτε ἐν ἡμέρᾳ ὀργῆς Κυρίου.

\vs{4}Διότι Γάζα διηρπασμένη ἔσται, καὶ Ἀσκάλων εἰς ἀφανισμὸν, καὶ Ἄζωτος μεσημβρίας ἐκριφήσεται, καὶ Ἀκκαρὼν ἐκριζωθήσεται.
\vs{5}Οὐαὶ οἱ κατοικοῦντες τὸ σχοίνισμα τῆς θαλάσσης, πάροικοι Κρητῶν· λόγος Κυρίου ἐφʼ ὑμᾶς Χαναὰν, γῆ ἀλλοφύλων, καὶ ἀπολῶ ὑμᾶς ἐκ κατοικίας.
\vs{6}Καὶ ἔσται Κρήτη νομὴ ποιμνίων, καὶ μάνδρα προβάτων.
\vs{7}Καὶ ἔσται τὸ σχοίνισμα τῆς θαλάσσης τοῖς καταλοίποις οἴκου Ἰούδα, ἐπʼ αὐτοὺς νεμήσονται ἐν τοῖς οἴκοις Ἀσκάλωνος, δείλης καταλύσουσιν ἀπὸ προσώπου υἱῶν Ἰούδα, ὅτι ἐπέσκεπται αὐτοὺς Κύριος ὁ Θεὸς αὐτῶν, καὶ ἀποστρέψει τὴν αἰχμαλωσίαν αὐτῶν.

\vs{8}Ἤκουσα ὀνειδισμοὺς Μωὰβ, καὶ κονδυλισμοὺς υἱῶν Ἀμμὼν, ἐν οἷς ὠνείδιζον τὸν λαόν μου, καὶ ἐμεγαλύνοντο ἐπὶ τὰ ὅριά μου.
\vs{9}Διατοῦτο, ζῶ ἐγὼ, λέγει Κύριος τῶν δυνάμεων ὁ Θεὸς Ἰσραὴλ, διότι Μωὰβ ὡς Σόδομα ἔσται, καὶ υἱοὶ Ἀμμὼν ὡς Γόμοῤῥα, καὶ Δαμασκὸς ἐκλελειμμένη ὡς θιμωνία ἅλωνος, καὶ ἠφανισμένη εἰς τὸν αἰῶνα· καὶ οἱ κατάλοιποι λαοῦ μου διαρπῶνται αὐτοὺς, καὶ οἱ κατάλοιποι ἔθνους μου κληρονομήσουσιν αὐτούς.
\vs{10}Αὕτη αὐτοῖς ἀντὶ τῆς ὕβρεως αὐτῶν, διότι ὠνείδισαν, καὶ ἐμεγαλύνθησαν ἐπὶ τὸν Κύριον τὸν παντοκράτορα.
\vs{11}Ἐπιφανήσεται Κύριος ἐπʼ αὐτοὺς, καὶ ἐξολοθρεύσει πάντας τοὺς θεοὺς τῶν ἐθνῶν τῆς γῆς, καὶ προσκυνήσουσιν αὐτῷ ἔκαστος ἐκ τοῦ τόπου αὐτοῦ, πᾶσαι αἱ νῆσοι τῶν ἐθνῶν.

\vs{12}Καὶ ὑμεῖς Αἰθίοπες τραυματίαι ῥομφαίας μου ἐστέ.

\vs{13}Καὶ ἐκτενεῖ τὴν χεῖρα αὐτοῦ ἐπὶ Βοῤῥᾶν, καὶ ἀπολεῖ τὸν Ἀσσύριον, καὶ θήσει τὴν Νινευὴ εἰς ἀφανισμὸν ἄνυδρον, ὡς ἔρημον.
\vs{14}Καὶ νεμήσονται ἐν μέσῳ αὐτῆς ποίμνια, καὶ πάντα τὰ θηρία τῆς γῆς, καὶ χαμαιλέοντες, καὶ ἐχῖνοι ἐν τοῖς φατνώμασιν αὐτῆς κοιτασθήσονται· καὶ θηρία φωνήσει ἐν τοῖς διορύγμασιν αὐτῆς καὶ κόρακες ἐν τοῖς πυλῶσιν αὐτῆς, διότι κέδρος τὸ ἀνάστημα αὐτῆς.

\vs{15}Αὕτη ἡ πόλις ἡ φαυλίστρια, ἡ κατοικοῦσα ἐπʼ ἐλπίδι, ἡ λέγουσα ἐν καρδίᾳ αὐτῆς, ἐγώ εἰμι, καὶ οὐκ ἔστι μετʼ ἐμὲ ἔτι· πῶς ἐγενήθη εἰς ἀφανισμὸν, νομὴ θηρίων; πᾶς ὁ διαπορευόμενος διʼ αὐτῆς συριεῖ, καὶ κινήσει τὰς χεῖρας αὐτοῦ.

\ch{3}
Ὢ ἡ ἐπιφανὴς καὶ ἀπολελυτρωμένη πόλις,
\vs{2}ἡ περιστερὰ οὐκ εἰσήκουσε φωνῆς· οὐκ ἐδέξατο παιδείαν, ἐπὶ τῷ Κυρίῳ οὐκ ἐπεποίθει, καὶ πρὸς τὸν Θεὸν αὐτῆς οὐκ ἤγγισεν.
\vs{3}Οἱ ἄρχοντες αὐτῆς ἐν αὐτῇ ὡς λέοντες ὠρυόμενοι, οἱ κριταὶ αὐτῆς ὡς λύκοι τῆς Ἀραβίας, οὐχ ὑπελίποντο εἰς τοπρωΐ.
\vs{4}Οἱ προφῆται αὐτῆς πνευματοφόροι, ἄνδρες καταφρονηταί· ἱερεῖς αὐτῆς βεβηλοῦσι τὰ ἅγια, καὶ ἀσεβοῦσι νόμον.

\vs{5}Ὁ δὲ Κύριος δίκαιος ἐν μέσῳ αὐτῆς, καὶ οὐ μὴ ποιήσῃ ἄδικον· πρωῒ πρωῒ δώσει κρίμα αὐτοῦ εἰς φῶς, καὶ οὐκ ἀπεκρύβη, καὶ οὐκ ἔγνω ἀδικίαν ἐν ἀπαιτήσει, καὶ οὐκ εἰς νεῖκος ἀδικίαν.
\vs{6}Ἐν διαφθορᾷ κατέσπασα ὑπερηφάνους, ἠφανίσθησαν γωνίαι αὐτῶν· ἐξερημώσω τὰς ὁδοὺς αὐτῶν τοπαράπαν, τοῦ μὴ διοδεύειν· ἐξέλιπον αἱ πόλεις αὐτῶν, παρὰ τὸ μηδένα ὑπάρχειν, μηδὲ κατοικεῖν.
\vs{7}Εἶπα, πλὴν φοβεῖσθέ με, καὶ δέξασθε παιδείαν, καὶ οὐ μὴ ἐξολοθρευθῆτε ἐξ ὀφθαλμῶν αὐτῆς πάντα ὅσα ἐξεδίκησα ἐπʼ αὐτήν· ἑτοιμάζου, ὄρθρισον, ἔφθαρται πᾶσα ἡ ἐπιφυλλὶς αὐτῶν.

\vs{8}Διατοῦτο ὑπόμεινόν με, λέγει Κύριος, εἰς ἡμέραν ἀναστάσεώς μου εἰς μαρτύριον· διὸ τὸ κρίμα μου εἰς συναγωγὰς ἐθνῶν, τοῦ εἰσδέκασθαι βασιλεῖς, τοῦ ἐκχέαι ἐπʼ αὐτοὺς πᾶσαν ὀργὴν θυμοῦ· διότι ἐν πυρὶ ζήλου μου καταναλωθήσεται πᾶσα ἡ γῆ.

\vs{9}Ὅτι τότε μεταστρέψω ἐπὶ λαοὺς γλῶσσαν εἰς γενεὰν αὐτῆς, τοῦ ἐπικαλεῖσθαι πάντας τὸ ὄνομα Κυρίου, τοῦ δουλεύειν αὐτῷ ὑπὸ ζυγὸν ἕνα.
\vs{10}Ἐκ περάτων ποταμῶν Αἰθιοπίας· προσδέξομαι ἐν διεσπαρμένοις μου, οἴσουσι θυσίας μοι.
\vs{11}Ἐν τῇ ἡμέρᾳ ἐκείνῃ, οὐ μὴ καταισχυνθῇς ἐκ πάντων τῶν ἐπιτηδευμάτων σου, ὧν ἠσέβησας εἰς ἐμέ· ὅτι τότε περιελῶ ἀπὸ σοῦ τὰ φαυλίσματα τῆς ὕβρεώς σου, καὶ οὐκ ἔτι μὴ προσθῇς, τοῦ μεγαλαυχῆσαι ἐπὶ τὸ ὄρος τὸ ἅγιόν μου.
\vs{12}Καὶ ὑπολήψομαι ἐν σοὶ λαὸν πρᾳῢν καὶ ταπεινὸν, καὶ εὐλαβηθήσονται ἀπὸ τοῦ ὀνόματος Κυρίου
\vs{13}οἱ κατάλοιποι τοῦ Ἰσραὴλ, καὶ οὐ ποιήσουσιν ἀδικίαν, καὶ οὐ λαλήσουσι μάταια, καὶ οὐ μὴ εὑρεθῇ ἐν τῷ στόματι αὐτῶν γλῶσσα δολία· διότι αὐτοὶ νεμήσονται, καὶ κοιτασθήσονται, καὶ οὐκ ἔσται ὁ ἐκφοβῶν αὐτούς.

\vs{14}Χαῖρε θύγατερ Σιὼν, κήρυσσε θύγατερ Ἱερουσαλήμ· εὐφραίνου καὶ κατατέρπου ἐξ ὅλης τῆς καρδίας σου θύγατερ Ἱερουσαλήμ.
\vs{15}Περιεῖλε Κύριος τὰ ἀδικήματά σου, λελύτρωταί σε ἐκ χειρὸς ἐχθρῶν σου· βασιλεὺς Ἰσραὴλ Κύριος ἐν μέσῳ σου, οὐκ ὄψῃ κακὰ οὐκέτι.

\vs{16}Ἐν τῷ καιρῷ ἐκείνῳ ἐρεῖ Κύριος τῇ Ἱερουσαλὴμ, θάρσει Σιὼν, μὴ παρείσθωσαν αἱ χεῖρές σου.
\vs{17}Κύριος ὁ Θεός σου ἐν σοὶ, ὁ δυνατὸς σώσει σε, ἐπάξει ἐπὶ δὲ εὐφροσύνην, καὶ καινιεῖ σε ἐν τῇ ἀγαπήσει αὐτοῦ· καὶ εὐφρανθήσεται ἐπὶ σὲ ἐν τέρψει ὡς ἐν ἡμέρᾳ ἑορτῆς.
\vs{18}Καὶ συνάξω τοὺς συντετριμμένους σου· οὐαὶ, τίς ἔλαβεν ἐπʼ αὐτὴν ὀνειδισμόν;

\vs{19}Ἰδοὺ ἐγὼ ποιῶ ἐν σοὶ ἔνεκέν σου ἐν τῷ καιρῷ ἐκείνῳ, λέγει Κύριος, καὶ σώσω τὴν ἐκπεπιεσμένην, καὶ τὴν ἀπωσμένην εἰσδέξομαι, καὶ θήσομαι αὐτοὺς εἰς καύχημα, καὶ ὀνομαστοὺς ἐν πάσῃ τῇ γῇ.
\vs{20}καὶ καταισχυνθήσονται ἐν τῷ καιρῷ ἐκείνῳ, ὅταν καλῶς ὑμῖν ποιήσω, καὶ ἐν τῷ καιρῷ, ὅταν εἰσδέξομαι ὑμᾶς· διότι δώσω ὑμᾶς ὀνομαστοὺς, καὶ εἰς καύχημα ἐν πᾶσι τοῖς λαοῖς τῆς γῆς, ἐν τῷ ἐπιστρέφειν με τὴν αἰχμαλωσίαν ὑμῶν ἐνώπιον ὑμῶν, λέγει Κύριος.


\end{multicols}
\chapter{ΑΓΓΑΙΟΣ}
\begin{multicols}{2}

\ch{1}
ἘΝ τῷ δευτέρῳ ἔτει ἐπὶ Δαρείου τοῦ βασιλέως, ἐν τῷ μηνὶ τῷ ἕκτῳ, μιᾷ τοῦ μηνὸς, ἐγένετο λόγος Κυρίου ἐν χειρὶ Ἀγγαίου τοῦ προφήτου, λέγων, εἰπὸν πρὸς Ζοροβάβελ τὸν τοῦ Σαλαθιὴλ ἐκ φυλῆς Ἰούδα, καὶ πρὸς Ἰησοῦν τὸν τοῦ Ἰωσεδὲκ τὸν ἱερέα τὸν μέγαν, λέγων,
\vs{2}τάδε λέγει Κύριος παντοκράτωρ, λέγων, ὁ λαὸς οὗτος λέγουσιν, οὐκ ἧκεν ὁ καιρὸς τοῦ οἰκοδομῆσαι τὸν οἶκον Κυρίου.
\vs{3}Καὶ ἐγένετο λόγος Κυρίου ἐν χειρὶ Ἀγγαίου τοῦ προφήτου, λέγων,

\vs{4}Εἰ καιρὸς μὲν ὑμῖν ἐστι τοῦ οἰκεῖν ἐν οἴκοις ὑμῶν κοιλοστάθμοις, ὁ δὲ οἶκος ἡμῶν ἐξηρήμωται;

\vs{5}Καὶ νῦν τάδε λέγει Κύριος παντοκράτωρ, τάξατε δὴ καρδίας ὑμῶν εἰς τὰς ὁδοὺς ὑμῶν.
\vs{6}Ἐσπείρατε πολλὰ καὶ εἰσηνέγκατε ὀλίγα, ἐφάγετε καὶ οὐκ εἰς πλησμονὴν, ἐπίετε καὶ οὐκ εἰς μέθην, περιεβάλεσθε καὶ οὐκ ἐθερμάνθητε ἐν αὐτοῖς, καὶ ὁ τοὺς μισθοὺς συνάγων, συνήγαγεν εἰς δεσμὸν τετρυπημένον.

\vs{7}Τάδε λέγει Κύριος παντοκράτωρ, θέσθε τὰς καρδίας ὑμῶν εἰς τὰς ὁδοὺς ὑμῶν.
\vs{8}Ἀνάβητε εἰς τὸ ὄρος, καὶ κόψατε ξύλα, οἰκοδομήσατε τὸν οἶκον, καὶ εὐδοκήσω ἐν αὐτῷ, καὶ ἐνδοξασθήσομαι, εἶπε Κύριος.
\vs{9}Ἐπεβλέψατε εἰς πολλὰ, καὶ ἐγένετο ὀλίγα· καὶ εἰσηνέχθη εἰς τὸν οἶκον, καὶ ἐξεφύσησα αὐτά· διατοῦτο τάδε λέγει Κύριος παντοκράτωρ, ἀνθʼ ὧν ὁ οἶκός μου ἐστὶν ἔρημος, ὑμεῖς δὲ διώκετε ἕκαστος εἰς τὸν οἶκον αὐτοῦ,
\vs{10}διατοῦτο ἀνέξει ὁ οὐρανὸς ἀπὸ δρόσου, καὶ ἡ γῆ ὑποστελεῖται τὰ ἐκφόρια αὐτῆς.
\vs{11}Καὶ ἐπάξω ῥομφαίαν ἐπὶ τὴν γῆν, καὶ ἐπὶ τὰ ὄρη, καὶ ἐπὶ τὸν σῖτον, καὶ ἐπὶ τὸν οἶνον, καὶ ἐπὶ τὸ ἔλαιον, καὶ ὅσα ἐκφέρει ἡ γῆ, καὶ ἐπὶ τοὺς ἀνθρώπους, καὶ ἐπὶ τὰ κτήνη, καὶ ἐπὶ πάντας τοὺς πόνους τῶν χειρῶν αὐτῶν.

\vs{12}Καὶ ἤκουσε Ζοροβάβελ ὁ τοῦ Σαλαθιὴλ ἐκ φυλῆς Ἰούδα, καὶ Ἰησοῦς ὁ τοῦ Ἰωσεδὲκ ὁ ἱερεὺς ὁ μέγας, καὶ πάντες οἱ κατάλοιποι τοῦ λαοῦ τῆς φωνῆς Κυρίου τοῦ Θεοῦ αὐτῶν, καὶ τῶν λόγων τοῦ Ἀγγαίου τοῦ προφήτου, καθότι ἐξαπέστειλεν αὐτὸν Κύριος ὁ Θεὸς αὐτῶν πρὸς αὐτοὺς, καὶ ἐφοβήθη ὁ λαὸς ἀπὸ προσώπου Κυρίου.
\vs{13}Καὶ εἶπεν Ἀγγαῖος ἄγγελος Κυρίου ἐν ἀγγέλοις Κυρίου τῷ λαῷ, ἐγώ εἰμι μεθʼ ὑμῶν, λέγει Κύριος.

\vs{14}Καὶ ἐξήγειρε Κύριος τὸ πνεῦμα Ζοροβάβελ τοῦ Σαλαθιὴλ ἐκ φυλῆς Ἰούδα, καὶ τὸ πνεῦμα Ἰησοῦ τοῦ Ἰωσεδὲκ τοῦ ἱερέως τοῦ μεγάλου, καὶ τὸ πνεῦμα τῶν καταλοίπων παντὸς τοῦ λαοῦ, καὶ εἰσῆλθον, καὶ ἐποίουν ἔργα ἐν τῷ οἴκῳ Κυρίου παντοκράτορος Θεοῦ αὐτῶν.

\vs{15}Τῇ τετράδι καὶ εἰκάδι τοῦ μηνὸς τοῦ ἕκτου, τῷ δευτέρῳ ἔτει, ἐπὶ Δαρείου τοῦ βασιλέως.

\ch{2}Τῷ μηνὶ τῷ ἑβδόμῳ, μιᾷ καὶ εἰκάδι τοῦ μηνὸς, ἐλάλησε Κύριος ἐν χειρὶ Ἀγγαίου τοῦ προφήτου, λέγων,
\vs{2}εἰπὸν δὴ πρὸς Ζοροβάβελ τὸν τοῦ Σαλαθιὴλ ἐκ φυλῆς Ἰούδα, καὶ πρὸς Ἰησοῦν τοῦ Ἰωσεδὲκ τὸν ἱερέα τὸν μέγαν, καὶ πρὸς πάντας τοὺς καταλοίπους τοῦ λαοῦ, λέγων,

\vs{3}Τίς ἐξ ὑμῶν, ὃς εἶδε τὸν οἶκον τοῦτον ἐν τῇ δόξῃ αὐτοῦ τῇ ἔμπροσθεν; καὶ πῶς ὑμεῖς βλέπετε αὐτὸν νῦν καθὼς οὐχ ὑπάρχοντα ἐνώπιον ὑμῶν;
\vs{4}Καὶ νῦν κατίσχυε Ζοροβάβελ, λέγει Κύριος, καὶ κατίσχυε Ἰησοῦ ὁ τοῦ Ἰωσεδὲκ ὁ ἱερεὺς ὁ μέγας, καὶ κατισχυέτω πᾶς ὁ λαὸς τῆς γῆς, λέγει Κύριος, καὶ ποιεῖτε, διότι μεθʼ ὑμῶν ἐγώ εἰμι, λέγει Κύριος ὁ παντοκράτωρ,
\vs{5}καὶ τὸ πνεῦμά μου ἐφέστηκεν ἐν μέσῳ ὑμῶν· θαρσεῖτε,

\vs{6}Διότι τάδε λέγει Κύριος παντοκράτωρ, ἔτι ἅπαξ ἐγὼ σείσω τὸν οὐρανὸν καὶ τὴν γῆν, καὶ τὴν θάλασσαν καὶ τὴν ξηρὰν,
\vs{7}καὶ συσσείσω πάντα τὰ ἔθνη, καὶ ἥξει τὰ ἐκλεκτὰ πάντων τῶν ἐθνῶν· καὶ πλήσω τὸν οἶκον τοῦτον δόξης, λέγει Κύριος παντοκράτωρ.
\vs{8}Ἐμὸν τὸ ἀργύριον, καὶ ἐμὸν τὸ χρυσίον, λέγει Κύριος παντοκράτωρ.
\vs{9}Διότι μεγάλη ἔσται ἡ δόξα τοῦ οἴκου τούτου, ἡ ἐσχάτη ὑπὲρ τὴν πρώτην, λέγει Κύριος παντοκράτωρ· καὶ ἐν τῷ τόπῳ τούτῳ δώσω εἰρήνην, λέγει Κύριος παντοκράτωρ, καὶ εἰρήνην ψυχῆς εἰς περιποίησιν παντὶ τῷ κτίζοντι, τοῦ ἀναστῆσαι τὸν ναὸν τοῦτον.

\vs{10}Τετράδι καὶ εἰκάδι τοῦ ἐννάτου μηνὸς, ἔτους δευτέρου, ἐπὶ Δαρείου, ἐγένετο λόγος Κυρίου πρὸς Ἀγγαῖον τὸν προφήτην, λέγων,
\vs{11}τάδε λέγει Κύριος παντοκράτωρ, ἐπερώτησον δὴ τοὺς ἱερεῖς νόμον, λέγων,
\vs{12}ἐὰν λάβῃ ἄνθρωπος κρέας ἅγιον ἐν τῷ ἄκρῳ τοῦ ἱματίου αὐτοῦ, καὶ ἅψηται τὸ ἄκρον τοῦ ἱματίου αὐτοῦ ἄρτου, ἢ ἑψήματος, ἢ οἴνου, ἢ ἐλαίου, ἢ παντὸς βρώματος, εἰ ἁγιασθήσεται; καὶ ἀπεκρίθησαν οἱ ἱερεῖς, καὶ εἶπαν, οὔ.
\vs{13}Καὶ εἶπεν Ἀγγαῖος, ἐὰν ἅψηται μεμιασμένος ἀκάθαρτος ἐπὶ ψυχῇ ἐπὶ παντὸς τούτων, εἰ μιανθήσεται; καὶ ἀπεκρίθησαν οἱ ἱερεῖς, καὶ εἶπαν, μιανθήσεται.
\vs{14}Καὶ ἀπεκρίθη Ἀγγαῖος, καὶ εἶπεν, οὕτως ὁ λαὸς οὗτος, καὶ οὕτως τὸ ἔθνος τοῦτο ἐνώπιον ἐμοῦ, λέγει Κύριος, καὶ οὕτως πάντα τὰ ἔργα τῶν χειρῶν αὐτῶν· καὶ ὃς ἐὰν ἐγγίσῃ ἐκεῖ, μιανθήσεται ἕνεκεν τῶν λημμάτων αὐτῶν τῶν ὀρθρινῶν, ὀδυνηθήσονται ἀπὸ προσώπου πόνων αὐτῶν, καὶ ἐμισεῖτε ἐν πύλαις ἐλέγχοντα.
\vs{15}Καὶ νῦν θέσθε δὴ εἰς τὰς καρδίας ὑμῶν ἀπὸ τῆς ἡμέρας ταύτης καὶ ὑπεράνω, πρὸ τοῦ θεῖναι λίθον ἐπὶ λίθον ἐν τῷ ναῷ Κυρίου
\vs{16}τίνες ἦτε, ὅτε ἐνεβάλλετε εἰς κυψέλην κριθῆς εἴκοσι σάτα, καὶ ἐγένετο κριθῆς δέκα σάτα· καὶ εἰσεπορεύεσθε εἰς τὸ ὑπολήνιον ἐξαντλῆσαι πεντήκοντα μετρητὰς καὶ ἐγένοντο εἴκοσι.
\vs{17}Ἐπάταξα ὑμᾶς ἐν ἀφορίᾳ, καὶ ἐν ἀνεμοφθορίᾳ, καὶ ἐν χαλάζῃ πάντα τὰ ἔργα τῶν χειρῶν ὑμῶν, καὶ οὐκ ἐπεστρέψατε πρὸς μὲ, λέγει Κύριος.

\vs{18}Ὑποτάξατε δὴ τὰς καρδίας ὑμῶν ἀπὸ τῆς ἡμέρας ταύτης, καὶ ἐπέκεινα, ἀπὸ τῆς τετράδος καὶ εἰκάδος τοῦ ἐννάτου μηνὸς, καὶ ἀπὸ τῆς ἡμέρας ἧς τεθεμελίωται ὁ ναὸς Κυρίου· θέσθε ἐν ταῖς καρδίαις ὑμῶν,
\vs{19}εἰ ἐπιγνωσθήσεται ἐπὶ τῆς ἅλω, καὶ εἰ ἔτι ἡ ἄμπελος, καὶ ἡ συκῆ, καὶ ἡ ῥοὰ, καὶ τὰ ξύλα τῆς ἐλαίας τὰ οὐ φέροντα καρπὸν, ἀπὸ τῆς ἡμέρας ταύτης εὐλογήσω.

\vs{20}Καὶ ἐγένετο λόγος Κυρίου ἐκ δευτέρου πρὸς Ἀγγαῖον τὸν προφήτην, τετράδι καὶ εἰκάδι τοῦ μηνὸς, λέγων,
\vs{21}εἰπὸν πρὸς Ζοροβάβελ τὸν τοῦ Σαλαθιὴλ ἐκ φυλῆς Ἰούδα, λέγων,

Ἐγὼ σείω τὸν οὐρανὸν καὶ τὴν γῆν, καὶ τὴν θάλασσαν καὶ τὴν ξηρὰν,
\vs{22}καὶ καταστρέψω θρόνους βασιλέων, καὶ ὀλοθρεύσω δύναμιν βασιλέων τῶν ἐθνῶν, καὶ καταστρέψω ἅρματα καὶ ἀναβάτας, καὶ καταβήσονται ἵπποι καὶ ἀναβάται αὐτῶν, ἕκαστος ἐν ῥομφαίᾳ πρὸς τὸν ἀδελφὸν αὐτοῦ.
\vs{23}Ἐν τῇ ἡμέρᾳ ἐκείνῃ, λέγει Κύριος παντοκράτωρ, λήψομαί σε Ζοροβάβελ τὸν τοῦ Σαλαθιὴλ, τὸν δοῦλόν μου, λέγει Κύριος, καὶ θήσομαί σε ὡς σφραγίδα, διότι σὲ ᾑρέτισα, λέγει Κύριος παντοκράτωρ.\end{multicols} % End two-column layout
\vfill
\setlength{\parindent}{0cm}
\fontsize{8}{10}\selectfont{This work is in the public domain. \\The source is available at: \underline{https://ebible.org/Scriptures/details.php?id=grcbrent}.}

\end{spacing}
\end{document}